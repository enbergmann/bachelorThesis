\documentclass[draft=false,twoside,12pt]{scrreprt}
\newcommand{\lang}{english}
\usepackage[\lang]{babel}
\usepackage{graphicx}
\usepackage{float} %TODO what does this do
\usepackage[a4paper, 
            width = 150mm, top = 25mm, bottom = 25mm,
            bindingoffset = 6mm]
            {geometry}
% \usepackage{fancyhdr}
\usepackage{datatool}


\pagestyle{headings}

% \setlength{\parindent}{0pt}

% used packages

\usepackage[utf8]{inputenc}
\usepackage{graphicx}

% --- REFERENCING I-------------------------------------------------------------

\usepackage[hidelinks]{hyperref} % has to be loaded before ams to avoid warnings

%TODO
\usepackage[backend=biber, style = alphabetic]{biblatex}
\addbibresource{sourcesBergmannBT.bib}

% --- MATH ---------------------------------------------------------------------

\usepackage{amsmath} % align
\usepackage{amsthm} % proof
\usepackage{amssymb}
\usepackage{mathtools} %coloneqq
\usepackage{mathabx} %vvvert


% --- REFERENCING II -----------------------------------------------------------

\usepackage[\lang]{cleveref} % has to be loaded after ams and hyperref

% --- MISCELLANEOUS ------------------------------------------------------------

\usepackage{ifthen}


%\usepackage{tikz}
%\usepackage{pgfplots}
\usetikzlibrary{arrows.meta}

%
% USER-DEFINED ENVIRONMENTS
%
\usepackage{amsthm}
\usepackage{listings}

% Re-define font of optional title
\makeatletter
\def\th@plain{%
  \thm@notefont{}% same as heading font
  \itshape % body font
}
\def\th@definition{%
  \thm@notefont{}% same as heading font
  \normalfont % body font
}
\makeatother

%%% MATH ENVIRONMENTS %%%%%%%%%%%%%%%%%%%%%%%%%%%%%%%%%%%%%%%%%%%%%%%%%%%%%%%%%
% statements
\theoremstyle{plain}
\newtheorem{theorem}{Theorem}[chapter]
\newtheorem{proposition}[theorem]{Proposition}
\newtheorem{lemma}[theorem]{Lemma}
\newtheorem{corollary}[theorem]{Corollary}
% normal text
\theoremstyle{definition}
\newtheorem{definition}[theorem]{Definition}
\newtheorem{example}[theorem]{Example}
\newtheorem{exercise}{Exercise}[chapter]
% minor text
\theoremstyle{remark}
\newtheorem{remark}[theorem]{Remark}
\newtheorem*{notation}{Notation}
% proofs
\newtheorem{claim}{Claim}
\newtheorem{subproof}{Proof~of~the~claim}
\newtheorem*{proofb}{Proof}
% algorithm
\newtheoremstyle{algorithm-style}%
  {\topsep}   % ABOVESPACE
  {\topsep}   % BELOWSPACE
  {\normalfont}% BODYFONT
  {0pt}       % INDENT (empty value is the same as 0pt)
  {\bfseries} % HEADFONT
  {.}         % HEADPUNCT
  {5pt plus 1pt minus 1pt} % HEADSPACE
  {\thmname{#1}\thmnumber{ #2}\thmnote{ (#3)}} % CUSTOM-HEAD-SPEC
\theoremstyle{algorithm-style}
\newtheorem{algorithm}[theorem]{Algorithm}

% axiom
\newtheoremstyle{axiom-style}%
  {\topsep}   % ABOVESPACE
  {\topsep}   % BELOWSPACE
  {\itshape}% BODYFONT
  {0pt}       % INDENT (empty value is the same as 0pt)
  {\bfseries} % HEADFONT
  {.}         % HEADPUNCT
  {5pt plus 1pt minus 1pt} % HEADSPACE
  {\thmname{#1} \thmnumber{A#2}\thmnote{ (#3)}} % CUSTOM-HEAD-SPEC
\theoremstyle{axiom-style}
\newtheorem{inneraxiom}{Axiom}
\newenvironment{axiom}[1]
  {\renewcommand\theinneraxiom{#1}\inneraxiom}
  {\endinneraxiom}
%To Reference an Axiom as ``(A1)'' use \cref
\crefname{inneraxiom}{}{}
%To Reference an Axiom as ``axiom (A1)'' use \Cref
\Crefname{inneraxiom}{axiom}{axioms}
\creflabelformat{inneraxiom}{#2(A#1)#3}

%Only for A12
\newtheoremstyle{axiomspecial-style}%
  {\topsep}   % ABOVESPACE
  {\topsep}   % BELOWSPACE
  {\itshape}% BODYFONT
  {0pt}       % INDENT (empty value is the same as 0pt)
  {\bfseries} % HEADFONT
  {.}         % HEADPUNCT
  {5pt plus 1pt minus 1pt} % HEADSPACE
  {\thmnumber{A#2}\thmnote{ (#3)}} % CUSTOM-HEAD-SPEC
\theoremstyle{axiomspecial-style}
\newtheorem{inneraxiomspecial}{}
\newenvironment{axiomspecial}[1]
  {\renewcommand\theinneraxiomspecial{#1}\inneraxiomspecial}
  {\endinneraxiomspecial}
  %To Reference A12 as ``(A12)'' use \crefspecial
\crefname{inneraxiomspecial}{}{}
\creflabelformat{inneraxiomspecial}{#2(A#1)#3}

%% THEOREM ENVIRONMENTS
% theorems
% \makeatletter
% \newtheoremstyle{algostyle}
% {\item[\rlap{\vbox{\hbox{\hskip\labelsep \theorem@headerfont
%    ##1\ ##2\theorem@separator}\hbox{\strut}}}]}%
% {\item[\rlap{\vbox{\hbox{\hskip\labelsep \theorem@headerfont
% ##1\ ##2\ ##3\theorem@separator}\hbox{\strut}}}]}
% \theoremstyle{algostyle}
% \theorempreskip{3\parskip}
% \theoremheaderfont{\itshape}


% % axioms
% % \makeatletter
% \newtheoremstyle{axiomstyle}%
%   {\item[\hskip\labelsep \theorem@headerfont ##1\ (A##2)\theorem@separator]}%
%   {\item[\hskip\labelsep \theorem@headerfont ##1\ (A##2)\ ##3\theorem@separator]}
% % \makeatother
% \theoremstyle{axiomstyle}

% \theoremheaderfont{\bf}

% \theoremstyle{changebreak}
% \theoremheaderfont{\bf}
% \newcommand\theoremname{Axiom}
% \newtheorem{innercustomthm}{Axiom}
% \newenvironment{axiom}[1]
%   {\renewcommand\theinnercustomthm{#1}\innercustomthm}
%   {\endinnercustomthm}

% \theoremstyle{changebreak}
% \theoremheaderfont{\bf}
% \newcommand\theoremname{Axiom}
% \newtheorem*{thm}{\theoremname}
% \newenvironment{axiom}[1][\theoremname]
%  {\edef\theoremname{#1}%
%   \begin{thm}%
%  }
%  {\end{thm}}

% \makeatother








%%% CODE ENVIRONMENTS %%%%%%%%%%%%%%%%%%%%%%%%%%%%%%%%%%%%%%%%%%%%%%%%%%%%%%%%%
\usepackage{listings}
\lstset{ %
  language=Matlab,
  basicstyle=\scriptsize\ttfamily,
  keywordstyle=\color{blue},
  commentstyle=\color{dkgreen},
  stringstyle=\color{mauve},
  numberstyle=\color{orange},
  numbers=left,
  numberstyle=\tiny\color{gray},
  stepnumber=1,
  numbersep=10pt,
  % backgroundcolor=\color{bgcolor},
  showspaces=false,
  showstringspaces=false,
  showtabs=false,
  tabsize=2,
  captionpos=t,
  breaklines=true,
  breakatwhitespace=false,
  belowskip=0pt plus 2pt minus 2pt,
}
% different font sizes
\lstdefinestyle{fullsrcsmall}{
  basicstyle=\small\ttfamily,
  xleftmargin=1em,
}
\lstdefinestyle{fullsrcfnsize}{
  basicstyle=\footnotesize\ttfamily,
  xleftmargin=2em,
}
\lstdefinestyle{fullsrcscsize}{
  basicstyle=\scriptsize\ttfamily,
  xleftmargin=2em,
}
\lstdefinestyle{inline}{
  basicstyle=\ttfamily,
  numbers=none,
  xleftmargin=1em,
}
\lstdefinestyle{inlinesmall}{
  basicstyle=\small\ttfamily,
  numbers=none,
  xleftmargin=1em,
}
\lstdefinestyle{inlinefnsize}{
  basicstyle=\footnotesize\ttfamily,
  numbers=none,
  xleftmargin=1em,
}

%%% ALGORITHM ENVIRONMENTS %%%%%%%%%%%%%%%%%%%%%%%%%%%%%%%%%%%%%%%%%%%%%%%%%%%%
% \usepackage{algorithm}
\usepackage{algpseudocode}
% endif CC style
\algrenewcommand\algorithmicrequire{\textbf{Input:}}
\algrenewcommand\algorithmicensure{\textbf{Output:}}
\newcommand\Od{\textbf{od}}
\newcommand\Fi{\textbf{fi}}
\algnotext{EndIf}
\let\oldEndIf\EndIf
\renewcommand\EndIf{\Fi\oldEndIf}
\algnotext{EndWhile}
\let\oldEndWhile\EndWhile
\renewcommand\EndWhile{\Od\oldEndWhile}
\algnotext{EndFor}
\let\oldEndFor\EndFor
\renewcommand\EndFor{\Od\oldEndFor}
% block for an inline for loop
\algblockdefx[IfInline]{IfInline}{EndIfInline}
[2]{\textbf{if} #1 \textbf{then} #2}
[0]{\textbf{fi}}
\algcblockdefx{IfInline}{ElseInline}{EndIfInline}
[1]{\textbf{else} #1}
[0]{\textbf{fi}}
\algnotext{EndIfInline}
\let\oldEndIfInline\EndIfInline
\renewcommand\EndIfInline{\Fi\oldEndIfInline}
% single else inline
\algcloopx[SingleElseInline]{If}{SingleElseInline}
[1]{\textbf{else} #1}

% ============================================================
% === USER-DEFINED COMMANDS ==================================
% ============================================================

\usepackage{amssymb,mathabx,textcomp,bm}

\newcommand{\mathbox}[2]{\makebox[#1]{$\displaystyle #2$}}
\newcommand{\Stackrel}[2]{\mathbox{1.25em}{\stackrel{#1}{#2}}}


% === INTEGRAL MEAN ==========================================

\def\Xint#1{\mathchoice
{\XXint\displaystyle\textstyle{#1}}%
{\XXint\textstyle\scriptstyle{#1}}%
{\XXint\scriptstyle\scriptscriptstyle{#1}}%
{\XXint\scriptscriptstyle\scriptscriptstyle{#1}}%
\!\int}
\def\XXint#1#2#3{{\setbox0=\hbox{$#1{#2#3}{\int}$}
\vcenter{\hbox{$#2#3$}}\kern-.5\wd0}}
\newcommand{\intmean}{\Xint-}

% === PERFECT BULLET =========================================

\let\oldbullet\bullet
\newlength{\raisebulletlen}
\setbox1=\hbox{$\bullet$}\setbox2=\hbox{\tiny$\bullet$}
\setlength{\raisebulletlen}{\dimexpr0.5\ht1-0.5\ht2}
\renewcommand\bullet{\raisebox{\raisebulletlen}{\,\tiny$\oldbullet$}\,}

% === ORTHOGONAL SUM =========================================

\newcommand\orthsum{
\tikz[baseline=(A.base),font=\small]
     \node[draw,ellipse,inner sep=0.15ex] (A){$\perp$};
}

% === CALIGRAPHIC LETTERS ====================================

\newcommand\Acal{\mathcal{A}}
\newcommand\Bcal{\mathcal{B}}
\newcommand\Ccal{\mathcal{C}}
\newcommand\Dcal{\mathcal{D}}
\newcommand\Ecal{\mathcal{E}}
\newcommand\Fcal{\mathcal{F}}
\newcommand\Jcal{\mathcal{J}}
\newcommand\Kcal{\mathcal{K}}
\newcommand\Lcal{\mathcal{L}}
\newcommand\Mcal{\mathcal{M}}
\newcommand\Ncal{\mathcal{N}}
\newcommand\Ocal{\mathcal{O}}
\newcommand\Pcal{\mathcal{P}}
\newcommand\Rcal{\mathcal{R}}
\newcommand\Tcal{\mathcal{T}}

% === MATHBB LETTERS =========================================

\newcommand\C{\mathbb{C}}
\newcommand\N{\mathbb{N}}
\newcommand\R{\mathbb{R}}
\newcommand\T{\mathbb{T}}
\newcommand\K{\mathbb{K}}

% === MATH OPERATORS =========================================

\DeclareMathOperator*{\argmin}{argmin} % the * adjusts MathOperator for indizes beneath
\DeclareMathOperator{\bisec}{bisec}
\DeclareMathOperator{\cond}{cond}
\DeclareMathOperator{\Conv}{conv}
\DeclareMathOperator{\Curl}{Curl}
\DeclareMathOperator{\curl}{curl}
\DeclareMathOperator{\dev}{dev}
\DeclareMathOperator{\diag}{diag}
\DeclareMathOperator{\diam}{diam}
\DeclareMathOperator{\Dim}{dim}
\DeclareMathOperator{\Div}{div}
\DeclareMathOperator{\Dist}{dist}
\DeclareMathOperator{\esssup}{ess\ supp}
\DeclareMathOperator{\grad}{\nabla}
\DeclareMathOperator{\Int}{int}
\DeclareMathOperator{\Ker}{ker}
\DeclareMathOperator{\Mid}{mid}
\DeclareMathOperator{\Osc}{osc}
\DeclareMathOperator{\Red}{red}
\DeclareMathOperator{\Ref}{Ref}
\DeclareMathOperator{\Res}{Res}
\DeclareMathOperator{\sign}{sgn}
\DeclareMathOperator{\Span}{span}
\DeclareMathOperator{\supp}{supp}
\DeclareMathOperator{\tr}{tr}
\DeclareMathOperator{\Width}{width}
\DeclareMathOperator*{\arginf}{arginf}

% === COMMANDS WITH INPUT ARGUMENTS ==========================

\newcommand\abs[1]{\lvert #1 \rvert}
\newcommand\average[1]{\langle #1 \rangle}
\newcommand\jump[1]{\lbrack #1 \rbrack}
\newcommand\NormEnergy[1]{\big\vvvert #1 \big\vvvert}
\newcommand\Norm[2]{\lVert #1 \rVert_{#2}}
\newcommand\scal[2]{\left\langle #1 , #2 \right\rangle}

% === SPACES WITH NAMES ======================================

\newcommand{\CONF}{\textup{C}}
\newcommand{\NC}{\textup{NC}}
\newcommand{\CR}{\textup{CR}}
\newcommand{\RT}{\textup{RT}}

% === GENERAL STUFF ==========================================

\newcommand{\splitter}{\,:\,} % split for definition of sets
\renewcommand{\d}{\, \textup{d}} % d for differentials, i.e., \d x


% === JUST TO COMPILE CHAPTERS 4  AND 5 ==========================================

\newcommand{\LO}[0]{L^2(\Omega)}
\newcommand{\restrict}[2]{\left. #1 \right\lvert_{#2}}
\DeclareMathOperator{\mPkt}{mid}
\newcommand{\nnorm}[2]{\left\lVert #1 \right\rVert_{#2}}
\newcommand{\volf}[2]{\nnorm{h_{#1}f}{#2}}
\newcommand{\volO}[1]{\nnorm{h_{#1}f}{\LO}}
\DeclareMathOperator\Kern{ker}
\DeclareMathOperator\setspan{span}

% === JUST TO COMPILE CHAPTER 6 ==========================================


\newcommand\NormL[3]{\Norm{#1}{L^{#2}\left( { #3 } \right)}}
 \newcommand\NormH[3]{\Norm{#1}{H^{#2}\left( { #3 } \right)}}
 \newcommand\NormHdiv[1]{\Norm{#1}{H(\div)}}
 \newcommand\NormW[3]{\Norm{#1}{W^{#2}\left( { #3 } \right)}}
\newcommand\NormLDom[2]{\NormL{#1}{#2}{\Omega}}
 \newcommand\NormHDom[2]{\NormH{#1}{#2}{\Omega}}
 \newcommand\NormHdivDom[1]{\NormHdiv{#1}}
 \newcommand\NormWDom[2]{\NormW{#1}{#2}{\Omega}}
 \newcommand\NormLz[2]{\NormL{#1}{2}{#2}}
 \newcommand\NormHz[2]{\NormH{#1}{2}{#2}}
 \newcommand\NormWkp[4]{\NormW{#1}{{#2,#3}}{#4}}
 \newcommand\NormWkpDom[3]{\NormWDom{#1}{{#2,#3}}}
 \newcommand\NormLzDom[1]{\NormLDom{#1}{2}}
 \newcommand\NormHzDom[1]{\NormHDom{#1}{2}}
\newcommand\NormMax[2]{\Norm{#1}{\C({#2})}}


% === JUST TO COMPILE CHAPTER 0 ==========================================

\newcommand{\dist}{\mathrm{dist}}
\newcommand{\fa}{\;\forall\;}
\newcommand{\Lra}{\Leftrightarrow}
\newcommand{\id}{\mathrm{id}}

% === JUST TO COMPILE EXERCISES ==========================================

\newcommand{\al}{\alpha}
\newcommand{\be}{\beta}
\newcommand{\ga}{\gamma}
\newcommand{\de}{\delta}
\newcommand{\eps}{\varepsilon}
\newcommand{\vphi}{\varphi}
\newcommand{\la}{\lambda}
\newcommand{\om}{\omega}
 

\begin{document}
\section{The continuous and discrete problem}
Let $\alpha>0$, $\Omega\subset\Rbb^n$ bounded polyhedral Lipschitz domain, and
$f\in L^2(\Omega)$.

The continuous problem minimizes 
\begin{align}
  \label{equ:contProb}
  E(v)\coloneqq \frac{\alpha}{2}\Vert v\Vert^2_{L^2(\Omega)}
  +|v|_{\BV(\Omega)}+ \Vert v\Vert_{L^1(\partial\Omega)}
  -\int_\Omega f\,v\dx
\end{align}
amongst all $v\in V\coloneqq \BV(\Omega)\cap L^2(\Omega)$ where the \BV 
seminorm $|v|_{\BV(\Omega)}$ is equal to the $W^{1,1}$ seminorm for any 
$v\in W^{1,1}(\Omega)$.

The nonconforming problem minimizes 
\begin{align}
  \label{equ:discProb}
  \Enc(\vcr)\coloneqq \frac{\alpha}{2}\Vert \vcr\Vert^2_{L^2(\Omega)}
  +|\vcr |_{1,1,\NC}-\int_\Omega f\, \vcr\dx
\end{align}
amongst all $\vcr\in\CR^1_0(\Tcal)$ where $|\bullet|_{1,1,\NC}\coloneqq
\Vert\gradnc\bullet\Vert_{L^1(\Omega)}$.

\section{Refinement indicator and guaranteed lower energy bound}

For some $n\in\mathbb{N}$ (here $n=2$) and $0<\gamma\leq 1$ define a refinement
indicator 
$\eta\coloneqq\sum_{T\in\mathcal{T}}\eta(T)$
with
\begin{align}
  \label{equ:errorEstimator}
  \eta (T)\coloneqq 
  \underbrace{|T|^{2/n}\Vert f-\alpha \ucr\Vert^2_{L^2(T)}}_{\eqqcolon 
  \eta_\text{Vol}(T)}
  +\underbrace{|T|^{\gamma/n}\sum_{F\in\Fcal(T)}\Vert [\ucr]_F\Vert_{L^1(F)}}_{
  \eqqcolon \eta_\text{Jumps}(T)}
\end{align}
for any $T\in\mathcal{T}$.

For $f\in H^1_0(\Omega)$ and $u\in H^1_0(\Omega)$ $\left(\ucr\in
\CR^1_0(\Omega)\right)$ continuous (discrete) minimizer with minimal energy
$E(u)$ $\left(\Enc(\ucr)\right)$ it holds
\begin{align}
  \label{equ:gleb}
  \Enc(\ucr)+\frac{\alpha}{2}\Vert u-\ucr\Vert_{L^2(\Omega)}^2
  -\frac{\kappa_\CR}{\alpha}\Vert
  h_\mathcal{T}(f-\alpha\ucr)\Vert_{L^2(\Omega)} |f|_{1,2}\leq E(u)
\end{align}
where $|\bullet|_{1,2}=\Vert\nabla \bullet\Vert_{L^2(\Omega)}$.

Hence, for $\textup{GLEB} \coloneqq 
  \Enc(\ucr) - \frac{\kappa_\CR}{\alpha}\Vert
  h_\mathcal{T}(f-\alpha\ucr)\Vert_{L^2(\Omega)} |f|_{1,2}$, it holds
  $\Enc(\ucr)\geq \textup{GLEB}$ and $E(u)\geq \textup{GLEB}$.

\section{Experiments}
In the following sections the termination criterion for the algorithm
was TODO $<\varepsilon=10^{-4}$. %TODO

\section{Examples with exact solution}
\subsection{Example 1}
For $\beta=1$ define $f$ as a function of the radius as
\begin{align}
  \label{equ:f01}
  f(r)\coloneqq 
  \begin{cases}
    \alpha-12(2-9r) & \text{if } 0\leq r\leq\frac{1}{6},\\
    \alpha(1+(6r-1)^\beta)-\frac{1}{r} & \text{if } \frac{1}{6}\leq r\leq
    \frac{1}{3},\\
    2\alpha+6\pi\sin(\pi(6r-2))-\frac{1}{r}\cos(\pi(6r-2)) &
    \text{if } \frac{1}{3}\leq r\leq\frac{1}{2},\\
    2\alpha(\frac{5}{2}-3r)^\beta+\frac{1}{r}&
    \text{if } \frac{1}{2}\leq r\leq\frac{5}{6},\\
    -3\pi\sin(\pi(6r-5))+\frac{1+\cos(\pi(6r-5))}{2r} &
    \text{if } \frac{5}{6}\leq r\leq 1,
  \end{cases}
\end{align}
with exact solution
\begin{align}
  \label{equ:f01exactSol}
  u(r)\coloneqq
  \begin{cases}
    1 & \text{if } 0\leq r\leq\frac{1}{6},\\
    1+(6r-1)^\beta & \text{if } \frac{1}{6}\leq r\leq
    \frac{1}{3},\\
    2 &
    \text{if } \frac{1}{3}\leq r\leq\frac{1}{2},\\
    2(\frac{5}{2}-3r)^\beta &
    \text{if } \frac{1}{2}\leq r\leq\frac{5}{6},\\
    0 &
    \text{if } \frac{5}{6}\leq r\leq 1.
  \end{cases}
\end{align}

For $\alpha=1$ the exact energy $E(u)$ was computed before the experiment.

\begin{minipage}[t]{0.3\textwidth}
  \begin{figure}[H]
	  \centering
		\includegraphics[width=1.1\textwidth]{tikzPlots/f01/alpha1/rhs.png} 
		\caption{$f$ for $\alpha=1$}
  \end{figure}
\end{minipage}
\hfill
\vline
\hfill
\begin{minipage}[t]{0.3\textwidth}
  \begin{figure}[H]
	  \centering
		\includegraphics[width=1.1\textwidth]{tikzPlots/f01/alpha1e4/rhs.png} 
		\caption{$f$ for $\alpha=10^4$}
  \end{figure}
\end{minipage}
\hfill
\vline
\hfill
\begin{minipage}[t]{0.3\textwidth}
  \begin{figure}[H]
	  \centering
		\includegraphics[width=1.1\textwidth]
    {tikzPlots/f01/alpha1/exactSolution.png} 
		\caption{$u$}
  \end{figure}
\end{minipage}

\begin{minipage}[t]{0.3\textwidth}
  \begin{figure}[H]
	  \centering
		\includegraphics[width=1.1\textwidth]{tikzPlots/f01/alpha1/rhsAxis.png} 
		\caption{$f$ along the axes for $\alpha=1$}
  \end{figure}
\end{minipage}
\hfill
\vline
\hfill
\begin{minipage}[t]{0.3\textwidth}
  \begin{figure}[H]
	  \centering
		\includegraphics[width=1.1\textwidth]{tikzPlots/f01/alpha1e4/rhsAxis.png} 
		\caption{$f$ along the axes for $\alpha=10^4$}
  \end{figure}
\end{minipage}
\hfill
\vline
\hfill
\begin{minipage}[t]{0.3\textwidth}
  \begin{figure}[H]
	  \centering
		\includegraphics[width=1.1\textwidth]
    {tikzPlots/f01/alpha1/exactSolutionAxis.png} 
		\caption{$u$ along the axes}
  \end{figure}
\end{minipage}

\vspace{-\parskip}
\begin{figure}[H]
	\centering
	\includegraphics[width=14cm]{tikzPlots/f01/alpha1/convergence.pdf}
  \caption{convergence history plot for $\alpha=1$}
\end{figure}

\vspace{-\parskip}
\begin{figure}[H]
	\centering
	\includegraphics[width=14cm]{tikzPlots/f01/alpha1e4/convergence.pdf}
  \caption{convergence history plot for $\alpha=10^4$}
\end{figure}

\vspace{-\parskip}
\begin{figure}[H]
	\centering
	\includegraphics[width=14cm]{tikzPlots/f01/alpha1/misc.pdf}
  \caption{development of the number of iterations and the elapsed time for 
  each iteration for $\alpha=1$}
\end{figure}

\vspace{-\parskip}
\begin{figure}[H]
	\centering
	\includegraphics[width=14cm]{tikzPlots/f01/alpha1e4/misc.pdf}
  \caption{development of the number of iterations and the elapsed time for 
  each iteration for $\alpha=10^4$}
\end{figure}

\DTLloaddb{db}{tikzPlots/f01/alpha1/1e-4/adaptive/lvl10/currentDataReduced.csv}
\DTLassign{db}{1}{\nrNodes=nrNodes, \nrDof=nrDof}

\DTLloaddb{dbAlt}{tikzPlots/f01/alpha1e4/1e-4/adaptive/lvl19/currentDataReduced.csv}
\DTLassign{dbAlt}{1}{\nrNodesAlt=nrNodes, \nrDofAlt=nrDof}

\vspace{-\parskip}
\begin{minipage}[t]{0.45\textwidth}
  \begin{figure}[H]
    \includegraphics[width=1.3\textwidth]
    {tikzPlots/f01/alpha1/1e-4/adaptive/lvl10/triangulation.png}
    \caption{adaptive mesh for $\alpha=1$ and $\theta = 0.5$ with \nrNodes\
    nodes and \nrDof\ degrees of freedom}
  \end{figure}
\end{minipage}
\hfill
\vline
\hfill
\begin{minipage}[t]{0.45\textwidth}
  \begin{figure}[H]
  	\centering
    \includegraphics[width=1.3\textwidth]
    {tikzPlots/f01/alpha1e4/1e-4/adaptive/lvl19/triangulation.png}
    \caption{adaptive mesh for $\alpha=10^4$ and $\theta = 0.5$ with
    \nrNodesAlt\ nodes and \nrDofAlt\ degrees of freedom}
  \end{figure}
\end{minipage}

\DTLgdeletedb{db}
\DTLgdeletedb{dbAlt}

\begin{minipage}[t]{0.45\textwidth}
  \begin{figure}[H]
	  \centering
		\includegraphics[width=1.1\textwidth]
      {tikzPlots/f01/alpha1/1e-4/adaptive/lvl15/solution.png} 
    \caption{last iterate for $\alpha=1$ and $\theta = 0.5$}
  \end{figure}
\end{minipage}
\hfill
\vline
\hfill
\begin{minipage}[t]{0.45\textwidth}
  \begin{figure}[H]
	  \centering
		\includegraphics[width=1.1\textwidth]
      {tikzPlots/f01/alpha1/1e-4/uniform/lvl8/solution.png} 
    \caption{last iterate for $\alpha=1$ and $\theta = 1$}
  \end{figure}
\end{minipage}

\vspace{-\parskip}
\begin{minipage}[t]{0.45\textwidth}
  \begin{figure}[H]
	  \centering
		\includegraphics[width=1.1\textwidth]
      {tikzPlots/f01/alpha1/1e-4/adaptive/lvl15/solutionAxis.png} 
    \caption{last iterate along the axes for $\alpha=1$ and $\theta = 0.5$}
  \end{figure}
\end{minipage}
\hfill
\vline
\hfill
\begin{minipage}[t]{0.45\textwidth}
  \begin{figure}[H]
	  \centering
		\includegraphics[width=1.1\textwidth]
      {tikzPlots/f01/alpha1/1e-4/uniform/lvl8/solutionAxis.png} 
    \caption{last iterate along the axes for $\alpha=1$ and $\theta = 1$}
  \end{figure}
\end{minipage}

\begin{minipage}[t]{0.45\textwidth}
  \begin{figure}[H]
	  \centering
		\includegraphics[width=1.1\textwidth]
      {tikzPlots/f01/alpha1e4/1e-4/adaptive/lvl25/solution.png} 
    \caption{last iterate for $\alpha=10^4$ and $\theta = 0.5$}
  \end{figure}
\end{minipage}
\hfill
\vline
\hfill
\begin{minipage}[t]{0.45\textwidth}
  \begin{figure}[H]
	  \centering
		\includegraphics[width=1.1\textwidth]
      {tikzPlots/f01/alpha1e4/1e-4/uniform/lvl8/solution.png} 
    \caption{last iterate for $\alpha=10^4$ and $\theta = 1$}
  \end{figure}
\end{minipage}

\vspace{-\parskip}
\begin{minipage}[t]{0.45\textwidth}
  \begin{figure}[H]
	  \centering
		\includegraphics[width=1.1\textwidth]
      {tikzPlots/f01/alpha1e4/1e-4/adaptive/lvl25/solutionAxis.png} 
    \caption{last iterate along the axes for $\alpha=10^4$ and $\theta = 0.5$}
  \end{figure}
\end{minipage}
\hfill
\vline
\hfill
\begin{minipage}[t]{0.45\textwidth}
  \begin{figure}[H]
	  \centering
		\includegraphics[width=1.1\textwidth]
      {tikzPlots/f01/alpha1e4/1e-4/uniform/lvl8/solutionAxis.png} 
    \caption{last iterate along the axes for $\alpha=10^4$ and $\theta = 1$}
  \end{figure}
\end{minipage}
\subsection{Example 2}

\section{Application to image}
For $\alpha = 10000$ let $f$ represent the grayscale of an 
image in $[0,1]^{256\times 256}$ scaled to the domain $\Omega\in(0,1)^2$ as
seen in \cref{fig:rhsCameraman}.

\begin{figure}[H]
	\centering
	\includegraphics[width=14cm]{tikzPlots/imageCameraman/rhsGrayscale.png}
  \caption{grayscale plot of the right-hand side $f$ (view from above onto the
  $x$-$y$ plane)}
  \label{fig:rhsCameraman}
\end{figure}

\begin{figure}[H]
	\centering
	\includegraphics[width=14cm]
  {tikzPlots/imageCameraman/convergence.pdf}
  \caption{convergence history plot for $\eta$,
  $\eta_\text{Vol}$, and $\eta_\text{Jumps}$}
\end{figure}

\begin{figure}[H]
	\centering
	\includegraphics[width=14cm]{tikzPlots/imageCameraman/misc.pdf}
  \caption{development of the number of iterations and the elapsed time for 
  each iteration}
\end{figure}

\begin{minipage}[t]{0.5\textwidth}
  \begin{figure}[H]
	  \centering
		\includegraphics[width=\textwidth]
      {tikzPlots/imageCameraman/1e-4/adaptive/lvl21/solutionGrayscale.png}
    \caption{grayscale plot of last iterate for $\theta = 0.5$}
  \end{figure}
\end{minipage}
\begin{minipage}[t]{0.5\textwidth}
  \begin{figure}[H]
	  \centering
		\includegraphics[width=\linewidth]
      {tikzPlots/imageCameraman/1e-4/uniform/lvl9/solutionGrayscale.png}
    \caption{grayscale plot of last iterate for $\theta = 1$}
  \end{figure}
\end{minipage}

\DTLloaddb{db}{
tikzPlots/imageCameraman/1e-4/adaptive/lvl16/currentDataReduced.csv}
\DTLassign{db}{1}{\nrNodes=nrNodes, \nrDof=nrDof}

\begin{figure}[H]
	\centering
  \includegraphics[width=14cm]
  {tikzPlots/imageCameraman/1e-4/adaptive/lvl16/triangulation.png}
  \caption{adaptive mesh for $\theta = 0.5$ with \nrNodes\ nodes and 
  \nrDof\ degrees of freedom}
\end{figure}

\DTLgdeletedb{db}

%TODO here and above not only plot the mesh but also the solution on this mesh
%(maybe with subfigure or maybe with minipages)

\section{Application to a function with discontinuity set}
For $\alpha = 100$ define 
\begin{align}
  \label{equ:defMidSqu}
  f(x)\coloneqq 
  \begin{cases}
    100 &\text{if } \Vert x\Vert_\infty\leq \frac{1}{2},\\
    0 &\text{else}
  \end{cases}
\end{align}
on $\Omega = (-1,1)^2$.

\begin{figure}[H]
	\centering
	\includegraphics[width=14cm]{tikzPlots/middleSquare/rhsGrayscale.png}
  \caption{grayscale plot of the right-hand side $f$ (view from above onto the
  $x$-$y$ plane)}
  \label{fig:rhsMiddleSquare}
\end{figure}

\begin{figure}[H]
	\centering
	\includegraphics[width=14cm]
  {tikzPlots/middleSquare/convergence.pdf}
  \caption{convergence history plot for $\eta$,
  $\eta_\text{Vol}$, and $\eta_\text{Jumps}$}
\end{figure}

\begin{figure}[H]
	\centering
	\includegraphics[width=14cm]{tikzPlots/middleSquare/misc.pdf}
  \caption{development of the number of iterations and the elapsed time for 
  each iteration}
\end{figure}

\begin{minipage}[t]{0.5\textwidth}
  \begin{figure}[H]
	  \centering
		\includegraphics[width=\textwidth]
      {tikzPlots/middleSquare/1e-4/adaptive/lvl16/solutionGrayscale.png}
    \caption{grayscale plot of last iterate for $\theta = 0.5$}
  \end{figure}
\end{minipage}
\begin{minipage}[t]{0.5\textwidth}
  \begin{figure}[H]
	  \centering
		\includegraphics[width=\linewidth]
      {tikzPlots/middleSquare/1e-4/uniform/lvl8/solutionGrayscale.png}
    \caption{grayscale plot of last iterate for $\theta = 1$}
  \end{figure}
\end{minipage}

\DTLloaddb{db}{
tikzPlots/middleSquare/1e-4/adaptive/lvl8/currentDataReduced.csv}
\DTLassign{db}{1}{\nrNodes=nrNodes, \nrDof=nrDof}

\begin{figure}[H]
	\centering
  \includegraphics[width=14cm]
  {tikzPlots/middleSquare/1e-4/adaptive/lvl8/triangulation.png}
  \caption{adaptive mesh for $\theta = 0.5$ with \nrNodes\ nodes and 
  \nrDof\ degrees of freedom}
\end{figure}

\DTLgdeletedb{db}

\end{document}
