\subsection{Iterative solve}
Theorem
\ref{thm:characterizationDiscreteSolutions}.c
characterizes the discrete solution to  
\eqref{e:defnonconfEintro} 
by $u_{\textup{CR}}\in\textup{CR}^1_0(\mathcal T)$ and 
the existence of some 
$
  \Lambda_0\in\operatorname{sign}\nabla_{\textup{NC}}u_{\textup{CR}}
$ 
with \eqref{eqLambda}
% \begin{equation}
%   \alpha (u_{\textup{CR}},v_{\textup{CR}})_{L^2(\Omega)} + (\Lambda_0,\nabla_{\textup{NC}}
%   v_{\textup{CR}})_{L^2(\Omega)}
%   =(f,v_{\textup{CR}})_{L^2(\Omega)}
% \end{equation}
for all $v_{\textup{CR}}\in\textup{CR}^1_0(\mathcal{T})$.
The pair $(u_\textup{CR},\Lambda_0)$ is unique in the first component $u_\textup{CR}$
and computed with the following algorithm with input $(u_0,\Lambda_0)$.

% TODO

\begin{algorithm}[primal-dual iteration]\label{alg:PrimalDualIteration}
\begin{algorithmic}\\
  \Require $u_0\in\textup{CR}_0^1(\mathcal{T}),$
  $\Lambda_0\in P_0(\mathcal{T};
  \overline{B(0,1)}),\tau>0$  \\
  Initialize $v_0\coloneqq 0$ in $\textup{CR}^1_0(\mathcal T)$.
  \For{$j = 1,2,\dots$}
  \begin{equation}
    \label{equ:PrimalDualAlgUj}
    \tilde{u}_j\coloneqq u_{j-1}+\tau v_{j-1},
  \end{equation}
  \begin{equation}
    \label{equ:PrimalDualAlgLambdaJ}
    \Lambda_j\coloneqq
    (\Lambda_{j-1}+\tau\nabla_{\textup{NC}} \tilde{u}_j)/
      (\operatorname{max}\{1,|\Lambda_{j-1}+\tau\nabla_{\textup{NC}}\tilde{u}_j|\}),
  \end{equation}
      \State
  \State solve linear system of equations
  \begin{align}
    \label{equ:linSysPrimalDualAlg}
    \frac{1}{\tau}&a_{\textup{NC}}(u_j,\bullet)+\alpha(u_j,\bullet)_{L^2(\Omega)}
    \notag \\
    &=
    \frac{1}{\tau}a_{\textup{NC}}(u_{j-1},\bullet) + (f,\bullet)_{L^2(\Omega)}
    - (\Lambda_j,\nabla_{\textup{NC}}\bullet)_{L^2(\Omega)} 
  \end{align}
in $\textup{CR}^1_0(\mathcal{T})$ for $u_j\in\textup{CR}^1_0(\mathcal{T})$
with the abbreviation
$v_j\coloneqq(u_j-u_{j-1})/\tau$.\\
  \EndFor
  \Ensure sequence $(u_j,\Lambda_j)_{j\in\mathbb N}$ in $\textup{CR}^1_0(\mathcal{T})\times
  P_0(\mathcal{T};\overline{B(0,1)})$   
  \end{algorithmic}
\end{algorithm}

\subsection{Convergence analysis}

\begin{theorem}[convergence]
  Let $u_{\textup{CR}}\in \textup{CR}^1_0(\mathcal T)$ solve \eqref{eqLambda}.
  If $0 < \tau \leq 1$, then the iterates of Algorithm
  \ref{alg:PrimalDualIteration} converge to $u_{\textup{CR}}$.
\end{theorem}

\begin{proof}
  Set $e_j \coloneqq u_{\textup{CR}}-u_j$, $E_j\coloneqq \Lambda_0-\Lambda_j$.\\
  Test \eqref{equ:linSysPrimalDualAlg} with $e_j$ for
  \begin{align*}
    a_\textup{NC}(v_j,e_j) + \alpha(u_j,e_j)_{L^2(\Omega)} 
    + (\Lambda_j,\nabla_\textup{NC}e_j)_{L^2(\Omega)}
    = 
    (f,e_j)_{L^2(\Omega)}.
  \end{align*}
  Since $u_{\textup{CR}}$ solves \eqref{eqLambda} this is equivalent to
  \begin{align}
    a_\textup{NC}(v_j,e_j) &= 
    \alpha(u_\textup{CR}-u_j,e_j)_{L^2(\Omega)} 
    + (\Lambda_0-\Lambda_j,\nabla_\textup{NC}e_j)_{L^2(\Omega)} \notag\\
    &= 
    \alpha\Vert e_j\Vert_{L^2(\Omega)}^2
    + (E_j,\nabla_\textup{NC}e_j)_{L^2(\Omega)}.
  \end{align}
  Abbreviate $\mu_j\coloneqq \max\{1,|\Lambda_{j-1}+\tau
  \nabla_\textup{NC}\tilde{u}_j|\}$.
  Utilize \eqref{equ:PrimalDualAlgLambdaJ} to compute
  \begin{align}
    \label{equ:convProofA}
    \Lambda_{j-1}-\Lambda_j+\tau \nabla_\textup{NC}\tilde{u}_j 
    = (\mu_j-1)\Lambda_j \quad\text{ a.e. in }\Omega.
  \end{align}
  For all $x\in\Omega$ the Cauchy-Schwarz inequality yields
  $\Lambda_j(x)\cdot\Lambda_0(x)\leq|\Lambda_j(x)|$ 
  and by definition of $\Lambda_j$ a simple case distinction leads to
  $(1-|\Lambda_j(x)|)(\mu_j(x)-1)=0$.
  Test \eqref{equ:convProofA} with $E_j$ to compute
  \begin{align*}
    ( (\Lambda_{j-1}-\Lambda_j)/\tau+\nabla_\textup{NC}\tilde{u}_j,
    E_j)_{L^2(\Omega)}
    &= 
    1/\tau ( (\mu_j-1)\Lambda_j,\Lambda_0-\Lambda_j)_{L^2(\Omega)}\\
    &\leq
    1/\tau \int_\Omega (\mu_j-1)(|\Lambda_j|-|\Lambda_j|^2)\,\mathrm dx\\
    &=
    1/\tau \int_\Omega |\Lambda_j|
    \underbrace{(1-|\Lambda_j|)(\mu_j-1)}_{=0}\,\mathrm dx =0.
  \end{align*}
  With $\Lambda_{j-1}-\Lambda_j=E_j-E_{j-1}$, 
  $ $
  for $j\geq 2$ and for $j=1$ the convention $e_{-1}\coloneqq e_0$
  this leads for all $j\in \mathbb N$ to
  \begin{align}
    \label{equ:convProofB}
    ( (E_j-E_{j-1})/\tau-\nabla_\textup{NC}(e_{j-1}-e_{j-2})+
    \nabla_\textup{NC}u_{j-1},E_j)_{L^2(\Omega)}\leq 0.
  \end{align}

  

\end{proof}

\subsection{Data structures }
The realization of Algorithm \ref{alg:PrimalDualIteration} utilizes the data
structures \texttt{c4n}, 
\texttt{n4e}, \texttt{s4e}, \texttt{n4s}, \texttt{n4sDb}, \texttt{n4sNb}, \texttt{s4n},
and \texttt{area4e}
from \cite{Carstensen2017script} and Section 7.8 of \cite{CarstensenBrennerFEM}.\\
Let $\mathcal{T}$ be a regular triangulation of the polygonal bounded
Lipschitz domain $\Omega$ into triangles with set of nodes $\mathcal{N}$ and
set of edges
$\mathcal{E}$, and the set of inner nodes $\mathcal{N}(\Omega)$ and set of inner edges
$\mathcal{E}(\Omega)$. The data structure of $\mathcal T$ and $\mathcal E$ is described
as follows.\\
The $j$-th row \texttt{c4n(j,:)=[$x$ $y$]} of \texttt{c4n} contains the
coordinates of the $j$-th 
node $z_j=(x,y)$  
of $\mathcal T$.
\texttt{n4e(j,:)=[$\alpha$ $\beta$ $\gamma$]} contains
the three node numbers of the $j$-th triangle 
$T_j=\operatorname{conv}\{P_\alpha,P_\beta,P_\gamma\}$
in counterclockwise order with the refinement edge
$E_\textup{Ref}=\operatorname{conv}
\{P_\alpha,P_\beta\}$
of
$T_j$.
Similar  for
$\mathcal E(T_j) = \{ E_\alpha,E_\beta,E_\gamma\}$
the $j$-th row 
\texttt{s4e(j,:)=[$\alpha$ $\beta$ $\gamma$]} 
of \texttt{s4e}
contains
the three side numbers of the $j$-th triangle $T_j$
in counterclockwise order 
where
$E_\textup{Ref}=E_\alpha$
is the refinement edge of
$T_j$. 
\texttt{n4s(j,:):=[$\alpha$ $\beta$]} contains
the two node numbers of the \(j\)-th edge 
$E_j = \operatorname*{conv} \{ P_\alpha, P_\beta \}$
of $\mathcal T$.
For a boundary edge
\(E_j = \operatorname*{conv} \{ P_\alpha, P_\beta \}\in\mathcal{E}\setminus\mathcal{E}(\Omega)\)
choose $\alpha$ and $\beta$ such that the domain \(\Omega\) is on the left hand side
of the vector from \(P_\alpha\) to \(P_\beta\).
The following three lines of Matlab code compute \texttt{n4s}.
\begin{lstlisting}[frame=single,numbers=none]
allSides=[n4e(:,[1 2]); n4e(:,[2 3]); n4e(:,[3 1])];
[b,ind]=unique(sort(allSides,2),'rows','first');
n4s=allSides(sort(ind),:);
\end{lstlisting}
\texttt{n4sDb} contains the numbers of the edges with Dirichlet boundary data and 
\texttt{n4sNb} contains those of the edges with Neumann boundary data.
With \texttt{n4s} the computation of the data
structure \texttt{s4n} is possible which satisfies
\begin{align*}
\texttt{s4n($\alpha$,$\beta$)} =
\texttt{s4n($\beta$,$\alpha$)} =
\begin{cases}
 \ell & \text{if there exist } E_\ell = \operatorname*{conv} \{ P_\alpha, P_\beta \},\\
    0 & \text{else}
\end{cases}
\end{align*}
for any given nodes $\alpha,\beta\in\mathcal{N}$.
\texttt{s4n=sparse(n4s(:,1),n4s(:,2),1:S,N,N)}
creates a sparse matrix with
\begin{equation*}
  \texttt{s4n(n4s(j,1),n4s(j,2))} = \texttt j \quad \text{ for } j = 1,\ldots,M := \lvert 
  \mathcal{E} \rvert
\end{equation*}
and the last two parameters ensure that the resulting
Matrix is \(N \times N\) with
$
\texttt N = \lvert \mathcal{N} \rvert
$. 
\begin{lstlisting}[frame=single,numbers=none]
M=size(n4s,1); N=size(c4n,1);
s4n=sparse(n4s(:,1),n4s(:,2),1:M,N,N);
s4n=s4n + s4n';
\end{lstlisting}
The AFEM function \texttt{computeArea4e.m} computes the 
array \texttt{area4e} that contains the area of the $j$-th 
triangle of $\mathcal T$ 
in its $j$-th row.
\subsection{Basis of Crouzeix-Raviart functions \cite{Carstensen2017script}}
Given an edge $E\in\mathcal E$, the associated edge-oriented basis function
$\psi_E\in\textup{CR}^1(\mathcal T)$ is characterized by
\[
\psi_E(\operatorname{mid}(F))=\delta_{EF} \text{ for all } F\in\mathcal E.
\]
Then $(\psi_j)_{j=1,\dots,|\mathcal{E}|}$ is a basis of $\textup{CR}^1(\Omega)$
and $(\psi_j \, | \, E_j\in\mathcal{E}(\Omega))$ is a basis of $\textup{CR}^1_0(\Omega)$.
It is established that on a triangle $T\in\mathcal T$ with $E\in\mathcal E (T)$ the
function
$\psi_E$ has the representation
\[
\psi_E = 1-2\varphi_P \text{ in } T
\] using the barycentric coordinate $\varphi_P$ with respect to 
the corner $P\in\mathcal N (T)$ of $T$ opposite to $E$.

\subsection{Discrete gradients }\label{sec:DiscreteGradients}

To compute the non-conforming gradient of a Crouzeix-Raviart function, the
gradients of the conforming basis $(\phi_1,\phi_2,\phi_3)$ functions on a triangle $T\in\mathcal T$
are (\cite{Carstensen2017script},\cite{CarstensenBrennerFEM})
\begin{equation*}
  \begin{pmatrix}
    1 & 1 & 1\\
    P_1 & P_2 & P_3
  \end{pmatrix}
  \begin{pmatrix}
    \nabla \varphi_1\\
    \nabla \varphi_2\\
    \nabla \varphi_3\\
  \end{pmatrix}
  =
  \begin{pmatrix}
    0 & 0 \\
    1 & 0\\
    0&1
  \end{pmatrix}.
\end{equation*}
Consequently
\begin{equation*}
  \begin{pmatrix}
    1 & 1 & 1\\
    P_1 & P_2 & P_3
  \end{pmatrix}
  \begin{pmatrix}
    \nabla \psi_1|_T\\
    \nabla \psi_2|_T\\
    \nabla \psi_3|_T\\
  \end{pmatrix}
  =
  \begin{pmatrix}
    1 & 1 & 1\\
    P_1 & P_2 & P_3
  \end{pmatrix}
  \begin{pmatrix}
   -2 \nabla \varphi_1\\
   -2 \nabla \varphi_2\\
   -2 \nabla \varphi_3\\
  \end{pmatrix}
  =
  \begin{pmatrix}
    0 & 0 \\
    -2 & 0\\
    0&-2
  \end{pmatrix}
\end{equation*}
computes the gradients of the CR basis functions. 
$\texttt{c4n(n4e(j,:),:)}\in\mathbb R^{3\times 2}$ 
are the coordinates of the nodes of a triangle
$T=\operatorname{conv}\{P_\alpha,P_\beta,P_\gamma\}$ with 
\texttt{n4e(elem,:)=[$\alpha$ $\beta$ $\gamma$]}, i.e. 
\[
\texttt{c4n(n4e(j,:),:)=[$P_\alpha^T$; $P_\beta^T$; $P_\gamma^T$]}.
\]
Hence for the triangle $T\in\mathcal T$ of number $j$,
\[
\texttt{[ones(1,3); c4n(n4e(j,:),:)']\textbackslash[zeros(1,2); -2*eye(2)]}
\]
are the gradients of the CR basis functions on $T$.

These can be used to calculate the gradient of 
$u_{\textup{CR}} \in \textup{CR}^1(\mathcal T)$ on $T$
since $u_{\textup{CR}}$ on $T$ can be represented as
\[
u_{\textup{CR}}|_T = \sum_{E\in\mathcal E (T)}u_E\psi_E|_T.
\]
Therefore
\[
\nabla u_{\textup{CR}}|_T = \sum_{E\in\mathcal E (T)}u_E\nabla\psi_E|_T.
\]
\texttt{u(s4e(j,:))} determines the three
coefficients $(u_E|E\in\mathcal E)$ of $u_{\textup{CR}}$
on $T$.
\texttt{s4e(j,:)} contains the global edge numbers 
of $T$ in counterclockwise order and
the first global edge number of $T$ in \texttt{s4e(j,:)} is the number
of the refinement edge $E_\textup{Ref}$.
This implies 
$$
\texttt{u(s4e(j,1))}=u_{E_\textup{Ref}}.
$$
On the other hand 
\begin{lstlisting}[frame = single, numbers = none]
[ones(1,3); c4n(n4e(j,:),:)']\[zeros(1,2); -2*eye(2)]
\end{lstlisting}
computes the three gradients of the basis functions on $T$ in counterclockwise 
order and the first gradient with respect to
the edge opposit to the first global node in \texttt{n4e(j,:)}, i.e. the
first of the gradients is not with respect to the refinement edge of $T$. \\
\texttt{c4n(n4e(j,[3 1 2]),:)'} permutates the coordinates of the nodes such that the
order of the gradients computed by
\begin{lstlisting}[frame = single, numbers = none]
[ones(1,3); c4n(n4e(j,[3 1 2]),:)']\[zeros(1,2); -2*eye(2)]
\end{lstlisting}
matches the order of the coefficients.
Hence
\begin{lstlisting}[frame=single,numbers=none]
du(j,:) = u(s4e(j,:))' * ...
    [ones(1,3); c4n(n4e(j,[3 1 2]),:)']\[zeros(1,2); -2*eye(2)]
\end{lstlisting}
computes the gradient of $u_{\textup{CR}}$ on the $j$-th element $T$.
The function \texttt{GradientNC.m}
uses this to compute the gradients of $u_{\textup{CR}}$ on every triangle
$T\in\mathcal T$. The inputs of the function are \texttt{c4n},
\texttt{n4e}, \texttt{s4e}, the number of elements \texttt{nrElems} in $\mathcal T$,
and the edge-oriented components of $u_{\textup{CR}}$ in \texttt{u}.
The output of the function is the array 
$\texttt{du}\in\mathbb R^{|\mathcal T|\times 2}$. It contains
the gradient of $u_{\textup{CR}}$ on
  the $j$-th element in the $j$-th component.
\begin{lstlisting}[frame = single,numbers=none]
function du = GradientNC(c4n,n4e,s4e,nrElems,u)
  du = zeros(nrElems,2);
  for elem = 1:nrElems
    du(elem,:) = u(s4e(elem,:))' * ...
        [ones(1,3);c4n(n4e(elem,[3 1 2]),:)']\[zeros(1,2);-2*eye(2)];          
  end
end
\end{lstlisting}




\subsection{Finite elemente matrices (\cite{Carstensen2017script},
\cite{CarstensenBrennerFEM})}
 
The entries of the non-conforming stiffness matrix are
\begin{equation*}
  \texttt{A}_{j,k} = 
  a_{\textup{NC}}(\psi_j,\psi_k)
  \coloneqq
  \int_{\Omega}\nabla_{\textup{NC}}\psi_j\cdot\nabla_{\textup{NC}}\psi_k\, \mathrm{dx}
\end{equation*}
and the entries of the non-conforming mass matrix are
\begin{equation*}
  \texttt{M}_{j,k} = 
  (\psi_j,\psi_k)_{L^2(\Omega)} =
  \int_{\Omega}\psi_j\psi_k\, \mathrm{dx}
\end{equation*}


The local non-conforming mass matrix on a triangle $T$ reads
\begin{equation*}
  \textup{M}(T) \coloneqq
  \left(
  \int_T \psi_j\psi_k \, \mathrm{dx}\right)_{j,k=1,2,3}
  =
  \frac{|T|}{3}
  I_{3\times3}
\end{equation*}

Furthermore the non-conforming gradient $\nabla_{\textup{NC}}$ of a CR basis function is 
constant on any triangle $T\in\mathcal{T}$, hence
\begin{align*}
  \textup{A}(T) &\coloneqq
  \left(
  \int_T \nabla \psi_j\cdot\nabla\psi_k \, \mathrm{dx}\right)_{j,k=1,2,3}
  =
   \int_T 1 \, \mathrm{dx}\left( \nabla \psi_j\cdot\nabla\psi_k \right)_{j,k=1,2,3}\\
 &=
  |T|
  \begin{pmatrix}
    \nabla \psi_1\\
    \nabla \psi_2\\
    \nabla \psi_3\\
  \end{pmatrix}
  \begin{pmatrix}
    \nabla \psi_1^T&
    \nabla \psi_2^T&
    \nabla \psi_3^T\\
  \end{pmatrix}.
\end{align*}
Using this observations the function \texttt{FeMatrices.m} assembles
the non-conforming mass matrix and non-conforming stiffness matrix.
The inputs of the function are \texttt{c4n},
\texttt{n4e}, \texttt{s4e}, \texttt{area4e}, and the number of elements \texttt{nrElems} in $\mathcal T$.
The outputs of the function are the non-conforming stiffness matrix \texttt{A} and the
non-conforming mass matrix \texttt{M} as sparse matrices.
First
\begin{lstlisting}[frame = single, numbers = none]
  Alocal = zeros(3,3,nrElems);
  Mlocal = zeros(3,3,nrElems);
\end{lstlisting}
preallocates memory for the local non-conforming matrices.
\texttt{Mlocal} and \texttt{Alocal} are both $3\times 3\times 
|\mathcal{T}|$ arrays. 
The following loop over all triangles uses this
and the computation of the discrete gradients in section \ref{sec:DiscreteGradients} 
to assemble the 
local non-conforming stiffness and mass matrices. 

\begin{lstlisting}[frame= single, numbers = none]
  for elem = 1 : nrElems
      gradsNC = [ones(1,3); c4n(n4e(elem,:),[3 1 2])']\ ...
          [zeros(1,2); -2*eye(2)]; 
      Alocal(:,:,elem) = area(elem) * (gradsNC * gradsNC'); 
      Mlocal(:,:,elem) = area(elem) * Mlocal(:,:,elem); 
  end
\end{lstlisting}
Then the global non-conforming stiffness and mass matrix can be assembled.
\texttt{I} and \texttt{J} are used to identify local side numbers of the elements
with the global side numbers of the triangulation by specifying the rows and colums
the entries must have in the global matrices. Note that while assembling the sparse matrices, 
in case there already is an entry at a position 
in the matrix, the \texttt{sparse} function sums up the new number and the already existing one.

\begin{lstlisting}[frame = single, numbers =none]
  s4eT = s4e';
  I = [s4eT;s4eT;s4eT];
  J = [s4eT(:),s4eT(:),s4eT(:)]';
  A = sparse(I(:),J(:),Alocal(:));
  M = sparse(I(:),J(:),Mlocal(:));
\end{lstlisting}

The complete function reads
\begin{lstlisting}[frame = single, numbers = none]
function [A,M] = FeMatrices(c4n,n4e,s4e,area4e,nrElems)
  Alocal = zeros(3,3,nrElems);
  Mlocal = zeros(3,3,nrElems);
  
  for elem = 1 : nrElems
    gradsNC = [ones(1,3); c4n(n4e(elem,[3 1 2]),:)']\ ...
        [zeros(1,2); -2*eye(2)]; 
    Alocal(:,:,elem) = area(elem) * (gradsNC * gradsNC'); 
    Mlocal(:,:,elem) = area(elem) * eye(3)/3; 
  end
  
  s4eT = s4e';
  I = [s4eT;s4eT;s4eT];
  J = [s4eT(:),s4eT(:),s4eT(:)]';
  A = sparse(I(:),J(:),Alocal(:));
  M = sparse(I(:),J(:),Mlocal(:));
end
\end{lstlisting}




\subsection{Reformulation of Algorithm \ref{alg:PrimalDualIteration} for implementation}

In an iteration step of Algorithm \ref{alg:PrimalDualIteration} with given
$\Lambda_{j-1}, u_{j-1}$ and $v_{j-1}\coloneqq\frac{u_{j-1}-u_{j-2}}{\tau}$ 
the equation that is to be solved in $\textup{CR}^1_0(\mathcal T)$ is
\begin{equation*} 
\frac{1}{\tau}a_{\textup{NC}}(u_j,\bullet)+\alpha(u_j,\bullet)_{L^2(\Omega)}
=
\frac{1}{\tau}a_{\textup{NC}}(u_{j-1},\bullet) + (f,\bullet)_{L^2(\Omega)}
- (\Lambda_j,\nabla_{\textup{NC}}\bullet)_{L^2(\Omega)}.
\end{equation*}
This leads to the problem of finding a solution to
\begin{equation}\label{equ:LinearSystem}
\left( \frac{1}{\tau}A+\alpha M\right) x=b  
\end{equation}
where
$b_k\coloneqq
\left(
\frac{1}{\tau}
\nabla_{\textup{NC}}u_{j-1}-\Lambda_j,\nabla_{\textup{NC}}\psi_k
\right)_{L^2(\Omega)}+(f,\psi_k)_{L^2(\Omega)}
$.


The left-hand side matrix can easily be calculated after the non-conforming
stiffness and mass matrix are assembled with
\begin{lstlisting}[style=inline]
A/tau+alpha*M; 
\end{lstlisting}





\subsection{Right-hand side $b$}
To obtain $b$ the function \texttt{Integrals.m} evaluates the integrals 
$$
\int_T f\psi_k\,\mathrm{dx} \text{ for all } T\in\mathcal T \text{ and for all }
k=1,\dots,|\mathcal E|.
$$
The inputs are a function handle of the function $f$ in \texttt{f}, \texttt{c4n},
\texttt{n4e}, the algebraic degree of exactness the numerical quadrature will have, and 
\texttt{area4e}.
The output is, for $j=1,2,3$, an array 
\texttt{tempj} that contains the value of the approximation of 
$
\int_{T_k} f\psi_j\,\mathrm{dx} 
$
in the $k$-th component for $k=1,\dots,|\mathcal T|$.

To use the AFEM packages function \texttt{integrate.m} the integrands have to
be defined and therefore function handles of the local CR basis functions are needed.

\begin{lstlisting}[frame = single, numbers = none]
  phi1 = @(x)( x(:,1) );
  phi2 = @(x)( x(:,2) );
  phi3 = @(x)( 1 - x(:,1) - x(:,2) );
  
  psi1 = @(x)( 1-2*phi3(x) );
  psi2 = @(x)( 1-2*phi1(x) );
  psi3 = @(x)( 1-2*phi2(x) );
\end{lstlisting}
\texttt{phij} for $j=1,2,3$ are function handles of the barycentric coordinates 
on the reference triangle
$T_\textup{Ref}=\operatorname{conv}((0,0),(1,0),(0,1))$. They are used 
to define the CR basis functions \texttt{psij} for $j=1,2,3$ on $T_{\textup{Ref}}$. 
The edge numeration convention from \texttt{s4e} is used and therefore the first edge is the 
refinement edge.\\
Then the integrands are defined where the CR basis functions will be evaluated on the reference
triangle \texttt{psij(Gpts4ref)}, $j=1,2,3$, and transformed to a element $T\in\mathcal T$.
\begin{lstlisting}[frame=single,numbers=none]
  INT1 = @(n4p,Gpts4p,Gpts4ref) psi1(Gpts4ref).*f(Gpts4p);
  INT2 = @(n4p,Gpts4p,Gpts4ref) psi2(Gpts4ref).*f(Gpts4p);        
  INT3 = @(n4p,Gpts4p,Gpts4ref) psi3(Gpts4ref).*f(Gpts4p);
\end{lstlisting}
The integrals are evaluated via \texttt{integrate.m} and \texttt{area4e} will be used
for the transformations from the reference triangle to the elements of $\mathcal T$.

\begin{lstlisting}[frame=single,numbers=none]
  temp1 = integrate(INT1, c4n, n4e, degree, area4e);
  temp2 = integrate(INT2, c4n, n4e, degree, area4e);
  temp3 = integrate(INT3, c4n, n4e, degree, area4e);
\end{lstlisting}
The complete function reads

\begin{lstlisting}[frame=single,numbers=none]
function [temp1,temp2,temp3] = computeIntegrals(f,c4n,n4e,degree,area4e)
  phi1 = @(x)( x(:,1) );
  phi2 = @(x)( x(:,2) );
  phi3 = @(x)( 1 - x(:,1) - x(:,2) );
  
  psi1 = @(x)( 1-2*phi3(x) );
  psi2 = @(x)( 1-2*phi1(x) );
  psi3 = @(x)( 1-2*phi2(x) );
  
  INT1 = @(n4p,Gpts4p,Gpts4ref) psi1(Gpts4ref).*f(Gpts4p);
  INT2 = @(n4p,Gpts4p,Gpts4ref) psi2(Gpts4ref).*f(Gpts4p);        
  INT3 = @(n4p,Gpts4p,Gpts4ref) psi3(Gpts4ref).*f(Gpts4p);
  
  temp1 = integrate(INT1, c4n, n4e, degree, area4e);
  temp2 = integrate(INT2, c4n, n4e, degree, area4e);
  temp3 = integrate(INT3, c4n, n4e, degree, area4e);
end
\end{lstlisting}

The function \texttt{RightHandSide.m} uses \texttt{tempj}, $j=1,2,3$, to compute the right-hand
side $b$ with
$$
b_k\coloneqq
\left(
\frac{1}{\tau}
\nabla_{\textup{NC}}u_{j-1}-\Lambda_j,\nabla_{\textup{NC}}\psi_k
\right)_{L^2(\Omega)}+(f,\psi_k)_{L^2(\Omega)} \text{ for all } k=1,\dots,|\mathcal E|.
$$
The inputs of the function are \texttt{c4n},
\texttt{n4e}, \texttt{s4e}, the number of sides \texttt{nrSides} in $\mathcal T$,
\texttt{area4e}, the discrete gradient \texttt{du} of $u_{j-1}$ computed by the function
\texttt{GradientNC} in section \ref{sec:DiscreteGradients}, the parameter $\texttt{tau}=\tau$,
$\texttt{Lambda}=\Lambda_j$, the number of elements \texttt{nrElems} in $\mathcal T$, and 
\texttt{tempj}, $j=1,2,3$, the arrays with the value of the integral
$
\int_{T_k} f\psi_j\,\mathrm{dx} 
$
in the $k$-th component for $k=1,\dots,|\mathcal T|$.
The outputs are the right-hand side $\texttt{b}=b$ and the vector 
$\texttt{temp}=\left( \int_\Omega f\, \psi_k \, \mathrm{dx} \right)_{k=1,\dots,|\mathcal E|}$ 
that will be used to compute the discrete energy.

First memory for \texttt{b} and \texttt{temp} is preallocated.
\begin{lstlisting}[frame=single,numbers=none]
b = zeros(nrSides,1);
temp = zeros(nrSides,1);
\end{lstlisting}
Given \texttt{elem} the number of an arbitrary element $T\in\mathcal T$
\begin{lstlisting}[frame=single,numbers=none]
gradsNC = [ones(1,3); c4n(n4e(elem,[3 1 2]),:)']\ ...
    [zeros(1,2); -2*eye(2)];
\end{lstlisting}
computes the gradients of the CR basis functions on T as seen in section 
\ref{sec:DiscreteGradients}. Furthermore 
for all $k=1,\dots,|\mathcal E|$
\begin{align*}
\tilde{b}_k 
&\coloneqq
\left(
\frac{1}{\tau}
\nabla_{\textup{NC}}u_{j-1}-\Lambda_j,\nabla_{\textup{NC}}\psi_k
\right)_{L^2(\Omega)}\\
&=
\sum_{T\in\mathcal T}
\left(
\frac{1}{\tau}
\nabla u_{j-1}|_T-\Lambda_j,\nabla \psi_k|_T
\right)_{L^2(T)} \\
&=
\sum_{T\in\mathcal T}
|T|
\left(\frac{1}{\tau}
\nabla u_{j-1}|_T-\Lambda_j\right)\cdot\nabla \psi_k|_T\\
&=
\sum_{\texttt{elem}=1}^{|\mathcal T|} \texttt{area4e(elem)}
\left(\texttt{du(elem,:)/tau-Lambda(elem,:)}\right)\texttt{gradsNC'}_k
\end{align*}
and
\begin{align*}
  \texttt{temp} &=(f,\psi_k)_{L^2(\Omega)} 
=\sum_{T\in\mathcal T}(f,\psi_k)_{L^2(T)} \\
.
\end{align*}
Note that $ b = \tilde{b}+\texttt{temp}.$
Therefore \begin{lstlisting}[frame=single,numbers=none]
for elem = 1 : nrElems
    gradsNC = [ones(1,3); c4n(n4e(elem,[3 1 2]),:)']\ ...
        [zeros(1,2); -2*eye(2)];
    bLocal = ( du(elem,:)/tau - Lambda(elem,:) ) * gradsNC';  
    temp(s4e(elem,:)) = temp(s4e(elem,:)) + ...
        [temp1(elem),temp2(elem),temp3(elem)]';
    b(s4e(elem,:)) = b(s4e(elem,:)) + area4e(elem)*bLocal'; 
end
b = b + temp;
\end{lstlisting}
computes \texttt{b} in a loop over all triangles in $\mathcal T$.

The complete function reads
\begin{lstlisting}[frame=single,numbers=none]
function [b,temp] = RightHandSide(c4n,n4e,s4e,nrSides,area4e,du,tau, ...
    Lambda,nrElems,temp1,temp2,temp3)
  b = zeros(nrSides,1);
  temp = zeros(nrSides,1);
  for elem = 1 : nrElems
      gradsNC = [ones(1,3); c4n(n4e(elem,[3 1 2]),:)']\ ...
          [zeros(1,2); -2*eye(2)];
      bLocal = ( du(elem,:)/tau - Lambda(elem,:) ) * gradsNC';  
      temp(s4e(elem,:)) = temp(s4e(elem,:)) + ...
          [temp1(elem),temp2(elem),temp3(elem)]';
      b(s4e(elem,:)) = b(s4e(elem,:)) + area4e(elem)*bLocal'; 
  end
  b = b + temp;
end
\end{lstlisting}

\subsection{Degrees of freedom}
Since the solution $u_j$ of the linear system of equations in algorithm 
\ref{alg:PrimalDualIteration} is a function in $\textup{CR}^1_0(\mathcal T)$ 
only the edge-oriented coefficients for non-Dirchlet edges, the 
degrees of freedom for this system, have to be computed.
The function \texttt{DegreesOfFreedom.m} has the inputs 
\texttt{n4e}, the number of sides \texttt{nrSides} of the triangulation $\mathcal T$, and
\texttt{n4sDb} and computes the degrees of freedom \texttt{dof} as output as a vector
that contains the global numbers of the sides in $\mathcal E$ that are not Dirichlet sides.
First \texttt{s4n} is assemled to compute the Dirichlet boundary sides in \texttt{DbSides}.
For the $j$-th pair of nodes in \texttt{n4sDb} 
\begin{lstlisting}[frame=single,numbers=none]
DbSides(j) = s4n(n4sDb(j,1),n4sDb(j,2));
\end{lstlisting}
computes the global number of the side that has this pair of nodes as end points.
This number is saved in \texttt{DbSides(j)} as the number of the $j$-Dirichlet boundary
side.
\begin{lstlisting}[frame=single,numbers=none]
s4n = computeS4n(n4e);

DbSides = zeros(1,size(n4sDb,1));
for j = 1:size(n4sDb,1)
    DbSides(j) = s4n(n4sDb(j,1),n4sDb(j,2));
end
\end{lstlisting}
Now the line
\begin{lstlisting}[frame=single,numbers=none]
dof = setdiff(1:nrSides,DbSides);
\end{lstlisting}
removes all numbers of Dirichlet boundary sides from a vector 
\texttt{1:nrSides = [1 2 3 ... nrSides]} and yields an array \texttt{dof} that
only contains the
global numbers of the degrees of freedom.
The complete code reads.

\begin{lstlisting}[frame=single,numbers=none]
function dof = DegreesOfFreedom(n4e,nrSides,n4sDb)
  s4n = computeS4n(n4e);
  
  DbSides = zeros(1,size(n4sDb,1));
  for i = 1:size(n4sDb,1)
      DbSides(i) = s4n(n4sDb(i,1),n4sDb(i,2));
  end

  dof = setdiff(1:nrSides,DbSides);
end
\end{lstlisting}

\section{Solving of \ref{equ:LinearSystem}}
With the left-hand side matrix
$
 \frac{1}{\tau}A+\alpha M
$
, the right-hand side $b$ 
and the degrees of freedom of the linear system \ref{equ:LinearSystem} computed
it is now solved for all non-Dirichlet boundary sides.
The solution \texttt{uNew} contains the edge-oriented components of $u_j\in\textup{CR}^1_0(
\mathcal T)$ and is initialized as
\begin{lstlisting}[frame=single, numbers=none]
uNew = zeros(nrSides,1);
\end{lstlisting}
The system is only solved for the degrees of freedom \texttt{dof} and the zeros remain
for the components with respect do Dirichlet boundary sides.
\begin{lstlisting}[frame=single, numbers=none]
uNew(dof) = A(dof,dof)\b(dof);
\end{lstlisting}

\section{Discrete energy}

The function \texttt{Energy.m} computes the discrete energy
\begin{align*}
  E_\nc (u) 
  &=    
  \frac{\alpha}{2}   \|  u \|_{L^2(\Omega)}^2 
  + | u |_{1,1,\nc }  -\int_\Omega f\,  u \dx. \\
\end{align*}
The inputs are
\texttt{area4e},
the edge-oriented components \texttt{u} of a Crouzeix-Raviart function $u_{\textup{CR}}
\in \textup{CR}^1_0(\mathcal T)$, the gradient \texttt{du} of $u_{\textup{CR}}$, 
an array \texttt{temp} that contains the value of the integral
$
\int_{\Omega} f\psi_j\,\mathrm{dx} 
$
in the $j$-th component for $j=1,\dots,|\mathcal E|$,
$\texttt{alpha}=\alpha$,
and the non-conforming mass matrix \texttt{MAMANC}.
It holds for $u\in\textup{CR}^1(\mathcal T)$
\begin{align*}
  E_\nc (u) 
  &=    
  \frac{\alpha}{2}   \|  u \|_{L^2(\Omega)}^2 
  + | u |_{1,1,\nc }  -\int_\Omega f\,  u \dx \\
  &=
  \frac{\alpha}{2} (u,u)_{L^2(\Omega)} 
  + \int_\Omega |\nabla_{\textup{NC}}u|\,\mathrm{dx} 
  - \sum_{j=1}^{|\mathcal E|}\texttt{u(j)} \int_\Omega f\psi_j\,\mathrm{dx} \\
  &= 
  \frac{\alpha}{2} 
  \sum_{j,k=1}^{|\mathcal E|}\texttt{u(j)}(\psi_j,\psi_k)_{L^2(\Omega)}\texttt{u(k)}
  + \sum_{T \in \mathcal T}\int_T |\nabla u|_T| \,\mathrm{dx} 
  - \sum_{j=1}^{|\mathcal E|}\texttt{u(j)} \texttt{temp(j)} \\
  &= 
  \frac{\alpha}{2} 
  \texttt{u'*MAMANC*u}
  + \sum_{j=1}^{|\mathcal T|}|T_j||\texttt{du(j,:)}|
  - \texttt{u'*temp}\\
  &=
  \frac{\texttt{alpha}}{2} \texttt{u'*MAMANC*u}
  + \texttt{area4e'*sqrt(sum(du.$\wedge$2,2))}
  - \texttt{u'*temp}.
\end{align*}
Therefore the complete function reads.
\begin{lstlisting}[frame=single,numbers=none]
function ENew = Energy(area4e,u,du,temp,alpha,MAMANC)
  ENew = alpha/2 * u'*MAMANC*u + area4e'*sqrt(sum(du.^2,2)) - u'*temp;
end
\end{lstlisting}


\section{Main function}\label{fct:tvRegPrimalDual}

The function \texttt{tvRegPrimalDual.m} realizes Algorithm \ref{alg:PrimalDualIteration}.
Its inputs are \texttt{c4n}, \texttt{n4e}, \texttt{n4sDb}, the initial data 
$\texttt{u}=u_0$ and $\texttt{Lambda}=\Lambda_0$, 
$\texttt{f}=f$, $\texttt{alpha}=\alpha$, and \texttt{epsStop} for the termination criterion
of the iteration. The outputs are the solution of the iteration \texttt{u}, 
the value \texttt{corr} of the last iteration step that satisfies $\texttt{corr}<\texttt{epsStop}$,
and arrays \texttt{corrVec} and \texttt{energyVec} that contain the values of $\texttt{corr}_j$
and the discrete energy $E_{\textup{NC}}(u_j)$ in the $j$-th component. 

First $\tau$ is choosen and the function computes the number \texttt{nrElems} 
of triangles in $\mathcal T$.
Then the AFEM functions \texttt{computeArea4e.m} and 
\texttt{computeS4e.m} compute the data structures
\texttt{area4e} and \texttt{s4e}. The function uses \texttt{s4e} to compute the number 
\texttt{nrSides} of edges of $\mathcal T$ in $\mathcal E$.
\begin{lstlisting}[frame=single,numbers=none]
tau = 1/2;

nrElems = size(n4e,1);
area4e = computeArea4e(c4n,n4e);
s4e = computeS4e(n4e);
nrSides = max(max(s4e));
\end{lstlisting}
The function \texttt{DegreesOfFreedom.m} computes \texttt{dof} that contains 
the global side numbers of non-Dirichlet sides, the function \texttt{FeMatrices.m} computes 
the non-conforming stiffness and mass matrices, and the function \texttt{computeIntegrals.m} 
computes
for $j=1,2,3$ the array
\texttt{tempj} that contains the value of the approximation of 
$
\int_{T_k} f\psi_j\,\mathrm{dx} 
$
in the $k$-th component for $k=1,\dots,|\mathcal T|$.
\begin{lstlisting}[frame=single,numbers=none]
dof = DegreesOfFreedom(n4e,nrSides,n4sDb);

[STIMANC,MAMANC] = FeMatrices(c4n,n4e,s4e,area4e,nrElems);
A = STIMANC/tau+alpha*MAMANC; 

[temp1,temp2,temp3] = computeIntegrals(f,c4n,n4e,200,area4e);
\end{lstlisting}
Then the function \texttt{GradientNC.m} computes the gradients of $u_0$ on every triangle 
in $\mathcal T$, $v_0=0$ is 
initialized, the value \texttt{corr} is initialized, \texttt{corrVec} and \texttt{energyVec} 
to save the corresponding information during the iteration are initialized, 
and the energy \texttt{E} is
initialized.

\begin{lstlisting}[frame=single,numbers=none]
du = GradientNC(c4n,n4e,u);

v = zeros(nrSides,1);    

corr = epsStop+1; 
corrVec = [];
energyVec = [];

E = 1;
\end{lstlisting}
After the initializations are completed a while loop realizes the iteration
described in Algorithm \ref{alg:PrimalDualIteration}. The termination criterion
is reached when 
$$
\texttt{epsStop}\geq|E_\textup{NC}(u_j)-E_\textup{NC}(u_{j-1}|).
$$
In the beginning of the loop the gradient of $v_{j-1}$ and 
$M=\Lambda +\tau\nabla_\textup{NC}+\tau^2\nabla_\textup{NC}$ are
computed.

\begin{lstlisting}[frame=single,numbers=none]
dv = GradientNC(c4n,n4e,v);
M = Lambda + tau*(du + tau*dv);
\end{lstlisting}

The line
\begin{lstlisting}[frame=single,numbers=none]
Lambda = bsxfun(@rdivide,M,max(1,sqrt(sum(M.^2,2))));
\end{lstlisting}
of MATLAB code computes $\Lambda_j$ by row-wise division of \texttt{M} by 
$\operatorname{max}\{1,|\texttt{M}_j|$\}.
The function \texttt{RightHandSide.m} computes the right-hand side $b$ and 
an array \texttt{temp} that contains the value of the integral
$
\int_{\Omega} f\psi_j\,\mathrm{dx} 
$
in the $j$-th component for $j=1,\dots,|\mathcal E|$.
\begin{lstlisting}[frame=single,numbers=none]
[b,temp] = RightHandSide(c4n,n4e,s4e,nrSides,area4e,du, ...
    tau,Lambda,nrElems,temp1,temp2,temp3);     
\end{lstlisting}
Now
\begin{lstlisting}[frame=single,numbers=none]
uNew = zeros(nrSides,1);
uNew(dof) = A(dof,dof)\b(dof);
\end{lstlisting}
computes $u_j=\texttt{uNew}$,
\begin{lstlisting}[frame=single,numbers=none]
v=(uNew-u)/tau;        
\end{lstlisting}
computes $v_j=\texttt{v}$, and
\begin{lstlisting}[frame=single,numbers=none]
u = uNew;

du = GradientNC(c4n,n4e,u);
ENew = Energy(area4e,u,du,temp,alpha,MAMANC);
\end{lstlisting}
computes the gradient \texttt{du} of $u_{j}$ for the next iteration step and to compute
the energy with the function \texttt{Energy.m}.
With the now computed energy \texttt{E}
\begin{lstlisting}[frame=single,numbers=none]
corr = abs(ENew-E);
E = ENew;
energyVec(end+1) = E;
corrVec(end+1) = corr;
\end{lstlisting}
computes \texttt{corr} to check wheter to terminate the iteration or not. Also
the values \texttt{E} and \texttt{corr} are saved in \texttt{energyVec} and 
\texttt{corrVec} for later use.

The complete function reads


\begin{lstlisting}[frame=single,numbers=none]
function  [u,corr,corrVec,energyVec] = tvRegPrimalDual(c4n,n4e,n4sDb, ...
    u,Lambda,f,alpha,epsStop) 
  
    tau = 1/2;

    nrElems = size(n4e,1);
    area4e = computeArea4e(c4n,n4e);
    s4e = computeS4e(n4e);
    nrSides = max(max(s4e));

    dof = DegreesOfFreedom(n4e,nrSides,n4sDb);

    [STIMANC,MAMANC] = FeMatrices(c4n,n4e,s4e,area4e,nrElems);
    A = STIMANC/tau+alpha*MAMANC; 

    [temp1,temp2,temp3] = computeIntegrals(f,c4n,n4e,200,area4e);

    du = GradientNC(c4n,n4e,u);

    v = zeros(nrSides,1);    

    corr = epsStop+1; 
    corrVec = [];
    energyVec = [];

    E = 1;
    while corr > epsStop
        dv = GradientNC(c4n,n4e,v);
        M = Lambda + tau*(du + tau*dv);
        Lambda = bsxfun(@rdivide,M,max(1,sqrt(sum(M.^2,2))));
        
        [b,temp] = RightHandSide(c4n,n4e,s4e,nrSides,area4e,du, ...
            tau,Lambda,nrElems,temp1,temp2,temp3);     

        uNew = zeros(nrSides,1);
        uNew(dof) = A(dof,dof)\b(dof);
        v=(uNew-u)/tau;        
        u = uNew;

        du = GradientNC(c4n,n4e,u);
        ENew = Energy(area4e,u,du,temp,alpha,MAMANC);

        corr = abs(ENew-E);
        E = ENew;
        energyVec(end+1) = E;
        corrVec(end+1) = corr;
    end
end
\end{lstlisting}
