The continuous problem involves a low-order term  with positive  parameter
$\alpha > 0$  and a given right-hand side  $f\in L^2(\Omega)$ in a bounded
polyhedral  Lipschitz domain 
$\Omega\subset \R^n$ and minimizes
\begin{align}\label{e:defEintro} 
  E(v)& :=    
  \frac{\alpha}{2} \|  v
  \|_{L^2(\Omega)}^2 + \lvert v\rvert_{\BV(\Omega)}
  + \| v \|_{L^1(\partial\Omega)}  -\int_\Omega f\, v\dx 
\end{align} amongst all
functions $v\in V:=\BV(\Omega)\cap L^2(\Omega)$.  The  $BV$ seminorm $\lvert
v\rvert_{\BV}$ is  the total variation $\int_\Omega |\nabla v|\, dx $ and
well defined for $v\in \BV(\Omega)$ and is equal to the $W^{1,1}$ seminorm
for $v\in W^{1,1}(\Omega)$. The trace theorem for $\BV$ functions shows that
they are Lebesgue functions along the boundary and so the  term $ \|  v
\|_{L^1(\partial\Omega)}$ is well defined and models homogeneous boundary
conditions. 

It is well established that the energy has a unique minimizer $u$ in $V$ and it
is a corollary of a lemma of Brezis that, for a convex domain and $f\in
H^1(\Omega)$,  $u$ belongs to $ H^1_0(\Omega)$.

The nonconforming discretization of this section is nonconforming in two
aspects. First it utilizes nonconforming $P_1$ finite element functions named
after Crouzeix-Raviart \cite{CrouzeixRaviart1973}, 
\begin{align*}
  \CR(\T):=\{\vcr\in P_1(\T)\;\vert\;\vcr \text{ is continuous at midpoints of
  interior}\\ \text{sides and vanishes at midpoints of boundary sides}\}.
\end{align*} 
Here and throughout this paper,  $P_k(\T)$ abbreviates the
piecewise polynomials of total degree at most $k$ with respect to a
shape-regular triangulation $\T$ of $\Omega$ into simplices with the set of
nodes $\N$ and set of sides $\F$ (resp. interior sides $\F(\Omega)$);  the
$L^2$ projection onto $P_0(\T)$  is the averaging operator $\Pi_0$.
 

The subsequent discrete problem is nonconforming in a second way for
it replaces the distributional 
gradient $\nabla$ by its piecewise action 
% of it, written $\nabla_\nc$ and,  for  any $v\in H^1(\T)$, 
% defined as   $\nabla_\nc v \in L^2(\Omega;\R^n)$ on $T\in\T$ by 
% $(\nabla_\nc v )|_T:= \nabla ( v|_{\operatorname{int}(T)})$; 
and abbreviate
% the discrete substitute
$|\bullet |_{1,1,\nc}\coloneqq
\|  \nabla_\nc\bullet\|_{L^1(\Omega)}$. 
% of the $\BV(\Omega)+L^2(\partial\Omega)$  seminorms. 
It is well understood that 
\[
\lvert \vcr \rvert_{\BV}+  \|  \vcr \|_{L^1(\partial\Omega)} 
=   | \vcr  |_{1,1,\nc} + \sum_{F\in\mathcal{F}} \int_F  | [\vcr]_F|\, ds 
\]
with the jump  $[\vcr]_F$ of $\vcr$ across the interior side $F$ 
and $[\vcr]_F \coloneqq \vcr|_F$ for a boundary side $F\subset\partial\Omega$
for all $\vcr\in \textup{CR}_0^1(\T)$. 


The nonconforming problem minimizes
\begin{align}\label{e:defnonconfEintro}
E_\nc (\vcr) :=    \frac{\alpha}{2}   \|  \vcr \|_{L^2(\Omega)}^2 
+ | \vcr |_{1,1,\nc }  -\int_\Omega f\,  \vcr \dx
\end{align}
amongst all $\vcr\in \CR(\T)$.  
 

The convex modulus function $|\bullet|$ has the multivalued subgradient $\sign$, 
defined by the singleton
$\sign F:= \{ F/|F| \}$ for $F\in\R^n\setminus\{0\}$ and 
the closed unit ball $\sign 0:= \overline{ B(0,1)}$ in 
$\R^n$ endowed   with  the Euclidean scalar product "$\cdot$".   


\begin{theorem}[characterization of discrete solutions] 
  \label{thm:characterizationDiscreteSolutions}
(a) There exist a unique discrete minimizer
$
\ucr=\operatorname{argmin} E_\nc ( \CR(\T)).
$\\  
(b) The minimizer 
$\ucr$ is equivalently characterized as the solution $\ucr\in \CR(\T)$ to the variational inequality
\begin{equation}\label{eqccvarioineq} \tag{VI}
(f-\alpha\ucr, \vcr-\ucr)_{L^2(\Omega)}\le | \vcr |_{1,1,\nc }-| \ucr |_{1,1,\nc }
\end{equation}
for all $\vcr\in \CR(\T)$.
\\
(c) The minimizer $\ucr\in \CR(\T)$ is equivalently characterized by 
the existence of some
$\Lambda\in P_0(\T; \overline{ B(0,1)})$ with 
$\Lambda\cdot\nabla_\nc \ucr=|\nabla_\nc\ucr|$ a.e. in $\Omega$ and 
\begin{equation}
\label{eqLambda} \tag{1.2a}
(f-\alpha\ucr, \vcr)_{L^2(\Omega)}=(\Lambda,\nabla_\nc \vcr)_{L^2(\Omega)}
\,\text{ for all }\vcr\in \CR(\T).
\end{equation}
\end{theorem}


\begin{proof}
Standard arguments on the quadratic growth and the continuity of the discrete 
energy $E_\nc (\vcr)$ with respect to 
$\vcr$ in the fixed finite-dimensional space  $\CR(\T)$ provide the existence 
of a minimizer $\ucr$ of $E_\nc$ in
$\CR(\T)$. Moreover, this minimizer is equivalently characterized as the 
solution of the variational inequality
and more details for (a) and (b) are omitted.  One constructive  way for the 
existence proof of either the discrete or the continuous minimization problem 
utilizes a regularization of the $L^1$ norm. 
For instance, the modulus 
 $|\bullet |$ may be replaced by a differentiable upper bound 
$|\bullet|_\varepsilon$, defined for any $\varepsilon>0$ by 
\[
| F |_\varepsilon:=  \sqrt{   \varepsilon^2+ F\cdot F }\quad\text{for all }F\in\R^n.
\]
This leads to a regularized  nonconforming  energy \eqref{e:defnonconfEintro} with the substitution of
$\int_\Omega |\nabla_\nc \vcr |_\varepsilon\dx$  for   $| \vcr |_{1,1,\nc }$. The same substitution applies to the
discrete variational inequality in (b),  which becomes an equality 
for $| \bullet |_\varepsilon$ is differentiable for any positive $\varepsilon$. For any $\varepsilon> 0$ there 
exists a unique minimizer to the regularized nonconforming energy and the necessary stationary condition applies for the smooth functional and results in 
that $\ucrvarepsilon \in \CR(\T)$  satisfies
\begin{equation*}
(f-\alpha\ucrvarepsilon , \vcr)_{L^2(\Omega)} =   \int_\Omega 
% \frac{ \nabla_\nc  \ucrvarepsilon \cdot \nabla_\nc \vcr}{ \sqrt{   \varepsilon^2+ |\nabla_\nc  \ucrvarepsilon|^2 }} 
\Lambda_\varepsilon \cdot \nabla_\nc \vcr \dx
\,\text{ for all }\vcr\in \CR(\T)
\end{equation*} 
with the abbreviation 
\begin{equation}\label{e:definepvarepsilon}
\Lambda_\varepsilon :=  
\frac{ \nabla_\nc  \ucrvarepsilon}{ \sqrt{   \varepsilon^2+ |\nabla_\nc  \ucrvarepsilon|^2 }}\in P_0(\T; \overline{ B(0,1)}).
\end{equation} 
The test with $\vcr=\ucrvarepsilon $ and standard arguments reveal that $\ucrvarepsilon \in \CR(\T)$ is bounded 
(in any norm for the fixed finite-dimensional vector space  $\CR(\T)$) as $\varepsilon \to 0^+$. 
Any accumulation point $(\ucr,\Lambda)\in \CR(\T)\times P_0(\T;\R^n)$ of bounded subsequences 
$(\ucrvarepsilon,\Lambda_\varepsilon)$  as $\varepsilon \to 0^+$ satisfies 
\begin{align}
\label{e:limitidentityforp0}
% \text{ and } \text{ }  \\ 
(f-\alpha\ucr, \vcr)_{L^2(\Omega)} &=   \int_\Omega \Lambda \cdot \nabla_\nc \vcr \dx
\,\text{ for all }\vcr\in \CR(\T) ; \\
\label{e:limitboundednessetalforp0}
|\Lambda|\le 1  \quad \text{and}&\quad
\Lambda\cdot \nabla_\nc\ucr=| \nabla_\nc\ucr| \quad\text{  a.e. in }\Omega.
\end{align}
Substitute the test function $\vcr\in \CR(\T)$ in  \eqref{e:limitidentityforp0} by  $\vcr-\ucr$ (for some fixed limit $\ucr$
and) any  $\vcr\in \CR(\T)$ to obtain with \eqref{e:limitboundednessetalforp0}
the identity 
\[
(f-\alpha\ucr, \vcr-\ucr)_{L^2(\Omega)}=  \int_\Omega \Lambda \cdot \nabla_\nc \vcr \dx - | \ucr |_{1,1,\nc }.
\]
Since    $|\Lambda|\le 1$  a.e. in $\Omega$ implies
$   \int_\Omega \Lambda \cdot \nabla_\nc \vcr \dx\le  | \vcr |_{1,1,\nc }$,
this becomes  the discrete variational inequality
(b). In other words, any selected accumulation point  $\ucr$ of the discrete solutions $\ucrvarepsilon \in \CR(\T)$
as $\varepsilon \to 0^+$ is equal to the unique solution $\ucr$  in (b). This implies convergence 
$\ucrvarepsilon \to \ucr$  as $\varepsilon \to 0^+$ but in general not for $\Lambda_\varepsilon$  from
\eqref{e:definepvarepsilon}. However,  any  accumulation point $\Lambda$ (of the possibly many choices) 
 leads to  \eqref{e:limitidentityforp0}-\eqref{e:limitboundednessetalforp0}.
\end{proof}

\begin{remark}[dual variable]
The condition \eqref{e:limitboundednessetalforp0} equivalently reads
$\Lambda\in  \sign \nabla_\nc \ucr$
for the discrete minimizer $\ucr$. If $\nabla_\nc\ucr \ne 0$ on $T\in\T$, then the dual 
variable $\Lambda=\nabla_\nc\ucr /|\nabla_\nc\ucr |$ is unique on $T$.  
\end{remark}

\begin{example}[nonuniquenss of dual variable]
Given $f\equiv 0$, the unique minimizer vanishes $\ucr\equiv 0$ a.e. in $\Omega$,
while any 
$v_C\in S^1(\T)$ with $|\text{Curl } v_C|\le 1$ a.e. in $\Omega$  leads to
$\Lambda:= \text{Curl } v_C\in P_0(\T; \overline{ B(0,1)})$ with \eqref{eqLambda}.
\end{example}


\begin{theorem}[convergence]
For any sequence $(\T_\ell)_{\ell\in\mathbb{N}}$ of meshes in $\mathbb{T}$ with 
respective discrete solutions $(\ucr^{(\ell)})_{\ell\in\mathbb{N}}$ to
\eqref{e:defnonconfEintro} 
and 
maximal mesh-sizes  $h_\ell \to 0^+$ as $\ell\to \infty$
the unique solution $u$ to \eqref{e:defEintro} satisfies
\[
\lim_{\ell\to\infty} || u- \ucr^{(\ell)}||_{L^2(\Omega)}=0. 
\]
%for the unique minimizer $u$ of the continuous problem.  

\end{theorem}

%The theorem leaves open the question whether the discrete energy converges 
%to the right function and hence numerical experiments need to tell us.







