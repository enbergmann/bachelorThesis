% 
% \newpage
% \section{Numerical Experiments}\label{sec:NumericalExperiments}
% Consider $f$ defined on the unit circle as a function of the radius $r$ by
% \[
% f(r)\coloneqq
% \begin{cases}
%   \alpha-12(2-9r) & \text{ for } 0\leq r \leq \frac{1}{6},\\
%   \alpha(1+(6r-1)^\beta)-\frac{1}{r} & 
%    \text{ for } \frac{1}{6}\leq r \leq \frac{1}{3},\\
%   2\alpha+6\pi\sin(\pi(6r-2))-\frac{1}{r}\cos(\pi(6r-2)) & 
%     \text{ for } \frac{1}{3}\leq r \leq \frac{1}{2},\\
%     2\alpha(\frac{5}{2}-3r)^\beta+\frac{1}{3} & 
%     \text{ for } \frac{1}{2}\leq r \leq \frac{5}{6},\\
%     -3\pi\sin(\pi(6r-5))+\frac{1}{r}\frac{(1+\cos(\pi(6r-5)))}{2}
%     & \text{ for } \frac{5}{6}\leq r \leq 1\\
% \end{cases}
% \]
% with $\alpha=\beta=1$ and the
% exact solution to \ref{e:defEintro} with right-hand side $f$
% \begin{align*}
% u(r)=
%   \begin{cases}
%   1 & \text{ for } 0\leq r \leq \frac{1}{6},\\
%   1 + (6r-1)^\beta& 
%    \text{ for } \frac{1}{6}\leq r \leq \frac{1}{3},\\
%   2 &
%     \text{ for } \frac{1}{3}\leq r \leq \frac{1}{2},\\
%   2(5/2-3r)^\beta &
%     \text{ for } \frac{1}{2}\leq r \leq \frac{5}{6},\\
%     0
%     & \text{ for } \frac{5}{6}\leq r \leq 1.\\
%   \end{cases}
% \end{align*}
% 
% 
% \hspace{-30pt}
% \begin{minipage}{0.49\textwidth}
% \hspace{-40pt}
% \includegraphics[scale=0.6]
% {usedPictures/fPlot3D.png}
% \captionof{figure}{Plot of the right-hand side $f$ with $\alpha=\beta=1$.}
% \end{minipage}
% \hspace{+20pt}
% \begin{minipage}{0.49\textwidth}
% \hspace{-20pt}
% \includegraphics[scale=0.6]
% {usedPictures/fPlotAxis.png}
% \captionof{figure}{Plot of the right-hand side $f$ along 
% the axes with $\alpha=\beta=1$.}
% \end{minipage}\\
% 
% \hspace{-30pt}
% \begin{minipage}{0.49\textwidth}
% \hspace{-40pt}
% \includegraphics[scale=0.6]
% {usedPictures/uExactPlot3D.png}
% \captionof{figure}{Plot of the exact solution $u$ with $\alpha=\beta=1$.}
% \end{minipage}
% \hspace{+20pt}
% \begin{minipage}{0.49\textwidth}
% \hspace{-20pt}
% \includegraphics[scale=0.6]
% {usedPictures/uExactPlotAxis.png}
% \captionof{figure}{Plot of the exact solution $u$ along 
% the axes with $\alpha=\beta=1$.}
% \end{minipage}\\
%  
% The discrete energy $E_{\textup{NC}}(I_\textup{NC}u)$ of the exact solution $u$
% on a triangulation that approximates the unit circle develops as in 
% Table \ref{table:energyDevelopment} with respect to the number of nodes.
% 
% \begin{center}
%   \includegraphics[scale=0.7]
%   {usedPictures/polygonSample.png}
%   \captionof{figure}{Example of a triangulation approximating the unit circle.}
% \end{center}
% 
% \begin{table}[h]
%   \centering
%   \begin{tabular}[H]{l|c c c c c c c}
%     Number of nodes & 545 & 2113 & 8321 & 33025 & 131585 & 523313 & 2099201 \\ 
%     \hline
%     $E_\textup{NC} (I_\textup{NC}u)$ & -1.9499 & -2.0388 & -2.0545 & 
%     -2.0573& -2.0579 
%     & -2.0580 & -2.0580
%   \end{tabular}
%   \caption{$E_\textup{NC} (I_\textup{NC}u)$ (rounded) on different triangulations of the 
%   unit circle.}
%   \label{table:energyDevelopment}
% \end{table}
% 
% The values of $E_\textup{NC} (I_\textup{NC}u)$ are rounded. For 
% the triangulation with 2099201 nodes the value of 
% $E_\textup{NC} (I_\textup{NC}u)$ is approximately $-2.05802391003896$.
% Therefore the discrete energy of the solutions of the discrete problem might
% converge from above to some real value close to $E_u\coloneqq
% -2.05802391003896$.
% 
% %TODO document how E INC U was calculated.
% 
% The following results were obtained by calling the function 
% \texttt{tvRegPrimalDual}
% (cf.\,Section \ref{fct:tvRegPrimalDual}) via nested iterations. 
% The inital choice for $u_0$ is $I_\textup{NC} f$ on a triangulation approximating 
% the unit circle. After every nested iteration, if the termination criterion
% $|E_\textup{NC}(u_j)-E_\textup{NC}(u_{j-1})|<\varepsilon$ for some $\varepsilon
% >0$ is not satisfied, the inital choice of $u_0$ for the next iteration 
% on a refined mesh is 
% the prolongation of the solution of the previous nested iteration to the new 
% triangulation.
% \texttt{tvRegPrimalDual.m} is executed with the termination
% criterion $|E_\textup{NC}(u_j)-E_\textup{NC}(u_{j-1})|<h^2/8$, where 
% $h\coloneqq 2^{-\texttt{red}-\texttt{itStep}}$ with \texttt{red} the number 
% of red refinements of the inital triangulation and \texttt{itStep} the
% number of nested iterations already executed.
% 
% The choice $\varepsilon=10^{-5}$ with varying inital red refinements
% \texttt{red} for the
% first triangulation of the nested iteration led to the following observations.\\
% 
% Initial $\texttt{red}=1.$ 
% 
% \hspace{-80pt}
% \begin{minipage}{0.49\textwidth}
% \hspace{-45pt}
% \includegraphics[scale=0.55]
% {usedPictures/experiments/initalRed1/solution_red_7.png}
% \captionof{figure}{Plot of the result of the seventh nested iteration.}
% \end{minipage}
% \hspace{+70pt}
% \begin{minipage}{0.49\textwidth}
% \hspace{-20pt}
% \includegraphics[scale=0.55]
% {usedPictures/experiments/initalRed1/solution_red_7_axis.png}
% \captionof{figure}{Plot of the result of the seventh nested iteration along
% the axes.}
% \end{minipage}\\
% 
% First the result of the nested iteration (Figure 6.6 and Figure 6.7) 
% starting with a coarse triangulation
% ($\texttt{red}=1$) closely resembles the exact solution (Figure 6.3 and 
% Figure 6.4). \\
% 
% \hspace{-100pt}
% \includegraphics[scale=1]
% {usedPictures/experiments/initalRed1/energy.png}
% \captionof{figure}{Energy development during every nested iteration.}
% 
% 
% \hspace{-70pt}
% \begin{minipage}{0.49\textwidth}
% \hspace{-45pt}
% \includegraphics[scale=0.55]
% {usedPictures/experiments/initalRed1/singleEnergy_red_7.png}
% \captionof{figure}{Energy development during the seventh nested iteration.}
% \end{minipage}
% \hspace{+70pt}
% \begin{minipage}{0.49\textwidth}
% \hspace{-20pt}
% \includegraphics[scale=0.55]
% {usedPictures/experiments/initalRed1/enDiffExact_red_7.png}
% \captionof{figure}{$|E_\textup{NC}(u_j)-E_\textup{NC}(u_\textup{NC})|$ 
% for the seventh nested iteration.}
% \end{minipage}\\
% 
% The discrete energy converges from above and is bounded from below by 
% the the approximated disrecte energy $E_u$ of $I_\textup{NC}u$ of the 
% exact solution $u.$ Furthermore in the final nested iteration 
% the absolute difference 
% between the energy of the discrete solution and $E_u$ is about $10^{-2}$.
% 
% 
% \hspace{-70pt}
% \begin{minipage}{0.49\textwidth}
% \hspace{-45pt}
% \includegraphics[scale=0.55]
% {usedPictures/experiments/initalRed1/singleExactError6.png}
% \captionof{figure}{Development of the exact error in the $L^2$ norm for the
% sixth nested iteration.}
% \end{minipage}
% \hspace{+70pt}
% \begin{minipage}{0.49\textwidth}
% \hspace{-20pt}
% \includegraphics[scale=0.55]
% {usedPictures/experiments/initalRed1/singleExactError7.png}
% \captionof{figure}{Development of the exact error in the $L^2$ norm for the
% seventh nested iteration.}
% \end{minipage}\\
% 
% The exact error (here the $L^2$ norm of $u_{\textup{NC}}-u$) shows some 
% unexpected behaviour. While it remains fairly large in the last 
% nested iteration, even though the plot of the solution suggests otherwise,
% the error becomes much smaller in the previous step of the nested iteration.
% 
% 
% Initial $\texttt{red}=6.$ 
% 
% \hspace{-80pt}
% \begin{minipage}{0.49\textwidth}
% \hspace{-45pt}
% \includegraphics[scale=0.55]
% {usedPictures/experiments/initalRed6/solution_red_6.png}
% \captionof{figure}{Plot of the result of the first nested iteration.}
% \end{minipage}
% \hspace{+70pt}
% \begin{minipage}{0.49\textwidth}
% \hspace{-20pt}
% \includegraphics[scale=0.55]
% {usedPictures/experiments/initalRed6/solution_red_6_axis.png}
% \captionof{figure}{Plot of the result of the first nested iteration along
% the axes.}
% \end{minipage}\\
% 
% Starting with an inital red refinement $red=6$ the termination of the iteration
% yields no further nested iterations, since the global termination criterion
% is already satisfied. Therefore this is in fact just the call of 
% \texttt{tvRegPrimalDual} on a relatively fine mesh. Some features of the 
% exact solution are visible, namely the plateaus are nearly at the right hight.
% However, this discrete solution looks fairly different from the exact solution
% $u$ and even the discrete solution in Figure 6.6 and Figure 6.7.
% 
% \hspace{-100pt}
% \includegraphics[scale=1]
% {usedPictures/experiments/initalRed6/energy.png}
% \captionof{figure}{Energy development during every nested iteration.}
% 
% \hspace{-70pt}
% \begin{minipage}{0.49\textwidth}
% \hspace{-45pt}
% \includegraphics[scale=0.55]
% {usedPictures/experiments/initalRed6/enDiffExact_red_6.png}
% \captionof{figure}{$|E_\textup{NC}(u_j)-E_\textup{NC}(u_\textup{NC})|$ 
% for the first nested iteration.}
% \end{minipage}
% \hspace{+70pt}
% \begin{minipage}{0.49\textwidth}
% \hspace{-20pt}
% \includegraphics[scale=0.55]
% {usedPictures/experiments/initalRed6/singleExactError6.png}
% \captionof{figure}{Development of the exact error in the $L^2$ norm for the
% first nested iteration.}
% \end{minipage}\\
% 
% Even though the discrete solution noticeably differs from the exact solution 
% the discrete
% energy still converges to the expected value and the exact error becomes
% fairly small (even smaller than the exact error in Figure 6.12, where the 
% discrete solution nearly resembled the exact solution).
% 
% The cause of this difference between an initial refinement level of 
% $\texttt{red}=1$ and $\texttt{red}=6$ might be the inital value
% $u_0 = I_\textup{NC}f$. For $\texttt{red}=1$ the first nested iteration is on 
% a triangulation with relatively few nodes. Therefore, after the first nested
% iteration, most information of $f$ is lost due to the lack of nodes.
% This is not the case for an inital $\texttt{red}=6$, where the maximal value
% of $f$ on the unit circle of about $20$ and the minimal value of about
% $-25$ seem to have a severe effect on the solution of the iteration.
% Further indication for this hypothesis is provided by an initial red 
% refinement between $1$ and $6$, where $I_\textup{NC}f$ resembles $f$
% better than on the mesh with $\texttt{red}=1$, but worse than on the mesh
% with $\texttt{red}=6$.
% 
% 
% Initial $\texttt{red}=4.$ 
% 
% \hspace{-80pt}
% \begin{minipage}{0.49\textwidth}
% \hspace{-45pt}
% \includegraphics[scale=0.55]
% {usedPictures/experiments/initalRed4/solution_red_7.png}
% \captionof{figure}{Plot of the result of the fourth nested iteration.}
% \end{minipage}
% \hspace{+70pt}
% \begin{minipage}{0.49\textwidth}
% \hspace{-20pt}
% \includegraphics[scale=0.55]
% {usedPictures/experiments/initalRed4/solution_red_7_axis.png}
% \captionof{figure}{Plot of the result of the fourth nested iteration along
% the axes.}
% \end{minipage}\\
% 
% Figure 6.18 and Figure 6.19 still resemble the exact solution in
% Figure 6.6 and Figure 6.7. However, on the inner slope the behaviour 
% of the discrete solution in Figure 6.13 and Figure 6.14 is visible.
% 
% \hspace{-100pt}
% \includegraphics[scale=1]
% {usedPictures/experiments/initalRed4/energy.png}
% \captionof{figure}{Energy development during every nested iteration.}
% 
% \hspace{-70pt}
% \begin{minipage}{0.49\textwidth}
% \hspace{-45pt}
% \includegraphics[scale=0.55]
% {usedPictures/experiments/initalRed4/singleEnergy_red_7.png}
% \captionof{figure}{Energy development during the fourth nested iteration.}
% \end{minipage}
% \hspace{+70pt}
% \begin{minipage}{0.49\textwidth}
% \hspace{-20pt}
% \includegraphics[scale=0.55]
% {usedPictures/experiments/initalRed4/enDiffExact_red_7.png}
% \captionof{figure}{$|E_\textup{NC}(u_j)-E_\textup{NC}(u_\textup{NC})|$ 
% for the fourth nested iteration.}
% \end{minipage}\\
% 
% 
% \hspace{-70pt}
% \begin{minipage}{0.49\textwidth}
% \hspace{-45pt}
% \includegraphics[scale=0.55]
% {usedPictures/experiments/initalRed4/singleExactError6.png}
% \captionof{figure}{Development of the exact error in the $L^2$ norm for the
% third nested iteration.}
% \end{minipage}
% \hspace{+70pt}
% \begin{minipage}{0.49\textwidth}
% \hspace{-20pt}
% \includegraphics[scale=0.55]
% {usedPictures/experiments/initalRed4/singleExactError7.png}
% \captionof{figure}{Development of the exact error in the $L^2$ norm for the
% fourth nested iteration.}
% \end{minipage}\\
% 
% 
% 
% 
% 
% 
% \newpage
% \subsection{Further experiments}
% 
% The experiments in this section are performed on
% only one mesh per experiment (without nested iterations) to study the 
% influence of the inital value
% $u_0$ on the result of the iteration. The termination criterion 
% is $\|E_\textup{NC}(u_j)-E_\textup{NC}(u_{j-1})\|_{L^2}<\varepsilon$.
% Also Algorithm \ref{alg:PrimalDualIteration} is compared to the
% conforming algorithm in \cite{Bart15-book} (p.314, Code on p. 316). 
% For the conforming algorithm the termination criterion is
% $\|E_\textup{NC}(I_\textup{NC}u_j)-
% E_\textup{NC}(I_\textup{NC}u_{j-1})\|_{L^2}<\varepsilon$.
% 
% In legends 'nonconform' and 'conform' indicate which algorithm was used, 
% followed by the choice of $u_0$ for 
% the experiment, and in parentheses the time (in hours, rounded) the iteration ran
% to obtain the corresponding result.
% 
% \subsubsection{Update on previous observations}
% 
% The result of the iteration for an inital red refinement level 
% $\texttt{red}=6$,
% $u_0 = I_\textup{NC}f$, and a termination criterion 
% $\varepsilon = 10^{-5}$ (Figure 6.13 and Figure 6.14)
% does not look like the exact solution (Figure 6.3 and Figure 6.4). However,
% for $\varepsilon = 10^{-9}$ the result closely resembles the exact solution.
% 
% \hspace{-80pt}
% \begin{minipage}{0.49\textwidth}
% \hspace{-45pt}
% \includegraphics[scale=0.55]
% {usedPictures/updateOnBadResults/solution_red_6.png}
% \captionof{figure}{Plot of the result for $\varepsilon=10^{-9}$, 
% $\texttt{red} = 6$, $u_0 = I_{\textup{NC}}f$.}
% \end{minipage}
% \hspace{+70pt}
% \begin{minipage}{0.49\textwidth}
% \hspace{-20pt}
% \includegraphics[scale=0.55]
% {usedPictures/updateOnBadResults/solution_red_6_axis.png}
% \captionof{figure}{Plot of the result for $\varepsilon=10^{-9}$, 
% $\texttt{red} = 6$, $u_0 = I_{\textup{NC}}f$ along
% the axes.}
% \end{minipage}\\
% 
% For $\varepsilon=10^{-5}$ the exact $L^2$ error of the result
% is approximately $10^{-1}$ (Figure 6.17), while for $\varepsilon=10^{-9}$ the
% exact $L^2$ error is approximately $2\cdot 10^{-3}$. To obtain these improved
% results the iteration runs approximately four hours longer than the 
% iteration for $\varepsilon=10^{-5}$.
% 
% \hspace{-100pt}
% \includegraphics[scale=1]
% {usedPictures/updateOnBadResults/exactError.png}
% \captionof{figure}{Development of the exact error in the $L^2$ norm for
% $\varepsilon=10^{-9}$, 
% $\texttt{red} = 6$, $u_0 = I_{\textup{NC}}f$.}
% 
% %\subsubsection{Comparison of Algorithm \ref{alg:PrimalDualIteration} 
% %and the conforming algorithm (\cite{Bart15-book}, p.314) for 
% %$u_0\in \{I_\textup{NC}f,0\}$}
% %
% %\newpage
% %\textit{Exact $L^2$ Error.}\\
% %
% %\begin{tikzpicture}
% %  %\begin{axis}[] 
% %    \begin{loglogaxis}[legend pos = south west, 
% %      scale = 2, xlabel = {number of iterations}, 
% %      ylabel = {exact $L^2$ error},
% %      my filter = every 50 between 250 and 1e5,filter discard warning = false]      
% %
% %      \addplot [solid, black] 
% %      table[x expr = \coordindex+1, y index = 0] 
% %      {usedPictures/comparison/red4/conform_0/errorExactVec.txt}; 
% %      \addlegendentry{conform, $u_0 = 0$ (0.17h)}
% %    
% %      \addplot [solid, blue] 
% %      table[x expr = \coordindex+1, y index = 0] 
% %      {usedPictures/comparison/red4/conform_f/errorExactVec.txt}; 
% %      \addlegendentry{conform, $u_0 = I_\textup C f$ (0.12h)}
% %   
% %      \addplot [solid, green] 
% %      table[x expr = \coordindex+1, y index = 0] 
% %      {usedPictures/comparison/red4/nonconform_0/errorExactVec.txt}; 
% %      \addlegendentry{nonconform, $u_0 = 0$ (0.15h)}
% %  
% %      \addplot [solid, red]
% %      table[x expr = \coordindex+1, y index = 0] 
% %      {usedPictures/comparison/red4/nonconform_f/errorExactVec.txt}; 
% %      \addlegendentry{nonconform, $u_0 = I_\textup{NC} f$ (0.84h)}
% %    \end{loglogaxis}
% %  %\end{axis}
% %\end{tikzpicture}
% %\captionof{figure}{Development of the exact error in the $L^2$ norm for
% %$\varepsilon=10^{-9}$ and 
% %$\texttt{red} = 4$.}
% %
% %
% %\begin{tikzpicture}
% %  %\begin{axis}[] 
% %    \begin{loglogaxis}[legend pos = south west, 
% %      scale = 2, xlabel = {number of iterations}, 
% %      ylabel = {exact $L^2$ error},
% %      my filter = every 50 between 250 and 1e5,filter discard warning = false]      
% %
% %      \addplot [solid, black] 
% %      table[x expr = \coordindex+1, y index = 0] 
% %      {usedPictures/comparison/red5/conform_0/errorExactVec.txt}; 
% %      \addlegendentry{conform, $u_0 = 0$ (0.27h)}
% %    
% %      \addplot [solid, blue] 
% %      table[x expr = \coordindex+1, y index = 0] 
% %      {usedPictures/comparison/red5/conform_f/errorExactVec.txt}; 
% %      \addlegendentry{conform, $u_0 = I_\textup C f$ (0.43h)}
% %   
% %      \addplot [solid, green] 
% %      table[x expr = \coordindex+1, y index = 0] 
% %      {usedPictures/comparison/red5/nonconform_0/errorExactVec.txt}; 
% %      \addlegendentry{nonconform, $u_0 = 0$ (1.52h)}
% %  
% %      \addplot [solid, red]
% %      table[x expr = \coordindex+1, y index = 0] 
% %      {usedPictures/comparison/red5/nonconform_f/errorExactVec.txt}; 
% %      \addlegendentry{nonconform, $u_0 = I_\textup{NC} f$ (1.78h)}
% %    \end{loglogaxis}
% %  %\end{axis}
% %\end{tikzpicture}
% %\captionof{figure}{Development of the exact error in the $L^2$ norm for
% %$\varepsilon=10^{-9}$ and 
% %$\texttt{red} = 5$.}
% %
% %\begin{tikzpicture}
% %  %\begin{axis}[] 
% %    \begin{loglogaxis}[legend pos = south west, 
% %      scale = 2, xlabel = {number of iterations}, 
% %      ylabel = {exact $L^2$ error},
% %      my filter = every 50 between 250 and 1e5,filter discard warning = false]      
% %
% %      \addplot [solid, black] 
% %      table[x expr = \coordindex+1, y index = 0] 
% %      {usedPictures/comparison/red6/conform_0/errorExactVec.txt}; 
% %      \addlegendentry{conform, $u_0 = 0$ (1.11h)}
% %    
% %      \addplot [solid, blue] 
% %      table[x expr = \coordindex+1, y index = 0] 
% %      {usedPictures/comparison/red6/conform_f/errorExactVec.txt}; 
% %      \addlegendentry{conform, $u_0 = I_\textup C f$ (2.11h)}
% %   
% %      \addplot [solid, green] 
% %      table[x expr = \coordindex+1, y index = 0] 
% %      {usedPictures/comparison/red6/nonconform_0/errorExactVec.txt}; 
% %      \addlegendentry{nonconform, $u_0 = 0$ (4.9h)}
% %  
% %      \addplot [solid, red]
% %      table[x expr = \coordindex+1, y index = 0] 
% %      {usedPictures/comparison/red6/nonconform_f/errorExactVec.txt}; 
% %      \addlegendentry{nonconform, $u_0 = I_\textup{NC} f$ (6.94h)}
% %    \end{loglogaxis}
% %  %\end{axis}
% %\end{tikzpicture}
% %\captionof{figure}{Development of the exact error in the $L^2$ norm for
% %$\varepsilon=10^{-9}$ and 
% %$\texttt{red} = 6$.}
% %
% %
% %\begin{tikzpicture}
% %  %\begin{axis}[] 
% %    \begin{loglogaxis}[legend pos = south west, 
% %      scale = 2, xlabel = {number of iterations}, 
% %      ylabel = {exact $L^2$ error},
% %      my filter = every 50 between 250 and 1e5,filter discard warning = false]      
% %
% %      \addplot [solid, black] 
% %      table[x expr = \coordindex+1, y index = 0] 
% %      {usedPictures/comparison/red7/conform_0/errorExactVec.txt}; 
% %      \addlegendentry{conform, $u_0 = 0$ (4.4h)}
% %    
% %      \addplot [solid, blue] 
% %      table[x expr = \coordindex+1, y index = 0] 
% %      {usedPictures/comparison/red7/conform_f/errorExactVec.txt}; 
% %      \addlegendentry{conform, $u_0 = I_\textup C f$ (10.4h)}
% %   
% %      \addplot [solid, green] 
% %      table[x expr = \coordindex+1, y index = 0] 
% %      {usedPictures/comparison/red7/nonconform_0/errorExactVec.txt}; 
% %      \addlegendentry{nonconform, $u_0 = 0$ (20.6h)}
% %  
% %      \addplot [solid, red]
% %      table[x expr = \coordindex+1, y index = 0] 
% %      {usedPictures/comparison/red7/nonconform_f/errorExactVec.txt}; 
% %      \addlegendentry{nonconform, $u_0 = I_\textup{NC} f$ (15.4h)}
% %    \end{loglogaxis}
% %  %\end{axis}
% %\end{tikzpicture}
% %\captionof{figure}{Development of the exact error in the $L^2$ norm for
% %$\varepsilon=10^{-9}$ and 
% %$\texttt{red} = 7$.}
% %
% %
% %\begin{tikzpicture}
% %  %\begin{axis}[] 
% %    \begin{loglogaxis}[legend pos = south west, 
% %      scale = 2, xlabel = {number of iterations}, 
% %      ylabel = {exact $L^2$ error},
% %      my filter = every 50 between 250 and 1e5,filter discard warning = false]      
% %      
% %      \addplot [solid, black] 
% %      table[x expr = \coordindex+1, y index = 0] 
% %      {usedPictures/comparison/red8/conform_0/errorExactVec.txt}; 
% %      \addlegendentry{conform, $u_0 = 0$ (18.5h)}
% %    
% %      \addplot [solid, blue] 
% %      table[x expr = \coordindex+1, y index = 0] 
% %      {usedPictures/comparison/red8/conform_f/errorExactVec.txt}; 
% %      \addlegendentry{conform, $u_0 = I_\textup C f$ (74.4h)}
% %   
% %      \addplot [solid, green] 
% %      table[x expr = \coordindex+1, y index = 0] 
% %      {usedPictures/comparison/red8/nonconform_0/errorExactVec.txt}; 
% %      \addlegendentry{nonconform, $u_0 = 0$ (13.3h)}
% %  
% %      \addplot [solid, red]
% %      table[x expr = \coordindex+1, y index = 0] 
% %      {usedPictures/comparison/red8/nonconform_f/errorExactVec.txt}; 
% %      \addlegendentry{nonconform, $u_0 = I_\textup{NC} f$ (123.7h)}
% %    \end{loglogaxis}
% %  %\end{axis}
% %\end{tikzpicture}
% %\captionof{figure}{Development of the exact error in the $L^2$ norm for
% %$\varepsilon=10^{-9}$ and 
% %$\texttt{red} = 8$.}
% %
% %Figures 6.28 - 6.32 depict that the conforming algorithm 
% %(respectively Algorithm \ref{alg:PrimalDualIteration}) converges to a similar
% %value independent of the choice of $u_0\in\{0,I_\textup{C} f\}$ 
% %(respectively $u_0\in\{0,I_\textup{NC} f\}$). 
% %These figures (except Figure 6.28 for the conforming algorithm and Figure
% %6.31 for Algorithm \ref{alg:PrimalDualIteration}) show that both algorithms
% %run significantly shorter for the initial value $u_0=0$. \\
% %In all figures Algorithm \ref{alg:PrimalDualIteration} stagnates at a smaller 
% %value than the conforming algorithm. For both algorithms the limit is smaller on
% %finer meshes.
% %
% %\begin{table}[h]
% %  \centering
% %  \begin{tabular}[H]{|l|c c|}
% %    \hline
% %    & Algorithm \ref{alg:PrimalDualIteration} & 
% %    Algorithm from \cite{Bart15-book} \\ 
% %    \hline
% %    $\texttt{red} = 4$ & $3\cdot 10^{-2}$ & $6\cdot 10^{-1}$ \\
% %    $\texttt{red} = 5$ & $9\cdot 10^{-3}$ & $2\cdot 10^{-1}$ \\
% %    $\texttt{red} = 6$ & $1\cdot 10^{-3}$ & $6\cdot 10^{-2}$ \\
% %    $\texttt{red} = 7$ & $5\cdot 10^{-4}$ & $3\cdot 10^{-2}$ \\
% %    $\texttt{red} = 8$ & $1\cdot 10^{-4}$ & $1\cdot 10^{-2}$ \\
% %    \hline
% %  \end{tabular}
% %  \caption{Approximated limits of the algorithms on the meshes.}
% %\end{table}
% %
% %\newpage
% %\textit{Discrete Energy.}\\
% %
% %Figures 6.28, 6.30, 6.32, 6.34, 6.36 depict the development of the discrete
% %energy of
% %Algorithm \ref{alg:PrimalDualIteration} and the algorithm from 
% %\cite{Bart15-book} on different meshes. In all settings the energy converges
% %from above to values in a neighborhood of the expected discrete energy $E_u$ 
% %from \ref{table:energyDevelopment}. Thus the algorithms minimize the discrete 
% %energy in this experiments.\\
% %Figures 6.29, 6.31, 6.33, 6.35, 6.37 depict the 
% %development of the absolute difference between
% %the discrete energy of the iterates and $E_u$. The graphs for 
% %Algorithm \ref{alg:PrimalDualIteration} suggest that the discrete energy
% %of the iterates yielded by this algorithm converge to a value
% %about $2\cdot 10^{-2}$ smaller
% %than $E_u$.
% %
% %% red 4
% %
% %\begin{tikzpicture}
% %  %\begin{axis}[] 
% %    \begin{axis}[legend pos = north east, 
% %      scale = 2, xlabel = {number of iterations}, 
% %      ylabel = {discrete energy},
% %      my filter = every 50 between 250 and 1e5,filter discard warning = false,
% %      ymax = 1]      
% %
% %      \addplot [solid, black] 
% %      table[x expr = \coordindex+1, y index = 0] 
% %      {usedPictures/comparison/red4/conform_0/energyVec.txt}; 
% %      \addlegendentry{conform, $u_0 = 0$ (0.17h)}
% %    
% %      \addplot [solid, blue] 
% %      table[x expr = \coordindex+1, y index = 0] 
% %      {usedPictures/comparison/red4/conform_f/energyVec.txt}; 
% %      \addlegendentry{conform, $u_0 = I_\textup C f$ (0.12h)}
% %   
% %      \addplot [solid, green] 
% %      table[x expr = \coordindex+1, y index = 0] 
% %      {usedPictures/comparison/red4/nonconform_0/energyVec.txt}; 
% %      \addlegendentry{nonconform, $u_0 = 0$ (0.15h)}
% %  
% %      \addplot [solid, red]
% %      table[x expr = \coordindex+1, y index = 0] 
% %      {usedPictures/comparison/red4/nonconform_f/energyVec.txt}; 
% %      \addlegendentry{nonconform, $u_0 = I_\textup{NC} f$ (0.84h)}
% %    \end{axis}
% %  %\end{axis}
% %\end{tikzpicture}
% %\captionof{figure}{Development of $E_\textup{NC}(u_j)$ for Algorithm 
% %\ref{alg:PrimalDualIteration} respectively  
% %      $E_\textup{NC}(I_\textup{NC}u_j)$ for the conforming algorithm in 
% %      \cite{Bart15-book} for
% %      $\varepsilon=10^{-9}$ and 
% %      $\texttt{red} = 4$.}
% %
% %
% %\begin{tikzpicture}
% %  %\begin{axis}[] 
% %    \begin{loglogaxis}[legend pos = south west, 
% %      scale = 2, xlabel = {number of iterations}, 
% %      ylabel = {absolute difference of the discrete energy and $E_u$},
% %      my filter = every 50 between 250 and 1e5,filter discard warning = false]      
% %
% %      \addplot [solid, black] 
% %      table[x expr = \coordindex+1, y index = 0] 
% %      {usedPictures/comparison/red4/conform_0/exactAbsoluteEnergyDifference.txt}; 
% %      \addlegendentry{conform, $u_0 = 0$ (0.17h)}
% %    
% %      \addplot [solid, blue] 
% %      table[x expr = \coordindex+1, y index = 0] 
% %      {usedPictures/comparison/red4/conform_f/exactAbsoluteEnergyDifference.txt}; 
% %      \addlegendentry{conform, $u_0 = I_\textup C f$ (0.12h)}
% %   
% %      \addplot [solid, green] 
% %      table[x expr = \coordindex+1, y index = 0] 
% %      {usedPictures/comparison/red4/nonconform_0/exactAbsoluteEnergyDifference.txt}; 
% %      \addlegendentry{nonconform, $u_0 = 0$ (0.15h)}
% %  
% %      \addplot [solid, red]
% %      table[x expr = \coordindex+1, y index = 0] 
% %      {usedPictures/comparison/red4/nonconform_f/exactAbsoluteEnergyDifference.txt}; 
% %      \addlegendentry{nonconform, $u_0 = I_\textup{NC} f$ (0.84h)}
% %    \end{loglogaxis}
% %  %\end{axis}
% %\end{tikzpicture}
% %\captionof{figure}{Development of $|E_\textup{NC}(u_j)-E_u|$ for Algorithm 
% %\ref{alg:PrimalDualIteration} respectively  
% %      $|E_\textup{NC}(I_\textup{NC}u_j)-E_u|$ for the conforming algorithm in 
% %      \cite{Bart15-book} for
% %      $\varepsilon=10^{-9}$ and 
% %      $\texttt{red} = 4$.}
% %
% %% red 5
% %
% %\begin{tikzpicture}
% %  %\begin{axis}[] 
% %    \begin{axis}[legend pos = north east, 
% %      scale = 2, xlabel = {number of iterations}, 
% %      ylabel = {discrete energy},
% %      my filter = every 50 between 250 and 1e5,filter discard warning = false,
% %      ymax = 1]      
% %
% %      \addplot [solid, black] 
% %      table[x expr = \coordindex+1, y index = 0] 
% %      {usedPictures/comparison/red5/conform_0/energyVec.txt}; 
% %      \addlegendentry{conform, $u_0 = 0$ (0.17h)}
% %    
% %      \addplot [solid, blue] 
% %      table[x expr = \coordindex+1, y index = 0] 
% %      {usedPictures/comparison/red5/conform_f/energyVec.txt}; 
% %      \addlegendentry{conform, $u_0 = I_\textup C f$ (0.12h)}
% %   
% %      \addplot [solid, green] 
% %      table[x expr = \coordindex+1, y index = 0] 
% %      {usedPictures/comparison/red5/nonconform_0/energyVec.txt}; 
% %      \addlegendentry{nonconform, $u_0 = 0$ (0.15h)}
% %  
% %      \addplot [solid, red]
% %      table[x expr = \coordindex+1, y index = 0] 
% %      {usedPictures/comparison/red5/nonconform_f/energyVec.txt}; 
% %      \addlegendentry{nonconform, $u_0 = I_\textup{NC} f$ (0.84h)}
% %    \end{axis}
% %  %\end{axis}
% %\end{tikzpicture}
% %\captionof{figure}{Development of $E_\textup{NC}(u_j)$ for Algorithm 
% %\ref{alg:PrimalDualIteration} respectively  
% %      $E_\textup{NC}(I_\textup{NC}u_j)$ for the conforming algorithm in 
% %      \cite{Bart15-book} for
% %      $\varepsilon=10^{-9}$ and 
% %      $\texttt{red} = 5$.}
% %
% %
% %\begin{tikzpicture}
% %  %\begin{axis}[] 
% %    \begin{loglogaxis}[legend pos = south west, 
% %      scale = 2, xlabel = {number of iterations}, 
% %      ylabel = {absolute difference of the discrete energy and $E_u$},
% %      my filter = every 50 between 250 and 1e5,filter discard warning = false]      
% %
% %      \addplot [solid, black] 
% %      table[x expr = \coordindex+1, y index = 0] 
% %      {usedPictures/comparison/red5/conform_0/exactAbsoluteEnergyDifference.txt}; 
% %      \addlegendentry{conform, $u_0 = 0$ (0.17h)}
% %    
% %      \addplot [solid, blue] 
% %      table[x expr = \coordindex+1, y index = 0] 
% %      {usedPictures/comparison/red5/conform_f/exactAbsoluteEnergyDifference.txt}; 
% %      \addlegendentry{conform, $u_0 = I_\textup C f$ (0.12h)}
% %   
% %      \addplot [solid, green] 
% %      table[x expr = \coordindex+1, y index = 0] 
% %      {usedPictures/comparison/red5/nonconform_0/exactAbsoluteEnergyDifference.txt}; 
% %      \addlegendentry{nonconform, $u_0 = 0$ (0.15h)}
% %  
% %      \addplot [solid, red]
% %      table[x expr = \coordindex+1, y index = 0] 
% %      {usedPictures/comparison/red5/nonconform_f/exactAbsoluteEnergyDifference.txt}; 
% %      \addlegendentry{nonconform, $u_0 = I_\textup{NC} f$ (0.84h)}
% %    \end{loglogaxis}
% %  %\end{axis}
% %\end{tikzpicture}
% %\captionof{figure}{Development of $|E_\textup{NC}(u_j)-E_u|$ for Algorithm 
% %\ref{alg:PrimalDualIteration} respectively  
% %      $|E_\textup{NC}(I_\textup{NC}u_j)-E_u|$ for the conforming algorithm in 
% %      \cite{Bart15-book} for
% %      $\varepsilon=10^{-9}$ and 
% %      $\texttt{red} = 5$.}
% %
% %% red 6
% %
% %\begin{tikzpicture}
% %  %\begin{axis}[] 
% %    \begin{axis}[legend pos = north east, 
% %      scale = 2, xlabel = {number of iterations}, 
% %      ylabel = {discrete energy},
% %      my filter = every 50 between 250 and 1e5,filter discard warning = false,
% %      ymax = 1]      
% %
% %      \addplot [solid, black] 
% %      table[x expr = \coordindex+1, y index = 0] 
% %      {usedPictures/comparison/red6/conform_0/energyVec.txt}; 
% %      \addlegendentry{conform, $u_0 = 0$ (0.17h)}
% %    
% %      \addplot [solid, blue] 
% %      table[x expr = \coordindex+1, y index = 0] 
% %      {usedPictures/comparison/red6/conform_f/energyVec.txt}; 
% %      \addlegendentry{conform, $u_0 = I_\textup C f$ (0.12h)}
% %   
% %      \addplot [solid, green] 
% %      table[x expr = \coordindex+1, y index = 0] 
% %      {usedPictures/comparison/red6/nonconform_0/energyVec.txt}; 
% %      \addlegendentry{nonconform, $u_0 = 0$ (0.15h)}
% %  
% %      \addplot [solid, red]
% %      table[x expr = \coordindex+1, y index = 0] 
% %      {usedPictures/comparison/red6/nonconform_f/energyVec.txt}; 
% %      \addlegendentry{nonconform, $u_0 = I_\textup{NC} f$ (0.84h)}
% %    \end{axis}
% %  %\end{axis}
% %\end{tikzpicture}
% %\captionof{figure}{Development of $E_\textup{NC}(u_j)$ for Algorithm 
% %\ref{alg:PrimalDualIteration} respectively  
% %      $E_\textup{NC}(I_\textup{NC}u_j)$ for the conforming algorithm in 
% %      \cite{Bart15-book} for
% %      $\varepsilon=10^{-9}$ and 
% %      $\texttt{red} = 6$.}
% %
% %
% %\begin{tikzpicture}
% %  %\begin{axis}[] 
% %    \begin{loglogaxis}[legend pos = south west, 
% %      scale = 2, xlabel = {number of iterations}, 
% %      ylabel = {absolute difference of the discrete energy and $E_u$},
% %      my filter = every 50 between 250 and 1e5,filter discard warning = false]      
% %
% %      \addplot [solid, black] 
% %      table[x expr = \coordindex+1, y index = 0] 
% %      {usedPictures/comparison/red6/conform_0/exactAbsoluteEnergyDifference.txt}; 
% %      \addlegendentry{conform, $u_0 = 0$ (0.17h)}
% %    
% %      \addplot [solid, blue] 
% %      table[x expr = \coordindex+1, y index = 0] 
% %      {usedPictures/comparison/red6/conform_f/exactAbsoluteEnergyDifference.txt}; 
% %      \addlegendentry{conform, $u_0 = I_\textup C f$ (0.12h)}
% %   
% %      \addplot [solid, green] 
% %      table[x expr = \coordindex+1, y index = 0] 
% %      {usedPictures/comparison/red6/nonconform_0/exactAbsoluteEnergyDifference.txt}; 
% %      \addlegendentry{nonconform, $u_0 = 0$ (0.15h)}
% %  
% %      \addplot [solid, red]
% %      table[x expr = \coordindex+1, y index = 0] 
% %      {usedPictures/comparison/red6/nonconform_f/exactAbsoluteEnergyDifference.txt}; 
% %      \addlegendentry{nonconform, $u_0 = I_\textup{NC} f$ (0.84h)}
% %    \end{loglogaxis}
% %  %\end{axis}
% %\end{tikzpicture}
% %\captionof{figure}{Development of $|E_\textup{NC}(u_j)-E_u|$ for Algorithm 
% %\ref{alg:PrimalDualIteration} respectively  
% %      $|E_\textup{NC}(I_\textup{NC}u_j)-E_u|$ for the conforming algorithm in 
% %      \cite{Bart15-book} for
% %      $\varepsilon=10^{-9}$ and 
% %      $\texttt{red} = 6$.}
% %
% %% red 7
% %
% %\begin{tikzpicture}
% %  %\begin{axis}[] 
% %    \begin{axis}[legend pos = north east, 
% %      scale = 2, xlabel = {number of iterations}, 
% %      ylabel = {discrete energy},
% %      my filter = every 50 between 250 and 1e5,filter discard warning = false,
% %      ymax = 1]      
% %
% %      \addplot [solid, black] 
% %      table[x expr = \coordindex+1, y index = 0] 
% %      {usedPictures/comparison/red7/conform_0/energyVec.txt}; 
% %      \addlegendentry{conform, $u_0 = 0$ (0.17h)}
% %    
% %      \addplot [solid, blue] 
% %      table[x expr = \coordindex+1, y index = 0] 
% %      {usedPictures/comparison/red7/conform_f/energyVec.txt}; 
% %      \addlegendentry{conform, $u_0 = I_\textup C f$ (0.12h)}
% %   
% %      \addplot [solid, green] 
% %      table[x expr = \coordindex+1, y index = 0] 
% %      {usedPictures/comparison/red7/nonconform_0/energyVec.txt}; 
% %      \addlegendentry{nonconform, $u_0 = 0$ (0.15h)}
% %  
% %      \addplot [solid, red]
% %      table[x expr = \coordindex+1, y index = 0] 
% %      {usedPictures/comparison/red7/nonconform_f/energyVec.txt}; 
% %      \addlegendentry{nonconform, $u_0 = I_\textup{NC} f$ (0.84h)}
% %    \end{axis}
% %  %\end{axis}
% %\end{tikzpicture}
% %\captionof{figure}{Development of $E_\textup{NC}(u_j)$ for Algorithm 
% %\ref{alg:PrimalDualIteration} respectively  
% %      $E_\textup{NC}(I_\textup{NC}u_j)$ for the conforming algorithm in 
% %      \cite{Bart15-book} for
% %      $\varepsilon=10^{-9}$ and 
% %      $\texttt{red} = 7$.}
% %
% %
% %\begin{tikzpicture}
% %  %\begin{axis}[] 
% %    \begin{loglogaxis}[legend pos = south west, 
% %      scale = 2, xlabel = {number of iterations}, 
% %      ylabel = {absolute difference of the discrete energy and $E_u$},
% %      my filter = every 50 between 250 and 1e5,filter discard warning = false]      
% %
% %      \addplot [solid, black] 
% %      table[x expr = \coordindex+1, y index = 0] 
% %      {usedPictures/comparison/red7/conform_0/exactAbsoluteEnergyDifference.txt}; 
% %      \addlegendentry{conform, $u_0 = 0$ (0.17h)}
% %    
% %      \addplot [solid, blue] 
% %      table[x expr = \coordindex+1, y index = 0] 
% %      {usedPictures/comparison/red7/conform_f/exactAbsoluteEnergyDifference.txt}; 
% %      \addlegendentry{conform, $u_0 = I_\textup C f$ (0.12h)}
% %   
% %      \addplot [solid, green] 
% %      table[x expr = \coordindex+1, y index = 0] 
% %      {usedPictures/comparison/red7/nonconform_0/exactAbsoluteEnergyDifference.txt}; 
% %      \addlegendentry{nonconform, $u_0 = 0$ (0.15h)}
% %  
% %      \addplot [solid, red]
% %      table[x expr = \coordindex+1, y index = 0] 
% %      {usedPictures/comparison/red7/nonconform_f/exactAbsoluteEnergyDifference.txt}; 
% %      \addlegendentry{nonconform, $u_0 = I_\textup{NC} f$ (0.84h)}
% %    \end{loglogaxis}
% %  %\end{axis}
% %\end{tikzpicture}
% %\captionof{figure}{Development of $|E_\textup{NC}(u_j)-E_u|$ for Algorithm 
% %\ref{alg:PrimalDualIteration} respectively  
% %      $|E_\textup{NC}(I_\textup{NC}u_j)-E_u|$ for the conforming algorithm in 
% %      \cite{Bart15-book} for
% %      $\varepsilon=10^{-9}$ and 
% %      $\texttt{red} = 7$.}
% %
% %% red 8
% %
% %\begin{tikzpicture}
% %  %\begin{axis}[] 
% %    \begin{axis}[legend pos = north east, 
% %      scale = 2, xlabel = {number of iterations}, 
% %      ylabel = {discrete energy},
% %      my filter = every 50 between 250 and 1e5,filter discard warning = false,
% %      ymax = 1]      
% %
% %      \addplot [solid, black] 
% %      table[x expr = \coordindex+1, y index = 0] 
% %      {usedPictures/comparison/red8/conform_0/energyVec.txt}; 
% %      \addlegendentry{conform, $u_0 = 0$ (0.17h)}
% %    
% %      \addplot [solid, blue] 
% %      table[x expr = \coordindex+1, y index = 0] 
% %      {usedPictures/comparison/red8/conform_f/energyVec.txt}; 
% %      \addlegendentry{conform, $u_0 = I_\textup C f$ (0.12h)}
% %   
% %      \addplot [solid, green] 
% %      table[x expr = \coordindex+1, y index = 0] 
% %      {usedPictures/comparison/red8/nonconform_0/energyVec.txt}; 
% %      \addlegendentry{nonconform, $u_0 = 0$ (0.15h)}
% %  
% %      \addplot [solid, red]
% %      table[x expr = \coordindex+1, y index = 0] 
% %      {usedPictures/comparison/red8/nonconform_f/energyVec.txt}; 
% %      \addlegendentry{nonconform, $u_0 = I_\textup{NC} f$ (0.84h)}
% %    \end{axis}
% %  %\end{axis}
% %\end{tikzpicture}
% %\captionof{figure}{Development of $E_\textup{NC}(u_j)$ for Algorithm 
% %\ref{alg:PrimalDualIteration} respectively  
% %      $E_\textup{NC}(I_\textup{NC}u_j)$ for the conforming algorithm in 
% %      \cite{Bart15-book} for
% %      $\varepsilon=10^{-9}$ and 
% %      $\texttt{red} = 8$.}
% %
% %
% %\begin{tikzpicture}
% %  %\begin{axis}[] 
% %    \begin{loglogaxis}[legend pos = south west, 
% %      scale = 2, xlabel = {number of iterations}, 
% %      ylabel = {absolute difference of the discrete energy and $E_u$},
% %      my filter = every 50 between 250 and 1e5,filter discard warning = false]      
% %
% %      \addplot [solid, black] 
% %      table[x expr = \coordindex+1, y index = 0] 
% %      {usedPictures/comparison/red8/conform_0/exactAbsoluteEnergyDifference.txt}; 
% %      \addlegendentry{conform, $u_0 = 0$ (0.17h)}
% %    
% %      \addplot [solid, blue] 
% %      table[x expr = \coordindex+1, y index = 0] 
% %      {usedPictures/comparison/red8/conform_f/exactAbsoluteEnergyDifference.txt}; 
% %      \addlegendentry{conform, $u_0 = I_\textup C f$ (0.12h)}
% %   
% %      \addplot [solid, green] 
% %      table[x expr = \coordindex+1, y index = 0] 
% %      {usedPictures/comparison/red8/nonconform_0/exactAbsoluteEnergyDifference.txt}; 
% %      \addlegendentry{nonconform, $u_0 = 0$ (0.15h)}
% %  
% %      \addplot [solid, red]
% %      table[x expr = \coordindex+1, y index = 0] 
% %      {usedPictures/comparison/red8/nonconform_f/exactAbsoluteEnergyDifference.txt}; 
% %      \addlegendentry{nonconform, $u_0 = I_\textup{NC} f$ (0.84h)}
% %    \end{loglogaxis}
% %  %\end{axis}
% %\end{tikzpicture}
% %\captionof{figure}{Development of $|E_\textup{NC}(u_j)-E_u|$ for Algorithm 
% %\ref{alg:PrimalDualIteration} respectively  
% %      $|E_\textup{NC}(I_\textup{NC}u_j)-E_u|$ for the conforming algorithm in 
% %      \cite{Bart15-book} for
% %      $\varepsilon=10^{-9}$ and 
% %      $\texttt{red} = 8$.}
% %
% %\newpage
% %
% %\subsubsection{Best Result}
% %
% %\begin{tikzpicture}
% %  %\begin{axis}[] 
% %    \begin{loglogaxis}[legend pos = south west, 
% %      scale = 2, xlabel = {number of iterations}, 
% %      ylabel = {exact $L^2$ error},
% %      my filter = every 50 between 250 and 1e5,filter discard warning = false]      
% %
% %      \addplot [solid, black] 
% %      table[x expr = \coordindex+1, y index = 0] 
% %      {usedPictures/bestResult/red7/errorExactVec.txt}; 
% %      \addlegendentry{nonconform, $u_0 = I_\textup{NC}f$ (34.6h)}
% %
% %    \end{loglogaxis}
% %  %\end{axis}
% %\end{tikzpicture}
% %\captionof{figure}{Development of the exact error in the $L^2$ norm for
% %$\varepsilon=10^{-11}$ and 
% %$\texttt{red} = 7$.}
% %
% %
% %\begin{tikzpicture}
% %  %\begin{axis}[] 
% %    \begin{axis}[legend pos = north east, 
% %      scale = 2, xlabel = {number of iterations}, 
% %      ylabel = {discrete energy},
% %      my filter = every 50 between 250 and 1e5,filter discard warning = false,
% %      ymax = 1]      
% %      \addplot [solid, black] 
% %      table[x expr = \coordindex+1, y index = 0] 
% %      {usedPictures/bestResult/red7/energyVec.txt}; 
% %      \addlegendentry{nonconform, $u_0 = I_\textup{NC}f$ (34.6h)}
% %    \end{axis}
% %  %\end{axis}
% %\end{tikzpicture}
% %\captionof{figure}{Development of $E_\textup{NC}(u_j)$ for Algorithm 
% %\ref{alg:PrimalDualIteration} for
% %      $\varepsilon=10^{-11}$ and 
% %      $\texttt{red} = 7$.}
% %
% %
% %\begin{tikzpicture}
% %  %\begin{axis}[] 
% %    \begin{loglogaxis}[legend pos = south west, 
% %      scale = 2, xlabel = {number of iterations}, 
% %      ylabel = {absolute difference of the discrete energy and $E_u$},
% %      my filter = every 50 between 250 and 1e5,filter discard warning = false]      
% %
% %      \addplot [solid, black] 
% %      table[x expr = \coordindex+1, y index = 0] 
% %      {usedPictures/bestResult/red7/exactAbsoluteEnergyDifference.txt}; 
% %      \addlegendentry{nonconform, $u_0 = I_\textup{NC}f$ (34.6h)}
% %
% %    \end{loglogaxis}
% %  %\end{axis}
% %\end{tikzpicture}
% %\captionof{figure}{Development of $|E_\textup{NC}(u_j)-E_u|$ for Algorithm 
% %\ref{alg:PrimalDualIteration} for
% %      $\varepsilon=10^{-11}$ and 
% %      $\texttt{red} = 7$.}
% 


%\newpage
%\begin{theorem}[convergence]
%For any sequence of meshes $(\T_\ell)_{\ell\in\mathbb{N}} $ in $\mathbb{T}$ with 
%respective discrete solutions $(\ucr^{(\ell)})_{\ell\in\mathbb{N}}$ and 
%maximal mesh-sizes  $h_\ell \to 0^+$ as $\ell\to \infty$
%\[
%\lim_{\ell\to\infty} || u- \ucr^{(\ell)}||_{L^2(\Omega)}=0
%\]
%for the unique minimizer $u$ of the continuous problem.  \qed
%\end{theorem}
%
%The theorem leaves open the question whether the discrete energy converges to the right function and hence numerical experiments need to tell us.
%
%\section{Convergence}
%
%Let $\mathbb{T}$ denote a set of uniformly shape-regular triangulations of the polyhedral bounded 
%Lipschitz domain $\Omega$. Given any $\delta>0$, the subset $\mathbb{T}(\delta)$ denotes the 
%triangulations in $\mathbb{T}$ with maximal mesh-size smaller than or equal to $\delta$. 
%
%\begin{theorem}[convergence]
%For any sequence of meshes $(\T_\ell)_{\ell\in\mathbb{N}} $ in $\mathbb{T}$ with 
%respective discrete solutions $(\ucr^{(\ell)})_{\ell\in\mathbb{N}}$ and 
%maximal mesh-sizes  $h_\ell \to 0^+$ as $\ell\to \infty$
%\[
%\lim_{\ell\to\infty} || u- \ucr^{(\ell)}||_{L^2(\Omega)}=0
%\]
%for the unique minimizer $u$ of the continuous problem. Moreover, the $L^1$ from of the jumps 
%of the discrete minimizers tends to zero?
%\end{theorem}
%
%\begin{proof}
%{\bf Step 1} establishes a priori bounds  for $|| \ucr  ||_{L^2(\Omega)}+ | \ucr |_{1,1,\nc}\lesssim 1 $.
%\\
%{\bf Step 2}  utilizes trace theorems to prove $ | \ucr |_{\BV} \lesssim | \ucr |_{1,1,\nc}$.
%\\
%{\bf Step 3} concerns an arbitrary sequence   of meshes $(\T_\ell)_{\ell\in\mathbb{N}} $
%with maximal mesh-sizes  $h_\ell \to 0^+$ as $\ell\to \infty$ and the weak compactness of $\BV$ and $L^2$
%to select a subsequence (not relabeled) so that
%\[
%\ucr^{(\ell)} \to u_\infty \text{ strongly in } L^1(\Omega) \quad\text{and}\quad
%\ucr^{(\ell)} \rightharpoonup  u_\infty \text{ weakly in } L^2(\Omega)
%\]
%plus the weak-star convergence of the distributional derivative $\nabla \ucr^{(\ell)}$  to some limit measure
%$\nabla u_\infty $ in  measure: for all $\Phi\in C_0(\R^n)$ the sequence $\langle    \nabla \ucr^{(\ell)} , \Phi\rangle$
%has the limit    $\langle   \nabla u_\infty, \Phi \rangle$ as $\ell\to \infty$.
%\\
%{\bf Step 4} proves the key result
%\begin{equation}\label{keyresult}
%E(u_\infty)\le \liminf_{\ell\to\infty}  E_\nc( \ucr^{(\ell)}).
%\end{equation}
%The point is that $u\infty\in \BV\cap L^2$ has a  $\BV$  seminorm defined by duality with a test function
%$\varphi\in C^1(\Omega;\R^n)$
%with maximum norm $\le 1$ and compact support. 
%The boundary integral   $\|  \bullet \|_{L^1(\partial\Omega)} $
%is included in $E$ and then  
%$ \lvert u_\infty \rvert_{\BV(\Omega)}+ \|  u_\infty \|_{L^1(\partial\Omega)} $
%is the supremum of all values 
%$\int_\Omega u_\infty\ddiv\varphi\dx $
%for some of those $\varphi\in C^1(\overline{\Omega};\R^n)$ with maximum norm $\le 1$.
%For such a fixed test function $\varphi$, $\int_\Omega u_\infty\ddiv\varphi\dx $ is the limit of 
%\[
%\int_\Omega \ucr^{(\ell)}\ddiv\varphi\dx = 
%\sum_{F\in\mathcal{F}} \int_F  (\varphi\cdot\nu_F)\,  [\ucr^{(\ell)}]_F  \, ds 
%-\int_\Omega \varphi\cdot \nabla_\nc  \ucr^{(\ell)}\dx.
%\]
%Since  $\int_F   [\ucr^{(\ell)}]_F  \, ds =0$ for any $F\in\mathcal{F}$, one may subtract the 
%constant value $\varphi(\midp (F))$
% on each face and utilize that $|  (\varphi(x)-\varphi(\midp (F)))\cdot \nu_F| \le \delta L $ 
% with a global Lipschitz continuity constant
% $L$ for $\varphi$ and the maximal mesh-size $h_\ell$. This proves
%\[
%|\int_F  (\varphi\cdot\nu_F)\,  [\ucr^{(\ell)}]_F  \, ds |
%\le   L h_\ell       \|[\ucr^{(\ell)}]_F\|_{L^1(F)} \quad\text{for any }F\in\mathcal{F} .
%\]
%Recall the trace theorems from Step~2 to observe that the jump contributions are bounded 
%by $C_{sr}    L h_\ell   | \ucr^{(\ell)} |_{1,1,\nc}$. Since the maximums norm  of $\varphi$ is  $\le 1$,
%it follows that
%\[
%\int_\Omega \ucr^{(\ell)}\ddiv\varphi\dx \le (1+ C_{sr}    L h_\ell  ) | \ucr^{(\ell)} |_{1,1,\nc}.
%\]
%Since  $\ucr^{(\ell)}$ convereges in $L^1$ to $u_\infty$, it follows that
%\[
%\int_\Omega u_\infty\ddiv\varphi\dx\le \liminf_{\ell\to\infty}   | \ucr^{(\ell)} |_{1,1,\nc}  \quad\text{for all }
%\varphi\in C^1(\overline{\Omega}; \overline{ B(0,1)} ).
%\]
%This implies  that $ \lvert u_\infty \rvert_{\BV(\Omega)}+ \|  u_\infty \|_{L^1(\partial\Omega)} $
%is also bounded by the limes inferior of the nonconforming semi-norms  $| \ucr^{(\ell)} |_{1,1,\nc}$.
%This and the weak convergence of the remaining  energy contributions prove the key result
%\eqref{keyresult}.
%\\
%{\bf Step 5} employs the nonconforming interpolation operator $I_\nc^{(\ell)}$ w.r.t. $\T_{\ell}\in\mathbb{T}$ 
%and his properties like
%$\nabla_\nc  I_\nc^{(\ell)} u=\Pi_0^{(\ell)} \nabla u$ for $u\in H^1_0(\Omega)$ followed by Jensen's inequality (and elementary algebra) 
%to prove 
%\[
%E_\nc(I_\nc^{(\ell)} u))\le E_\nc( u) +(  f-\alpha(u+I^{(\ell)}_\nc u)/2, u- I^{(\ell)}_\nc u) _{L^2(\Omega)}
%\]
%The last term convergence even with rate $h_\ell ^{1/2}$ to zero because of $u\in H^1_0(\Omega)$.  
%It follows 
%\begin{equation}\label{equationfromstep5}
%\limsup_{\ell\to\infty}  E_\nc(I_\nc^{(\ell)} u))\le E(u).
%\end{equation}
%{\bf Step 6} concludes the proof of weak convergence.  In fact, the weak limit $u_\infty$ from Step~3 satisfies 
%the key inequality  \eqref{keyresult} and the immediate result  \eqref{equationfromstep5}. 
%Since  the discrete minimizer
%provides the estimate $E_\nc(\ucr^{(\ell)} ) \le E_\nc(I_\nc^{(\ell)} u))$, it follows 
%\[
%E(u_\infty)\le \liminf_{\ell\to\infty}  E_\nc( \ucr^{(\ell)})\le \liminf_{\ell\to\infty}  E_\nc( \ucr^{(\ell)})
%\le 
%\limsup_{\ell\to\infty}  E_\nc(I_\nc^{(\ell)} u))\le E(u).
%\]
%In other words, the weak limit $u_\infty$ is also a minimizer of the continuous problem and so 
%$u=u_\infty$. Since the limit is unique, the weak convergence follows for all selected subsequences with mesh-sizes 
%tending to zero. Moreover, it follows that
%\begin{equation}\label{equationfromstep5onlimitsofenergy}
%\lim_{\ell\to\infty}  E_\nc( \ucr^{(\ell)})=E(u).
%\end{equation}
%\\
%{\bf Step 7} deduces the proof of strong convergence from the weak convergence plus the energy convergence 
%in Step~6 that implies the norm convergence in $L^2(\Omega)$.  Moreover, 
%\eqref{equationfromstep5onlimitsofenergy} proves 
%\[
%\limsup_{\ell\to\infty} \sum_{F\in\mathcal{F}_\ell} \|   [\ucr^{(\ell)}]_F  \|_F\|_{L^1(F)}=
%\limsup_{\ell\to\infty}   \left(E( \ucr^{(\ell)})- E_\nc( \ucr^{(\ell)})\right).
%\]
%\end{proof}
%
% 
%
%
%\section{a posteriori error control} 
%
%TBC
%
%\begin{lemma}
%Any $a,b\in\R^n$ with $\alpha \in\sign a$ and $\beta \in\sign b$ satisfy
%\[
%\frac{|b|}{2}\, |\alpha-\beta |^2\le   |b| - |a| + \alpha\cdot (a-b)
%\]
%and 
%\[
%\frac{ |a|+|b|}{2}\, |\alpha-\beta |^2\le (a-b)\cdot (\alpha-\beta).
%\]
%The latter inequality is strict if and only if either ($|\alpha|<1=|\beta|$ and $|a|=0< |b|$) or if  
%($|\beta|<1=|\alpha|$ and $|b|=0<|a|$).
%\end{lemma}
%
%\begin{proof}
%This is most likely well known and a short proof is added for completeness: The multiplication of 
%the obvious inequality 
%\[
%\frac{1}{2}\, |\alpha-\beta |^2=(|\alpha|^2+|\beta|^2)/2 -\alpha\cdot \beta \le 1-\alpha\cdot \beta 
%\]
%with $|b|=\beta\cdot b \ge 0$ is followed by the observation $\alpha\cdot a=|a|$ to rewrite the 
%obtained upper bound  $|b|(1-\alpha\cdot \beta)=  |b| - |a| + \alpha\cdot (a-b)$.
%
%The second asserted inequality follows from the sum of the first inequality with  its analog
%$\frac{|a|}{2}\, |\alpha-\beta |^2\le   |a| - |b| + \beta\cdot (b-a)$.  
%\end{proof}
%
%
%
%
%
%
%
%
%
%The Raviart-Thomas finite element space $\RT(\T)$
%and its convex subset $Q(\T)$, 
%\begin{align*}
%\RT(\T)&:=\left\lbrace \left. q \in H(\ddiv,\Omega) \right\vert \,  \forall T \in \T \, 
%\exists a_T \in \R^n\,  \exists b_T \in \R  \, 
%\forall x \in T  \right.\\ 
%&  \qquad\quad \left. q(x)=\mathbf a_T+ b_T ( x -\operatorname{mid}(T))\right\rbrace,  
%%Q(\T)&:=  \{ \qrt\in \RT(\T): \, | \Pi_0 \qrt |\le 1\text{ a.e. in }\Omega\}, 
%%\mathcal{D}:=\{    (\vcr,\qrt\)in \CR(\T)\times \RT(\T): | \Pi_0 \qrt |\le 1\text{ a.e. in }\Omega\}, 
%\end{align*}
%are defined as globally $H(\ddiv,\Omega)$ conforming. Given $f\in L^2(\Omega)$ let $F_1\in P_0(\T;\R^n)$ and
%$S(\T)\in P_0(\T;\R^{n\times n})$ be a constant vector $F_1|_T$ and SPD matrix  $S(\T)|_T$ on $T\in\T$ by
%\begin{align*}
%F_1|_T&:= \mint_T  (x-\midp(T))f(x)\dx\\
%S(\T)|_T&:=\mint_T  (x-\midp(T))\otimes (x-\midp(T))\dx;  \\ 
%\mathcal{C}&:=\{    (\vcr,\qrt)\in \CR(\T)\times \RT(\T): \\
%& \qquad| \Pi_0 \qrt  -\alpha S(\T)\nabla_\nc \vcr+F_1 |\le 1
%\text{ a.e. in }\Omega
%\}.
%\end{align*}
%
%\begin{theorem}[discrete duality]
%(a) There exist a unique discrete minimizer
%$\ucr=\operatorname{argmin} E_\nc ( \CR(\T))$.  
% \\
%(b) The discrete minimizer  $\ucr$ in (a)  is equivalently characterized as the unique solution 
%$\ucr\in \CR(\T)$ to the discrete variational inequality
%\[
%(f-\alpha\ucr, \vcr-\ucr)_{L^2(\Omega)}\le | \vcr |_{1,1,\nc }-| \ucr |_{1,1,\nc }\,\text{ for all }
%\vcr\in \CR(\T).
%\]
%(c)  Given the discrete solution  $\ucr$ from (a)-(b), there exists at least one selection  
%$p_0\in  P_0(\T;\R^n)$ with $ p_0\in \sign (\nabla_\nc \ucr)$ a.e. in $\Omega$, abbreviated as
%$p_0\in  \left( \sign (\nabla_\nc \ucr)\right) \cap P_0(\T;\R^n)$, 
%such that 
%\[
%\prt(x):=  p_0+\alpha S(\T) \nabla_\nc \ucr -F_1 + \frac{\Pi_0(\alpha \ucr- f) }{n}(x-\operatorname{mid}(T))
%\] 
% at $x\in \operatorname{int}(T)$ for  any $ T\in\T $
%defines an $H(\ddiv,\Omega)$ conforming Raviart-Thomas function 
%$\prt\in RT_0(\T)\subset H(\ddiv,\Omega)$. \\
%(d) Suppose that $\vcr\in \CR(\T)$ and $\qrt\in RT_0(\T)\subset H(\ddiv,\Omega)$ satisfy 
%\[
%\Pi_0 \qrt-\alpha S(\T) \nabla_\nc \vcr +F_1\in\sign (\nabla_\nc \vcr)
%\text{ and } \ddiv \qrt=\Pi_0(\alpha \vcr- f). 
%\]
%Then $\vcr=\ucr$ solves (a)-(b). \\
%% and $p_0:=\Pi_0 \qrt+\alpha S(\T) \nabla_\nc \ucr  +F_1\in\sign (\nabla_\nc \ucr)$.\\
%(e) Any pair $(\ucr,\prt)$ with (a)-(c) is a saddle point of the functional
%\[
%L(\vcr,\qrt):=  \frac{\alpha}{2}   \|  \Pi_0 \vcr \|_{L^2(\Omega)}^2  
%-\int_\Omega (f +\ddiv\qrt)\Pi_0 \vcr \dx
%\]
%for $(\vcr,\qrt)$ in $\mathcal{C}$. %:=  \{ \rrt\in \RT(\T): \, | \Pi_0 \rrt |\le 1\text{ a.e. in }\Omega\}$.
%Conversely, any saddle point $(\ucr,\prt)$ of $L$ in $\mathcal{C}$
%satisfies (a)-(c). \\
%(f)  Any $\prt$ in (c)  is a maximizer of the dual functional $E_\nc^*$ in  $Q(\T)$, where 
%\[
%E_\nc ^*(\qrt):=  (f +\ddiv\qrt, \vcr)_{L^2(\Omega)} - \frac{\alpha}{2}   \|  \Pi_0 \vcr \|_{L^2(\Omega)}^2  
%\;\text{ for all }\qrt\in \RT(\T).
%\]
%Conversely, any maximizer $\prt\in \operatorname{argmax} E_\nc^* ( Q(\T))$  leads to a selection $p_0:=\Pi_0 \prt$
%with (c). \\
%(g) There is no discrete duality gap in  that
%$ \max E_\nc^*(Q(\T))=\min E_\nc(  \CR(\T))$.
%\end{theorem}
%
%
%\begin{proof}
%Standard arguments on the quadratic growth and the continuity of the discrete energy $E_\nc (\vcr)$ with respect to 
%$\vcr$ in the fixed finite-dimensional space  $\CR(\T)$ provide the existence of a minimizer $\ucr$ of $E_\nc$ in
%$\CR(\T)$. Moreover, this minimizer is equivalently characterized as the solution of the variational inequality
%and more details for (a) and (b) are omitted.  One constructive  way for the existence proof of either the discrete or the continuous minimization problem utilizes a regularization of the $L^1$ norm. For instance, the modulus 
% $|\bullet |$ may be replaced by a differentiable upper bound 
%$|\bullet|_\varepsilon$, defined for any $\varepsilon>0$ by 
%\[
%| F |_\varepsilon:=  \sqrt{   \varepsilon^2+ F\cdot F }\quad\text{for all }F\in\R^n;
%\]
%This leads to a regularized  nonconforming  energy \eqref{e:defnonconfEintro} with the substitution of
%$\int_\Omega |\nabla_\nc \vcr |_\varepsilon\dx$  for   $| \vcr |_{1,1,\nc }$. The same substitution applies to the
%discrete variational inequality in (b)  which becomes an equality 
%for $| \bullet |_\varepsilon$ is differentiable for any positive $\varepsilon$. For any $\varepsilon> 0$ there 
%exists a unique minimizer to the regularized nonconforming energy and the necessary stationary condition applies for the smooth functional and results in 
%that $\ucrvarepsilon \in \CR(\T)$  satisfies
%\begin{equation*}
%(f-\alpha\ucrvarepsilon , \vcr)_{L^2(\Omega)} =   \int_\Omega 
%% \frac{ \nabla_\nc  \ucrvarepsilon \cdot \nabla_\nc \vcr}{ \sqrt{   \varepsilon^2+ |\nabla_\nc  \ucrvarepsilon|^2 }} 
%p_\varepsilon \cdot \nabla_\nc \vcr \dx
%\,\text{ for all }\vcr\in \CR(\T)
%\end{equation*} 
%with the abbreviation 
%\begin{equation}\label{e:definepvarepsilon}
%p_\varepsilon :=  \frac{ \nabla_\nc  \ucrvarepsilon}{ \sqrt{   \varepsilon^2+ |\nabla_\nc  \ucrvarepsilon|^2 }}\in P_0(\T;\R^n)
%\quad \text{and}\quad |p_\varepsilon|\le 1\text{ a.e. in }\Omega.
%\end{equation} 
%The test with $\vcr=\ucrvarepsilon $ and standard arguments reveal that $\ucrvarepsilon \in \CR(\T)$ is bounded 
%(in any norm for the fixed finite-dimensional vector space  $\CR(\T)$) as $\varepsilon \to 0^+$. 
%Any accumulation point $(\ucr,p_0)\in \CR(\T)\times P_0(\T;\R^n)$ of bounded subsequences 
%$(\ucrvarepsilon,p_\varepsilon)$  as $\varepsilon \to 0^+$ satisfies 
%\begin{align}
%\label{e:limitidentityforp0}
%% \text{ and } \text{ }  \\ 
%(f-\alpha\ucr, \vcr)_{L^2(\Omega)} &=   \int_\Omega p_0 \cdot \nabla_\nc \vcr \dx
%\,\text{ for all }\vcr\in \CR(\T) ; \\
%\label{e:limitboundednessetalforp0}
%|p_0|\le 1  \quad \text{and}&\quad
%p_0\cdot \nabla_\nc\ucr=| \nabla_\nc\ucr| \quad\text{  a.e. in }\Omega.
%\end{align}
%Substitute the test function $\vcr\in \CR(\T)$ in  \eqref{e:limitidentityforp0} by  $\vcr-\ucr$ (for some fixed limit $\ucr$
%and) any  $\vcr\in \CR(\T)$ to obtain with \eqref{e:limitboundednessetalforp0}
%the identity 
%\[
%(f-\alpha\ucr, \vcr-\ucr)_{L^2(\Omega)}=  \int_\Omega p_0 \cdot \nabla_\nc \vcr \dx - | \ucr |_{1,1,\nc }.
%\]
%Since    $|p_0|\le 1$  a.e. in $\Omega$ implies
%$   \int_\Omega p_0 \cdot \nabla_\nc \vcr \dx\le  | \vcr |_{1,1,\nc }$, this becomes  the discrete variational inequality
%(b). In other words, any selected accumulation point  $\ucr$ of the discrete solutions $\ucrvarepsilon \in \CR(\T)$
%as $\varepsilon \to 0^+$ is equal to the unique solution $\ucr$  in (b). This implies convergence 
%$\ucrvarepsilon \to \ucr$  as $\varepsilon \to 0^+$ but in general not for $p_\varepsilon$  from
%\eqref{e:definepvarepsilon}. However,  any  accumulation point (of the possibly many choices) for 
%$p_0$ leads to  \eqref{e:limitidentityforp0}-\eqref{e:limitboundednessetalforp0}.
%
%The conditions in \eqref{e:limitboundednessetalforp0} also read
%$p_0\in  \left( \sign (\nabla_\nc \ucr)\right) \cap P_0(\T;\R^n)$.  
%The abbreviations $F_1\in P_0(\T;\R^n)$ and $S(\T)\in P_0(\T;\R^{n\times n})$ rewrite  the 
%left-hand side of \eqref{e:limitidentityforp0} as
%%\[(f-\alpha\ucr, \vcr)_{L^2(\Omega)}
% $(\Pi_0 f -\alpha\Pi_0 \ucr,  \vcr)_{L^2(\Omega)}+ (F_1-\alpha S(\T)  \nabla_\nc \ucr, \nabla_\nc \vcr)_{L^2(\Omega)}$.
%Hence,  for all $\vcr\in \CR(\T)$, \eqref{e:limitidentityforp0} is equivalent to 
%\[
%(\Pi_0 f -\alpha\Pi_0 \ucr,  \vcr)_{L^2(\Omega)}=(p_0+\alpha S(\T) \nabla_\nc \ucr -F_1 ,\nabla_\nc \vcr)_{L^2(\Omega)}.
%\]
%It is behind the well-established Marini identity that the foregoing equality 
%for all Crouzeix-Raviart function guarantees the unique existence of a Raviart-Thomas function 
%$\prt\in H(\ddiv,\Omega)$ with   $\Pi_0(\alpha\ucr-f)=\ddiv \prt$ and $p_0+\alpha S(\T) \nabla_\nc \ucr -F_1=\Pi_0 \prt $.  
%This concludes the proof of (c).  
%
%%\bigskip
%% Alternative notation of (c) 
%%\[
%%\Pi_0(\alpha\ucr-f)=\ddiv \prt\qquad
%%\Pi_0 \prt +\alpha S(\T) \nabla_\nc \ucr +F_1\in  \sign (\nabla_\nc \ucr).
%%\]
%Given a pair $(\vcr, \qrt)\in \CR(\T)\times RT_0(\T)$ with the assumptions in (d), set
%$q_0:= \Pi_0 \qrt-\alpha S(\T) \nabla_\nc \vcr +F_1$ and 
%let $\wcr \in \CR(\T)$ denote an arbitrary 
%test function. A careful analysis of interface and boundary terms within 
%a piecewise  integration by parts shows
%$(\qrt, \nabla_\nc(\wcr-\vcr))_{L^2(\Omega)}=(  \Pi_0(f- \alpha \vcr)  ,\wcr-\vcr)_{L^2(\Omega)}$.
%This and the aforementioned split into piecewise constant and remaining affine terms result in 
%\begin{equation}\label{e:someidentityinbetweentohelptheproofofc}
%( f- \alpha \vcr  ,\wcr-\vcr)_{L^2(\Omega)}
%=(q_0,  \nabla_\nc (\wcr-\vcr))_{L^2(\Omega)} 
%\end{equation}
%On the other hand,  $q_0\in\sign (\nabla_\nc \vcr)$, implies
%$| \vcr |_{1,1,\nc}=(q_0, \nabla_\nc \vcr  )_{L^2(\Omega)}$ and 
%$(q_0, \nabla_\nc \wcr  )_{L^2(\Omega)}\le | \wcr |_{1,1,\nc}$.
%Consequently, \eqref{e:someidentityinbetweentohelptheproofofc} is smaller than or equal to 
%$ | \wcr |_{1,1,\nc}-| \vcr |_{1,1,\nc}$. In other words, $\vcr$ solves the discrete variational inequality (b).
%This concludes the proof of (d). 
%
%\bigskip
%
%In the proof of first implication in (e) suppose that  $(\ucr,\prt)$ satisfies  (a)-(c) and so $(\ucr,\prt)\in\mathcal{C}$.
%Since $\ddiv\prt= \Pi_0(\alpha \ucr- f)$, elementary algebra shows for any $\vcr\in\CR(\T)$ that 
% \[
% \frac \alpha 2 \|  \Pi_0( \ucr-\vcr)\|_{L^2(\Omega)}^2+ L(\ucr,\prt)=L(\vcr,\prt).
% \]
%Elementary calculations  and a piecewise integration by parts show for any $\qrt\in\RT(\T)$ that
%\begin{align} 
%\label{e:piecewiseorthogonalitiesandintrationbypartsinproofofd}
%L(\ucr,\qrt) & +(f-\alpha \ucr, \ucr)_{L^2(\Omega)}+ \frac \alpha 2\|  \Pi_0\ucr\|^2 \\
%&= (\nabla_\nc\ucr, \Pi_0\qrt+F_1-\alpha S(\T)\nabla_\nc \ucr )_{L^2(\Omega)}
%\nonumber
%\end{align}
%The variable $\qrt$ solely enters on the 
%right-hand side of \eqref{e:piecewiseorthogonalitiesandintrationbypartsinproofofd}, 
%which is bounded from above by $|\ucr|_{1,1,\nc}$ 
%whenever $(\ucr,\qrt)\in\mathcal{C}$ with  $|\Pi_0\qrt+F_1-\alpha S(\T)\nabla_\nc \ucr|\le 1$. Moreover, 
%the upper bound  $|\ucr|_{1,1,\nc}$  is attained at  $\qrt=\prt$ from (c). This proves 
%$L(\ucr,\qrt)\le  L(\ucr,\prt) $ whenever $(\ucr,\qrt)\in\mathcal{C}$. In conclusion, 
% $(\ucr,\prt)$ is a saddle point in that
%\begin{equation}\label{e:saddlepointdiscrete}
%L(\ucr,\qrt)\le  L(\ucr,\prt)\le L(\vcr,\prt).
%\end{equation}
%
%In the proof of reverse implication in (e), 
%let $(\ucr,\prt)\in \CR(\T)\times  Q(\T)$ denote some solution to  (a)-(c) and
%suppose  that  $(\widetilde\ucr,\widetilde\prt)\in \CR(\T)\times  Q(\T)$ is another saddle point of $L$
%(i.e., \eqref{e:saddlepointdiscrete} holds with  $(\widetilde\ucr,\widetilde\prt)$ replacing $(\ucr,\prt)$).
%The  functional $L(\vcr,\widetilde\prt)$ is quadratic in the 
%variable $\vcr$ in $\CR(\T)$ and attains a minimum at $\widetilde\ucr$.  Hence the partial derivative 
%has to vanish there;  i.e.,  $\Pi_0(\alpha \widetilde\ucr-f)=\ddiv \widetilde\prt$. 
%This also holds for the saddle point $(\ucr,\prt)$. Consequently, 
%$\alpha \Pi_0( \ucr-\widetilde\ucr )=   \ddiv (\prt- \widetilde\prt )$. On the other hand, 
%the  functional $L(\widetilde\ucr,\qrt)$ in the variable $\qrt$  with
%$(\widetilde\ucr,\qrt)\in\mathcal{C}$ attains a maximum at $\widetilde\prt$
%; i.e.,
% $(\widetilde \ucr, \ddiv(\widetilde \prt-\qrt))_{L^2(\Omega)}\le 0 $.
% 
% 
% \newpage
% 
%  for all $\qrt\in Q(\T)$ and, e.g.,
%for $\qrt:=\prt$. Analog  arguments apply to the saddle point $(\ucr,\prt)$ and prove
%$( \ucr, \ddiv( \prt-\widetilde \prt))_{L^2(\Omega)}\le 0 $. The sum of the two resulting inequalities reads
%$( \ucr-\widetilde \ucr, \ddiv( \prt-\widetilde \prt))_{L^2(\Omega)}\le 0$ and  so 
%$\alpha \Pi_0( \ucr-\widetilde\ucr )=   \ddiv (\prt- \widetilde\prt )$ proves 
%$\alpha \|\Pi_0( \ucr-\widetilde\ucr )\|^2_{L^2(\Omega)}\le 0$. Consequently 
% $\Pi_0 \ucr=\Pi_0 \widetilde\ucr $ and  $\ddiv \prt =  \ddiv\widetilde\prt $. 
% 
% 
% \newpage
% 
% Moreover, each of the two aforementioned inequalities  cannot be strict and has to be an equality 
% (for their sum is zero), i.e., $\ucr$ and $\widetilde \ucr$ are $L^2$ orthogonal to $\ddiv( \prt-\widetilde \prt)$.
% 
%$
%( \ucr, )_{L^2(\Omega)} = 0 =(\widetilde \ucr, \ddiv(\widetilde \prt-\prt))_{L^2(\Omega)}
%$
%
%\newpage
%
%
%satisfies 
%for all  $(\vcr,\qrt)$  in    $\CR(\T)\times  Q(\T)$. The functional $L(\ucr,\qrt)$ with respect to 
%$ \qrt \in Q(\T)$ is maximal
%at $\prt$. The calculation in \eqref{e:piecewiseorthogonalitiesandintrationbypartsinproofofd}
%proves $(\nabla_\nc\ucr, \Pi_0(\qrt-\prt))_{L^2(\Omega)}\le 0 $.
%
%
%\eqref{e:piecewiseorthogonalitiesandintrationbypartsinproofofd}
%
%
%Therefore, $(\ucr, \prt)$ satisfy (d) and so solve (a)-(c).  
%\end{proof}
%
%
%
%
%The evaluation of the $L^2$ integrals in midpoints 
%and the piecewise constant average $\Pi_0 f$ of the data $f$ is not just 
%faster than the full scalar product (without $\Pi_0$ on \eqref{e:defnonconfEintro}) but another discretization step 
%that shall be useful in the discrete duality principle.
%
%
%\begin{align*}
%& \\
%&
%\end{align*}
%
%
%
%\section{The continuous  and discrete problem}
%The continuous problem involves a low-order term 
%with nonnegative  parameters $\alpha,\mu \ge 0$ with $\alpha+\mu>0$ and reads 
%\begin{align}\label{e:defEintro}
%E(v)& :=    \frac{\alpha}{2} \|  v \|_{L^2(\Omega)}^2 + \frac{\mu}{2}\| \nabla v\|^2_{L^2(\Omega)}
%+ \lvert v\rvert_{\BV(\Omega)}+ \|  v \|_{L^1(\partial\Omega)}  -\int_\Omega f\, v\dx
%\end{align}
%for $V:=\BV(\Omega)\cap L^2(\Omega)$. The meaning of  $\lvert\nabla v\rvert_{\BV}$ is that of a $\BV$ 
%seminorm in case of $\mu=0<\alpha$ with the 
%solution space $\BV(\Omega)\cap L^2(\Omega)$  minimizers  belong to. It is a result of Brezis that there exists a minimizer 
%in $H^1_0(\Omega)$ under certain extra conditions. Then, and also  if $\mu>0$, the meaning of  $\lvert v\rvert_{BV}$ 
%is that of the seminorm of $v$ in $W^{1,1}(\Omega)$.
%
%The discrete situation concerns $E$ with a test function $v_{\Cr}\in \CR(\T)$ and the same meaning of $E(v_\Cr)$ as in 
%\eqref{e:defEintro}. 
%A closer inspection of the $\BV$ seminorm shows that it reads 
%\begin{equation}\label{e:introBVCR}
%\lvert v_\Cr\rvert_{\BV(\Omega)} = \|  \nabla_\nc v_\Cr \|_{L^1(\Omega)}+ \|  [ v_\Cr]_\mathcal{F} \|_{L^1(\cup\mathcal{F}(\Omega) )}
%\quad\text{for all }v_{\Cr}\in \CR(\T)
%\end{equation}
%with the jumps $[\bullet]_\mathcal{F}$ of the, in general, discontinuous discrete functions  across the interior
%element faces $\mathcal{F}$.
%
%The homogeneous boundary conditions are modeled by the extra term  $\|  v \|_{L^1(\partial\Omega)} $, which exists 
%as a corollary of a trace theorem for BV functions. The sum 
%$\lvert v_\Cr\rvert_{\BV(\Omega)} +  \|  v_\Cr \|_{L^1(\partial\Omega)} $ is abbreviated as
%\[
%\lvert v_\Cr\rvert_{\overline{\BV}}= \|  \nabla_\nc v_\Cr \|_{L^1(\Omega)}+ \|  [ v_\Cr]_\mathcal{F} \|_{L^1(\cup\mathcal{F} )}
%\quad\text{for all }v_{\Cr}\in \CR(\T)
%\]
%with the sum over all faces and the associated convention  $[ v_\Cr]_\mathcal{F} |_F:= [ v_\Cr]_F := v_\Cr|_F $ along a boundary face $F\in \mathcal{F} (\partial\Omega)$. 
%
%
%The continuous and the discrete problem have unique minimizers $u\in V$ and $u_\Cr\in \CR(\T)$.
%Those solutions are characterized by the following  respective variational inequalities with unique 
%$u\in V$ and  $u_\Cr\in \CR(\T)$ with 
%\begin{align*}%\label{e:defVI}
%&(\alpha u-f , u-v)_{L^2(\Omega)} + \mu a_\nc (u, u-v) \le  \\
%&\hspace{55mm}  \lvert v\rvert_{\BV(\Omega)}+\|  v \|_{L^1(\partial\Omega)}
% - \lvert u\rvert_{\BV(\Omega)}-\|  u \|_{L^1(\partial\Omega)} , \\
%&(\alpha u_\Cr-f , u_\Cr-v_\Cr)_{L^2(\Omega)} + \mu a_\nc (u_\Cr, u_\Cr-v_\Cr) \le 
%\lvert v_\Cr\rvert_{\overline{\BV}} - \lvert u_\Cr\rvert_{\overline{\BV}} 
%\end{align*}
%for all $v\in V$ and all $v_{\Cr}\in \CR(\T)$. Since nothing else then boundedness is known a~priori about 
%the $ \BV(\Omega)$ seminorm of  $ u_\Cr$,   $v:= v_\Cr$ appears a natural choice for the test function on the
%continuous level while $v_\Cr$ will be an approximation to $I_\nc u$ below. The sum of the continuous and discrete 
%variational inequalities lead to
%\begin{align}\label{e:sumofVIs}
%\alpha \|  u-u_\Cr \|_{L^2(\Omega)}^2 &+ \mu \||   u-u_\Cr \||_\nc ^2
%\le \lvert v_\Cr\rvert_{\BV(\Omega)}   - \lvert u\rvert_{\BV(\Omega)}\\
%& +(f-\alpha u_\Cr, u -v_\Cr)_{L^2(\Omega)} - \mu a_\nc (u_\Cr, u-v_\Cr). \nonumber
%\end{align}
%The difficulty with the standard choice  $v_\Cr$ as $I_\nc u$ is that the $ \BV(\Omega)$ seminorm of 
%$I_\nc u$ involves its jumps and an straightforward analysis of them leads to plain convergence but with no 
%rates even under the optimistic assumption $u\in H^1_0(\Omega)$ --- which holds under particular additional conditions. 
%
%The design of $v_\Cr:= I_\nc u_\delta$ hence follows the literature and first applies a regularization $u_\delta$ to $u$
%by regularization with a parameter $\delta$. Assuming $u\in H_0^1(\Omega)$ on a polyhedral bounded Lipschitz 
%domain $\Omega\subset \R^n$, this leads to some $u_\delta\in C^\infty_0(\Omega)$ with 
%the approximation properties 
%\[
%\| u- u_\delta \|_{L^1(\Omega)} \le  \delta    C(0)  \, \lvert u  \rvert_{\BV(\Omega)} 
%\quad\text{and}\quad  \lvert u_\delta  \rvert_{\BV(\Omega)} \le (1+\delta) \lvert u  \rvert_{\BV(\Omega)} 
%\]
%and, for any partial derivative $D_\alpha $ with respect to the multi-index $\alpha\in \mathbb{N}_0^n $
%of order $|\alpha|\ge 2$,
%the stability  property 
%\[
%\| D_\alpha u_\delta \|_{L^1(\Omega)} \le  \delta^{1-|\alpha|}    C(|\alpha|)  \, \lvert u  \rvert_{\BV(\Omega)} 
%\]
%for a generic constant  $C(|\alpha|)$, which 
%depends on the domain $\Omega$ and on the order  $|\alpha|$  but not on $\delta$.
%Jensen's inequality and $\Pi_0 \nabla u_\delta=  \nabla_\nc I_\nc u_\delta$ 
%imply 
%\[
%\| \nabla_\nc  I_\nc u_\delta  \|_{L^1(\Omega)}\le \| \nabla u_\delta  \|_{L^1(\Omega)}.
%\]
%This, \eqref{e:introBVCR}, and the definition of the Crouzeix-Raviart %finite element 
%functions  lead to
%\[
%\lvert  I_\nc u_\delta  \rvert_{\BV(\Omega)}   - \lvert u_\delta \rvert_{\BV(\Omega)}
%\le  \|  [ I_\nc u_\delta]_\mathcal{F} \|_{L^1(\cup\mathcal{F} )}
%\approx \sum_{F\in \mathcal{F}}
% h_F \, \|  [ \nabla_\nc  I_\nc u_\delta]_F\|_{L^1(F)}.
%\]
%The jump $[ \nabla_\nc  I_\nc u_\delta]_F=-[ \nabla_\nc  (1-I_\nc) u_\delta)]_F$ is treated with a triangle inequality
%followed by a trace inequality to derive  
%\[
%\lvert  I_\nc u_\delta  \rvert_{\BV(\Omega)}   - \lvert u_\delta \rvert_{\BV(\Omega)}
%\lesssim    
%\|\nabla_\nc  (1-I_\nc) u_\delta)  \|_{L^1(\Omega)}+  \|   h_\T D^2    u_\delta  \|_{L^1(\Omega)}.
%\]
%A standard approximation result on $K\in\T$ shows  $\|\nabla_\nc  (1-I_\nc) u_\delta)  \|_{L^1(K)}
%\lesssim  h_K\,  \| D^2 u_\delta \|_{L^1(K)}$. The combination with the previous estimate and 
%the aforementioned stability estimate leads with the maximal mesh-size $h:=\max h_\T$ to
%\[
%\lvert  I_\nc u_\delta  \rvert_{\BV(\Omega)}   - \lvert u_\delta \rvert_{\BV(\Omega)}
%\le C_1   h/\delta\, \lvert u \rvert_{\BV(\Omega)}
%\]
%The second estimate for  $\|| I_\nc (u-u_\delta) \||_\nc$ starts with an inverse followed by a triangle inequality  to deduce
%\begin{align*}
%\|| I_\nc (u-u_\delta) \||_\nc & \lesssim 
% \|  h^{-1}_\T  I_\nc (u-u_\delta) \|  \\
% &\le  \|  h^{-1}_\T  (u-I_\nc  u_\delta) \| +
%  \|  h^{-1}_\T  (u-I_\nc  u) \|   \\
%  & \lesssim \||  u \|| + \|  h^{-1}_\T  (u-I_\nc  u_\delta) \| 
%\end{align*}
%with a standard estimate in the last step. It remains to control the third term 
%\begin{align*}
%\|  h^{-1}_\T  (u-I_\nc  u_\delta) \| 
%\lesssim 
% \|  h^{-1}_\T  (u- u_\delta)  \| + \|| u_\delta - I_\nc u_\delta \||_\nc
%\end{align*}
%
%
%These arguments plus the previous key estimate result in \eqref{e:sumofVIs} to
%\begin{align}\label{e:mainresult}
%\alpha \|  u-u_\Cr \|_{L^2(\Omega)}^2 &+ \mu \||   u-u_\Cr \||_\nc ^2
%\le (\delta  + C_1 h/\delta)  \lvert u\rvert_{\BV(\Omega)}\\
%& \| f-\alpha u_\Cr\|_{L^2(\Omega)}   ( \| u - u_\delta \|_{L^2(\Omega)}  + h/\delta  \|   \mu a_\nc (u_\Cr, u-u_\delta ). \nonumber
%\end{align}
%
%\section{One little lemma}
%
%
%
%The convex modulus function $|\bullet|$ has the multivalued subgradient $\sign$, defined by
%$\sign F:= \{ F/|F| \}$ for $F\in\R^n\setminus\{0\}$ while $\sign 0:= \overline{ B(0,1)}$ is the closed unit ball in 
%$\R^n$ endowed   with  the Euclidean scalar product "$\cdot$". 
%
%
%
%\begin{lemma}
%Any $a,b\in\R^n$ with $\alpha \in\sign a$ and $\beta \in\sign b$ satisfy
%\[
%\frac{|b|}{2}\, |\alpha-\beta |^2\le   |b| - |a| + \alpha\cdot (a-b)
%\]
%and 
%\[
%\frac{ |a|+|b|}{2}\, |\alpha-\beta |^2\le (a-b)\cdot (\alpha-\beta).
%\]
%The latter inequality is strict if and only if either ($|A|<1=|B|$ and $|a|=0< |b|$) or if  ($|B|<1=|A|$ and $|b|=0<|a|$).
%\end{lemma}
%
%\begin{proof}
%This is most likely well known and a short proof is added for completeness: The multiplication of 
%the obvious inequality 
%\[
%\frac{1}{2}\, |\alpha-\beta |^2=(|\alpha|^2+|\beta|^2)/2 -\alpha\cdot \beta \le 1-\alpha\cdot \beta 
%\]
%with $|b|=\beta\cdot b \ge 0$ is followed by the observation $\alpha\cdot a=|a|$ to rewrite the 
%obtained upper bound  $|b|(1-\alpha\cdot \beta)=  |b| - |a| + \alpha\cdot (a-b)$.
%
%The second asserted inequality follows from the sum of the first inequality with  its analog
%$\frac{|a|}{2}\, |\alpha-\beta |^2\le   |a| - |b| + \beta\cdot (b-a)$.  
%\end{proof}
%
%
%
%
%
%
%
%\begin{proof}
%
%
%
%
%In the situation of the lemma, 
%$\alpha:=|a|$ and $\beta:=|b|$ are nonnegative while $|A|\le 1$ and $|B|\le 1$. Elementary algebra 
%shows that right-hand side $(a-b)\cdot (A-B)$ of the lemma is equal to
%\[
%(\alpha A-\beta B)\cdot(A-B)= \frac{\alpha+\beta}{2}|A-B|^2+\frac{\alpha-\beta}{2}(|A|^2-|B|^2).
%\]
%The first term on the right-hand side is the left-hand side and it remains to prove that 
%$(\alpha-\beta)(|A|^2-|B|^2)\ge 0$. Since the assertion is symmetric in $a$ and $b$, we may and will assume without loss of generality that $\alpha\le \beta$. In case of equality $\alpha=\beta$, the critical term vanishes and it remains to consider
%$\alpha<\beta$. Then $|A|\le 1= |B|$ and so $(\alpha-\beta)(|A|^2-|B|^2)\ge 0$. This concludes the proof. 
%\end{proof}
%\subsection{Overview/Main Contributions}
%The analysis focusses on the Euler-Lagrange equations with sub gradients from convex analysis in Section~2.
%Their pointwise definition allows a first estimate in Subsection~2.3, which controls the nonlinearity completely for this paper. What follows is the appropriate treatment of the nonconforming contributions with the help of the medius analysis in the form of interpolation operators and conforming companions of Section~3. The combination leads to the a~priori error estimate of 
%Secion~4.
% 
%
%\subsection{Reguarized Moslov Modell Problem}
%The  Mosolov's problem  \cite{FM77,Glowinski76,Glowinski08,GLT76} in the Bingham flow
%leads to the energy density 
%\begin{equation}
% W(F) =  \frac{\mu}{2} |F|^2 +  |F|
% \qquad\text{for all }F\in\mathbb{R}^2\label{energypot}
%\end{equation}
%with two positive material parameters $g$ and $\mu$. 
%
%
%With homogeneous Dirichlet boundary 
%conditions on $\partial \Omega$, the resulting variational problem seeks
%$u\in H^1_0(\Omega)$ with
%\begin{align}\label{e:MinProbIntro}
%E(u)=\min_{v\in H^1_0(\Omega)}E(v).
%\end{align}
%The parameters $\alpha,g,\mu$ are all positive, but the regime this analysis aims at is that $\mu$ is very small 
%in comparison to $g$ and $\alpha$. This allows for a functional analysis in $H^1_0(\Omega)$, 
%but the energy density is driven more by the norm $W^{1,1}_0(\Omega)$. As a rule of thumb, the energy norm
%$\|| \bullet \|| := \| \nabla\bullet\|:= |\bullet|_{H^1(\Omega)}$ is finite but too large while is product with $\mu^{1/2}$
%is bounded or the seminorm   $j(\bullet):=\| \nabla\bullet \|_{L^1(\Omega)}:=|\bullet|_{W^{1,1}(\Omega)}$  in    
%$W^{1,1}(\Omega)$ is reasonably bounded. 
%
%
%\begin{theorem}[Euler-Lagrange]\label{t:EulerLagrange}
%There exists a unique  minimizer $u\in H^1_0(\Omega)$ in \eqref{e:MinProbIntro} and 
%the Euler-Lagrange equations hold in the sense
%that there exists $\sigma\in H(\operatorname{div},\Omega)$ with
%$\sigma\in\partial W(\nabla u)$ a.e. in $\Omega$
%and
%\begin{align*}
% \big( \sigma , \nabla v \big)_{L^2(\Omega)}=(f-\alpha\, u, v)_{L^2(\Omega)}
% \qquad\text{for all }v\in H^1_0(\Omega).
% \end{align*}
% The quantity $p:=\sigma-\mu\nabla u\in L^2(\Omega;\R^2)$ satisfies $p\in \partial j(u)$
% a.e. in $\Omega$ in the sense that, for a.e. $x\in\Omega$, 
% \[
% p(x) \cdot (q-\nabla u(x))\le  g|q| -   g|\nabla u(x)|\qquad\text{for all } q\in\R^2.
% \]
%\end{theorem}
%
%\begin{proof}
%This is a minor generalisation of  \cite[Chapter II, Theorem~6.3]{Glowinski08} 
%and \cite[Theorem~1.1]{MM66}.
%\end{proof}
%
%
%\section{Discretization}
%\subsection{Triangulations and Finite Element Spaces}\label{ss:triangFEM}
%A shape-regular triangulation %$\tri$ of a polygonal
%bounded Lipschitz domain $\Omega\subseteq \R^n$ with $n=2$ or $n=3$ 
%is a set of simplices (triangles if $n=2$ and tetrahedra of $n=3$)
%%such that $\overline{\Omega}=\bigcup\tri$ and any two distinct 
%simplices are 
%either disjoint or share exactly one common face, edge or vertex. 
%%Let $\E$ denote the set of edges for $n=2$ and the 
%%set of faces for $n=3$ of $\tri$ and $\N$ the set of vertices.
%Let
%%\begin{align*}
%%\begin{array}{rl}
%%P_k(T;\R^m) &:= \{v_k:T\rightarrow \R^m\;|\;\forall j=1,\ldots,m,\text{ the component}\\
%%      &\qquad\qquad\qquad\;v_k(j) \text{ of }v_k
%%\text{ is a polynomial of total degree}\leq k\},\\
%%P_k(\tri;\R^m) &:= \{v_k:\Omega\rightarrow \R^m\;| \;
%%\forall T\in\tri,v_k|_T \in P_k(T;\R^m)\}
%%\end{array}
%%\end{align*}
%denote the set of piecewise polynomials;
%The piecewise constant function 
%%$\midp(\tri)\in P_0(\tri)$ is defined by 
%%$\midp(\tri)\vert_T=\midp(T)$ for a
%%simplex $T\in\tri$ with barycenter $\midp(T)$.
%For an edge or face $E\in\mathcal{E}$, 
%$\midp(E)$ denotes the midpoint of $E$.
%%The $L^2$-projection onto $\tri$-piecewise constant functions or vectors
%%${\Pi_0:L^2(\Omega;\R^m)\rightarrow P_0(\tri;\R^m)}$ 
%is defined by 
%$(\Pi_0 f)\vert_T=\fint_T f \dx:=\int_T f\dx/\vert T\vert$ 
%%for all $T\in\tri$ with area $\vert T\vert$ for $n=2$ and 
%volume $\vert T\vert$ for $n=3$
%and all $f\in L^2(\Omega;\mathbb{R}^m)$. 
%%Let $h_\tri\in P_0(\tri)$ denote the piecewise constant mesh-size 
%%with $h_\tri\vert_T:=\operatorname{diam}(T)$ for all $T\in\tri$.
%The oscillations of $f$ are defined by 
%%$\osc(f,\tri):=\left\|h_\tri (f-\Pi_0 f)\right\|_{L^2(\Omega)}$.
%The jump along an interior edge or face $E$ with
%adjacent simplices $T_+$ and $T_-$, i.e., $E=T_+\cap T_-$,
%is defined by $[v]_E:=v\vert_{T_+}-v\vert_{T_-}$. 
%The jump along 
%%boundary edges or faces $E\in\E(\Gamma_D)$ reads
%%$[v]_E:=v\vert_{T_+}$ for that simplex $T_+\in\tri$ with $E\subset T_+$
%due to the homogeneous Dirichlet boundary conditions.
%
%%For piecewise affine functions $v_h\in P_1(\tri)$ the 
%$\T$-piecewise gradient $\gradnc v_h$ with 
%$(\gradnc v_h)\vert_T=\nabla (v_h\vert_T)$ for all $T\in\mathcal T$ 
%and, accordingly, $\divnc(\tau_h)$ for $\tau_h\in P_1(\mathcal T;\mathbb{R}^2)$, 
%exists and $\gradnc v_h\in P_0(\mathcal T;\mathbb{R}^{2})$
%and $\divnc(v_h)\in P_0(\mathcal T)$. 
%
%\subsection{$P_1$ Nonconforming Discretization}\label{ss:NCFEM}
%The $P_1$ nonconforming finite element space \cite{CR73},
%named after Crouzeix and Raviart, reads
%\begin{align*}
% \CR(\mathcal T):=\{\vcr\in P_1(\mathcal T)\;\vert\;\vcr \text{ is continuous at midpoints of interior}\\ 
% \text{edges and vanishes at midpoints of boundary edges}\}
%\end{align*}
%and motivate the discrete energy
%\begin{align}\label{e:dMinProbIntro}
% E_\nc(\vcr):= \frac{\alpha}{2} \||  \vcr  ||_{L^2(\Omega)}^2+ \int_\Omega W(\gradnc\vcr)\dx   -F(\vcr).
%\end{align}
%The nonconforming discretization to \eqref{e:MinProbIntro} minimizes 
%$E_\nc$ in $\CR(\mathcal T)$.
%
%\begin{theorem}[discrete Euler-Lagrange]\label{t:dEulerLagrange}
%There exists a unique  minimizer $u_{CR}$ of \eqref{e:dMinProbIntro} in $CR^1_0(\mathcal T)$ and 
%the Euler-Lagrange equations hold in the sense
%that there exists $\sigma_{CR}\in P_0(\mathcal T;\R^2)$ with
%$\sigma_{CR}\in\partial W(\nabla_\nc u_{CR})$ a.e. in $\Omega$
%and
%\begin{align*}
% \big( \sigma_{CR} , \nabla_\nc \vcr \big)_{L^2(\Omega)}=(f-\alpha\, \ucr, \vcr )_{L^2(\Omega)}
% \qquad\text{for all }\vcr\in CR^1_0(\mathcal T).
% \end{align*}
% The quantity $p_{CR}:=\sigma_{CR}-\mu\nabla u_{CR}\in P_0(\tri;\R^2)$ satisfies $p_{CR}\in \partial j(u_{CR})$
% a.e. in $\Omega$ in the sense that, for a.e. $x\in\Omega$, 
% \[
% p_{CR}(x) \cdot (q-\nabla_\nc u_{CR}(x))\le  g|q| -   g|\nabla_\nc u_{CR}(x)|\qquad\text{for all } q\in\R^2.
% \]
%\end{theorem}
%
%\begin{proof}
%This is the discrete   analog to Theorem~\ref{t:EulerLagrange}.
%\end{proof}
%
%\subsection{First Estimate}
%The detailed discussion of the pointwise subgradients implies the starting point 
%of our error analysis.
%
%\begin{lemma}\label{lemma1}
%The unique minimizers $u$ of \eqref{e:MinProbIntro} in $H^1_0(\Omega)$
%and $\ucr$ of  \eqref{e:dMinProbIntro} in $CR^1_0(\tri)$ with their subordinated stress $\sigma$ from Theorem~\ref{t:EulerLagrange}
%and discrete stress $\sigma_{CR}$  from Theorem~\ref{t:dEulerLagrange} satisfy
%\[
%\mu \, |||  u-\ucr |||_\nc^2\le (\sigma-\sigma_{CR}, \nabla_\nc (u-u_{CR}))_{L^2(\Omega)}.
%\]
%\end{lemma}
%
%\begin{proof}
%Theorem~\ref{t:EulerLagrange} allows  $q:=\nabla_\nc \ucr(x) $ to deduce 
%\[
%p(x) \cdot \nabla_\nc (\ucr-u)(x)\le  g|\nabla_\nc \ucr(x)| -   g|\nabla u(x)| 
%\quad\text{for a.e. }x\in\Omega.
%\]
%Theorem~\ref{t:dEulerLagrange} allows  $q:=\nabla u(x) $ to deduce 
% \[
% p_{CR}(x) \cdot \nabla_\nc (u-u_{CR})(x) \le  g|\nabla u(x)| -   g|\nabla_\nc u_{CR}(x)|
% \quad\text{for a.e. }x\in\Omega.
% \]
%The sum of those inequalities is integrated over the domain to prove
%\[
%0\le (p-p_{CR},  \nabla_\nc (u-u_{CR}))_{L^2(\Omega)}.
%\]
%The substitution of $\sigma=\mu\nabla u+p$ and $\sigma_{CR}=\mu\nabla_\nc u_{CR}+p_{CR}$
%concludes the proof.
%\end{proof}
%
%Some interpolation and approximation operators are required to analyze further  the 
%right-hand side of the estimate of the lemma. 
%
%The analysis in \cite{CRS16} utilizes Jensen's inequality  $\|\Pi_0 p\|_{L^1(\Omega)} \leq \|p\|_{L^1(\Omega)} $  \cite{Evans2010}. That improves the estimate of the lemma only by an additional non-negative term
% $\|p\|_{L^1(\Omega)} -\|\Pi_0 p\|_{L^1(\Omega)} $ in the lower bound. 
%
%\section{Interpolation Error Analysis}
%The nonconforming interpolation $I_\nc$ maps $W^{1,1}(\Omega)$ onto  $CR_0^1(\tri)$ via
%\[
%(I_\nc v)(\mid(E))= \int  v\, ds /|E| \quad\text{for all }E\in \mathcal{E}.
%\]
%The stability and approximation main properties of this nonconforming interpolation operator 
%are well known for $L^2$ norms and completed here for $W^{1,1}$ sind Sobolev functions
%in $W^{1,1}$  allow for integrable traces  along the edges of triangles.  One remarkable property is the
%integral mean property of the gradient, i.e., for any $v\in W^{1,1}(\Omega)$, 
%\[
%\Pi_0  \nabla v=\nabla_\nc  I_\nc v\in P_0(\tri;\R^2).
%\]
%Given the shape-regular triangulation $\tri$, let 
%$h_\tri$ denote the piecewise constant mesh-size 
%function with $h_\tri\vert_T:=\operatorname{diam}(T)$
%on $T\in\tri$ and maximal mesh-size
%$h_{\max} :=\max h_\tri:=\left\|h_\tri\right\|_{L^\infty(\Omega)}$.
%
%\begin{theorem}
%The aforementioned operator $I_\nc v\in CR^1(\tri)$ is well defined for all $v\in W^{1,1}(\Omega) $ 
%and satisfies the first-order approximation and stability property
%\begin{align}\label{e:INCApproxStabLp}
% \|  h_\tri^{-1}  (v-I_\nc v)   \|_{L^p(\Omega)}
%+  \| \nabla_\nc  I_\nc v\|_{L^p(\Omega)}
%\lesssim \| \nabla  v\|_{L^p(\Omega)}
%\end{align}
%for all $v\in W^{1,p}(\Omega) $ for all $1\le p\le \infty$. 
%\end{theorem}
%
%\begin{proof}TBC\end{proof}
%
%The  conforming $P_4$ companion $J_4\vcr $ of the nonconforming function $\vcr$ of \cite{CGS}
%leads to a linear operator  $J_4:   CR_0^1(\tri)\to P_4(\tri)\cap C_0(\Omega)$ which is some polynomial
%right-inverse of $I_\nc$.
%
%\begin{proposition}\label{proponJ4}
% For any $\vcr\in\CR(\tri)$ there exists some
% $J_4\vcr\in P_4(\Omega)\cap C_0(\Omega)$
%such that 
%(a) $\vcr-J_4\vcr$ is $L^2$ orthogonal on
% the space $P_1(\tri)$ of piecewise first-order
%polynomials,
%\[
%\vcr-J_4\vcr\perp P_1(\tri),
%\]
%(b) it enjoys the integral mean property
%of the gradient 
%\begin{align*}
% \Pi_0(\nabla_\nc(\vcr-J_4\vcr))=0\quad\text{and so}\quad
% J_4 |_\nc = 1 \text{ on } CR_0^1(\tri),
%\end{align*}
%and (c) satisfies the approximation and stability
%property
%\begin{align}\label{e:J4ApproxStabL2}
% \|  h_\tri^{-1}  (\vcr-J_4\vcr) \|_{L^2(\Omega)}
%+\||  \vcr-J_4\vcr\||_\nc \lesssim \min_{v\in H_0^1(\Omega)}\|| \vcr-v\||_\nc .
%\end{align}
%The operator is (c) stable in $L^1$ in that  
%\[
% \|  J_4\vcr \|_{L^1(\Omega)} \lesssim\|  \vcr \|_{L^1(\Omega)}
% \qquad \text{for all }v_{CR}\in  CR_0^1(\tri),
% \] 
% (d) is stable in $W^{1,1}$ in that  
%\[
% \|  \nabla_\nc J_4\vcr \|_{L^1(\Omega)} \lesssim\|  \nabla_\nc \vcr \|_{L^1(\Omega)}
% \qquad \text{for all }v_{CR}\in  CR_0^1(\tri),
% \] 
%and  (e) satisfies the first-order approximation and the stability property in  $W^{1,1}$ 
%in the sense that, for all $v_{CR}\in  CR_0^1(\tri)$, 
%\begin{align}\label{e:J4ApproxStabL1}
% \|  h_\tri^{-1}  (\vcr-J_4\vcr) \|_{L^1(\Omega)}
%  \lesssim \min_{v\in H_0^1(\Omega)}\|| \nabla_\nc (\vcr-v) \||_{L^1(\Omega)} .
%\end{align}
%\end{proposition}
%
%\begin{proof}
%The parts (a)-(c) are  explicitly included as Prop 4.1 in  \cite{CGS}; the other parts  are minor generalizations.
%TBC
%\end{proof}
%
%\section{Error Analysis}
%The a~priori error analysis is kept explicit in the positive parameters $\alpha$, $g$, and $\mu$.
%
%\begin{theorem}[main result]\label{t:main}
%The  minimizer $u$ and the discrete minimizer  $u_{CR}$ 
%satisfy 
%\begin{align*}
%&\alpha \| u-\ucr\|_{L^2(\Omega)}^2 +\mu \||   u-\ucr \||_\nc^2
% \le \alpha \| u-I_\nc u \|_{L^2(\Omega)}^2
%+\mu  C_1^2\| |   u-I_\nc u\| |_\nc^2 \\
%&+2(f-\alpha \, u, (1-I_\nc)(u-J_4 \ucr))_{L^2(\Omega)} 
%+2 \big( p-\Pi_0 p  , \nabla_\nc J_4  \ucr \big)_{L^2(\Omega)}.
% \end{align*}
%\end{theorem}
%
%\begin{proof}
%The easier part is the discrete contribution in the 
%further analysis  of  the upper bound in Lemma~\ref{lemma1}.
%Since $\sigma_{CR}\perp   \nabla_\nc (u- I_\nc u)$, the discrete Euler Lagrange equations show
%\[
%-( \sigma_{CR}, \nabla_\nc (u-u_{CR}))_{L^2(\Omega)}= (\alpha \ucr-f,  I_\nc u-\ucr)_{L^2(\Omega)}.
%\]
%Since $\Pi_0\sigma\perp   \nabla_\nc (\ucr-J_4\ucr ) $ in $L^2(\Omega)$ and 
%since $v:= u- J_4 u_{CR}$ is an admissible test function  
%in  the Euler Lagrange equations, it follows
%\[
% (\sigma, \nabla_\nc (u-u_{CR}))_{L^2(\Omega)}
% = ( f-\alpha u, u- J_4 u_{CR})_{L^2(\Omega)}
% + (\sigma-\Pi_0\sigma, \nabla_\nc J_4\ucr )_{L^2(\Omega)}.
%\]
%The combination of the preceding two identities with  Lemma~\ref{lemma1}  proves
%\begin{align}
%&\alpha ||u-\ucr||_{L^2(\Omega)}^2+\mu \, |||  u-\ucr |||_\nc^2
%\label{eq1section4}
%\le    (\sigma-\Pi_0\sigma, \nabla_\nc J_4\ucr ))_{L^2(\Omega)} \\
%&+( f-\alpha\, u , (1-  I_\nc)(   u- J_4 u_{CR}))_{L^2(\Omega)}  \nonumber
%+\alpha(u-\ucr , u-  I_\nc  u)_{L^2(\Omega)} .
%\end{align}
%Since $\sigma-\Pi_0\sigma=\mu\nabla_\nc(u-I_\nc u)+p-\Pi_0$,
%\begin{align*}
%(\sigma-\Pi_0\sigma, \nabla_\nc J_4\ucr ))_{L^2(\Omega)} 
%\le & \mu\|| u-I_\nc u\|_\nc \,  \|| \ucr- J_4\ucr \||_\nc  \\
%&+ (p-\Pi_0, \nabla_\nc J_4\ucr ))_{L^2(\Omega)}. 
%\end{align*}
%With \eqref{e:J4ApproxStabL2} and a constant $C_1$ hidden therein,
% $\|| \ucr- J_4\ucr \||_\nc\le C_1\|| u- \ucr \||_\nc$.
%The combination with the previous estimate shows
%\begin{align*}
%(\sigma-\Pi_0\sigma, \nabla_\nc J_4\ucr ))_{L^2(\Omega)} 
%\le &\frac\mu 2 C_1^2 \|| u-I_\nc u\|_\nc^2 +  \frac\mu 2\| \ucr- J_4\ucr \||_\nc^2  \\
%&+ (p-\Pi_0, \nabla_\nc J_4\ucr ))_{L^2(\Omega)}. 
%\end{align*}
%This and the obvious inequality
%\[
%2 (u-\ucr , u-  I_\nc   u)_{L^2(\Omega)}\le 
%\| u-\ucr \| _{L^2(\Omega)}^2+  \|  u-  I_\nc   u\|_{L^2(\Omega)}^2
%\] 
%lead in  \eqref{eq1section4}  to the assertion and so conclude the proof.
%\end{proof}
%
%\section{Comments}
%\subsection{A priori bounds}
%The obvious fact $E_\nc (\ucr)\le 0$ implies
%\[
%\alpha ||\ucr||_{L^2(\Omega)}^2+\mu \, |||  \ucr |||_\nc^2+ g \| \nabla_\nc \ucr \|_{L^1(\Omega)}
%\le \frac 1\alpha ||f||_{L^2(\Omega)}^2.
%\]
%As a consequence, $\ucr$ and $u$ are uniformly bounded in $W^{1,1}(\tri)$ and so
%\[
%\|h_\tri^{-1}(1-I_\nc)(u-J_4 \ucr)\|_{L^1(\Omega)}  \lesssim 1
%\]
%Under the extra regularity assumption $f-\alpha u\in L^\infty(\Omega)$, this proves the optimistic result
%\[
%2(f-\alpha \, u, (1-I_\nc)(u-J_4 \ucr) )_{L^2(\Omega)}  
%\lesssim \| h_\tri (f-\alpha u) \|_{L^\infty(\Omega)}.
%\]
%The remaining term crucial term is 
%\[
% \big( p-\Pi_0 p  , \nabla_\nc J_4  \ucr \big)_{L^2(\Omega)}\lesssim 
% \|  p-\Pi_0 p\|_{L^\infty(\Omega)} \| \nabla_\nc \ucr \|_{L^1(\Omega)}.
%\] 
%It is clear that $p\in L^\infty(\Omega;\R^2)$ is pointwise bounded $|p|\le 1$ a.e. but the convergence 
%of  $\|  p-\Pi_0 p\|_{L^\infty(\Omega)}$ does not hold without further assumptions. 
%
%\subsection{Large Viscosity}
%In case that $1\lesssim \mu$, the above error control allows refinements and so recovers the analysis of 
%\cite{CRS16} in the present situation and enlarges those results to the slightly more general situation at hand:
%\begin{align*}
%\alpha \| u-\ucr\|_{L^2(\Omega)}^2 +\mu \||   u-\ucr \||_\nc^2
% &\le \alpha \| u-I_\nc u \|_{L^2(\Omega)}^2
%+\mu  C_1^2\| |   u-I_\nc u\| |_\nc^2 \\
%&+ \|  h_\tri( f-\alpha \, u)\|_{L^2(\Omega)}^2
%+\| p-\Pi_0 p  \|_{L^2(\Omega)}^2.
% \end{align*}
%
%
%
%
%
%
%\subsection{Regularized Problem}\label{ss:defReg}
%
%The regularisation is one possibility to approximate the discrete solution. 
%Given any $\varepsilon>0$,
%define $W_\varepsilon\in C^1(\mathbb{R}^2;\mathbb{R}^2)$ by
%\begin{align*}
% W_\varepsilon(F):=(\mu/2) \lvert F\rvert^2 
%    + g\left(\sqrt{\lvert F\rvert^2+\varepsilon^2}-\varepsilon\right).
%\end{align*}
%The regularized problem seeks 
%$u_{\varepsilon,\operatorname{CR}}\in\CR(\tri)$
%with
%\begin{align}\label{e:dReg}
%E_{\varepsilon,\nc} (u_{\varepsilon,\operatorname{CR}})
%=\min_{\vcr\in\CR(\tri)} E_{\varepsilon,\nc} (\vcr)
%\end{align}
%for the modified energy
%\begin{align*}
%E_{\varepsilon,\nc}(\vcr)&:=\int_\Omega W_\varepsilon(\gradnc\vcr)\dx-F(\vcr).
%\end{align*}
%
%
%\begin{theorem}\label{t:apriori}
%Any solution $(u,p,\sigma)\in X\times H(\operatorname{div},\Omega)$
%to \eqref{e:3FF} and any
%discrete solution 
%$(u_h,p_h,\sigma_h)\in (P_0(\tri)\times P_0(\tri;\R^2))\times\RT(\tri)$
%to \eqref{e:d3FF} satisfies
%\begin{align}
%&\left\|\ddiv(\sigma-\sigma_h)\right\|_{L^2(\Omega)}
%    = \left\|f-\Pi_0 f\right\|_{L^2(\Omega)},\label{e:divsigmaEqf}\\
%%
% &\mu\left\|p-\Pi_0 p\right\|_{L^2(\Omega)}^2/2+\vert j(p)-j(\Pi_0 p)\vert
%   \leq \left\|\sigma-\Pi_0\sigma\right\|_{L^2(\Omega)}^2/(2\mu),
%                           \label{e:EstpPi0p}
%                           \\
%%
% &\begin{aligned}
%  \mu\left\|p-p_h\right\|_{L^2(\Omega)}^2/2&+\vert j(p)-j(\Pi_0 p)\vert\\
% &\leq \Big(\min_{\tau_h\in Q(f,\tri)}\left\|\sigma-\tau_h\right\|_{L^2(\Omega)}^2 
%  + \left\|h_\tri f\right\|_{L^2(\Omega)}^2/2\Big)/\mu,
%  \end{aligned}
%  \label{e:Estpph}
%  \\
%%
%  &\left\|u-u_h\right\|_{L^2(\Omega)}
%  \lesssim \left\|u-\Pi_0 u\right\|_{L^2(\Omega)}
%      + \left\|p-p_h\right\|_{L^2(\Omega)}\label{e:Estuuh}.
%\end{align}
%\end{theorem}
%
%\begin{corollary}\label{c:aprioriCR}
%The solution $u\in H^1_0(\Omega)$ to \eqref{e:MinProbIntro} and
%the solution $\ucr\in\CR(\tri)$ to \eqref{e:CRVI}
%satisfy
%\begin{align*}
% \mu \left\|\gradnc(u-\ucr)\right\|_{L^2(\Omega)}^2
% \leq \Big( 4\min_{\tau_h\in Q(f,\tri)} \left\|\sigma-\tau_h\right\|_{L^2(\Omega)}^2
%    &+2\left\| h_\tri f\right\|_{L^2(\Omega)}^2\Big)/\mu \\
%    &+4\osc^2(f,\tri).
%\end{align*}
%\end{corollary}
%
%\begin{theorem}[direct analysis of $P_1$ nonconforming FEM]\label{t:aprioriCR}
%The solution $u\in H^1_0(\Omega)$ and the approximation $\ucr\in\CR(\tri)$
%satisfy
%\begin{align*}
% \left\|\gradnc(u-\ucr)\right\|_{L^2(\Omega)}
%   \lesssim \left\|\sigma-\Pi_0\sigma\right\|_{L^2(\Omega)}
%      + \osc(f,\tri).
%\end{align*}
%\end{theorem}
%\begin{theorem}\label{t:aPrioriReg}
%The discrete solution $\ucreps\in\CR(\tri)$ to \eqref{e:dReg}
%satisfies
%\begin{equation}\label{e:aprioriNC}
%\begin{aligned}
% \mu\left\|\gradnc( u-\ucreps)\right\|_{L^2(\Omega)}^2/2
%\leq \varepsilon g &\lvert\Omega\rvert
%    + 4\min_{\tau_h\in Q(f,\tri)} \left\|\sigma-\tau_h\right\|_{L^2(\Omega)}^2/\mu\\
%    &+2\left\| h_\tri f\right\|_{L^2(\Omega)}^2/\mu
%     +4\osc^2(f,\tri)\ .
%\end{aligned}
%\end{equation}
%\end{theorem}
%
%
%
%
%
%
%
%\subsection{Proof of Theorem~\ref{t:aprioriCR}}
%Let $\sigma\in H(\ddiv,\Omega)$ from  
%Theorem~\ref{t:EulerLagrange}. 
%Then $\sigma(x)\in\partial W(\nabla u(x))=\mu\nabla u(x)+\partial j(\nabla u(x))$
%implies, for all $q\in L^2(\Omega)$, that
%\begin{align*}
% (\sigma-\mu \nabla u,q-\nabla u)_{L^2(\Omega)}
%  \leq j(q)-j(\nabla u).
%\end{align*}
%Let $\sigma_{\operatorname{CR}}\in P_0(\tri;\mathbb{R}^2)$
%from Theorem~\ref{t:dEulerLagrange}.
%Then $\sigma_{\operatorname{CR}}\in \partial W(\gradnc\ucr)
%=\mu\gradnc\ucr+\partial j(\gradnc\ucr)$ implies
%for $q_h\in P_0(\tri;\mathbb{R}^2)$
%\begin{align*}
% (\sigma_{\operatorname{CR}}-\mu\gradnc\ucr,q_h-\gradnc\ucr)_{L^2(\Omega)}
%   \leq j(q_h)-j(\gradnc\ucr).
%\end{align*}
%Let $q=\gradnc\ucr$ and $q_h=\Pi_0\nabla u$
%in the previous inequalities. The sum of the 
%two inequalities yields
%\begin{equation}\label{e:proofCRdirect}
%\begin{aligned}
% \mu\left\|\gradnc(u-\ucr)\right\|^2
%  &\leq (\sigma-\sigma_{\operatorname{CR}},\gradnc(u-\ucr))_{L^2(\Omega)}\\
%   &\qquad 
%    + (\sigma_{\operatorname{CR}}-\mu\gradnc\ucr,\nabla u-\Pi_0\nabla u)_{L^2(\Omega)}\\
%   &\qquad+j(\Pi_0\nabla u)-j(\nabla u).
%\end{aligned}
%\end{equation}
%Since $\sigma_{\operatorname{CR}}$ and $\gradnc\ucr$ are piecewise 
%constant, the second term vanishes.
%Jensen's inequality \cite{Evans2010} leads to $j(\Pi_0\nabla u)-j(\nabla u)\leq 0$.
%Let $\N(\Omega)=\N\cap\Omega$ denote the set of the interior
%nodes and $\tri(z):=\{T\in\tri\mid z\in\tri\}$ the 
%set of triangles that share the node $z$. 
%Let $J_3:\CR(\tri)\to(P_3(\tri)\cap H^1_0(\Omega))$
%be defined as in \cite[Subsection~2.4]{CKPS14}, 
%\cite{CGS2014_EVP,CarstensenSchedensack14}
%with the conservation property 
%\begin{align}\label{e:J3conservation}
% \fint_T \nabla J_3 \vcr\,dx &= \fint_T \gradnc \vcr\,dx
%  \qquad\text{for all }T\in\tri
%\end{align}
%and the approximation and stability property
%\begin{align*}
% \left\|h_\tri^{-1}(\vcr-J_3\vcr)\right\|_{L^2(\Omega)}
%  &\approx\left\|\gradnc(\vcr-J_3\vcr)\right\|_{L^2(\Omega)}\\
% &\approx \min_{\varphi\in H^1_0(\Omega)} 
%           \left\|\gradnc(\vcr-\varphi)\right\|_{L^2(\Omega)}\\
%&\leq\left\|\gradnc(\vcr-u)\right\|_{L^2(\Omega)}
%\end{align*}
%for all $\vcr\in\CR(\tri)$.
%Let $I_\nc:H^1_0(\Omega)\to \CR(\tri)$ denote the 
%nonconforming interpolation operator defined by
%\begin{align*}
% I_\nc v(\midp(E)) =\fint_E v\,ds
%\end{align*}
%for all interior edges $E\in\E(\Omega)$.
%A calculation reveals the integral mean property
%$\gradnc I_\nc v=\Pi_0\nabla v$ for all $T\in\tri$ and  
%$v\in H^1_0(\Omega)$.
%This and the conservation property \eqref{e:J3conservation}
%leads to 
%\begin{align*}
% (\sigma-\sigma_{\operatorname{CR}},\gradnc(u-\ucr))_{L^2(\Omega)}
%    = &(\sigma,\nabla(u-J_3\ucr))_{L^2(\Omega)}\\
%      &\qquad + (\sigma-\Pi_0\sigma,\gradnc(J_3\ucr-\ucr))_{L^2(\Omega)}\\
%      &\qquad- (\sigma_{\operatorname{CR}},\gradnc(I_\nc u-\ucr))_{L^2(\Omega)}.
%\end{align*}
%Theorem~\ref{t:EulerLagrange} and \ref{t:dEulerLagrange}
%imply
%\begin{align*}
% (\sigma,\nabla(u-J_3\ucr)&)_{L^2(\Omega)}
%  -(\sigma_{\operatorname{CR}},\gradnc(I_\nc u-\ucr))_{L^2(\Omega)}\\
% &= (f,u-I_\nc u)_{L^2(\Omega)} + (f,\ucr-J_3\ucr)_{L^2(\Omega)}.
%\end{align*}
%The combination of the previous inequalities with 
%the approximation properties of $I_\nc$ and $J_3$ 
%yield
%\begin{align*}
%  (\sigma-\sigma_{\operatorname{CR}},&\gradnc(u-\ucr))_{L^2(\Omega)}\\
%   & \lesssim \big(\left\|h_\tri f\right\|_{L^2(\Omega)}
%       + \left\|\sigma-\Pi_0\sigma\right\|_{L^2(\Omega)}\big)
%         \left\|\gradnc(u-\ucr)\right\|_{L^2(\Omega)}.
%\end{align*}
%The efficiency of $\left\|h_\tri f\right\|_{L^2(\Omega)}$, namely
%\begin{align*}
%  \left\|h_\tri f\right\|_{L^2(\Omega)}
%  \lesssim \left\|\sigma-\Pi_0\sigma\right\|_{L^2(\Omega)}
%   +\osc(f,\tri),
%\end{align*}
%follows from the bubble function technique of \cite[Chapter I]{Verfuerth96}.
%The combination with \eqref{e:proofCRdirect} 
%yields the assertion.
%
%
%\subsection{Proof of Theorem~\ref{t:aPrioriReg}}
%\label{ss:proofReg}
%
%\begin{lemma}\label{l:strongEllipticityReg}
%Any $A,B\in\mathbb{R}^2$ satisfies
%\begin{align*}
% \mu\lvert A-B\rvert^2\leq W_\varepsilon(B)-W_\varepsilon(A)-DW_\varepsilon(A)\cdot(B-A).
%\end{align*}
%\end{lemma}
%
%\begin{proof}
%An elementary calculation reveals
%\begin{align*}
% W_\varepsilon&(B)-W_\varepsilon(A)-DW_\varepsilon(A)\,(B-A)\\
% &= \mu \lvert A-B\rvert^2 
%  + g\big(\sqrt{\lvert B\rvert^2+\varepsilon^2}-\sqrt{\lvert A\rvert^2+\varepsilon^2}
%    - A\cdot (B-A)/\sqrt{\lvert A\rvert^2+\varepsilon^2}\big).
%\end{align*}
%The formula $2ab\leq a^2+b^2$ together with the Cauchy 
%inequality prove
%\begin{align*}
% (A\cdot B+\varepsilon^2)^2
% \leq \lvert A\rvert^2\lvert B\rvert^2
%    +\varepsilon^2(\lvert A\rvert^2+\lvert B\rvert^2)
%     + \varepsilon^4
%  = (\lvert A\rvert^2+\varepsilon^2)(\lvert B\rvert^2+\varepsilon^2).
%\end{align*}
%This yields
%\begin{align*}
% \sqrt{\lvert B\rvert^2+\varepsilon^2}&-\sqrt{\lvert A\rvert^2+\varepsilon^2}
%    - A\cdot (B-A)/\sqrt{\lvert A\rvert^2+\varepsilon^2}\\
%  &= \sqrt{\lvert B\rvert^2+\varepsilon^2} 
%      - (A\cdot B+\varepsilon^2)/\sqrt{\lvert A\rvert^2+\varepsilon^2}
%  \geq 0
%\end{align*}
%This yields the assertion.
%\end{proof}
%
%
%
%\begin{proof}[Proof of Theorem~\ref{t:aPrioriReg}]
%Since $\sqrt{a^2+b^2}\leq a+b$ for $a,b>0$, the functional 
%$j_\varepsilon$ defined by
%\begin{align*}
% j_\varepsilon(F)
% :=g\int_\Omega (\sqrt{\lvert F\rvert^2+\varepsilon^2}
%                                       -\varepsilon)\dx
% \qquad\text{for all }F\in L^2(\Omega;\R^{2\times 2}),
%\end{align*}
%satisfies,
%for all $v_\nc\in H^1_0(\Omega) + \CR(\tri)$, that
%\begin{align*}
% j_\varepsilon(\gradnc v_\nc)
%\leq j(\gradnc v_\nc)
%\leq j_\varepsilon(\gradnc v_\nc) + g\varepsilon \lvert\Omega\rvert.
%\end{align*}
%This implies
%\begin{align}\label{e:distRegEnergies}
% E_{\varepsilon,\nc}(v_\nc)
% \leq E_\nc(v_\nc)
% \leq E_{\varepsilon,\nc}(v_\nc) + g\varepsilon \lvert\Omega\rvert.
%\end{align}
%Lemma~\ref{l:strongEllipticityReg} leads to
%\begin{align*}
%& \mu\left\|\gradnc(u_{\operatorname{CR}}-\ucreps)\right\|_{L^2(\Omega)}^2
%\leq E_{\varepsilon,\nc}(u_{\operatorname{CR}}) - E_{\varepsilon,\nc}(\ucreps)\\
%   &\qquad- \big( DW_\varepsilon(\gradnc u_{\varepsilon,\operatorname{CR}}) , 
%            \gradnc(u_{\operatorname{CR}}-\ucreps\big)_{L^2(\Omega)} 
%         + F(u_{\operatorname{CR}}-\ucreps).
%\end{align*}
%The Euler-Lagrange Equations for the smooth $W_\varepsilon$
%and the previous inequalities lead to 
%\begin{align*}
% \mu\left\|\gradnc(u_{\operatorname{CR}}-\ucreps)\right\|_{L^2(\Omega)}^2
%&\leq E_{\varepsilon,\nc}(u_{\operatorname{CR}}) - E_{\varepsilon,\nc}(\ucreps)\\
%&\leq E_\nc(u_{\operatorname{CR}}) - E_\nc(\ucreps) + g\varepsilon\lvert\Omega\rvert.
%\end{align*}
%Since $u_{\operatorname{CR}}$ minimizes $E_\nc$ in $\CR(\tri)$, 
%this implies
%\begin{align*}
% \mu\left\|\gradnc(u_{\operatorname{CR}}-\ucreps)\right\|_{L^2(\Omega)}^2
%\leq g\varepsilon\vert\Omega\rvert.
%\end{align*}
%A triangle inequality
%%, $\sqrt{a^2+b^2}\leq a+b$ for $a,b>0$,
%and Corollary~\ref{c:aprioriCR}
%conclude the proof.
%\end{proof}
%section{Example with Known Solution}
%\label{ss:resultsForCircularDomain}
%\section{Generalisation to 3D}\label{s:3D}
%
%This section describes the variational inequality 
%for the 3D Bingham problem with its discretization in 
%Subsection~\ref{ss:3DVI}, the  
%three-field formulation with its discretization in Subsection~\ref{ss:3D3FF} and 
%proves in Subsection~\ref{ss:3Dapriori} a priori error bounds.
%
%\subsection{Variational Inequality}
%\label{ss:3DVI}
%
%Let $\mathbb{S}=\{A\in \mathbb{R}^{3\times 3}\mid A=A^\top\}$
%be the space of symmetric matrices and $\mathrm{sym}A=(A+A^\top)/2$
%and define 
%for $q\in L^2(\Omega;\mathbb{R}^{3\times 3})$ 
%\begin{align*}
% j(q) = g\int_\Omega \lvert \mathrm{sym}(q)\rvert\,dx.
%\end{align*}
%The variational inequality for the Bingham flow problem in 3D seeks
%$u\in Z:=\{w\in H^1_0(\Omega;\mathbb{R}^3)\mid \ddiv w=0\}$ with 
%\begin{equation}
%\begin{aligned}\label{e:3DVI}
% \int_\Omega f\cdot(v&-u)\,dx
%  \leq \mu\int_\Omega \varepsilon(u):\varepsilon(v-u)\,dx\\
%  &   + j(\nabla v)
%    - j(\nabla u)
%  \qquad\text{for all }v\in H^1_0(\Omega;\mathbb{R}^3).
%\end{aligned}
%\end{equation}
%% The three-field formulation in $\mathbb{R}^3$ seeks 
%% $u\in L^2(\Omega;\mathbb{R}^3)$, $p\in L^2(\Omega;\mathbb{S})$,
%% and $\sigma\in H(\ddiv,\Omega;\mathbb{S})$ with 
%% \footnote{$\ddiv u=0$?}
%% \begin{align*}
%%  \int_\Omega f\cdot (v-u)\,dx
%%   &\leq \mu\int_\Omega p:(q-p)\,dx + g\int_\Omega |p|\,dx - g\int_\Omega |q|\,dx\\
%%   &\qquad\qquad -\int_\Omega v\cdot\ddiv\sigma\,dx -\int_\Omega q:\tau\,dx,\\
%%   -\int_\Omega u\cdot\ddiv \tau\,dx - \int_\Omega p:\tau\,dx&=0
%% \end{align*}
%% for all $v\in L^2(\Omega;\mathbb{R}^3)$, $q\in L^2(\Omega;\mathbb{S})$,
%% and $\tau\in H(\ddiv,\Omega;\mathbb{S})$.
%A direct discretization of the variational inequality with 
%$P_1$ nonconforming finite elements is not possible,
%since $\int_\Omega \varepsilon_\nc(\bullet):\varepsilon_\nc(\bullet)\,dx$
%is not positive definite on the $P_1$ nonconforming finite element space.
%% A discretization of the three-field formulation with Raviart-Thomas 
%% finite elements is also problematic because of the 
%% symmetry of $\sigma$.
%% However, a discretization with symmetric finite element methods 
%% as the Arnold-Winther FEM \cite{ArnoldWinther2002} is possible.
%% \footnote{How to choose the space for $p_h$? 
%% Existence of solutions?}
%% However, the crucial point in the proof of Theorem~\ref{t:apriori}
%% is an application of Jensen's inequality and uses, that the 
%% approximation space for $p$ is piecewise constant.
%% Thus, a quasi-optimal error bound for this method is not available
%% with the techniques from this paper.
%% \bigskip
%% 
%% 
%For homogeneous Dirichlet boundary conditions, a straightforward 
%calculation reveals that
%\begin{align*}
% 2 \int_\Omega \varepsilon(u):\varepsilon(v)\,dx
%  = \int_\Omega (\nabla u:\nabla v + \ddiv u\ddiv v)\,dx.
%\end{align*}
%Since $\ddiv u=0$,
%this leads to the alternative formulation of \eqref{e:3DVI}:
%Seek $u\in Z$ with 
%\begin{align*}
% \int_\Omega  f\cdot(v-u)\,dx
%  &\leq (\mu/2)\int_\Omega \nabla u:\nabla (v-u)\,dx\\
%  &\qquad  + j(\nabla v)
%    - j(\nabla u) 
% \qquad\text{for all }v\in Z.
%\end{align*}
%Define the energy 
%\begin{align*}
% E_{3D}(v):=\int_\Omega W_{3D}(\nabla v)\,dx - F(v)
%\qquad\text{for all }v\in H^1_0(\Omega;\R^3)
%\end{align*}
%with $W_{3D}(A):= (\mu/4) \lvert A\rvert^2 + g \lvert \mathrm{sym}A\vert$.
%The unique existence of a solution $u$ follows from the equivalence 
%of \eqref{e:3DVI} with the minimization of $E_{3D}$ over $Z$
%as in the two-dimensional case.
%%
%The discretization with $P_1$ nonconforming finite elements
%seeks $\ucr\in Z_\Cr(\tri):=\{w_\Cr\in\CR(\tri;\mathbb{R}^3)\mid 
%\ddiv_\nc w_\Cr=0\}$ with 
%\begin{equation}\label{e:3DVIdisc}
%\begin{aligned}
% \int_\Omega  f\cdot(\vcr-\ucr)\,dx
%  &\leq (\mu/2)\int_\Omega \gradnc \ucr:\gradnc (\vcr-\ucr)\,dx\\
%  &\qquad  + j(\gradnc\vcr) - j(\gradnc\ucr)
%\end{aligned}
%\end{equation}
%for all $\vcr\in Z_\Cr(\tri)$.
%The unique existence of a discrete solution to \eqref{e:3DVIdisc}
%follows with the equivalence to the minimization of 
%\begin{align*}
% E_{\nc,3D}(\vcr):=\int_\Omega W_{3D}(\gradnc\vcr)\,dx - F(\vcr)
% \qquad\text{over }Z_\Cr(\tri).
%\end{align*}
%
%
%The following theorem is the point of departure for the a~priori 
%error analysis from Subsection~\ref{ss:3Dapriori}.
%
%\begin{theorem}[Euler-Lagrange equations for 3D Bingham flow]\label{t:3DEL}
%The solution $u\in H^1_0(\Omega;\R^3)$ to \eqref{e:3DVI} 
%satisfies the Euler-Lagrange equations in the sense that 
%there exist $\sigma\in H(\ddiv,\Omega;\R^{3\times 3})$ 
%and $\xi \in L^2_0(\Omega)$ with 
%\begin{align*}
% \sigma - \xi I_{3\times 3}\in\partial W_{3D}(\nabla u) 
% \qquad\text{and}\qquad 
% f+\ddiv\sigma=0 \text{ a.e. in }\Omega.
%\end{align*}
%The discrete solution $\ucr\in\CR(\tri;\R^3)$ satisfies the 
%discrete Euler-Lagrange equations in the sense that there exist 
%$\sigma_\Cr\in P_0(\tri;\R^{3\times 3})$ and $\xi_0\in P_0(\tri)\cap L^2_0(\Omega)$ 
%with 
%\begin{align*}
% \sigma_\Cr-\xi_0 I_{3\times 3}\in \partial W_{3D}(\gradnc\ucr) 
%\text{ a.e. in }\Omega
%\end{align*}
%and
%\begin{align}\label{e:3DELdisc}
% (\sigma_\Cr,\gradnc \vcr)_{L^2(\Omega)} = F(\vcr)
%        \qquad\text{for all }\vcr\in \CR(\tri;\R^3).
%\end{align}
%\end{theorem}
%
%\begin{proof}
%The proof is analogous to that of \cite[Theorem~6.3]{Glowinski08}
%and is outlined below.
%The regularisation 
%\begin{align*}
% j_\delta(q) = g\int_\Omega \sqrt{\delta^2 + \lvert\mathrm{sym}(q)\rvert^2}\,dx
%\qquad\text{for all }q\in L^2(\Omega;\R^{3\times 3})
%\end{align*}
%of $j$ motivates the minimization of 
%\begin{align*}
% E_{\delta,3D}(v):=(\mu/4)\int_\Omega \lvert\nabla v\rvert^2\,dx
%    + j_\delta(\nabla v) - F(v) 
%\end{align*}
%over $Z$ with unique minimizer $u_\delta$.
%The arguments of \cite[p.83, l.10--21]{Glowinski08} lead to  
%\begin{align*}
% u_\delta\to u
%\text{ strongly in }H^1_0(\Omega;\R^3)\text{ as }\delta\to 0.
%\end{align*}
%
%Since $j_\delta$ is differentiable on $Z$, the solution $u_\delta$
%is characterised by the Euler-Lagrange equations
%\begin{align*}
% (\mu/2) \int_\Omega \nabla u_\delta\cdot\nabla v\,dx
%  + g \int_\Omega (\varepsilon(u_\delta) : \varepsilon(v)
%          /\sqrt{\delta^2+\lvert\varepsilon(u_\delta)\rvert^2})\,dx
%  = F(v)
%\end{align*}
%for all $v\in Z$.
%Define $p_\delta:=\varepsilon(u_\delta)
%/\sqrt{\delta^2+\lvert\varepsilon(u_\delta)\rvert^2}$.
%Then 
%\begin{align*}
% p_\delta\in\Lambda:=\{q\in L^2(\Omega;\mathbb{S})\mid 
%\lvert q(x)\rvert\leq 1\text{ a.e. and }\mathrm{tr}(q)=0\}
%\end{align*}
%and 
%the arguments of \cite[p.84]{Glowinski08} prove 
%the existence of a weak limit $p\in \Lambda$ with
%$p_\delta \rightharpoonup p$ weakly in $L^2(\Omega;\R^{3\times 3})$ as 
%$\delta\to 0$,  
%\begin{align*}
% ( \mu\nabla u +g p, \nabla v )_{L^2(\Omega)} = F(v)
%\qquad\text{for all }v\in Z,
%\end{align*}
%and $\mu\nabla u+g p\in\partial W_{3D}(\nabla u)$.
%In order to involve $\xi\in L^2(\Omega)$ and generalise  
%the equilibrium to all test functions $v\in H^1_0(\Omega;\R^3)$,
%let $\alpha\in H^1_0(\Omega;\R^3)$ and $\xi\in L^2_0(\Omega)$ 
%be the solutions to the Stokes equations 
%\begin{align*}
% (\nabla \alpha,\nabla v)_{L^2(\Omega)} - (\xi,\ddiv v)_{L^2(\Omega)}
% &= F(v)-(\nabla u+ p , \nabla v)_{L^2(\Omega)},\\
% (\ddiv \alpha,\zeta)_{L^2(\Omega)} &= 0
%\end{align*}
%for all $v\in H^1_0(\Omega;\R^3)$ and all $\zeta\in L^2_0(\Omega)$.
%The choice $v=\alpha\in Z$ proves $\alpha=0$. 
%Hence, $\sigma:= \nabla u+ p - \xi I_{3\times 3}\in H(\ddiv,\Omega;\R^{3\times 3})$
%fulfils 
%\begin{align*}
% (\sigma,\nabla v)_{L^2(\Omega)} = F(v)
% \qquad\text{for all }v\in H^1_0(\Omega;\R^3).
%\end{align*}
%
%The discrete Euler-Lagrange equations \eqref{e:3DELdisc} 
%follow with the same arguments. Indeed, since the discrete 
%spaces are finite dimensional, the convergences are even strong.
%The existence of $\xi_h\in P_0(\tri)\cap L^2_0(\Omega)$ 
%follows from the existence of solutions of the discrete 
%Stokes equations \cite{CR73}.
%\end{proof}
%
%\subsection{Three-Field Formulation}
%\label{ss:3D3FF}
%
%As in two dimensions, 
%define
%for $\psi=(v,q)\in L^2(\Omega;\R^3)\times L^2(\Omega;\R^{3\times 3})$
%and $\tau \in H(\ddiv,\Omega;\R^{3\times 3})$
%the bilinear form 
%\begin{align*}
% b(\psi,\tau) = -(v,\ddiv\tau)_{L^2(\Omega)} - (q,\tau)_{L^2(\Omega)}.
%\end{align*}
%The three-field formulation seeks $p=\nabla u$ with values 
%in $\R^{3\times 3}_{\mathrm{dev}}:=\{A\in \R^{3\times 3}\mid 
%\mathrm{tr}(A) =0\}$ and in this way incorporates 
%incompressibility.
%The three-field formulation in 3D seeks 
%$\varphi=(u,p)\in L^2(\Omega;\R^3)\times L^2(\Omega;\R^{3\times 3}_{\mathrm{dev}})$
%and
%$\sigma\in H(\ddiv,\Omega;\R^{3\times 3})$
%%and $\xi\in L^2_0(\Omega)$ 
%with 
%\begin{equation}\label{e:3D3FF}
%\begin{aligned}
% F(v-u) &\leq \mu (p,q-p)_{L^2(\Omega)} + j(q)-j(p)+b(\psi-\varphi,\sigma),
% %  + (\xi,\mathrm{tr}(q-p))_{L^2(\Omega)},
% \\
% b(\varphi,\tau) &= 0
% %(\zeta,\mathrm{tr}(p))_{L^2(\Omega)} &= 0
%\end{aligned}
%\end{equation}
%for all 
%$\psi = (v,q)\in L^2(\Omega;\R^3)\times L^2(\Omega;\R^{3\times 3}_{\mathrm{dev}})$
%and
%$\tau \in H(\ddiv,\Omega;\R^{3\times 3})$.
%%, and $\zeta\in L^2_0(\Omega)$.
%Let $\RT(\tri;\R^{3\times 3})$ denote the space of row-wise Raviart-Thomas 
%functions.
%The discrete three-field formulation in 3D seeks 
%$\varphi_h=(u_h,p_h)\in P_0(\tri;\R^3)\times P_0(\tri;\R^{3\times 3}_{\mathrm{dev}})$
%and 
%$\sigma_h\in \RT (\tri;\R^{3\times 3})$ 
%%and $\xi_h\in P_0(\tri)\cap L^2_0(\Omega)$ 
%with 
%\begin{equation}\label{e:3D3FFd}
%\begin{aligned}
% F(v_h-u_h) &\leq \mu (p_h,q_h-p)_{L^2(\Omega)} + j(q_h)-j(p_h) 
%   +b(\psi_h-\varphi_h,\sigma_h),
%  % + (\xi_h,\mathrm{tr}(q_h-p_h))_{L^2(\Omega)},
%  \\
% b(\varphi_h,\tau_h) &= 0
%% (\zeta_h,\mathrm{tr}(p_h))_{L^2(\Omega)} &= 0
%\end{aligned}
%\end{equation}
%for all 
%$\psi_h = (v_h,q_h)\in P_0(\tri;\R^3)\times P_0(\tri;\R^{3\times 3}_{\mathrm{dev}})$
%and 
%$\tau_h \in \RT (\tri;\R^{3\times 3})$.
%%, and $\zeta_h\in P_0(\tri)\cap L^2_0(\Omega)$.
%
%The equivalence of the $P_1$ nonconforming discretization of the 
%variational inequality and the discretization of the 
%three-field formulation follows with the arguments of 
%Theorem~\ref{t:equivalenced3FFVI} and the observation that 
%$\mathrm{tr}(\gradnc\ucr(x))=\mathrm{tr}(p_h(x))=0$ implies that 
%$\ddiv_\nc\ucr = 0$.
%
%
%\subsection{A Priori Analysis}\label{ss:3Dapriori}
%
%This section generalises Theorems~\ref{t:apriori} and 
%\ref{t:aprioriCR} of Section~\ref{s:apriori}
%to 3D.
%
%
%\begin{theorem}[direct analysis of $P_1$ nonconforming FEM in 3D]
%The solution $u\in Z$ of \eqref{e:3DVI} and the 
%discrete solution $\ucr\in Z_\Cr(\tri)$ of \eqref{e:3DVIdisc} satisfy 
%\begin{align*}
%  \left\|\gradnc(u-\ucr)\right\|_{L^2(\Omega)}
%   \lesssim \left\|\sigma-\Pi_0\sigma\right\|_{L^2(\Omega)}
%      + \osc(f,\tri).
%\end{align*}
%\end{theorem}
%
%\begin{proof}
%The crucial points in the proof are analogous to those of 
%Theorem~\ref{t:aprioriCR}. The outline given here shows how 
%$\xi$ (the Lagrange multiplier for the incompressibility 
%condition from Theorem~\ref{t:3DEL}) comes into play.
%
%Let $\sigma\in H(\ddiv,\Omega;\R^{3\times 3})$ and $\xi\in L^2_0(\Omega)$
%from Theorem~\ref{t:3DEL}.
%The sum rule for subderivatives implies 
%$\partial W_{3D}(\nabla u)=\mu\nabla u +\partial j(\nabla u)$.
%Then $\sigma-\xi I_{3\times 3}\in\partial W_{3D}(\nabla u)$
%implies for all $q\in L^2(\Omega;\R^{3\times 3})$
%\begin{align*}
% (\sigma-\xi I_{3\times 3} - \mu\nabla u, q-\nabla u)_{L^2(\Omega)}
%   \leq j(q) - j(\nabla u).
%\end{align*}
%For $\sigma_\Cr\in P_0(\tri;\R^{3\times 3})$ and 
%$\xi_h\in P_0(\tri)\cap L^2_0(\Omega)$ from Theorem~\ref{t:3DEL},
%the same arguments prove for all $q_h\in P_0(\tri;\R^{3\times 3})$
%\begin{align*}
%  (\sigma_\Cr-\xi_h I_{3\times 3} - \mu\gradnc \ucr, q_h-\gradnc \ucr)_{L^2(\Omega)}
%   \leq j(q_h) - j(\gradnc \ucr).
%\end{align*}
%The choice $q=\gradnc\ucr$ and $q_h=\Pi_0\nabla u$ 
%in the two above displayed inequalities and the sum of those prove 
%\begin{align*}
% \mu &\left\|\gradnc(u-\ucr)\right\|^2
%  \leq j(\Pi_0\nabla u) - j(\nabla u)
%     + \mu(\gradnc \ucr, \Pi_0\nabla u - \nabla u)_{L^2(\Omega)}\\
%  &\qquad\qquad+ (\sigma-\sigma_\Cr,\gradnc(u-\ucr)_{L^2(\Omega)}
%     + (\sigma_\Cr,\nabla u-\Pi_0\nabla u)_{L^2(\Omega)}\\
%  &\qquad\qquad+ (\xi-\xi_\Cr,\ddiv_\nc(u-\ucr))_{L^2(\Omega)} 
%     + (\xi_\Cr,\ddiv u-\Pi_0\ddiv u)_{L^2(\Omega)}.
%\end{align*}
%As in the proof of Theorem~\ref{t:aprioriCR},
%Jensen's inequality \cite{Evans2010} yields $j(\Pi_0\nabla u) - j(\nabla u)\leq 0$ 
%and, since $\gradnc \ucr$, $\sigma_\Cr$, and $\xi_\Cr$ are piecewise 
%constant, the third, fifth and seventh term on the right-hand side 
%vanish. The fourth term is estimated by means of a conforming 
%companion operator analogously to the proof of Theorem~\ref{t:3DEL}.
%Since $u\in Z$ and $\ucr\in Z_\Cr(\tri)$, the remaining term 
%vanishes, namely 
%\begin{equation*}
% (\xi-\xi_\Cr,\ddiv_\nc(u-\ucr))_{L^2(\Omega)}=0.
% \qedhere
%\end{equation*}
%\end{proof}
%
%
%\begin{theorem}\label{t:3Dapriori}
%Any solution $(u,p,\sigma)\in L^2(\Omega;\R^3)\times L^2(\Omega;\R^{3\times 3})
%\times H(\operatorname{div},\Omega;\R^{3\times 3})$
%of \eqref{e:3D3FF} and any
%discrete solution 
%$(u_h,p_h,\sigma_h)\in (P_0(\tri;\R^3)\times P_0(\tri;\R^{3\times 3}))
%\times\RT(\tri;\R^{3\times 3})$
%of \eqref{e:3D3FFd} satisfies
%\begin{align}
%&\left\|\ddiv(\sigma-\sigma_h)\right\|_{L^2(\Omega)}
%    = \left\|f-\Pi_0 f\right\|_{L^2(\Omega)},\label{e:3DdivsigmaEqf}\\
%%
% &\mu\left\|p-\Pi_0 p\right\|_{L^2(\Omega)}^2/2+\vert j(p)-j(\Pi_0 p)\vert
%   \leq \left\|\sigma-\Pi_0\sigma\right\|_{L^2(\Omega)}^2/(2\mu),
%                           \label{e:3DEstpPi0p}
%                           \\
%%
% &\begin{aligned}
%  \mu\left\|p-p_h\right\|_{L^2(\Omega)}^2/2&+\vert j(p)-j(\Pi_0 p)\vert\\
% &\leq \Big(\min_{\tau_h\in Q(f,\tri)}\left\|\sigma-\tau_h\right\|_{L^2(\Omega)}^2 
%  + \left\|h_\tri f\right\|_{L^2(\Omega)}^2/2\Big)/\mu,
%  \end{aligned}
%  \label{e:3DEstpph}
%  \\
%%
%  &\left\|u-u_h\right\|_{L^2(\Omega)}
%  \lesssim \left\|u-\Pi_0 u\right\|_{L^2(\Omega)}
%      + \left\|p-p_h\right\|_{L^2(\Omega)}\label{e:3DEstuuh}.
%\end{align}
%\end{theorem}
%
%\begin{proof}
%The proof of the theorem is analogous to that of 
%Theorem~\ref{t:apriori} and therefore omitted. 
%\end{proof}
%






% \section{Numerical Experiments}\label{s:numexp}
% 
% \begin{figure}
%  \begin{center}
%    \subfloat[\label{f:redrefinement}A red-refinement of a triangle.]{
%      \begin{tikzpicture}[scale=1.5]
% 	\draw (0,0)--(2,0)--(1,1)--cycle;
% 	\draw (1,0)--(1.5,0.5)--(0.5,0.5)--cycle;
%      \end{tikzpicture}
%    }
%    \hfill
%    \subfloat[\label{f:squaredomain}The initial mesh of the 
%       square domain 
%       from Subsection~\ref{ss:Square}.]{
%       \includegraphics[width=0.3\textwidth]{SquareDomain}
%    }
%    \hfill
%    \subfloat[\label{f:Ldomain}The initial mesh of the 
%    L-shaped domain 
%    from Subsection~\ref{ss:Lshaped}.]{
%    \includegraphics[width=0.3\textwidth]{LshapedDomain}
%    }
%  \end{center}
%  \caption{\label{f:squareLdomain}A red-refinement of a triangle and 
%    initial meshs of the domains  
%    from Subsections~\ref{ss:Square}--\ref{ss:Lshaped}.} 
% \end{figure}
% in\Omega$ with $\nabla u_0(x)\neq 0$.
% 
% TODO 
\newpage



%\newpage
%For an initial number of red refinements of 0 the results are
%
%%%%%%%%%%%% INTRODUCTION
%\begin{minipage}{0.49\textwidth}
%\hspace{-130pt}
%\includegraphics[scale=0.55]
%{pictures/Experiment0001/quotientAbsVu/alpha_1_beta_1_initalRed_0/solution_red_1.png}
%\captionof{figure}{Best approximation obtained by this run of the program.}
%\end{minipage}
%\begin{minipage}{0.49\textwidth}
%\includegraphics[scale=0.55]
%{pictures/Experiment0001/quotientAbsVu/alpha_1_beta_1_initalRed_0/energy.png}
%\captionof{figure}{Behaviour of the energy in each iteration.}
%\end{minipage}\\
%
%
%%%%%%%%%%%%% CONTENT
%\foreach \red in {0,...,1}{
%\newpage
%\begin{minipage}{0.49\textwidth}
%\hspace{-130pt}
%\includegraphics[scale=0.55]
%{pictures/Experiment0001/quotientAbsVu/alpha_1_beta_1_initalRed_0/corr_red_\red.png}
%\end{minipage}
%\begin{minipage}{0.49\textwidth}
%\hspace{-10pt}
%\includegraphics[scale=0.55]
%{pictures/Experiment0001/quotientAbsVu/alpha_1_beta_1_initalRed_0/corr_red_\red_loglog.png}
%\end{minipage}\\
%
%\begin{minipage}{0.49\textwidth}
%\hspace{-130pt}
%\includegraphics[scale=0.55]
%{pictures/Experiment0001/quotientAbsVu/alpha_1_beta_1_initalRed_0/solution_red_\red.png}
%\end{minipage}
%\begin{minipage}{0.49\textwidth}
%\hspace{-10pt}
%\includegraphics[scale=0.55]
%{pictures/Experiment0001/quotientAbsVu/alpha_1_beta_1_initalRed_0/solution_red_\red_axis.png}
%\end{minipage}
%\\
%}
%\newpage
%
%An initial number of red refinements of 4 yields
%
%%%%%%%%%%%% INTRODUCTION
%\begin{minipage}{0.49\textwidth}
%\hspace{-130pt}
%\includegraphics[scale=0.55]
%{pictures/Experiment0001/quotientAbsVu/alpha_1_beta_1_initalRed_4/solution_red_6.png}
%\captionof{figure}{Best approximation obtained by this run of the program.}
%\end{minipage}
%\begin{minipage}{0.49\textwidth}
%\includegraphics[scale=0.55]
%{pictures/Experiment0001/quotientAbsVu/alpha_1_beta_1_initalRed_4/energy.png}
%\captionof{figure}{Behaviour of the energy in each iteration.}
%\end{minipage}\\
%
%
%%%%%%%%%%%%% CONTENT
%\foreach \red in {4,...,6}{
%\newpage
%\begin{minipage}{0.49\textwidth}
%\hspace{-130pt}
%\includegraphics[scale=0.55]
%{pictures/Experiment0001/quotientAbsVu/alpha_1_beta_1_initalRed_4/corr_red_\red.png}
%\end{minipage}
%\begin{minipage}{0.49\textwidth}
%\hspace{-10pt}
%\includegraphics[scale=0.55]
%{pictures/Experiment0001/quotientAbsVu/alpha_1_beta_1_initalRed_4/corr_red_\red_loglog.png}
%\end{minipage}\\
%
%\begin{minipage}{0.49\textwidth}
%\hspace{-130pt}
%\includegraphics[scale=0.55]
%{pictures/Experiment0001/quotientAbsVu/alpha_1_beta_1_initalRed_4/solution_red_\red.png}
%\end{minipage}
%\begin{minipage}{0.49\textwidth}
%\hspace{-10pt}
%\includegraphics[scale=0.55]
%{pictures/Experiment0001/quotientAbsVu/alpha_1_beta_1_initalRed_4/solution_red_\red_axis.png}
%\end{minipage}
%\\
%}
%\newpage
%
%Choosing $\texttt{corr}\coloneqq |E(u_j)-E(u_{j-1})|$ an initial number of 
%red refinements results in
%
%
%
%%%%%%%%%%%% INTRODUCTION
%\begin{minipage}{0.49\textwidth}
%\hspace{-130pt}
%\includegraphics[scale=0.55]
%{pictures/Experiment0001/energyDiff/alpha_1_beta_1_initalRed_0/solution_red_6.png}
%\captionof{figure}{Best approximation obtained by this run of the program.}
%\end{minipage}
%\begin{minipage}{0.49\textwidth}
%\includegraphics[scale=0.55]
%{pictures/Experiment0001/energyDiff/alpha_1_beta_1_initalRed_0/energy.png}
%\captionof{figure}{Behaviour of the energy in each iteration.}
%\end{minipage}\\
%
%
%%%%%%%%%%%%% CONTENT
%\foreach \red in {0,...,6}{
%\newpage
%\begin{minipage}{0.49\textwidth}
%\hspace{-130pt}
%\includegraphics[scale=0.55]
%{pictures/Experiment0001/energyDiff/alpha_1_beta_1_initalRed_0/corr_red_\red.png}
%\end{minipage}
%\begin{minipage}{0.49\textwidth}
%\hspace{-10pt}
%\includegraphics[scale=0.55]
%{pictures/Experiment0001/energyDiff/alpha_1_beta_1_initalRed_0/corr_red_\red_loglog.png}
%\end{minipage}\\
%
%\begin{minipage}{0.49\textwidth}
%\hspace{-130pt}
%\includegraphics[scale=0.55]
%{pictures/Experiment0001/energyDiff/alpha_1_beta_1_initalRed_0/solution_red_\red.png}
%\end{minipage}
%\begin{minipage}{0.49\textwidth}
%\hspace{-10pt}
%\includegraphics[scale=0.55]
%{pictures/Experiment0001/energyDiff/alpha_1_beta_1_initalRed_0/solution_red_\red_axis.png}
%\end{minipage}
%\\
%}
%
%
%\newpage
%
%An initial number of red refinements of 3 yields
%
%
%
%%%%%%%%%%%% INTRODUCTION
%\begin{minipage}{0.49\textwidth}
%\hspace{-130pt}
%\includegraphics[scale=0.55]
%{pictures/Experiment0001/energyDiff/alpha_1_beta_1_initalRed_3/solution_red_6.png}
%\captionof{figure}{Best approximation obtained by this run of the program.}
%\end{minipage}
%\begin{minipage}{0.49\textwidth}
%\includegraphics[scale=0.55]
%{pictures/Experiment0001/energyDiff/alpha_1_beta_1_initalRed_3/energy.png}
%\captionof{figure}{Behaviour of the energy in each iteration.}
%\end{minipage}\\
%
%
%%%%%%%%%%%%% CONTENT
%\foreach \red in {3,...,6}{
%\newpage
%\begin{minipage}{0.49\textwidth}
%\hspace{-130pt}
%\includegraphics[scale=0.55]
%{pictures/Experiment0001/energyDiff/alpha_1_beta_1_initalRed_3/corr_red_\red.png}
%\end{minipage}
%\begin{minipage}{0.49\textwidth}
%\hspace{-10pt}
%\includegraphics[scale=0.55]
%{pictures/Experiment0001/energyDiff/alpha_1_beta_1_initalRed_3/corr_red_\red_loglog.png}
%\end{minipage}\\
%
%\begin{minipage}{0.49\textwidth}
%\hspace{-130pt}
%\includegraphics[scale=0.55]
%{pictures/Experiment0001/energyDiff/alpha_1_beta_1_initalRed_3/solution_red_\red.png}
%\end{minipage}
%\begin{minipage}{0.49\textwidth}
%\hspace{-10pt}
%\includegraphics[scale=0.55]
%{pictures/Experiment0001/energyDiff/alpha_1_beta_1_initalRed_3/solution_red_\red_axis.png}
%\end{minipage}
%\\
%}
%
%
%\newpage
%
%An initial number of red refinements of 4 yields
%
%%%%%%%%%%%% INTRODUCTION
%\begin{minipage}{0.49\textwidth}
%\hspace{-130pt}
%\includegraphics[scale=0.55]
%{pictures/Experiment0001/energyDiff/alpha_1_beta_1_initalRed_4/solution_red_5.png}
%\captionof{figure}{Best approximation obtained by this run of the program.}
%\end{minipage}
%\begin{minipage}{0.49\textwidth}
%\includegraphics[scale=0.55]
%{pictures/Experiment0001/energyDiff/alpha_1_beta_1_initalRed_4/energy.png}
%\captionof{figure}{Behaviour of the energy in each iteration.}
%\end{minipage}\\
%
%
%%%%%%%%%%%%% CONTENT
%\foreach \red in {4,...,5}{
%\newpage
%\begin{minipage}{0.49\textwidth}
%\hspace{-130pt}
%\includegraphics[scale=0.55]
%{pictures/Experiment0001/energyDiff/alpha_1_beta_1_initalRed_4/corr_red_\red.png}
%\end{minipage}
%\begin{minipage}{0.49\textwidth}
%\hspace{-10pt}
%\includegraphics[scale=0.55]
%{pictures/Experiment0001/energyDiff/alpha_1_beta_1_initalRed_4/corr_red_\red_loglog.png}
%\end{minipage}\\
%
%\begin{minipage}{0.49\textwidth}
%\hspace{-130pt}
%\includegraphics[scale=0.55]
%{pictures/Experiment0001/energyDiff/alpha_1_beta_1_initalRed_4/solution_red_\red.png}
%\end{minipage}
%\begin{minipage}{0.49\textwidth}
%\hspace{-10pt}
%\includegraphics[scale=0.55]
%{pictures/Experiment0001/energyDiff/alpha_1_beta_1_initalRed_4/solution_red_\red_axis.png}
%\end{minipage}
%\\
%}
%
