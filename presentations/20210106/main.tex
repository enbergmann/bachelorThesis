\documentclass[xcolor=svgnames,english]{beamer}
\usetheme{DarkHU}

%\usepackage{pgfpages}
%\pgfpagesuselayout{2 on 1}[a4paper,border shrink=5mm]

% ============================================================
% === USER-DEFINED COMMANDS ==================================
% ============================================================

\usepackage{amssymb,mathabx,textcomp,bm}

\newcommand{\mathbox}[2]{\makebox[#1]{$\displaystyle #2$}}
\newcommand{\Stackrel}[2]{\mathbox{1.25em}{\stackrel{#1}{#2}}}


% === INTEGRAL MEAN ==========================================

\def\Xint#1{\mathchoice
{\XXint\displaystyle\textstyle{#1}}%
{\XXint\textstyle\scriptstyle{#1}}%
{\XXint\scriptstyle\scriptscriptstyle{#1}}%
{\XXint\scriptscriptstyle\scriptscriptstyle{#1}}%
\!\int}
\def\XXint#1#2#3{{\setbox0=\hbox{$#1{#2#3}{\int}$}
\vcenter{\hbox{$#2#3$}}\kern-.5\wd0}}
\newcommand{\intmean}{\Xint-}

% === PERFECT BULLET =========================================

\let\oldbullet\bullet
\newlength{\raisebulletlen}
\setbox1=\hbox{$\bullet$}\setbox2=\hbox{\tiny$\bullet$}
\setlength{\raisebulletlen}{\dimexpr0.5\ht1-0.5\ht2}
\renewcommand\bullet{\raisebox{\raisebulletlen}{\,\tiny$\oldbullet$}\,}

% === ORTHOGONAL SUM =========================================

\newcommand\orthsum{
\tikz[baseline=(A.base),font=\small]
     \node[draw,ellipse,inner sep=0.15ex] (A){$\perp$};
}

% === CALIGRAPHIC LETTERS ====================================

\newcommand\Acal{\mathcal{A}}
\newcommand\Bcal{\mathcal{B}}
\newcommand\Ccal{\mathcal{C}}
\newcommand\Dcal{\mathcal{D}}
\newcommand\Ecal{\mathcal{E}}
\newcommand\Fcal{\mathcal{F}}
\newcommand\Jcal{\mathcal{J}}
\newcommand\Kcal{\mathcal{K}}
\newcommand\Lcal{\mathcal{L}}
\newcommand\Mcal{\mathcal{M}}
\newcommand\Ncal{\mathcal{N}}
\newcommand\Ocal{\mathcal{O}}
\newcommand\Pcal{\mathcal{P}}
\newcommand\Rcal{\mathcal{R}}
\newcommand\Tcal{\mathcal{T}}

% === MATHBB LETTERS =========================================

\newcommand\C{\mathbb{C}}
\newcommand\N{\mathbb{N}}
\newcommand\R{\mathbb{R}}
\newcommand\T{\mathbb{T}}
\newcommand\K{\mathbb{K}}

% === MATH OPERATORS =========================================

\DeclareMathOperator*{\argmin}{argmin} % the * adjusts MathOperator for indizes beneath
\DeclareMathOperator{\bisec}{bisec}
\DeclareMathOperator{\cond}{cond}
\DeclareMathOperator{\Conv}{conv}
\DeclareMathOperator{\Curl}{Curl}
\DeclareMathOperator{\curl}{curl}
\DeclareMathOperator{\dev}{dev}
\DeclareMathOperator{\diag}{diag}
\DeclareMathOperator{\diam}{diam}
\DeclareMathOperator{\Dim}{dim}
\DeclareMathOperator{\Div}{div}
\DeclareMathOperator{\Dist}{dist}
\DeclareMathOperator{\esssup}{ess\ supp}
\DeclareMathOperator{\grad}{\nabla}
\DeclareMathOperator{\Int}{int}
\DeclareMathOperator{\Ker}{ker}
\DeclareMathOperator{\Mid}{mid}
\DeclareMathOperator{\Osc}{osc}
\DeclareMathOperator{\Red}{red}
\DeclareMathOperator{\Ref}{Ref}
\DeclareMathOperator{\Res}{Res}
\DeclareMathOperator{\sign}{sgn}
\DeclareMathOperator{\Span}{span}
\DeclareMathOperator{\supp}{supp}
\DeclareMathOperator{\tr}{tr}
\DeclareMathOperator{\Width}{width}
\DeclareMathOperator*{\arginf}{arginf}

% === COMMANDS WITH INPUT ARGUMENTS ==========================

\newcommand\abs[1]{\lvert #1 \rvert}
\newcommand\average[1]{\langle #1 \rangle}
\newcommand\jump[1]{\lbrack #1 \rbrack}
\newcommand\NormEnergy[1]{\big\vvvert #1 \big\vvvert}
\newcommand\Norm[2]{\lVert #1 \rVert_{#2}}
\newcommand\scal[2]{\left\langle #1 , #2 \right\rangle}

% === SPACES WITH NAMES ======================================

\newcommand{\CONF}{\textup{C}}
\newcommand{\NC}{\textup{NC}}
\newcommand{\CR}{\textup{CR}}
\newcommand{\RT}{\textup{RT}}

% === GENERAL STUFF ==========================================

\newcommand{\splitter}{\,:\,} % split for definition of sets
\renewcommand{\d}{\, \textup{d}} % d for differentials, i.e., \d x


% === JUST TO COMPILE CHAPTERS 4  AND 5 ==========================================

\newcommand{\LO}[0]{L^2(\Omega)}
\newcommand{\restrict}[2]{\left. #1 \right\lvert_{#2}}
\DeclareMathOperator{\mPkt}{mid}
\newcommand{\nnorm}[2]{\left\lVert #1 \right\rVert_{#2}}
\newcommand{\volf}[2]{\nnorm{h_{#1}f}{#2}}
\newcommand{\volO}[1]{\nnorm{h_{#1}f}{\LO}}
\DeclareMathOperator\Kern{ker}
\DeclareMathOperator\setspan{span}

% === JUST TO COMPILE CHAPTER 6 ==========================================


\newcommand\NormL[3]{\Norm{#1}{L^{#2}\left( { #3 } \right)}}
 \newcommand\NormH[3]{\Norm{#1}{H^{#2}\left( { #3 } \right)}}
 \newcommand\NormHdiv[1]{\Norm{#1}{H(\div)}}
 \newcommand\NormW[3]{\Norm{#1}{W^{#2}\left( { #3 } \right)}}
\newcommand\NormLDom[2]{\NormL{#1}{#2}{\Omega}}
 \newcommand\NormHDom[2]{\NormH{#1}{#2}{\Omega}}
 \newcommand\NormHdivDom[1]{\NormHdiv{#1}}
 \newcommand\NormWDom[2]{\NormW{#1}{#2}{\Omega}}
 \newcommand\NormLz[2]{\NormL{#1}{2}{#2}}
 \newcommand\NormHz[2]{\NormH{#1}{2}{#2}}
 \newcommand\NormWkp[4]{\NormW{#1}{{#2,#3}}{#4}}
 \newcommand\NormWkpDom[3]{\NormWDom{#1}{{#2,#3}}}
 \newcommand\NormLzDom[1]{\NormLDom{#1}{2}}
 \newcommand\NormHzDom[1]{\NormHDom{#1}{2}}
\newcommand\NormMax[2]{\Norm{#1}{\C({#2})}}


% === JUST TO COMPILE CHAPTER 0 ==========================================

\newcommand{\dist}{\mathrm{dist}}
\newcommand{\fa}{\;\forall\;}
\newcommand{\Lra}{\Leftrightarrow}
\newcommand{\id}{\mathrm{id}}

% === JUST TO COMPILE EXERCISES ==========================================

\newcommand{\al}{\alpha}
\newcommand{\be}{\beta}
\newcommand{\ga}{\gamma}
\newcommand{\de}{\delta}
\newcommand{\eps}{\varepsilon}
\newcommand{\vphi}{\varphi}
\newcommand{\la}{\lambda}
\newcommand{\om}{\omega}
 

\usepackage[backend=biber, style = alphabetic]{biblatex}
\usepackage{csquotes} % to fix biblatex warning expecting csquotes
\addbibresource{sourcesBergmannBT.bib}


\usepackage[T1]{fontenc}
\usepackage{lmodern}

%\usepackage{eurosym}

%\usepackage[backend=biber, style=authortitle-icomp]{biblatex}
%\usepackage[babel,english=guillemets]{csquotes}
%\addbibresource{bib.bib}

%\usepackage{algorithm}
%\usepackage{algpseudocode}

\usepackage{tikz}
%\usetikzlibrary{shapes, fit, positioning}
%\usepackage{hyperref}

%%%%%%%%%%%%%%%%%%%%%%%
% UTF-8 input encoding
\usepackage[utf8]{inputenc}
\usepackage{microtype}\usepackage{blindtext}

% Page layout formating of koma-script
\usepackage[automark]{scrpage2}
\usepackage{lmodern}

% Language support
\usepackage[english]{babel}

% AMSMath package
\usepackage{amsmath}
\usepackage{amsfonts}
\usepackage{amssymb}
\usepackage{amstext}
\usepackage{amsopn}
\usepackage{amsthm}
\usepackage{url}
\usepackage{listings}
\usepackage{xspace}
\usepackage{mathabx}
%\usepackage{mathptmx}
\usepackage{esint}
\usepackage{graphicx}
\usepackage{mathtools}
\usepackage{hyperref}

%%%%%%%%%%%%%%%%%%%%%%%%%
\usepackage{enumitem,amssymb}
\newlist{todolist}{itemize}{2}
\setlist[todolist]{label=$\square$}
\usepackage{pifont}
\newcommand{\cmark}{\ding{51}}%
\newcommand{\xmark}{\ding{55}}%
\newcommand{\done}{\rlap{$\square$}{\raisebox{2pt}{\large\hspace{1pt}\cmark}}%
\hspace{-2.5pt}}
\newcommand{\wontfix}{\rlap{$\square$}{\large\hspace{1pt}\xmark}}
%%%%%%%%%%%%%%%%%%%%%%%%%%%


\parindent 0ex


\providecommand{\integral}[3]{\ensuremath{\int\limits_{#1} \! {#2} \, \mathrm{d}#3}}
\providecommand{\T}{\ensuremath{\mathcal{T}}}
\providecommand{\Hh}{\ensuremath{\mathrm{H}}}
\providecommand{\Ll}{\ensuremath{\mathrm{L}}}
 
%%%%%%%%%%
\setbeamercolor{green box}{use=base, fg=base.fg, bg=green}
\newenvironment{question}[1][center]{\begin{beamercolorbox}[#1, rounded=true,
  sep=1pt]{green box}}{\end{beamercolorbox}}

\setbeamercolor{yellow box}{use=base, fg=base.fg, bg=yellow}
\newenvironment{emphbox}[1][center]{\begin{beamercolorbox}[#1, rounded=true,
  sep=1pt]{yellow box}}{\end{beamercolorbox}}

\renewcommand{\emph}[1]{\textcolor{magenta}{\bfseries #1}}
\providecommand{\abs}[1]{\ensuremath{\left\lvert#1\right\rvert}}

%\setbeamercovered{transparent}
%\setbeamertemplate{navigation symbols}{}

%\expandafter\def\expandafter\insertshorttitle\expandafter{%
%  \insertshorttitle\hfill%
%    \insertframenumber\,/\,\inserttotalframenumber}

%\addtobeamertemplate{navigation symbols}{}{%
%  \usebeamerfont{footline}%
%  \usebeamercolor[fg]{footline}%
%  \hspace{1em}%
%  \insertframenumber/\inserttotalframenumber}

%\expandafter\def\expandafter\insertshorttitle\expandafter{%
%  \insertshorttitle\hfill%
%    \insertframenumber\,/\,\inserttotalframenumber}

%%%%%%%%%%%%%%%%%%%%%%%%%%%%%%%%%%%%%%%%%%%%%%%%%%%%%%%%%%%%
\author{Enrico Bergmann}
\title{Minimization of a Functional on the Space of BV Functions and 
Nonconforming Discretization of the Problem}
\subtitle{I. Theoretical Basics and Characterization of Minimizers}
\institute{Humboldt-Universität zu Berlin}
\date{January 6, 2021}
\titlegraphic{\includegraphics[height=1.6cm]{hukombi_bbw_rgb_op}}
%%%%%%%%%%%%%%%%%%%%%%%%%%%%%%%%%%%%%%%%%%%%%%%%%%%%%%%%%%%%

\usepackage{microtype}

\begin{document}

\begin{frame}[plain, noframenumbering]
	\maketitle
\end{frame}
  
\begin{frame}{Table of Contents}
  \tableofcontents
\end{frame}

%%%%%%%%%%%%%%%%%%%%%%%%%%%%%%%%%%%%%%%%%%%%%%%%%%%%%%%%%%%%%%%%%%%%%%%%%%%%%%

\section{Introduction}

\begin{frame}[noframenumbering]{Table of Contents}
  \tableofcontents[currentsection, currentsubsection]
\end{frame}

\begin{frame}
  \fullcite[Chapter 10, p. 297-319]{Bar15}

  \bigskip
  \pause

  Let $\Omega\subset\Rbb^n$ be a bounded polyhedral Lipschitz domain.

  \medskip

  For given $g\in L^2(\Omega)$ and $\alpha>0$ minimize the functional 
  \begin{align*}
    I(v)=|v|_{\BV(\Omega)}+\frac{\alpha}{2}\Vert v-g\Vert^2
  \end{align*}
  amongst all $v\in \BV(\Omega)\cap L^2(\Omega)$.
\end{frame}

\begin{frame}{Functions of Bounded Variation}
  A function $v\in L^1(\Omega)$ with distributional derivative
  $Dv:C^{\infty}_C(\Omega;\Rbb^n)\to\Rbb$ is said to be of bounded variation if 
  there exists $c>0$ such that 
  \begin{align*}
    \langle Dv, \phi\rangle\coloneqq -\int_\Omega v\Div(\phi)\dx\leq
    c\Vert\phi\Vert_{L^\infty(\Omega)}
  \end{align*}
  for all $\phi\in C^1_C(\Omega;\Rbb^n)$.

  \pause  

  The minimal constant $c\geq 0$ satisfying this property is called 
  total variation of $Dv$ and is given by
  \begin{align*}
    |v|_{\BV(\Omega)} = \sup_{\substack{\phi\in C^1_C(\Omega;\Rbb^n)\\
    \Vert\phi\Vert_{L^\infty(\Omega)}\leq 1}}-\int_\Omega v\Div (\phi)\dx.
  \end{align*}

  \pause

  The space of all such functions is denoted by $\BV(\Omega)$.
\end{frame}

\begin{frame}{Properties of $\BV(\Omega)$}
  $\BV(\Omega)$ is a nonseparable Banach space equipped with the norm
  \begin{align*}
    \Vert v \Vert_{\BV(\Omega)} \coloneqq \Vert v\Vert_{L^1(\Omega)} +
    |v|_{\BV(\Omega)}\quad\text{for all } v\in\BV(\Omega).
  \end{align*}
  
  \pause
  \medskip
  $W^{1,1}(\Omega)\subset\BV(\Omega)$ with $\Vert v\Vert_{\BV(\Omega)}=
  \Vert v\Vert_{W^{1,1}(\Omega)}$ for all $v\in W^{1,1}(\Omega)$.
\end{frame}

\begin{frame}{Notions of convergence on $\BV(\Omega)$}
  Let $(v_n)_{n\in\Nbb}\subset \BV(\Omega)$ and $v\in \BV(\Omega)$ such that
  $v_n\rightarrow v$ in $L^1(\Omega)$ as $n\rightarrow\infty$.
  \begin{itemize}
    \item[(i)]
      $(v_n)_{n\in\Nbb}$ converges intermediately or strictly to $v$
      if $|v_n|_{\BV(\Omega)}\rightarrow |v|_{\BV(\Omega)}$ as
      $n\rightarrow\infty$.
      \pause
    \item[(ii)] $(v_n)_{n\in\Nbb}$ converges weakly to
      $v$ if
      $\langle Dv_n,\phi\rangle\rightarrow \langle Dv,\phi\rangle$ 
      for all $\phi\in C_0(\Omega;\Rbb^n)$ as 
      $n\rightarrow\infty$.
  \end{itemize}
\end{frame}

\begin{frame}{Further Properties of $\BV(\Omega)$}
  $C^\infty(\overline\Omega)$ and $C^\infty(\Omega)\cap\BV(\Omega)$ are dense
  in $\BV(\Omega)$ with respect to intermediate convergence.
  
  \pause
  \bigskip

  The embedding $\BV(\Omega)\to L^p(\Omega)$ is continuous for
  $1\leq p\leq n/(n-1)$ and compact for $1\leq p< n/(n-1)$
  
  \pause
  \bigskip

  There exists a linear operator $T:\BV(\Omega)\to L^1(\partial\Omega)$
  such that $T(v) = v|_{\partial\Omega}$ for all $v\in\BV(\Omega)\cap
  C(\overline\Omega)$.

  $T$ is continuous with respect to intermediate convergence in $\BV(\Omega)$
  but not with respect to weak convergence in $\BV(\Omega)$. 
\end{frame}

%%%%%%%%%%%%%%%%%%%%%%%%%%%%%%%%%%%%%%%%%%%%%%%%%%%%%%%%%%%%%%%%%%%%%%%%%%%%%%%

\section{Continuous Problem}
\begin{frame}{Table of Contents}
  \tableofcontents[currentsection, currentsubsection]
\end{frame}

\begin{frame}
  For given $f\in L^2(\Omega)$ and $\alpha>0$ minimize the functional 
  \begin{align*}\label{eq:continuousProblem}
    E(v)\coloneqq \frac{\alpha}{2}\Vert v\Vert_{L^2(\Omega)}^2 + |v|_{\BV(\Omega)}
    +\Vert v\Vert_{L^1(\partial\Omega)}-\int_\Omega f\,v\dx
  \end{align*}
  amongst all $v\in \BV(\Omega)\cap L^2(\Omega)$.

  \pause
  \bigskip
  

  For $f = \alpha g$ we have
  \begin{align*}
    I(v) 
    = |v|_{\BV(\Omega)}+\frac{\alpha}{2}\Vert v-g\Vert^2
    = E(v) - \Vert v\Vert_{L^1(\partial \Omega)}+ \frac{\alpha}{2}\Vert
    g\Vert_{L^2(\Omega)}^2
  \end{align*}
  for all $v\in \BV(\Omega)\cap L^2(\Omega)$.
  
  \pause
  \medskip

  $I$ and $E$ have the same minimizers in $\left\{v\in\BV(\Omega)\cap
  L^2(\Omega)\mid \Vert v\Vert_{L^1(\partial\Omega)}=0\right\}$.
\end{frame}

\subsection{Existence of Minimizers}
\begin{frame}{Table of Contents}
  \tableofcontents[currentsection, currentsubsection]
\end{frame}

\begin{frame}
  \begin{align*}
    E(v)
    \only<1->{&=
    \frac{\alpha}{2}\Vert v\Vert_{L^2(\Omega)}^2 + |v|_{\BV(\Omega)}
    +\Vert v\Vert_{L^1(\partial\Omega)}-\int_\Omega fv\dx}\\
    \only<2->{&\geq 
    \frac{\alpha}{2}\Vert v\Vert_{L^2(\Omega)}^2 + |v|_{\BV(\Omega)}
    +\Vert v\Vert_{L^1(\partial\Omega)}
    -\Vert f\Vert_{L^2(\Omega)}\Vert v\Vert_{L^2(\Omega)}}\\
    \only<3->{&\geq 
    \frac{\alpha}{2}\Vert v\Vert_{L^2(\Omega)}^2 + |v|_{\BV(\Omega)}
    +\Vert v\Vert_{L^1(\partial\Omega)}
    -\frac{1}{\alpha}\Vert f\Vert_{L^2(\Omega)}^2
    -\frac{\alpha}{4}\Vert v\Vert_{L^2(\Omega)}^2}\\
    \only<4->{&\geq 
    \frac{\alpha}{4}\Vert v\Vert_{L^2(\Omega)}^2 + |v|_{\BV(\Omega)}
    +\Vert v\Vert_{L^1(\partial\Omega)}-\frac{1}{\alpha}\Vert
    f\Vert_{L^2(\Omega)}^2}\\
    \only<5->{&\geq 
    \frac{\alpha}{4|\Omega|}\Vert v\Vert_{L^1(\Omega)}^2 + |v|_{\BV(\Omega)}
    +\Vert v\Vert_{L^1(\partial\Omega)}-\frac{1}{\alpha}\Vert
    f\Vert_{L^2(\Omega)}^2}\\
    \only<6->{&\geq 
    -\frac{1}{\alpha}\Vert f\Vert_{L^2(\Omega)}^2.}
  \end{align*}
\end{frame}

\begin{frame}
  \begin{align*}
    E(v)
    &\geq 
    \frac{\alpha}{4}\Vert v\Vert_{L^2(\Omega)}^2 + |v|_{\BV(\Omega)}
    +\Vert v\Vert_{L^1(\partial\Omega)}-\frac{1}{\alpha}\Vert
    f\Vert_{L^2(\Omega)}^2\\
    &\geq 
    \frac{\alpha}{4|\Omega|}\Vert v\Vert_{L^1(\Omega)}^2 + |v|_{\BV(\Omega)}
    +\Vert v\Vert_{L^1(\partial\Omega)}-\frac{1}{\alpha}\Vert
    f\Vert_{L^2(\Omega)}^2\\
    &\geq -\frac{1}{\alpha}\Vert f\Vert_{L^2(\Omega)}^2.
  \end{align*}

  \pause

  \begin{itemize}[label=$\bullet$]
    \item\alt<1>{$E$ bounded from below}{$\exists
      (u_n)_{n\in\Nbb}\subset\BV(\Omega)\cap L^2(\Omega)$ infimizing sequence
      of $E$}
    \item $\Vert u_n\Vert_{\BV(\Omega)}\to\infty$ as $n\to\infty$
      $\Rightarrow$ $E(u_n)\to\infty$ as $n\to\infty$
    \item $(u_n)_{n\in\Nbb}$ bounded
  \end{itemize}

\end{frame}

\begin{frame}
  
\end{frame}

\begin{frame}
  \fullcite[p. 183, Theorem 1]{EG92}

  \bigbreak

  Let $v\in\BV(\Omega)$.
  For all $x\in\Rbb^n$ define
  \begin{align*}
    \tilde{v}(x)\coloneqq
    \begin{cases}
      v(x)  &\text{ if } x\in\Omega,\\
      0     &\text{ if } x\in\Rbb^n\setminus\Omega.
    \end{cases} 
  \end{align*}
  Then $\tilde{v}\in\BV\left(\Rbb^n\right)$ and
  $|\tilde{v}|_{\BV\left(\Rbb^n\right)}
  = |v|_{\BV(\Omega)}+\Vert v\Vert_{L^1(\partial\Omega)}$.
\end{frame}

\subsection{Uniqueness and Stability}
\begin{frame}{Table of Contents}
  \tableofcontents[currentsection, currentsubsection]
\end{frame}


\section{Discrete Problem}
\begin{frame}{Table of Contents}
  \tableofcontents[currentsection, currentsubsection]
\end{frame}
%TODO maybe show refinement indicator to show that the we refine where the 
% jumps are large

% TODO include Prop 10.1 assumption
\subsection{Equivalent Saddle Point Problem}
\begin{frame}{Table of Contents}
  \tableofcontents[currentsection, currentsubsection]
\end{frame}
\subsection{Characterization of Minimizers}
\begin{frame}{Table of Contents}
  \tableofcontents[currentsection, currentsubsection]
\end{frame}

\section{Outlook}
\begin{frame}{Table of Contents}
  \tableofcontents[currentsection, currentsubsection]
\end{frame}

%\begin{frame}{Checkliste}
%\begin{itemize}
%  \item
%  \begin{todolist}
%  \item[\done] $P_1$ implementieren.
%  \item[\done] $P_2$ implementieren.
%\pause  
%  \item[\done] $P_3$ implementieren.
%  \item[\done] Implementation des eingebauten Fehlerschätzers.
%  \pause
%  \item Nachstellen der Experimente aus dem Paper.
%  \pause
%  \item[\wontfix] Optimierung des Codes.
%  \item[\wontfix] Durchführung weiterer Experimente.
%  \item[\wontfix] Auswertung und Dokumentation.
%  
%  \end{todolist}
%\end{itemize}
%\end{frame}		

\end{document}
