\subsection{Primal-Dual Iteration}
\begin{frame}[noframenumbering]{Table of Contents}
  \tableofcontents[currentsection, currentsubsection]
\end{frame}

\begin{frame}
  \begin{algorithm}[Primal-dual iteration]
  \begin{algorithmic}
    \Require $\left(u_0,\Lambda_0\right)
    \in\textup{CR}_0^1(\mathcal{T})\times P_0\!\left(\mathcal{T}; 
    %\left\{w\in\Rbb^2\,\middle|\,|w|\leq 1\right\}\right),
    \overline{B_{\Rbb^2}}\right)$, 
    %\overline{B_1(0)}\right),
    \pause$\tau>0$, \only<12>{{\color{red}$\varepsilon_\textup{stop}>0$}}  \\
    \pause Initialize $v_0\coloneqq 0$ in $\textup{CR}^1_0(\mathcal T)$.
    \pause\For{$j = 1,2,\dots$}
    \begin{align*}
      \tilde{u}_j&\coloneqq u_{j-1}+\tau v_{j-1},
      &&&\Lambda_j
      &\coloneqq
      \frac{\Lambda_{j-1}+\tau\nabla_{\textup{NC}} \tilde{u}_j}
      {\max\left\{1,
      \left|\Lambda_{j-1}+\tau\nabla_{\textup{NC}}\tilde{u}_j\right|\right\}},
    \end{align*}
    \pause\State solve  % TODO habe Antwort parat für ,,Was ist diese Gleichung
    \begin{align*}
      \label{eq:linSysPrimalDualAlg}
      \frac{1}{\tau}{\color<8,10-11>{red}{a_{\textup{NC}}(u_j,\bullet)}}
      +\alpha{\color<9>{red}{(u_j,\bullet)}}
      &=
      \frac{1}{\tau}{\color<10-11>{red}{a_{\textup{NC}}(u_{j-1},\bullet)}} 
      + (f,\bullet)
      - \left(\Lambda_j,\nabla_{\textup{NC}}\bullet\right) 
    \end{align*}
    \State in $\CR^1_0(\Tcal)$ for $u_j$, \pause and set
    \begin{equation*}
      {\color<11>{green}{v_j\coloneqq\frac{u_j-u_{j-1}}{\tau}}}.
      {\only<12>{\color{red}\text{ Terminate iteration if }\vvvert v_j\vvvert<
      \varepsilon_\textup{stop}.}}
    \end{equation*}
    \EndFor
    \pause\Ensure Folge $(u_j,\Lambda_j)_{j\in\mathbb N}$ in
    $\CR^1_0(\mathcal{T})\times
     P_0\!\left(\mathcal{T};\overline{B_{\Rbb^2}}\right)$   
    \end{algorithmic}
  \end{algorithm}
\end{frame}

\begin{frame}
  \begin{theorem}[Convergence of the primal-dual iteration]
    Let $\ucr\in \CR^1_0(\Tcal)$ solve the discrete problem,
    $\bar\Lambda_0\in P_0\!\left(\Tcal;\Rbb^2\right)$ satisfy
    $\left|\bar\Lambda_0(\bullet)\right|\leq 1$ a.e.\ in $\Omega$ as well as
    \begin{equation*}
      \bar\Lambda_0(\bullet)\cdot\gradnc\ucr(\bullet)
      =
      \left|\gradnc\ucr(\bullet)\right| 
      \quad\text{a.e.\ in } \Omega 
    \end{equation*}
    and
    \begin{equation*}
      \left(\bar\Lambda_0,\gradnc\vcr\right)
      = \left(f-\alpha\ucr,
      \vcr\right)
      \quad\text{for all } \vcr\in\CR^1_0(\Tcal),
    \end{equation*}
    and $\tau \in (0, 1]$.
    Then the iterates $(u_j)_{j\in\Nbb}$ of
    the primal-dual iteration converge to $\ucr$ in $L^2(\Omega)$.
  \end{theorem}
\end{frame}

\begin{frame}
  Beweis skizzieren und insbesondere auf meine Hypothesen bzgl Wahl von $\tau$
  eingehen und für Prolongation argumentieren, da initiale Fehler da sind etc

  wohl auch Konvergenztheorem aufführen, Bereich für $\tau$ kurz erläutern,
  vielleicht beim groben erläutern der Beweisidee
\end{frame}

