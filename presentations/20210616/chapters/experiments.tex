\begin{frame}
  drüber nachdenken, was hier gezeigt werden soll. Idealerweise viele 
  subsections mit Themenbereichen (f01, cam, termCrit, tau\ldots)

  termination criteria experiments only in the end if questions arise, only
  mention the possible termination criteria and that they seem equally valid
  (except for energy difference)

  show tau experiments

  energy during a iteration (convergence of subsequences from above, i.e.
  also choose one exampe with osscilating convergence)

  find good alpha for denoising

  show adaptive mesh for camerman and maybe for square to show the working
  of the refinement indicator

  vom Kapitel continuous problem auch die Konstruktion einer exakten Lösung
  anreißen

  L2 Sprünge vielleicht auswerten (bleiben sie konstant\ldots, if we consider
  them, it becomes conforming


  Verfeinerungsindikator, strikte Konvexität, EGLEB alles hier genau dann,
  wenn danach ein Plot dazu kommen soll.

  Probably etaJumps and etaVol Vergleich und eta und Fehler in einem getrennten
  Plot, in einem gesamt Plot dann irgendwann, wo alles drin ist, die etaAnteile
  nicht mehr extra, und wahrscheinlich auch nicht beide Differenzen zu EGLEB

  single all eta and error plot

  single alle gleb and energy differences plot (mit NC und Cont Egleb Differenz
  sind fast gleich, da Differenz zwischen diskreten und kont. gegen 0 geht
  bla bla bla
\end{frame}

\begin{frame}
  tien schicken spätestens am Wochenende vor der Präsi, CC vor der Präsi
  die fertige Präsi + akuteller Stand der Arbeit schicken
\end{frame}
