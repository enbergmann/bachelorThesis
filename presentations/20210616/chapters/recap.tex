\begin{frame}[noframenumbering]{Table of Contents}
  \tableofcontents[currentsection, currentsubsection]
\end{frame}

\subsection{Rudin-Osher-Fatemi Model Problem}
\begin{frame}
  ROF Problem beschreiben mit kurzer Def von $\BV(\Omega)$, $\alpha$ Bedeutung
  nachliefern, vielleicht auch hier schon die Bilder zeigen mit Entrauschen (s.
  Intro BA)
\end{frame}
\subsection{Continuous Problem}
\begin{frame}
  nochmal Existenz und Eindeutigkeitsaussagen zeigen und gaaaanz grob 
  beschreiben, wie die bewiesen werden
\end{frame}

\subsection{Discrete Problem}
\begin{frame}
  nochmal Existenz und Eindeutigkeitsaussagen zeigen und gaaaanz grob 
  beschreiben, wie die bewiesen werden
\end{frame}

\subsection{Discrete Problem continues}
\begin{frame}
  vielleicht in eigener Section: noch die Aussagen aus der Arbeit verarbeiten,
  die über Existenz und Eindeutigkeit hinausgehen
\end{frame}


\begin{frame}
  \fullcite[Chapter 10, p. 297-319]{Bar15}

  \bigskip
  \pause

  Let $\Omega\subset\Rbb^n$ be a bounded polyhedral Lipschitz domain.

  \medskip

  For given $g\in L^2(\Omega)$ and $\alpha>0$ minimize the functional 
  \begin{align*}
    I(v)=|v|_{\BV(\Omega)}+\frac{\alpha}{2}\Vert v-g\Vert^2
  \end{align*}
  amongst all $v\in \BV(\Omega)\cap L^2(\Omega)$.
\end{frame}

\begin{frame}{Functions of Bounded Variation}
  A function $v\in L^1(\Omega)$ with distributional derivative
  $Dv:C^{\infty}_C(\Omega;\Rbb^n)\to\Rbb$ is said to be of bounded variation if 
  there exists $c>0$ such that 
  \begin{align*}
    \langle Dv, \phi\rangle\coloneqq -\int_\Omega v\Div(\phi)\dx\leq
    c\Vert\phi\Vert_{L^\infty(\Omega)}
  \end{align*}
  for all $\phi\in C^1_C(\Omega;\Rbb^n)$.

  \pause  

  The minimal constant $c\geq 0$ satisfying this property is called 
  total variation of $Dv$ and is given by
  \begin{align*}
    |v|_{\BV(\Omega)} = \sup_{\substack{\phi\in C^1_C(\Omega;\Rbb^n)\\
    \Vert\phi\Vert_{L^\infty(\Omega)}\leq 1}}-\int_\Omega v\Div (\phi)\dx.
  \end{align*}

  \pause

  The space of all such functions is denoted by $\BV(\Omega)$.
\end{frame}

\begin{frame}{Properties of $\BV(\Omega)$}
  $\BV(\Omega)$ is a Banach space equipped with the norm
  \begin{align*}
    \Vert v \Vert_{\BV(\Omega)} \coloneqq \Vert v\Vert_{L^1(\Omega)} +
    |v|_{\BV(\Omega)}\quad\text{for all } v\in\BV(\Omega).
  \end{align*}
  
  \pause
  \medskip
  $W^{1,1}(\Omega)\subset\BV(\Omega)$ with $\Vert v\Vert_{\BV(\Omega)}=
  \Vert v\Vert_{W^{1,1}(\Omega)}$ for all $v\in W^{1,1}(\Omega)$.
\end{frame}

\begin{frame}{Notions of convergence on $\BV(\Omega)$}
  Let $(v_n)_{n\in\Nbb}\subset \BV(\Omega)$ and $v\in \BV(\Omega)$ such that
  $v_n\rightarrow v$ in $L^1(\Omega)$ as $n\rightarrow\infty$.
  \pause
  \begin{itemize}
    \item[(i)]
      $(v_n)_{n\in\Nbb}$ converges intermediately or strictly to $v$
      if $|v_n|_{\BV(\Omega)}\rightarrow |v|_{\BV(\Omega)}$ as
      $n\rightarrow\infty$.
      \pause
    \item[(ii)] $(v_n)_{n\in\Nbb}$ converges weakly to
      $v$ if
      $\langle Dv_n,\phi\rangle\rightarrow \langle Dv,\phi\rangle$ 
      for all $\phi\in C_0(\Omega;\Rbb^n)$ as 
      $n\rightarrow\infty$.
  \end{itemize}
\end{frame}

\begin{frame}{Further Properties of $\BV(\Omega)$}
  $C^\infty(\overline\Omega)$ and $C^\infty(\Omega)\cap\BV(\Omega)$ are dense
  in $\BV(\Omega)$ with respect to intermediate convergence.
  
  \pause
  \bigskip

  The embedding $\BV(\Omega)\to L^p(\Omega)$ is continuous for
  $1\leq p\leq n/(n-1)$ and compact for $1\leq p< n/(n-1)$.
  
  \pause
  \bigskip

  There exists a linear operator $T:\BV(\Omega)\to L^1(\partial\Omega)$
  such that $T(v) = v|_{\partial\Omega}$ for all $v\in\BV(\Omega)\cap
  C(\overline\Omega)$.

  $T$ is continuous with respect to intermediate convergence in $\BV(\Omega)$
  but not with respect to weak convergence in $\BV(\Omega)$. 
\end{frame}

