% --- SETTINGS -----------------------------------------------------------------

\documentclass[draft=false,twoside,11pt]{scrreprt}
\newcommand{\lang}{ngerman}
\usepackage[\lang]{babel}
\usepackage[a4paper, 
            width = 150mm, top = 25mm, bottom = 25mm,
            bindingoffset = 6mm]
            {geometry}
% \usepackage{fancyhdr}
\pagestyle{headings}

% \setlength{\parindent}{0pt}

% used packages

\usepackage[utf8]{inputenc}
\usepackage{graphicx}

% --- REFERENCING I-------------------------------------------------------------

\usepackage[hidelinks]{hyperref} % has to be loaded before ams to avoid warnings

%TODO
\usepackage[backend=biber, style = alphabetic]{biblatex}
\addbibresource{sourcesBergmannBT.bib}

% --- MATH ---------------------------------------------------------------------

\usepackage{amsmath} % align
\usepackage{amsthm} % proof
\usepackage{amssymb}
\usepackage{mathtools} %coloneqq
\usepackage{mathabx} %vvvert


% --- REFERENCING II -----------------------------------------------------------

\usepackage[\lang]{cleveref} % has to be loaded after ams and hyperref

% --- MISCELLANEOUS ------------------------------------------------------------

\usepackage{ifthen}


\usepackage{tikz}
%\usepackage{pgfplots}
\usetikzlibrary{arrows.meta}

% --- AMSTHM ENVIRONMENtS ------------------------------------------------------

\theoremstyle{plain}
\newtheorem{theorem}{Theorem}[chapter]
\newtheorem{lemma}[theorem]{Lemma}
\newtheorem{corollary}[theorem]{Corollary}

\theoremstyle{definition}
\newtheorem{definition}[theorem]{Definition}

\theoremstyle{remark}
\newtheorem{remark}[theorem]{Remark}

% --- OPERATORS ----------------------------------------------------------------

\DeclareMathOperator{\interior}{int}
\DeclareMathOperator{\closure}{cl}
\DeclareMathOperator{\sign}{sign}
\DeclareMathOperator{\sgn}{sgn}
\DeclareMathOperator{\Div}{div}
\DeclareMathOperator{\atan}{atan2}
\DeclareMathOperator{\Curl}{Curl}
\newcommand{\conv}{\textup{conv}}
\newcommand{\Mid}{\textup{mid}}

%\DeclareMathOperator*{\liminf}{lim\,inf} % lim inf

% --- SPACES -------------------------------------------------------------------

\newcommand{\BV}{\textup{BV}}
\newcommand{\CR}{\textup{CR}}
\newcommand{\C}{\textup{C}}
\newcommand{\Vnc}{\ensuremath{V_\textup{NC}}}
% already defined\newcommand{\P}{\textup{P}}

% --- VARIABLES ----------------------------------------------------------------

\newcommand{\ucr}{u_\textup{CR}}
\newcommand{\vcr}{v_\textup{CR}}
\newcommand{\snr}{\textup{SNR}}
\newcommand{\Egleb}{E_\textup{GLEB}}

% --- FUNCTIONS ----------------------------------------------------------------

\newcommand{\Enc}{E_\textup{NC}}
\newcommand{\anc}{a_\textup{NC}}

% --- SYMBOLS/OPERATORS --------------------------------------------------------

\newcommand{\nc}{\textup{NC}}
\newcommand{\NC}{\textup{NC}}
\newcommand{\dx}{\,\mathrm{d}x}
\newcommand{\ds}{\,\mathrm{d}s}
\newcommand{\dt}{\,\mathrm{d}t}
\newcommand{\dmu}{\,\mathrm{d}\mu}
\newcommand{\gradnc}{\nabla_\textup{NC}}

% --- RELATIONS ----------------------------------------------------------------

\newcommand{\ssubset}{\subset\joinrel\subset}

% --- CALIGRAPHIC LETTERS ------------------------------------------------------

\newcommand\Acal{\mathcal{A}}
\newcommand\Bcal{\mathcal{B}}
\newcommand\Ccal{\mathcal{C}}
\newcommand\Dcal{\mathcal{D}}
\newcommand\Ecal{\mathcal{E}}
\newcommand\Fcal{\mathcal{F}}
\newcommand\Gcal{\mathcal{G}}
\newcommand\Hcal{\mathcal{H}}
\newcommand\Ical{\mathcal{I}}
\newcommand\Jcal{\mathcal{J}}
\newcommand\Kcal{\mathcal{K}}
\newcommand\Lcal{\mathcal{L}}
\newcommand\Mcal{\mathcal{M}}
\newcommand\Ncal{\mathcal{N}}
\newcommand\Ocal{\mathcal{O}}
\newcommand\Pcal{\mathcal{P}}
\newcommand\Qcal{\mathcal{Q}}
\newcommand\Rcal{\mathcal{R}}
\newcommand\Scal{\mathcal{S}}
\newcommand\Tcal{\mathcal{T}}
\newcommand\Ucal{\mathcal{U}}
\newcommand\Vcal{\mathcal{V}}
\newcommand\Wcal{\mathcal{W}}
\newcommand\Xcal{\mathcal{X}}
\newcommand\Ycal{\mathcal{Y}}
\newcommand\Zcal{\mathcal{Z}}

% --- MATHBB LETTERS -----------------------------------------------------------

\newcommand\Abb{\mathbb{A}}
%\newcommand\Bbb{\mathbb{B}}
\newcommand\Cbb{\mathbb{C}}
\newcommand\Dbb{\mathbb{D}}
\newcommand\Ebb{\mathbb{E}}
\newcommand\Fbb{\mathbb{F}}
\newcommand\Gbb{\mathbb{G}}
\newcommand\Hbb{\mathbb{H}}
\newcommand\Ibb{\mathbb{I}}
\newcommand\Jbb{\mathbb{J}}
\newcommand\Kbb{\mathbb{K}}
\newcommand\Lbb{\mathbb{L}}
\newcommand\Mbb{\mathbb{M}}
\newcommand\Nbb{\mathbb{N}}
\newcommand\Obb{\mathbb{O}}
\newcommand\Pbb{\mathbb{P}}
\newcommand\Qbb{\mathbb{Q}}
\newcommand\Rbb{\mathbb{R}}
\newcommand\Sbb{\mathbb{S}}
\newcommand\Tbb{\mathbb{T}}
\newcommand\Ubb{\mathbb{U}}
\newcommand\Vbb{\mathbb{V}}
\newcommand\Wbb{\mathbb{W}}
\newcommand\Xbb{\mathbb{X}}
\newcommand\Ybb{\mathbb{Y}}
\newcommand\Zbb{\mathbb{Z}}

% --- INTEGRAL MEAN ------------------------------------------------------------

\def\Xint#1{\mathchoice
{\XXint\displaystyle\textstyle{#1}}%
{\XXint\textstyle\scriptstyle{#1}}%
{\XXint\scriptstyle\scriptscriptstyle{#1}}%
{\XXint\scriptscriptstyle\scriptscriptstyle{#1}}%
\!\int}
\def\XXint#1#2#3{{\setbox0=\hbox{$#1{#2#3}{\int}$}
\vcenter{\hbox{$#2#3$}}\kern-.5\wd0}}
\newcommand{\intmean}{\Xint-}

% --- PERFECT BULLET -----------------------------------------------------------

\let\oldbullet\bullet
\newlength{\raisebulletlen}
\setbox1=\hbox{$\bullet$}\setbox2=\hbox{\tiny$\bullet$}
\setlength{\raisebulletlen}{\dimexpr0.5\ht1-0.5\ht2}
\renewcommand\bullet{\raisebox{\raisebulletlen}{\,\tiny$\oldbullet$}\,}
 

% --- DEBUG -------------------------------------------------------------------
\overfullrule=2cm % marks overfull hboxes by black bar in document
\usepackage{todonotes} % for the todo's
%\usepackage[disable]{todonotes} % for the todo's
\usepackage{xcolor}


% --- META INFORMATION --------------------------------------------------------

\titlehead{%
  \begin{minipage}{.7\textwidth}%
  Humboldt-Universit\"at zu Berlin\\
  Mathematisch-Naturwissenschaftliche Fakult\"at\\
  Institut f\"ur Mathematik
  \end{minipage}
  \begin{minipage}{.29\textwidth}%
    \begin{flushright}
      \includegraphics*[scale=.6]{pictures/logos/husiegel_bw.pdf}%
    \end{flushright}
  \end{minipage}
}
\title{Die Crouzeix\--Raviart\--Finite\--Elemente\--Methode für
eine nichtkonforme Formulierung des Rudin\--Osher\--Fatemi\--Modellproblems}
\author{Enrico Bergmann}
\date{Version:~\today}

% --- DOCUMENT -----------------------------------------------------------------

\begin{document}

\begin{titlepage}
  \begin{minipage}{.65\textwidth}
  {\scshape{\Large Humboldt-Universität zu Berlin}\\
  Mathematisch-Naturwissenschaftliche Fakultät\\
  Institut für Mathematik\par}
  \end{minipage}
  \begin{minipage}{.25\textwidth}
    \begin{flushright}
      \includegraphics*[scale=.6]{pictures/logos/husiegel_bw.pdf}
    \end{flushright}
  \end{minipage}

  \centering

  \vspace{3cm}

  {\huge\bfseries Die Crouzeix\--Raviart\--Finite\--Elemente\--Methode für eine
  nichtkonforme Formulierung des Rudin\--Osher\--Fatemi\--Modellproblems\par}

  \vspace{2cm}

  {\huge Bachelorarbeit}

  \vspace{1.5cm}

  {\LARGE zur Erlangung des akademischen Grades}

  \vspace{0.3cm}

  {\LARGE Bachelor of Science (B.\ Sc.)}

  \vspace{3cm}

  \flushleft
  {\LARGE
  \begin{tabular}{ll}
  eingereicht von: &Enrico Bergmann\\
  geboren am: &13.10.1995\\
  geboren in: &Berlin\\
    Gutachter/innen: &Prof.\ Dr.\ Carsten Carstensen\\
    &Dr.\ Philipp Bringmann
  \end{tabular}
  }
  \vfill

  %{Eingereicht am Institut für Mathematik der Humboldt-Universität zu
  %Berlin am:}
  {\small Eingereicht am Institut für Mathematik der Humboldt-Universität zu
  Berlin am: }
\end{titlepage}


\tableofcontents

%\chapter*{Zusammenfassung}
%In dieser Arbeit untersuchen wir eine primale-duale Iteration zur numerischen
Lösung einer mit Crouzeix\--Raviart\--Finite\--Elemente\--Funktionen
diskretisierten, nichtkonformen Formulierung des
Rudin\--Osher\--Fatemi\--Mi\-ni\-mie\-rungs\-problems, welches die
Rauschunterdrückung in der Bildverarbeitung mithilfe von Funktionen
beschränkter Variation modelliert.
TODO




\newpage

% TODO to toc
% \todototoc
%\listoftodos
% TODO

\chapter{Einleitung}
\label{chap:introduction}
write the introduction similar to
[bartelsErrorControlAndAdaptivityForBV], just more details. See below.

In particular, drop infos needed about BV (wir werden sehen, dass BV Funktionen
dies und das erfüllen und deshalb diese und jene Terme definiert sind)
or (wir werden zuerst grundlagen einführen aus der Optimierung und über BV 
Funktionen [dann verweis auf das entsprechende Kapitel], anschließend
das konitnuierliche Problem betrachten [verweis auf kapitel] \ldots)
\bigskip

1. BV Funtionen allgemein: lassen unstetigkeiten zu, mehr als Sobolev 
Funktionen,
[hier nochmal überlegen, wie ich dieses über 'manifold' (one codimensional
(ABM)) springen schreibe, ohne ins Detail gehen zu müssen]
\medskip

deshalb Anwendung zum Beispiel in
\medskip

[\cite{ABM14} modelization of a large number of problems in physics, mechanics,
or image processing requires the introducion of new functionals spaces
permitting discontinuities of the solution. In phase transitions, image
segmentation, plasticity theory, the study of crasks and fissures, the study of
the wake in fluid dynamics , and so forth, the solution of the problem presents
discontinuities along one-odimensionalmanifolds.
-solution of these problems cannot be found in classical Sobolev spaces
\medskip

Viele physikalischen Anwendungen können mit kontinuierlichen Funktionen nicht
beschrieben werden [Bartels, Error Control and Adaptivity for a variational
model problem defined on functions of bounded variation, und darin zitierte
Quellen].
\medskip

introduction aus [Braides] zitieren einfach mit Bemerkung, dass man dort
Beispiele finden kann (vielleicht auch nicht, da sie direkt auf SBV
eingeht und nicht auf BV)
\medskip

insbesondere in der 'Bildbearbeitung' (finde
noch das richtige Wort, Computer Vision, computergestützte Analyse von Bildern
oder so änhlich) wird das genutzt
(Quellen: \cite{AK06}, \ldots)

\cite{Get12} ist mglw ein guter Verweis für weitere Infos zum Denoisung und
wahrscheinlich in Kapitel 6 auch zur Wahl von alpha, sagt
aber auch viel über die Interpretation der Parameter
Wahrscheinlich auch sowas schreiben wie ,Die Anwendung ist kein zentraler
Punkt in dieser Arbeit, sonder \ldots, für weitere Details zum Denoising siehe
\ldots.' Dann im Experimente Kapitel eine kleine Rand-Section, in der geguckt
wird, wie verschiedene Alpha sich auf Denoising auswirken, rein optisch 'nur
ein kleiner experementeller Exkurs zur Wahr des Tuning Parameters alpha'

Vielleicht auch den Press Release zum Schwarzen Loch Bild erwähnen oder ist
das zu random?


\medskip

2. als Modell Problem dient das ROF-Modell (\cite[sec 6.2.1]{CP10}, beschreibt
auch,
was die Parameter und Terme sollen, aber vielleicht auch noch andere Quellen 
dazu (''siehe auch JZ13'')

\medskip
natürlich an dieser Stelle das Funktional einmal wie in Bartels hinschreiben

an der stelle vielleicht auch kurz schreiben, was Bartels tut und dann was wir
hier tun im Unterschied dazu

\medskip

benannt nach Rudin, Osher und Fatemi, die das Modell vorschlugen in
\cite{ROF92} (das ist auch aus CP10, also mit in das Zitat von CP10 rein)

\medskip
u ist die exakte Lösung, g das gesehene verrauschte Bild und alpha Gewichtet 
das Verhältniss zwischen Regularisierung und data fitting (tradeoff
between regularization and data fitting, CP10)
(zitiere eines der Paper/Bücher, CP10 above klärt das ganz gut)
große alpha wenig Entrauschen, kleine alpha blurry, oversmoothed u
\cite{JZ13}

\medskip
TV Term verringert Osszilationen, lässt aber Unstetigkeiten der Lösung zu.
Regularisierungs Term, der Rauschen und kleine Details entfernt.

\medskip
der zweite Term versucht die Lösung nahe an g zu halten (Treue, fidelity) (cite
getreuer und JZ13, hat aber keinen mathscinet entry), vielleicht auch nur
für mich

\medskip
INSBESONDERE erst ROF beschreiben und Parameter erklären und dann CCs 
Variante hinchreiben

\medskip
3. wir betrachten hier eine leicht abgewandlete Variante, die wie folgt
zusammenhängt (dann dieses Remark (aber wohl nicht mehr als Remark)

\begin{remark}
  In \cite[Kapitel~10.1.3]{Bar15} wird \Cref{prob:continuousProblem} für ein
  gegebenes $g\in L^2(\Omega)$ formuliert
  mit dem Funktional 
  \begin{align*}
    I(v)\coloneqq |v|_{\BV(\Omega)} + \frac{\alpha}{2}\int_\Omega (v-g)^2\dx
  \end{align*}
  für $v\in \BV(\Omega)\cap L^2(\Omega)$.

  Nun wählen wir $f = \alpha g$. Dann gilt
  $I(v) = E(v) - \Vert v\Vert_{L^1(\partial \Omega)}+ 
  \frac{\alpha}{2}\Vert g\Vert_{L^2(\Omega)}^2$ für alle 
  $v\in \BV(\Omega)\cap L^2(\Omega)$. Da der Term $\frac{\alpha}{2}\Vert
  g\Vert_{L^2(\Omega)}^2$ konstant ist, haben die Funktionale $E$ und $I$ somit
  die gleichen Minimierer in $\left\{v\in\BV(\Omega)\cap L^2(\Omega)\mid 
  \Vert v\Vert_{L^1(\partial\Omega)}=0\right\}$.
\end{remark}

4. Kurzen Überblick über Aufbau der Arbeit (,,In Kap 1 machen wir das, dann das
in Kapitel 2, dann das in Kapitel 3, \ldots``)

5. auch ein hübsches Bild wie bei Philipp (siehe Paper, sowas wie: Originalbild 
links, verrauschtes Bild (Art des Rauschens nennen) mitte, entrauschtes Bild
rechts). Falls ich die Experimente so eingestellt bekomme, dass sie entrauschen
 (todo)


\chapter{Theoretische Grundlagen}
\label{chap:theoreticalBasics}
\section{Notation}
\todo[inline]{TODO usw alle Notationen einführen, die in der ganzen Arbeit
gelten, vergleiche Bringmann 2.1 Einleitung. Da wir aber erst sehr allgemein
sind bzw hier für Subdiffs und Variationsrechnung etc zwischen Räumen 
umhersprigen, dieses Kapitel doch lieber nicht machen sondern diese Themen (eben
theoretische Grundlagen) abarbeiten und dann immer weiter einschränken Kapitel
weise (continuous Problem schränkt Omega ein, dann discrete schränkt zu 2D ein
und führt CR ein, usw.) Notation hier also mglw einstampfen und on the fly 
machen}
\todo[inline]{FRAGE Dann zB CR Notation erst im entsprechenden Kapitel 
einführen oder auch das schon hier? Oder hier nur ABSOLUTE Grundlagen, extrem 
basic, und alles andere dann on the fly?}

\todo[inline]{TODO
 nur die Sachen rausschreiben/zusammentragen/zitieren (um später Theoreme und
 Gleichungen zitieren zu können statt Bücher) die auch wirklich gebraucht 
 werden später in Beweisen. Insbesondere am Ende nochmal durchgucken, was 
 wirklich gebraucht wurde und ungebrauchtes und/oder uninteressanter und/oder
 unwichtiges rauswerfen}


\section{Benötigte Begriffe der Variationsrechnung in Banachräumen}
\todo[inline]{im Zeidler lokal konvex Raum. Das irgendwie noch ausdrücken und
die Aussage, dass B-spaces lokal konvex sind, einmal rechtzeitig erwähnen.}

Die folgenden Aussagen basieren auf \cite[S. 189-192]{Zei85}.
Wir betrachten einen Banachraum $X$, eine Teilmenge $V\subseteq X$ und ein
Funktional $F:V\to\Rbb$. Sei $u$ ein innerer Punkt von $V$. Außerdem
definieren wir für $h\in X$ die Funktion $\varphi_h:\Rbb\to\Rbb$ durch 
$\varphi_h(t)\coloneqq F(u+th)$ für alle $t\in\Rbb$.

\begin{definition}[$n$-te Variation]
  Die $n$-te Variation von $F$ an der Stelle $u$ in Richtung $h\in X$ ist 
  \begin{align*}
    \delta^n F(u;h)\coloneqq \varphi_h^{(n)}(0)=
    \left. \frac{d^n F(u+th)}{dt^n}\right|_{t=0},
  \end{align*}
  falls $\varphi_h^{(n)}(0)$ existiert. Wir schreiben $\delta$ für $\delta^1$.
\end{definition}

\begin{definition}[G\^ateaux- und Fr\'echet-Differential]
  $F$ heißt G\^ateaux-differenzierbar an der Stelle $u$, falls ein 
  Funktional $F'(u)\in X^\ast$ existiert mit 
  \begin{align*}
    \lim_{t\to 0}\frac{F(u+th)-F(u)}{t} = \langle F'(u), h\rangle\quad
    \text{für alle } h\in X.
  \end{align*}
  $F'(u)$ heißt dann G\^ateaux-Differential von $F$ an der Stelle $u$.

  $F$ heißt Fr\'echet-differenzierbar an der Stelle $u$, falls ein Funktional
  $F'(u)\in X^\ast$ existiert, sodass
  \begin{align*}
    \lim_{\Vert h\Vert_X\to 0}\frac{|F(u+th)-F(u)-
    \langle F'(u),h\rangle|}{\Vert h\Vert_X} =0.
  \end{align*}
  $F'(u)$ heißt dann Fr\'echet-Differential von $F$ an der Stelle $u$.
  Das Fr\'echet-Differential von $F$ an der Stelle $u$ in Richtung $h\in X$
  ist definiert durch $dF(u;h)\coloneq \langle F'(u),h\rangle.$
\end{definition}

Die wichtigsten Eigenschaften sind noch einmal in folgender Bemerkung 
zusammengefasst.

\begin{remark}
  \begin{itemize}
    \item Existiert das Fr\'echet-Differential $F'(u)$ von $F$ an der Stelle
      $u$, so ist $F'(u)$ auch das G\^ateaux-Differential von $F$ an der Stelle
      $u$ und es gilt 
      \begin{align*}
        \delta F(u;h)=dF(u;h)=\langle F'(u),h\rangle\quad\text{für alle } h\in
        X.
      \end{align*}
%      Außerdem ist $F$ dann stetig an der Stelle $u$.
  \end{itemize}
\end{remark}

Damit können wir eine wichtige Aussage der Variationsrechnung formulieren,
basierend auf \cite[S. 193ff., Theorem 40.A, Theorem 40.B]{Zei85}.

\begin{theorem}[Notwendige Optimalitätsbedingung erster Ordnung]
  \label{thm:necessaryConditionFreeLocalExtrema}
  Sei $u\in \interior(V)$ lokaler Minimierer von $F$, das heißt
  es existiere eine Umgebung 
  \todo[inline]{Definiere Umgebung so wie in Zeilder 'neigborhood', 
  \cite[S. 751, (5)]{Zei86} (also es gibt eine offene Menge in der Umgebung,
  die den Punkt enthällt} 
  $U$ von $u$, so dass $F(v)\geq F(u)$ für alle $v\in U$. Dann gilt für alle
  $h\in X$, dass $\delta F(u;h) = 0$, falls diese Variation für alle $h\in X$
  existiert, beziehungsweise $F'(u) = 0$, falls $F'(u)$ als 
  G\^ateaux- oder Fr\'echet-Differential existiert.
\end{theorem}

\section{Subdifferential}
In diesem Abschnitt trage ich die in dieser Arbeit benötigten Eigenschaften 
des Subdifferentials eines Funktionals $F:X\to [-\infty,\infty]$ 
auf einem Banachraum 
$(X,\Vert\bullet\Vert_X)$ und die dafür benötigten Begriffe zusammen.

Zunächst eine grundlegende Definition.

\begin{definition}[\protect{\cite[S. 245, Definition 42.1]{Zei85}}]
  Sei $X$ ein Vektorraum, $M\subseteq X$ und $F:M\to\Rbb$. 
  
  Dann heißt die Menge $M$ konvex, wenn für alle $u,v\in M$ und alle $t\in
  [0,1]$ gilt $(1-t)u+tv\in M$.

  Ist $M$ konvex, so heißt $F$ konvex, falls für alle $u,v\in M$ und alle
  $t\in[0,1]$ gilt $F\big( (1-t)u+tv\big)\leq (1-t)F(u)+t F(v).$
\end{definition}

In \cite{Zei85} werden einige der folgenden Aussagen auf reellen lokal konvexen
Räumen $X$ formuliert.
Da nach \cite[S. 781, (43)]{Zei86} alle Banachräume 
(in \cite{Zei86} und \cite{Zei85} genannt \glqq B-spaces\grqq, \cite[S.
786]{Zei86}) lokal
konvex sind und in dieser Arbeit die Aussagen nur auf Banachräumen benötigt
werden, beschränke ich
die folgenden Aussagen, falls nicht anders spezifiziert, wie folgt.
Sei $(X,\Vert\bullet\Vert_X)$ ein reeller Banachraum und
$F:X\to [-\infty,\infty]$ ein Funktional auf $X$.

\begin{definition}[Subdifferential, \protect{\cite[S. 385,
  Definition 47.8]{Zei85}}]
  \label{def:subdifferential}
  Für $u\in X$ mit $F(u)\neq\pm\infty$ heißt
  \begin{equation}
    \label{equ:subdifferential}
    \partial F(u)\coloneq 
    \{u^\ast\in X^\ast\ |\ 
    \forall v\in X\quad F(v)\geq F(u)+\langle u^\ast,v-u\rangle\}  
  \end{equation}
  Subdifferential von $F$ an der Stelle $u$. Für $F(u)=\pm\infty$ definiere
  $\partial F(u)\coloneq\emptyset$.

  Ein Element $u^\ast\in\partial F(u)$ heißt Subgradient von $F$ an der Stelle
  $u$.
\end{definition}

\begin{theorem}[\protect{\cite[S. 387, Proposition 47.12]{Zei85}}]
  \label{thm:extremalprinciple}
  Falls $F: X\to (-\infty,\infty]$ mit $F\nequiv\infty$, gilt
  $F(u)=\inf_{v\in X}F(v)$ genau dann, wenn $0\in\partial F(u)$.
\end{theorem}

\begin{theorem}[\protect{\cite[S. 387, Proposition 47.13]{Zei85}}]
  \label{thm:subdiffGateaux}
  Falls $F$ konvex ist und G\^{a}teaux-differenzierbar
  (in \cite{Zei86} und \cite{Zei85} genannt \glqq G-differentiable\grqq, \cite[S.
  135f.]{Zei86})
  an der Stelle $u\in X$ mit G\^{a}teaux-Differential $F'(u)$,
  gilt $\partial F(u)=\{F'(u)\}$.
  \todo[inline]{TODO checke Notation und Definition mit der
  Gateaux/Frechet-differenzierbarkeit, für die ich mich entschieden habe (d.h.
  meint Zeidler das gleiche  
  
  Bemerke, die Prop liefert noch wann das umgekehrte gilt, aber nur
  aufschreiben, wenn das mal benötigt wird in dieser Arbeit

  nutze vielleicht Zeidler I als Quelle für die Differentiale und vielleicht
  auch Notation?}
\end{theorem}

Das folgende Theorem folgt aus \cite[S. 389, Theorem 47.B]{Zei85} unter 
Beachtung der Tatsache, dass die Addition von Funktionalen 
$F_1,F_2,\ldots,F_n:X\to (-\infty,\infty]$ und die Addition von
Menge in $X^\ast$ kommutieren.

\begin{theorem}
  \label{thm:subdifferentialSumRule}
  Seien für $n\geq 2$ die Funktionale $F_1,F_2,\ldots,F_n: X\to
  (-\infty,\infty]$ konvex und es existiere
  ein $u_0\in X$ und ein $j\in\{1,2,\ldots,n\}$ mit $F_k(u_0)<\infty$
  für alle $k\in\{1,2\ldots,n\}$, 
  sodass für alle $k\in\{1,2,\ldots,n\}\setminus\{j\}$ das Funktional
  $F_k$ stetig an der Stelle $u_0$ ist.

  Dann gilt 
  \begin{align*}
    \partial (F_1+F_2+\ldots+ F_n)(u) 
    = \partial F_1(u)+\partial F_2(u)+ \ldots + \partial F_n(u) \quad\text{für
    alle } u\in X.
  \end{align*}
\end{theorem}

\begin{theorem}[\protect{\cite[S. 396f., Definition 47.15, Theorem
  47.F]{Zei85}}]
  \label{thm:subdifferentialMonotonicity}
  Sei $F:X\to (-\infty,\infty]$ konvex und unterhalbstetig mit $F\nequiv\infty$.

  Dann ist $\partial F(\bullet)$ monoton, das heißt 
  \begin{align*}
    \langle u^\ast-v^\ast,u-v\rangle\geq 0\quad \text{für alle } u,v\in X, 
    u^\ast \in \partial F(u), v^\ast \in \partial F(v).
  \end{align*}
\end{theorem}


\section{Funktionen Beschränkter Variation}

Dieser Abschnitt präsentiert die für diese Arbeite benötigten Aussagen 
über Funktionen beschrankter Variation und basiert auf
Kapitel 10
in \cite{ABM14}.
\todo[inline]{Schreibe vlt noch sowas wie ,Für weitere Aussagen und Details
zu BV Funktionen und den maßtheoretischen Grundlagen dafür siehe [EG92,
Braides, ABM14]'}
Dabei sei $\Omega$ eine offene Teilmenge des $\Rbb^n$.
Der Raum aller $\Rbb^n$-wertigen Borelmaße wird bezeichnet mit
$M(\Omega;\Rbb^n)$ und ist nach Riesz identifizierbar mit dem Dualraum von
$C_0(\Omega;\Rbb^n)$ ausgestattet mit der Norm
$\Vert\phi\Vert_\infty\coloneqq
(\sum_{j=1}^n\sup_{x\in\Omega}|\phi_j(x)|^2)^{1/2}$ für $\phi\in
C_0(\Omega;\Rbb^n)$.

Die folgende Definition basiert auf \cite[S. 393 f.]{ABM14}.
\todo[inline]{Wenn einmal gesagt wurde worauf diese Section basiert, sind
dann noch Zitate wie dieses notwendig? Ich würde einfach nur noch zitieren,
wenn eine Aussage mal aus einer anderen Quelle kommt.}

\begin{definition}[Funktionen beschränkter Variation]
  Eine Funktion $u\in L^1(\Omega)$ ist von beschränkter Variation, wenn ihre
  distributionelle Ableitung $Du$ ein Element in $M(\Omega;\Rbb^n)$ definiert.
  Das ist äquivalent zu der Bedingung
  \begin{align}
    \label{eq:boundedVariation}
    |u|_{\BV(\Omega)}
    \coloneqq
    \sup_{\substack{\phi\in C^1_C(\Omega;\Rbb^n)\\
    \Vert\phi\Vert_{L^\infty(\Omega)}\leq 1}}\int_\Omega u\Div (\phi)\dx
    <
    \infty.
  \end{align}

  Durch $|\bullet|_{\BV(\Omega)}$ ist eine Seminorm auf $\BV(\Omega)$
  gegeben.

  Der Raum aller Funktionen beschränkter Variation $\BV(\Omega)$
  ist ausgestattet mit der Norm 
  \begin{align*}
    \Vert u \Vert_{\BV(\Omega)} \coloneqq \Vert u\Vert_{L^1(\Omega)} +
    |u|_{\BV(\Omega)}
  \end{align*}
  für $u\in\BV(\Omega)$.

  Nach \cite[S. 395, Theorem 10.1.1.]{ABM14} ist $(\BV(\Omega),
  \Vert\bullet\Vert_{\BV(\Omega)})$ ein Banachraum.
\end{definition}

\begin{remark}
  Es gilt $W^{1,1}(\Omega)\subset\BV(\Omega)$ und 
  $\Vert u \Vert_{\BV(\Omega)}=\Vert u \Vert_{W^{1,1}(\Omega)}$ für alle
  $u\in W^{1,1}(\Omega)$. (\cite[S. 394]{ABM14})
\end{remark}

\begin{definition}
  Sei $(u_n)_{n\in\Nbb}\subset \BV(\Omega)$ und sei $u\in \BV(\Omega)$ mit
  $u_n\rightarrow u$ in $L^1(\Omega)$ für $n\rightarrow\infty$.
  \begin{itemize}
    \item[(i)]
      Die Folge $(u_n)_{n\in\Nbb}$ konvergiert strikt gegen $u$,
      wenn $|u_n|_{\BV(\Omega)}\rightarrow |u|_{\BV(\Omega)}$ für
      $n\rightarrow\infty$
      (unter Beachtung von \cite[Remark 10.1.1]{ABM14}).
    \item[(ii)] Die Folge $(u_n)_{n\in\Nbb}$ konvergiert
      schwach gegen $u$, wenn
      $Du_n$ in 
      $M(\Omega;\Rbb^n)$ schwach gegen $Du$ konvergiert.
  \end{itemize}
\end{definition}

\begin{theorem}[Schwache Unterhalbstetigkeit, Prop. 10.1.1]
  \label{thm:wlsc}
  Sei $(u_n)_{n\in\Nbb}$ eine Folge in $\BV(\Omega)$ mit
  $\sup_{n\in\Nbb}|u_n|_{\BV(\Omega)}< \infty$ und $u\in L^1(\Omega)$ 
  mit $u_n\rightarrow u$ in $L^1(\Omega)$ für $n\rightarrow\infty$.

  Dann gilt $u\in\BV(\Omega)$ und $|u|_{\BV(\Omega)}\leq
  \liminf_{n\rightarrow\infty}|u_n|_{\BV(\Omega)}.$
  Außerdem konvergiert $u_n$ schwach gegen $u$ in $\BV(\Omega)$.
\end{theorem}

Mit Blick auf die folgenden Kapitel, sei $\Omega\subset\Rbb^n$ nun ein
polygonal berandetes Lipschitz-Gebiet.
\todo[inline]{Nochmal nachfragen: das heißt tatsächlich offen, beschränkt mit
Lipschitz Rand, korrekt?}
\todo[inline]{Nochmal nachfragen: $\sup u_k<\infty\Leftrightarrow u_k$ bounded,
korrekt? Ich übersehe da nichts, oder? Falls doch, alle Theoreme nochmal 
nachschlagen und sichergehen, dass sie richtig zitiert sind.}
\todo[inline]{Die BV Aussagen aus den nächsten Kapitel wieder hierher holen, soweit 
sinnvoll. Vielleicht auch nur die Spuroperator Aussage. Kann kopiert werden
aus delTexts.tex.}

\begin{theorem}[\protect{\cite[S. 176, Theorem 4]{EG92}}]
  \label{thm:l1ConvergentSubsequence}
  Sei $(u_n)_{n\in\Nbb}\subset \BV(\Omega)$ eine beschränkte Folge. Dann 
  existiert eine Teilfolge $(u_{n_k})_{k\in\Nbb}$ von
  $(u_n)_{n\in\Nbb}$ und ein $u\in\BV(\Omega)$, sodass
  $u_{n_k}\to u$ in $L^1(\Omega)$ für $k\to \infty$.
\end{theorem}
\todo[inline]{Das kann vielleciht auch noch im kontinuierlichen Existenzbeweis 
eingebracht werden und den etwas vereinfachen/verkürzen. Prüf das Zukunfts-Ich}

\begin{theorem}
  \label{thm:compactness}
  Sei $(u_n)_{n\in\Nbb}\subset \BV(\Omega)$ eine beschränkte Folge. Dann 
  existiert eine Teilfolge $(u_{n_k})_{k\in\Nbb}$ und ein $u\in\BV(\Omega)$,
  sodass $u_{n_k}\rightharpoonup u$ in $\BV(\Omega)$ für $k\rightarrow\infty$.
\end{theorem}

\begin{proof}
  Nach \Cref{thm:l1ConvergentSubsequence} besitzt $(u_n)_{n\in\Nbb}$ eine
  Teilfolge $(u_{n_k})_{k\in\Nbb}$, die in $L^1(\Omega)$ gegen ein
  $u\in\BV(\Omega)$ konvergiert.
  Diese Folge ist nach Voraussetzung ebenfalls beschränkt in 
  $\BV(\Omega)$, woraus nach Definition der Norm auf $\BV(\Omega)$ insbesondere
  folgt, dass
  $\sup_{k\in\Nbb}|u_{n_k}|_{\BV(\Omega)}< \infty$. 
  
  Insgesamt liefert \Cref{thm:wlsc} dann die schwache Konvergenz von
  $(u_{n_k})_{k\in\Nbb}$ in $\BV(\Omega)$ gegen $u\in\BV(\Omega)$.
\end{proof}


\chapter{Das kontinuierliche Problem}
\label{chap:continuousProblem}
In diesem Kapitel wollen wir für einen Parameter $\alpha\in\Rbb_+$ und eine
Funktion $f\in L^2(\Omega)$ folgendes Minimierungsproblem untersuchen.

\begin{problem}\label{prob:continuousProblem}
  Finde $u\in \BV(\Omega)\cap L^2(\Omega)$, sodass
  $u$ das Funktional
  \begin{align}\label{eq:continuousProblem}
    E(v)\coloneqq \frac{\alpha}{2}\Vert v\Vert^2 + |v|_{\BV(\Omega)}
    +\Vert v\Vert_{L^1(\partial\Omega)}-\int_\Omega fv\dx
  \end{align}
  unter allen $v\in\BV(\Omega)\cap L^2(\Omega)$ minimiert.
\end{problem}
Dabei ist der Term $\Vert v\Vert_{L^1(\partial\Omega)}$ wohldefiniert, da
nach \cite[S. 400, Theorem 10.2.1]{ABM14} eine lineare, stetige Abbildung
$T:\BV(\Omega)\to L^1(\partial\Omega)$ existiert mit $T(v) =
v|_{\partial\Omega}$ für alle $v\in\BV(\Omega)\cap C(\overline\Omega)$.
Wie in \Cref{chap:introduction} beschrieben, hat \Cref{prob:continuousProblem}
für homogene Randdaten die gleichen Minimierer wie das ROF-Modellproblem mit
Eingangssignal $g\in L^2(\Omega)$, falls $f=\alpha g$.

\begin{remark}
  \label{rem:embedding}
  Für $d\in\{2,3\}$ und ein beschränktes Lipschitz-Gebiet $U\subset\Rbb^d$ ist
  nach \cite[S. 302, Remark 10.5 (i)]{Bar15} die Einbettung
  $\BV(U)\hookrightarrow L^p(U)$ stetig, wenn $1\leq p\leq d/(d-1)$. 
  Damit ist $\BV(\Omega)$ für das polygonal berandete Lipschitz-Gebiet
  $\Omega\subset \Rbb^2$ Teilmenge von $L^2(\Omega)$ und die Lösung von
  \Cref{prob:continuousProblem} kann in $\BV(\Omega)$ gesucht werden. 
  Wir vernachlässigen diese Vereinfachung und erreichen dadurch, dass alle
  Aussagen im nachfolgenden Abschnitt auch gelten, falls $\Omega\subset\Rbb^d$
  mit $d\in\Nbb$ ein beschränktes Lipschitz-Gebiet ist.
\end{remark}

In den folgenden Abschnitten zeigen wir, dass
\Cref{prob:continuousProblem} eine eindeutige Lösung $u\in\BV(\Omega)\cap
L^2(\Omega)$ besitzt. 
Außerdem beschreiben wir, wie $f$ für eine gegebene Lösung $u$ konstruiert
werden kann und welche Eigenschaften $u$ dafür erfüllen muss.


\section{Existenz eines eindeutigen Minimierers}
Nach \Cref{rem:embedding} kann in diesem Abschnitt $\Omega\subset\Rbb^d$ für
$d\in\Nbb$ ein beliebiges beschränktes Lipschitz-Gebiet sein.
Zunächst zeigen wir, dass \Cref{prob:continuousProblem} eine Lösung besitzt.
Dafür benötigen wir die folgende Formulierung der Youngschen Ungleichung.

\begin{lemma}[Youngsche Ungleichung]
  \label{lem:young}
  Seien $a,b\in\Rbb$ und $\varepsilon\in\Rbb_+$ beliebig. Dann gilt
  \begin{align*}
    ab\leq\frac{1}{\varepsilon}a^2+\frac{\varepsilon}{4}b^2. 
  \end{align*}
\end{lemma}

Außerdem wird die folgende Aussage benötigt, die direkt aus \cite[S. 183,
Theorem 1]{EG92} folgt, da
$0\in\BV\!\left(\Rbb^d\setminus\overline\Omega\right)$ mit
$|0|_{\BV\!\left(\Rbb^d\setminus\overline\Omega\right)}=0$ und
$0|_{\partial\Omega}=0$.

\begin{lemma}
  \label{lem:bvExtension}
  Sei $v\in\BV(\Omega)$.
  Definiere die Fortsetzung $\tilde{v}$ von $v$ für alle $x\in\Rbb^d$ durch
  \begin{align*}
    \tilde{v}(x)\coloneqq
    \begin{cases}
      v(x),  &\text{ falls } x\in\Omega,\\
      0,     &\text{ falls } x\in\Rbb^d\setminus\overline\Omega.
    \end{cases} 
  \end{align*}
  Dann gilt $\tilde{v}\in\BV\!\left(\Rbb^d\right)$ und
  $\left|\tilde{v}\right|_{\BV\!\left(\Rbb^d\right)}
  = |v|_{\BV(\Omega)}+\Vert v\Vert_{L^1(\partial\Omega)}$.
\end{lemma}

\begin{theorem}[Existenz einer Lösung]
  \label{thm:contProblemExistence}
  \Cref{prob:continuousProblem} besitzt eine Lösung \\$u\in\BV(\Omega)\cap
  L^2(\Omega)$.
\end{theorem}

\begin{proof}
  Die Beweisidee ist die Anwendung der direkten Methode der Variationsrechnung
  (cf.\ \cite{Dac89}) unter Nutzung der in \Cref{sec:bvFunctions}
  aufgeführten Eigenschaften der schwachen Konvergenz in $\BV(\Omega)$.

  Für alle $v\in L^2(\Omega)$ gilt mit der Hölderschen
  Ungleichung und der Youngschen Ungleichung aus \cref{lem:young}, dass
  \begin{equation}
    \label{eq:hoelderYoungL2ScalarProduct}
    \int_\Omega fv\dx
    \leq 
    \Vert f\Vert\Vert v\Vert
    \leq 
    \frac{1}{\alpha}\Vert f\Vert^2 + \frac{\alpha}{4}\Vert v\Vert^2.
  \end{equation}
  Die Höldersche Ungleichung impliziert außerdem für alle $v\in
  L^2(\Omega)\subseteq L^1(\Omega)$, dass
  \begin{equation}\label{eq:hoelderL2BiggerL1}
    \Vert v\Vert_{L^1(\Omega)} 
    = \Vert 1\cdot v\Vert_{L^1(\Omega)}
    \leq \Vert 1\Vert\Vert v\Vert
    =\sqrt{|\Omega|} \Vert v\Vert.
  \end{equation}
  Für das Funktional $E$ aus \eqref{eq:continuousProblem} folgt dann mit den
  Ungleichungen \eqref{eq:hoelderYoungL2ScalarProduct} und
  \eqref{eq:hoelderL2BiggerL1} für alle $v\in \BV(\Omega)\cap L^2(\Omega)$,
  dass
  \begin{equation}
    \label{eq:contProbBddFromBelow}
    \begin{aligned}
      E(v)
      &\geq 
      \frac{\alpha}{4}\Vert v\Vert^2 + |v|_{\BV(\Omega)}
      +\Vert v\Vert_{L^1(\partial\Omega)}-\frac{1}{\alpha}\Vert
      f\Vert^2\\
      &\geq 
      \frac{\alpha}{4|\Omega|}\Vert v\Vert_{L^1(\Omega)}^2 + |v|_{\BV(\Omega)}
      +\Vert v\Vert_{L^1(\partial\Omega)}-\frac{1}{\alpha}\Vert
      f\Vert^2
      \geq -\frac{1}{\alpha}\Vert f\Vert^2.
    \end{aligned}
  \end{equation}
  Somit ist $E$ nach unten beschränkt, was die Existenz einer infimierenden
  Folge $(u_n)_{n\in\Nbb}\subset\BV(\Omega)\cap L^2(\Omega)$ von $E$ 
  impliziert, das heißt,
  $(u_n)_{n\in\Nbb}$ erfüllt $$\lim_{n\rightarrow\infty}E(u_n) =
  \inf_{v\in\BV(\Omega)\cap L^2(\Omega)}E(v).$$ 
  Ungleichung \eqref{eq:contProbBddFromBelow} impliziert außerdem, dass
  $E(u_n)\to\infty$ für $n\to\infty$, falls $|u_n|_{\BV(\Omega)}\to\infty$ oder
  $\Vert u_n\Vert_{L^1(\Omega)}\to\infty$ für $n\to\infty$. 
  Daraus folgt insbesondere, dass $E(u_n)\to\infty$ für $n\to\infty$, falls
  $\Vert u_n\Vert_{\BV(\Omega)}\to\infty$ für $n\to\infty$ .
  Deshalb muss die Folge $(u_n)_{n\in\Nbb}$ beschränkt in $\BV(\Omega)$ sein.
  Nun garantiert \cref{thm:compactness} die Existenz einer in $\BV(\Omega)$
  schwach konvergenten Teilfolge $(u_{n_k})_{k\in\Nbb}$ von $(u_n)_{n\in\Nbb}$
  mit schwachem Grenzwert $u\in\BV(\Omega)$. 
  Ohne Beschränkung der Allgemeinheit sei
  $(u_{n_k})_{k\in\Nbb}=(u_n)_{n\in\Nbb}$.
  Aus der schwachen Konvergenz von $(u_n)_{n\in\Nbb}$ in $\BV(\Omega)$ gegen
  $u$ folgt nach Definition, dass $(u_n)_{n\in\Nbb}$ stark, und damit
  insbesondere auch schwach, in $L^1(\Omega)$ gegen $u$ konvergiert.

  Weiterhin folgt aus (\ref{eq:contProbBddFromBelow}), dass
  $E(v)\rightarrow\infty$ für $\Vert v\Vert\rightarrow\infty$. 
  Somit muss $(u_n)_{n\in\Nbb}$ auch beschränkt sein bezüglich der Norm
  $\Vert\bullet\Vert$ und besitzt deshalb, wegen der Reflexivität von
  $L^2(\Omega)$, eine Teilfolge (ohne Beschränkung der Allgemeinheit weiterhin
  bezeichnet mit $(u_n)_{n\in\Nbb}$), die in $L^2(\Omega)$ schwach gegen einen
  Grenzwert $\overline{u}\in L^2(\Omega)$ konvergiert. 
  Damit gilt für alle $w\in L^2(\Omega)\cong L^2(\Omega)^\ast$ und, da
  $L^\infty(\Omega)\subseteq L^2(\Omega)$, insbesondere auch für alle $w\in
  L^\infty(\Omega)\cong L^1(\Omega)^\ast$, dass 
  \begin{align*}
    \lim_{n\to\infty}\int_\Omega u_n w\dx =\int_\Omega \overline{u} w\dx.
  \end{align*}
  Das bedeutet, dass $(u_n)_{n\in\Nbb}$ auch schwach in $L^1(\Omega)$ gegen
  $\overline{u}\in L^2(\Omega)\subseteq L^1(\Omega)$ konvergiert. 
  Da schwache Grenzwerte eindeutig bestimmt sind, gilt insgesamt $u=\overline u
  \in L^2(\Omega)$, das heißt, $u\in\BV(\Omega)\cap L^2(\Omega)$.
  Nun definieren wir für alle
  $n\in\Nbb$ und für alle 
  $x\in\Rbb^d$ die Fortsetzungen
  \begin{align*}
    \tilde{u}_n(x)
    &\coloneqq
    \begin{cases}
      u_n(x),  &\text{ falls } x\in\Omega,\\
      0,     &\text{ falls } x\in\Rbb^d\setminus\overline\Omega
    \end{cases} 
    &&\text{und }
    &\tilde{u}(x)
    &\coloneqq
    \begin{cases}
      u(x),  &\text{ falls } x\in\Omega,\\
      0,     &\text{ falls } x\in\Rbb^d\setminus\overline\Omega.
    \end{cases} 
  \end{align*}
  Dann gilt nach \cref{lem:bvExtension} sowohl
  \begin{align*}
    \tilde{u}_n
    &\in
    \BV\!\big(\Rbb^d\big)
    &&\text{und}
    &\left|\tilde{u}_n\right|_{\BV\!\left(\Rbb^d\right)} 
    &= 
    |u_n|_{\BV(\Omega)}+\Vert u_n\Vert_{L^1(\partial\Omega)}
    \quad\text{für alle }n\in\Nbb \text{ als auch}\\
    \tilde{u}
    &\in
    \BV\!\big(\Rbb^d\big)
    &&\text{und}
    &\left|\tilde{u}\right|_{\BV\!\left(\Rbb^d\right)} 
    &=
    |u|_{\BV(\Omega)}+\Vert u\Vert_{L^1(\partial\Omega)}.
  \end{align*}
  Da $(u_n)_{n\in \Nbb}$ infimierende Folge von $E$ ist, muss die Folge
  \begin{align*}
    \left(\left|\tilde{u}_n\right|_{\BV\!\left(\Rbb^d\right)}\right)_{n\in\Nbb} 
    = \left(|u_n|_{\BV(\Omega)}+
    \Vert u_n\Vert_{L^1(\partial\Omega)}\right)_{n\in\Nbb}
  \end{align*}
  beschränkt sein.
  Außerdem folgt aus den Definitionen von $\tilde{u}$ und 
  $\tilde{u}_n$ für alle $n\in\Nbb$ und der bereits bekannten Eigenschaft 
  $u_n\to u$ in $L^1(\Omega)$ für $n\to\infty$, dass
  \begin{align*}
    \left\Vert \tilde{u}_n - \tilde{u}\right\Vert_{L^1\!\left(\Rbb^d\right)} 
    &= \int_{\Rbb^d} \left|\tilde{u}_n - \tilde{u}\right|\dx
    = \int_\Omega |u_n - u|\dx
    = \Vert u_n - u\Vert_{L^1(\Omega)}\to 0\quad\text{für }n\to\infty,
  \end{align*}
  das heißt, $\tilde{u}_n \to \tilde{u}$ in $L^1\left(\Rbb^d\right)$ für
  $n\to\infty$.
  Insgesamt ist also $\left(\tilde{u}_n\right)_{n\in\Nbb}$ eine Folge in
  $\BV\!\left(\Rbb^d\right)$, die in $L^1\!\left(\Rbb^d\right)$ gegen
  $\tilde{u}\in\BV\!\left(\Rbb^d\right)\subseteq L^1\!\left(\Rbb^d\right)$
  konvergiert und 
  $\sup_{n\in\Nbb} \left|\tilde{u}_n\right|_{\BV\!\left(\Rbb^d\right)}<\infty$
  erfüllt.
  Somit folgt mit
  \cref{thm:wlsc}  
  \begin{equation}
    \label{eq:wlscOfExtension}
    \begin{aligned}
      |u|_{\BV(\Omega)} +\Vert u\Vert_{L^1(\partial\Omega)}
      = \left|\tilde{u}\right|_{\BV\left(\Rbb^d\right)}
      &\leq\liminf_{n\to\infty}
      \left|\tilde{u}_n\right|_{\BV\left(\Rbb^d\right)}\\
      &= \liminf_{n\to\infty} \left(|u_n|_{\BV(\Omega)} +
      \Vert u_n\Vert_{L^1(\partial\Omega)}\right).
    \end{aligned}
  \end{equation}
  Die Funktionen $\Vert\bullet\Vert^2$ und $-\int_\Omega
  f\bullet\dx$ sind auf $L^2(\Omega)$ stetig und konvex, was impliziert,
  dass sie schwach unterhalbstetig auf $L^2(\Omega)$ sind. Da wir bereits
  wissen, dass $u_n\rightharpoonup u$ in $L^2(\Omega)$ für $n\to\infty$, 
  folgt
  \begin{align*}
    \frac{\alpha}{2}\Vert u\Vert-\int_\Omega fu\dx
    \leq \liminf_{n\to\infty}
    \left(\frac{\alpha}{2}\Vert u_n\Vert
    -\int_\Omega fu_n\dx\right).
  \end{align*}
  Damit und mit Ungleichung \eqref{eq:wlscOfExtension} gilt insgesamt
  \begin{align*}
    \inf_{v\in\BV(\Omega)\cap L^2(\Omega)}E(v)\leq
    E(u)\leq\liminf_{n\rightarrow\infty} E\left(u_n\right) =
    \lim_{n\rightarrow\infty}E\left(u_n\right) = \inf_{v\in\BV(\Omega)\cap
    L^2(\Omega)}E(v),
  \end{align*}
  das heißt, $\min_{v\in\BV(\Omega)\cap L^2(\Omega)} E(v) = E(u)$.
\end{proof}

Nachdem wir nun gezeigt haben, dass eine Lösung von
\Cref{prob:continuousProblem} existiert, können wir aus dem folgenden
Theorem unmittelbar die Eindeutigkeit dieser Lösung folgern. 

\begin{theorem}[Stabilität]
  \label{thm:contProbStabAndUniqu}
  Seien $u_1,u_2\in \BV(\Omega)\cap L^2(\Omega)$ die Lösungen von
  \Cref{prob:continuousProblem} mit $f_1,f_2\in L^2(\Omega)$ anstelle von $f$,
  das heißt, für $\ell\in\{1,2\}$ minimiere $u_\ell$ das Funktional
  \begin{align*}
    E_\ell(v)
    \coloneqq 
    \frac{\alpha}{2}\Vert v\Vert^2 + |v|_{\BV(\Omega)} 
    + \Vert v\Vert_{L^1(\partial\Omega)} - \int_\Omega f_\ell v\dx
  \end{align*}
  unter allen $v\in\BV(\Omega)\cap L^2(\Omega)$.
  Dann gilt 
  \begin{align*}
    \Vert u_1 - u_2\Vert 
    \leq\frac{1}{\alpha}\Vert f_1-f_2\Vert.
  \end{align*}
\end{theorem}

\begin{proof}
  Wir folgen der Argumentation im Beweis von \cite[S. 304, Theorem 10.6]{Bar15}.

  Zunächst definieren wir die Funktionale $F: L^2(\Omega)\to
  [0,\infty]$ und $G_\ell:L^2(\Omega)\to \Rbb$, $\ell\in\{1,2\}$, für
  alle $v\in L^2(\Omega)$ durch
  \begin{align*}
    F(v) 
    &\coloneqq 
    \begin{cases}
      |v|_{\BV(\Omega)} + \Vert v \Vert_{L^1(\partial\Omega)}, 
      &\text{ falls } v\in\BV(\Omega)\cap L^2(\Omega),\\
      \infty,&\text{ falls } v\in L^2(\Omega)\setminus\BV(\Omega)
    \end{cases}
    \quad\text{und }\\
    G_\ell(v)
    &\coloneqq 
    \frac{\alpha}{2}\Vert v\Vert^2 - \int_\Omega f_\ell v\dx.
  \end{align*}
  Damit gilt für $\ell\in\{1,2\}$ und alle
  $v\in\BV(\Omega)\cap L^2(\Omega)$, dass $E_\ell(v) =  F(v)+G_\ell(v)$.
  Für $\ell\in\{1,2\}$ ist $G_\ell$ Fr\'echet-differenzierbar und die
  Fr\'echet-Ableitung $G_\ell'(v): L^2(\Omega)\to\Rbb$ von $G_\ell$ an der
  Stelle $v\in L^2(\Omega)$ ist für alle $w\in L^2(\Omega)$
  gegeben durch
  \begin{align*}
    dG_\ell(v;w) = \alpha (v,w) - \int_\Omega f_\ell w\dx 
    = (\alpha v-f_\ell ,w).
  \end{align*}
  Das Funktional $F$ ist konvex, unterhalbstetig und es gilt $F\nequiv\infty$.
  Deshalb ist nach \cref{thm:subdifferentialMonotonicity} das Subdifferential
  $\partial F(\bullet)$ von $F$ monoton. 
  Damit gilt für alle $\mu_\ell\in
  \partial F(u_\ell)$, $\ell\in\{1,2\}$, dass
  \begin{align}\label{eq:stabilityAndUniqueness:monotonicityOfSubdifferential}
    (\mu_1-\mu_2,u_1-u_2)\geq 0.
  \end{align}
  Für $\ell\in\{1,2\}$ gilt, dass $E_\ell$ konvex ist und von $u_\ell$ in
  $\BV(\Omega)\cap L^2(\Omega)$ minimiert wird. 
  Außerdem gilt $E_\ell\nequiv\infty$ und $G_\ell$ ist stetig.
  % Somit gilt nach \cref{thm:extremalprinciple}, 
  % \cref{thm:subdifferentialSumRule} und \cref{thm:subdiffGateaux}, dass
  Somit gilt nach den Theoremen \ref{thm:extremalprinciple} --
  \ref{thm:subdifferentialSumRule}, dass
  \begin{align*}
    0\in\partial E_\ell(u_\ell) = \partial F(u_\ell)+\partial
    G_\ell(u_\ell)=\partial F(u_\ell)+ \{G_\ell'(u_\ell)\}.
  \end{align*}
  Daraus folgt
  $-G_\ell'(u_\ell)\in\partial F(u_\ell)$.
  Zusammen mit Ungleichung
  \eqref{eq:stabilityAndUniqueness:monotonicityOfSubdifferential}
  impliziert das
  \begin{align*}
    \big( -(\alpha u_1 - f_1) -(- (\alpha u_2 - f_2)), u_1 - u_2\big)
    \geq 0.
  \end{align*}
  Durch Umformen und Anwenden der Cauchy-Schwarzschen Ungleichung erhalten wir
  \begin{align*}
    \alpha \Vert u_1 - u_2 \Vert^2
    &\leq
    \big(f_1 -f_2, u_1-u_2 \big)
    \leq
    \Vert f_1-f_2\Vert\Vert u_1 - u_2\Vert.
  \end{align*}
  Falls $\Vert u_1 - u_2 \Vert = 0$, gilt die zu zeigende Aussage.
  Ansonsten führt die Division durch $\alpha\Vert u_1 - u_2 \Vert\neq 0$ den
  Beweis zum Abschluss.
\end{proof}

Zum Ende dieses Abschnitts beweisen wir, dass der Abstand einer Funktion
zu einem Minimierer des Funktionals $E$ durch die Differenz der Werte des
Funktionals kontrolliert wird.
Auch aus diesem Theorem folgt die Eindeutigkeit der nach
\Cref{thm:contProblemExistence} existierenden Lösung von
\Cref{prob:continuousProblem}.

\begin{theorem}
  \label{thm:convexity}
  Sei $u\in\BV(\Omega)\cap L^2(\Omega)$ Lösung von 
  \Cref{prob:continuousProblem}.
  Dann gilt 
  \begin{align*}
    \frac{\alpha}{2}\Vert u-v\Vert^2 \leq E(v)-E(u)\quad
    \text{für alle } v\in\BV(\Omega)\cap L^2(\Omega).
  \end{align*}
\end{theorem}

\begin{proof}
  Wir folgen der Argumentation im Beweis von \cite[S. 309, Lemma 10.2]{Bar15}.

  Wir betrachten die konvexen Funktionale
  $F:L^2(\Omega)\to [0,\infty]$ und $G:L^2(\Omega)\to \Rbb$, wobei 
  $F$ wie im Beweis von \Cref{thm:contProbStabAndUniqu} definiert ist und $G$
  für alle $v\in L^2(\Omega)$ gegeben ist durch 
  \begin{align*}
    G(v)\coloneqq \frac{\alpha}{2}\Vert v\Vert^2 - \int_\Omega f v\dx.
  \end{align*}
  Es gilt $E(v) = F(v)+G(v)$ für alle $v\in \BV(\Omega)\cap L^2(\Omega)$.
  Die Fr\'echet-Ableitung $G'(u): L^2(\Omega)\to\Rbb$ von $G$ an der Stelle
  $u\in \BV(\Omega)\cap L^2(\Omega)$ ist für alle $v\in L^2(\Omega)$ gegeben
  durch
  \begin{align*}
    dG(u;v) = \alpha (u,v) - \int_\Omega f v\dx 
    = (\alpha u-f ,v).
  \end{align*}
  Das impliziert mit wenigen Rechenschritten
  \begin{align}\label{eq:strongConvexityG}
    dG(u;v-u) +\frac{\alpha}{2}\Vert u-v\Vert^2+G(u) 
    =
    G(v)
    \quad\text{für alle } v\in L^2(\Omega).
  \end{align}
  Da $u$ der Minimierer von $E$ ist, erhalten wir mit den Theoremen
  %\ref{thm:extremalprinciple}, \ref{thm:subdifferentialSumRule} und
  %\ref{thm:subdiffGateaux} die Aussage
  \ref{thm:extremalprinciple} -- \ref{thm:subdifferentialSumRule} die Aussage
  \begin{align*}
    0\in\partial E(u) = \partial F(u)+\{G'(u)\},
  \end{align*}
  woraus folgt 
  $ -G'(u)\in\partial F(u).$
  Das ist nach \Cref{def:subdifferential} äquivalent zu
  \begin{align*}
    -dG(u;v-u)\leq F(v)-F(u)\quad\text{für alle }v\in\BV(\Omega)\cap
    L^2(\Omega).
  \end{align*}
  Für alle $v\in \BV(\Omega)\cap L^2(\Omega)$ folgt daraus, zusammen mit
  \Cref{eq:strongConvexityG}, dass
  \begin{align*}
    \frac{\alpha}{2}\Vert u-v\Vert^2+G(u)-G(v)+F(u)
    = -dG(u;v-u)+F(u)\leq F(v).
  \end{align*}
  Da $E=F+G$ auf $\BV(\Omega)\cap L^2(\Omega)$, impliziert das die zu zeigende
  Aussage.
\end{proof}


\section{Konstruktion eines Eingangssignals zu einer gegebenen Lösung}
\label{sec:constructionInputSignal}

Für die numerische Untersuchung der primalen-dualen Iteration aus
\Cref{chap:algorithm} ist es sinnvoll, Eingangssignale $f$ für
\Cref{prob:continuousProblem} gegeben zu haben, für die der entsprechende
gesuchte Minimierer bekannt ist. 
Die folgende Konstruktion solcher Signale basiert auf einer Aussage von
Professor Carstensen.

Sei $u:\Omega\to\Rbb$ gegeben als Funktion in Polarkoordinaten. 
Dabei beschränken wir uns auf vom Polarwinkel unabhängige Funktionen, das
heißt, für alle $x\in\Omega$ gelte
$u(x)\coloneqq u_P\big(|x|\big)$ für $u_P:[0,\infty)\to\Rbb$. 
Weiterhin fordern wir $u_P(r)=0$ für $r\geq 1$ und die Existenz
der partiellen Ableitung $\partial_r u_P$ fast überall in $[0,\infty)$.
Außerdem existiere fast überall in $[0,\infty)$ die partielle Ableitung des für
$r\in[0,\infty)$ mithilfe einer Funktion $q:\Rbb\to[0,1]$ definierten Ausdrucks
\begin{align*}
  \sgn\big(\partial_r u_P(r)\big)
  \coloneqq
  \begin{cases}
    -1 &\text{für }\partial_r u_P(r)<0,\\
    q(r) &\text{für }\partial_r u_P(r)=0,\\ 
    1 &\text{für }\partial_r u_P(r)>0.
  \end{cases}
\end{align*}
Des Weiteren fordern wir $\sgn\big(\partial_r u_P(r)\big)/r\to 0$ für $r\to 0$, 
damit $f_P$ in der folgenden Definition stetig in $0$ fortgesetzt werden kann.
Sei $f_P:[0,\infty)\to\Rbb$ mit $\alpha$ aus \Cref{prob:continuousProblem}
gegeben durch
\begin{align}
  \label{eq:constructionInputSignal}
  f_P(r)
  \coloneqq 
  \alpha u_P(r) - \partial_r\left(\sgn\big(\partial_r u_P(r)\big)\right) 
  - \frac{\sgn\big(\partial_r u_P(r)\big)}{r}
  \quad\text{für alle }r\in[0,\infty).
\end{align}
Dann ist $u$ Lösung von \Cref{prob:continuousProblem}, wenn das Eingangssignal
auf $\Omega\supseteq \left\{w\in\Rbb^2\,\middle|\, |w|\leq 1\right\}$ für fast
alle $x\in\Omega$ gegeben ist durch $f(x)\coloneqq f_P\big(|x|\big)$.

Für die Experimente in \Cref{chap:experiments} wollen wir die
garantierte untere Energieschranke aus \Cref{thm:gleb} berechnen können. 
Da dieses Theorem für das Eingangssignal voraussetzt, dass $f\in
H^1_0(\Omega)$, müssen wir noch die folgenden Bedingungen an $u_P$ formulieren.
Hinreichend für $f\in H^1_0(\Omega)$ ist nach \Cref{eq:constructionInputSignal},
dass  $u_P$, $\partial_r\big(\sgn(\partial_r u_P)\big)$ und $\sgn(\partial_r
u_P)$ stetig sind und 
\begin{align*}
  u_P(1)
  =
  \partial_r\left( \sgn\big(\partial_r u_P(1)\big)\right)
  =
  \sgn\big(\partial_r u_P(1)\big)
  =
  0.
\end{align*}
Mit diesen Einschränkung gilt insbesondere $u\in H^1_0(\Omega)$, weshalb 
die exakte Energie $E(u)$ nach \Cref{rem:bvSeminorm} berechnet werden kann
durch 
\begin{align*}
  E(u)
  =
  \frac{\alpha}{2}\Vert u\Vert^2 + \Vert u\Vert_{W^{1,1}(\Omega)} 
  - \int fu\dx.
\end{align*}
Um also $E(u)$ berechnen zu können, wird der schwache Gradient $\nabla u$ von
$u$ benötigt und um die garantierte untere Energieschranke $\Egleb$ aus
\Cref{eq:gleb} zu berechnen, wird der schwache Gradient $\nabla f$ von $f$
benötigt.
Deshalb betrachten wir an dieser Stelle noch kurz die benötigten Zusammenhänge
zwischen den partiellen Ableitungen in kartesischen Koordinaten und in
Polarkoordinaten für eine hinreichend glatte Funktion $g_P:\Rbb^2\to\Rbb$.
Für $x_1,x_2\in\Rbb$ sei
\begin{align*}
  \atan(x_2,x_1)\coloneqq
  \begin{cases}
    \arctan\left( \frac{x_2}{x_1} \right),& \text{wenn }x_1>0,\\
    \arctan\left( \frac{x_2}{x_1} \right) +\pi,& \text{wenn }x_1<0,x_2\geq 0,\\
    \arctan\left( \frac{x_2}{x_1} \right) -\pi,& \text{wenn }x_1<0,x_2<0,\\
    \frac{\pi}{2},& \text{wenn }x_1=0,x_2>0,\\
    -\frac{\pi}{2},& \text{wenn }x_1=0,x_2<0,\\
    \text{undefiniert},& \text{wenn }x_1=x_2=0.
  \end{cases}
\end{align*}
Ein Argument $x=(x_1,x_2)\in\Rbb^2$ von $g_P$ kann dann in Polarkoordinaten 
charakterisiert werden durch die Länge $r=\sqrt{x_1^2+x_2^2}$ und den Winkel
$\varphi = \atan(x_2,x_1)$.
Mit dieser Notation gelten für die partiellen Ableitungen die Zusammenhänge
\begin{align*}
  \partial_{x_1} &= 
  \cos(\varphi)\partial_r - \frac{1}{r}\sin(\varphi)\partial_\varphi
  &&\text{und}
  &\partial_{x_2} &= 
  \sin(\varphi)\partial_r - \frac{1}{r}\cos(\varphi)\partial_\varphi.
\end{align*}
Ist nun $g_P$ vom Winkel $\varphi$ unabhängig, so erhalten wir
\begin{align}
  \label{eq:gradInPolarCoordinates}
  \nabla g_P 
  = 
  \begin{pmatrix}
    \cos(\varphi)\\
    \sin(\varphi)
  \end{pmatrix}
  \partial_r g_P.
\end{align}
Unter Beachtung der trigonometrischen Zusammenhänge
\begin{align*}
  \sin\big(\arctan(y)\big) &= \frac{y}{\sqrt{1+y^2}} &&\text{und}
  &\cos\big(\arctan(y)\big) &= \frac{1}{\sqrt{1+y^2}}
\end{align*}
für alle $y\in\Rbb$ gilt
\begin{align*}
  \begin{pmatrix}
    \cos(\varphi)\\
    \sin(\varphi)
  \end{pmatrix}
  = 
  \frac{1}{r}
  \begin{pmatrix}
    x_1\\
    x_2
  \end{pmatrix}.
\end{align*}
Aus \Cref{eq:gradInPolarCoordinates} folgt damit
\begin{align*}
  \nabla g_P
  = 
  \frac{\partial_r g_P}{r}
  \begin{pmatrix}
    x_1\\
    x_2
  \end{pmatrix}.
\end{align*} 
Zum Bestimmen des Gradienten in kartesischen Koordinaten einer
in Polarkoordinaten gegegebenen Funktion $g_P$, die
vom Polarwinkel unabhängig ist, muss also lediglich 
die partielle Ableitung $\partial_r g_P$ berechnet werden.
Konkrete Beispiele formulieren wir in \Cref{chap:experiments}.


\chapter{Das diskrete Problem}
\label{chap:discreteProblem}
\section{Formulierung}
\label{sec:discreteProblemFormulation}
Bevor wir \Cref{prob:continuousProblem} diskretisieren, merken wir an,
dass $\CR^1(\Tcal)\subset\BV(\Omega)$, da
\begin{align*}
  |\vcr|_{\BV(\Omega)} 
  = 
  \Vert \gradnc \vcr\Vert_{L^1(\Omega)} 
  + \sum_{F\in\Ecal(\Omega)}\Vert[\vcr]_F\Vert_{L^1(F)}
  \quad\text{für alle }\vcr\in\CR^1(\Tcal).
\end{align*} 
Dies wird für $|\Tcal|=2$ zum Beispiel von \cites[S. 404, Example
10.2.1]{ABM14}[S. 301, Proposition 10.1]{Bar15} impliziert und kann
analog für beliebige reguläre Triangulierungen von $\Omega$ bewiesen
werden.
Damit gilt dann für alle $\vcr\in\CR^1(\Tcal)$ insbesondere
\begin{align*}
  |\vcr|_{\BV(\Omega)} +\Vert\vcr\Vert_{L^1(\partial\Omega)} 
  = \Vert \gradnc \vcr\Vert_{L^1(\Omega)} +
  \sum_{F\in\Ecal}\Vert[\vcr]_F\Vert_{L^1(F)}.
\end{align*}
Um eine nichtkonforme Formulierung von \Cref{prob:continuousProblem} zu 
erhalten, ersetzen wir die Terme 
$|\bullet|_{\BV(\Omega)} +\Vert\bullet\Vert_{L^1(\partial\Omega)}$ des
Funktionals $E$ durch 
$\Vert \gradnc \bullet\Vert_{L^1(\Omega)}$, das heißt wir vernachlässigen
bei der nichtkonformen Formulierung die Terme
$\sum_{F\in\Ecal}\Vert[\bullet]_F\Vert_{L^1(F)}$.
Somit erhalten wir das folgende Minimierungsproblem für den Parameter
$\alpha\in\Rbb_+$ und die rechte Seite $f\in
L^2(\Omega)$.

\begin{problem}\label{prob:discreteProblem}
  Finde $\ucr\in \CR^1_0(\Tcal)$,
  sodass $\ucr$ das Funktional
  \begin{align}\label{eq:discreteProblem}
    \Enc(\vcr)\coloneqq \frac{\alpha}{2}\Vert \vcr\Vert^2
    +\Vert \gradnc\vcr\Vert_{L^1(\Omega)}-\int_\Omega f\vcr\dx
  \end{align}
  unter allen $\vcr\in \CR^1_0(\Tcal)$ minimiert.
\end{problem}

\section{Charakterisierung und Existenz eines eindeutigen Minimierers}

In diesen Abschnitt führen wir die Argumente in \cite[S. 313]{Bar15}, angepasst
für unsere Formulierung in \Cref{prob:discreteProblem}, detailiert aus. 
Zunächst zeigen wir, dass \Cref{prob:discreteProblem} eine eindeutige Lösung
besitzt. Dafür benötigen wir folgendes Lemma.
\begin{lemma}
  \label{lem:normOfGradNcContiuous}
  Das Funktional $\Enc$ aus \Cref{eq:discreteProblem} ist stetig bezüglich der
  Konvergenz in $L^2(\Omega)$.
\end{lemma}

\begin{proof}
  Die Folge $(v_n)_{n\in\Nbb}\subset\CR^1_0(\Tcal)$ konvergiere
  gegen $\vcr\in\CR^1_0(\Tcal)$ bezüglich der Norm $\Vert\bullet\Vert$.
  Damit ist $(v_n)_{n\in\Nbb}$ insbesondere beschränkt in $L^2(\Omega)$ und es
  gilt mit einer binomischen Formel und der umgekehrten Dreiecksungleichung,
  dass
  \begin{align*}
    \left|\Vert\vcr\Vert^2-\Vert v_n\Vert^2\right|
    &=
    \big|\Vert\vcr\Vert-\Vert v_k\Vert\big|\, 
    \big|\Vert \vcr\Vert+\Vert v_k\Vert\big|\\
    &\leq
    \Vert\vcr- v_k\Vert\, \big|\Vert \vcr\Vert+\Vert v_k\Vert\big|
    \to 0\quad\text{für }n\to\infty.
  \end{align*}
  Außerdem gilt mit der Hölderschen Ungleichung
  \begin{align*}
    \left|\int_\Omega f(\vcr-v_k)\dx\right|
    \leq \Vert f\Vert \Vert\vcr-v_k\Vert\to 0\quad\text{für }n\to\infty.
  \end{align*}
  Schließlich gilt für alle $n\in\Nbb$ und alle $T\in\Tcal$ mit der 
  inversen Ungleichung (cf. \cite[S. 53, Lemma 3.5]{Bar15})
  mit Konstante $c_T\in\Rbb_+$ und der Hölderschen Ungleichung, dass
  \begin{equation*}
    \label{eq:continuityProofTriangleWiseEstimate}
    \Vert\gradnc(\vcr- v_n)\Vert_{L^1(T)}
    \leq
    c_T h_T^{-1}\Vert\vcr- v_n\Vert_{L^1(T)}
    \leq
    c_T h_T^{-1}\sqrt{|T|}\Vert\vcr- v_n\Vert_{L^2(T)}.
  \end{equation*}
  Damit folgt zusammen mit der umgekehrten Dreiecksungleichung
  \begin{align*}
    \left|\Vert\gradnc\vcr\Vert-\Vert \gradnc v_n\Vert\right|
    &\leq 
    \Vert\gradnc(\vcr- v_n)\Vert_{L^1(\Omega)}\\
    &=
    \sum_{T\in\Tcal}\Vert\gradnc(\vcr- v_n)\Vert_{L^1(T)}\\
    &\leq
    \max_{T\in\Tcal}\left(c_T h_T^{-1}\sqrt{|T|}\right)
    \sum_{T\in\Tcal}\Vert\vcr- v_n\Vert_{L^2(T)}\\
    &=
    \max_{T\in\Tcal}\left(c_T h_T^{-1}\sqrt{|T|}\right) \Vert\vcr- v_n\Vert
    \to 0\quad\text{für }n\to\infty.
  \end{align*}
  Somit ist $\Enc$ Summe von drei Termen, die bezüglich der Norm
  $\Vert\bullet\Vert$ folgenstetig sind, und deshalb stetig bezüglich der
  Konvergenz in $L^2(\Omega)$.
\end{proof}

\begin{theorem}
  \label{thm:discreteProblemExistenceUniqueness}
  Es existiert eine eindeutige Lösung $\ucr\in\CR^1_0(\Tcal)$ von
  \Cref{prob:discreteProblem}.
\end{theorem}

\begin{proof}
  Mit analogen Abschätzungen wie in \eqref{eq:contProbBddFromBelow}
  erhalten wir für das Funktional $\Enc$ aus \Cref{prob:discreteProblem} 
  für alle $\vcr\in\CR^1_0(\Tcal)\subset L^2(\Omega)$ die Ungleichung 
  \begin{equation}
    \label{eq:discreteEnergyCoercivity}
    \Enc(\vcr) 
    \geq 
    \frac{\alpha}{4}\Vert \vcr\Vert^2
    +\Vert \gradnc\vcr\Vert_{L^1(\Omega)}
    -\frac{1}{\alpha}\Vert f\Vert^2
    \geq 
    -\frac{1}{\alpha}\Vert f\Vert^2.
  \end{equation}
  Somit ist $\Enc$ nach unten beschränkt und es existiert eine infimierende
  Folge $(v_n)_{n\in\Nbb} \subset \CR^1_0(\Tcal)$ von $\Enc$. 
  Ungleichung \eqref{eq:discreteEnergyCoercivity} impliziert weiterhin, dass
  diese Folge beschränkt bezüglich der Norm $\Vert\bullet\Vert$ sein muss.
  Der endlichdimensionale Raum $\CR^1_0(\Tcal)$ ist, ausgestattet mit der Norm
  $\Vert\bullet\Vert$, ein Banachraum und damit reflexiv. 
  Demnach existiert eine in $\CR^1_0(\Tcal)$ schwach konvergente Teilfolge von
  $(v_n)_{n\in\Nbb}$.
  Da $\CR^1_0(\Tcal)$ endlichdimensional ist, konvergiert diese sogar stark
  in $L^2(\Omega)$. 
  Weil $\CR^1_0(\Tcal)$ ein Banachraum und damit abgeschlossen bezüglich der
  Konvergenz in $\Vert\bullet\Vert$ ist, gilt für den Grenzwert $\ucr$ dieser
  Teilfolge, dass $\ucr\in\CR^1_0(\Tcal)$.
  Nach \Cref{lem:normOfGradNcContiuous} ist $\Enc$ stetig bezüglich der
  Konvergenz in $L^2(\Omega)$, was impliziert, dass $\ucr$ Minimierer von
  $\Enc$ in $\CR^1_0(\Tcal)$ sein muss.   
  Dieser Minimierer $\ucr$ ist eindeutig, da $\Enc$ strikt konvex ist.
\end{proof}

Als nächstes wollen wir äquivalente Charakterisierungen der eindeutigen Lösung
von \Cref{prob:discreteProblem} beweisen, die von Professor Carstensen
formuliert wurden.
Dazu leiten wir zunächst ein zu \Cref{prob:discreteProblem} äquivalentes
Minimaxproblem nach \cite[Section 36]{Roc70} her.
Wir betrachten die konvexe Menge 
\begin{align*}
  K
  \coloneqq 
  \left\{\Lambda\in L^\infty\!\left(\Omega;\Rbb^2\right)
  \,\middle|\,|\Lambda(\bullet)| \leq 1 \text{ fast überall in }\Omega\right\}
\end{align*}
und das dazugehörige Indikatorfunktional
$I_K:L^\infty\!\left(\Omega;\Rbb^2\right)\to\Rbb\cup\{\infty\}$, das für
$\Lambda\in L^\infty\!\left(\Omega;\Rbb^2\right)$ gegeben ist durch
\begin{align*}
  I_K(\Lambda)
  &\coloneqq
  \begin{cases}
    \infty, & \text{falls } \Lambda\notin K,\\
    0,       & \text{falls } \Lambda\in K.
  \end{cases}
\end{align*} 
Aufgrund der Konvexität von $K$ ist $I_K$ konvex.
Für $\vcr\in\CR^1_0(\Tcal)$ und $\Lambda_0\in
P_0\!\left(\Tcal;\Rbb^2\right)\subset L^\infty\!\left(\Omega;\Rbb^2\right)$
können wir damit die Sattelfunktion $L:\CR^1_0(\Tcal)\times
P_0\!\left(\Tcal;\Rbb^2\right)\to [-\infty,\infty)$ nach \cite[Section
33]{Roc70} definieren durch
\begin{align}\label{eq:discreteProblemLagrangeFunctional}
  L(\vcr,\Lambda_0) \coloneqq \int_\Omega\Lambda_0\cdot\gradnc\vcr\dx +
  \frac{\alpha}{2}\Vert \vcr\Vert^2 -\int_\Omega f\vcr\dx
  - I_K(\Lambda_0).
\end{align}
Nun wählen wir $\vcr\in\CR^1_0(\Tcal)$ beliebig. 
Mit der Cauchy-Schwarzschen Ungleichung gilt für alle
$\Lambda_0\in P_0\!\left(\Tcal;\Rbb^2\right)\cap K$, dass
\begin{align*}
  \int_\Omega \Lambda_0\cdot\gradnc\vcr\dx
  \leq 
  \int_\Omega |\Lambda_0||\gradnc\vcr|\dx
  \leq 
  \Vert\gradnc\vcr\Vert_{L^1(\Omega)}.
\end{align*}
Daraus folgt
\begin{align}
  \label{eq:saddlepointLeqEnergy}
  \sup_{\Lambda_0\in P_0\left(\Tcal;\Rbb^2\right)\cap K}L(\vcr,\Lambda_0)
  \leq \Enc(\vcr).
\end{align}
Weiterhin gilt, wenn wir $\Lambda_0\in P_0\!\left(\Tcal;\Rbb^2\right)\cap K$
mit der Signumfunktion aus \Cref{eq:signumFunction} elementweise auf allen
$T\in\Tcal$ definieren durch $\Lambda_0(x)\in\sign\left(\gradnc\vcr(x)\right)$
für alle $x\in \interior(T)$, dass $L(\vcr,\Lambda_0)=\Enc(\vcr)$ und deshalb
auch
\begin{align}
  \label{eq:saddlepointGeqEnergy}
  \Enc(\vcr)
  \leq
  \sup_{\Lambda_0\in P_0\left(\Tcal;\Rbb^2\right)\cap K}L(\vcr,\Lambda_0).
\end{align}
Außerdem ist $L(\vcr,\Lambda_0)>-\infty$ genau dann, wenn $\Lambda_0\in K$.
Damit folgt aus den Ungleichungen \eqref{eq:saddlepointLeqEnergy} und
\eqref{eq:saddlepointGeqEnergy} insgesamt
\begin{equation*}
  \label{eq:discreteEnergySaddlefunctionalEquality}
  \Enc(\vcr)
  =\sup_{\Lambda_0\in P_0\left(\Tcal;\Rbb^2\right)\cap K}L(\vcr,\Lambda_0)
  =\sup_{\Lambda_0\in P_0\left(\Tcal;\Rbb^2\right)}L(\vcr,\Lambda_0).
\end{equation*}
Wenn also das folgende Minimaxproblem
\ref{prob:discreteSaddlepointProblem} eine Lösung $\left(
\tilde{u}_\CR,\bar\Lambda_0 \right)\in\CR^1_0(\Tcal)\times
P_0\!\left(\Tcal;\Rbb^2\right)$ hat, dann löst die Funktion $\tilde{u}_\CR$
\Cref{prob:discreteProblem}.

\begin{problem}\label{prob:discreteSaddlepointProblem}
  Finde $\left( \tilde{u}_\CR,\bar\Lambda_0 \right)\in\CR^1_0(\Tcal)\times
  P_0\!\left(\Tcal;\Rbb^2\right)$,
  sodass
  \begin{align*}
    L(\tilde{u}_\CR,\bar\Lambda_0) 
    = 
    \inf_{\vcr\in\CR^1_0(\Tcal)}\sup_{\Lambda_0\in P_0\left(\Tcal;\Rbb^2\right)}
    L(\vcr,\Lambda_0).
  \end{align*}
\end{problem}

\begin{lemma}
  \label{lem:existenceSaddlepoint}
  Es existiert eine Lösung $\left( \tilde{u}_\CR,\bar\Lambda_0
  \right)\in\CR^1_0(\Tcal)\times \left(P_0\!\left(\Tcal;\Rbb^2\right)\cap
  K\right)$ von \Cref{prob:discreteSaddlepointProblem}.
\end{lemma}

\begin{proof}
  Die Sattelfunktion $L$ aus \Cref{eq:discreteProblemLagrangeFunctional} ist in
  ihrer ersten Komponente eine konvexe, unterhalbstetige, auf $\CR^1_0(\Tcal)$
  reellwertige Funktion und
  in ihrer zweiten Komponente eine konkave, oberhalbstetige, auf
  $P_0\!\left(\Tcal;\Rbb^2\right)\cap K$ reellwertige Funktion.
  Somit ist $L$ in beiden Komponenten abgeschlossen 
  nach \cite[S. 52, 308]{Roc70}.  
  Insgesamt ist $L$ damit eine konvex-konkave, propere und abgeschlossene
  Funktion nach \cite[S. 349, 362 f.]{Roc70}, deren effektiver
  Definitionsbereich nach \cite[362]{Roc70} die Menge $\CR^1_0(\Tcal)\times
  \left( P_0\!\left(\Tcal;\Rbb^2\right)\cap K\right)$ ist.
  Unter Beachtung der Isomorphie von $\CR^1_0(\Tcal)$ zu
  $\Rbb^{|\Ecal(\Omega)|}$ und der Isomorphie von
  $P_0\!\left(\Tcal;\Rbb^2\right)$ zu $\Rbb^{2|\Tcal|}$, folgt aus 
  \cite[S. 397, Theorem 37.6]{Roc70} die Existenz eines Sattelpunkts
  $\left(\tilde{u}_\CR,\bar\Lambda_0\right)\in \CR^1_0(\Tcal)\times \left(
  P_0\!\left(\Tcal;\Rbb^2\right)\cap K\right)$ von $L$ nach \cite[380]{Roc70}.
  Für diesen impliziert \cite[S. 380, Lemma 36.2]{Roc70}, dass 
  \begin{align*}
    \sup_{\Lambda_0\in P_0\left(\Tcal;\Rbb^2\right)}\inf_{\vcr\in\CR^1_0(\Tcal)}
    L(\vcr,\Lambda_0)
    =
    L(\tilde{u}_\CR,\bar\Lambda_0) 
    = 
    \inf_{\vcr\in\CR^1_0(\Tcal)}\sup_{\Lambda_0\in P_0\left(\Tcal;\Rbb^2\right)}
    L(\vcr,\Lambda_0).
  \end{align*}
  Somit löst $\left(\tilde{u}_\CR,\bar\Lambda_0\right)\in \CR^1_0(\Tcal)\times
  \left( P_0\!\left(\Tcal;\Rbb^2\right)\cap K\right)$
  \Cref{prob:discreteSaddlepointProblem}.
\end{proof}

Nachdem diese Vorbereitungen abgeschlossen sind, können wir nun folgendes
Theorem beweisen.

\begin{theorem}
  \label{thm:discProbCharacterizationOfDiscreteSolutions}
  Für eine Funktion $\tilde{u}_\CR\in\CR^1_0(\Tcal)$ sind die folgenden drei
  Aussagen äquivalent.
  \begin{itemize}
    \item[(i)] \Cref{prob:discreteProblem} wird von $\tilde{u}_\CR$ gelöst.
    \item[(ii)] Es existiert ein
      $\bar\Lambda_0\in P_0\!\left(\Tcal;\Rbb^2\right)$ mit
      $\left|\bar\Lambda_0(\bullet)\right|\leq 1$
      fast überall in $\Omega$, sodass
      \begin{equation}
        \label{eq:discreteMultiplierScalerProductEquality}
        \bar\Lambda_0(\bullet)\cdot\gradnc\tilde{u}_\CR(\bullet)
        =
        \left|\gradnc\tilde{u}_\CR(\bullet)\right| 
        \quad\text{fast überall in } \Omega 
      \end{equation}
      und
      \begin{equation}
        \label{eq:discreteMultiplierL2Equality}
        \left(\bar\Lambda_0,\gradnc\vcr\right)
        = \left(f-\alpha\tilde{u}_\CR,
        \vcr\right)
        \quad\text{für alle } \vcr\in\CR^1_0(\Tcal).
      \end{equation}
    \item[(iii)] Für alle $\vcr\in\CR^1_0(\Tcal)$ gilt
      \begin{equation}
        \label{eq:discreteVariationalInequality}
        \left(f-\alpha\tilde{u}_\CR,\vcr-\tilde{u}_\CR\right)\leq
        \Vert\gradnc\vcr\Vert_{L^1(\Omega)} -
        \left\Vert\gradnc\tilde{u}_\CR\right\Vert_{L^1(\Omega)}.
      \end{equation}
  \end{itemize}
\end{theorem}

\begin{proof} 
  Sei $\tilde{u}_\CR\in\CR^1_0(\Tcal)$.

  \textit{(i) $\Rightarrow$ (ii).}
  Sei $\tilde{u}_\CR$ Lösung von \Cref{prob:discreteProblem}.
  Nach \Cref{lem:existenceSaddlepoint} existiert eine Lösung
  $\left(\hat{u}_\CR,\bar\Lambda_0\right)\in \CR^1_0(\Tcal)\times \left(
  P_0\!\left(\Tcal;\Rbb^2\right)\cap K\right)$ 
  von \Cref{prob:discreteSaddlepointProblem}. 
  Außerdem wissen wir, dass damit $\hat{u}_\CR$ Lösung von
  \Cref{prob:discreteProblem} ist.
  Daraus folgt, da nach \Cref{thm:discreteProblemExistenceUniqueness} die
  Lösung von \Cref{prob:discreteProblem} eindeutig ist, dass
  $\hat{u}_\CR=\tilde{u}_\CR$ in $\CR^1_0(\Tcal)$.
  Weiterhin wissen wir aus dem Beweis von \Cref{lem:existenceSaddlepoint}, dass
  $\left(\tilde{u}_\CR,\bar\Lambda_0\right)$ Sattelpunkt der Funktion $L$ aus
  \Cref{eq:discreteProblemLagrangeFunctional} ist.
  Das bedeutet nach \cite[380]{Roc70} insbesondere, dass $\tilde{u}_\CR$
  Minimierer von $L(\bullet, \bar\Lambda_0)$ in $\CR^1_0(\Tcal)$ und
  $\bar\Lambda_0$ Maximierer von $L\!\left(\tilde{u}_\CR,\bullet\right)$ über $
  P_0\!\left(\Tcal;\Rbb^2\right)$ ist.  
  Mit dieser Erkenntnis können wir nun die entsprechenden
  Optimalitätsbedingungen diskutieren.
  Zunächst gilt, da
  $L\!\left(\tilde{u}_\CR,\bullet\right):P_0\!\left(\Tcal;\Rbb^2\right)\to
  [-\infty,\infty)$ konkav und
  $\bar\Lambda_0$ Maximierer von $L\!\left(\tilde{u}_\CR,\bullet\right)$ in 
  $ P_0\!\left(\Tcal;\Rbb^2\right)$ ist, dass das konvexe Funktional
  $-L(\tilde{u}_\CR,\bullet):P_0\!\left(\Tcal;\Rbb^2\right)\to
  (-\infty,\infty]$ von $\bar\Lambda_0$ in $ P_0\!\left(\Tcal;\Rbb^2\right)$
  minimiert wird.
  %Nach den Theoremen \ref{thm:extremalprinciple},
  %\ref{thm:subdifferentialSumRule} und \ref{thm:subdiffGateaux} gilt somit
  Nach den Theoremen \ref{thm:extremalprinciple} --
  \ref{thm:subdifferentialSumRule} gilt somit
  \begin{align*}
    0
    \in 
    \partial \left(-L\!\left(\tilde{u}_\CR,\bullet\right)\right)
    \left(\bar\Lambda_0\right) 
    =
    \left\{-\!\left(\gradnc\tilde{u}_\CR,\bullet\right)\right\}+\partial I_K
    \left(\bar\Lambda_0\right).
  \end{align*}
  Äquivalent zu dieser Aussage ist, dass
  $\left(\gradnc\tilde{u}_\CR,\bullet\right)\in \partial
  I_K \left(\bar\Lambda_0\right)$. 
  Da $\bar\Lambda_0\in K$, folgt mit \Cref{def:subdifferential},
  dass für alle $\Lambda_0\in  P_0\!\left(\Tcal;\Rbb^2\right)$ gilt
  \begin{align*}
    \left(\gradnc\tilde{u}_\CR,\Lambda_0-\bar\Lambda_0\right) 
    \leq 
    I_K (\Lambda_0) - I_K\!\left(\bar\Lambda_0\right)
    =
    I_K (\Lambda_0).
  \end{align*}
  Falls $\Lambda_0\in  P_0\!\left(\Tcal;\Rbb^2\right)\cap K$, folgt insbesondere
  \begin{align}
    \label{eq:scalarProductInequDiscreteProof}
    \left(\gradnc\tilde{u}_\CR,\Lambda_0-\bar\Lambda_0\right) 
    &\leq 
    0,\quad\text{also }\notag\\ 
    \left(\gradnc\tilde{u}_\CR,\Lambda_0\right)
    &\leq
    \left(\gradnc\tilde{u}_\CR,\bar\Lambda_0\right).
  \end{align}
  Sei nun $\Lambda_0\in P_0\!\left(\Tcal;\Rbb^2\right)\cap K$ elementweise auf
  allen $T\in\Tcal$ durch $\Lambda_0(x)\in\sign\left(\gradnc\tilde{u}_\CR(x)\right)$
  definiert für alle $x\in\interior(T)$.
  Mit dieser Wahl von $\Lambda_0$, Ungleichung
  \eqref{eq:scalarProductInequDiscreteProof}, der Cauchy\--Schwarz\-schen
  Ungleichung und $\bar\Lambda_0\in K$ erhalten wir die Abschätzung
  \begin{align}
    \label{eq:sumOverAllTrianglesDualVariable}
    \int_\Omega\left|\gradnc\tilde{u}_\CR\right|\dx
    &=
    \int_\Omega\gradnc\tilde{u}_\CR\cdot\Lambda_0\dx
    \leq 
    \int_\Omega\gradnc\tilde{u}_\CR\cdot\bar\Lambda_0\dx \notag\\
    &\leq 
    \int_\Omega\left|\gradnc\tilde{u}_\CR\right|\left|\bar\Lambda_0\right|\dx
    \leq
    \int_\Omega\left|\gradnc\tilde{u}_\CR\right|\dx,
    \quad\text{das heißt }\notag\\
    \int_\Omega\left|\gradnc\tilde{u}_\CR\right|\dx 
    &= 
    \int_\Omega\gradnc\tilde{u}_\CR\cdot\bar\Lambda_0\dx
    \quad\text{beziehungsweise }\notag\\
    \sum_{T\in\Tcal}|T|\,\big|(\gradnc\tilde{u}_\CR)\!|_T\big|
    &=
    \sum_{T\in\Tcal}|T|\left(\gradnc\tilde{u}_\CR\cdot \bar\Lambda_0\right)\!\!|_T.
  \end{align}
  Außerdem gilt für alle $T\in\Tcal$ mit der Cauchy-Schwarzschen Ungleichung
  und $\bar\Lambda_0\in K$, dass 
  \begin{align*}
    \left(\gradnc\tilde{u}_\CR\cdot \bar\Lambda_0\right)\!\!|_T
  \leq
  \big|(\gradnc\tilde{u}_\CR)\!|_{T}\big|\,\left|\bar\Lambda_0|_T\right|
  \leq
  \big|(\gradnc\tilde{u}_\CR)\!|_{T}\big|.
  \end{align*}
  Mit \Cref{eq:sumOverAllTrianglesDualVariable} folgt daraus für alle
  $T\in\Tcal$, dass $\left(\gradnc\tilde{u}_\CR\cdot
  \bar\Lambda_0\right)\!\!|_T=\big|(\gradnc\tilde{u}_\CR)\!|_T\big|$, das heißt fast
  überall in $\Omega$ gilt $\bar\Lambda_0(\bullet)\cdot\gradnc\tilde{u}_\CR(\bullet)
  =|\gradnc\tilde{u}_\CR(\bullet)|$. 
  Damit ist \Cref{eq:discreteMultiplierScalerProductEquality} gezeigt.
  Als Nächstes betrachten wir das reellwertige Funktional
  $L\left(\bullet,\bar\Lambda_0\right):\CR^1_0(\Tcal)\to\Rbb$.
  Es ist Fr\'echet-differenzierbar mit
  \begin{align*}
    dL\!\left(\bullet,\bar\Lambda_0\right)\!\left(\tilde{u}_\CR;\vcr\right)
    =
    \int_\Omega\bar\Lambda_0\cdot \gradnc\vcr\dx
    +\alpha\! \left(\tilde{u}_\CR,\vcr\right) - \int_\Omega f\vcr\dx
  \end{align*}
  für alle $\vcr\in\CR^1_0(\Tcal)$.
  Da $\tilde{u}_\CR$ Minimierer von  $L\!\left(\bullet, \bar\Lambda_0\right)$
  in $\CR^1_0(\Tcal)$ ist, gilt nach
  \Cref{thm:necessaryConditionFreeLocalExtrema}, dass $0 =
  dL\!\left(\bullet,\bar\Lambda_0\right)\!\left(\tilde{u}_\CR;\vcr\right)$ für
  alle $\vcr\in\CR^1_0(\Tcal)$.
  Diese Bedingung ist für alle $\vcr\in\CR^1_0(\Tcal)$ äquivalent zu
  $\left(\bar\Lambda_0,\gradnc\vcr\right) = (f-\alpha \tilde{u}_\CR,\vcr)$.
  Somit ist auch \Cref{eq:discreteMultiplierL2Equality} gezeigt.

  \textit{(ii) $\Rightarrow$ (iii).}
  Die Funktion $\bar\Lambda_0\in P_0\!\left(\Tcal;\Rbb^2\right)$ erfülle
  $\left|\bar\Lambda_0(\bullet)\right|\leq 1$ fast überall in $\Omega$ sowie
  die Gleichungen \eqref{eq:discreteMultiplierScalerProductEquality} und 
  \eqref{eq:discreteMultiplierL2Equality}. 
  Sei $\vcr\in\CR^1_0(\Tcal)$.
  Mit den Gleichungen 
  \eqref{eq:discreteMultiplierL2Equality} und 
  \eqref{eq:discreteMultiplierScalerProductEquality} gilt
  \begin{equation}
    \label{eq:equivalentCharacterizationApplicationTwoEquations}
    \begin{aligned}
      \left(f-\alpha\tilde{u}_\CR,\vcr-\tilde{u}_\CR\right) 
      &=
      \left(\bar\Lambda_0,\gradnc\vcr\right)
      - \left(\bar\Lambda_0,\gradnc\tilde{u}_\CR\right)\\
      &=
      \int_\Omega\bar\Lambda_0\cdot\gradnc\vcr\dx
      - \int_\Omega\left|\gradnc\tilde{u}_\CR\right|\dx.
    \end{aligned}
  \end{equation}
  Weiterhin gilt mit der Cauchy-Schwarzschen
  Ungleichung und $\left|\bar\Lambda_0(\bullet)\right|\leq 1$ fast überall in
  $\Omega$, dass
  \begin{align*}
    \int_\Omega\bar\Lambda_0\cdot\gradnc\vcr\dx
    &\leq 
    \int_\Omega\left|\bar\Lambda_0\right|\,|\gradnc\vcr|\dx
    \leq 
    \int_\Omega|\gradnc\vcr|\dx.
  \end{align*}
  Zusammen mit \Cref{eq:equivalentCharacterizationApplicationTwoEquations}
  folgt daraus Ungleichung \eqref{eq:discreteVariationalInequality}.

  \textit{(iii) $\Rightarrow$ (i)}.
  Es gelte Ungleichung \eqref{eq:discreteVariationalInequality} für alle
  $\vcr\in\CR^1_0(\Tcal)$, also
  \begin{align*}
    \left(f-\alpha\tilde{u}_\CR,\vcr-\tilde{u}_\CR\right) 
    \leq
    \left\Vert\gradnc\vcr\right\Vert_{L^1(\Omega)}
    -\left\Vert\gradnc\tilde{u}_\CR\right\Vert_{L^1(\Omega)}.
  \end{align*}
  Nach \Cref{thm:discreteProblemExistenceUniqueness} exisitert eine eindeutige
  Lösung $\ucr\in\CR^1_0(\Tcal)$ von \Cref{prob:discreteProblem}.
  Wir haben bereits gezeigt, dass somit für alle
  $\vcr\in\CR^1_0(\Tcal)$ gilt
  \begin{align*}
    \left(f-\alpha\ucr,\vcr-\ucr\right) 
    \leq
    \left\Vert\gradnc\vcr\right\Vert_{L^1(\Omega)}
    -\Vert\gradnc\ucr\Vert_{L^1(\Omega)}.
  \end{align*}
  Um nun zu beweisen, dass $\tilde{u}_\CR$ \Cref{prob:discreteProblem} löst, genügt
  es $\tilde{u}_\CR=\ucr$ in $\CR^1_0(\Tcal)$ zu zeigen.
  Es gilt
  \begin{align*}
    \left(f-\alpha\ucr,\tilde{u}_\CR-\ucr\right) 
    &\leq
    \left\Vert\gradnc\tilde{u}_\CR\right\Vert_{L^1(\Omega)}
    -\Vert\gradnc\ucr\Vert_{L^1(\Omega)}\quad\text{und }\\
    \left(f-\alpha\tilde{u}_\CR,\ucr-\tilde{u}_\CR\right) 
    &\leq
    \left\Vert\gradnc\ucr\right\Vert_{L^1(\Omega)}
    -\Vert\gradnc\tilde{u}_\CR\Vert_{L^1(\Omega)}. 
  \end{align*}
  Die Addition dieser Ungleichungen
  impliziert
  \begin{align*}
    \alpha\left\Vert\tilde{u}_\CR-\ucr\right\Vert^2=
    \left(-\alpha\ucr,\tilde{u}_\CR-\ucr\right) 
    + \left(-\alpha\tilde{u}_\CR,\ucr-\tilde{u}_\CR\right) 
    \leq
    0.
  \end{align*}
  Da $\alpha>0$, folgt daraus $\left\Vert\tilde{u}_\CR-\ucr\right\Vert^2=0$,
  also $\tilde{u}_\CR=\ucr$ in $\CR^1_0(\Tcal)$.
\end{proof}

Zum Schluss dieses Abschnitts wollen wir noch zwei Bemerkungen von Professor
Carstensen erwähnen und kurz deren Gültigkeit begründen.
Die erste ist in eine äquivalente Charakterisierung der dualen Variable
$\bar\Lambda_0\in P_0\!\left(\Tcal;\Rbb^2\right)$ aus
\Cref{thm:discProbCharacterizationOfDiscreteSolutions} zur diskreten Lösung
$\ucr\in\CR^1_0(\Tcal)$ von \Cref{prob:discreteProblem}.

\begin{remark}
  Dass $\bar\Lambda_0\in P_0\!\left(\Tcal;\Rbb^2\right)$ fast überall in $\Omega$
  \Cref{eq:discreteMultiplierScalerProductEquality} und
  $|\bar\Lambda_0(\bullet)|\leq 1$ erfüllt, ist äquivalent zu der Bedingung
  $\bar\Lambda_0(x)\in\sign(\gradnc \ucr(x))$ für alle $x\in\interior(T)$ für
  alle $T\in\Tcal$.   
\end{remark}

\begin{proof}
  Dass die genannte Bedingung hinreichend ist, folgt direkt aus der Definition
  der Signumfunktion.
  Ihre Notwendigkeit folgt aus der folgenden Beobachtung.
  Da $\left|\bar\Lambda_0(\bullet)\right|\leq 1$ fast überall in $\Omega$, ist
  \Cref{eq:discreteMultiplierScalerProductEquality} eine Cauchy-Schwarzsche
  Ungleichung, bei der sogar Gleichheit gilt. 
  Dies ist genau dann der Fall, wenn $\bar\Lambda_0(\bullet)$ und
  $\gradnc\ucr(\bullet)$ fast überall in $\Omega$ linear abhängig sind.
\end{proof}
 
Daraus können wir folgern, unter welchen Umständen die duale Variable
$\bar\Lambda_0$ auf einem Dreieck $T\in\Tcal$ eindeutig bestimmt ist.

\begin{remark}
  Falls $\gradnc\ucr\neq 0$ auf $T\in\Tcal$, gilt nach Definition der
  Signumfunktion, dass $\bar\Lambda_0=\gradnc\ucr/|\gradnc\ucr|$ eindeutig
  bestimmt ist auf $T$.
  Im Allgemeinen ist $\bar\Lambda_0$ allerdings nicht eindeutig bestimmbar. 
  Betrachten wir zum Beispiel $f\equiv 0$ in \Cref{prob:discreteProblem} mit
  eindeutiger Lösung $\ucr\equiv 0$ fast überall in $\Omega$. 
  Dann erfüllt nach der diskreten Helmholtz Zerlegung \cite[S. 193, Theorem
  3.32]{Car09b} die Wahl $\bar\Lambda_0\coloneqq \Curl v_\C$ für ein beliebiges
  $v_\C\in S^1(\Tcal)$ mit $|\Curl v_\C|\leq 1$ die Charakterisierung
  \textit{(ii)} aus \Cref{thm:discProbCharacterizationOfDiscreteSolutions}.
\end{remark}

\section{Verfeinerungsindikator und garantierte untere Energieschranke}
\todo[inline]{vielleicht: Verfeinerungsindikator und garantierte 
Energieschranken/ garantierte unter und obere Energieschranke}

Professor Carstensen stellte für die numerischen Untersuchungen 
einen Verfeinerungsindikator zur adaptiven Netzverfeinerung und eine 
Aussage über eine garantierte untere Energieschranke zur Verfügung.

\begin{theorem}
  \label{thm:gleb}
  \todo[inline]{,,Mit minimaler Energie \ldots`` rausnehmen? Das ist halt
  wirklich überflüssig als Info}
  Sei $\Omega$ konvex, $f\in H^1_0(\Omega)$ das Eingangssignal für
  \Cref{prob:continuousProblem} mit Lösung $u\in H^1_0(\Omega)$ und minimaler
  Energie $E(u)$ sowie für \Cref{prob:discreteProblem} mit Lösung $\ucr\in
  \CR^1_0(\Omega)$ und minimaler Energie $\Enc(\ucr)$.
  Dann gilt
  \begin{align*}
    \Enc(\ucr)+\frac{\alpha}{2}\Vert u-\ucr\Vert^2
    -\frac{\kappa_\CR}{\alpha}\Vert
    h_\Tcal(f-\alpha\ucr)\Vert \Vert\nabla f\Vert\leq E(u).
  \end{align*}
  Dabei ist die Konstante $\kappa_\CR\coloneqq\sqrt{1/48+1/j_{1,1}^2}$ mit der
  kleinsten positiven Nullstelle $j_{1,1}$ der Bessel-Funktion erster Art.
  Insbesondere gilt dann für 
  \begin{align}
    \label{eq:gleb}
    \Egleb 
    \coloneqq 
    \Enc(\ucr) - \frac{\kappa_\CR}{\alpha}\Vert h_\Tcal(f-\alpha\ucr)\Vert
    \Vert \nabla f\Vert,
  \end{align}
    dass $\Enc(\ucr)\geq \Egleb$ und $E(u)\geq \Egleb$.
\end{theorem}

\begin{definition}[Verfeinerungsindikator]
  \label{def:refinementIndicator}
  Für $d\in\mathbb{N}$ (in dieser Arbeit stets $d=2$) und $0<\gamma\leq 1$
  definieren wir für alle $T\in\Tcal$ und $\ucr\in\CR^1_0(\Tcal)$ die
  Funktionen
  \begin{align*}
    \eta_\text{V}(T)
    &\coloneqq
    |T|^{2/d}\Vert f-\alpha \ucr\Vert^2_{L^2(T)}\quad\text{und }\\
    \eta_\text{J}(T)
    &\coloneqq
    |T|^{\gamma/d}\sum_{F\in\Ecal(T)}\left\Vert [\ucr]_F\right\Vert_{L^1(F)}.
  \end{align*} 
  Damit definieren wir den Verfeinerungsindikator
  $\eta\coloneqq\sum_{T\in\Tcal}\eta(T)$, wobei
  \begin{align} \label{eq:refinementIndicator} 
    \eta (T)
    \coloneqq
    \eta_\text{V}(T) + \eta_\text{J}(T)\quad\text{für alle } T\in\Tcal.
  \end{align} 
\end{definition}

\section{Kontrolle des Abstandes zwischen diskreter und kontinuierlicher
Lösung}
\todo[inline]{section womöglich rausnehmen und das alles zwischen EGLEB Thm
und Verfeinerungsind Def schieben.}
Wir möchten in diesen Abschnitt eine Abschätzungen herleiten, mit welcher
der $L^2$-Abstand zwischen der Lösung des diskreten Problems
\ref{prob:discreteProblem} und der Lösung des kontinuierlichen Problems
\ref{prob:continuousProblem} kontrolliert werden kann.
Dafür benötigen wir zunächst folgendes Theorem.
\todo[inline]{Dann wohl auch den Text hier ändern und sagen, dass es als
Vorbereitung für die GUEB gebraucht wird}

\todo[inline]{definitiv reinnehmen, aber vielleicht nur als Bemerkung und 
für Beweis auf Bartels verweisen}
\begin{theorem}[Eher ein Lemma draus machen?]
  \label{thm:convexity}
  Sei $u\in\BV(\Omega)\cap L^2(\Omega)$ Lösung von 
  \Cref{prob:continuousProblem}.
  Dann gilt 
  \begin{align*}
    \frac{\alpha}{2}\Vert u-v\Vert^2 \leq E(v)-E(u)\quad
    \text{für alle } v\in\BV(\Omega)\cap L^2(\Omega).
  \end{align*}
\end{theorem}

\begin{proof}
  Wir folgen der Argumentation im Beweis von \cite[S. 309, Lemma 10.2]{Bar15}.
  Da viele der Schritte ähnlich zum Beweis von \Cref{thm:contProbStabAndUniqu}
  sind, präsentieren wir die entsprechenden Argumente verkürzt.

  Wir definieren die konvexen Funktionale
  $F:L^2(\Omega)\to \Rbb\cup\{\infty\}$ und $G:L^2(\Omega)\to \Rbb$, wobei 
  $F$ wie im Beweis von \Cref{thm:contProbStabAndUniqu} definiert ist und $G$
  für alle $v\in L^2(\Omega)$ gegeben ist durch 
  \begin{align*}
    G(v)\coloneqq \frac{\alpha}{2}\Vert v\Vert^2 - \int_\Omega f v\dx.
  \end{align*}
  Es gilt $E = F+G$.
  Die Fr\'echet-Ableitung $G'(u): L^2(\Omega)\to\Rbb$ von $G$ an der Stelle
  $u\in \BV(\Omega)\cap L^2(\Omega)$ ist für alle $v\in L^2(\Omega)$ gegeben
  durch
  \begin{align*}
    dG(u;v) = \alpha (u,v) - \int_\Omega f v\dx 
    = (\alpha u-f ,v).
  \end{align*}
  Das impliziert mit wenigen Rechenschritten
  \begin{align}\label{eq:strongConvexityG}
    dG(u;v-u) +\frac{\alpha}{2}\Vert u-v\Vert^2+G(u) 
    =
    G(v)
    \quad\text{für alle } v\in L^2(\Omega).
  \end{align}
  Da $u$ der Minimierer von $E$ ist, erhalten wir mit den Theoremen
  %\ref{thm:extremalprinciple}, \ref{thm:subdifferentialSumRule} und
  %\ref{thm:subdiffGateaux} die Aussage
  \ref{thm:extremalprinciple} -- \ref{thm:subdifferentialSumRule} die Aussage
  \begin{align*}
    0\in\partial E(u) = \partial F(u)+\{G'(u)\},
  \end{align*}
  woraus folgt 
  $ -G'(u)\in\partial F(u).$
  Das ist nach \Cref{def:subdifferential} äquivalent zu
  \begin{align*}
    -dG(u;v-u)\leq F(v)-F(u)\quad\text{für alle }v\in\BV(\Omega)\cap
    L^2(\Omega).
  \end{align*}
  Daraus folgt zusammen mit \Cref{eq:strongConvexityG} für alle $v\in
  \BV(\Omega)\cap L^2(\Omega)$, dass
  \begin{align*}
    \frac{\alpha}{2}\Vert u-v\Vert^2+G(u)-G(v)+F(u)
    = -dG(u;v-u)+F(u)\leq F(v).
  \end{align*}
  Da $E=F+G$, folgt daraus die zu zeigende Aussage.
\end{proof}

\begin{theorem}[Garantierte obere Energieschranke]
  \label{thm:gueb}
  \todo[inline]{Write it, aber Tien noch fragen stellen}
  Sei $\Omega$ konvex, $f\in H^1_0(\Omega)$ das Eingangssignal für
  \Cref{prob:continuousProblem} mit Lösung $u\in H^1_0(\Omega)$ und minimaler
  Energie $E(u)$ sowie für \Cref{prob:discreteProblem} mit Lösung $\ucr\in
  \CR^1_0(\Omega)$ und minimaler Energie $\Enc(\ucr)$.
  Dann gilt
  \begin{align*}
    \Enc(\ucr)+\frac{\alpha}{2}\Vert u-\ucr\Vert^2
    -\frac{\kappa_\CR}{\alpha}\Vert
    h_\Tcal(f-\alpha\ucr)\Vert \Vert\nabla f\Vert\leq E(u).
  \end{align*}
  Dabei ist die Konstante $\kappa_\CR\coloneqq\sqrt{1/48+1/j_{1,1}^2}$ mit der
  kleinsten positiven Nullstelle $j_{1,1}$ der Bessel-Funktion erster Art.
  Insbesondere gilt dann für 
  \begin{align}
    \label{eq:gleb}
    \Egleb 
    \coloneqq 
    \Enc(\ucr) - \frac{\kappa_\CR}{\alpha}\Vert h_\Tcal(f-\alpha\ucr)\Vert
    \Vert \nabla f\Vert,
  \end{align}
    dass $\Enc(\ucr)\geq \Egleb$ und $E(u)\geq \Egleb$.

\end{theorem}

Zusammen mit den Betrachtungen in \Cref{sec:discreteProblemFormulation} 
erhalten wir damit die folgende Abschätzung.

\begin{corollary}[TODO löschen]
  Sei $u\in\BV(\Omega)\cap L^2(\Omega)$ Lösung von
  \Cref{prob:continuousProblem} und $\ucr\in\CR^1_0(\Tcal)$ Lösung von
  \Cref{prob:discreteProblem}.
  Dann gilt
  \begin{align*}
    \frac{\alpha}{2}\Vert u-\ucr\Vert^2\leq
    E(\ucr)-E(u)=\Enc(\ucr)+\sum_{F\in\Ecal}\Vert[\ucr]_F\Vert_{L^1(F)}-E(u).
  \end{align*}
  Insbesondere gilt auch 
  \begin{align*}
    \frac{\alpha}{2}\Vert u-\ucr\Vert^2
    \leq
    \left|\Enc(\ucr)-E(u)\right|+
    \left|\sum_{F\in\Ecal}\Vert[\ucr]_F\Vert_{L^1(F)}\right|.
  \end{align*}
\end{corollary}

\todo[inline]{Hier Fragen und Formeln}
Von Theorem 4.9:
  \begin{align*}
    \Enc(\ucr)+\frac{\alpha}{2}\Vert u-\ucr\Vert^2
    -\frac{\kappa_\CR}{\alpha}\Vert
    h_\Tcal(f-\alpha\ucr)\Vert \Vert\nabla f\Vert\leq E(u).
  \end{align*}
  impliziert für
  \begin{align*}
    \Egleb 
    \coloneqq 
    \Enc(\ucr) - \frac{\kappa_\CR}{\alpha}\Vert h_\Tcal(f-\alpha\ucr)\Vert
    \Vert \nabla f\Vert,
  \end{align*}
    dass $\Enc(\ucr)\geq \Egleb$ und $E(u)\geq \Egleb$. 

      deshalb garantierte untere Energieschranke?

      Falls ja, ist dann noch eine Abschätzung $E(u)\leq E_\textup{gueb}$ gewollt?
      Diese bekommen wir nicht so wirklich aus Thm. 4.11
      Es gilt natürlich z.B.
      $E(u)\leq E(J_1\ucr)=\Enc(J_1\ucr)$. Bringt das irgendwas? Hat nur
      leider keine h Potenz dabei.

      Weitere von Thm 4.9 implizierte Aussagen, die nützlich aussehen:

      \begin{align*}
        \Vert u-\ucr\Vert\leq 
        \sqrt{\frac{\alpha}{2}}\sqrt{E(u)-\Egleb}
        \lesssim
        \sqrt{E(u)-\Egleb}
      \end{align*}

      diese rechte Seite, ohne die Konstante, wurde geplottet (quadriert)
      und war dann parallel. Also gilt in den Experimenten sogar mehr. 
      Parallelität gilt aber damit nicht. (Bild zeigen, auch das angepasste
      mit sqrt)

      Diese Aussage ist keine A-priori Abschätzung, weil die rechte
      Seite immer noch von $\ucr$ abhängt, oder?
      Bringt der plot dieser RHS dann noch irgendwas
      

      Theorem 4.11 liefert
  \begin{align*}
    \frac{\alpha}{2}\Vert u-v\Vert^2 \leq E(v)-E(u)\quad
    \text{für alle } v\in\BV(\Omega)\cap L^2(\Omega).
  \end{align*}

  Daraus folgt
  \begin{align*}
    \Vert u-\ucr\Vert &\leq
    \sqrt{2/\alpha}\sqrt{\Enc(J_1{\ucr})-\Egleb} + \Vert J_1\ucr-\ucr\Vert\\
    &\lesssim
    \sqrt{\Enc(J_1{\ucr})-\Egleb} + \Vert J_1\ucr-\ucr\Vert
  \end{align*}

  Die unterscheidet sich dann aber nicht mehr von der anderen Aussage, die auch 
  eine obere Schranke für $\Vert u-\ucr\Vert$ war.
  Ich hatte mit einer unteren Schranke gerechnet, weil CC was von ,,zwischen
  zwei Graphen`` liegen meinte.
  ACHSO, die hier ist natürlich besser, weil die auch ohne Kentniss der exakten
  Lösung berechnet werden kann (A-posteriori, korrekt?). Den Gradienten von $f$
  braucht man allerdings noch, also kann man das für Cameraman immer noch nicht
  nutzen.

  Außerdem impliziert Theorem 4.11
  \begin{align*}
    E(u)\leq E(J_1 \ucr)-\frac{\alpha}{2}\Vert u - J_1 \ucr\Vert^2 
    = \Enc(J_1 \ucr)-\frac{\alpha}{2}\Vert u - J_1 \ucr\Vert^2 
  \end{align*}
  Bringt das was als GUEB oder zählt das nicht, da die rechte Seite immernoch von
  $u$ abhängt.



\chapter{Iterative Lösung}
\label{chap:algorithm}
\section{Primale-duale Iteration}

In diesem Abschnitt präsentieren wir ein iteratives Verfahren mit dem wir
\Cref{prob:discreteProblem} numerisch lösen möchten. 
Dieses basiert auf der primalen-dualen Iteration \cite[S. 314, Algorithm
10.1]{Bar15} unter Beachtung von \cite[S. 314, Remark 10.11]{Bar15}. 
%Diese realisiert Gradientenverfahren zum Finden eines Sattelpunkts, dessen 
%Komponenten die primale und die duale Formulierung des Minimierungsproblems
%lösen. 
Details dazu und weitere Referenzen finden sich in \cites{Bar12}[S.
118-121]{Bar15}.
Angepasst an unser Problem und die Notation dieser Arbeit lautet der
Algorithmus wie folgt.

\begin{algorithm}[Primale-duale Iteration]
  \label{alg:primalDualIteration}
\begin{algorithmic}\\
  \Require $\left(u_0,\Lambda_0\right)
  \in\textup{CR}_0^1(\mathcal{T})\times P_0\!\left(\mathcal{T}; 
  %\left\{w\in\Rbb^2\,\middle|\,|w|\leq 1\right\}\right),
  \overline{B_{\Rbb^2}}\right),
  %\overline{B_1(0)}\right),
  \tau>0$  \\
  Initialisiere $v_0\coloneqq 0$ in $\textup{CR}^1_0(\mathcal T)$.
  \For{$j = 1,2,\dots$}
  \begin{align}
    %\label{eq:primalDualAlgUj}
    \tilde{u}_j&\coloneqq u_{j-1}+\tau v_{j-1},\nonumber\\
    \label{eq:primalDualAlgLambdaJ}
    \Lambda_j
    &\coloneqq
    \frac{\Lambda_{j-1}+\tau\nabla_{\textup{NC}} \tilde{u}_j}
    {\max\left\{1,
    \left|\Lambda_{j-1}+\tau\nabla_{\textup{NC}}\tilde{u}_j\right|\right\}},\\
    \text{bestimme }u_j\in\textup{CR}^1_0(\mathcal{T})
    \text{ a}&\text{ls Lösung des linearen Gleichungssystems }\nonumber\\
    \label{eq:linSysPrimalDualAlg}
    \frac{1}{\tau}a_{\textup{NC}}(u_j,\bullet)+\alpha(u_j,\bullet)
    &=
    \frac{1}{\tau}a_{\textup{NC}}(u_{j-1},\bullet) + (f,\bullet)
    - \left(\Lambda_j,\nabla_{\textup{NC}}\bullet\right) 
    \text{ in }\CR^1_0(\Tcal),\\
    v_j &\coloneqq \frac{u_j-u_{j-1}}{\tau}.\nonumber
  \end{align}
  %\State bestimme $u_j\in\textup{CR}^1_0(\mathcal{T})$
  %als Lösung des linearen Gleichungssystems
  %\begin{align}
  %  \label{eq:linSysPrimalDualAlg}
  %  \frac{1}{\tau}a_{\textup{NC}}(u_j,\bullet)+\alpha(u_j,\bullet)
  %  &=
  %  \frac{1}{\tau}a_{\textup{NC}}(u_{j-1},\bullet) + (f,\bullet)
  %  - \left(\Lambda_j,\nabla_{\textup{NC}}\bullet\right) 
  %  \quad\text{in }\CR^1_0(\Tcal),
  %\end{align}
  %\begin{equation*}
  %  v_j\coloneqq\frac{u_j-u_{j-1}}{\tau}.
  %\end{equation*}
  \EndFor
  \Ensure Folge $(u_j,\Lambda_j)_{j\in\mathbb N}$ in
  $\CR^1_0(\mathcal{T})\times
   P_0\!\left(\mathcal{T};\overline{B_{\Rbb^2}}\right)$   
  \end{algorithmic}
\end{algorithm}

In \cite{Bar15} wird in \Cref{eq:linSysPrimalDualAlg} anstelle des diskreten
Skalarprodukts $\anc(\bullet,\bullet)$ ein diskretes Skalarprodukt
$(\bullet,\bullet)_{h,s}$, welches ungleich dem $L^2$-Skalarprodukt sein
kann, genutzt. Die Iteration wird ebenda abgebrochen, wenn die von
$(\bullet,\bullet)_{h,s}$ induzierte Norm des Terms $(u_j-u_{j-1})/\tau$
kleiner einer gegebenen Toleranz ist.
Dementsprechend nutzen wir mit $\epsstop > 0$ für
\Cref{alg:primalDualIteration} das Abbruchkriterium 
\begin{align}
  \label{eq:terminationCriterion}
  \left\vvvert \frac{u_j-u_{j-1}}{\tau}\right\vvvert_\NC<\epsstop.
\end{align}

\begin{remark} 
  \label{rem:primalDualMatrixEquations}
  Mit den kantenorientierten Crouzeix-Raviart-Basisfunktionen
  $\{\psi_E\,\mid\,E\in\Ecal\}$ aus \Cref{sec:crouzeixRaviartFunctions} können
  wir die Steifigkeitsmatrix $A\in\Rbb^{|\Ecal|\times|\Ecal|}$ und die
  Massenmatrix $M\in\Rbb^{|\Ecal|\times|\Ecal|}$ für alle
  $k,\ell\in\{1,2,\ldots,|\Ecal|\}$ definieren durch
  \begin{align*}
    A_{k\ell}\coloneqq \anc\!\left(\psi_{E_k},\psi_{E_\ell}\right)
    \quad\text{und}\quad
    M_{k\ell}\coloneqq \left(\psi_{E_k},\psi_{E_\ell}\right).
  \end{align*}
  Außerdem definieren wir mit $u_{j-1}$ und $\Lambda_j$ aus
  \Cref{alg:primalDualIteration} den Vektor $b\in\Rbb^{|\Ecal|}$ für alle
  $k\in\Nbb$ durch
  \begin{align*}
    b_k\coloneqq 
    \left(\frac{1}{\tau}\gradnc u_{j-1}-\Lambda_j,\gradnc\psi_{E_k}\right)
    + \left( f,\psi_{E_k} \right).
  \end{align*}
  Sei nun $\Ecal=\left\{E_1,E_2,\ldots,E_{|\Ecal|}\right\}$ und sei ohne
  Beschränkung der Allgemeinheit 
  $\Ecal(\Omega)\coloneqq\left\{E_1,E_2,\ldots,E_{|\Ecal(\Omega)|}\right\}$.  
  Dann ist $J\coloneqq\{|\Ecal(\Omega)|+1,|\Ecal(\Omega)|+2,\ldots,|\Ecal|\}$
  die Menge der Indizes der Randkanten in $\Ecal$.
  Damit definieren wir die Matrix $\bar
  A\in\Rbb^{|\Ecal(\Omega)|\times|\Ecal(\Omega)|}$, die durch Streichen der
  Zeilen und Spalten von $A$ mit den Indizes aus $J$ entsteht, die Matrix $\bar
  M\in\Rbb^{|\Ecal(\Omega)|\times|\Ecal(\Omega)|}$, die ebenso aus $M$
  hervorgeht und den Vektor $\bar b\in\Rbb^{|\Ecal(\Omega)|}$, der durch
  Streichen der Komponenten von $b$ mit Indizes in $J$ entsteht.
  Weiterhin sei $x\in\Rbb^{|\Ecal(\Omega)|}$, wobei $x_k$ für alle
  $k\in\{1,2\ldots,|\Ecal(\Omega)|\}$ der Koeffizient der Lösung 
  $u_j\in\CR^1_0(\Tcal)$ des
  Gleichungssystems \eqref{eq:linSysPrimalDualAlg} zur $k$-ten Basisfunktion
  von $\CR^1_0(\Tcal)$ sei, das heißt es gelte
  \begin{align*}
    u_j=\sum_{k=1}^{|\Ecal(\Omega)|} x_k\psi_{E_k}.
  \end{align*}
  Da wir das Gleichungssystem \eqref{eq:linSysPrimalDualAlg} in
  $\CR^1_0(\Tcal)$ lösen, lässt sich somit $u_j$ durch Lösen einer
  Matrixgleichung nach $x$ bestimmen. 
  Diese lautet
  \begin{align}
    \label{eq:linSysPrimalDualAlgMatrixEq}
    \left(\frac{1}{\tau}\bar A+\alpha \bar M\right)x=\bar b.
  \end{align}
\end{remark}

\section{Konvergenz der Iteration}
In diesem Abschnitt beweisen wir die Konvergenz der Iterate von
\Cref{alg:primalDualIteration} gegen die Lösung von
\Cref{prob:discreteProblem}. 
Dabei bedienen wir uns unter anderem der äquivalenten Charakterisierungen aus
\Cref{thm:discProbCharacterizationOfDiscreteSolutions}.

\begin{theorem}
  Sei $\ucr\in \CR^1_0(\Tcal)$ Lösung von \Cref{prob:discreteProblem} und
  $\bar\Lambda_0\in P_0\!\left(\Tcal;\Rbb^2\right)$ erfülle
  $\left|\bar\Lambda_0(\bullet)\right|\leq 1$ fast überall in $\Omega$ sowie
  \Cref{eq:discreteMultiplierScalerProductEquality} und
  \Cref{eq:discreteMultiplierL2Equality} aus
  \Cref{thm:discProbCharacterizationOfDiscreteSolutions} mit
  $\tilde{u}_\CR=\ucr$.
  Falls $\tau \in (0, 1]$, dann konvergieren die Iterate $(u_j)_{j\in\Nbb}$ von
  \Cref{alg:primalDualIteration} in $L^2(\Omega)$ gegen $\ucr.$
\end{theorem}

\begin{proof}
  Der Beweis folgt einer Skizze von Professor Carstensen.
  
  Sei $j\in\Nbb$. 
  Seien weiterhin $u_0$, $\Lambda_0$ und $v_0$ sowie $\tilde{u}_j$,
  $\Lambda_j$, $u_j$ und $v_j$ definiert wie in \Cref{alg:primalDualIteration}.
  Außerdem definieren wir $\mu_j\coloneqq \max\{1,|\Lambda_{j-1}+\tau \gradnc
  \tilde{u}_j|\}$ und für alle $k\in\Nbb_0$ die Abkürzungen $e_k \coloneqq
  \ucr-u_k$, $E_k\coloneqq \bar\Lambda_0-\Lambda_k$.
  Dabei nutzen wir die Konvention $e_{-1}\coloneqq e_0$.
  Wir testen zunächst \eqref{eq:linSysPrimalDualAlg} mit $e_j$ und formen das
  Resultat um. 
  Damit erhalten wir
  \begin{align*}
    \anc(v_j,e_j) + \alpha(u_j,e_j) 
    + (\Lambda_j,\gradnc e_j)
    = 
    (f,e_j).
  \end{align*}
  Zusammen mit \Cref{eq:discreteMultiplierL2Equality} folgt daraus
  \begin{equation}
    \label{eq:convProofE}
    \begin{aligned}
      \anc(v_j,e_j) &= 
      \alpha(\ucr-u_j,e_j) 
      + \left(\bar\Lambda_0-\Lambda_j,\gradnc e_j\right) 
      = 
      \alpha\Vert e_j\Vert^2 + \left(E_j,\gradnc e_j\right).
    \end{aligned}
  \end{equation}
  Als Nächstes betrachten wir \Cref{eq:primalDualAlgLambdaJ}. Es gilt
  \begin{align}
    \label{eq:convProofA}
    \Lambda_{j-1}-\Lambda_j+\tau \gradnc \tilde{u}_j 
    = (\mu_j-1)\Lambda_j \quad\text{fast überall in }\Omega.
  \end{align}
  Außerdem folgt aus \Cref{eq:primalDualAlgLambdaJ} und einer
  Fallunterscheidung zwischen $1\geq |\Lambda_{j-1}+\tau\gradnc \tilde{u}_j|$
  und $1< |\Lambda_{j-1}+\tau\gradnc \tilde{u}_j|$, dass
  \begin{align}
    \label{eq:convergenceIterationMuProductZero}
    \left(1-|\Lambda_j|\right)(\mu_j-1)=0
    \quad\text{fast überall in } \Omega.
  \end{align}
  Testen wir nun \Cref{eq:convProofA} in $L^2(\Omega)$ mit $E_j$, erhalten wir 
  unter Nutzung von $\mu_j\geq 1$, der Cauchy-Schwarzschen Ungleichung,
  $\left|\bar\Lambda_0(\bullet)\right|\leq 1$ fast überall in $\Omega$ und
  \Cref{eq:convergenceIterationMuProductZero}, dass
  \begin{align*}
    \left( \Lambda_{j-1}-\Lambda_j+\tau\gradnc \tilde{u}_j, E_j\right)
    &= 
    \left( (\mu_j-1)\Lambda_j,\bar\Lambda_0-\Lambda_j\right)\\
    &\leq
    \int_\Omega (\mu_j-1)\left(|\Lambda_j|-|\Lambda_j|^2\right)\dx\\
    &=
    \int_\Omega |\Lambda_j| (1-|\Lambda_j|)(\mu_j-1)\dx 
    =
    0.
  \end{align*}
  Daraus folgt mit $\Lambda_{j-1}-\Lambda_j = E_j-E_{j-1}$ und $\tilde{u}_j =
  u_{j-1}-(e_{j-1}-e_{j-2})$, dass nach Division durch $\tau$ gilt
  \begin{align}
    \label{eq:convProofB}
    \left(\frac{E_j-E_{j-1}}{\tau}+ \gradnc u_{j-1}-\gradnc
    (e_{j-1}-e_{j-2}),E_j\right)\leq 0.
  \end{align}
  Aus der Cauchy-Schwarzschen Ungleichung,
  \Cref{eq:discreteMultiplierScalerProductEquality} und
  $|\Lambda_j(\bullet)|\leq 1$ fast überall in $\Omega$ folgt, dass
  \begin{align*}
    \gradnc\ucr\cdot E_j 
    &=
    \gradnc\ucr\cdot\bar\Lambda_0 - \gradnc\ucr\cdot\Lambda_j\\
    &\geq 
    \gradnc\ucr\cdot\bar\Lambda_0 - |\gradnc\ucr||\Lambda_j| \\
    &= 
    |\gradnc\ucr|(1-|\Lambda_j|)
    \geq
    0\quad\text{fast überall in }\Omega.
  \end{align*}
  Daraus folgt
  \begin{align}
    \label{eq:convProofC}
    (\gradnc\ucr,E_j)=\int_\Omega \gradnc\ucr\cdot E_j\dx\geq 0.
  \end{align}
  Aus den Ungleichungen \eqref{eq:convProofB} und \eqref{eq:convProofC} folgt
  insgesamt
  \begin{align*}
    \left( \frac{E_j-E_{j-1}}{\tau}+ \gradnc u_{j-1}
    -\nabla_\nc(e_{j-1}-e_{j-2}),E_j\right)
    \leq
    (\gradnc\ucr,E_j).
  \end{align*}
  Das ist äquivalent zu
  \begin{align}
    \label{eq:convProofD}
    \left( \frac{E_j-E_{j-1}}{\tau} 
    -\gradnc(2e_{j-1}-e_{j-2}),E_j\right)\leq 0.
  \end{align}
  Weiterhin gilt
  \begin{equation}
    \begin{aligned}
      \label{eq:longVvvertFormula}
      &\vvvert e_j \vvvert^2_\nc   -
      \vvvert e_{j-1}\vvvert_\nc^2 +
      \Vert E_j \Vert^2 - \Vert E_{j-1}\Vert^2 +
      \vvvert e_j-e_{j-1}\vvvert_\nc^2 +
      \Vert E_j - E_{j-1} \Vert^2\\
      &=
      2a_\nc(e_j,e_j-e_{j-1}) + 2(E_j,E_j-E_{j-1}).
    \end{aligned}
  \end{equation}
  Unter Nutzung von $e_j-e_{j-1}=-\tau v_j$ und \Cref{eq:convProofE} 
  gilt außerdem
  \begin{align*}
    2a_\nc(e_j,e_j-e_{j-1}) + 2(E_j,E_j-E_{j-1})
    &=
    -2\tau a_\nc(e_j,v_j) + 2(E_j,E_j-E_{j-1})\\
    &=
    -2\tau\alpha\Vert e_j\Vert^2 + 2\tau\left(E_j,
    -\nabla_\nc e_j+\frac{E_j-E_{j-1}}{\tau}\right) 
  \end{align*}
  Daraus folgt durch Ungleichung \eqref{eq:convProofD} zusammen mit $\tau>0$,
  dass
  \begin{align*}
    &2a_\nc(e_j,e_j-e_{j-1}) + 2(E_j,E_j-E_{j-1})\\
    &\leq
    -2\tau\alpha\Vert e_j\Vert^2 + 2\tau\left(E_j,
    -\nabla_\nc e_j+\frac{E_j-E_{j-1}}{\tau}\right)\\
    &\quad\quad-2\tau\left( \frac{E_j-E_{j-1}}{\tau}
    -\gradnc(2e_{j-1}-e_{j-2}),E_j\right)\\
    &=
    -2\tau\alpha\Vert e_j\Vert^2 - 
    2\tau\big(E_j,\gradnc(e_j-2e_{j-1}+e_{j-2})\big).
  \end{align*}
  Damit und mit \Cref{eq:longVvvertFormula} erhalten wir insgesamt
  \begin{align*}
    &\vvvert e_j \vvvert^2_\nc   -
    \vvvert e_{j-1}\vvvert_\nc^2 +
    \Vert E_j \Vert^2 - \Vert E_{j-1}\Vert^2 +
    \vvvert e_j-e_{j-1}\vvvert_\nc^2 +
    \Vert E_j - E_{j-1} \Vert^2\\
    &\leq
    -2\tau\alpha\Vert e_j\Vert^2 - 
    2\tau\big(E_j,\gradnc(e_j-2e_{j-1}+e_{j-2})\big).
  \end{align*}
  Für jedes $J\in\Nbb$ führt die Summation dieser Ungleichung über
  $j=1,\ldots,J$ und eine Äquivalenzumfomung zu
  \begin{equation}
    \label{eq:convProofF}
    \begin{aligned}
      &\vvvert e_J \vvvert^2_\nc +\Vert E_J \Vert^2 
      +\sum_{j=1}^J\left(\vvvert e_j-e_{j-1} \vvvert_\nc^2 + 
      \Vert E_j-E_{j-1}\Vert^2\right)\\
      &\leq 
      \vvvert e_0 \vvvert_\nc^2 + \Vert E_0 \Vert^2 
      -2\tau\alpha\sum_{j=1}^J \Vert e_j\Vert^2 
      -2\tau \sum_{j=1}^J\big(E_j,\gradnc
      (e_j-2e_{j-1}+e_{j-2})\big).
    \end{aligned}
  \end{equation}
  Für die letzte Summe auf der rechten Seite dieser Ungleichung gilt, unter
  Beachtung von $e_{-1}=e_0$, dass
  \begin{align*}
    &\sum_{j=1}^J\big(E_j,\gradnc
    (e_j-2e_{j-1}+e_{j-2})\big)\\
    &=\sum_{j=1}^J\big(E_j,\gradnc(e_j-e_{j-1})\big)
    -
    \sum_{j=0}^{J-1}\big(E_{j+1},\gradnc(e_j-e_{j-1})\big) \\
    &= 
    \sum_{j=1}^{J-1} 
    \big(E_j-E_{j+1},\gradnc(e_j-e_{j-1})\big)
    +\big(E_J,\gradnc(e_J-e_{J-1})\big)
    - \big(E_1, \gradnc(e_0-e_{-1})\big) \\
    &= 
    \sum_{j=1}^{J-1} 
    \big(E_j-E_{j+1},\gradnc(e_j-e_{j-1})\big)
    +\big(E_J,\gradnc(e_J-e_{J-1})\big).
  \end{align*}
  Mit dieser Umformung erhalten wir aus Ungleichung \eqref{eq:convProofF} für
  jedes $\tau\in(0,1]$, das heißt $\tau^{-1}\geq 1$, dass
  \begin{equation}
    \label{eq:convProofG}
    \begin{aligned}
      &\vvvert e_J \vvvert^2_\nc +\Vert E_J \Vert^2 
      +\sum_{j=1}^J\left(\vvvert e_j-e_{j-1} \vvvert_\nc^2 + 
      \Vert E_j-E_{j-1}\Vert^2\right) \\
      &\leq 
      \tau^{-1}\left(\vvvert e_0 \vvvert_\nc^2 + \Vert E_0 \Vert^2 \right)
      -2\alpha\sum_{j=1}^J \Vert e_j\Vert^2 \\
      &\quad\quad
      -2 \sum_{j=1}^{J-1} \big(E_j-E_{j+1},\gradnc(e_j-e_{j-1})\big)
      -2\big(E_J,\gradnc(e_J-e_{J-1})\big).
    \end{aligned}
  \end{equation}
  Außerdem gilt
  \begin{align*}
    2\alpha\sum_{j=1}^J\Vert e_j\Vert^2 
    &\leq
    2\alpha\sum_{j=1}^J\Vert e_j\Vert^2
    +\Vert E_J + \gradnc(e_J-e_{J-1}) \Vert^2 
    + \vvvert e_J \vvvert^2_\nc 
    + \Vert E_1 - E_0 \Vert^2 \\
    &\quad\quad
    + \sum_{j=1}^{J-1}  
      \Vert \gradnc(e_j-e_{j-1}) - (E_{j+1} - E_j ) \Vert^2 \\
    &= 
    2\alpha\sum_{j=1}^J\Vert e_j\Vert^2
    +\vvvert e_J \vvvert^2_\nc + \Vert E_J \Vert^2 
    + \sum_{j=1}^J \left( \vvvert e_j-e_{j-1} \vvvert^2_\nc
    + \Vert E_j - E_{j-1} \Vert^2 \right)\\
    &\quad\quad
    + 2\sum_{j=1}^{J-1}\big(E_j-E_{j+1},\gradnc(e_j-e_{j-1})\big)
    + 2\big(E_{J},\gradnc(e_J-e_{J-1})\big).
  \end{align*}
  Zusammen mit Ungleichung \eqref{eq:convProofG} folgt daraus
  \begin{align}
    \label{eq:updateBound}
    2\alpha\sum_{j=1}^J\Vert e_j\Vert^2 
    \leq
    \tau^{-1}\left(\vvvert e_0\vvvert^2_\nc + \Vert E_0\Vert^2\right).
  \end{align}
  Somit gilt, dass $\sum_{j=1}^\infty \Vert e_j\Vert^2$ beschränkt
  ist, was impliziert $\Vert\ucr-u_j\Vert=\Vert e_j\Vert\rightarrow 0$ für
  $j\rightarrow \infty$.
\end{proof}


\chapter{Implementierung}
\label{chap:implementation}
\section{Algorithmus}
\todo[inline]{Irgendwo, wahrscheinlich bei ,,alles zu $\CR_0$'' muss noch
$\anc(u,v)\coloneqq\int_\Omega\gradnc u\cdot\gradnc v\dx$ erwähnt werden (und
warum das ein SP ist muss angerissen werden, Stichwort Friedrichs Ungleichung)} 

Für unsere Formulierung \Cref{prob:discreteProblem} nutzen wir \cite[S. 314,
Algorithm 10.1]{Bar15} unter Beachtung von \cite[S. 314, Remark 10.11]{Bar15}
als Algorithmus als iterativen Löser und benutzen als inneres Produkt $\anc$
(definiert in Kapitel \ldots hier in dieser Arbeit).
Weitere Details dazu finden sich in \cite[S. 118-121]{Bar15}.
\todo[inline]{Beim Zitieren z.B. 'Remark' lassen, weil es in Bartels so heißt,
oder das lieber übersetzen? Außerdem natürlich, passt das so als Einleitung 
für den Alg?}
\begin{algorithm}[Primale-Duale Iteration]
  \label{alg:primalDualIteration}
\begin{algorithmic}\\
  \Require $u_0\in\textup{CR}_0^1(\mathcal{T}),$
  $\Lambda_0\in \Pbb_0(\mathcal{T};
  \overline{B(0,1)}),\tau>0$  \\
  Initialisiere $v_0\coloneqq 0$ in $\textup{CR}^1_0(\mathcal T)$.
  \For{$j = 1,2,\dots$}
  \begin{equation}
    \label{eq:primalDualAlgUj}
    \tilde{u}_j\coloneqq u_{j-1}+\tau v_{j-1},
  \end{equation}
  \begin{equation}
    \label{eq:primalDualAlgLambdaJ}
    \Lambda_j\coloneqq
    (\Lambda_{j-1}+\tau\nabla_{\textup{NC}} \tilde{u}_j)/
      (\operatorname{max}\{1,|\Lambda_{j-1}+\tau\nabla_{\textup{NC}}\tilde{u}_j|\}),
  \end{equation}
      \State
  \State bestimme $u_j\in\textup{CR}^1_0(\mathcal{T})$
  als Lösung des linearen Gleichungssystems
  \begin{align}
    \label{eq:linSysPrimalDualAlg}
    \frac{1}{\tau}&a_{\textup{NC}}(u_j,\bullet)+\alpha(u_j,\bullet)_{L^2(\Omega)}
    \notag \\
    &=
    \frac{1}{\tau}a_{\textup{NC}}(u_{j-1},\bullet) + (f,\bullet)_{L^2(\Omega)}
    - (\Lambda_j,\nabla_{\textup{NC}}\bullet)_{L^2(\Omega)} 
  \end{align}
  \State in $\CR^1_0(\mathcal{T})$, \\
  \begin{equation*}
    v_j\coloneqq(u_j-u_{j-1})/\tau.
  \end{equation*}
  \EndFor
  \Ensure Folge $(u_j,\Lambda_j)_{j\in\mathbb N}$ in
  $\CR^1_0(\mathcal{T})\times
  \Pbb_0(\mathcal{T};\overline{B(0,1)})$   
  \end{algorithmic}
\end{algorithm}

\begin{theorem}
  Sei $\ucr\in \CR^1_0(\Tcal)$ Lösung von \Cref{prob:discreteProblem}
  und $\bar\Lambda\in\operatorname{sign}(\gradnc\ucr)$ erfülle Eigenschaft 
  \textit{(ii)} in \Cref{thm:discProbCharacterizationOfDiscreteSolutions}.
  Falls $0 < \tau \leq 1$, dann konvergieren die Iterate $(u_j)_{j\in\Nbb}$ von
  \Cref{alg:primalDualIteration}
  gegen $\ucr.$
\end{theorem}

\begin{proof}
  Der Beweis folgt einer Skizze von Prof. Carstensen.
  
  Seien $\tilde{u}_j$, $v_j$ und $\Lambda_j$ definiert wie in
  \Cref{alg:primalDualIteration}.
  Definiere außerdem $e_j \coloneqq \ucr-u_j$ und $E_j\coloneqq
  \bar\Lambda-\Lambda_j$. 

  Testen wir nun \eqref{eq:linSysPrimalDualAlg} mit $e_j$, erhalten wir
  \begin{align*}
    \anc(v_j,e_j) + \alpha(u_j,e_j)_{L^2(\Omega)} 
    + (\Lambda_j,\gradnc e_j)_{L^2(\Omega)}
    = 
    (f,e_j)_{L^2(\Omega)}.
  \end{align*}
  Äquivalent dazu ist, da $\ucr$ \Cref{eq:discreteMultiplierL2Equality} löst, 
  \begin{align}
    \label{eq:convProofE}
    \anc(v_j,e_j) &= 
    \alpha(\ucr-u_j,e_j)_{L^2(\Omega)} 
    + (\bar\Lambda-\Lambda_j,\gradnc e_j)_{L^2(\Omega)} \notag\\
    &= 
    \alpha\Vert e_j\Vert_{L^2(\Omega)}^2
    + (E_j,\gradnc e_j)_{L^2(\Omega)}.
  \end{align}
  Sei $\mu_j\coloneqq \max\{1,|\Lambda_{j-1}+\tau
  \gradnc \tilde{u}_j|\}$.
  Nutzen wir \eqref{eq:primalDualAlgLambdaJ} erhalten wir damit
  \begin{align}
    \label{eq:convProofA}
    \Lambda_{j-1}-\Lambda_j+\tau \gradnc \tilde{u}_j 
    = (\mu_j-1)\Lambda_j \quad\text{fast überall in }\Omega.
  \end{align}
  Für fast alle $x\in\Omega$ liefert die CSU, da $|\bar\Lambda|\leq 1$ fast
  überall in $\Omega$,
  $\Lambda_j(x)\cdot\bar\Lambda(x)\leq|\Lambda_j(x)|$ und damit folgt 
  aus \Cref{eq:primalDualAlgLambdaJ} und einer einfachen Fallunterscheidung
  zwischen $1\geq |\Lambda_{j-1}+\tau\gradnc \tilde{u}_j|$ und
  $1< |\Lambda_{j-1}+\tau\gradnc \tilde{u}_j|$,
  dass $(1-|\Lambda_j(x)|)(\mu_j(x)-1)=0$.
  Testen wir nun \eqref{eq:convProofA} mit $E_j$, erhalten wir 
  unter Nutzung von $\mu_j\geq 1$ und der CSU damit
  \begin{align*}
    ( \Lambda_{j-1}-\Lambda_j+\tau\gradnc \tilde{u}_j,
    E_j)_{L^2(\Omega)}
    &= 
    ( (\mu_j-1)\Lambda_j,\bar\Lambda-\Lambda_j)_{L^2(\Omega)}\\
    &=
    \int_\Omega
    (\mu_j-1)(\Lambda_j\cdot\bar\Lambda-\Lambda_j\cdot\Lambda_j)\,\mathrm dx\\
    &\leq
    \int_\Omega (\mu_j-1)(|\Lambda_j|-|\Lambda_j|^2)\,\mathrm dx\\
    &=
    \int_\Omega |\Lambda_j|
    (1-|\Lambda_j|)(\mu_j-1)\,\mathrm dx \\
    &=
    \int_\Omega |\Lambda_j|\cdot
    0\,\mathrm dx =0.
  \end{align*}
  Damit und mit $\Lambda_{j-1}-\Lambda_j=E_j-E_{j-1}$, 
  $\tilde{u}_j=u_{j-1}+\tau v_{j-1}=u_{j-1}+u_{j-1}-u_{j-2}=
  u_{j-1}-(e_{j-1}-e_{j-2})$
  für $j\geq 2$ und der Konvention $e_{-1}\coloneqq e_0$ für $j=1$ erhalten wir
  insgesamt
  \begin{align}
    \label{eq:convProofB}
    \left(\frac{E_j-E_{j-1}}{\tau}+ \gradnc u_{j-1}-\gradnc
    (e_{j-1}-e_{j-2}),E_j\right)_{L^2(\Omega)}\leq 0\quad\text{für alle }
    j\in\Nbb.
  \end{align}
  Falls $|\gradnc \ucr|\neq 0$, gilt somit zusammen mit
  der CSU, $\bar\Lambda\in\sign\gradnc\ucr$ und $|\Lambda_j|\leq 1$,  dass
  \begin{align*}
    \gradnc\ucr\cdot E_j 
    &=
    \gradnc\ucr\cdot\bar\Lambda - \gradnc\ucr\cdot\Lambda_j\\
    &\geq 
    \gradnc\ucr\cdot\bar\Lambda - |\gradnc\ucr||\Lambda_j| \\
    &=
    |\gradnc\ucr|^2/|\gradnc\ucr|-|\gradnc\ucr||\Lambda_j| \\
    &= 
    |\gradnc\ucr|(1-|\Lambda_j|)\\
    &\geq
    0. 
  \end{align*}
  Falls $|\gradnc\ucr|=0$, gilt diese Ungleichung ebenfalls.
  Daraus folgt
  \begin{align}
    \label{eq:convProofC}
    (\gradnc\ucr,E_j)_{L^2(\Omega)}=\int_\Omega \gradnc\ucr\cdot E_j\dx\geq 0.
  \end{align}
  Mit \eqref{eq:convProofB} und \eqref{eq:convProofC} folgt nun
  \begin{align*}
    \left( \frac{E_j-E_{j-1}}{\tau}+ \gradnc u_{j-1}
    -\nabla_\nc(e_{j-1}-e_{j-2}),E_j\right)_{L^2(\Omega)}
    \leq
    (\gradnc\ucr,E_j)_{L^2(\Omega)}
  \end{align*}
  für alle $j\in\Nbb$, was nach Definition von $e_{j-1}$
  äquivalent ist zu
  \begin{align}
    \label{eq:convProofD}
    \left( \frac{E_j-E_{j-1}}{\tau} 
    -\gradnc(2e_{j-1}-e_{j-2}),E_j\right)_{L^2(\Omega)}\leq 0.
  \end{align}
  Unter Nutzung von $-v_j=(e_j-e_{j-1})/\tau$, \eqref{eq:convProofE}, $\tau>0$
  und \eqref{eq:convProofD} erhalten wir 
  \begin{align*}
    &\vvvert e_j \vvvert^2_\nc   -
    \vvvert e_{j-1}\vvvert_\nc^2 +
    \Vert E_j \Vert_{L^2(\Omega)}^2 - \Vert E_{j-1}\Vert_{L^2(\Omega)}^2 +
    \vvvert e_j-e_{j-1}\vvvert_\nc^2 +
    \Vert E_j - E_{j-1} \Vert_{L^2(\Omega)}^2\\
    &\quad =
    2a_\nc(e_j,e_j-e_{j-1}) + 2(E_j,E_j-E_{j-1})_{L^2(\Omega)}\\
    &\quad =
    -2\tau a_\nc(e_j,v_j) + 2(E_j,E_j-E_{j-1})_{L^2(\Omega)}\\
    &\quad =
    -2\tau\alpha\Vert e_j\Vert_{L^2(\Omega)}^2 + 2\tau\left(E_j,
    -\nabla_\nc e_j+\frac{E_j-E_{j-1}}{\tau}\right)_{L^2(\Omega)} \\
    &\quad \leq
    -2\tau\alpha\Vert e_j\Vert_{L^2(\Omega)}^2 + 2\tau\left(E_j,
    -\nabla_\nc e_j+\frac{E_j-E_{j-1}}{\tau}\right)_{L^2(\Omega)}\\ 
    &\quad\quad -2\tau\left( \frac{E_j-E_{j-1}}{\tau}
    -\gradnc(2e_{j-1}-e_{j-2}),E_j\right)_{L^2(\Omega)}\\
    &\quad =
    -2\tau\alpha\Vert e_j\Vert_{L^2(\Omega)}^2 - 
    2\tau\big(E_j,\gradnc(e_j-2e_{j-1}+e_{j-2})\big)_{L^2(\Omega)}.
  \end{align*}
  Für jedes $J\in\Nbb$ führt die Summation über $j=1,\ldots,J$ und eine
  Äquivalenz\-umfomung zu
  \begin{align}
    \label{eq:convProofF}
    &\vvvert e_J \vvvert^2_\nc +\Vert E_J \Vert_{L^2(\Omega)}^2 
    +\sum_{j=1}^J\left(\vvvert e_j-e_{j-1} \vvvert_\nc^2 + 
    \Vert E_j-E_{j-1}\Vert_{L^2(\Omega)}^2\right)\notag \\
    &\quad \leq 
    \vvvert e_0 \vvvert_\nc^2 + \Vert E_0 \Vert_{L^2(\Omega)}^2 
    -2\tau\alpha\sum_{j=1}^J \Vert e_j\Vert^2_{L^2(\Omega)} \\
    &\quad\quad
    -2\tau \sum_{j=1}^J\big(E_j,\gradnc
    (e_j-2e_{j-1}+e_{j-2})\big)_{L^2(\Omega)}.\notag
  \end{align}
  Dabei lässt sich die letzt Summe, unter Beachtung von $e_{-1}=e_0$, umformen
  zu
  \begin{align*}
    &\sum_{j=1}^J\big(E_j,\gradnc
    (e_j-2e_{j-1}+e_{j-2})\big)_{L^2(\Omega)}\\
    &\quad=\sum_{j=1}^J(E_j,\gradnc(e_j-e_{j-1}))_{L^2(\Omega)}
    -
    \sum_{j=0}^{J-1}(E_{j+1},\gradnc(e_j-e_{j-1}))_{L^2(\Omega)} \\
    &\quad = 
    \sum_{j=1}^{J-1} 
    \big(E_j-E_{j+1},\gradnc(e_j-e_{j-1})\big)_{L^2(\Omega)}
    +(E_J,\gradnc(e_J-e_{J-1}))_{L^2(\Omega)}\\
    &\quad\quad 
    - (E_1, \gradnc(e_0-e_{-1}))_{L^2(\Omega)} \\
    &\quad = 
    \sum_{j=1}^{J-1} 
    \big(E_j-E_{j+1},\gradnc(e_j-e_{j-1})\big)_{L^2(\Omega)}
    +(E_J,\gradnc(e_J-e_{J-1}))_{L^2(\Omega)}
  \end{align*}
  und da die linke Seite von
  \eqref{eq:convProofF} nicht negativ ist, gilt damit
  für jedes $0<\tau\leq 1$, dass
  \begin{align*}
    &\tau\left(\vvvert e_J \vvvert^2_\nc +\Vert E_J \Vert_{L^2(\Omega)}^2 
    +\sum_{j=1}^J\left(\vvvert e_j-e_{j-1} \vvvert_\nc^2 + 
    \Vert E_j-E_{j-1}\Vert_{L^2(\Omega)}^2\right)\right) \\
    &\quad \leq 
    \vvvert e_0 \vvvert_\nc^2 + \Vert E_0 \Vert_{L^2(\Omega)}^2 
    -2\tau\alpha\sum_{j=1}^J \Vert e_j\Vert^2_{L^2(\Omega)} \\
    &\quad\quad
    -2\tau\left( 
    \sum_{j=1}^{J-1} 
    (E_j-E_{j+1},\gradnc(e_j-e_{j-1}))_{L^2(\Omega)}
    +(E_J,\gradnc(e_J-e_{J-1}))_{L^2(\Omega)}\right).
  \end{align*}
  Division durch $\tau$ ergibt
  \begin{align}
    \label{eq:convProofG}
    &\vvvert e_J \vvvert^2_\nc +\Vert E_J \Vert_{L^2(\Omega)}^2 
    +\sum_{j=1}^J\left(\vvvert e_j-e_{j-1} \vvvert_\nc^2 + 
    \Vert E_j-E_{j-1}\Vert_{L^2(\Omega)}^2\right) \notag\\
    &\quad \leq 
    \tau^{-1}(\vvvert e_0 \vvvert_\nc^2 + \Vert E_0 \Vert_{L^2(\Omega)}^2 )
    -2\alpha\sum_{j=1}^J \Vert e_j\Vert^2_{L^2(\Omega)} \\
    &\quad\quad
    -2 \sum_{j=1}^{J-1} (E_j-E_{j+1},\gradnc(e_j-e_{j-1}))_{L^2(\Omega)}
    -2(E_J,\gradnc(e_J-e_{J-1}))_{L^2(\Omega)}.\notag
  \end{align}

%  \begin{align*}
%    &2\tau\sum_{j=1}^J(E_j,\gradnc(-e_j+e_{j-1}))_{L^2(\Omega)} +
%    2\tau\sum_{j=0}^{J-1}(E_{j+1},\gradnc(e_j-e_{j-1}))_{L^2(\Omega)}\\
%    &\quad \le
%    2\sum_{j=1}^J(E_j,\gradnc(-e_j+e_{j-1}))_{L^2(\Omega)} +
%    2\sum_{j=0}^{J-1}(E_{j+1},\gradnc(e_j-e_{j-1}))_{L^2(\Omega)}\\
%    &\quad =
%    2\sum_{j=1}^J(E_j,\gradnc(-e_j+e_{j-1}))_{L^2(\Omega)} 
%    +
%    2\sum_{j=1}^{J-1}(E_{j+1},\gradnc(e_j-e_{j-1}))_{L^2(\Omega)} 
%    \tag{since $e_{-1}\coloneqq e_0$}\\
%    &\quad =
%    2\sum_{j=1}^{J-1}(E_{j+1}-E_j,\gradnc(e_j-e_{j-1}))_{L^2(\Omega)}
%    -2(E_{J},\gradnc(e_J-e_{J-1}))_{L^2(\Omega)}
%    .
%  \end{align*}

  Schließlich ergibt eine Abschätzung unter Nutzung von \eqref{eq:convProofG}, 
  dass
  \begin{align*}
    2\alpha\sum_{j=1}^J\Vert e_j\Vert^2_{L^2(\Omega)} 
    &\leq
    2\alpha\sum_{j=1}^J\Vert e_j\Vert^2_{L^2(\Omega)}\\
    &\quad
    +\Vert E_J + \gradnc(e_J-e_{J-1}) \Vert_{L^2(\Omega)}^2 
    + \vvvert e_J \vvvert^2_\nc 
    + \Vert E_1 - E_0 \Vert^2_{L^2(\Omega)} \\
    &\quad 
    + \sum_{j=1}^{J-1}  
      \Vert \gradnc(e_j-e_{j-1}) - (E_{j+1} - E_j ) \Vert^2_{L^2(\Omega)} \\
    & = 
    2\alpha\sum_{j=1}^J\Vert e_j\Vert^2_{L^2(\Omega)}\\
    &\quad 
    +\vvvert e_J \vvvert^2_\nc + \Vert E_J \Vert_{L^2(\Omega)}^2 
    + \sum_{j=1}^J ( \vvvert e_j-e_{j-1} \vvvert^2_\nc
    + \Vert E_j - E_{j-1} \Vert^2_{L^2(\Omega)} )\\
    &\quad
    + 2\sum_{j=1}^{J-1}(E_j-E_{j+1},\gradnc(e_j-e_{j-1}))_{L^2(\Omega)}\\
    &\quad 
    + 2(E_{J},\gradnc(e_J-e_{J-1}))_{L^2(\Omega)} \\
    &\leq
    \tau^{-1}(\vvvert e_0\vvvert^2_\nc + \Vert E_0\Vert^2_{L^2(\Omega)}).
  \end{align*}

  Das zeigt, dass
  $\sum_{j=1}^\infty \Vert e_j\Vert _{L^2(\Omega)}^2$ nach oben beschränkt ist,
  was impliziert, dass $\Vert e_j\Vert_{L^2(\Omega)}\rightarrow 0$
  für $j\rightarrow \infty$.
\end{proof}


\section{Seitennummerierungskonvention lokal}
welche Funktionen haben welche, an 
welche Stellen wird also umnummeriert

\section{Aufstellung des zu lösenden LGS}
Aufstellung der Gradienten etc.
 
\section{Berechnung der L1 Norm der Sprünge}
Für die Berechnung des Verfeinerungsindikators [verweis auf entsprechende
section] und zur Auswertung der kontinuierlichen Energie $E(\vcr)$ einer 
Crouzeix-Raviart Funktion $\vcr$, deren diskrete Energie $\Enc(\vcr)$
bereits bekannt ist, werden die $L^1$ Normen der Kantensprünge 
$[\vcr]_F$ für alle Kanten $F\in\Fcal$ der Triangulierung benötigt,
wobei für eine Innenkante $F\in\Fcal(\Omega)$, die gemeinsame Kante der
Dreiecke $T_+$ und $T_-$ ist, gilt
$[\vcr]_F\coloneqq (\vcr|_{T_+})|_F-(\vcr|_{T_-})|_F$
und $[\vcr]_F \coloneqq \vcr|_F$ für eine Randkante
$F\in\Fcal(\partial\Omega)$. Die Konvention der Wahl von
$T_+$ und $T_-$ ist hier irrelevant, da wir
zur Berechung von $\Vert [\vcr]_F\Vert_{L^1(\Omega)}$
ausschließlich den Betrag $|[\vcr]_F|$ benötigen.

Da $\vcr\in\CR^1(\Tcal)$,
ist $[\vcr]_F$ affin linear und es gilt $[\vcr]_F(\Mid(F))=0$ für 
alle Innenkanten $F\in\Fcal(\Omega)$ und, falls 
$\vcr\in\CR^1_0(\Tcal)$, auch für alle Randkanten $F\in\Fcal(\partial\Omega)$.

Die folgenden Aussagen gelten also für Innenkanten beliebiger
Crouzeix-Raviart Funktionen, wir beschränken uns aber von nun
an auf Funktionen $\vcr\in\CR^1_0(\Tcal)$.

Betrachten wir also eine beliebige Kante $F\in\Fcal$
mit $F=\conv\{P_1,P_2\}$. 
Wir definieren eine Parametrisierung $\gamma:[0,2]\to\Rbb^2$ von $F$ durch
$\gamma(t)\coloneqq \frac{t}{2}(P_2-P_1)+P_1$. 
Es gilt $|\gamma'|\equiv \frac{1}{2}|P_2-P_1|=\frac{1}{2}|F|$.

Sei außerdem
$p(t)\coloneqq [\vcr]_F(\gamma(t))$. Dann gilt nach
 [cite Wegintegrale] 
\begin{align*}
  \Vert [\vcr]_F\Vert_{L^1(F)} 
  &=
  \int_F |[\vcr]_F|\ds 
  = \int_0^2 |p(t)|\,|\gamma'(t)|\dt
  = \frac{|F|}{2}\int_0^2 |p(t)|\dt\\
  &= \frac{|F|}{2}\left(\int_0^1 |p(t)|\dt + \int_1^2 |p(t)|\dt\right).
\end{align*}

Da $\vcr\in\CR^1_0(\Tcal)$, ist $|p|$ auf $[0,1]$ und $[1,2]$ jeweils
ein Polynom
vom Grad $1$ mit $p(1)=[\vcr]_F(\Mid(F))=0$, womit sich $|p|$ jeweils
explizit ausdrücken lässt durch
$|p|(t)=(1-t)|p|(0)$ für alle $t\in[0,1]$ und 
$|p|(t)=(t-1)|p|(2)$ für alle $t\in[1,2]$.
Die Mittelpunktsregel $\int_a^b f(x)\dx\approx (b-a)f( (a+b)/2)$ [cite] ist
exakt für Polynome vom Grad $1$ und somit gilt
\begin{align*}
  \int_0^1 |p(t)|\dt 
  &= 
  (1-0)|p|\left( \frac{1}{2} \right)
  =
  \frac{|p|(0)}{2}\quad\text{und }\\
  \int_1^2 |p(t)|\dt 
  &= 
  (2-1)|p|\left( \frac{3}{2} \right)
  =
  \frac{|p|(2)}{2}.
\end{align*}

Somit erhalten wir insgesamt 
\begin{align*}
  \Vert [\vcr]_F\Vert_{L^1(F)} 
  &=
  \frac{|F|}{2}\left(\frac{|p|(0)}{2} + \frac{|p|(2)}{2}\right)
  =
  \frac{|F|}{4}(|p|(0)+|p|(2))\\
  &= 
  \frac{|F|}{4}\big(|[\vcr]_F|(P_1)+|[\vcr]_F|(P_2)\big),
\end{align*}
beziehungsweise 
  $\Vert [\vcr]_F\Vert_{L^1(F)} =
  \frac{|F|}{4}\big(|\vcr|(P_1)+|\vcr|(P_2)\big)$ für eine Randkanten
  $F\in\Fcal(\partial\Omega)$.

Diese Berechnung ist realisiert durch die 
Funktionen \texttt{computeBlaJumps}, die die absoluten Sprünge
in den Endpunkte einer Kante berchnet, \texttt{computeAbsJumps}, die \ldots,
und \texttt{computeL1NormOfJumps}, die schließlich die $L^1$ Norm aller
Kantensprünge berechnet\ldots.



\section{Implementation der GLEB}

\section{Implementation des Refinement Indicators}

\section{Implementaition der exakten Energie Berechnung}


\chapter{Numerische Beispiele}
\label{chap:experiments}
\section{Konstruktion eines Experiments mit exakter Lösung}
Um eine rechte Seite zu finden, zu der die exakte Lösung bekannt
ist, wähle eine Funktion des Radius $u\in H^1_0([0,1])$ mit Träger im 
zweidimensionalen Einheitskreis. Insbesondere muss damit gelten $u(1)=0$ und
$u$ stetig.
Die rechte Seite als Funktion des Radius $f\in L^2([0,1])$ ist dann gegeben
durch 
\begin{align*}
  f \coloneqq 
  \alpha u - \partial_r(\sign(\partial_r u)) - \frac{\sign(\partial_r u)}{r},
\end{align*}
wobei für $F\in\Rbb^2\setminus\{0\}$ gilt 
$\sign(F)\coloneqq \left\{\frac{F}{|F|}\right\}$ 
und $\sign(0)\in B_1(0)$.
Damit außerdem gilt $f\in H^1_0([0,1])$, was z.B.\ für GLEB relevant ist, 
muss also noch Stetigkeit von $\sign(\partial_r u)$ und 
$\partial_r(\sign(\partial_r u))$ verlangt werden und 
$\partial_r(\sign(\partial_r u(1))=\sign(\partial_r u(1))=0$.
Damit $f$ in $0$ definierbar ist, muss auch gelten 
$\sign(\partial_r u) \in o(r)$ für $r\to 0$.

Damit erhält man für die Funktion
\begin{align*}
  u_1(r)\coloneqq
  \begin{cases}
    1, & \text{wenn } 0\leq r\leq\frac{1}{6},\\
    1+(6r-1)^\beta, & \text{wenn } \frac{1}{6}\leq r\leq\frac{1}{3},\\
    2, &\text{wenn } \frac{1}{3}\leq r\leq\frac{1}{2},\\
    2(\frac{5}{2}-3r)^\beta, &\text{wenn } \frac{1}{2}\leq r\leq\frac{5}{6},\\
    0, &\text{wenn } \frac{5}{6}\leq r,
  \end{cases}
\end{align*}
wobei $\beta\geq 1/2$, mit der Wahl
\begin{align*}
  \sign(\partial_r u_1(r)) =
  \begin{cases}
    12r-36r^2, & \text{wenn } 0\leq r\leq\frac{1}{6},\\
    1, & \text{wenn } \frac{1}{6}\leq r\leq\frac{1}{3},\\
    \cos(\pi(6r-2)), &\text{wenn } \frac{1}{3}\leq r\leq\frac{1}{2},\\
    -1, &\text{wenn } \frac{1}{2}\leq r\leq\frac{5}{6},\\
    -\frac{1+\cos(\pi(6r-5))}{2}, &\text{wenn } \frac{5}{6}\leq r\leq 1,
  \end{cases}
\end{align*}
die rechte Seite
\begin{align*}
  f_1(r)\coloneqq 
  \begin{cases}
    \alpha-12(2-9r), & \text{wenn } 0\leq r\leq\frac{1}{6},\\
    \alpha(1+(6r-1)^\beta)-\frac{1}{r}, & \text{wenn } \frac{1}{6}\leq r\leq
    \frac{1}{3},\\
    2\alpha+6\pi\sin(\pi(6r-2))-\frac{1}{r}\cos(\pi(6r-2)), &
    \text{wenn } \frac{1}{3}\leq r\leq\frac{1}{2},\\
    2\alpha(\frac{5}{2}-3r)^\beta+\frac{1}{r},&
    \text{wenn } \frac{1}{2}\leq r\leq\frac{5}{6},\\
    -3\pi\sin(\pi(6r-5))+\frac{1+\cos(\pi(6r-5))}{2r}, &
    \text{wenn } \frac{5}{6}\leq r\leq 1.
  \end{cases}
\end{align*}

Für die Funktion
\begin{align*}
  u_2(r)\coloneqq 
  \begin{cases}
    1, & \text{wenn } 0\leq r\leq\frac{1-\beta}{2},\\
    -\frac{1}{\beta}r + \frac{1+\beta}{2\beta}, & 
    \text{wenn } \frac{1-\beta}{2}\leq r\leq \frac{1+\beta}{2},\\
    0, & \text{wenn } \frac{1+\beta}{2}\leq r,
  \end{cases}
\end{align*}
erhält man mit der Wahl
\begin{align*}
  \sign&(\partial_r u_2(r)) \\
  &\coloneqq 
  \begin{cases}
    \frac{4}{1-\beta}r\left(\frac{1}{1-\beta}r -1\right), &
    \text{wenn } 0\leq r\leq\frac{1-\beta}{2},\\
    -1, & \text{wenn } \frac{1-\beta}{2}\leq r\leq \frac{1+\beta}{2},\\
    \frac{4}{(\beta-1)^3}
    \left( 4r^3-3(\beta+3)r^2 +6(\beta+1)r-3\beta-1\right), & 
    \text{wenn } \frac{1+\beta}{2}\leq r\leq 1,
  \end{cases}
\end{align*}
die rechte Seite
\begin{align*}
  f_2(r)\coloneqq 
  \begin{cases}
    \alpha - \frac{4}{1-\beta}\left(\frac{3}{1-\beta}r - 2\right), &
    \text{wenn } 0\leq r\leq\frac{1-\beta}{2},\\
    -\frac{\alpha}{\beta}\left( r-\frac{1+\beta}{2} \right) +\frac{1}{r}, & 
    \text{wenn } \frac{1-\beta}{2}\leq r\leq \frac{1+\beta}{2},\\
    \frac{-4}{(\beta-1)^3}
    \left( 16r^2 -9(\beta+3)r + 12(\beta+1) - \frac{3\beta+1}{r}\right), & 
    \text{wenn } \frac{1+\beta}{2}\leq r\leq 1.
  \end{cases}
\end{align*}

Damit können Experimente durchgeführt werden bei denen 
\texttt{exactSolutionKnown = true} gesetzt werden kann und entsprechend auch 
der $L^2$-Fehler berechnet wird.

Soll nun auch die Differenz der exakten Energie mit der garantierten unteren 
Energie Schranke (GLEB) berechnet werden, dann werden die stückweisen
Gradienten der exakten Lösung und der rechten Seite benötigt.

Dabei gelten folgende Ableitungsregeln für die Ableitungen einer Funktion 
$g$, wenn man ihr Argument $x=(x_1,x_2)\in\Rbb^2$ in Polarkoordinaten mit Länge
$r=\sqrt{x_1^2+x_2^2}$ und Winkel
$\varphi = \atan(x_2,x_1)$, wobei 
\begin{align*}
  \atan(x_2,x_1)\coloneqq
  \begin{cases}
    \arctan\left( \frac{x_2}{x_1} \right),& \text{wenn }x_1>0,\\
    \arctan\left( \frac{x_2}{x_1} \right) +\pi,& \text{wenn }x_1<0,x_2\geq 0,\\
    \arctan\left( \frac{x_2}{x_1} \right) -\pi,& \text{wenn }x_1<0,x_2<0,\\
    \frac{\pi}{2},& \text{wenn }x_1=0,x_2>0,\\
    -\frac{\pi}{2},& \text{wenn }x_1=0,x_2<0,\\
    \text{undefiniert},& \text{wenn }x_1=x_2=0,\\
  \end{cases}
\end{align*}
auffasst,
\begin{align*}
  \partial_{x_1} &= 
  \cos(\varphi)\partial_r - \frac{1}{r}\sin(\varphi)\partial_\varphi,\\
  \partial_{x_2} &= 
  \sin(\varphi)\partial_r - \frac{1}{r}\cos(\varphi)\partial_\varphi.
\end{align*}
Ist $g$ vom Winkel $\varphi$ unabhängig, so ergibt sich
\begin{align*}
  \nabla_{(x_1,x_2)}g = (\cos(\varphi),\sin(\varphi))\partial_r g.
\end{align*}
Unter Beachtung der trigonometrischen Zusammenhänge
\begin{align*}
  \sin(\arctan(y)) = \frac{y}{\sqrt{1+y^2}},\\
  \cos(\arctan(y)) = \frac{1}{\sqrt{1+y^2}}
\end{align*}
ergibt sich 
\begin{align*}
  (\cos(\varphi),\sin(\varphi)) = (x_1,x_2)\frac{1}{r}
\end{align*}
und damit 
\begin{align*}
  \nabla_{(x_1,x_2)}g = (x_1,x_2)\frac{\partial_r g}{r},
\end{align*} 
es muss also nur $\partial_r g$ bestimmt werden.

Die entsprechenden Ableitung lauten
\begin{align*}
  \partial_r f_1(r) &= 
  \begin{cases}
    108,&
    \text{wenn }0\leq r\leq\frac{1}{6},\\
    6\alpha\beta(6r-1)^{\beta-1} +\frac{1}{r^2}, &
    \text{wenn } \frac{1}{6}\leq r\leq\frac{1}{3},\\
    (36\pi^2+\frac{1}{r^2})\cos(\pi(6r-2))+
    \frac{6\pi}{r}\sin(\pi(6r-2)), &
    \text{wenn } \frac{1}{3}\leq r\leq\frac{1}{2},\\
    -\left(6\alpha\beta\left( \frac{5}{2}-3r \right)^{\beta-1}+
    \frac{1}{r^2}\right),&
    \text{wenn } \frac{1}{2}\leq r\leq\frac{5}{6},\\
    -\left( \left( 18\pi^2+\frac{1}{2r^2} \right)\cos(\pi(6r-5))
    +\frac{1}{2r^2}+\frac{3\pi}{r}\sin(\pi(6r-5))\right),
    &\text{wenn } \frac{5}{6}\leq r\leq 1,
  \end{cases}\\
  \partial_r u_1(r) &= 
  \begin{cases}
    0,&\text{wenn }0\leq r\leq\frac{1}{6},\\
    6\beta(6r-1)^{\beta-1}, &\text{wenn } \frac{1}{6}\leq r\leq\frac{1}{3},\\
    0, &\text{wenn } \frac{1}{3}\leq r\leq\frac{1}{2},\\
    -6\beta\left( \frac{5}{2}-3r \right)^{\beta-1},&
    \text{wenn } \frac{1}{2}\leq r\leq\frac{5}{6},\\
    0,&\text{wenn } \frac{5}{6}\leq r,
  \end{cases}\\
  \partial_r f_2(r) &= 
  \begin{cases}
    -\frac{12}{(1-\beta)^2},&\text{wenn }0\leq r\leq\frac{1-\beta}{2},\\
    -\frac{\alpha}{\beta}-\frac{1}{r^2},&
    \text{wenn } \frac{1-\beta}{2}\leq r\leq \frac{1+\beta}{2},\\
    -\frac{4}{(1-\beta)^3}\left( 32r-9(\beta+3)+\frac{3\beta+1}{r^2} \right),&
    \text{wenn } \frac{1+\beta}{2}\leq r\leq 1,\\
  \end{cases}\\
  \partial_r u_2(r) &= 
  \begin{cases}
    0,&\text{wenn }0\leq r\leq\frac{1-\beta}{2},\\
    -\frac{1}{\beta},&
    \text{wenn } \frac{1-\beta}{2}\leq r\leq \frac{1+\beta}{2},\\
    0,&\text{wenn } \frac{1+\beta}{2}\leq r.
  \end{cases}
\end{align*}

Mit diesen Informationen kann mit \texttt{computeExactEnergyBV.m} die exakte 
Energie berechnet werden und somit durch eintragen der exakten Energie
in die Variable \texttt{exactEnergy} im Benchmark und setzen der Flag
\texttt{useExactEnergy=true} das Experiment durch anschließendes Ausführen
von \texttt{startAlgorithmCR.m} gestartet werden.


\todo[inline]{Take a look at denoising and read [ROF92] for that.}


%% --- APPENDIX -------------------------------------------------------------
%
%\appendix
%\chapter{Appendix}
%  %tien schicken spätestens am Wochenende vor der Präsi, CC vor der Präsi
  %die fertige Präsi + akuteller Stand der Arbeit schicken

\begin{frame}
  L2 Sprünge vielleicht auswerten (bleiben sie konstant\ldots, if we consider
  them, it becomes conforming

  die L2 Sprung entwicklung einiger experiment (iteration auswerten, iteration 
  selbst und Afem loop insgesamt). bleiben sicherlich konstant oder sowas
\end{frame}

\begin{frame}
  differenct norm for termination criteria comparison (energy difference not
  good because oscillations, everything else (dont show L2 error squared)
  is similar, just different height
\end{frame}

\begin{frame}
  compare times without preallocating and with (inform about the extreme 
  improvement of performance
\end{frame}

\begin{frame}
  Let $u_P:[0,\infty)\to\Rbb$ with $u_P(r)=0$ for $r\geq 1$,
  and, for all $x\in\Omega$, $u(x)= u_P\big(|x|\big)$. 
  \pause
  Furthermore, assume the existence  of $\partial_r u_P$ a.e.\ in $[0,\infty)$,
  the existence of the derivative of
  \begin{align*}
    \operatorname{sgn}\big(\partial_r u_P(r)\big)
    \coloneqq
    \begin{cases}
      -1 &\text{für }\partial_r u_P(r)<0,\\
      x\in[0,1] &\text{für }\partial_r u_P(r)=0,\\ 
      1 &\text{für }\partial_r u_P(r)>0.
    \end{cases}
  \end{align*}
  a.e.\ in $[0,\infty)$, and
  that $\operatorname{sgn}\big(\partial_r u_P(r)\big)/r\to 0$ as $r\to 0$.
  \pause
  For all $r\in[0,\infty)$, define 
  \begin{align*}
    f_P(r)
    \coloneqq 
    \alpha u_P(r) - \partial_r\left(\operatorname{sgn}\big(\partial_r u_P(r)\big)\right) 
    - \frac{\operatorname{sgn}\big(\partial_r u_P(r)\big)}{r}
  \end{align*}
  \pause
  Then $u$ solves the continuous problem
  on $\Omega\supseteq \left\{w\in\Rbb^2\,\middle|\, |w|\leq 1\right\}$ if
  the input signal is $f(x)\coloneqq f_P\big(|x|\big)$.
\end{frame}


% --- BIBLIOGRAPHY -------------------------------------------------------------

\printbibliography
\addcontentsline{toc}{chapter}{Literatur}

\vfill

\noindent\textbf{\huge{Anhang}} 

\vspace{.8cm}

\noindent Datenträger mit dem Programm und einer digitalen Version dieser Arbeit

\vspace{3cm}

\chapter*{Selbstständigkeitserklärung}

Ich erkläre, dass ich die vorliegende Arbeit selbständig verfasst und noch nicht 
für andere Prüfungen eingereicht habe. Sämtliche Quellen, einschließlich
Internetquellen, die unverändert oder abgewandelt wiedergegeben werden,
insbesondere Quellen für Texte, Grafiken, Tabellen und Bilder, sind als solche
kenntlich gemacht. Mir ist bekannt, dass bei Verstößen gegen diese Grundsätze ein
Verfahren wegen Täuschungsversuchs bzw.\ Täuschung eingeleitet wird. 
\bigbreak
\noindent Berlin, den \today, 
\end{document}
