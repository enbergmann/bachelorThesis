% --- SETTINGS -----------------------------------------------------------------

\documentclass[draft=false,twoside,11pt]{scrreprt}
\newcommand{\lang}{ngerman}
\usepackage[\lang]{babel}
\usepackage[a4paper, 
            width = 150mm, top = 25mm, bottom = 25mm,
            bindingoffset = 6mm]
            {geometry}
% \usepackage{fancyhdr}
\pagestyle{headings}

% \setlength{\parindent}{0pt}

\documentclass[xcolor=svgnames,english]{beamer}
\usetheme{DarkHU}

%\usepackage{pgfpages}
%\pgfpagesuselayout{2 on 1}[a4paper,border shrink=5mm]

% ============================================================
% === USER-DEFINED COMMANDS ==================================
% ============================================================

\usepackage{amssymb,mathabx,textcomp,bm}

\newcommand{\mathbox}[2]{\makebox[#1]{$\displaystyle #2$}}
\newcommand{\Stackrel}[2]{\mathbox{1.25em}{\stackrel{#1}{#2}}}


% === INTEGRAL MEAN ==========================================

\def\Xint#1{\mathchoice
{\XXint\displaystyle\textstyle{#1}}%
{\XXint\textstyle\scriptstyle{#1}}%
{\XXint\scriptstyle\scriptscriptstyle{#1}}%
{\XXint\scriptscriptstyle\scriptscriptstyle{#1}}%
\!\int}
\def\XXint#1#2#3{{\setbox0=\hbox{$#1{#2#3}{\int}$}
\vcenter{\hbox{$#2#3$}}\kern-.5\wd0}}
\newcommand{\intmean}{\Xint-}

% === PERFECT BULLET =========================================

\let\oldbullet\bullet
\newlength{\raisebulletlen}
\setbox1=\hbox{$\bullet$}\setbox2=\hbox{\tiny$\bullet$}
\setlength{\raisebulletlen}{\dimexpr0.5\ht1-0.5\ht2}
\renewcommand\bullet{\raisebox{\raisebulletlen}{\,\tiny$\oldbullet$}\,}

% === ORTHOGONAL SUM =========================================

\newcommand\orthsum{
\tikz[baseline=(A.base),font=\small]
     \node[draw,ellipse,inner sep=0.15ex] (A){$\perp$};
}

% === CALIGRAPHIC LETTERS ====================================

\newcommand\Acal{\mathcal{A}}
\newcommand\Bcal{\mathcal{B}}
\newcommand\Ccal{\mathcal{C}}
\newcommand\Dcal{\mathcal{D}}
\newcommand\Ecal{\mathcal{E}}
\newcommand\Fcal{\mathcal{F}}
\newcommand\Jcal{\mathcal{J}}
\newcommand\Kcal{\mathcal{K}}
\newcommand\Lcal{\mathcal{L}}
\newcommand\Mcal{\mathcal{M}}
\newcommand\Ncal{\mathcal{N}}
\newcommand\Ocal{\mathcal{O}}
\newcommand\Pcal{\mathcal{P}}
\newcommand\Rcal{\mathcal{R}}
\newcommand\Tcal{\mathcal{T}}

% === MATHBB LETTERS =========================================

\newcommand\C{\mathbb{C}}
\newcommand\N{\mathbb{N}}
\newcommand\R{\mathbb{R}}
\newcommand\T{\mathbb{T}}
\newcommand\K{\mathbb{K}}

% === MATH OPERATORS =========================================

\DeclareMathOperator*{\argmin}{argmin} % the * adjusts MathOperator for indizes beneath
\DeclareMathOperator{\bisec}{bisec}
\DeclareMathOperator{\cond}{cond}
\DeclareMathOperator{\Conv}{conv}
\DeclareMathOperator{\Curl}{Curl}
\DeclareMathOperator{\curl}{curl}
\DeclareMathOperator{\dev}{dev}
\DeclareMathOperator{\diag}{diag}
\DeclareMathOperator{\diam}{diam}
\DeclareMathOperator{\Dim}{dim}
\DeclareMathOperator{\Div}{div}
\DeclareMathOperator{\Dist}{dist}
\DeclareMathOperator{\esssup}{ess\ supp}
\DeclareMathOperator{\grad}{\nabla}
\DeclareMathOperator{\Int}{int}
\DeclareMathOperator{\Ker}{ker}
\DeclareMathOperator{\Mid}{mid}
\DeclareMathOperator{\Osc}{osc}
\DeclareMathOperator{\Red}{red}
\DeclareMathOperator{\Ref}{Ref}
\DeclareMathOperator{\Res}{Res}
\DeclareMathOperator{\sign}{sgn}
\DeclareMathOperator{\Span}{span}
\DeclareMathOperator{\supp}{supp}
\DeclareMathOperator{\tr}{tr}
\DeclareMathOperator{\Width}{width}
\DeclareMathOperator*{\arginf}{arginf}

% === COMMANDS WITH INPUT ARGUMENTS ==========================

\newcommand\abs[1]{\lvert #1 \rvert}
\newcommand\average[1]{\langle #1 \rangle}
\newcommand\jump[1]{\lbrack #1 \rbrack}
\newcommand\NormEnergy[1]{\big\vvvert #1 \big\vvvert}
\newcommand\Norm[2]{\lVert #1 \rVert_{#2}}
\newcommand\scal[2]{\left\langle #1 , #2 \right\rangle}

% === SPACES WITH NAMES ======================================

\newcommand{\CONF}{\textup{C}}
\newcommand{\NC}{\textup{NC}}
\newcommand{\CR}{\textup{CR}}
\newcommand{\RT}{\textup{RT}}

% === GENERAL STUFF ==========================================

\newcommand{\splitter}{\,:\,} % split for definition of sets
\renewcommand{\d}{\, \textup{d}} % d for differentials, i.e., \d x


% === JUST TO COMPILE CHAPTERS 4  AND 5 ==========================================

\newcommand{\LO}[0]{L^2(\Omega)}
\newcommand{\restrict}[2]{\left. #1 \right\lvert_{#2}}
\DeclareMathOperator{\mPkt}{mid}
\newcommand{\nnorm}[2]{\left\lVert #1 \right\rVert_{#2}}
\newcommand{\volf}[2]{\nnorm{h_{#1}f}{#2}}
\newcommand{\volO}[1]{\nnorm{h_{#1}f}{\LO}}
\DeclareMathOperator\Kern{ker}
\DeclareMathOperator\setspan{span}

% === JUST TO COMPILE CHAPTER 6 ==========================================


\newcommand\NormL[3]{\Norm{#1}{L^{#2}\left( { #3 } \right)}}
 \newcommand\NormH[3]{\Norm{#1}{H^{#2}\left( { #3 } \right)}}
 \newcommand\NormHdiv[1]{\Norm{#1}{H(\div)}}
 \newcommand\NormW[3]{\Norm{#1}{W^{#2}\left( { #3 } \right)}}
\newcommand\NormLDom[2]{\NormL{#1}{#2}{\Omega}}
 \newcommand\NormHDom[2]{\NormH{#1}{#2}{\Omega}}
 \newcommand\NormHdivDom[1]{\NormHdiv{#1}}
 \newcommand\NormWDom[2]{\NormW{#1}{#2}{\Omega}}
 \newcommand\NormLz[2]{\NormL{#1}{2}{#2}}
 \newcommand\NormHz[2]{\NormH{#1}{2}{#2}}
 \newcommand\NormWkp[4]{\NormW{#1}{{#2,#3}}{#4}}
 \newcommand\NormWkpDom[3]{\NormWDom{#1}{{#2,#3}}}
 \newcommand\NormLzDom[1]{\NormLDom{#1}{2}}
 \newcommand\NormHzDom[1]{\NormHDom{#1}{2}}
\newcommand\NormMax[2]{\Norm{#1}{\C({#2})}}


% === JUST TO COMPILE CHAPTER 0 ==========================================

\newcommand{\dist}{\mathrm{dist}}
\newcommand{\fa}{\;\forall\;}
\newcommand{\Lra}{\Leftrightarrow}
\newcommand{\id}{\mathrm{id}}

% === JUST TO COMPILE EXERCISES ==========================================

\newcommand{\al}{\alpha}
\newcommand{\be}{\beta}
\newcommand{\ga}{\gamma}
\newcommand{\de}{\delta}
\newcommand{\eps}{\varepsilon}
\newcommand{\vphi}{\varphi}
\newcommand{\la}{\lambda}
\newcommand{\om}{\omega}
 

\usepackage[backend=biber, style = alphabetic]{biblatex}
\usepackage{csquotes} % to fix biblatex warning expecting csquotes
\addbibresource{sourcesBergmannBT.bib}


\usepackage[T1]{fontenc}
\usepackage{lmodern}

%\usepackage{eurosym}

%\usepackage[backend=biber, style=authortitle-icomp]{biblatex}
%\usepackage[babel,english=guillemets]{csquotes}
%\addbibresource{bib.bib}

%\usepackage{algorithm}
%\usepackage{algpseudocode}

\usepackage{tikz}
%\usetikzlibrary{shapes, fit, positioning}
%\usepackage{hyperref}

%%%%%%%%%%%%%%%%%%%%%%%
% UTF-8 input encoding
\usepackage[utf8]{inputenc}
\usepackage{microtype}\usepackage{blindtext}

% Page layout formating of koma-script
\usepackage[automark]{scrpage2}
\usepackage{lmodern}

% Language support
\usepackage[english]{babel}

% AMSMath package
\usepackage{amsmath}
\usepackage{amsfonts}
\usepackage{amssymb}
\usepackage{amstext}
\usepackage{amsopn}
\usepackage{amsthm}
\usepackage{url}
\usepackage{listings}
\usepackage{xspace}
\usepackage{mathabx}
%\usepackage{mathptmx}
\usepackage{esint}
\usepackage{graphicx}
\usepackage{mathtools}
\usepackage{hyperref}

%%%%%%%%%%%%%%%%%%%%%%%%%
\usepackage{enumitem,amssymb}
\newlist{todolist}{itemize}{2}
\setlist[todolist]{label=$\square$}
\usepackage{pifont}
\newcommand{\cmark}{\ding{51}}%
\newcommand{\xmark}{\ding{55}}%
\newcommand{\done}{\rlap{$\square$}{\raisebox{2pt}{\large\hspace{1pt}\cmark}}%
\hspace{-2.5pt}}
\newcommand{\wontfix}{\rlap{$\square$}{\large\hspace{1pt}\xmark}}
%%%%%%%%%%%%%%%%%%%%%%%%%%%


\parindent 0ex


\providecommand{\integral}[3]{\ensuremath{\int\limits_{#1} \! {#2} \, \mathrm{d}#3}}
\providecommand{\T}{\ensuremath{\mathcal{T}}}
\providecommand{\Hh}{\ensuremath{\mathrm{H}}}
\providecommand{\Ll}{\ensuremath{\mathrm{L}}}
 
%%%%%%%%%%
\setbeamercolor{green box}{use=base, fg=base.fg, bg=green}
\newenvironment{question}[1][center]{\begin{beamercolorbox}[#1, rounded=true,
  sep=1pt]{green box}}{\end{beamercolorbox}}

\setbeamercolor{yellow box}{use=base, fg=base.fg, bg=yellow}
\newenvironment{emphbox}[1][center]{\begin{beamercolorbox}[#1, rounded=true,
  sep=1pt]{yellow box}}{\end{beamercolorbox}}

\renewcommand{\emph}[1]{\textcolor{magenta}{\bfseries #1}}
\providecommand{\abs}[1]{\ensuremath{\left\lvert#1\right\rvert}}

%\setbeamercovered{transparent}
%\setbeamertemplate{navigation symbols}{}

%\expandafter\def\expandafter\insertshorttitle\expandafter{%
%  \insertshorttitle\hfill%
%    \insertframenumber\,/\,\inserttotalframenumber}

%\addtobeamertemplate{navigation symbols}{}{%
%  \usebeamerfont{footline}%
%  \usebeamercolor[fg]{footline}%
%  \hspace{1em}%
%  \insertframenumber/\inserttotalframenumber}

%\expandafter\def\expandafter\insertshorttitle\expandafter{%
%  \insertshorttitle\hfill%
%    \insertframenumber\,/\,\inserttotalframenumber}

%%%%%%%%%%%%%%%%%%%%%%%%%%%%%%%%%%%%%%%%%%%%%%%%%%%%%%%%%%%%
\author{Enrico Bergmann}
\title{Minimization of a Functional on the Space of BV Functions and 
Nonconforming Discretization of the Problem}
\subtitle{I. Theoretical Basics and Characterization of Minimizers}
\institute{Humboldt-Universität zu Berlin}
\date{January 6, 2021}
\titlegraphic{\includegraphics[height=1.6cm]{hukombi_bbw_rgb_op}}
%%%%%%%%%%%%%%%%%%%%%%%%%%%%%%%%%%%%%%%%%%%%%%%%%%%%%%%%%%%%

\usepackage{microtype}

\begin{document}

\begin{frame}[plain, noframenumbering]
	\maketitle
\end{frame}
  
\begin{frame}{Table of Contents}
  \tableofcontents
\end{frame}

%%%%%%%%%%%%%%%%%%%%%%%%%%%%%%%%%%%%%%%%%%%%%%%%%%%%%%%%%%%%%%%%%%%%%%%%%%%%%%

\section{Introduction}
\begin{frame}[noframenumbering]{Table of Contents}
  \tableofcontents[currentsection, currentsubsection]
\end{frame}

\begin{frame}
  \fullcite[Chapter 10, p. 297-319]{Bar15}

  \bigskip
  \pause

  Let $\Omega\subset\Rbb^n$ be a bounded polyhedral Lipschitz domain.

  \medskip

  For given $g\in L^2(\Omega)$ and $\alpha>0$ minimize the functional 
  \begin{align*}
    I(v)=|v|_{\BV(\Omega)}+\frac{\alpha}{2}\Vert v-g\Vert^2
  \end{align*}
  amongst all $v\in \BV(\Omega)\cap L^2(\Omega)$.
\end{frame}

\begin{frame}{Functions of Bounded Variation}
  A function $v\in L^1(\Omega)$ with distributional derivative
  $Dv:C^{\infty}_C(\Omega;\Rbb^n)\to\Rbb$ is said to be of bounded variation if 
  there exists $c>0$ such that 
  \begin{align*}
    \langle Dv, \phi\rangle\coloneqq -\int_\Omega v\Div(\phi)\dx\leq
    c\Vert\phi\Vert_{L^\infty(\Omega)}
  \end{align*}
  for all $\phi\in C^1_C(\Omega;\Rbb^n)$.

  \pause  

  The minimal constant $c\geq 0$ satisfying this property is called 
  total variation of $Dv$ and is given by
  \begin{align*}
    |v|_{\BV(\Omega)} = \sup_{\substack{\phi\in C^1_C(\Omega;\Rbb^n)\\
    \Vert\phi\Vert_{L^\infty(\Omega)}\leq 1}}-\int_\Omega v\Div (\phi)\dx.
  \end{align*}

  \pause

  The space of all such functions is denoted by $\BV(\Omega)$.
\end{frame}

\begin{frame}{Properties of $\BV(\Omega)$}
  $\BV(\Omega)$ is a Banach space equipped with the norm
  \begin{align*}
    \Vert v \Vert_{\BV(\Omega)} \coloneqq \Vert v\Vert_{L^1(\Omega)} +
    |v|_{\BV(\Omega)}\quad\text{for all } v\in\BV(\Omega).
  \end{align*}
  
  \pause
  \medskip
  $W^{1,1}(\Omega)\subset\BV(\Omega)$ with $\Vert v\Vert_{\BV(\Omega)}=
  \Vert v\Vert_{W^{1,1}(\Omega)}$ for all $v\in W^{1,1}(\Omega)$.
\end{frame}

\begin{frame}{Notions of convergence on $\BV(\Omega)$}
  Let $(v_n)_{n\in\Nbb}\subset \BV(\Omega)$ and $v\in \BV(\Omega)$ such that
  $v_n\rightarrow v$ in $L^1(\Omega)$ as $n\rightarrow\infty$.
  \pause
  \begin{itemize}
    \item[(i)]
      $(v_n)_{n\in\Nbb}$ converges intermediately or strictly to $v$
      if $|v_n|_{\BV(\Omega)}\rightarrow |v|_{\BV(\Omega)}$ as
      $n\rightarrow\infty$.
      \pause
    \item[(ii)] $(v_n)_{n\in\Nbb}$ converges weakly to
      $v$ if
      $\langle Dv_n,\phi\rangle\rightarrow \langle Dv,\phi\rangle$ 
      for all $\phi\in C_0(\Omega;\Rbb^n)$ as 
      $n\rightarrow\infty$.
  \end{itemize}
\end{frame}

\begin{frame}{Further Properties of $\BV(\Omega)$}
  $C^\infty(\overline\Omega)$ and $C^\infty(\Omega)\cap\BV(\Omega)$ are dense
  in $\BV(\Omega)$ with respect to intermediate convergence.
  
  \pause
  \bigskip

  The embedding $\BV(\Omega)\to L^p(\Omega)$ is continuous for
  $1\leq p\leq n/(n-1)$ and compact for $1\leq p< n/(n-1)$.
  
  \pause
  \bigskip

  There exists a linear operator $T:\BV(\Omega)\to L^1(\partial\Omega)$
  such that $T(v) = v|_{\partial\Omega}$ for all $v\in\BV(\Omega)\cap
  C(\overline\Omega)$.

  $T$ is continuous with respect to intermediate convergence in $\BV(\Omega)$
  but not with respect to weak convergence in $\BV(\Omega)$. 
\end{frame}



\section{Continuous Problem}
In diesen Kapitel wollen wir für einen Parameter $\alpha\in\Rbb_+$ und eine
Funktion $f\in L^2(\Omega)$ folgendes Minimierungsproblem untersuchen.

\begin{problem}\label{prob:continuousProblem}
  Finde $u\in \BV(\Omega)\cap L^2(\Omega)$, sodass
  $u$ das Funktional
  \begin{align}\label{eq:continuousProblem}
    E(v)\coloneqq \frac{\alpha}{2}\Vert v\Vert^2 + |v|_{\BV(\Omega)}
    +\Vert v\Vert_{L^1(\partial\Omega)}-\int_\Omega fv\dx
  \end{align}
  unter allen $v\in\BV(\Omega)\cap L^2(\Omega)$ minimiert.
\end{problem}
Dabei ist der Term $\Vert v\Vert_{L^1(\partial\Omega)}$ wohldefiniert, da
nach \cite[S. 400, Theorem 10.2.1]{ABM14} eine lineare, stetige Abbildung
$T:\BV(\Omega)\to L^1(\partial\Omega)$ existiert mit $T(v) =
v|_{\partial\Omega}$ für alle $v\in\BV(\Omega)\cap C(\overline\Omega)$.
Wie in \Cref{chap:introduction} beschrieben, hat \Cref{prob:continuousProblem}
für homogene Randdaten die gleichen Minimierer wie das ROF-Modell-Problem mit
Eingangssignal $g\in L^2(\Omega)$, wenn $f=\alpha g$.

\begin{remark}
  Für $d\in\{2,3\}$ und ein beschränktes Lipschitz-Gebiet $U\subset\Rbb^d$ ist
  nach \cite[S. 302, Remark 10.5 (i)]{Bar15} die Einbettung
  $\BV(U)\hookrightarrow L^p(U)$ stetig, wenn $1\leq p\leq d/(d-1)$. 
  % Nach \cite[S. 399, Theorem 10.1.3]{ABM14} gilt diese Aussage auch für
  % $d\in\Nbb$, wenn $U\subset\Rbb^d$ offen, beschränkt und 1-regulär ist.
  Damit ist $\BV(\Omega)$ Teilmenge von $L^2(\Omega)$ und die Lösung von
  \Cref{prob:continuousProblem} kann in $\BV(\Omega)$ gesucht werden. 
  Wir vernachlässigen diese Vereinfachung und erreichen dadurch, dass alle
  Aussagen dieses Kapitels auch gelten würden, wenn
  $\Omega\subset\Rbb^d$ mit $d\in\Nbb$ ein beschränktes Lipschitz-Gebiet wäre.
\end{remark}

In den folgenden Abschnitten zeigen wir, dass
\Cref{prob:continuousProblem} eine eindeutige Lösung $u\in\BV(\Omega)\cap
L^2(\Omega)$ besitzt. 
Außerdem beschreiben wir, wie $f$ für eine gegebene Lösung $u$ konstruiert
werden kann und welche Eigenschaften $u$ dafür erfüllen muss.


\section{Existenz eines eindeutigen Minimierers}
Zunächst zeigen wir, dass \Cref{prob:continuousProblem} eine Lösung besitzt.
Dafür benötigen wir die folgende Formulierung der Youngschen Ungleichung.

\begin{lemma}[Youngsche Ungleichung]
  \label{lem:young}
  Seien $a,b\in\Rbb$ und $\varepsilon\in\Rbb_+$ beliebig. Dann gilt
  \begin{align*}
    ab\leq\frac{1}{\varepsilon}a^2+\frac{\varepsilon}{4}b^2. 
  \end{align*}
\end{lemma}

Außerdem wird im Beweis folgende Aussage benötigt, die direkt aus \cite[S. 183,
Theorem 1]{EG92} folgt, da
$0\in\BV\!\left(\Rbb^d\setminus\overline\Omega\right)$ mit
$|0|_{\BV\!\left(\Rbb^d\setminus\overline\Omega\right)}=0$ und
$0|_{\partial\Omega}=0$.

\begin{lemma}
  \label{lem:bvExtension}
  Sei $v\in\BV(\Omega)$.
  Definere die Fortsetzung $\tilde{v}$ von $v$ für alle $x\in\Rbb^d$ durch
  \begin{align*}
    \tilde{v}(x)\coloneqq
    \begin{cases}
      v(x),  &\text{ falls } x\in\Omega,\\
      0,     &\text{ falls } x\in\Rbb^d\setminus\overline\Omega.
    \end{cases} 
  \end{align*}
  Dann gilt $\tilde{v}\in\BV\!\left(\Rbb^d\right)$ und
  $\left|\tilde{v}\right|_{\BV\!\left(\Rbb^d\right)}
  = |v|_{\BV(\Omega)}+\Vert v\Vert_{L^1(\partial\Omega)}$.
\end{lemma}

\begin{theorem}[Existenz einer Lösung]
  \label{thm:contProblemExistence}
  \Cref{prob:continuousProblem} besitzt eine Lösung \\$u\in\BV(\Omega)\cap
  L^2(\Omega)$.
\end{theorem}

\begin{proof}
  Die Beweisidee ist die Anwendung der direkten Methode der Variationsrechnung
  (cf.\ z.B.\ \cite{Dac89}) unter Nutzung der in \Cref{sec:bvFunctions}
  aufgeführten Eigenschaften der schwachen Konvergenz in $\BV(\Omega)$.

  Für alle $v\in L^2(\Omega)\subseteq L^1(\Omega)$ gilt mit der Hölderschen
  Ungleichung für $p=q=2$, dass
  \begin{equation}\label{eq:hoelderL2BiggerL1}
    \Vert v\Vert_{L^1(\Omega)} 
    = \Vert 1\cdot v\Vert_{L^1(\Omega)}
    \leq \Vert 1\Vert\Vert v\Vert
    =\sqrt{|\Omega|} \Vert v\Vert.
  \end{equation}
  Dann folgt für das Funktional $E$ in \eqref{eq:continuousProblem} für alle
  $v\in \BV(\Omega)\cap L^2(\Omega)$ durch die Cauchy-Schwarzsche Ungleichung,
  die Youngsche Ungleichung aus \cref{lem:young} und Ungleichung
  \eqref{eq:hoelderL2BiggerL1}, dass

  \begin{equation}
    \label{eq:contProbBddFromBelow}
    \begin{aligned}
      E(v)&=\frac{\alpha}{2}\Vert v\Vert^2 + |v|_{\BV(\Omega)}
      +\Vert v\Vert_{L^1(\partial\Omega)}-\int_\Omega fv\dx\\
      &\geq 
      \frac{\alpha}{2}\Vert v\Vert^2 + |v|_{\BV(\Omega)}
      +\Vert v\Vert_{L^1(\partial\Omega)}
      -\Vert f\Vert\Vert v\Vert\\
      &\geq 
      \frac{\alpha}{2}\Vert v\Vert^2 + |v|_{\BV(\Omega)}
      +\Vert v\Vert_{L^1(\partial\Omega)}
      -\frac{1}{\alpha}\Vert f\Vert^2
      -\frac{\alpha}{4}\Vert v\Vert^2\\
      &\geq 
      \frac{\alpha}{4}\Vert v\Vert^2 + |v|_{\BV(\Omega)}
      +\Vert v\Vert_{L^1(\partial\Omega)}-\frac{1}{\alpha}\Vert
      f\Vert^2\\
      &\geq 
      \frac{\alpha}{4|\Omega|}\Vert v\Vert_{L^1(\Omega)}^2 + |v|_{\BV(\Omega)}
      +\Vert v\Vert_{L^1(\partial\Omega)}-\frac{1}{\alpha}\Vert
      f\Vert^2\\
      &\geq -\frac{1}{\alpha}\Vert f\Vert^2.
    \end{aligned}
  \end{equation}
  Somit ist $E$ nach unten beschränkt, was die Existenz einer infimierenden
  Folge $(u_n)_{n\in\Nbb}\subset\BV(\Omega)\cap L^2(\Omega)$ von $E$ 
  impliziert, das heißt
  $(u_n)_{n\in\Nbb}$ erfüllt $$\lim_{n\rightarrow\infty}E(u_n) =
  \inf_{v\in\BV(\Omega)\cap L^2(\Omega)}E(v).$$ 
  Ungleichung \eqref{eq:contProbBddFromBelow} impliziert außerdem, dass
  $E(u_n)\to\infty$ für $n\to\infty$, falls $|u_n|_{\BV(\Omega)}\to\infty$ oder
  $\Vert u_n\Vert_{L^1(\Omega)}\to\infty$ für $n\to\infty$. 
  Daraus folgt insbesondere, dass $E(u_n)\to\infty$ für $n\to\infty$, falls
  $\Vert u_n\Vert_{\BV(\Omega)}\to\infty$ für $n\to\infty$ .
  Deshalb muss die Folge $(u_n)_{n\in\Nbb}$ beschränkt in $\BV(\Omega)$ sein.
  Nun garantiert \cref{thm:compactness} die Existenz einer in $\BV(\Omega)$
  schwach konvergenten Teilfolge $(u_{n_k})_{k\in\Nbb}$ von $(u_n)_{n\in\Nbb}$
  mit schwachen Grenzwert $u\in\BV(\Omega)$. 
  Ohne Beschränkung der Allgemeinheit ist
  $(u_{n_k})_{k\in\Nbb}=(u_n)_{n\in\Nbb}$.
  Aus der schwachen Konvergenz von $(u_n)_{n\in\Nbb}$ in $\BV(\Omega)$ gegen
  $u$ folgt nach Definition, dass $(u_n)_{n\in\Nbb}$ stark, und damit
  insbesondere auch schwach, in $L^1(\Omega)$ gegen $u$ konvergiert.

  Weiterhin folgt aus (\ref{eq:contProbBddFromBelow}), dass
  $E(v)\rightarrow\infty$ für $\Vert v\Vert\rightarrow\infty$. 
  Somit muss $(u_n)_{n\in\Nbb}$ auch beschränkt sein bezüglich der Norm
  $\Vert\bullet\Vert$ und besitzt deshalb, wegen der Reflexivität von
  $L^2(\Omega)$, eine Teilfolge (ohne Beschränkung der Allgemeinheit weiterhin
  bezeichnet mit $(u_n)_{n\in\Nbb}$), die in $L^2(\Omega)$ schwach gegen einen
  Grenzwert $\overline{u}\in L^2(\Omega)$ konvergiert. 
  Damit gilt für alle $w\in L^2(\Omega)\cong L^2(\Omega)^\ast$ und, da
  $L^\infty(\Omega)\subseteq L^2(\Omega)$, insbesondere auch für alle $w\in
  L^\infty(\Omega)\cong L^1(\Omega)^\ast$, dass 
  \begin{align*}
    \lim_{n\to\infty}\int_\Omega u_n w\dx =\int_\Omega \overline{u} w\dx.
  \end{align*}
  Das bedeutet, dass $(u_n)_{n\in\Nbb}$ auch schwach in $L^1(\Omega)$ gegen
  $\overline{u}\in L^2(\Omega)\subseteq L^1(\Omega)$ konvergiert. 
  Da schwache Grenzwerte eindeutig bestimmt sind, gilt insgesamt $u=\overline u
  \in L^2(\Omega)$, das heißt $u\in\BV(\Omega)\cap
  L^2(\Omega)$.
  Nun definieren wir für alle
  $n\in\Nbb$ und für alle 
  $x\in\Rbb^d$ die Fortsetzungen
  \begin{align*}
    \tilde{u}_n(x)
    &\coloneqq
    \begin{cases}
      u_n(x),  &\text{ falls } x\in\Omega,\\
      0,     &\text{ falls } x\in\Rbb^d\setminus\overline\Omega
    \end{cases} 
    &&\text{und }
    &\tilde{u}(x)
    &\coloneqq
    \begin{cases}
      u(x),  &\text{ falls } x\in\Omega,\\
      0,     &\text{ falls } x\in\Rbb^d\setminus\overline\Omega.
    \end{cases} 
  \end{align*}
  Dann gilt nach \cref{lem:bvExtension} sowohl
  \begin{align*}
    \tilde{u}_n
    &\in
    \BV\!\big(\Rbb^d\big)
    &&\text{und}
    &\left|\tilde{u}_n\right|_{\BV\!\left(\Rbb^d\right)} 
    &= 
    |u_n|_{\BV(\Omega)}+\Vert u_n\Vert_{L^1(\partial\Omega)}
    \quad\text{für alle }n\in\Nbb \text{ als auch}\\
    \tilde{u}
    &\in
    \BV\!\big(\Rbb^d\big)
    &&\text{und}
    &\left|\tilde{u}\right|_{\BV\!\left(\Rbb^d\right)} 
    &=
    |u|_{\BV(\Omega)}+\Vert u\Vert_{L^1(\partial\Omega)}.
  \end{align*}
  Da $(u_n)_{n\in \Nbb}$ infimierende Folge von $E$ ist, muss die Folge
  \begin{align*}
    \left(\left|\tilde{u}_n\right|_{\BV\!\left(\Rbb^d\right)}\right)_{n\in\Nbb} 
    = \left(|u_n|_{\BV(\Omega)}+
    \Vert u_n\Vert_{L^1(\partial\Omega)}\right)_{n\in\Nbb}
  \end{align*}
  beschränkt sein.
  Außerdem folgt aus den Definitionen von $\tilde{u}$ und 
  $\tilde{u}_n$ für alle $n\in\Nbb$ und der bereits bekannten Eigenschaft 
  $u_n\to u$ in $L^1(\Omega)$ für $n\to\infty$, dass
  \begin{align*}
    \left\Vert \tilde{u}_n - \tilde{u}\right\Vert_{L^1\!\left(\Rbb^d\right)} 
    &= \int_{\Rbb^d} \left|\tilde{u}_n - \tilde{u}\right|\dx
    = \int_\Omega |u_n - u|\dx
    = \Vert u_n - u\Vert_{L^1(\Omega)}\to 0\quad\text{für }n\to\infty,
  \end{align*}
  das heißt $\tilde{u}_n \to \tilde{u}$ in $L^1\left(\Rbb^d\right)$ für
  $n\to\infty$.
  Insgesamt ist also $\left(\tilde{u}_n\right)_{n\in\Nbb}$ eine Folge in
  $\BV\!\left(\Rbb^d\right)$, die in $L^1\!\left(\Rbb^d\right)$ gegen
  $\tilde{u}\in\BV\!\left(\Rbb^d\right)\subseteq L^1\!\left(\Rbb^d\right)$
  konvergiert und 
  $\sup_{n\in\Nbb} \left|\tilde{u}_n\right|_{\BV\!\left(\Rbb^d\right)}<\infty$
  erfüllt.
  Somit folgt mit
  \cref{thm:wlsc}  
  \begin{equation}
    \label{eq:wlscOfExtension}
    \begin{aligned}
      |u|_{\BV(\Omega)} +\Vert u\Vert_{L^1(\partial\Omega)}
      = \left|\tilde{u}\right|_{\BV\left(\Rbb^d\right)}
      &\leq\liminf_{n\to\infty}
      \left|\tilde{u}_n\right|_{\BV\left(\Rbb^d\right)}\\
      &= \liminf_{n\to\infty} \left(|u_n|_{\BV(\Omega)} +
      \Vert u_n\Vert_{L^1(\partial\Omega)}\right).
    \end{aligned}
  \end{equation}
  Die Funktionen $\Vert\bullet\Vert^2$ und $-\int_\Omega
  f\bullet\dx$ sind auf $L^2(\Omega)$ stetig und konvex, was impliziert,
  dass sie schwach unterhalbstetig auf $L^2(\Omega)$ sind. Da wir bereits
  wissen, dass $u_n\rightharpoonup u$ in $L^2(\Omega)$ für $n\to\infty$, 
  folgt
  \begin{align*}
    \frac{\alpha}{2}\Vert u\Vert-\int_\Omega fu\dx
    \leq \liminf_{n\to\infty}
    \left(\frac{\alpha}{2}\Vert u_n\Vert
    -\int_\Omega fu_n\dx\right).
  \end{align*}
  Damit und mit Ungleichung \eqref{eq:wlscOfExtension} gilt insgesamt
  \begin{align*}
    \inf_{v\in\BV(\Omega)\cap L^2(\Omega)}E(v)\leq
    E(u)\leq\liminf_{n\rightarrow\infty} E\left(u_n\right) =
    \lim_{n\rightarrow\infty}E\left(u_n\right) = \inf_{v\in\BV(\Omega)\cap
    L^2(\Omega)}E(v),
  \end{align*}
  das heißt $\min_{v\in\BV(\Omega)\cap L^2(\Omega)} E(v) = E(u)$.
\end{proof}

Nachdem wir gezeigt haben, dass für \Cref{prob:continuousProblem} eine
Lösung existiert, beweisen wir als nächstes ein Theorem, das direkt impliziert,
dass diese Lösung eindeutig ist. 
\begin{theorem}[Stabilität und Eindeutigkeit]
  \label{thm:contProbStabAndUniqu}
  Seien $u_1,u_2\in \BV(\Omega)\cap L^2(\Omega)$ die Minimierer des Problems
  \ref{prob:continuousProblem} mit $f_1,f_2\in L^2(\Omega)$ anstelle von $f$,
  das heißt für $\ell\in\{1,2\}$ minimiere $u_\ell$ das Funktional
  \begin{align*}
    E_\ell
    \coloneqq 
    \frac{\alpha}{2}\Vert v\Vert^2 + |v|_{\BV(\Omega)} 
    + \Vert v\Vert_{L^1(\partial\Omega)} - \int_\Omega f_\ell v\dx
  \end{align*}
  unter allen $v\in\BV(\Omega)\cap L^2(\Omega)$.
  Dann gilt 
  \begin{align*}
    \Vert u_1 - u_2\Vert 
    \leq\frac{1}{\alpha}\Vert f_1-f_2\Vert.
  \end{align*}
\end{theorem}

\begin{proof}
  Wir folgen der Argumentation im Beweis von \cite[S. 304, Theorem 10.6]{Bar15}.

  Zunächst definieren wir die Funktionale $F: L^2(\Omega)\to
  \Rbb\cup\{\infty\}$ und $G_\ell:L^2(\Omega)\to \Rbb$, $\ell\in\{1,2\}$, für
  alle $u\in L^2(\Omega)$ durch
  \begin{align*}
    F(u) 
    &\coloneqq 
    \begin{cases}
      |u|_{\BV(\Omega)} + \Vert u \Vert_{L^1(\partial\Omega)}, 
      &\text{ falls } u\in\BV(\Omega)\cap L^2(\Omega),\\
      \infty,&\text{ falls } u\in L^2(\Omega)\setminus\BV(\Omega)
    \end{cases}
    \quad\text{und }\\
    G_\ell(u)
    &\coloneqq 
    \frac{\alpha}{2}\Vert u\Vert^2 - \int_\Omega f_\ell u\dx.
  \end{align*}
  Damit gilt für $\ell\in\{1,2\}$ und alle
  $u\in\BV(\Omega)\cap L^2(\Omega)$, dass $E_\ell(u) =  F(u)+G_\ell(u)$.
  Für $\ell\in\{1,2\}$ ist $G_\ell$ Fr\'echet-differenzierbar und die
  Fr\'echet-Ableitung $G_\ell'(u): L^2(\Omega)\to\Rbb$ von $G_\ell$ an der
  Stelle $u\in L^2(\Omega)$ ist für alle $v\in L^2(\Omega)$
  gegeben durch
  \begin{align*}
    dG_\ell(u;v) = \alpha (u,v) - \int_\Omega f_\ell v\dx 
    = (\alpha u-f_\ell ,v).
  \end{align*}
  Das Funktional $F$ ist konvex, unterhalbstetig und es gilt $F\nequiv\infty$.
  Deshalb ist nach \cref{thm:subdifferentialMonotonicity} das Subdifferential
  $\partial F$ von $F$ monoton, das heißt für alle $\mu_\ell\in \partial
  F(u_\ell)$, $\ell\in\{1,2\}$, gilt
  \begin{align}\label{eq:stabilityAndUniqueness:monotonicityOfSubdifferential}
    (\mu_1-\mu_2,u_1-u_2)\geq 0.
  \end{align}
  Für $\ell\in\{1,2\}$ gilt, dass $E_\ell$ konvex ist und von $u_\ell$ in
  $\BV(\Omega)\cap L^2(\Omega)$ minimiert wird. 
  Außerdem gilt $E_\ell\nequiv\infty$ und $G_\ell$ ist stetig.
  Somit gilt nach \cref{thm:extremalprinciple}, 
  \cref{thm:subdifferentialSumRule} und \cref{thm:subdiffGateaux}, dass
  $0\in\partial E_\ell(u_\ell) = \partial F(u_\ell)+\partial
  G_\ell(u_\ell)=\partial F(u_\ell)+ \{G_\ell'(u_\ell)\}.$ 
  Daraus folgt
  $-G_\ell'(u_\ell)\in\partial F(u_\ell)$.
  Zusammen mit Ungleichung
  \eqref{eq:stabilityAndUniqueness:monotonicityOfSubdifferential}
  impliziert das
  \begin{align*}
    \big( -(\alpha u_1 - f_1) -(- (\alpha u_2 - f_2)), u_1 - u_2\big)
    \geq 0.
  \end{align*}
  Durch Umformen und Anwenden der Cauchy-Schwarzschen Ungleichung erhalten wir
  \begin{align*}
    \alpha \Vert u_1 - u_2 \Vert^2
    &\leq
    \big(f_1 -f_2, u_1-u_2 \big)\\
    &\leq
    \Vert f_1-f_2\Vert\Vert u_1 - u_2\Vert.
  \end{align*}
  Falls $\Vert u_1 - u_2 \Vert = 0$, gilt die zu zeigende Aussage.
  Ansonsten führt die Division durch $\alpha\Vert u_1 - u_2 \Vert\neq 0$ den
  Beweis zum Abschluss.
\end{proof}

\section{Konstruktion eines Experiments mit bekannter Lösung}
%\section{Konstruktion eines Eingangssignals für eine gegebene Lösung}
\label{sec:constructionInputSignal}

Für die numerische Untersuchung der primalen-dualen Iteration aus
\Cref{chap:algorithm} ist es sinnvoll
Eingangssignale $f$ für \Cref{prob:continuousProblem} gegeben zu haben, für die
der entsprechende gesuchte Minimierer bekannt ist. 
Als Grundlage für die Konstruktion solcher Signale nutzen wir die folgende 
Aussage von Professor Carstensen.

Sei $u:\Omega\to\Rbb$ gegeben als Funktion in Polarkoordinaten. 
Dabei beschränken wir uns auf vom Polarwinkel unabhängige Funktionen, das heißt
für alle $x\in\Omega$ gelte
$u(x)\coloneqq u_P\big(|x|\big)$ für $u_P:[0,\infty)\to\Rbb$. 
Weiterhin fordern wir $u_P(r)=0$ für $r\geq 1$ und die Existenz
der partiellen Ableitung $\partial_r u_P$ fast überall in $[0,\infty)$.
Außerdem existiere fast überall in $[0,\infty)$ die partielle Ableitung
des für $r\in[0,\infty)$ definierten Ausdrucks
\begin{align*}
  \sgn\big(\partial_r u_P(r)\big)
  \coloneqq
  \begin{cases}
    -1 &\text{für }\partial_r u_P(r)<0,\\
    x\in[0,1] &\text{für }\partial_r u_P(r)=0,\\ 
    1 &\text{für }\partial_r u_P(r)>0.
  \end{cases}
\end{align*}
Des Weiteren fordern wir $\sgn\big(\partial_r u_P(r)\big)/r\to 0$ für $r\to 0$, 
damit $f_P$ in der folgenden Definition stetig in $0$ fortgesetzt werden kann.
Sei $f_P:[0,\infty)\to\Rbb$ gegeben durch
\begin{align}
  \label{eq:constructionInputSignal}
  f_P(r)
  \coloneqq 
  \alpha u_P(r) - \partial_r\left(\sgn\big(\partial_r u_P(r)\big)\right) 
  - \frac{\sgn\big(\partial_r u_P(r)\big)}{r}
  \quad\text{für alle }r\in[0,\infty).
\end{align}
Dann ist $u$ Lösung von \Cref{prob:continuousProblem}, wenn das Eingangssignal
auf $\Omega\supseteq \left\{w\in\Rbb^2\,\middle|\, |w|\leq 1\right\}$ für fast
alle $x\in\Omega$ gegeben ist durch $f(x)\coloneqq f_P\big(|x|\big)$.

In unseren Experimenten wird uns die garantierte untere Energieschranke aus
\Cref{thm:gleb} interessieren. 
Da dieses Theorem für das Eingangssignal voraussetzt, dass $f\in
H^1_0(\Omega)$, müssen wir noch die folgenden Bedingungen an $u_P$ formulieren.
Hinreichend für $f\in H^1_0(\Omega)$ ist nach \Cref{eq:constructionInputSignal},
dass  $u_P$, $\partial_r\big(\sgn(\partial_r u_P)\big)$ und $\sgn(\partial_r
u_P)$ stetig sind und 
\begin{align*}
  u_P(1)
  =
  \partial_r\left( \sgn\big(\partial_r u_P(1)\big)\right)
  =
  \sgn\big(\partial_r u_P(1)\big)
  =
  0.
\end{align*}
Mit diesen Einschränkung gilt insbesondere $u\in H^1_0(\Omega)$, weshalb 
die exakte Energie $E(u)$ nach \Cref{rem:bvSeminorm} berechnet werden kann
durch 
\begin{align*}
  E(u)
  =
  \frac{\alpha}{2}\Vert u\Vert^2 + \Vert u\Vert_{W^{1,1}(\Omega)} 
  - \int fu\dx.
\end{align*}
Um also $E(u)$ berechnen zu können, wird der schwache Gradient $\nabla u$ von
$u$ benötigt und um die garantierte untere Energieschranke $\Egleb$ aus
\Cref{eq:gleb} zu berechnen, wird der schwache Gradient $\nabla f$ von $f$
benötigt.
Deshalb betrachten wir an dieser Stelle noch kurz die benötigten
Zusammenhänge zwischen den partiellen Ableitungen in kartesischen Koordianten
und in Polarkoordinaten für eine hinreichend glatte Funktion $g_P$.
Sei
\begin{align*}
  \atan(x_2,x_1)\coloneqq
  \begin{cases}
    \arctan\left( \frac{x_2}{x_1} \right),& \text{wenn }x_1>0,\\
    \arctan\left( \frac{x_2}{x_1} \right) +\pi,& \text{wenn }x_1<0,x_2\geq 0,\\
    \arctan\left( \frac{x_2}{x_1} \right) -\pi,& \text{wenn }x_1<0,x_2<0,\\
    \frac{\pi}{2},& \text{wenn }x_1=0,x_2>0,\\
    -\frac{\pi}{2},& \text{wenn }x_1=0,x_2<0,\\
    \text{undefiniert},& \text{wenn }x_1=x_2=0.
  \end{cases}
\end{align*}
Ein Argument $x=(x_1,x_2)\in\Rbb^2$ von $g_P$ kann dann in Polarkoordinaten 
charakterisiert werden durch die Länge $r=\sqrt{x_1^2+x_2^2}$ und den Winkel
$\varphi = \atan(x_2,x_1)$.
Mit dieser Notation gelten für die partiellen Ableitungen die Zusammenhänge
\begin{align*}
  \partial_{x_1} &= 
  \cos(\varphi)\partial_r - \frac{1}{r}\sin(\varphi)\partial_\varphi
  \quad\text{und }\\
  \partial_{x_2} &= 
  \sin(\varphi)\partial_r - \frac{1}{r}\cos(\varphi)\partial_\varphi.
\end{align*}
Ist nun $g_P$ vom Winkel $\varphi$ unabhängig, so erhalten wir
\begin{align*}
  \nabla g_P 
  = 
  \begin{pmatrix}
    \cos(\varphi)\\
    \sin(\varphi)
  \end{pmatrix}
  \partial_r g_P.
\end{align*}
Unter Beachtung der trigonometrischen Zusammenhänge
\begin{align*}
  \sin\big(\arctan(y)\big) &= \frac{y}{\sqrt{1+y^2}} &&\text{und}
  &\cos\big(\arctan(y)\big) &= \frac{1}{\sqrt{1+y^2}}
\end{align*}
ergibt sich 
\begin{align*}
  \begin{pmatrix}
    \cos(\varphi)\\
    \sin(\varphi)
  \end{pmatrix}
  = 
  \frac{1}{r}
  \begin{pmatrix}
    x_1\\
    x_2
  \end{pmatrix}
\end{align*}
und somit 
\begin{align*}
  \nabla g_P
  = 
  \frac{\partial_r g_P}{r}
  \begin{pmatrix}
    x_1\\
    x_2
  \end{pmatrix}.
\end{align*} 
Zum Bestimmen des Gradienten in kartesischen Koordianten einer
in Polarkoordinaten gegegebenen Funktion $g_P$, die
vom Polarwinkel unabhängig ist, muss also lediglich 
die partielle Ableitung $\partial_r g_P$ berechnet werden.
Konkrete Beispiele formulieren wir in \Cref{chap:experiments}.


\section{Discrete Problem}
\section{Formulierung}
\label{sec:discreteProblemFormulation}
Bevor wir \Cref{prob:continuousProblem} diskretisieren, merken wir an,
dass $\CR^1(\Tcal)\subset\BV(\Omega)$, da
\begin{align*}
  |\vcr|_{\BV(\Omega)} 
  = 
  \Vert \gradnc \vcr\Vert_{L^1(\Omega)} 
  + \sum_{F\in\Ecal(\Omega)}\Vert[\vcr]_F\Vert_{L^1(F)}
  \quad\text{für alle }\vcr\in\CR^1(\Tcal).
\end{align*} 
Dies wird für $|\Tcal|=2$ zum Beispiel von \cites[S. 404, Example
10.2.1]{ABM14}[S. 301, Proposition 10.1]{Bar15} impliziert und kann
analog für beliebige reguläre Triangulierungen von $\Omega$ bewiesen
werden.
Damit gilt für alle $\vcr\in\CR^1(\Tcal)$ insbesondere
\begin{align*}
  |\vcr|_{\BV(\Omega)} +\Vert\vcr\Vert_{L^1(\partial\Omega)} 
  = \Vert \gradnc \vcr\Vert_{L^1(\Omega)} +
  \sum_{F\in\Ecal}\Vert[\vcr]_F\Vert_{L^1(F)}.
\end{align*}
Um eine nichtkonforme Formulierung von \Cref{prob:continuousProblem} zu 
erhalten, ersetzen wir die Terme 
$|\bullet|_{\BV(\Omega)} +\Vert\bullet\Vert_{L^1(\partial\Omega)}$ des
Funktionals $E$ durch 
$\Vert \gradnc \bullet\Vert_{L^1(\Omega)}$, das heißt, wir vernachlässigen
bei der nichtkonformen Formulierung die Terme
$\sum_{F\in\Ecal}\Vert[\bullet]_F\Vert_{L^1(F)}$.
Somit erhalten wir das folgende Minimierungsproblem für den Parameter
$\alpha\in\Rbb_+$ und das Eingangssignal $f\in L^2(\Omega)$.

\begin{problem}\label{prob:discreteProblem}
  Finde $\ucr\in \CR^1_0(\Tcal)$,
  sodass $\ucr$ das Funktional
  \begin{align}\label{eq:discreteProblem}
    \Enc(\vcr)\coloneqq \frac{\alpha}{2}\Vert \vcr\Vert^2
    +\Vert \gradnc\vcr\Vert_{L^1(\Omega)}-\int_\Omega f\vcr\dx
  \end{align}
  unter allen $\vcr\in \CR^1_0(\Tcal)$ minimiert.
\end{problem}

\section{Charakterisierung und Existenz eines eindeutigen Minimierers}

In diesem Abschnitt führen wir die Argumente in \cite[S. 313]{Bar15}, angepasst
für unsere Formulierung in \Cref{prob:discreteProblem}, detailliert aus. 
Zunächst zeigen wir, dass \Cref{prob:discreteProblem} eine eindeutige Lösung
besitzt. Dafür benötigen wir folgendes Lemma.
\begin{lemma}
  \label{lem:normOfGradNcContiuous}
  Das Funktional $\Enc$ aus \Cref{eq:discreteProblem} ist stetig bezüglich der
  Konvergenz in $L^2(\Omega)$.
\end{lemma}

\begin{proof}
  Die Folge $(v_n)_{n\in\Nbb}\subset\CR^1_0(\Tcal)$ konvergiere
  gegen $\vcr\in\CR^1_0(\Tcal)$ bezüglich der Norm $\Vert\bullet\Vert$.
  Damit ist $(v_n)_{n\in\Nbb}$ insbesondere beschränkt in $L^2(\Omega)$ und es
  gilt mit einer binomischen Formel und der umgekehrten Dreiecksungleichung,
  dass
  \begin{align*}
    \left|\Vert\vcr\Vert^2-\Vert v_n\Vert^2\right|
    &=
    \big|\Vert\vcr\Vert-\Vert v_k\Vert\big|\, 
    \big|\Vert \vcr\Vert+\Vert v_k\Vert\big|\\
    &\leq
    \Vert\vcr- v_k\Vert\, \big|\Vert \vcr\Vert+\Vert v_k\Vert\big|
    \to 0\quad\text{für }n\to\infty.
  \end{align*}
  Außerdem gilt mit der Hölderschen Ungleichung
  \begin{align*}
    \left|\int_\Omega f(\vcr-v_k)\dx\right|
    \leq \Vert f\Vert \Vert\vcr-v_k\Vert\to 0\quad\text{für }n\to\infty.
  \end{align*}
  Schließlich gilt für alle $n\in\Nbb$ und alle $T\in\Tcal$ mit der 
  inversen Ungleichung (cf. \cite[S. 53, Lemma 3.5]{Bar15})
  mit Konstante $c_T\in\Rbb_+$ und der Hölderschen Ungleichung, dass
  \begin{equation*}
    \label{eq:continuityProofTriangleWiseEstimate}
    \Vert\gradnc(\vcr- v_n)\Vert_{L^1(T)}
    \leq
    c_T h_T^{-1}\Vert\vcr- v_n\Vert_{L^1(T)}
    \leq
    c_T h_T^{-1}\sqrt{|T|}\Vert\vcr- v_n\Vert_{L^2(T)}.
  \end{equation*}
  Damit folgt zusammen mit der umgekehrten Dreiecksungleichung
  \begin{align*}
    \left|\Vert\gradnc\vcr\Vert_{L^1(\Omega)}-
    \Vert \gradnc v_n\Vert_{L^1(\Omega)}\right|
    &\leq 
    \Vert\gradnc(\vcr- v_n)\Vert_{L^1(\Omega)}\\
    &=
    \sum_{T\in\Tcal}\Vert\gradnc(\vcr- v_n)\Vert_{L^1(T)}\\
    &\leq
    \max_{T\in\Tcal}\left(c_T h_T^{-1}\sqrt{|T|}\right)
    \sum_{T\in\Tcal}\Vert\vcr- v_n\Vert_{L^2(T)}\\
    &=
    \max_{T\in\Tcal}\left(c_T h_T^{-1}\sqrt{|T|}\right) \Vert\vcr- v_n\Vert
    \to 0\quad\text{für }n\to\infty.
  \end{align*}
  Somit ist $\Enc$ Summe von drei Termen, die bezüglich der Norm
  $\Vert\bullet\Vert$ folgenstetig sind, und deshalb stetig bezüglich der
  Konvergenz in $L^2(\Omega)$.
\end{proof}

\begin{theorem}
  \label{thm:discreteProblemExistenceUniqueness}
  Es existiert eine eindeutige Lösung $\ucr\in\CR^1_0(\Tcal)$ von
  \Cref{prob:discreteProblem}.
\end{theorem}

\begin{proof}
  Mit analogen Abschätzungen wie in \eqref{eq:contProbBddFromBelow}
  erhalten wir für das Funktional $\Enc$ aus \Cref{prob:discreteProblem} 
  für alle $\vcr\in\CR^1_0(\Tcal)\subset L^2(\Omega)$ die Ungleichung 
  \begin{equation}
    \label{eq:discreteEnergyCoercivity}
    \Enc(\vcr) 
    \geq 
    \frac{\alpha}{4}\Vert \vcr\Vert^2
    +\Vert \gradnc\vcr\Vert_{L^1(\Omega)}
    -\frac{1}{\alpha}\Vert f\Vert^2
    \geq 
    -\frac{1}{\alpha}\Vert f\Vert^2.
  \end{equation}
  Somit ist $\Enc$ nach unten beschränkt und es existiert eine infimierende
  Folge $(v_n)_{n\in\Nbb} \subset \CR^1_0(\Tcal)$ von $\Enc$. 
  Ungleichung \eqref{eq:discreteEnergyCoercivity} impliziert weiterhin, dass
  diese Folge beschränkt bezüglich der Norm $\Vert\bullet\Vert$ sein muss.
  Der endlichdimensionale Raum $\CR^1_0(\Tcal)$ ist, ausgestattet mit der Norm
  $\Vert\bullet\Vert$, ein Banachraum und damit reflexiv. 
  Demnach existiert eine in $\CR^1_0(\Tcal)$ schwach konvergente Teilfolge von
  $(v_n)_{n\in\Nbb}$.
  Da $\CR^1_0(\Tcal)$ endlichdimensional ist, konvergiert diese sogar stark
  in $L^2(\Omega)$. 
  Weil $\CR^1_0(\Tcal)$ ein Banachraum und damit abgeschlossen bezüglich der
  Konvergenz in $\Vert\bullet\Vert$ ist, gilt für den Grenzwert $\ucr$ dieser
  Teilfolge, dass $\ucr\in\CR^1_0(\Tcal)$.
  Nach \Cref{lem:normOfGradNcContiuous} ist $\Enc$ stetig bezüglich der
  Konvergenz in $L^2(\Omega)$, was impliziert, dass $\ucr$ Minimierer von
  $\Enc$ in $\CR^1_0(\Tcal)$ sein muss.   
  Dieser Minimierer $\ucr$ ist eindeutig, da $\Enc$ strikt konvex ist.
\end{proof}

Als Nächstes wollen wir äquivalente Charakterisierungen der eindeutigen Lösung
von \Cref{prob:discreteProblem} beweisen, die von Professor Carstensen
formuliert wurden.
Dazu leiten wir zunächst ein zu \Cref{prob:discreteProblem} äquivalentes
Minimaxproblem nach \cite[Section 36]{Roc70} her.
Wir betrachten die konvexe Menge 
\begin{align*}
  K
  \coloneqq 
  \left\{\Lambda\in L^\infty\!\left(\Omega;\Rbb^2\right)
  \,\middle|\,|\Lambda(\bullet)| \leq 1 \text{ fast überall in }\Omega\right\}
\end{align*}
und das dazugehörige Indikatorfunktional
$I_K:L^\infty\!\left(\Omega;\Rbb^2\right)\to\Rbb\cup\{\infty\}$, das für
$\Lambda\in L^\infty\!\left(\Omega;\Rbb^2\right)$ gegeben ist durch
\begin{align*}
  I_K(\Lambda)
  &\coloneqq
  \begin{cases}
    \infty, & \text{falls } \Lambda\notin K,\\
    0,       & \text{falls } \Lambda\in K.
  \end{cases}
\end{align*} 
Aufgrund der Konvexität von $K$ ist $I_K$ konvex.
Für $\vcr\in\CR^1_0(\Tcal)$ und $\Lambda_0\in
P_0\!\left(\Tcal;\Rbb^2\right)\subset L^\infty\!\left(\Omega;\Rbb^2\right)$
können wir damit die Sattelfunktion $L:\CR^1_0(\Tcal)\times
P_0\!\left(\Tcal;\Rbb^2\right)\to [-\infty,\infty)$ nach \cite[Section
33]{Roc70} definieren durch
\begin{align}\label{eq:discreteProblemLagrangeFunctional}
  L(\vcr,\Lambda_0) \coloneqq \int_\Omega\Lambda_0\cdot\gradnc\vcr\dx +
  \frac{\alpha}{2}\Vert \vcr\Vert^2 -\int_\Omega f\vcr\dx
  - I_K(\Lambda_0).
\end{align}
Nun wählen wir $\vcr\in\CR^1_0(\Tcal)$ beliebig. 
Mit der Cauchy-Schwarzschen Ungleichung gilt für alle
$\Lambda_0\in P_0\!\left(\Tcal;\Rbb^2\right)\cap K$, dass
\begin{align*}
  \int_\Omega \Lambda_0\cdot\gradnc\vcr\dx
  \leq 
  \int_\Omega |\Lambda_0||\gradnc\vcr|\dx
  \leq 
  \Vert\gradnc\vcr\Vert_{L^1(\Omega)}.
\end{align*}
Daraus folgt
\begin{align}
  \label{eq:saddlepointLeqEnergy}
  \sup_{\Lambda_0\in P_0\left(\Tcal;\Rbb^2\right)\cap K}L(\vcr,\Lambda_0)
  \leq \Enc(\vcr).
\end{align}
Weiterhin gilt, wenn wir $\Lambda_0\in P_0\!\left(\Tcal;\Rbb^2\right)\cap K$
mit der Signumfunktion aus \Cref{eq:signumFunction} elementweise auf allen
$T\in\Tcal$ definieren durch $\Lambda_0(x)\in\sign\left(\gradnc\vcr(x)\right)$
für alle $x\in \interior(T)$, dass $L(\vcr,\Lambda_0)=\Enc(\vcr)$ und deshalb
auch
\begin{align}
  \label{eq:saddlepointGeqEnergy}
  \Enc(\vcr)
  \leq
  \sup_{\Lambda_0\in P_0\left(\Tcal;\Rbb^2\right)\cap K}L(\vcr,\Lambda_0).
\end{align}
Außerdem ist $L(\vcr,\Lambda_0)>-\infty$ genau dann, wenn $\Lambda_0\in K$.
Damit folgt aus den Ungleichungen \eqref{eq:saddlepointLeqEnergy} und
\eqref{eq:saddlepointGeqEnergy} insgesamt
\begin{equation*}
  \label{eq:discreteEnergySaddlefunctionalEquality}
  \Enc(\vcr)
  =\sup_{\Lambda_0\in P_0\left(\Tcal;\Rbb^2\right)\cap K}L(\vcr,\Lambda_0)
  =\sup_{\Lambda_0\in P_0\left(\Tcal;\Rbb^2\right)}L(\vcr,\Lambda_0).
\end{equation*}
Wenn also das folgende Minimaxproblem
\ref{prob:discreteSaddlepointProblem} eine Lösung $\left(
\tilde{u}_\CR,\bar\Lambda_0 \right)\in\CR^1_0(\Tcal)\times
P_0\!\left(\Tcal;\Rbb^2\right)$ hat, dann löst die Funktion $\tilde{u}_\CR$
\Cref{prob:discreteProblem}.

\begin{problem}\label{prob:discreteSaddlepointProblem}
  Finde $\left( \tilde{u}_\CR,\bar\Lambda_0 \right)\in\CR^1_0(\Tcal)\times
  P_0\!\left(\Tcal;\Rbb^2\right)$,
  sodass
  \begin{align*}
    L(\tilde{u}_\CR,\bar\Lambda_0) 
    = 
    \inf_{\vcr\in\CR^1_0(\Tcal)}\sup_{\Lambda_0\in P_0\left(\Tcal;\Rbb^2\right)}
    L(\vcr,\Lambda_0).
  \end{align*}
\end{problem}

\begin{lemma}
  \label{lem:existenceSaddlepoint}
  Es existiert eine Lösung $\left( \tilde{u}_\CR,\bar\Lambda_0
  \right)\in\CR^1_0(\Tcal)\times \left(P_0\!\left(\Tcal;\Rbb^2\right)\cap
  K\right)$ von \Cref{prob:discreteSaddlepointProblem}.
\end{lemma}

\begin{proof}
  Die Sattelfunktion $L$ aus \Cref{eq:discreteProblemLagrangeFunctional} ist,
  wenn ihre zweite Komponente in $P_0\!\left(\Tcal;\Rbb^2\right)\cap K$ fixiert
  ist, in ihrer ersten Komponente eine konvexe, unterhalbstetige, auf
  $\CR^1_0(\Tcal)$ reellwertige Funktion und in ihrer zweiten Komponente eine
  konkave, oberhalbstetige, auf $P_0\!\left(\Tcal;\Rbb^2\right)\cap K$
  reellwertige Funktion.
  Somit ist $L$ in beiden Komponenten abgeschlossen nach \cite[S. 52,
  308]{Roc70}.  
  Insgesamt ist $L$ damit eine konvex-konkave, propere und abgeschlossene
  Funktion nach \cite[S. 349, 362 f.]{Roc70}, deren effektiver
  Definitionsbereich nach \cite[362]{Roc70} die Menge $\CR^1_0(\Tcal)\times
  \left( P_0\!\left(\Tcal;\Rbb^2\right)\cap K\right)$ ist.
  Unter Beachtung der Isomorphie von $\CR^1_0(\Tcal)$ zu
  $\Rbb^{|\Ecal(\Omega)|}$ und der Isomorphie von
  $P_0\!\left(\Tcal;\Rbb^2\right)$ zu $\Rbb^{2|\Tcal|}$, folgt aus 
  \cite[S. 397, Theorem 37.6]{Roc70} die Existenz eines Sattelpunkts
  $\left(\tilde{u}_\CR,\bar\Lambda_0\right)\in \CR^1_0(\Tcal)\times \left(
  P_0\!\left(\Tcal;\Rbb^2\right)\cap K\right)$ von $L$ nach \cite[380]{Roc70}.
  Für diesen impliziert \cite[S. 380, Lemma 36.2]{Roc70}, dass 
  \begin{align*}
    \sup_{\Lambda_0\in P_0\left(\Tcal;\Rbb^2\right)}\inf_{\vcr\in\CR^1_0(\Tcal)}
    L(\vcr,\Lambda_0)
    =
    L(\tilde{u}_\CR,\bar\Lambda_0) 
    = 
    \inf_{\vcr\in\CR^1_0(\Tcal)}\sup_{\Lambda_0\in P_0\left(\Tcal;\Rbb^2\right)}
    L(\vcr,\Lambda_0).
  \end{align*}
  Somit löst $\left(\tilde{u}_\CR,\bar\Lambda_0\right)\in \CR^1_0(\Tcal)\times
  \left( P_0\!\left(\Tcal;\Rbb^2\right)\cap K\right)$
  \Cref{prob:discreteSaddlepointProblem}.
\end{proof}

Nachdem diese Vorbereitungen abgeschlossen sind, können wir nun folgendes
Theorem beweisen.

\begin{theorem}
  \label{thm:discProbCharacterizationOfDiscreteSolutions}
  Für eine Funktion $\tilde{u}_\CR\in\CR^1_0(\Tcal)$ sind die folgenden drei
  Aussagen äquivalent.
  \begin{itemize}
    \item[(i)] \Cref{prob:discreteProblem} wird von $\tilde{u}_\CR$ gelöst.
    \item[(ii)] Es existiert ein
      $\bar\Lambda_0\in P_0\!\left(\Tcal;\Rbb^2\right)$ mit
      $\left|\bar\Lambda_0(\bullet)\right|\leq 1$
      fast überall in $\Omega$, sodass
      \begin{equation}
        \label{eq:discreteMultiplierScalerProductEquality}
        \bar\Lambda_0(\bullet)\cdot\gradnc\tilde{u}_\CR(\bullet)
        =
        \left|\gradnc\tilde{u}_\CR(\bullet)\right| 
        \quad\text{fast überall in } \Omega 
      \end{equation}
      und
      \begin{equation}
        \label{eq:discreteMultiplierL2Equality}
        \left(\bar\Lambda_0,\gradnc\vcr\right)
        = \left(f-\alpha\tilde{u}_\CR,
        \vcr\right)
        \quad\text{für alle } \vcr\in\CR^1_0(\Tcal).
      \end{equation}
    \item[(iii)] Für alle $\vcr\in\CR^1_0(\Tcal)$ gilt
      \begin{equation}
        \label{eq:discreteVariationalInequality}
        \left(f-\alpha\tilde{u}_\CR,\vcr-\tilde{u}_\CR\right)\leq
        \Vert\gradnc\vcr\Vert_{L^1(\Omega)} -
        \left\Vert\gradnc\tilde{u}_\CR\right\Vert_{L^1(\Omega)}\!.
      \end{equation}
  \end{itemize}
\end{theorem}

\begin{proof} 
  Sei $\tilde{u}_\CR\in\CR^1_0(\Tcal)$.

  \textit{(i) $\Rightarrow$ (ii).}
  Sei $\tilde{u}_\CR$ Lösung von \Cref{prob:discreteProblem}.
  Nach \Cref{lem:existenceSaddlepoint} existiert eine Lösung
  $\left(\hat{u}_\CR,\bar\Lambda_0\right)\in \CR^1_0(\Tcal)\times \left(
  P_0\!\left(\Tcal;\Rbb^2\right)\cap K\right)$ 
  von \Cref{prob:discreteSaddlepointProblem}. 
  Außerdem wissen wir, dass damit $\hat{u}_\CR$ Lösung von
  \Cref{prob:discreteProblem} ist.
  Daraus folgt, da nach \Cref{thm:discreteProblemExistenceUniqueness} die
  Lösung von \Cref{prob:discreteProblem} eindeutig ist, dass
  $\hat{u}_\CR=\tilde{u}_\CR$ in $\CR^1_0(\Tcal)$.
  Weiterhin wissen wir aus dem Beweis von \Cref{lem:existenceSaddlepoint}, dass
  $\left(\tilde{u}_\CR,\bar\Lambda_0\right)$ Sattelpunkt der Funktion $L$ aus
  \Cref{eq:discreteProblemLagrangeFunctional} ist.
  Das bedeutet nach \cite[380]{Roc70} insbesondere, dass $\tilde{u}_\CR$
  Minimierer von $L(\bullet, \bar\Lambda_0)$ in $\CR^1_0(\Tcal)$ und
  $\bar\Lambda_0$ Maximierer von $L\!\left(\tilde{u}_\CR,\bullet\right)$ in 
  $P_0\!\left(\Tcal;\Rbb^2\right)$ ist.  
  Mit dieser Erkenntnis können wir nun die entsprechenden
  Optimalitätsbedingungen diskutieren.
  Zunächst gilt, da
  $L\!\left(\tilde{u}_\CR,\bullet\right):P_0\!\left(\Tcal;\Rbb^2\right)\to
  [-\infty,\infty)$ konkav und
  $\bar\Lambda_0$ Maximierer von $L\!\left(\tilde{u}_\CR,\bullet\right)$ in 
  $ P_0\!\left(\Tcal;\Rbb^2\right)$ ist, dass das konvexe Funktional
  $-L(\tilde{u}_\CR,\bullet):P_0\!\left(\Tcal;\Rbb^2\right)\to
  (-\infty,\infty]$ von $\bar\Lambda_0$ in $ P_0\!\left(\Tcal;\Rbb^2\right)$
  minimiert wird.
  %Nach den Theoremen \ref{thm:extremalprinciple},
  %\ref{thm:subdifferentialSumRule} und \ref{thm:subdiffGateaux} gilt somit
  Nach den Theoremen \ref{thm:extremalprinciple} --
  \ref{thm:subdifferentialSumRule} gilt somit
  \begin{align*}
    0
    \in 
    \partial \left(-L\!\left(\tilde{u}_\CR,\bullet\right)\right)
    \left(\bar\Lambda_0\right) 
    =
    \left\{-\!\left(\gradnc\tilde{u}_\CR,\bullet\right)\right\}+\partial I_K
    \left(\bar\Lambda_0\right)\!.
  \end{align*}
  Äquivalent zu dieser Aussage ist, dass
  $\left(\gradnc\tilde{u}_\CR,\bullet\right)\in \partial
  I_K \left(\bar\Lambda_0\right)$. 
  Da $\bar\Lambda_0\in K$, folgt mit \Cref{def:subdifferential},
  dass für alle $\Lambda_0\in  P_0\!\left(\Tcal;\Rbb^2\right)$ gilt
  \begin{align*}
    \left(\gradnc\tilde{u}_\CR,\Lambda_0-\bar\Lambda_0\right) 
    \leq 
    I_K (\Lambda_0) - I_K\!\left(\bar\Lambda_0\right)
    =
    I_K (\Lambda_0).
  \end{align*}
  Falls $\Lambda_0\in  P_0\!\left(\Tcal;\Rbb^2\right)\cap K$, folgt insbesondere
  \begin{align}
    \label{eq:scalarProductInequDiscreteProof}
    \left(\gradnc\tilde{u}_\CR,\Lambda_0-\bar\Lambda_0\right) 
    &\leq 
    0,\quad\text{also }\notag\\ 
    \left(\gradnc\tilde{u}_\CR,\Lambda_0\right)
    &\leq
    \left(\gradnc\tilde{u}_\CR,\bar\Lambda_0\right)\!.
  \end{align}
  Sei nun $\Lambda_0\in P_0\!\left(\Tcal;\Rbb^2\right)\cap K$ elementweise auf
  allen $T\in\Tcal$ durch $\Lambda_0(x)\in\sign\left(\gradnc\tilde{u}_\CR(x)\right)$
  definiert für alle $x\in\interior(T)$.
  Mit dieser Wahl von $\Lambda_0$, Ungleichung
  \eqref{eq:scalarProductInequDiscreteProof}, der Cauchy\--Schwarz\-schen
  Ungleichung und $\bar\Lambda_0\in K$ erhalten wir die Abschätzung
  \begin{align}
    \label{eq:sumOverAllTrianglesDualVariable}
    \int_\Omega\left|\gradnc\tilde{u}_\CR\right|\dx
    &=
    \int_\Omega\gradnc\tilde{u}_\CR\cdot\Lambda_0\dx
    \leq 
    \int_\Omega\gradnc\tilde{u}_\CR\cdot\bar\Lambda_0\dx \notag\\
    &\leq 
    \int_\Omega\left|\gradnc\tilde{u}_\CR\right|\left|\bar\Lambda_0\right|\dx
    \leq
    \int_\Omega\left|\gradnc\tilde{u}_\CR\right|\dx,
    \quad\text{das heißt, }\notag\\
    \int_\Omega\left|\gradnc\tilde{u}_\CR\right|\dx 
    &= 
    \int_\Omega\gradnc\tilde{u}_\CR\cdot\bar\Lambda_0\dx
    \quad\text{beziehungsweise }\notag\\
    \sum_{T\in\Tcal}|T|\,\big|(\gradnc\tilde{u}_\CR)\!|_T\big|
    &=
    \sum_{T\in\Tcal}|T|\left(\gradnc\tilde{u}_\CR\cdot \bar\Lambda_0\right)\!\!|_T.
  \end{align}
  Außerdem gilt für alle $T\in\Tcal$ mit der Cauchy-Schwarzschen Ungleichung
  und $\bar\Lambda_0\in K$, dass 
  \begin{align*}
    \left(\gradnc\tilde{u}_\CR\cdot \bar\Lambda_0\right)\!\!|_T
  \leq
  \big|(\gradnc\tilde{u}_\CR)\!|_{T}\big|\,\left|\bar\Lambda_0|_T\right|
  \leq
  \big|(\gradnc\tilde{u}_\CR)\!|_{T}\big|.
  \end{align*}
  Mit \Cref{eq:sumOverAllTrianglesDualVariable} folgt daraus für alle
  $T\in\Tcal$, dass $\left(\gradnc\tilde{u}_\CR\cdot
  \bar\Lambda_0\right)\!\!|_T=\big|(\gradnc\tilde{u}_\CR)\!|_T\big|$, das
  heißt, fast überall in $\Omega$ gilt
  $\bar\Lambda_0(\bullet)\cdot\gradnc\tilde{u}_\CR(\bullet)
  =|\gradnc\tilde{u}_\CR(\bullet)|$. 
  Damit ist \Cref{eq:discreteMultiplierScalerProductEquality} gezeigt.
  Als Nächstes betrachten wir das reellwertige Funktional
  $L\left(\bullet,\bar\Lambda_0\right):\CR^1_0(\Tcal)\to\Rbb$.
  Es ist Fr\'echet-differenzierbar mit
  \begin{align*}
    dL\!\left(\bullet,\bar\Lambda_0\right)\!\left(\tilde{u}_\CR;\vcr\right)
    =
    \int_\Omega\bar\Lambda_0\cdot \gradnc\vcr\dx
    +\alpha\! \left(\tilde{u}_\CR,\vcr\right) - \int_\Omega f\vcr\dx
  \end{align*}
  für alle $\vcr\in\CR^1_0(\Tcal)$.
  Da $\tilde{u}_\CR$ Minimierer von  $L\!\left(\bullet, \bar\Lambda_0\right)$
  in $\CR^1_0(\Tcal)$ ist, gilt nach
  \Cref{thm:necessaryConditionFreeLocalExtrema}, dass $0 =
  dL\!\left(\bullet,\bar\Lambda_0\right)\!\left(\tilde{u}_\CR;\vcr\right)$ für
  alle $\vcr\in\CR^1_0(\Tcal)$.
  Diese Bedingung ist für alle $\vcr\in\CR^1_0(\Tcal)$ äquivalent zu
  $\left(\bar\Lambda_0,\gradnc\vcr\right) = (f-\alpha \tilde{u}_\CR,\vcr)$.
  Somit ist auch \Cref{eq:discreteMultiplierL2Equality} gezeigt.

  \textit{(ii) $\Rightarrow$ (iii).}
  Die Funktion $\bar\Lambda_0\in P_0\!\left(\Tcal;\Rbb^2\right)$ erfülle
  $\left|\bar\Lambda_0(\bullet)\right|\leq 1$ fast überall in $\Omega$ sowie
  die Gleichungen \eqref{eq:discreteMultiplierScalerProductEquality} und 
  \eqref{eq:discreteMultiplierL2Equality}. 
  Sei $\vcr\in\CR^1_0(\Tcal)$.
  Mit den Gleichungen 
  \eqref{eq:discreteMultiplierL2Equality} und 
  \eqref{eq:discreteMultiplierScalerProductEquality} gilt
  \begin{equation}
    \label{eq:equivalentCharacterizationApplicationTwoEquations}
    \begin{aligned}
      \left(f-\alpha\tilde{u}_\CR,\vcr-\tilde{u}_\CR\right) 
      &=
      \left(\bar\Lambda_0,\gradnc\vcr\right)
      - \left(\bar\Lambda_0,\gradnc\tilde{u}_\CR\right)\\
      &=
      \int_\Omega\bar\Lambda_0\cdot\gradnc\vcr\dx
      - \int_\Omega\left|\gradnc\tilde{u}_\CR\right|\dx.
    \end{aligned}
  \end{equation}
  Weiterhin gilt mit der Cauchy-Schwarzschen
  Ungleichung und $\left|\bar\Lambda_0(\bullet)\right|\leq 1$ fast überall in
  $\Omega$, dass
  \begin{align*}
    \int_\Omega\bar\Lambda_0\cdot\gradnc\vcr\dx
    &\leq 
    \int_\Omega\left|\bar\Lambda_0\right|\,|\gradnc\vcr|\dx
    \leq 
    \int_\Omega|\gradnc\vcr|\dx.
  \end{align*}
  Zusammen mit \Cref{eq:equivalentCharacterizationApplicationTwoEquations}
  folgt daraus Ungleichung \eqref{eq:discreteVariationalInequality}.

  \textit{(iii) $\Rightarrow$ (i)}.
  Es gelte Ungleichung \eqref{eq:discreteVariationalInequality} für alle
  $\vcr\in\CR^1_0(\Tcal)$, also
  \begin{align*}
    \left(f-\alpha\tilde{u}_\CR,\vcr-\tilde{u}_\CR\right) 
    \leq
    \left\Vert\gradnc\vcr\right\Vert_{L^1(\Omega)}
    -\left\Vert\gradnc\tilde{u}_\CR\right\Vert_{L^1(\Omega)}\!.
  \end{align*}
  Nach \Cref{thm:discreteProblemExistenceUniqueness} existiert eine eindeutige
  Lösung $\ucr\in\CR^1_0(\Tcal)$ von \Cref{prob:discreteProblem}.
  Wir haben bereits gezeigt, dass somit für alle
  $\vcr\in\CR^1_0(\Tcal)$ gilt
  \begin{align*}
    \left(f-\alpha\ucr,\vcr-\ucr\right) 
    \leq
    \left\Vert\gradnc\vcr\right\Vert_{L^1(\Omega)}
    -\Vert\gradnc\ucr\Vert_{L^1(\Omega)}.
  \end{align*}
  Um nun zu beweisen, dass $\tilde{u}_\CR$ \Cref{prob:discreteProblem} löst, genügt
  es $\tilde{u}_\CR=\ucr$ in $\CR^1_0(\Tcal)$ zu zeigen.
  Es gilt
  \begin{align*}
    \left(f-\alpha\ucr,\tilde{u}_\CR-\ucr\right) 
    &\leq
    \left\Vert\gradnc\tilde{u}_\CR\right\Vert_{L^1(\Omega)}
    -\Vert\gradnc\ucr\Vert_{L^1(\Omega)}\quad\text{und }\\
    \left(f-\alpha\tilde{u}_\CR,\ucr-\tilde{u}_\CR\right) 
    &\leq
    \left\Vert\gradnc\ucr\right\Vert_{L^1(\Omega)}
    -\Vert\gradnc\tilde{u}_\CR\Vert_{L^1(\Omega)}. 
  \end{align*}
  Die Addition dieser Ungleichungen
  impliziert
  \begin{align*}
    \alpha\left\Vert\tilde{u}_\CR-\ucr\right\Vert^2=
    \left(-\alpha\ucr,\tilde{u}_\CR-\ucr\right) 
    + \left(-\alpha\tilde{u}_\CR,\ucr-\tilde{u}_\CR\right) 
    \leq
    0.
  \end{align*}
  Da $\alpha>0$, folgt daraus $\left\Vert\tilde{u}_\CR-\ucr\right\Vert^2=0$,
  also $\tilde{u}_\CR=\ucr$ in $\CR^1_0(\Tcal)$.
\end{proof}

Zum Abschluss dieses Abschnitts wollen wir noch zwei Bemerkungen von Professor
Carstensen erwähnen und kurz deren Gültigkeit begründen.
Die erste Bemerkung ist eine äquivalente Charakterisierung der dualen Variable
$\bar\Lambda_0\in P_0\!\left(\Tcal;\Rbb^2\right)$ aus
\Cref{thm:discProbCharacterizationOfDiscreteSolutions} zur diskreten Lösung
$\ucr\in\CR^1_0(\Tcal)$ von \Cref{prob:discreteProblem}.

\begin{remark}
  Dass $\bar\Lambda_0\in P_0\!\left(\Tcal;\Rbb^2\right)$ fast überall in $\Omega$
  \Cref{eq:discreteMultiplierScalerProductEquality} und
  $|\bar\Lambda_0(\bullet)|\leq 1$ erfüllt, ist äquivalent zu der Bedingung
  $\bar\Lambda_0(x)\in\sign(\gradnc \ucr(x))$ für alle $x\in\interior(T)$ für
  alle $T\in\Tcal$.   
\end{remark}

\begin{proof}
  Dass die genannte Bedingung hinreichend ist, folgt direkt aus der Definition
  der Signumfunktion.
  Ihre Notwendigkeit folgt aus der folgenden Beobachtung.
  Da $\left|\bar\Lambda_0(\bullet)\right|\leq 1$ fast überall in $\Omega$, ist
  \Cref{eq:discreteMultiplierScalerProductEquality} eine Cauchy-Schwarzsche
  Ungleichung, bei der sogar Gleichheit gilt. 
  Dies ist genau dann der Fall, wenn $\bar\Lambda_0(\bullet)$ und
  $\gradnc\ucr(\bullet)$ fast überall in $\Omega$ linear abhängig sind.
\end{proof}
 
Daraus können wir folgern, unter welchen Umständen die duale Variable
$\bar\Lambda_0$ auf einem Dreieck $T\in\Tcal$ eindeutig bestimmt ist.

\begin{remark}
  Falls $\gradnc\ucr\neq 0$ auf $T\in\Tcal$, gilt nach Definition der
  Signumfunktion, dass $\bar\Lambda_0=\gradnc\ucr/|\gradnc\ucr|$ eindeutig
  bestimmt ist auf $T$.
  Im Allgemeinen ist $\bar\Lambda_0$ allerdings nicht eindeutig bestimmbar. 
  Betrachten wir zum Beispiel $f\equiv 0$ in \Cref{prob:discreteProblem} mit
  eindeutiger Lösung $\ucr\equiv 0$ fast überall in $\Omega$. 
  Dann erfüllt nach der diskreten Helmholtz-Zerlegung \cite[S. 193, Theorem
  3.32]{Car09b} die Wahl $\bar\Lambda_0\coloneqq \Curl v_\C$ für ein beliebiges
  $v_\C\in S^1(\Tcal)$ mit $|\Curl v_\C|\leq 1$ die Charakterisierung
  \textit{(ii)} aus \Cref{thm:discProbCharacterizationOfDiscreteSolutions}.
\end{remark}


\section{Verfeinerungsindikator und garantierte Energieschranken}

Professor Carstensen stellte für die numerischen Untersuchungen eine Aussage
über eine garantierte untere Energieschranke und einen Verfeinerungsindikator
zur adaptiven Netzverfeinerung zur Verfügung, die wir in diesem Abschnitt
aufführen wollen.

\begin{theorem}[Garantierte untere Energieschranke]
  \label{thm:gleb}
  Sei $\Omega$ konvex, $f\in H^1_0(\Omega)$ das Eingangssignal für
  \Cref{prob:continuousProblem} mit Lösung $u\in H^1_0(\Omega)$ sowie für
  \Cref{prob:discreteProblem} mit Lösung $\ucr\in \CR^1_0(\Tcal)$.
  Dann gilt
  \begin{align*}
    \Enc(\ucr)+\frac{\alpha}{2}\Vert u-\ucr\Vert^2
    -\frac{\kappa_\CR}{\alpha}\Vert
    h_\Tcal(f-\alpha\ucr)\Vert \Vert\nabla f\Vert\leq E(u).
  \end{align*}
  Dabei ist die Konstante $\kappa_\CR\coloneqq\sqrt{1/48+1/j_{1,1}^2}$ mit der
  kleinsten positiven Nullstelle $j_{1,1}$ der Bessel-Funktion erster Art.
  Insbesondere gilt dann für 
  \begin{align}
    \label{eq:gleb}
    \Egleb 
    \coloneqq 
    \Enc(\ucr) - \frac{\kappa_\CR}{\alpha}\Vert h_\Tcal(f-\alpha\ucr)\Vert
    \Vert \nabla f\Vert,
  \end{align}
    dass $\Enc(\ucr)\geq \Egleb$ und $E(u)\geq \Egleb$.
\end{theorem}

\begin{definition}[Verfeinerungsindikator]
  \label{def:refinementIndicator}
  Für $d\in\mathbb{N}$ (in dieser Arbeit stets $d=2$) und $0<\gamma\leq 1$
  definieren wir für alle $T\in\Tcal$ und $\ucr\in\CR^1_0(\Tcal)$ die
  Funktionen
  \begin{align*}
    \eta_{\textup{V}, \Tcal}(T)
    &\coloneqq
    |T|^{2/d}\Vert f-\alpha \ucr\Vert^2_{L^2(T)}\quad\text{und }\\
    \eta_{\textup{J}, \Tcal}(T)
    &\coloneqq
    |T|^{\gamma/d}\sum_{F\in\Ecal(T)}\left\Vert [\ucr]_F\right\Vert_{L^1(F)}\!.
  \end{align*} 
  Damit definieren wir den Verfeinerungsindikator
  $\eta_\Tcal\coloneqq\sum_{T\in\Tcal}\eta_\Tcal(T)$, wobei
  \begin{align} \label{eq:refinementIndicator} 
    \eta_\Tcal (T)
    \coloneqq
    \eta_{\textup{V}, \Tcal}(T) + 
    \eta_{\textup{J}, \Tcal}(T)\quad\text{für alle } T\in\Tcal.
  \end{align} 
\end{definition}

Darüber hinaus können wir eine garantierte obere Energieschranke formulieren.
Dabei nutzen wir den Operator $J_{1,\Tcal}:\CR^1(\Tcal)\to P_1(\Tcal)\cap
C_0(\Omega)$ (cf.\ \cite[Section 4]{CH18}), wobei $J_{1, \Tcal}\vcr$ für eine
Funktion $\vcr\in\CR^1(\Tcal)$ in allen Innenknoten $z\in\Ncal(\Omega)$
definiert ist durch
\begin{align}
  \label{eq:enrichmentOperator}
  J_{1,  \Tcal}\vcr(z)\coloneqq |\Tcal(z)|^{-1}\sum_{T\in\Tcal(z)}\vcr|_T(z).
\end{align}
%Dieser erfüllt nach \cite[Section 4]{CH18} für eine vom Innenwinkel (?)
%abhängige Konstante $c>0$, dass
%\begin{align*}
%  \Vert h_\Tcal^{-1}(1-J_1)\vcr\Vert\leq c\Vert\gradnc \vcr\Vert.
%\end{align*}
Da für die Lösung $u$ von \Cref{prob:continuousProblem} und die Lösung
$\ucr$ von \Cref{prob:discreteProblem} gilt 
\begin{align}
  \label{eq:gueb}
  E(u)\leq E(J_{1, \Tcal}\ucr)=\Enc\left(J_{1, \Tcal}\ucr\right)
  \eqqcolon\Egueb,
\end{align}
wählen wir $\Egueb$ als garantierte obere Energieschranke.

\end{document}

\usepackage{tikz}
%\usepackage{pgfplots}
\usetikzlibrary{arrows.meta}

% --- AMSTHM ENVIRONMENtS ------------------------------------------------------

%\theoremstyle{plain}
%\newtheorem{theorem}{Theorem}[chapter]
%\newtheorem{lemma}[theorem]{Lemma}
%\newtheorem{corollary}[theorem]{Corollary}
%
%\theoremstyle{definition}
%\newtheorem{definition}[theorem]{Definition}
%\newtheorem{problem}[theorem]{Problem}
%
%\theoremstyle{remark}
%\newtheorem{remark}[theorem]
%            {\ifthenelse{\equal{\lang}{ngerman}}{Bemerkung}{Remark}}

%%% ALGORITHM ENVIRONMENTS %%%%%%%%%%%%%%%%%%%%%%%%%%%%%%%%%%%%%%%%%%%%%%%%%%%%
\usepackage{algpseudocode}

\renewcommand\algorithmicdo{}
\algrenewcommand\algorithmicrequire{\textbf{Input:}}
\algrenewcommand\algorithmicensure{\textbf{Output:}}
%\newcommand\Od{\textbf{od}}
%\newcommand\Fi{\textbf{fi}}
%\algnotext{EndIf}
%\let\oldEndIf\EndIf
%\renewcommand\EndIf{\Fi\oldEndIf}
%\algnotext{EndWhile}
%\let\oldEndWhile\EndWhile
%\renewcommand\EndWhile{\Od\oldEndWhile}
\algnotext{EndFor}
%\let\oldEndFor\EndFor
%% \renewcommand\EndFor{\Od\oldEndFor}
%% block for an inline for loop
%\algblockdefx[IfInline]{IfInline}{EndIfInline}
%[2]{\textbf{if} #1 \textbf{then} #2}
%[0]{\textbf{fi}}
%\algcblockdefx{IfInline}{ElseInline}{EndIfInline}
%[1]{\textbf{else} #1}
%[0]{\textbf{fi}}
%\algnotext{EndIfInline}
%\let\oldEndIfInline\EndIfInline
%\renewcommand\EndIfInline{\Fi\oldEndIfInline}
%% single else inline
%\algcloopx[SingleElseInline]{If}{SingleElseInline}
%[1]{\textbf{else} #1}

%% algorithm
%\newtheoremstyle{algorithm-style}%
%  {\topsep}   % ABOVESPACE
%  {\topsep}   % BELOWSPACE
%  {\normalfont}% BODYFONT
%  {0pt}       % INDENT (empty value is the same as 0pt)
%  {\bfseries} % HEADFONT
%  {.}         % HEADPUNCT
%  {5pt plus 1pt minus 1pt} % HEADSPACE
%  {\thmname{#1}\thmnumber{ #2}\thmnote{ (#3)}} % CUSTOM-HEAD-SPEC
%\theoremstyle{algorithm-style}
\theoremstyle{plain}
\newtheorem{algorithm}[theorem]{Algorithm}
%\newtheorem{algorithm}[theorem]
%            {\ifthenelse{\equal{\lang}{ngerman}}{Algorithmus}{Algorithm}}

% ============================================================
% === USER-DEFINED COMMANDS ==================================
% ============================================================

\usepackage{amssymb,mathabx,textcomp,bm}

\newcommand{\mathbox}[2]{\makebox[#1]{$\displaystyle #2$}}
\newcommand{\Stackrel}[2]{\mathbox{1.25em}{\stackrel{#1}{#2}}}


% === INTEGRAL MEAN ==========================================

\def\Xint#1{\mathchoice
{\XXint\displaystyle\textstyle{#1}}%
{\XXint\textstyle\scriptstyle{#1}}%
{\XXint\scriptstyle\scriptscriptstyle{#1}}%
{\XXint\scriptscriptstyle\scriptscriptstyle{#1}}%
\!\int}
\def\XXint#1#2#3{{\setbox0=\hbox{$#1{#2#3}{\int}$}
\vcenter{\hbox{$#2#3$}}\kern-.5\wd0}}
\newcommand{\intmean}{\Xint-}

% === PERFECT BULLET =========================================

\let\oldbullet\bullet
\newlength{\raisebulletlen}
\setbox1=\hbox{$\bullet$}\setbox2=\hbox{\tiny$\bullet$}
\setlength{\raisebulletlen}{\dimexpr0.5\ht1-0.5\ht2}
\renewcommand\bullet{\raisebox{\raisebulletlen}{\,\tiny$\oldbullet$}\,}

% === ORTHOGONAL SUM =========================================

\newcommand\orthsum{
\tikz[baseline=(A.base),font=\small]
     \node[draw,ellipse,inner sep=0.15ex] (A){$\perp$};
}

% === CALIGRAPHIC LETTERS ====================================

\newcommand\Acal{\mathcal{A}}
\newcommand\Bcal{\mathcal{B}}
\newcommand\Ccal{\mathcal{C}}
\newcommand\Dcal{\mathcal{D}}
\newcommand\Ecal{\mathcal{E}}
\newcommand\Fcal{\mathcal{F}}
\newcommand\Jcal{\mathcal{J}}
\newcommand\Kcal{\mathcal{K}}
\newcommand\Lcal{\mathcal{L}}
\newcommand\Mcal{\mathcal{M}}
\newcommand\Ncal{\mathcal{N}}
\newcommand\Ocal{\mathcal{O}}
\newcommand\Pcal{\mathcal{P}}
\newcommand\Rcal{\mathcal{R}}
\newcommand\Tcal{\mathcal{T}}

% === MATHBB LETTERS =========================================

\newcommand\C{\mathbb{C}}
\newcommand\N{\mathbb{N}}
\newcommand\R{\mathbb{R}}
\newcommand\T{\mathbb{T}}
\newcommand\K{\mathbb{K}}

% === MATH OPERATORS =========================================

\DeclareMathOperator*{\argmin}{argmin} % the * adjusts MathOperator for indizes beneath
\DeclareMathOperator{\bisec}{bisec}
\DeclareMathOperator{\cond}{cond}
\DeclareMathOperator{\Conv}{conv}
\DeclareMathOperator{\Curl}{Curl}
\DeclareMathOperator{\curl}{curl}
\DeclareMathOperator{\dev}{dev}
\DeclareMathOperator{\diag}{diag}
\DeclareMathOperator{\diam}{diam}
\DeclareMathOperator{\Dim}{dim}
\DeclareMathOperator{\Div}{div}
\DeclareMathOperator{\Dist}{dist}
\DeclareMathOperator{\esssup}{ess\ supp}
\DeclareMathOperator{\grad}{\nabla}
\DeclareMathOperator{\Int}{int}
\DeclareMathOperator{\Ker}{ker}
\DeclareMathOperator{\Mid}{mid}
\DeclareMathOperator{\Osc}{osc}
\DeclareMathOperator{\Red}{red}
\DeclareMathOperator{\Ref}{Ref}
\DeclareMathOperator{\Res}{Res}
\DeclareMathOperator{\sign}{sgn}
\DeclareMathOperator{\Span}{span}
\DeclareMathOperator{\supp}{supp}
\DeclareMathOperator{\tr}{tr}
\DeclareMathOperator{\Width}{width}
\DeclareMathOperator*{\arginf}{arginf}

% === COMMANDS WITH INPUT ARGUMENTS ==========================

\newcommand\abs[1]{\lvert #1 \rvert}
\newcommand\average[1]{\langle #1 \rangle}
\newcommand\jump[1]{\lbrack #1 \rbrack}
\newcommand\NormEnergy[1]{\big\vvvert #1 \big\vvvert}
\newcommand\Norm[2]{\lVert #1 \rVert_{#2}}
\newcommand\scal[2]{\left\langle #1 , #2 \right\rangle}

% === SPACES WITH NAMES ======================================

\newcommand{\CONF}{\textup{C}}
\newcommand{\NC}{\textup{NC}}
\newcommand{\CR}{\textup{CR}}
\newcommand{\RT}{\textup{RT}}

% === GENERAL STUFF ==========================================

\newcommand{\splitter}{\,:\,} % split for definition of sets
\renewcommand{\d}{\, \textup{d}} % d for differentials, i.e., \d x


% === JUST TO COMPILE CHAPTERS 4  AND 5 ==========================================

\newcommand{\LO}[0]{L^2(\Omega)}
\newcommand{\restrict}[2]{\left. #1 \right\lvert_{#2}}
\DeclareMathOperator{\mPkt}{mid}
\newcommand{\nnorm}[2]{\left\lVert #1 \right\rVert_{#2}}
\newcommand{\volf}[2]{\nnorm{h_{#1}f}{#2}}
\newcommand{\volO}[1]{\nnorm{h_{#1}f}{\LO}}
\DeclareMathOperator\Kern{ker}
\DeclareMathOperator\setspan{span}

% === JUST TO COMPILE CHAPTER 6 ==========================================


\newcommand\NormL[3]{\Norm{#1}{L^{#2}\left( { #3 } \right)}}
 \newcommand\NormH[3]{\Norm{#1}{H^{#2}\left( { #3 } \right)}}
 \newcommand\NormHdiv[1]{\Norm{#1}{H(\div)}}
 \newcommand\NormW[3]{\Norm{#1}{W^{#2}\left( { #3 } \right)}}
\newcommand\NormLDom[2]{\NormL{#1}{#2}{\Omega}}
 \newcommand\NormHDom[2]{\NormH{#1}{#2}{\Omega}}
 \newcommand\NormHdivDom[1]{\NormHdiv{#1}}
 \newcommand\NormWDom[2]{\NormW{#1}{#2}{\Omega}}
 \newcommand\NormLz[2]{\NormL{#1}{2}{#2}}
 \newcommand\NormHz[2]{\NormH{#1}{2}{#2}}
 \newcommand\NormWkp[4]{\NormW{#1}{{#2,#3}}{#4}}
 \newcommand\NormWkpDom[3]{\NormWDom{#1}{{#2,#3}}}
 \newcommand\NormLzDom[1]{\NormLDom{#1}{2}}
 \newcommand\NormHzDom[1]{\NormHDom{#1}{2}}
\newcommand\NormMax[2]{\Norm{#1}{\C({#2})}}


% === JUST TO COMPILE CHAPTER 0 ==========================================

\newcommand{\dist}{\mathrm{dist}}
\newcommand{\fa}{\;\forall\;}
\newcommand{\Lra}{\Leftrightarrow}
\newcommand{\id}{\mathrm{id}}

% === JUST TO COMPILE EXERCISES ==========================================

\newcommand{\al}{\alpha}
\newcommand{\be}{\beta}
\newcommand{\ga}{\gamma}
\newcommand{\de}{\delta}
\newcommand{\eps}{\varepsilon}
\newcommand{\vphi}{\varphi}
\newcommand{\la}{\lambda}
\newcommand{\om}{\omega}
 

% --- DEBUG -------------------------------------------------------------------
\overfullrule=2cm % marks overfull hboxes by black bar in document
\usepackage{todonotes} % for the todo's
%\usepackage[disable]{todonotes} % for the todo's
\usepackage{xcolor}


% --- META INFORMATION --------------------------------------------------------


\titlehead{%
  \begin{minipage}{.7\textwidth}%
  Humboldt-Universit\"at zu Berlin\\
  Mathematisch-Naturwissenschaftliche Fakult\"at\\
  Institut f\"ur Mathematik
  \end{minipage}
  \begin{minipage}{.29\textwidth}%
    \begin{flushright}
      \includegraphics*[scale=.6]{pictures/logos/husiegel_bw.pdf}%
    \end{flushright}
  \end{minipage}
}
\title{Die Crouzeix\--Raviart\--Finite\--Elemente\--Methode für
eine nichtkonforme Formulierung des Rudin\--Osher\--Fatemi\--Modellproblems}
\author{Enrico Bergmann}
\date{Version:~\today}

% --- DOCUMENT -----------------------------------------------------------------

\begin{document}
\maketitle
\tableofcontents

\chapter*{Zusammenfassung}
todo, vielleicht auch eher for dem Inhaltsverzeichnis


% TODO to toc
% \todototoc
%\listoftodos
% TODO

\todo[inline]{Titelseite nach Vorgaben in Formalien-Lesezeichen-Ordner
anpassen, Muster befindet sich auf Merkblatt, dies exakt umsetzen}
\bigskip
In Zitaten 'Remark', 'Section' etc. übersetzen oder  im Original lassen? 
Entscheiden und einmal konsequent überprüfen und durchziehen.

\chapter{Einleitung}
\label{chap:introduction}
\begin{frame}[noframenumbering]{Table of Contents}
  \tableofcontents[currentsection, currentsubsection]
\end{frame}

\begin{frame}
  \fullcite[Chapter 10, p. 297-319]{Bar15}

  \bigskip
  \pause

  Let $\Omega\subset\Rbb^n$ be a bounded polyhedral Lipschitz domain.

  \medskip

  For given $g\in L^2(\Omega)$ and $\alpha>0$ minimize the functional 
  \begin{align*}
    I(v)=|v|_{\BV(\Omega)}+\frac{\alpha}{2}\Vert v-g\Vert^2
  \end{align*}
  amongst all $v\in \BV(\Omega)\cap L^2(\Omega)$.
\end{frame}

\begin{frame}{Functions of Bounded Variation}
  A function $v\in L^1(\Omega)$ with distributional derivative
  $Dv:C^{\infty}_C(\Omega;\Rbb^n)\to\Rbb$ is said to be of bounded variation if 
  there exists $c>0$ such that 
  \begin{align*}
    \langle Dv, \phi\rangle\coloneqq -\int_\Omega v\Div(\phi)\dx\leq
    c\Vert\phi\Vert_{L^\infty(\Omega)}
  \end{align*}
  for all $\phi\in C^1_C(\Omega;\Rbb^n)$.

  \pause  

  The minimal constant $c\geq 0$ satisfying this property is called 
  total variation of $Dv$ and is given by
  \begin{align*}
    |v|_{\BV(\Omega)} = \sup_{\substack{\phi\in C^1_C(\Omega;\Rbb^n)\\
    \Vert\phi\Vert_{L^\infty(\Omega)}\leq 1}}-\int_\Omega v\Div (\phi)\dx.
  \end{align*}

  \pause

  The space of all such functions is denoted by $\BV(\Omega)$.
\end{frame}

\begin{frame}{Properties of $\BV(\Omega)$}
  $\BV(\Omega)$ is a Banach space equipped with the norm
  \begin{align*}
    \Vert v \Vert_{\BV(\Omega)} \coloneqq \Vert v\Vert_{L^1(\Omega)} +
    |v|_{\BV(\Omega)}\quad\text{for all } v\in\BV(\Omega).
  \end{align*}
  
  \pause
  \medskip
  $W^{1,1}(\Omega)\subset\BV(\Omega)$ with $\Vert v\Vert_{\BV(\Omega)}=
  \Vert v\Vert_{W^{1,1}(\Omega)}$ for all $v\in W^{1,1}(\Omega)$.
\end{frame}

\begin{frame}{Notions of convergence on $\BV(\Omega)$}
  Let $(v_n)_{n\in\Nbb}\subset \BV(\Omega)$ and $v\in \BV(\Omega)$ such that
  $v_n\rightarrow v$ in $L^1(\Omega)$ as $n\rightarrow\infty$.
  \pause
  \begin{itemize}
    \item[(i)]
      $(v_n)_{n\in\Nbb}$ converges intermediately or strictly to $v$
      if $|v_n|_{\BV(\Omega)}\rightarrow |v|_{\BV(\Omega)}$ as
      $n\rightarrow\infty$.
      \pause
    \item[(ii)] $(v_n)_{n\in\Nbb}$ converges weakly to
      $v$ if
      $\langle Dv_n,\phi\rangle\rightarrow \langle Dv,\phi\rangle$ 
      for all $\phi\in C_0(\Omega;\Rbb^n)$ as 
      $n\rightarrow\infty$.
  \end{itemize}
\end{frame}

\begin{frame}{Further Properties of $\BV(\Omega)$}
  $C^\infty(\overline\Omega)$ and $C^\infty(\Omega)\cap\BV(\Omega)$ are dense
  in $\BV(\Omega)$ with respect to intermediate convergence.
  
  \pause
  \bigskip

  The embedding $\BV(\Omega)\to L^p(\Omega)$ is continuous for
  $1\leq p\leq n/(n-1)$ and compact for $1\leq p< n/(n-1)$.
  
  \pause
  \bigskip

  There exists a linear operator $T:\BV(\Omega)\to L^1(\partial\Omega)$
  such that $T(v) = v|_{\partial\Omega}$ for all $v\in\BV(\Omega)\cap
  C(\overline\Omega)$.

  $T$ is continuous with respect to intermediate convergence in $\BV(\Omega)$
  but not with respect to weak convergence in $\BV(\Omega)$. 
\end{frame}



\chapter{Theoretische Grundlagen}
\label{chap:theoreticalBasics}
%Dies(easymotion-prefix)er Abschnitt folgt dem einführenden Kapitel von [braidesApprox], 
%mit der Einschränkung $\Omega \subset \Rbb^n$ offen und beschränkt, und für
%genauere Informationen sei an dieser Stelle darauf verwiesen.
%
%Die Familie $\Bcal(\Omega)$ aller Borelmengen und die Familie
%$\Bcal_C(\Omega)$ aller Borelmengen mit kompaktem Abschluss in
%$\Omega$ stimmen in diesem Fall überein. 

\todo[inline]{TODO quote \cite[chapter 10, maybe also 4]{ABM14} for BV things,
e.g. BV is Banach space is proven there and also some wlsc statements, so 
quote it for many things instead of Bar15}
\section{Notation}
In dieser Arbeit sei $\Omega\subset\mathbb{R}^2$ stets ein polygonal-berandetes
Lipschitz-Gebiet.

\todo[inline]{TODO usw alle Notationen einführen, die in der ganzen Arbeit
gelten}
\todo[inline]{FRAGE Dann zB CR Notation erst im entsprechenden Kapitel 
einführen oder auch das schon hier? Oder hier nur ABSOLUTE Grundlagen, extrem 
basic, und alles andere dann on the fly?}

\todo[inline]{TODO
 nur die Sachen rausschreiben/zusammentragen/zitieren (um später Theoreme und
 Gleichungen zitieren zu können statt Bücher) die auch wirklich gebraucht 
 werden später in Beweisen. Insbesondere am Ende nochmal durchgucken, was 
 wirklich gebraucht wurde und ungebrauchtes und/oder uninteressanter und/oder
 unwichtiges rauswerfen}

\section{Maßtheoretische Grundlagen}
 
\todo[inline]{TODO doch noch wenigstens eine einfach Def für Radon Maße (vor
allem mit Zitat zu einer Quelle}

\todo[inline]{TODO Maßtheorie für Vektormaße ist absolut nicht notwenig, da
meine Anwendung ausschließlich von R2 nach R geht Möglicherweise kann ich also
doch eine durchgehende Geschichte erzählen und insbesondere alles verstehen,
im Quelltext sind noch auskommentierte Theorem zur Maßtheorie}

% \begin{definition}[braides]\label{def:masz}
%   Eine Funktion $\mu:\Bcal(\Omega)\to\Rbb^N$ heißt (Vektor-) Maß auf $\Omega$,
%   wenn sie abzählbar additiv ist, d.h.\ für alle
%   $(B_j)_{j\in\Nbb}\subseteq\Bcal(\Omega)$ mit $B_j \cap B_k = \emptyset$ für
%   $j\neq k$ gilt
%   \begin{align*}
%     \mu\left( \dot{\bigcup_{j\in\Nbb}} B_j  \right) = \sum_{j\in\Nbb} \mu(B_j).
%   \end{align*}
%   Die Menge aller dieser Maße sei $\Mcal(\Omega;\Rbb^N)$.
% 
%   Im Fall $N=1$ heißt $\mu$ skalares Maß. Falls $\mu$ zusätzlich nur Werte in 
%   $[0,\infty)$
%   annimmt, heißt es positives Maß. Die Menge aller skalaren Maße sei 
%   $\Mcal(\Omega)$ und die Menge aller positiven Maße sei 
%   $\Mcal_+(\Omega)$.
% \end{definition}
% 
% \begin{remark}
%   Die übliche Anforderung $\mu(\emptyset) = 0$  an ein Maß ist äquivalent zur 
%   Bedingung, dass $A\in \Bcal(\Omega)$ existiert, sodass $\mu(A)$ in jeder 
%   Komponente endliche Werte annimmt. 
% 
%   Da in \Cref{def:masz} als Wertebereich $\Rbb^N$ gefordert wird, ist
%   diese äquivalente Bedingung erfüllt.
% \end{remark}
% 
% \begin{definition}\label{def:radonmasz}
%   Eine Funktion $\mu:\Bcal(\Omega)\to\Rbb^N$ heißt Radonmaß auf 
%   $\Omega$, wenn $\mu|_{\Bcal(\Omega')}$ ein Maß auf jeder Menge 
%   $\Omega'\ssubset \Omega$ (d.h.\,$\closure({\Omega'})\subset \interior({\Omega})$) 
%   ist.
% \end{definition}

\section{Direkte Methode der Variationsrechnung}
\todo[inline]{braucht es eigentlich nicht, im Beweis selbst werden ja alle 
Argumente gebracht, also diese section ist überflüssig}

\section{Subdifferential}
In diesem Abschnitt trage ich die in dieser Arbeit benötigten Eigenschaften 
des Subdifferentials eines Funktionals $F:X\to [-\infty,\infty]$ 
auf einem Banachraum 
$(X,\Vert\bullet\Vert_X)$ und die dafür benötigten Begriffe zusammen.

Zunächst eine grundlegende Definition.

\begin{definition}[\protect{\cite[S. 245, Definition 42.1]{Zei85}}]
  Sei $X$ ein Vektorraum, $M\subseteq X$ und $F:M\to\Rbb$. 
  
  Dann heißt die Menge $M$ konvex, wenn für alle $u,v\in M$ und alle $t\in
  [0,1]$ gilt $(1-t)u+tv\in M$.

  Ist $M$ konvex, so heißt $F$ konvex, falls für alle $u,v\in M$ und alle
  $t\in[0,1]$ gilt $F\big( (1-t)u+tv\big)\leq (1-t)F(u)+t F(v).$
\end{definition}

In \cite{Zei85} werden einige der folgenden Aussagen auf reellen lokal konvexen
Räumen $X$ formuliert.
Da nach \cite[S. 781, (43)]{Zei86} alle Banachräume 
(in \cite{Zei86} und \cite{Zei85} genannt \glqq B-spaces\grqq, \cite[S.
786]{Zei86}) lokal
konvex sind und in dieser Arbeit die Aussagen nur auf Banachräumen benötigt
werden, beschränke ich
die folgenden Aussagen, falls nicht anders spezifiziert, wie folgt.
Sei $(X,\Vert\bullet\Vert_X)$ ein reeller Banachraum und
$F:X\to [-\infty,\infty]$ ein Funktional auf $X$.

\begin{definition}[Subdifferential, \protect{\cite[S. 385,
  Definition 47.8]{Zei85}}]
  Für $u\in X$ mit $F(u)\neq\pm\infty$ heißt
  \begin{equation}
    \label{equ:subdifferential}
    \partial F(u)\coloneq 
    \{u^\ast\in X^\ast\ |\ 
    \forall v\in X\quad F(v)\geq F(u)+\langle u^\ast,v-u\rangle\}  
  \end{equation}
  Subdifferential von $F$ an der Stelle $u$. Für $F(u)=\pm\infty$ definiere
  $\partial F(u)\coloneq\emptyset$.

  Ein Element $u^\ast\in\partial F(u)$ heißt Subgradient von $F$ an der Stelle
  $u$.
\end{definition}

\begin{theorem}[\protect{\cite[S. 387, Proposition 47.12]{Zei85}}]
  \label{thm:extremalprinzip}
  Falls $F: X\to (-\infty,\infty]$ mit $F\nequiv\infty$, gilt
  $F(u)=\inf_{v\in X}F(v)$ genau dann, wenn $0\in\partial F(u)$.
\end{theorem}

\begin{theorem}[\protect{\cite[S. 387, Proposition 47.13]{Zei85}}]
  \label{thm:subdiffGateaux}
  Falls $F$ konvex ist und G\^{a}teaux-differenzierbar
  (in \cite{Zei86} und \cite{Zei85} genannt \glqq G-differentiable\grqq, \cite[S.
  135f.]{Zei86})
  an der Stelle $u\in X$ mit G\^{a}teaux-Differential $F'(u)$,
  gilt $\partial F(u)=\{F'(u)\}$.
  \todo[inline]{TODO checke Notation und Definition mit der
  Gateaux/Frechet-differenzierbarkeit, für die ich mich entschieden habe (d.h.
  meint Zeidler das gleiche  
  
  Bemerke, die Prop liefert noch wann das umgekehrte gilt, aber nur
  aufschreiben, wenn das mal benötigt wird in dieser Arbeit

  nutze vielleicht Zeidler I als Quelle für die Differentiale und vielleicht
  auch Notation?}
\end{theorem}

Das folgende Theorem folgt aus \cite[S. 389, Theorem 47.B]{Zei85} unter 
Beachtung der Tatsache, dass die Addition von Funktionalen 
$F_1,F_2,\ldots,F_n:X\to (-\infty,\infty]$ und die Addition von
Menge in $X^\ast$ kommutieren.

\begin{theorem}
  \label{thm:subdifferentialSumRule}
  Seien für $n\geq 2$ die Funktionale $F_1,F_2,\ldots,F_n: X\to
  (-\infty,\infty]$ konvex und es existiere
  ein $u_0\in X$ und ein $j\in\{1,2,\ldots,n\}$ mit $F_k(u_0)<\infty$
  für alle $k\in\{1,2\ldots,n\}$, 
  sodass für alle $k\in\{1,2,\ldots,n\}\setminus\{j\}$ das Funktional
  $F_k$ stetig an der Stelle $u_0$ ist.

  Dann gilt 
  \begin{align*}
    \partial (F_1+F_2+\ldots+ F_n)(u) 
    = \partial F_1(u)+\partial F_2(u)+ \ldots + \partial F_n(u) \quad\text{für
    alle } u\in X.
  \end{align*}
\end{theorem}

\begin{theorem}[\protect{\cite[S. 396f., Definition 47.15, Theorem
  47.F]{Zei85}}]
  \label{thm:subdifferentialMonotonicity}
  Sei $F:X\to (-\infty,\infty]$ konvex und unterhalbstetig mit $F\nequiv\infty$.

  Dann ist $\partial F(\bullet)$ monoton, das heißt 
  \begin{align*}
    \langle u^\ast-v^\ast,u-v\rangle\geq 0\quad \text{für alle } u,v\in X, 
    u^\ast \in \partial F(u), v^\ast \in \partial F(v).
  \end{align*}
\end{theorem}

\section{Karush-Kuhn-Tucker Bedingungen}

\section{Sattelpunktsprobleme}
In diesem Abschnitt können wir angelehnt an \cite[S. 4ff., S. 165ff.]{Aub79}
eine propere Funktion $f:U\to (-\infty,\infty]$ betrachten, wobei $U$ ein
Vektorraum ist.


\section{Funktionen Beschränkter Variation}

Dieser Abschnitt folgt Kapitel 10 von \cite{Bar15}.
Dabei sei $\Omega \subset \Rbb^n$ ein offenes, polygonal berandetes
Lipschitz-Gebiet.
\todo[inline]{Direkt von R2 ausgehen, weil mehr im Programm nicht geht?}
\todo[inline]{TODO jede übernommene Definition/Theorem/etc. zitieren trotz
Disclaimer oben?  jede Notation erklären bzw. definieren? Falls ja; am Anfang
oder Ende der Arbeit?}

\begin{definition}
  \todo{TODO vielleicht zu Grundlagen über Radonmaße verschieben}
  Die Vervollständigung des Raums $C^\infty_C(\Omega;\Rbb^m)$ bezüglich der 
  Norm
  $\Vert\bullet\Vert_{L^\infty(\Omega)}$ ist ein separabler Banachraum und wird
  bezeichnet mit 
  $C_0(\Omega; \Rbb^m)$.
  Der Dualraum $\Mcal(\Omega;\Rbb^m)$ von $C_0(\Omega; \Rbb^m)$ wird
  durch den Riesz'schen Darstellungssatz \todo{TODO zitiere?} identifiziert mit
  dem Raum aller (vektoriellen) Radonmaße. Dabei wird die Anwendung
  von $\mu\in \Mcal(\Omega;\Rbb^m)$
  auf $\phi\in C_0(\Omega;\Rbb^m)$ identifiziert mit
  \begin{align*}
    \langle \mu, \phi\rangle \coloneqq \int_\Omega \phi \dmu =
    \int_\Omega \phi(x) \dmu(x).
  \end{align*}
\end{definition}

\begin{definition}[Funktionen beschränkter Variation]
  Eine Funktion $u\in L^1(\Omega)$ ist von beschränkter Variation, wenn ihre
  distributionelle Ableitung ein Radonmaß definiert, d.h.\ eine Konstante
  $c\geq 0$ existiert, sodass 
  \begin{align}
    \label{eq:boundedVariation}
    \langle Du,\phi\rangle \coloneqq - \int_\Omega u\Div (\phi) \dx 
    \leq c\Vert\phi\Vert_{L^\infty(\Omega)}
  \end{align}
  für alle $\phi\in C^1_C(\Omega;\Rbb^n)$.

  Die minimale Konstante $c\geq 0$, die \eqref{eq:boundedVariation} erfüllt,
  heißt totale Variation von $Du$ und besitzt die Darstellung
  \begin{align*}
    |u|_{\BV(\Omega)} = \sup_{\substack{\phi\in C^1_C(\Omega;\Rbb^n)\\
    \Vert\phi\Vert_{L^\infty(\Omega)}\leq 1}}-\int_\Omega u\Div (\phi)\dx.
  \end{align*}

  Durch $|\bullet|_{\BV(\Omega)}$ ist eine Seminorm auf $\BV(\Omega)$
  gegeben.

  Der Raum aller Funktionen beschränkter Variation $\BV(\Omega)$
  ist ausgestattet mit der Norm 
  \begin{align*}
    \Vert u \Vert_{\BV(\Omega)} \coloneqq \Vert u\Vert_{L^1(\Omega)} +
    |u|_{\BV(\Omega)}
  \end{align*}
  für $u\in\BV(\Omega)$.
\end{definition}

\begin{remark}
  Es gilt $W^{1,1}(\Omega)\subset\BV(\Omega)$ und 
  $\Vert u \Vert_{\BV(\Omega)}=\Vert u \Vert_{W^{1,1}(\Omega)}$ für alle
  $u\in W^{1,1}(\Omega)$.
  {\color{red}es gilt für diese u tatsächlich (nach BV lecture04) ca.
  $|u|_{\BV(\Omega)=\Vert \nabla u \Vert_{L^1(\Omega)}}$}
\end{remark}

\begin{definition}
  Sei $(u_n)_{n\in\Nbb}\subset \BV(\Omega)$ und sei $u\in \BV(\Omega)$ mit
  $u_n\rightarrow u$ in $L^1(\Omega)$ für $n\rightarrow\infty$.
  \begin{itemize}
    \item[(i)]
      Die Folge $(u_n)_{n\in\Nbb}$ konvergiert strikt gegen $u$,
      wenn $|u_n|_{\BV(\Omega)}\rightarrow |u|_{\BV(\Omega)}$ für $n\rightarrow\infty$.
      {\color{red} strikte Konvergenz gdw. ($\Vert u-u_n\Vert_{L^1(\Omega)}
      \to 0$ und $|u_n|_{\BV(\Omega)}\rightarrow |u|_{\BV(\Omega)}$)
      was impliziert $\Vert u_n\Vert_\BV\to \Vert u\Vert_{\BV}$ aber nicht
      unbedingt $\Vert u_n - u \Vert_\BV\to 0$, da nicht folgt, dass 
      $|u_n - u|_\BV\to 0$
      
      aus BV Konvergenz, also $\Vert u_n - u \Vert_\BV\to 0$, folgt hingegen 
      aber ($\Vert u-u_n\Vert_{L^1(\Omega)}
      \to 0$ und $|u_n - u|_{\BV(\Omega)}\rightarrow 0$), also insbesondere
      ($\Vert u-u_n\Vert_{L^1(\Omega)}
      \to 0$ und $|u_n|_{\BV(\Omega)}\rightarrow |u|_{\BV(\Omega)}$), d.h.
      strikte Konvergenz
      
      jede BV konvergente Folge ist also strikt konvergent aber nicht umgekehrt,
      es gibt also mehr strikt konvergente Folgen, deshalb klingt es sinnvoll,
      dass wir BV Funktionen durch $C^\infty$ Funktionen (usw.) approximieren
      können bzgl strikter Konvergenz aber nicht bzgl BV Konvergenz (strong 
      topology, vgl.  Ende von BV lecture04}
    \item[(ii)] Die Folge $(u_n)_{n\in\Nbb}$ konvergiert
      schwach gegen $u$, wenn
      $Du_n\rightharpoonup^\ast Du$ in 
      $\Mcal(\Omega;\Rbb^n)$ für $n\rightarrow\infty$, d.h.\ für alle
      $\phi\in C_0(\Omega;\Rbb^n)$ gilt 
      $\langle Du_n,\phi\rangle\rightarrow \langle Du,\phi\rangle$ für 
      $n\rightarrow\infty$.
  \end{itemize}
\end{definition}

\begin{theorem}[Schwache Unterhalbstetigkeit]
  \label{thm:wlsc}
  Seien $(u_n)_{n\in\Nbb}\subset\BV(\Omega)$ und $u\in L^1(\Omega)$ mit
  $|u_n|_{\BV(\Omega)}\leq c$ für ein $c>0$ und alle $n\in\Nbb$ und
  $u_n\rightarrow u$ in $L^1(\Omega)$ für $n\rightarrow\infty$.

  Dann gilt $u\in\BV(\Omega)$ und $|u|_{\BV(\Omega)}\leq
  \liminf_{n\rightarrow\infty}|u_n|_{\BV(\Omega)}.$
  Außerdem gilt $u_n\rightharpoonup u$ in $\BV(\Omega)$ für $n\rightarrow
  \infty$.
\end{theorem}

\begin{theorem}[Appoximation mit glatten Funktionen]
  \label{thm:approximationBySmoothFunctions}
  Die Räume $C^\infty(\overline\Omega)$ und $C^\infty(\Omega)\cap\BV(\Omega)$
  liegen dicht in $\BV(\Omega)$ bezüglich strikter Konvergenz.

  {\color{red}BV lecture05 Thm 2.4 liefert sogar (Folge in
  $C^\infty\cap W^{1,1}$) sowohl strikte als auch schwache Konvergenz gegen
  gegebenes $u\in\BV$, also wir haben nach diesem Thm die Dichte von
  $C^\infty\cap W^{1,1}$ bzgl. strikter und schwacher BV Konvergenz
  
  da für Folgen in $W^{1,1}$ BV und $W^{1,1}$ Norm übereinstimmen und da
  $W^{1,1}$ der Abschluss von $C^\infty$ bzgl. der $W^{1,1}$ Norm ist (
  $C^\infty$ dicht in $W^{ 1,1 }$ bzgl. W11 Norm), ist
  für $W^{1,1}$ Funktionen W11 auch Abschluss von Cinfty bzgl
  der BV Norm (Cinfty dicht in W11 bzhl BV Norm)
  
  JETZT DER KNACKPUNKT und wie aus Thmeorem 2.6 gefolgert werden kann was in BV lecture
  steht: da BV Konvergenz strikte Konvergenz impliziert, ist 
  also Cinfty auch dicht in W11 bzgl strikter Konvergenz und W11 ist Teilmenge
  von BV. Da A dicht in B und B Teilmenge C impliziert das B dicht in C (wiki,
  natürlich beides bzgl gleicher Metrik), folgt insgesamt W11 dicht in BV bzgl
  strikter Konvergenz
  
  MORGEN CONTINUE IN WITH THIS IN EXISTENCE PROOF: NOW I might know WHY
  infimizing sequence in BV can simply be choosen in W11 or something
  (Think about it and what bartels did)}
\end{theorem}

\begin{theorem}
  \label{thm:compactness}
  Sei $(u_n)_{n\in\Nbb}\subset \BV(\Omega)$ eine beschränkte Folge. Dann 
  existiert eine Teilfolge $(u_{n_k})_{k\in\Nbb}$ und ein $u\in\BV(\Omega)$,
  sodass $u_{n_k}\rightharpoonup u$ in $\BV(\Omega)$ für $k\rightarrow\infty$.
  {\color{red}augenscheinlich nicht in BV lecture}
\end{theorem}

\begin{theorem}
  \label{thm:embeddingBVintoLp}
  Die Einbettung $\BV(\Omega)\to L^p(\Omega)$ ist stetig für 
  $1\leq p\leq n/(n-1)$ und kompakt für $1\leq p< n/(n-1)$
\end{theorem}\todo[inline]{TODO vielleicht wichtig, Quelle braucht es noch}

\begin{theorem}[Spuroperator]
  \label{thm:traceOperator}
  Es existiert ein linearer Operator $T:\BV(\Omega)\to L^1(\partial\Omega)$
  mit $T(u) = u|_{\partial\Omega}$ für alle $u\in\BV(\Omega)\cap
  C(\overline\Omega)$.

  Der Operator $T$ ist stetig bezüglich strikter Konvergenz in $\BV(\Omega)$,
  aber nicht stetig bezüglich schwacher Konvergenz in $\BV(\Omega)$. 
\end{theorem}\todo[inline]{TODO finde Quelle, nur ein Remark in Bartels (wird 
aber für CCs Funktional offensichtlich gebraucht, es gibt noch weitere Aussagen
in Bartels (zB integration by parts aber erstmal nur Existens und Stetigkeit 
hier (wie gesagt, nur was gebraucht wird zitieren, oder?)}


\chapter{Das kontinuierliche Problem}
\label{chap:continuousProblem}
In diesen Kapitel wollen wir für einen Parameter $\alpha\in\Rbb_+$ und eine
Funktion $f\in L^2(\Omega)$ folgendes Minimierungsproblem untersuchen.

\begin{problem}\label{prob:continuousProblem}
  Finde $u\in \BV(\Omega)\cap L^2(\Omega)$, sodass
  $u$ das Funktional
  \begin{align}\label{eq:continuousProblem}
    E(v)\coloneqq \frac{\alpha}{2}\Vert v\Vert^2 + |v|_{\BV(\Omega)}
    +\Vert v\Vert_{L^1(\partial\Omega)}-\int_\Omega fv\dx
  \end{align}
  unter allen $v\in\BV(\Omega)\cap L^2(\Omega)$ minimiert.
\end{problem}
Dabei ist der Term $\Vert v\Vert_{L^1(\partial\Omega)}$ wohldefiniert, da
nach \cite[S. 400, Theorem 10.2.1]{ABM14} eine lineare, stetige Abbildung
$T:\BV(\Omega)\to L^1(\partial\Omega)$ existiert mit $T(v) =
v|_{\partial\Omega}$ für alle $v\in\BV(\Omega)\cap C(\overline\Omega)$.
Wie in \Cref{chap:introduction} beschrieben, hat \Cref{prob:continuousProblem}
für homogene Randdaten die gleichen Minimierer wie das ROF-Modell-Problem mit
Eingangssignal $g\in L^2(\Omega)$, wenn $f=\alpha g$.

\begin{remark}
  Für $d\in\{2,3\}$ und ein beschränktes Lipschitz-Gebiet $U\subset\Rbb^d$ ist
  nach \cite[S. 302, Remark 10.5 (i)]{Bar15} die Einbettung
  $\BV(U)\hookrightarrow L^p(U)$ stetig, wenn $1\leq p\leq d/(d-1)$. 
  % Nach \cite[S. 399, Theorem 10.1.3]{ABM14} gilt diese Aussage auch für
  % $d\in\Nbb$, wenn $U\subset\Rbb^d$ offen, beschränkt und 1-regulär ist.
  Damit ist $\BV(\Omega)$ Teilmenge von $L^2(\Omega)$ und die Lösung von
  \Cref{prob:continuousProblem} kann in $\BV(\Omega)$ gesucht werden. 
  Wir vernachlässigen diese Vereinfachung und erreichen dadurch, dass alle
  Aussagen dieses Kapitels auch gelten würden, wenn
  $\Omega\subset\Rbb^d$ mit $d\in\Nbb$ ein beschränktes Lipschitz-Gebiet wäre.
\end{remark}

In den folgenden Abschnitten zeigen wir, dass
\Cref{prob:continuousProblem} eine eindeutige Lösung $u\in\BV(\Omega)\cap
L^2(\Omega)$ besitzt. 
Außerdem beschreiben wir, wie $f$ für eine gegebene Lösung $u$ konstruiert
werden kann und welche Eigenschaften $u$ dafür erfüllen muss.


\section{Existenz eines eindeutigen Minimierers}
Zunächst zeigen wir, dass \Cref{prob:continuousProblem} eine Lösung besitzt.
Dafür benötigen wir die folgende Formulierung der Youngschen Ungleichung.

\begin{lemma}[Youngsche Ungleichung]
  \label{lem:young}
  Seien $a,b\in\Rbb$ und $\varepsilon\in\Rbb_+$ beliebig. Dann gilt
  \begin{align*}
    ab\leq\frac{1}{\varepsilon}a^2+\frac{\varepsilon}{4}b^2. 
  \end{align*}
\end{lemma}

Außerdem wird im Beweis folgende Aussage benötigt, die direkt aus \cite[S. 183,
Theorem 1]{EG92} folgt, da
$0\in\BV\!\left(\Rbb^d\setminus\overline\Omega\right)$ mit
$|0|_{\BV\!\left(\Rbb^d\setminus\overline\Omega\right)}=0$ und
$0|_{\partial\Omega}=0$.

\begin{lemma}
  \label{lem:bvExtension}
  Sei $v\in\BV(\Omega)$.
  Definere die Fortsetzung $\tilde{v}$ von $v$ für alle $x\in\Rbb^d$ durch
  \begin{align*}
    \tilde{v}(x)\coloneqq
    \begin{cases}
      v(x),  &\text{ falls } x\in\Omega,\\
      0,     &\text{ falls } x\in\Rbb^d\setminus\overline\Omega.
    \end{cases} 
  \end{align*}
  Dann gilt $\tilde{v}\in\BV\!\left(\Rbb^d\right)$ und
  $\left|\tilde{v}\right|_{\BV\!\left(\Rbb^d\right)}
  = |v|_{\BV(\Omega)}+\Vert v\Vert_{L^1(\partial\Omega)}$.
\end{lemma}

\begin{theorem}[Existenz einer Lösung]
  \label{thm:contProblemExistence}
  \Cref{prob:continuousProblem} besitzt eine Lösung \\$u\in\BV(\Omega)\cap
  L^2(\Omega)$.
\end{theorem}

\begin{proof}
  Die Beweisidee ist die Anwendung der direkten Methode der Variationsrechnung
  (cf.\ z.B.\ \cite{Dac89}) unter Nutzung der in \Cref{sec:bvFunctions}
  aufgeführten Eigenschaften der schwachen Konvergenz in $\BV(\Omega)$.

  Für alle $v\in L^2(\Omega)\subseteq L^1(\Omega)$ gilt mit der Hölderschen
  Ungleichung für $p=q=2$, dass
  \begin{equation}\label{eq:hoelderL2BiggerL1}
    \Vert v\Vert_{L^1(\Omega)} 
    = \Vert 1\cdot v\Vert_{L^1(\Omega)}
    \leq \Vert 1\Vert\Vert v\Vert
    =\sqrt{|\Omega|} \Vert v\Vert.
  \end{equation}
  Dann folgt für das Funktional $E$ in \eqref{eq:continuousProblem} für alle
  $v\in \BV(\Omega)\cap L^2(\Omega)$ durch die Cauchy-Schwarzsche Ungleichung,
  die Youngsche Ungleichung aus \cref{lem:young} und Ungleichung
  \eqref{eq:hoelderL2BiggerL1}, dass

  \begin{equation}
    \label{eq:contProbBddFromBelow}
    \begin{aligned}
      E(v)&=\frac{\alpha}{2}\Vert v\Vert^2 + |v|_{\BV(\Omega)}
      +\Vert v\Vert_{L^1(\partial\Omega)}-\int_\Omega fv\dx\\
      &\geq 
      \frac{\alpha}{2}\Vert v\Vert^2 + |v|_{\BV(\Omega)}
      +\Vert v\Vert_{L^1(\partial\Omega)}
      -\Vert f\Vert\Vert v\Vert\\
      &\geq 
      \frac{\alpha}{2}\Vert v\Vert^2 + |v|_{\BV(\Omega)}
      +\Vert v\Vert_{L^1(\partial\Omega)}
      -\frac{1}{\alpha}\Vert f\Vert^2
      -\frac{\alpha}{4}\Vert v\Vert^2\\
      &\geq 
      \frac{\alpha}{4}\Vert v\Vert^2 + |v|_{\BV(\Omega)}
      +\Vert v\Vert_{L^1(\partial\Omega)}-\frac{1}{\alpha}\Vert
      f\Vert^2\\
      &\geq 
      \frac{\alpha}{4|\Omega|}\Vert v\Vert_{L^1(\Omega)}^2 + |v|_{\BV(\Omega)}
      +\Vert v\Vert_{L^1(\partial\Omega)}-\frac{1}{\alpha}\Vert
      f\Vert^2\\
      &\geq -\frac{1}{\alpha}\Vert f\Vert^2.
    \end{aligned}
  \end{equation}
  Somit ist $E$ nach unten beschränkt, was die Existenz einer infimierenden
  Folge $(u_n)_{n\in\Nbb}\subset\BV(\Omega)\cap L^2(\Omega)$ von $E$ 
  impliziert, das heißt
  $(u_n)_{n\in\Nbb}$ erfüllt $$\lim_{n\rightarrow\infty}E(u_n) =
  \inf_{v\in\BV(\Omega)\cap L^2(\Omega)}E(v).$$ 
  Ungleichung \eqref{eq:contProbBddFromBelow} impliziert außerdem, dass
  $E(u_n)\to\infty$ für $n\to\infty$, falls $|u_n|_{\BV(\Omega)}\to\infty$ oder
  $\Vert u_n\Vert_{L^1(\Omega)}\to\infty$ für $n\to\infty$. 
  Daraus folgt insbesondere, dass $E(u_n)\to\infty$ für $n\to\infty$, falls
  $\Vert u_n\Vert_{\BV(\Omega)}\to\infty$ für $n\to\infty$ .
  Deshalb muss die Folge $(u_n)_{n\in\Nbb}$ beschränkt in $\BV(\Omega)$ sein.
  Nun garantiert \cref{thm:compactness} die Existenz einer in $\BV(\Omega)$
  schwach konvergenten Teilfolge $(u_{n_k})_{k\in\Nbb}$ von $(u_n)_{n\in\Nbb}$
  mit schwachen Grenzwert $u\in\BV(\Omega)$. 
  Ohne Beschränkung der Allgemeinheit ist
  $(u_{n_k})_{k\in\Nbb}=(u_n)_{n\in\Nbb}$.
  Aus der schwachen Konvergenz von $(u_n)_{n\in\Nbb}$ in $\BV(\Omega)$ gegen
  $u$ folgt nach Definition, dass $(u_n)_{n\in\Nbb}$ stark, und damit
  insbesondere auch schwach, in $L^1(\Omega)$ gegen $u$ konvergiert.

  Weiterhin folgt aus (\ref{eq:contProbBddFromBelow}), dass
  $E(v)\rightarrow\infty$ für $\Vert v\Vert\rightarrow\infty$. 
  Somit muss $(u_n)_{n\in\Nbb}$ auch beschränkt sein bezüglich der Norm
  $\Vert\bullet\Vert$ und besitzt deshalb, wegen der Reflexivität von
  $L^2(\Omega)$, eine Teilfolge (ohne Beschränkung der Allgemeinheit weiterhin
  bezeichnet mit $(u_n)_{n\in\Nbb}$), die in $L^2(\Omega)$ schwach gegen einen
  Grenzwert $\overline{u}\in L^2(\Omega)$ konvergiert. 
  Damit gilt für alle $w\in L^2(\Omega)\cong L^2(\Omega)^\ast$ und, da
  $L^\infty(\Omega)\subseteq L^2(\Omega)$, insbesondere auch für alle $w\in
  L^\infty(\Omega)\cong L^1(\Omega)^\ast$, dass 
  \begin{align*}
    \lim_{n\to\infty}\int_\Omega u_n w\dx =\int_\Omega \overline{u} w\dx.
  \end{align*}
  Das bedeutet, dass $(u_n)_{n\in\Nbb}$ auch schwach in $L^1(\Omega)$ gegen
  $\overline{u}\in L^2(\Omega)\subseteq L^1(\Omega)$ konvergiert. 
  Da schwache Grenzwerte eindeutig bestimmt sind, gilt insgesamt $u=\overline u
  \in L^2(\Omega)$, das heißt $u\in\BV(\Omega)\cap
  L^2(\Omega)$.
  Nun definieren wir für alle
  $n\in\Nbb$ und für alle 
  $x\in\Rbb^d$ die Fortsetzungen
  \begin{align*}
    \tilde{u}_n(x)
    &\coloneqq
    \begin{cases}
      u_n(x),  &\text{ falls } x\in\Omega,\\
      0,     &\text{ falls } x\in\Rbb^d\setminus\overline\Omega
    \end{cases} 
    &&\text{und }
    &\tilde{u}(x)
    &\coloneqq
    \begin{cases}
      u(x),  &\text{ falls } x\in\Omega,\\
      0,     &\text{ falls } x\in\Rbb^d\setminus\overline\Omega.
    \end{cases} 
  \end{align*}
  Dann gilt nach \cref{lem:bvExtension} sowohl
  \begin{align*}
    \tilde{u}_n
    &\in
    \BV\!\big(\Rbb^d\big)
    &&\text{und}
    &\left|\tilde{u}_n\right|_{\BV\!\left(\Rbb^d\right)} 
    &= 
    |u_n|_{\BV(\Omega)}+\Vert u_n\Vert_{L^1(\partial\Omega)}
    \quad\text{für alle }n\in\Nbb \text{ als auch}\\
    \tilde{u}
    &\in
    \BV\!\big(\Rbb^d\big)
    &&\text{und}
    &\left|\tilde{u}\right|_{\BV\!\left(\Rbb^d\right)} 
    &=
    |u|_{\BV(\Omega)}+\Vert u\Vert_{L^1(\partial\Omega)}.
  \end{align*}
  Da $(u_n)_{n\in \Nbb}$ infimierende Folge von $E$ ist, muss die Folge
  \begin{align*}
    \left(\left|\tilde{u}_n\right|_{\BV\!\left(\Rbb^d\right)}\right)_{n\in\Nbb} 
    = \left(|u_n|_{\BV(\Omega)}+
    \Vert u_n\Vert_{L^1(\partial\Omega)}\right)_{n\in\Nbb}
  \end{align*}
  beschränkt sein.
  Außerdem folgt aus den Definitionen von $\tilde{u}$ und 
  $\tilde{u}_n$ für alle $n\in\Nbb$ und der bereits bekannten Eigenschaft 
  $u_n\to u$ in $L^1(\Omega)$ für $n\to\infty$, dass
  \begin{align*}
    \left\Vert \tilde{u}_n - \tilde{u}\right\Vert_{L^1\!\left(\Rbb^d\right)} 
    &= \int_{\Rbb^d} \left|\tilde{u}_n - \tilde{u}\right|\dx
    = \int_\Omega |u_n - u|\dx
    = \Vert u_n - u\Vert_{L^1(\Omega)}\to 0\quad\text{für }n\to\infty,
  \end{align*}
  das heißt $\tilde{u}_n \to \tilde{u}$ in $L^1\left(\Rbb^d\right)$ für
  $n\to\infty$.
  Insgesamt ist also $\left(\tilde{u}_n\right)_{n\in\Nbb}$ eine Folge in
  $\BV\!\left(\Rbb^d\right)$, die in $L^1\!\left(\Rbb^d\right)$ gegen
  $\tilde{u}\in\BV\!\left(\Rbb^d\right)\subseteq L^1\!\left(\Rbb^d\right)$
  konvergiert und 
  $\sup_{n\in\Nbb} \left|\tilde{u}_n\right|_{\BV\!\left(\Rbb^d\right)}<\infty$
  erfüllt.
  Somit folgt mit
  \cref{thm:wlsc}  
  \begin{equation}
    \label{eq:wlscOfExtension}
    \begin{aligned}
      |u|_{\BV(\Omega)} +\Vert u\Vert_{L^1(\partial\Omega)}
      = \left|\tilde{u}\right|_{\BV\left(\Rbb^d\right)}
      &\leq\liminf_{n\to\infty}
      \left|\tilde{u}_n\right|_{\BV\left(\Rbb^d\right)}\\
      &= \liminf_{n\to\infty} \left(|u_n|_{\BV(\Omega)} +
      \Vert u_n\Vert_{L^1(\partial\Omega)}\right).
    \end{aligned}
  \end{equation}
  Die Funktionen $\Vert\bullet\Vert^2$ und $-\int_\Omega
  f\bullet\dx$ sind auf $L^2(\Omega)$ stetig und konvex, was impliziert,
  dass sie schwach unterhalbstetig auf $L^2(\Omega)$ sind. Da wir bereits
  wissen, dass $u_n\rightharpoonup u$ in $L^2(\Omega)$ für $n\to\infty$, 
  folgt
  \begin{align*}
    \frac{\alpha}{2}\Vert u\Vert-\int_\Omega fu\dx
    \leq \liminf_{n\to\infty}
    \left(\frac{\alpha}{2}\Vert u_n\Vert
    -\int_\Omega fu_n\dx\right).
  \end{align*}
  Damit und mit Ungleichung \eqref{eq:wlscOfExtension} gilt insgesamt
  \begin{align*}
    \inf_{v\in\BV(\Omega)\cap L^2(\Omega)}E(v)\leq
    E(u)\leq\liminf_{n\rightarrow\infty} E\left(u_n\right) =
    \lim_{n\rightarrow\infty}E\left(u_n\right) = \inf_{v\in\BV(\Omega)\cap
    L^2(\Omega)}E(v),
  \end{align*}
  das heißt $\min_{v\in\BV(\Omega)\cap L^2(\Omega)} E(v) = E(u)$.
\end{proof}

Nachdem wir gezeigt haben, dass für \Cref{prob:continuousProblem} eine
Lösung existiert, beweisen wir als nächstes ein Theorem, das direkt impliziert,
dass diese Lösung eindeutig ist. 
\begin{theorem}[Stabilität und Eindeutigkeit]
  \label{thm:contProbStabAndUniqu}
  Seien $u_1,u_2\in \BV(\Omega)\cap L^2(\Omega)$ die Minimierer des Problems
  \ref{prob:continuousProblem} mit $f_1,f_2\in L^2(\Omega)$ anstelle von $f$,
  das heißt für $\ell\in\{1,2\}$ minimiere $u_\ell$ das Funktional
  \begin{align*}
    E_\ell
    \coloneqq 
    \frac{\alpha}{2}\Vert v\Vert^2 + |v|_{\BV(\Omega)} 
    + \Vert v\Vert_{L^1(\partial\Omega)} - \int_\Omega f_\ell v\dx
  \end{align*}
  unter allen $v\in\BV(\Omega)\cap L^2(\Omega)$.
  Dann gilt 
  \begin{align*}
    \Vert u_1 - u_2\Vert 
    \leq\frac{1}{\alpha}\Vert f_1-f_2\Vert.
  \end{align*}
\end{theorem}

\begin{proof}
  Wir folgen der Argumentation im Beweis von \cite[S. 304, Theorem 10.6]{Bar15}.

  Zunächst definieren wir die Funktionale $F: L^2(\Omega)\to
  \Rbb\cup\{\infty\}$ und $G_\ell:L^2(\Omega)\to \Rbb$, $\ell\in\{1,2\}$, für
  alle $u\in L^2(\Omega)$ durch
  \begin{align*}
    F(u) 
    &\coloneqq 
    \begin{cases}
      |u|_{\BV(\Omega)} + \Vert u \Vert_{L^1(\partial\Omega)}, 
      &\text{ falls } u\in\BV(\Omega)\cap L^2(\Omega),\\
      \infty,&\text{ falls } u\in L^2(\Omega)\setminus\BV(\Omega)
    \end{cases}
    \quad\text{und }\\
    G_\ell(u)
    &\coloneqq 
    \frac{\alpha}{2}\Vert u\Vert^2 - \int_\Omega f_\ell u\dx.
  \end{align*}
  Damit gilt für $\ell\in\{1,2\}$ und alle
  $u\in\BV(\Omega)\cap L^2(\Omega)$, dass $E_\ell(u) =  F(u)+G_\ell(u)$.
  Für $\ell\in\{1,2\}$ ist $G_\ell$ Fr\'echet-differenzierbar und die
  Fr\'echet-Ableitung $G_\ell'(u): L^2(\Omega)\to\Rbb$ von $G_\ell$ an der
  Stelle $u\in L^2(\Omega)$ ist für alle $v\in L^2(\Omega)$
  gegeben durch
  \begin{align*}
    dG_\ell(u;v) = \alpha (u,v) - \int_\Omega f_\ell v\dx 
    = (\alpha u-f_\ell ,v).
  \end{align*}
  Das Funktional $F$ ist konvex, unterhalbstetig und es gilt $F\nequiv\infty$.
  Deshalb ist nach \cref{thm:subdifferentialMonotonicity} das Subdifferential
  $\partial F$ von $F$ monoton, das heißt für alle $\mu_\ell\in \partial
  F(u_\ell)$, $\ell\in\{1,2\}$, gilt
  \begin{align}\label{eq:stabilityAndUniqueness:monotonicityOfSubdifferential}
    (\mu_1-\mu_2,u_1-u_2)\geq 0.
  \end{align}
  Für $\ell\in\{1,2\}$ gilt, dass $E_\ell$ konvex ist und von $u_\ell$ in
  $\BV(\Omega)\cap L^2(\Omega)$ minimiert wird. 
  Außerdem gilt $E_\ell\nequiv\infty$ und $G_\ell$ ist stetig.
  Somit gilt nach \cref{thm:extremalprinciple}, 
  \cref{thm:subdifferentialSumRule} und \cref{thm:subdiffGateaux}, dass
  $0\in\partial E_\ell(u_\ell) = \partial F(u_\ell)+\partial
  G_\ell(u_\ell)=\partial F(u_\ell)+ \{G_\ell'(u_\ell)\}.$ 
  Daraus folgt
  $-G_\ell'(u_\ell)\in\partial F(u_\ell)$.
  Zusammen mit Ungleichung
  \eqref{eq:stabilityAndUniqueness:monotonicityOfSubdifferential}
  impliziert das
  \begin{align*}
    \big( -(\alpha u_1 - f_1) -(- (\alpha u_2 - f_2)), u_1 - u_2\big)
    \geq 0.
  \end{align*}
  Durch Umformen und Anwenden der Cauchy-Schwarzschen Ungleichung erhalten wir
  \begin{align*}
    \alpha \Vert u_1 - u_2 \Vert^2
    &\leq
    \big(f_1 -f_2, u_1-u_2 \big)\\
    &\leq
    \Vert f_1-f_2\Vert\Vert u_1 - u_2\Vert.
  \end{align*}
  Falls $\Vert u_1 - u_2 \Vert = 0$, gilt die zu zeigende Aussage.
  Ansonsten führt die Division durch $\alpha\Vert u_1 - u_2 \Vert\neq 0$ den
  Beweis zum Abschluss.
\end{proof}

\section{Konstruktion eines Experiments mit bekannter Lösung}
%\section{Konstruktion eines Eingangssignals für eine gegebene Lösung}
\label{sec:constructionInputSignal}

Für die numerische Untersuchung der primalen-dualen Iteration aus
\Cref{chap:algorithm} ist es sinnvoll
Eingangssignale $f$ für \Cref{prob:continuousProblem} gegeben zu haben, für die
der entsprechende gesuchte Minimierer bekannt ist. 
Als Grundlage für die Konstruktion solcher Signale nutzen wir die folgende 
Aussage von Professor Carstensen.

Sei $u:\Omega\to\Rbb$ gegeben als Funktion in Polarkoordinaten. 
Dabei beschränken wir uns auf vom Polarwinkel unabhängige Funktionen, das heißt
für alle $x\in\Omega$ gelte
$u(x)\coloneqq u_P\big(|x|\big)$ für $u_P:[0,\infty)\to\Rbb$. 
Weiterhin fordern wir $u_P(r)=0$ für $r\geq 1$ und die Existenz
der partiellen Ableitung $\partial_r u_P$ fast überall in $[0,\infty)$.
Außerdem existiere fast überall in $[0,\infty)$ die partielle Ableitung
des für $r\in[0,\infty)$ definierten Ausdrucks
\begin{align*}
  \sgn\big(\partial_r u_P(r)\big)
  \coloneqq
  \begin{cases}
    -1 &\text{für }\partial_r u_P(r)<0,\\
    x\in[0,1] &\text{für }\partial_r u_P(r)=0,\\ 
    1 &\text{für }\partial_r u_P(r)>0.
  \end{cases}
\end{align*}
Des Weiteren fordern wir $\sgn\big(\partial_r u_P(r)\big)/r\to 0$ für $r\to 0$, 
damit $f_P$ in der folgenden Definition stetig in $0$ fortgesetzt werden kann.
Sei $f_P:[0,\infty)\to\Rbb$ gegeben durch
\begin{align}
  \label{eq:constructionInputSignal}
  f_P(r)
  \coloneqq 
  \alpha u_P(r) - \partial_r\left(\sgn\big(\partial_r u_P(r)\big)\right) 
  - \frac{\sgn\big(\partial_r u_P(r)\big)}{r}
  \quad\text{für alle }r\in[0,\infty).
\end{align}
Dann ist $u$ Lösung von \Cref{prob:continuousProblem}, wenn das Eingangssignal
auf $\Omega\supseteq \left\{w\in\Rbb^2\,\middle|\, |w|\leq 1\right\}$ für fast
alle $x\in\Omega$ gegeben ist durch $f(x)\coloneqq f_P\big(|x|\big)$.

In unseren Experimenten wird uns die garantierte untere Energieschranke aus
\Cref{thm:gleb} interessieren. 
Da dieses Theorem für das Eingangssignal voraussetzt, dass $f\in
H^1_0(\Omega)$, müssen wir noch die folgenden Bedingungen an $u_P$ formulieren.
Hinreichend für $f\in H^1_0(\Omega)$ ist nach \Cref{eq:constructionInputSignal},
dass  $u_P$, $\partial_r\big(\sgn(\partial_r u_P)\big)$ und $\sgn(\partial_r
u_P)$ stetig sind und 
\begin{align*}
  u_P(1)
  =
  \partial_r\left( \sgn\big(\partial_r u_P(1)\big)\right)
  =
  \sgn\big(\partial_r u_P(1)\big)
  =
  0.
\end{align*}
Mit diesen Einschränkung gilt insbesondere $u\in H^1_0(\Omega)$, weshalb 
die exakte Energie $E(u)$ nach \Cref{rem:bvSeminorm} berechnet werden kann
durch 
\begin{align*}
  E(u)
  =
  \frac{\alpha}{2}\Vert u\Vert^2 + \Vert u\Vert_{W^{1,1}(\Omega)} 
  - \int fu\dx.
\end{align*}
Um also $E(u)$ berechnen zu können, wird der schwache Gradient $\nabla u$ von
$u$ benötigt und um die garantierte untere Energieschranke $\Egleb$ aus
\Cref{eq:gleb} zu berechnen, wird der schwache Gradient $\nabla f$ von $f$
benötigt.
Deshalb betrachten wir an dieser Stelle noch kurz die benötigten
Zusammenhänge zwischen den partiellen Ableitungen in kartesischen Koordianten
und in Polarkoordinaten für eine hinreichend glatte Funktion $g_P$.
Sei
\begin{align*}
  \atan(x_2,x_1)\coloneqq
  \begin{cases}
    \arctan\left( \frac{x_2}{x_1} \right),& \text{wenn }x_1>0,\\
    \arctan\left( \frac{x_2}{x_1} \right) +\pi,& \text{wenn }x_1<0,x_2\geq 0,\\
    \arctan\left( \frac{x_2}{x_1} \right) -\pi,& \text{wenn }x_1<0,x_2<0,\\
    \frac{\pi}{2},& \text{wenn }x_1=0,x_2>0,\\
    -\frac{\pi}{2},& \text{wenn }x_1=0,x_2<0,\\
    \text{undefiniert},& \text{wenn }x_1=x_2=0.
  \end{cases}
\end{align*}
Ein Argument $x=(x_1,x_2)\in\Rbb^2$ von $g_P$ kann dann in Polarkoordinaten 
charakterisiert werden durch die Länge $r=\sqrt{x_1^2+x_2^2}$ und den Winkel
$\varphi = \atan(x_2,x_1)$.
Mit dieser Notation gelten für die partiellen Ableitungen die Zusammenhänge
\begin{align*}
  \partial_{x_1} &= 
  \cos(\varphi)\partial_r - \frac{1}{r}\sin(\varphi)\partial_\varphi
  \quad\text{und }\\
  \partial_{x_2} &= 
  \sin(\varphi)\partial_r - \frac{1}{r}\cos(\varphi)\partial_\varphi.
\end{align*}
Ist nun $g_P$ vom Winkel $\varphi$ unabhängig, so erhalten wir
\begin{align*}
  \nabla g_P 
  = 
  \begin{pmatrix}
    \cos(\varphi)\\
    \sin(\varphi)
  \end{pmatrix}
  \partial_r g_P.
\end{align*}
Unter Beachtung der trigonometrischen Zusammenhänge
\begin{align*}
  \sin\big(\arctan(y)\big) &= \frac{y}{\sqrt{1+y^2}} &&\text{und}
  &\cos\big(\arctan(y)\big) &= \frac{1}{\sqrt{1+y^2}}
\end{align*}
ergibt sich 
\begin{align*}
  \begin{pmatrix}
    \cos(\varphi)\\
    \sin(\varphi)
  \end{pmatrix}
  = 
  \frac{1}{r}
  \begin{pmatrix}
    x_1\\
    x_2
  \end{pmatrix}
\end{align*}
und somit 
\begin{align*}
  \nabla g_P
  = 
  \frac{\partial_r g_P}{r}
  \begin{pmatrix}
    x_1\\
    x_2
  \end{pmatrix}.
\end{align*} 
Zum Bestimmen des Gradienten in kartesischen Koordianten einer
in Polarkoordinaten gegegebenen Funktion $g_P$, die
vom Polarwinkel unabhängig ist, muss also lediglich 
die partielle Ableitung $\partial_r g_P$ berechnet werden.
Konkrete Beispiele formulieren wir in \Cref{chap:experiments}.


\chapter{Das diskrete Problem}
\label{chap:discreteProblem}
\section{Formulierung}
\label{sec:discreteProblemFormulation}
Bevor wir \Cref{prob:continuousProblem} diskretisieren, merken wir an,
dass $\CR^1(\Tcal)\subset\BV(\Omega)$, da
\begin{align*}
  |\vcr|_{\BV(\Omega)} 
  = 
  \Vert \gradnc \vcr\Vert_{L^1(\Omega)} 
  + \sum_{F\in\Ecal(\Omega)}\Vert[\vcr]_F\Vert_{L^1(F)}
  \quad\text{für alle }\vcr\in\CR^1(\Tcal).
\end{align*} 
Dies wird für $|\Tcal|=2$ zum Beispiel von \cites[S. 404, Example
10.2.1]{ABM14}[S. 301, Proposition 10.1]{Bar15} impliziert und kann
analog für beliebige reguläre Triangulierungen von $\Omega$ bewiesen
werden.
Damit gilt für alle $\vcr\in\CR^1(\Tcal)$ insbesondere
\begin{align*}
  |\vcr|_{\BV(\Omega)} +\Vert\vcr\Vert_{L^1(\partial\Omega)} 
  = \Vert \gradnc \vcr\Vert_{L^1(\Omega)} +
  \sum_{F\in\Ecal}\Vert[\vcr]_F\Vert_{L^1(F)}.
\end{align*}
Um eine nichtkonforme Formulierung von \Cref{prob:continuousProblem} zu 
erhalten, ersetzen wir die Terme 
$|\bullet|_{\BV(\Omega)} +\Vert\bullet\Vert_{L^1(\partial\Omega)}$ des
Funktionals $E$ durch 
$\Vert \gradnc \bullet\Vert_{L^1(\Omega)}$, das heißt, wir vernachlässigen
bei der nichtkonformen Formulierung die Terme
$\sum_{F\in\Ecal}\Vert[\bullet]_F\Vert_{L^1(F)}$.
Somit erhalten wir das folgende Minimierungsproblem für den Parameter
$\alpha\in\Rbb_+$ und das Eingangssignal $f\in L^2(\Omega)$.

\begin{problem}\label{prob:discreteProblem}
  Finde $\ucr\in \CR^1_0(\Tcal)$,
  sodass $\ucr$ das Funktional
  \begin{align}\label{eq:discreteProblem}
    \Enc(\vcr)\coloneqq \frac{\alpha}{2}\Vert \vcr\Vert^2
    +\Vert \gradnc\vcr\Vert_{L^1(\Omega)}-\int_\Omega f\vcr\dx
  \end{align}
  unter allen $\vcr\in \CR^1_0(\Tcal)$ minimiert.
\end{problem}

\section{Charakterisierung und Existenz eines eindeutigen Minimierers}

In diesem Abschnitt führen wir die Argumente in \cite[S. 313]{Bar15}, angepasst
für unsere Formulierung in \Cref{prob:discreteProblem}, detailliert aus. 
Zunächst zeigen wir, dass \Cref{prob:discreteProblem} eine eindeutige Lösung
besitzt. Dafür benötigen wir folgendes Lemma.
\begin{lemma}
  \label{lem:normOfGradNcContiuous}
  Das Funktional $\Enc$ aus \Cref{eq:discreteProblem} ist stetig bezüglich der
  Konvergenz in $L^2(\Omega)$.
\end{lemma}

\begin{proof}
  Die Folge $(v_n)_{n\in\Nbb}\subset\CR^1_0(\Tcal)$ konvergiere
  gegen $\vcr\in\CR^1_0(\Tcal)$ bezüglich der Norm $\Vert\bullet\Vert$.
  Damit ist $(v_n)_{n\in\Nbb}$ insbesondere beschränkt in $L^2(\Omega)$ und es
  gilt mit einer binomischen Formel und der umgekehrten Dreiecksungleichung,
  dass
  \begin{align*}
    \left|\Vert\vcr\Vert^2-\Vert v_n\Vert^2\right|
    &=
    \big|\Vert\vcr\Vert-\Vert v_k\Vert\big|\, 
    \big|\Vert \vcr\Vert+\Vert v_k\Vert\big|\\
    &\leq
    \Vert\vcr- v_k\Vert\, \big|\Vert \vcr\Vert+\Vert v_k\Vert\big|
    \to 0\quad\text{für }n\to\infty.
  \end{align*}
  Außerdem gilt mit der Hölderschen Ungleichung
  \begin{align*}
    \left|\int_\Omega f(\vcr-v_k)\dx\right|
    \leq \Vert f\Vert \Vert\vcr-v_k\Vert\to 0\quad\text{für }n\to\infty.
  \end{align*}
  Schließlich gilt für alle $n\in\Nbb$ und alle $T\in\Tcal$ mit der 
  inversen Ungleichung (cf. \cite[S. 53, Lemma 3.5]{Bar15})
  mit Konstante $c_T\in\Rbb_+$ und der Hölderschen Ungleichung, dass
  \begin{equation*}
    \label{eq:continuityProofTriangleWiseEstimate}
    \Vert\gradnc(\vcr- v_n)\Vert_{L^1(T)}
    \leq
    c_T h_T^{-1}\Vert\vcr- v_n\Vert_{L^1(T)}
    \leq
    c_T h_T^{-1}\sqrt{|T|}\Vert\vcr- v_n\Vert_{L^2(T)}.
  \end{equation*}
  Damit folgt zusammen mit der umgekehrten Dreiecksungleichung
  \begin{align*}
    \left|\Vert\gradnc\vcr\Vert_{L^1(\Omega)}-
    \Vert \gradnc v_n\Vert_{L^1(\Omega)}\right|
    &\leq 
    \Vert\gradnc(\vcr- v_n)\Vert_{L^1(\Omega)}\\
    &=
    \sum_{T\in\Tcal}\Vert\gradnc(\vcr- v_n)\Vert_{L^1(T)}\\
    &\leq
    \max_{T\in\Tcal}\left(c_T h_T^{-1}\sqrt{|T|}\right)
    \sum_{T\in\Tcal}\Vert\vcr- v_n\Vert_{L^2(T)}\\
    &=
    \max_{T\in\Tcal}\left(c_T h_T^{-1}\sqrt{|T|}\right) \Vert\vcr- v_n\Vert
    \to 0\quad\text{für }n\to\infty.
  \end{align*}
  Somit ist $\Enc$ Summe von drei Termen, die bezüglich der Norm
  $\Vert\bullet\Vert$ folgenstetig sind, und deshalb stetig bezüglich der
  Konvergenz in $L^2(\Omega)$.
\end{proof}

\begin{theorem}
  \label{thm:discreteProblemExistenceUniqueness}
  Es existiert eine eindeutige Lösung $\ucr\in\CR^1_0(\Tcal)$ von
  \Cref{prob:discreteProblem}.
\end{theorem}

\begin{proof}
  Mit analogen Abschätzungen wie in \eqref{eq:contProbBddFromBelow}
  erhalten wir für das Funktional $\Enc$ aus \Cref{prob:discreteProblem} 
  für alle $\vcr\in\CR^1_0(\Tcal)\subset L^2(\Omega)$ die Ungleichung 
  \begin{equation}
    \label{eq:discreteEnergyCoercivity}
    \Enc(\vcr) 
    \geq 
    \frac{\alpha}{4}\Vert \vcr\Vert^2
    +\Vert \gradnc\vcr\Vert_{L^1(\Omega)}
    -\frac{1}{\alpha}\Vert f\Vert^2
    \geq 
    -\frac{1}{\alpha}\Vert f\Vert^2.
  \end{equation}
  Somit ist $\Enc$ nach unten beschränkt und es existiert eine infimierende
  Folge $(v_n)_{n\in\Nbb} \subset \CR^1_0(\Tcal)$ von $\Enc$. 
  Ungleichung \eqref{eq:discreteEnergyCoercivity} impliziert weiterhin, dass
  diese Folge beschränkt bezüglich der Norm $\Vert\bullet\Vert$ sein muss.
  Der endlichdimensionale Raum $\CR^1_0(\Tcal)$ ist, ausgestattet mit der Norm
  $\Vert\bullet\Vert$, ein Banachraum und damit reflexiv. 
  Demnach existiert eine in $\CR^1_0(\Tcal)$ schwach konvergente Teilfolge von
  $(v_n)_{n\in\Nbb}$.
  Da $\CR^1_0(\Tcal)$ endlichdimensional ist, konvergiert diese sogar stark
  in $L^2(\Omega)$. 
  Weil $\CR^1_0(\Tcal)$ ein Banachraum und damit abgeschlossen bezüglich der
  Konvergenz in $\Vert\bullet\Vert$ ist, gilt für den Grenzwert $\ucr$ dieser
  Teilfolge, dass $\ucr\in\CR^1_0(\Tcal)$.
  Nach \Cref{lem:normOfGradNcContiuous} ist $\Enc$ stetig bezüglich der
  Konvergenz in $L^2(\Omega)$, was impliziert, dass $\ucr$ Minimierer von
  $\Enc$ in $\CR^1_0(\Tcal)$ sein muss.   
  Dieser Minimierer $\ucr$ ist eindeutig, da $\Enc$ strikt konvex ist.
\end{proof}

Als Nächstes wollen wir äquivalente Charakterisierungen der eindeutigen Lösung
von \Cref{prob:discreteProblem} beweisen, die von Professor Carstensen
formuliert wurden.
Dazu leiten wir zunächst ein zu \Cref{prob:discreteProblem} äquivalentes
Minimaxproblem nach \cite[Section 36]{Roc70} her.
Wir betrachten die konvexe Menge 
\begin{align*}
  K
  \coloneqq 
  \left\{\Lambda\in L^\infty\!\left(\Omega;\Rbb^2\right)
  \,\middle|\,|\Lambda(\bullet)| \leq 1 \text{ fast überall in }\Omega\right\}
\end{align*}
und das dazugehörige Indikatorfunktional
$I_K:L^\infty\!\left(\Omega;\Rbb^2\right)\to\Rbb\cup\{\infty\}$, das für
$\Lambda\in L^\infty\!\left(\Omega;\Rbb^2\right)$ gegeben ist durch
\begin{align*}
  I_K(\Lambda)
  &\coloneqq
  \begin{cases}
    \infty, & \text{falls } \Lambda\notin K,\\
    0,       & \text{falls } \Lambda\in K.
  \end{cases}
\end{align*} 
Aufgrund der Konvexität von $K$ ist $I_K$ konvex.
Für $\vcr\in\CR^1_0(\Tcal)$ und $\Lambda_0\in
P_0\!\left(\Tcal;\Rbb^2\right)\subset L^\infty\!\left(\Omega;\Rbb^2\right)$
können wir damit die Sattelfunktion $L:\CR^1_0(\Tcal)\times
P_0\!\left(\Tcal;\Rbb^2\right)\to [-\infty,\infty)$ nach \cite[Section
33]{Roc70} definieren durch
\begin{align}\label{eq:discreteProblemLagrangeFunctional}
  L(\vcr,\Lambda_0) \coloneqq \int_\Omega\Lambda_0\cdot\gradnc\vcr\dx +
  \frac{\alpha}{2}\Vert \vcr\Vert^2 -\int_\Omega f\vcr\dx
  - I_K(\Lambda_0).
\end{align}
Nun wählen wir $\vcr\in\CR^1_0(\Tcal)$ beliebig. 
Mit der Cauchy-Schwarzschen Ungleichung gilt für alle
$\Lambda_0\in P_0\!\left(\Tcal;\Rbb^2\right)\cap K$, dass
\begin{align*}
  \int_\Omega \Lambda_0\cdot\gradnc\vcr\dx
  \leq 
  \int_\Omega |\Lambda_0||\gradnc\vcr|\dx
  \leq 
  \Vert\gradnc\vcr\Vert_{L^1(\Omega)}.
\end{align*}
Daraus folgt
\begin{align}
  \label{eq:saddlepointLeqEnergy}
  \sup_{\Lambda_0\in P_0\left(\Tcal;\Rbb^2\right)\cap K}L(\vcr,\Lambda_0)
  \leq \Enc(\vcr).
\end{align}
Weiterhin gilt, wenn wir $\Lambda_0\in P_0\!\left(\Tcal;\Rbb^2\right)\cap K$
mit der Signumfunktion aus \Cref{eq:signumFunction} elementweise auf allen
$T\in\Tcal$ definieren durch $\Lambda_0(x)\in\sign\left(\gradnc\vcr(x)\right)$
für alle $x\in \interior(T)$, dass $L(\vcr,\Lambda_0)=\Enc(\vcr)$ und deshalb
auch
\begin{align}
  \label{eq:saddlepointGeqEnergy}
  \Enc(\vcr)
  \leq
  \sup_{\Lambda_0\in P_0\left(\Tcal;\Rbb^2\right)\cap K}L(\vcr,\Lambda_0).
\end{align}
Außerdem ist $L(\vcr,\Lambda_0)>-\infty$ genau dann, wenn $\Lambda_0\in K$.
Damit folgt aus den Ungleichungen \eqref{eq:saddlepointLeqEnergy} und
\eqref{eq:saddlepointGeqEnergy} insgesamt
\begin{equation*}
  \label{eq:discreteEnergySaddlefunctionalEquality}
  \Enc(\vcr)
  =\sup_{\Lambda_0\in P_0\left(\Tcal;\Rbb^2\right)\cap K}L(\vcr,\Lambda_0)
  =\sup_{\Lambda_0\in P_0\left(\Tcal;\Rbb^2\right)}L(\vcr,\Lambda_0).
\end{equation*}
Wenn also das folgende Minimaxproblem
\ref{prob:discreteSaddlepointProblem} eine Lösung $\left(
\tilde{u}_\CR,\bar\Lambda_0 \right)\in\CR^1_0(\Tcal)\times
P_0\!\left(\Tcal;\Rbb^2\right)$ hat, dann löst die Funktion $\tilde{u}_\CR$
\Cref{prob:discreteProblem}.

\begin{problem}\label{prob:discreteSaddlepointProblem}
  Finde $\left( \tilde{u}_\CR,\bar\Lambda_0 \right)\in\CR^1_0(\Tcal)\times
  P_0\!\left(\Tcal;\Rbb^2\right)$,
  sodass
  \begin{align*}
    L(\tilde{u}_\CR,\bar\Lambda_0) 
    = 
    \inf_{\vcr\in\CR^1_0(\Tcal)}\sup_{\Lambda_0\in P_0\left(\Tcal;\Rbb^2\right)}
    L(\vcr,\Lambda_0).
  \end{align*}
\end{problem}

\begin{lemma}
  \label{lem:existenceSaddlepoint}
  Es existiert eine Lösung $\left( \tilde{u}_\CR,\bar\Lambda_0
  \right)\in\CR^1_0(\Tcal)\times \left(P_0\!\left(\Tcal;\Rbb^2\right)\cap
  K\right)$ von \Cref{prob:discreteSaddlepointProblem}.
\end{lemma}

\begin{proof}
  Die Sattelfunktion $L$ aus \Cref{eq:discreteProblemLagrangeFunctional} ist,
  wenn ihre zweite Komponente in $P_0\!\left(\Tcal;\Rbb^2\right)\cap K$ fixiert
  ist, in ihrer ersten Komponente eine konvexe, unterhalbstetige, auf
  $\CR^1_0(\Tcal)$ reellwertige Funktion und in ihrer zweiten Komponente eine
  konkave, oberhalbstetige, auf $P_0\!\left(\Tcal;\Rbb^2\right)\cap K$
  reellwertige Funktion.
  Somit ist $L$ in beiden Komponenten abgeschlossen nach \cite[S. 52,
  308]{Roc70}.  
  Insgesamt ist $L$ damit eine konvex-konkave, propere und abgeschlossene
  Funktion nach \cite[S. 349, 362 f.]{Roc70}, deren effektiver
  Definitionsbereich nach \cite[362]{Roc70} die Menge $\CR^1_0(\Tcal)\times
  \left( P_0\!\left(\Tcal;\Rbb^2\right)\cap K\right)$ ist.
  Unter Beachtung der Isomorphie von $\CR^1_0(\Tcal)$ zu
  $\Rbb^{|\Ecal(\Omega)|}$ und der Isomorphie von
  $P_0\!\left(\Tcal;\Rbb^2\right)$ zu $\Rbb^{2|\Tcal|}$, folgt aus 
  \cite[S. 397, Theorem 37.6]{Roc70} die Existenz eines Sattelpunkts
  $\left(\tilde{u}_\CR,\bar\Lambda_0\right)\in \CR^1_0(\Tcal)\times \left(
  P_0\!\left(\Tcal;\Rbb^2\right)\cap K\right)$ von $L$ nach \cite[380]{Roc70}.
  Für diesen impliziert \cite[S. 380, Lemma 36.2]{Roc70}, dass 
  \begin{align*}
    \sup_{\Lambda_0\in P_0\left(\Tcal;\Rbb^2\right)}\inf_{\vcr\in\CR^1_0(\Tcal)}
    L(\vcr,\Lambda_0)
    =
    L(\tilde{u}_\CR,\bar\Lambda_0) 
    = 
    \inf_{\vcr\in\CR^1_0(\Tcal)}\sup_{\Lambda_0\in P_0\left(\Tcal;\Rbb^2\right)}
    L(\vcr,\Lambda_0).
  \end{align*}
  Somit löst $\left(\tilde{u}_\CR,\bar\Lambda_0\right)\in \CR^1_0(\Tcal)\times
  \left( P_0\!\left(\Tcal;\Rbb^2\right)\cap K\right)$
  \Cref{prob:discreteSaddlepointProblem}.
\end{proof}

Nachdem diese Vorbereitungen abgeschlossen sind, können wir nun folgendes
Theorem beweisen.

\begin{theorem}
  \label{thm:discProbCharacterizationOfDiscreteSolutions}
  Für eine Funktion $\tilde{u}_\CR\in\CR^1_0(\Tcal)$ sind die folgenden drei
  Aussagen äquivalent.
  \begin{itemize}
    \item[(i)] \Cref{prob:discreteProblem} wird von $\tilde{u}_\CR$ gelöst.
    \item[(ii)] Es existiert ein
      $\bar\Lambda_0\in P_0\!\left(\Tcal;\Rbb^2\right)$ mit
      $\left|\bar\Lambda_0(\bullet)\right|\leq 1$
      fast überall in $\Omega$, sodass
      \begin{equation}
        \label{eq:discreteMultiplierScalerProductEquality}
        \bar\Lambda_0(\bullet)\cdot\gradnc\tilde{u}_\CR(\bullet)
        =
        \left|\gradnc\tilde{u}_\CR(\bullet)\right| 
        \quad\text{fast überall in } \Omega 
      \end{equation}
      und
      \begin{equation}
        \label{eq:discreteMultiplierL2Equality}
        \left(\bar\Lambda_0,\gradnc\vcr\right)
        = \left(f-\alpha\tilde{u}_\CR,
        \vcr\right)
        \quad\text{für alle } \vcr\in\CR^1_0(\Tcal).
      \end{equation}
    \item[(iii)] Für alle $\vcr\in\CR^1_0(\Tcal)$ gilt
      \begin{equation}
        \label{eq:discreteVariationalInequality}
        \left(f-\alpha\tilde{u}_\CR,\vcr-\tilde{u}_\CR\right)\leq
        \Vert\gradnc\vcr\Vert_{L^1(\Omega)} -
        \left\Vert\gradnc\tilde{u}_\CR\right\Vert_{L^1(\Omega)}\!.
      \end{equation}
  \end{itemize}
\end{theorem}

\begin{proof} 
  Sei $\tilde{u}_\CR\in\CR^1_0(\Tcal)$.

  \textit{(i) $\Rightarrow$ (ii).}
  Sei $\tilde{u}_\CR$ Lösung von \Cref{prob:discreteProblem}.
  Nach \Cref{lem:existenceSaddlepoint} existiert eine Lösung
  $\left(\hat{u}_\CR,\bar\Lambda_0\right)\in \CR^1_0(\Tcal)\times \left(
  P_0\!\left(\Tcal;\Rbb^2\right)\cap K\right)$ 
  von \Cref{prob:discreteSaddlepointProblem}. 
  Außerdem wissen wir, dass damit $\hat{u}_\CR$ Lösung von
  \Cref{prob:discreteProblem} ist.
  Daraus folgt, da nach \Cref{thm:discreteProblemExistenceUniqueness} die
  Lösung von \Cref{prob:discreteProblem} eindeutig ist, dass
  $\hat{u}_\CR=\tilde{u}_\CR$ in $\CR^1_0(\Tcal)$.
  Weiterhin wissen wir aus dem Beweis von \Cref{lem:existenceSaddlepoint}, dass
  $\left(\tilde{u}_\CR,\bar\Lambda_0\right)$ Sattelpunkt der Funktion $L$ aus
  \Cref{eq:discreteProblemLagrangeFunctional} ist.
  Das bedeutet nach \cite[380]{Roc70} insbesondere, dass $\tilde{u}_\CR$
  Minimierer von $L(\bullet, \bar\Lambda_0)$ in $\CR^1_0(\Tcal)$ und
  $\bar\Lambda_0$ Maximierer von $L\!\left(\tilde{u}_\CR,\bullet\right)$ in 
  $P_0\!\left(\Tcal;\Rbb^2\right)$ ist.  
  Mit dieser Erkenntnis können wir nun die entsprechenden
  Optimalitätsbedingungen diskutieren.
  Zunächst gilt, da
  $L\!\left(\tilde{u}_\CR,\bullet\right):P_0\!\left(\Tcal;\Rbb^2\right)\to
  [-\infty,\infty)$ konkav und
  $\bar\Lambda_0$ Maximierer von $L\!\left(\tilde{u}_\CR,\bullet\right)$ in 
  $ P_0\!\left(\Tcal;\Rbb^2\right)$ ist, dass das konvexe Funktional
  $-L(\tilde{u}_\CR,\bullet):P_0\!\left(\Tcal;\Rbb^2\right)\to
  (-\infty,\infty]$ von $\bar\Lambda_0$ in $ P_0\!\left(\Tcal;\Rbb^2\right)$
  minimiert wird.
  %Nach den Theoremen \ref{thm:extremalprinciple},
  %\ref{thm:subdifferentialSumRule} und \ref{thm:subdiffGateaux} gilt somit
  Nach den Theoremen \ref{thm:extremalprinciple} --
  \ref{thm:subdifferentialSumRule} gilt somit
  \begin{align*}
    0
    \in 
    \partial \left(-L\!\left(\tilde{u}_\CR,\bullet\right)\right)
    \left(\bar\Lambda_0\right) 
    =
    \left\{-\!\left(\gradnc\tilde{u}_\CR,\bullet\right)\right\}+\partial I_K
    \left(\bar\Lambda_0\right)\!.
  \end{align*}
  Äquivalent zu dieser Aussage ist, dass
  $\left(\gradnc\tilde{u}_\CR,\bullet\right)\in \partial
  I_K \left(\bar\Lambda_0\right)$. 
  Da $\bar\Lambda_0\in K$, folgt mit \Cref{def:subdifferential},
  dass für alle $\Lambda_0\in  P_0\!\left(\Tcal;\Rbb^2\right)$ gilt
  \begin{align*}
    \left(\gradnc\tilde{u}_\CR,\Lambda_0-\bar\Lambda_0\right) 
    \leq 
    I_K (\Lambda_0) - I_K\!\left(\bar\Lambda_0\right)
    =
    I_K (\Lambda_0).
  \end{align*}
  Falls $\Lambda_0\in  P_0\!\left(\Tcal;\Rbb^2\right)\cap K$, folgt insbesondere
  \begin{align}
    \label{eq:scalarProductInequDiscreteProof}
    \left(\gradnc\tilde{u}_\CR,\Lambda_0-\bar\Lambda_0\right) 
    &\leq 
    0,\quad\text{also }\notag\\ 
    \left(\gradnc\tilde{u}_\CR,\Lambda_0\right)
    &\leq
    \left(\gradnc\tilde{u}_\CR,\bar\Lambda_0\right)\!.
  \end{align}
  Sei nun $\Lambda_0\in P_0\!\left(\Tcal;\Rbb^2\right)\cap K$ elementweise auf
  allen $T\in\Tcal$ durch $\Lambda_0(x)\in\sign\left(\gradnc\tilde{u}_\CR(x)\right)$
  definiert für alle $x\in\interior(T)$.
  Mit dieser Wahl von $\Lambda_0$, Ungleichung
  \eqref{eq:scalarProductInequDiscreteProof}, der Cauchy\--Schwarz\-schen
  Ungleichung und $\bar\Lambda_0\in K$ erhalten wir die Abschätzung
  \begin{align}
    \label{eq:sumOverAllTrianglesDualVariable}
    \int_\Omega\left|\gradnc\tilde{u}_\CR\right|\dx
    &=
    \int_\Omega\gradnc\tilde{u}_\CR\cdot\Lambda_0\dx
    \leq 
    \int_\Omega\gradnc\tilde{u}_\CR\cdot\bar\Lambda_0\dx \notag\\
    &\leq 
    \int_\Omega\left|\gradnc\tilde{u}_\CR\right|\left|\bar\Lambda_0\right|\dx
    \leq
    \int_\Omega\left|\gradnc\tilde{u}_\CR\right|\dx,
    \quad\text{das heißt, }\notag\\
    \int_\Omega\left|\gradnc\tilde{u}_\CR\right|\dx 
    &= 
    \int_\Omega\gradnc\tilde{u}_\CR\cdot\bar\Lambda_0\dx
    \quad\text{beziehungsweise }\notag\\
    \sum_{T\in\Tcal}|T|\,\big|(\gradnc\tilde{u}_\CR)\!|_T\big|
    &=
    \sum_{T\in\Tcal}|T|\left(\gradnc\tilde{u}_\CR\cdot \bar\Lambda_0\right)\!\!|_T.
  \end{align}
  Außerdem gilt für alle $T\in\Tcal$ mit der Cauchy-Schwarzschen Ungleichung
  und $\bar\Lambda_0\in K$, dass 
  \begin{align*}
    \left(\gradnc\tilde{u}_\CR\cdot \bar\Lambda_0\right)\!\!|_T
  \leq
  \big|(\gradnc\tilde{u}_\CR)\!|_{T}\big|\,\left|\bar\Lambda_0|_T\right|
  \leq
  \big|(\gradnc\tilde{u}_\CR)\!|_{T}\big|.
  \end{align*}
  Mit \Cref{eq:sumOverAllTrianglesDualVariable} folgt daraus für alle
  $T\in\Tcal$, dass $\left(\gradnc\tilde{u}_\CR\cdot
  \bar\Lambda_0\right)\!\!|_T=\big|(\gradnc\tilde{u}_\CR)\!|_T\big|$, das
  heißt, fast überall in $\Omega$ gilt
  $\bar\Lambda_0(\bullet)\cdot\gradnc\tilde{u}_\CR(\bullet)
  =|\gradnc\tilde{u}_\CR(\bullet)|$. 
  Damit ist \Cref{eq:discreteMultiplierScalerProductEquality} gezeigt.
  Als Nächstes betrachten wir das reellwertige Funktional
  $L\left(\bullet,\bar\Lambda_0\right):\CR^1_0(\Tcal)\to\Rbb$.
  Es ist Fr\'echet-differenzierbar mit
  \begin{align*}
    dL\!\left(\bullet,\bar\Lambda_0\right)\!\left(\tilde{u}_\CR;\vcr\right)
    =
    \int_\Omega\bar\Lambda_0\cdot \gradnc\vcr\dx
    +\alpha\! \left(\tilde{u}_\CR,\vcr\right) - \int_\Omega f\vcr\dx
  \end{align*}
  für alle $\vcr\in\CR^1_0(\Tcal)$.
  Da $\tilde{u}_\CR$ Minimierer von  $L\!\left(\bullet, \bar\Lambda_0\right)$
  in $\CR^1_0(\Tcal)$ ist, gilt nach
  \Cref{thm:necessaryConditionFreeLocalExtrema}, dass $0 =
  dL\!\left(\bullet,\bar\Lambda_0\right)\!\left(\tilde{u}_\CR;\vcr\right)$ für
  alle $\vcr\in\CR^1_0(\Tcal)$.
  Diese Bedingung ist für alle $\vcr\in\CR^1_0(\Tcal)$ äquivalent zu
  $\left(\bar\Lambda_0,\gradnc\vcr\right) = (f-\alpha \tilde{u}_\CR,\vcr)$.
  Somit ist auch \Cref{eq:discreteMultiplierL2Equality} gezeigt.

  \textit{(ii) $\Rightarrow$ (iii).}
  Die Funktion $\bar\Lambda_0\in P_0\!\left(\Tcal;\Rbb^2\right)$ erfülle
  $\left|\bar\Lambda_0(\bullet)\right|\leq 1$ fast überall in $\Omega$ sowie
  die Gleichungen \eqref{eq:discreteMultiplierScalerProductEquality} und 
  \eqref{eq:discreteMultiplierL2Equality}. 
  Sei $\vcr\in\CR^1_0(\Tcal)$.
  Mit den Gleichungen 
  \eqref{eq:discreteMultiplierL2Equality} und 
  \eqref{eq:discreteMultiplierScalerProductEquality} gilt
  \begin{equation}
    \label{eq:equivalentCharacterizationApplicationTwoEquations}
    \begin{aligned}
      \left(f-\alpha\tilde{u}_\CR,\vcr-\tilde{u}_\CR\right) 
      &=
      \left(\bar\Lambda_0,\gradnc\vcr\right)
      - \left(\bar\Lambda_0,\gradnc\tilde{u}_\CR\right)\\
      &=
      \int_\Omega\bar\Lambda_0\cdot\gradnc\vcr\dx
      - \int_\Omega\left|\gradnc\tilde{u}_\CR\right|\dx.
    \end{aligned}
  \end{equation}
  Weiterhin gilt mit der Cauchy-Schwarzschen
  Ungleichung und $\left|\bar\Lambda_0(\bullet)\right|\leq 1$ fast überall in
  $\Omega$, dass
  \begin{align*}
    \int_\Omega\bar\Lambda_0\cdot\gradnc\vcr\dx
    &\leq 
    \int_\Omega\left|\bar\Lambda_0\right|\,|\gradnc\vcr|\dx
    \leq 
    \int_\Omega|\gradnc\vcr|\dx.
  \end{align*}
  Zusammen mit \Cref{eq:equivalentCharacterizationApplicationTwoEquations}
  folgt daraus Ungleichung \eqref{eq:discreteVariationalInequality}.

  \textit{(iii) $\Rightarrow$ (i)}.
  Es gelte Ungleichung \eqref{eq:discreteVariationalInequality} für alle
  $\vcr\in\CR^1_0(\Tcal)$, also
  \begin{align*}
    \left(f-\alpha\tilde{u}_\CR,\vcr-\tilde{u}_\CR\right) 
    \leq
    \left\Vert\gradnc\vcr\right\Vert_{L^1(\Omega)}
    -\left\Vert\gradnc\tilde{u}_\CR\right\Vert_{L^1(\Omega)}\!.
  \end{align*}
  Nach \Cref{thm:discreteProblemExistenceUniqueness} existiert eine eindeutige
  Lösung $\ucr\in\CR^1_0(\Tcal)$ von \Cref{prob:discreteProblem}.
  Wir haben bereits gezeigt, dass somit für alle
  $\vcr\in\CR^1_0(\Tcal)$ gilt
  \begin{align*}
    \left(f-\alpha\ucr,\vcr-\ucr\right) 
    \leq
    \left\Vert\gradnc\vcr\right\Vert_{L^1(\Omega)}
    -\Vert\gradnc\ucr\Vert_{L^1(\Omega)}.
  \end{align*}
  Um nun zu beweisen, dass $\tilde{u}_\CR$ \Cref{prob:discreteProblem} löst, genügt
  es $\tilde{u}_\CR=\ucr$ in $\CR^1_0(\Tcal)$ zu zeigen.
  Es gilt
  \begin{align*}
    \left(f-\alpha\ucr,\tilde{u}_\CR-\ucr\right) 
    &\leq
    \left\Vert\gradnc\tilde{u}_\CR\right\Vert_{L^1(\Omega)}
    -\Vert\gradnc\ucr\Vert_{L^1(\Omega)}\quad\text{und }\\
    \left(f-\alpha\tilde{u}_\CR,\ucr-\tilde{u}_\CR\right) 
    &\leq
    \left\Vert\gradnc\ucr\right\Vert_{L^1(\Omega)}
    -\Vert\gradnc\tilde{u}_\CR\Vert_{L^1(\Omega)}. 
  \end{align*}
  Die Addition dieser Ungleichungen
  impliziert
  \begin{align*}
    \alpha\left\Vert\tilde{u}_\CR-\ucr\right\Vert^2=
    \left(-\alpha\ucr,\tilde{u}_\CR-\ucr\right) 
    + \left(-\alpha\tilde{u}_\CR,\ucr-\tilde{u}_\CR\right) 
    \leq
    0.
  \end{align*}
  Da $\alpha>0$, folgt daraus $\left\Vert\tilde{u}_\CR-\ucr\right\Vert^2=0$,
  also $\tilde{u}_\CR=\ucr$ in $\CR^1_0(\Tcal)$.
\end{proof}

Zum Abschluss dieses Abschnitts wollen wir noch zwei Bemerkungen von Professor
Carstensen erwähnen und kurz deren Gültigkeit begründen.
Die erste Bemerkung ist eine äquivalente Charakterisierung der dualen Variable
$\bar\Lambda_0\in P_0\!\left(\Tcal;\Rbb^2\right)$ aus
\Cref{thm:discProbCharacterizationOfDiscreteSolutions} zur diskreten Lösung
$\ucr\in\CR^1_0(\Tcal)$ von \Cref{prob:discreteProblem}.

\begin{remark}
  Dass $\bar\Lambda_0\in P_0\!\left(\Tcal;\Rbb^2\right)$ fast überall in $\Omega$
  \Cref{eq:discreteMultiplierScalerProductEquality} und
  $|\bar\Lambda_0(\bullet)|\leq 1$ erfüllt, ist äquivalent zu der Bedingung
  $\bar\Lambda_0(x)\in\sign(\gradnc \ucr(x))$ für alle $x\in\interior(T)$ für
  alle $T\in\Tcal$.   
\end{remark}

\begin{proof}
  Dass die genannte Bedingung hinreichend ist, folgt direkt aus der Definition
  der Signumfunktion.
  Ihre Notwendigkeit folgt aus der folgenden Beobachtung.
  Da $\left|\bar\Lambda_0(\bullet)\right|\leq 1$ fast überall in $\Omega$, ist
  \Cref{eq:discreteMultiplierScalerProductEquality} eine Cauchy-Schwarzsche
  Ungleichung, bei der sogar Gleichheit gilt. 
  Dies ist genau dann der Fall, wenn $\bar\Lambda_0(\bullet)$ und
  $\gradnc\ucr(\bullet)$ fast überall in $\Omega$ linear abhängig sind.
\end{proof}
 
Daraus können wir folgern, unter welchen Umständen die duale Variable
$\bar\Lambda_0$ auf einem Dreieck $T\in\Tcal$ eindeutig bestimmt ist.

\begin{remark}
  Falls $\gradnc\ucr\neq 0$ auf $T\in\Tcal$, gilt nach Definition der
  Signumfunktion, dass $\bar\Lambda_0=\gradnc\ucr/|\gradnc\ucr|$ eindeutig
  bestimmt ist auf $T$.
  Im Allgemeinen ist $\bar\Lambda_0$ allerdings nicht eindeutig bestimmbar. 
  Betrachten wir zum Beispiel $f\equiv 0$ in \Cref{prob:discreteProblem} mit
  eindeutiger Lösung $\ucr\equiv 0$ fast überall in $\Omega$. 
  Dann erfüllt nach der diskreten Helmholtz-Zerlegung \cite[S. 193, Theorem
  3.32]{Car09b} die Wahl $\bar\Lambda_0\coloneqq \Curl v_\C$ für ein beliebiges
  $v_\C\in S^1(\Tcal)$ mit $|\Curl v_\C|\leq 1$ die Charakterisierung
  \textit{(ii)} aus \Cref{thm:discProbCharacterizationOfDiscreteSolutions}.
\end{remark}


\section{Verfeinerungsindikator und garantierte Energieschranken}

Professor Carstensen stellte für die numerischen Untersuchungen eine Aussage
über eine garantierte untere Energieschranke und einen Verfeinerungsindikator
zur adaptiven Netzverfeinerung zur Verfügung, die wir in diesem Abschnitt
aufführen wollen.

\begin{theorem}[Garantierte untere Energieschranke]
  \label{thm:gleb}
  Sei $\Omega$ konvex, $f\in H^1_0(\Omega)$ das Eingangssignal für
  \Cref{prob:continuousProblem} mit Lösung $u\in H^1_0(\Omega)$ sowie für
  \Cref{prob:discreteProblem} mit Lösung $\ucr\in \CR^1_0(\Tcal)$.
  Dann gilt
  \begin{align*}
    \Enc(\ucr)+\frac{\alpha}{2}\Vert u-\ucr\Vert^2
    -\frac{\kappa_\CR}{\alpha}\Vert
    h_\Tcal(f-\alpha\ucr)\Vert \Vert\nabla f\Vert\leq E(u).
  \end{align*}
  Dabei ist die Konstante $\kappa_\CR\coloneqq\sqrt{1/48+1/j_{1,1}^2}$ mit der
  kleinsten positiven Nullstelle $j_{1,1}$ der Bessel-Funktion erster Art.
  Insbesondere gilt dann für 
  \begin{align}
    \label{eq:gleb}
    \Egleb 
    \coloneqq 
    \Enc(\ucr) - \frac{\kappa_\CR}{\alpha}\Vert h_\Tcal(f-\alpha\ucr)\Vert
    \Vert \nabla f\Vert,
  \end{align}
    dass $\Enc(\ucr)\geq \Egleb$ und $E(u)\geq \Egleb$.
\end{theorem}

\begin{definition}[Verfeinerungsindikator]
  \label{def:refinementIndicator}
  Für $d\in\mathbb{N}$ (in dieser Arbeit stets $d=2$) und $0<\gamma\leq 1$
  definieren wir für alle $T\in\Tcal$ und $\ucr\in\CR^1_0(\Tcal)$ die
  Funktionen
  \begin{align*}
    \eta_{\textup{V}, \Tcal}(T)
    &\coloneqq
    |T|^{2/d}\Vert f-\alpha \ucr\Vert^2_{L^2(T)}\quad\text{und }\\
    \eta_{\textup{J}, \Tcal}(T)
    &\coloneqq
    |T|^{\gamma/d}\sum_{F\in\Ecal(T)}\left\Vert [\ucr]_F\right\Vert_{L^1(F)}\!.
  \end{align*} 
  Damit definieren wir den Verfeinerungsindikator
  $\eta_\Tcal\coloneqq\sum_{T\in\Tcal}\eta_\Tcal(T)$, wobei
  \begin{align} \label{eq:refinementIndicator} 
    \eta_\Tcal (T)
    \coloneqq
    \eta_{\textup{V}, \Tcal}(T) + 
    \eta_{\textup{J}, \Tcal}(T)\quad\text{für alle } T\in\Tcal.
  \end{align} 
\end{definition}

Darüber hinaus können wir eine garantierte obere Energieschranke formulieren.
Dabei nutzen wir den Operator $J_{1,\Tcal}:\CR^1(\Tcal)\to P_1(\Tcal)\cap
C_0(\Omega)$ (cf.\ \cite[Section 4]{CH18}), wobei $J_{1, \Tcal}\vcr$ für eine
Funktion $\vcr\in\CR^1(\Tcal)$ in allen Innenknoten $z\in\Ncal(\Omega)$
definiert ist durch
\begin{align}
  \label{eq:enrichmentOperator}
  J_{1,  \Tcal}\vcr(z)\coloneqq |\Tcal(z)|^{-1}\sum_{T\in\Tcal(z)}\vcr|_T(z).
\end{align}
%Dieser erfüllt nach \cite[Section 4]{CH18} für eine vom Innenwinkel (?)
%abhängige Konstante $c>0$, dass
%\begin{align*}
%  \Vert h_\Tcal^{-1}(1-J_1)\vcr\Vert\leq c\Vert\gradnc \vcr\Vert.
%\end{align*}
Da für die Lösung $u$ von \Cref{prob:continuousProblem} und die Lösung
$\ucr$ von \Cref{prob:discreteProblem} gilt 
\begin{align}
  \label{eq:gueb}
  E(u)\leq E(J_{1, \Tcal}\ucr)=\Enc\left(J_{1, \Tcal}\ucr\right)
  \eqqcolon\Egueb,
\end{align}
wählen wir $\Egueb$ als garantierte obere Energieschranke.


\chapter{Iterative Lösung}
\label{chap:algorithm}
\section{Primale-duale Iteration}

In diesem Abschnitt präsentieren wir ein iteratives Verfahren mit dem wir
\Cref{prob:discreteProblem} numerisch lösen möchten. 
Dieses basiert auf der primalen-dualen Iteration \cite[S. 314, Algorithm
10.1]{Bar15} unter Beachtung von \cite[S. 314, Remark 10.11]{Bar15}. 
%Diese realisiert Gradientenverfahren zum Finden eines Sattelpunkts, dessen 
%Komponenten die primale und die duale Formulierung des Minimierungsproblems
%lösen. 
Details dazu und weitere Referenzen finden sich in \cites{Bar12}[S.
118-121]{Bar15}.
Angepasst an unser Problem und die Notation dieser Arbeit lautet der
Algorithmus wie folgt.

\begin{algorithm}[Primale-duale Iteration]
  \label{alg:primalDualIteration}
\begin{algorithmic}\\
  \Require $\left(u_0,\Lambda_0\right)
  \in\textup{CR}_0^1(\mathcal{T})\times P_0\!\left(\mathcal{T}; 
  %\left\{w\in\Rbb^2\,\middle|\,|w|\leq 1\right\}\right),
  \overline{B_{\Rbb^2}}\right),
  %\overline{B_1(0)}\right),
  \tau>0$  \\
  Initialisiere $v_0\coloneqq 0$ in $\textup{CR}^1_0(\mathcal T)$.
  \For{$j = 1,2,\dots$}
  \begin{align}
    %\label{eq:primalDualAlgUj}
    \tilde{u}_j&\coloneqq u_{j-1}+\tau v_{j-1},\nonumber\\
    \label{eq:primalDualAlgLambdaJ}
    \Lambda_j
    &\coloneqq
    \frac{\Lambda_{j-1}+\tau\nabla_{\textup{NC}} \tilde{u}_j}
    {\max\left\{1,
    \left|\Lambda_{j-1}+\tau\nabla_{\textup{NC}}\tilde{u}_j\right|\right\}},\\
    \text{bestimme }u_j\in\textup{CR}^1_0(\mathcal{T})
    \text{ a}&\text{ls Lösung des linearen Gleichungssystems }\nonumber\\
    \label{eq:linSysPrimalDualAlg}
    \frac{1}{\tau}a_{\textup{NC}}(u_j,\bullet)+\alpha(u_j,\bullet)
    &=
    \frac{1}{\tau}a_{\textup{NC}}(u_{j-1},\bullet) + (f,\bullet)
    - \left(\Lambda_j,\nabla_{\textup{NC}}\bullet\right) 
    \text{ in }\CR^1_0(\Tcal),\\
    v_j &\coloneqq \frac{u_j-u_{j-1}}{\tau}.\nonumber
  \end{align}
  %\State bestimme $u_j\in\textup{CR}^1_0(\mathcal{T})$
  %als Lösung des linearen Gleichungssystems
  %\begin{align}
  %  \label{eq:linSysPrimalDualAlg}
  %  \frac{1}{\tau}a_{\textup{NC}}(u_j,\bullet)+\alpha(u_j,\bullet)
  %  &=
  %  \frac{1}{\tau}a_{\textup{NC}}(u_{j-1},\bullet) + (f,\bullet)
  %  - \left(\Lambda_j,\nabla_{\textup{NC}}\bullet\right) 
  %  \quad\text{in }\CR^1_0(\Tcal),
  %\end{align}
  %\begin{equation*}
  %  v_j\coloneqq\frac{u_j-u_{j-1}}{\tau}.
  %\end{equation*}
  \EndFor
  \Ensure Folge $(u_j,\Lambda_j)_{j\in\mathbb N}$ in
  $\CR^1_0(\mathcal{T})\times
   P_0\!\left(\mathcal{T};\overline{B_{\Rbb^2}}\right)$   
  \end{algorithmic}
\end{algorithm}

In \cite{Bar15} wird in \Cref{eq:linSysPrimalDualAlg} anstelle des diskreten
Skalarprodukts $\anc(\bullet,\bullet)$ ein diskretes Skalarprodukt
$(\bullet,\bullet)_{h,s}$, welches ungleich dem $L^2$-Skalarprodukt sein
kann, genutzt. Die Iteration wird ebenda abgebrochen, wenn die von
$(\bullet,\bullet)_{h,s}$ induzierte Norm des Terms $(u_j-u_{j-1})/\tau$
kleiner einer gegebenen Toleranz ist.
Dementsprechend nutzen wir mit $\epsstop > 0$ für
\Cref{alg:primalDualIteration} das Abbruchkriterium 
\begin{align}
  \label{eq:terminationCriterion}
  \left\vvvert \frac{u_j-u_{j-1}}{\tau}\right\vvvert_\NC<\epsstop.
\end{align}

\begin{remark} 
  \label{rem:primalDualMatrixEquations}
  Mit den kantenorientierten Crouzeix-Raviart-Basisfunktionen
  $\{\psi_E\,\mid\,E\in\Ecal\}$ aus \Cref{sec:crouzeixRaviartFunctions} können
  wir die Steifigkeitsmatrix $A\in\Rbb^{|\Ecal|\times|\Ecal|}$ und die
  Massenmatrix $M\in\Rbb^{|\Ecal|\times|\Ecal|}$ für alle
  $k,\ell\in\{1,2,\ldots,|\Ecal|\}$ definieren durch
  \begin{align*}
    A_{k\ell}\coloneqq \anc\!\left(\psi_{E_k},\psi_{E_\ell}\right)
    \quad\text{und}\quad
    M_{k\ell}\coloneqq \left(\psi_{E_k},\psi_{E_\ell}\right)\!.
  \end{align*}
  Außerdem definieren wir mit $u_{j-1}$ und $\Lambda_j$ aus
  \Cref{alg:primalDualIteration} den Vektor $b\in\Rbb^{|\Ecal|}$ für alle
  $k\in\Nbb$ durch
  \begin{align*}
    b_k\coloneqq 
    \left(\frac{1}{\tau}\gradnc u_{j-1}-\Lambda_j,\gradnc\psi_{E_k}\right)
    + \left( f,\psi_{E_k} \right)\!.
  \end{align*}
  Sei nun $\Ecal=\left\{E_1,E_2,\ldots,E_{|\Ecal|}\right\}$ und sei ohne
  Beschränkung der Allgemeinheit 
  $\Ecal(\Omega)\coloneqq\left\{E_1,E_2,\ldots,E_{|\Ecal(\Omega)|}\right\}$.  
  Dann ist $J\coloneqq\{|\Ecal(\Omega)|+1,|\Ecal(\Omega)|+2,\ldots,|\Ecal|\}$
  die Menge der Indizes der Randkanten in $\Ecal$.
  Damit definieren wir die Matrix $\bar
  A\in\Rbb^{|\Ecal(\Omega)|\times|\Ecal(\Omega)|}$, die durch Streichen der
  Zeilen und Spalten von $A$ mit den Indizes aus $J$ entsteht, die Matrix $\bar
  M\in\Rbb^{|\Ecal(\Omega)|\times|\Ecal(\Omega)|}$, die ebenso aus $M$
  hervorgeht und den Vektor $\bar b\in\Rbb^{|\Ecal(\Omega)|}$, der durch
  Streichen der Komponenten von $b$ mit Indizes in $J$ entsteht.
  Weiterhin sei $x\in\Rbb^{|\Ecal(\Omega)|}$, wobei $x_k$ für alle
  $k\in\{1,2\ldots,|\Ecal(\Omega)|\}$ der Koeffizient der Lösung 
  $u_j\in\CR^1_0(\Tcal)$ des
  Gleichungssystems \eqref{eq:linSysPrimalDualAlg} zur $k$-ten Basisfunktion
  von $\CR^1_0(\Tcal)$ sei, das heißt, es gelte
  \begin{align*}
    u_j=\sum_{k=1}^{|\Ecal(\Omega)|} x_k\psi_{E_k}.
  \end{align*}
  Da wir das Gleichungssystem \eqref{eq:linSysPrimalDualAlg} in
  $\CR^1_0(\Tcal)$ lösen, lässt sich somit $u_j$ durch Lösen einer
  Matrixgleichung nach $x$ bestimmen. 
  Diese lautet
  \begin{align}
    \label{eq:linSysPrimalDualAlgMatrixEq}
    \left(\frac{1}{\tau}\bar A+\alpha \bar M\right)x=\bar b.
  \end{align}
\end{remark}

\section{Konvergenz der Iteration}
In diesem Abschnitt beweisen wir die Konvergenz der Iterate von
\Cref{alg:primalDualIteration} gegen die Lösung von
\Cref{prob:discreteProblem}. 
Dabei bedienen wir uns unter anderem der äquivalenten Charakterisierungen aus
\Cref{thm:discProbCharacterizationOfDiscreteSolutions}.

\begin{theorem}
  \label{thm:convergenceIteration}
  Sei $\ucr\in \CR^1_0(\Tcal)$ Lösung von \Cref{prob:discreteProblem} und
  $\bar\Lambda_0\in P_0\!\left(\Tcal;\Rbb^2\right)$ erfülle
  $\left|\bar\Lambda_0(\bullet)\right|\leq 1$ fast überall in $\Omega$ sowie
  \Cref{eq:discreteMultiplierScalerProductEquality} und
  \Cref{eq:discreteMultiplierL2Equality} aus
  \Cref{thm:discProbCharacterizationOfDiscreteSolutions} mit
  $\tilde{u}_\CR=\ucr$.
  Falls $\tau \in (0, 1]$, dann konvergieren die Iterate $(u_j)_{j\in\Nbb}$ von
  \Cref{alg:primalDualIteration} in $L^2(\Omega)$ gegen $\ucr.$
\end{theorem}

\begin{proof}
  Der Beweis folgt einer Skizze von Professor Carstensen.
  
  Sei $j\in\Nbb$. 
  Seien weiterhin $u_0$, $\Lambda_0$ und $v_0$ sowie $\tilde{u}_j$,
  $\Lambda_j$, $u_j$ und $v_j$ definiert wie in \Cref{alg:primalDualIteration}.
  Außerdem definieren wir $\mu_j\coloneqq \max\{1,|\Lambda_{j-1}+\tau \gradnc
  \tilde{u}_j|\}$ und für alle $k\in\Nbb_0$ die Abkürzungen $e_k \coloneqq
  \ucr-u_k$, $E_k\coloneqq \bar\Lambda_0-\Lambda_k$.
  Dabei nutzen wir die Konvention $e_{-1}\coloneqq e_0$.
  Wir testen zunächst \eqref{eq:linSysPrimalDualAlg} mit $e_j$ und formen das
  Resultat um. 
  Damit erhalten wir
  \begin{align*}
    \anc(v_j,e_j) + \alpha(u_j,e_j) 
    + (\Lambda_j,\gradnc e_j)
    = 
    (f,e_j).
  \end{align*}
  Zusammen mit \Cref{eq:discreteMultiplierL2Equality} folgt daraus
  \begin{equation}
    \label{eq:convProofE}
    \begin{aligned}
      \anc(v_j,e_j) &= 
      \alpha(\ucr-u_j,e_j) 
      + \left(\bar\Lambda_0-\Lambda_j,\gradnc e_j\right) 
      = 
      \alpha\Vert e_j\Vert^2 + \left(E_j,\gradnc e_j\right).
    \end{aligned}
  \end{equation}
  Als Nächstes betrachten wir \Cref{eq:primalDualAlgLambdaJ}. Es gilt
  \begin{align}
    \label{eq:convProofA}
    \Lambda_{j-1}-\Lambda_j+\tau \gradnc \tilde{u}_j 
    = (\mu_j-1)\Lambda_j \quad\text{fast überall in }\Omega.
  \end{align}
  Außerdem folgt aus \Cref{eq:primalDualAlgLambdaJ} und einer
  Fallunterscheidung zwischen $1\geq |\Lambda_{j-1}+\tau\gradnc \tilde{u}_j|$
  und $1< |\Lambda_{j-1}+\tau\gradnc \tilde{u}_j|$, dass
  \begin{align}
    \label{eq:convergenceIterationMuProductZero}
    \left(1-|\Lambda_j|\right)(\mu_j-1)=0
    \quad\text{fast überall in } \Omega.
  \end{align}
  Testen wir nun \Cref{eq:convProofA} in $L^2(\Omega)$ mit $E_j$, erhalten wir 
  unter Nutzung von $\mu_j\geq 1$, der Cauchy-Schwarzschen Ungleichung,
  $\left|\bar\Lambda_0(\bullet)\right|\leq 1$ fast überall in $\Omega$ und
  \Cref{eq:convergenceIterationMuProductZero}, dass
  \begin{align*}
    \left( \Lambda_{j-1}-\Lambda_j+\tau\gradnc \tilde{u}_j, E_j\right)
    &= 
    \left( (\mu_j-1)\Lambda_j,\bar\Lambda_0-\Lambda_j\right)\\
    &\leq
    \int_\Omega (\mu_j-1)\left(|\Lambda_j|-|\Lambda_j|^2\right)\dx\\
    &=
    \int_\Omega |\Lambda_j| (1-|\Lambda_j|)(\mu_j-1)\dx 
    =
    0.
  \end{align*}
  Daraus folgt mit $\Lambda_{j-1}-\Lambda_j = E_j-E_{j-1}$ und $\tilde{u}_j =
  u_{j-1}-(e_{j-1}-e_{j-2})$, dass nach Division durch $\tau$ gilt
  \begin{align}
    \label{eq:convProofB}
    \left(\frac{E_j-E_{j-1}}{\tau}+ \gradnc u_{j-1}-\gradnc
    (e_{j-1}-e_{j-2}),E_j\right)\leq 0.
  \end{align}
  Aus der Cauchy-Schwarzschen Ungleichung,
  \Cref{eq:discreteMultiplierScalerProductEquality} und
  $|\Lambda_j(\bullet)|\leq 1$ fast überall in $\Omega$ folgt, dass
  \begin{align*}
    \gradnc\ucr\cdot E_j 
    &=
    \gradnc\ucr\cdot\bar\Lambda_0 - \gradnc\ucr\cdot\Lambda_j\\
    &\geq 
    \gradnc\ucr\cdot\bar\Lambda_0 - |\gradnc\ucr||\Lambda_j| \\
    &= 
    |\gradnc\ucr|(1-|\Lambda_j|)
    \geq
    0\quad\text{fast überall in }\Omega.
  \end{align*}
  Daraus folgt
  \begin{align}
    \label{eq:convProofC}
    (\gradnc\ucr,E_j)=\int_\Omega \gradnc\ucr\cdot E_j\dx\geq 0.
  \end{align}
  Aus den Ungleichungen \eqref{eq:convProofB} und \eqref{eq:convProofC} folgt
  insgesamt
  \begin{align*}
    \left( \frac{E_j-E_{j-1}}{\tau}+ \gradnc u_{j-1}
    -\nabla_\nc(e_{j-1}-e_{j-2}),E_j\right)
    \leq
    (\gradnc\ucr,E_j).
  \end{align*}
  Das ist äquivalent zu
  \begin{align}
    \label{eq:convProofD}
    \left( \frac{E_j-E_{j-1}}{\tau} 
    -\gradnc(2e_{j-1}-e_{j-2}),E_j\right)\leq 0.
  \end{align}
  Weiterhin gilt
  \begin{equation}
    \begin{aligned}
      \label{eq:longVvvertFormula}
      &\vvvert e_j \vvvert^2_\nc   -
      \vvvert e_{j-1}\vvvert_\nc^2 +
      \Vert E_j \Vert^2 - \Vert E_{j-1}\Vert^2 +
      \vvvert e_j-e_{j-1}\vvvert_\nc^2 +
      \Vert E_j - E_{j-1} \Vert^2\\
      &=
      2a_\nc(e_j,e_j-e_{j-1}) + 2(E_j,E_j-E_{j-1}).
    \end{aligned}
  \end{equation}
  Unter Nutzung von $e_j-e_{j-1}=-\tau v_j$ und \Cref{eq:convProofE} 
  gilt außerdem
  \begin{align*}
    2a_\nc(e_j,e_j-e_{j-1}) + 2(E_j,E_j-E_{j-1})
    &=
    -2\tau a_\nc(e_j,v_j) + 2(E_j,E_j-E_{j-1})\\
    &=
    -2\tau\alpha\Vert e_j\Vert^2 + 2\tau\left(E_j,
    -\nabla_\nc e_j+\frac{E_j-E_{j-1}}{\tau}\right)\!.
  \end{align*}
  Daraus folgt durch Ungleichung \eqref{eq:convProofD} zusammen mit $\tau>0$,
  dass
  \begin{align*}
    &2a_\nc(e_j,e_j-e_{j-1}) + 2(E_j,E_j-E_{j-1})\\
    &\leq
    -2\tau\alpha\Vert e_j\Vert^2 + 2\tau\left(E_j,
    -\nabla_\nc e_j+\frac{E_j-E_{j-1}}{\tau}\right)\\
    &\quad\quad-2\tau\left( \frac{E_j-E_{j-1}}{\tau}
    -\gradnc(2e_{j-1}-e_{j-2}),E_j\right)\\
    &=
    -2\tau\alpha\Vert e_j\Vert^2 - 
    2\tau\big(E_j,\gradnc(e_j-2e_{j-1}+e_{j-2})\big).
  \end{align*}
  Damit und mit \Cref{eq:longVvvertFormula} erhalten wir insgesamt
  \begin{align*}
    &\vvvert e_j \vvvert^2_\nc   -
    \vvvert e_{j-1}\vvvert_\nc^2 +
    \Vert E_j \Vert^2 - \Vert E_{j-1}\Vert^2 +
    \vvvert e_j-e_{j-1}\vvvert_\nc^2 +
    \Vert E_j - E_{j-1} \Vert^2\\
    &\leq
    -2\tau\alpha\Vert e_j\Vert^2 - 
    2\tau\big(E_j,\gradnc(e_j-2e_{j-1}+e_{j-2})\big).
  \end{align*}
  Für jedes $J\in\Nbb$ führt die Summation dieser Ungleichung über
  $j=1,\ldots,J$ und eine Äquivalenzumfomung zu
  \begin{equation}
    \label{eq:convProofF}
    \begin{aligned}
      &\vvvert e_J \vvvert^2_\nc +\Vert E_J \Vert^2 
      +\sum_{j=1}^J\left(\vvvert e_j-e_{j-1} \vvvert_\nc^2 + 
      \Vert E_j-E_{j-1}\Vert^2\right)\\
      &\leq 
      \vvvert e_0 \vvvert_\nc^2 + \Vert E_0 \Vert^2 
      -2\tau\alpha\sum_{j=1}^J \Vert e_j\Vert^2 
      -2\tau \sum_{j=1}^J\big(E_j,\gradnc
      (e_j-2e_{j-1}+e_{j-2})\big).
    \end{aligned}
  \end{equation}
  Für die letzte Summe auf der rechten Seite dieser Ungleichung gilt, unter
  Beachtung von $e_{-1}=e_0$, dass
  \begin{align*}
    &\sum_{j=1}^J\big(E_j,\gradnc
    (e_j-2e_{j-1}+e_{j-2})\big)\\
    &=\sum_{j=1}^J\big(E_j,\gradnc(e_j-e_{j-1})\big)
    -
    \sum_{j=0}^{J-1}\big(E_{j+1},\gradnc(e_j-e_{j-1})\big) \\
    &= 
    \sum_{j=1}^{J-1} 
    \big(E_j-E_{j+1},\gradnc(e_j-e_{j-1})\big)
    +\big(E_J,\gradnc(e_J-e_{J-1})\big)
    - \big(E_1, \gradnc(e_0-e_{-1})\big) \\
    &= 
    \sum_{j=1}^{J-1} 
    \big(E_j-E_{j+1},\gradnc(e_j-e_{j-1})\big)
    +\big(E_J,\gradnc(e_J-e_{J-1})\big).
  \end{align*}
  Mit dieser Umformung erhalten wir aus Ungleichung \eqref{eq:convProofF} für
  jedes $\tau\in(0,1]$, das heißt, $\tau^{-1}\geq 1$, dass
  \begin{equation}
    \label{eq:convProofG}
    \begin{aligned}
      &\vvvert e_J \vvvert^2_\nc +\Vert E_J \Vert^2 
      +\sum_{j=1}^J\left(\vvvert e_j-e_{j-1} \vvvert_\nc^2 + 
      \Vert E_j-E_{j-1}\Vert^2\right) \\
      &\leq 
      \tau^{-1}\left(\vvvert e_0 \vvvert_\nc^2 + \Vert E_0 \Vert^2 \right)
      -2\alpha\sum_{j=1}^J \Vert e_j\Vert^2 \\
      &\quad\quad
      -2 \sum_{j=1}^{J-1} \big(E_j-E_{j+1},\gradnc(e_j-e_{j-1})\big)
      -2\big(E_J,\gradnc(e_J-e_{J-1})\big).
    \end{aligned}
  \end{equation}
  Außerdem gilt
  \begin{align*}
    2\alpha\sum_{j=1}^J\Vert e_j\Vert^2 
    &\leq
    2\alpha\sum_{j=1}^J\Vert e_j\Vert^2
    +\Vert E_J + \gradnc(e_J-e_{J-1}) \Vert^2 
    + \vvvert e_J \vvvert^2_\nc 
    + \Vert E_1 - E_0 \Vert^2 \\
    &\quad\quad
    + \sum_{j=1}^{J-1}  
      \Vert \gradnc(e_j-e_{j-1}) - (E_{j+1} - E_j ) \Vert^2 \\
    &= 
    2\alpha\sum_{j=1}^J\Vert e_j\Vert^2
    +\vvvert e_J \vvvert^2_\nc + \Vert E_J \Vert^2 
    + \sum_{j=1}^J \left( \vvvert e_j-e_{j-1} \vvvert^2_\nc
    + \Vert E_j - E_{j-1} \Vert^2 \right)\\
    &\quad\quad
    + 2\sum_{j=1}^{J-1}\big(E_j-E_{j+1},\gradnc(e_j-e_{j-1})\big)
    + 2\big(E_{J},\gradnc(e_J-e_{J-1})\big).
  \end{align*}
  Zusammen mit Ungleichung \eqref{eq:convProofG} folgt daraus
  \begin{align}
    \label{eq:upperBoundIterationError}
    2\alpha\sum_{j=1}^J\Vert e_j\Vert^2 
    \leq
    \tau^{-1}\left(\vvvert e_0\vvvert^2_\nc + \Vert E_0\Vert^2\right)\!.
  \end{align}
  Somit gilt, dass $\sum_{j=1}^\infty \Vert e_j\Vert^2$ beschränkt
  ist, was impliziert $\Vert\ucr-u_j\Vert=\Vert e_j\Vert\rightarrow 0$ für
  $j\rightarrow \infty$.
\end{proof}


\chapter{Implementierung}
\label{chap:implementation}
\section{Hinweise zur Benutzung des Programms}
%\subsection{Aufbau des Programms}
Ziel der für diese Arbeit implementierten Methoden ist die Realisierung von
\Cref{alg:primalDualIteration} im Solve-Schritt des AFEM-Algorithmus aus
\Cref{fig:afemLoop}. 
Wir gehen davon aus, dass dieser und die im
AFEM-Softwarepaket realisierten Methoden sowie deren Datenstrukturen bekannt
sind und verweisen für weitere Details auf \cite{CGKNRR10}.

\begin{figure}[h]
  \centering
  \tikzstyle{block} = [rectangle, text width=5em, text centered, rounded corners,
  thick, draw = black]
\tikzstyle{line} = [draw = black, thick, -Triangle, rounded corners]

\begin{tikzpicture}[node distance = 3.5cm]
	%Placing nodes
  \node [block] (solve) {\texttt{Solve}};
  \node [block, right of=solve] (estimate) {\texttt{Estimate}};
  \node [block, right of=estimate] (mark) {\texttt{Mark}};
  \node [block, right of=mark] (refine) {\texttt{Refine}};
	
	\path [line] (solve) -- (estimate);
	\path [line] (estimate) -- (mark);
	\path [line] (mark) -- (refine);
	\path [line] (refine) |- (0,-1) -| (solve);
\end{tikzpicture}

  \caption{AFEM-Schleife}
  \label{fig:afemLoop}
\end{figure}

Die zur korrekten Funktionsweise dieses Programms nötigen Methoden und
Dateien des AFEM-Softwarepakets sind enthalten im Ordner
\texttt{./utils/afemPackage/} sowie \texttt{./utils/geometries/}.

Alle Ein- und Ausgabeparameter der im Rahmen dieser Arbeit implementierten
Methoden sind in den entsprechenden Dateien detailiert dokumentiert. 
Ausgenommen davon sind die Standard-Datenstrukturen aus dem AFEM-Softwarepaket,
die dort lediglich namentlich genannt werden.
Die mathematischen Grundlagen für die Realisierung einiger Methoden diskutieren
wir in \Cref{sec:mathematicalBasicsForMethods}.

Die ausführbare Methode, welche den AFEM-Algorithmus realisiert, ist
\begin{center}
  \texttt{./nonconforming/startAlgorithmCR.m}.
\end{center}
Als optionaler Eingabeparameter ist dabei ein String \texttt{benchmark}
möglich. 
Wird die Methode ohne Übergabe eines solchen Parameters ausgeführt, nutzt
sie als Standardwert \texttt{benchmark = 'editable'}.
Nach Ausführen von \texttt{startAlgorithmCR(benchmark)} werden die nötigen
Parameter und Einstellungen für das jeweilige Experiment aus der Datei
\begin{center}
  \texttt{./nonconforming/benchmarks/benchmark.m}
\end{center}
geladen und als Felder des Structure Arrays \texttt{params} übergeben. Für
jedes in dieser Arbeit dokumentierte Experiment verweisen wir in
\Cref{chap:experiments} an entsprechender Stelle auf das dafür benutzte 
Benchmark, welches in \texttt{./nonconforming/benchmarks/} zu finden ist und
somit die Reproduzierbarkeit des Experiments garantiert. Als Muster für 
eine Bechmark-Datei dient
\begin{center}
  \texttt{./nonconforming/benchmarks/editable.m}.
\end{center}
In dieser Datei sind auch die wählbaren Parameter und Einstellungen
dokumentiert.
\todo{alle Benchmark einstellungen hier in einer Tabelle auflisten und
dokumentieren? Wahrscheinlich ja, so kann dokumentiert werden, dass zB 
Prolongation möglich ist. ``Alle Einstellungen und Paramter dokumentieren wir
in Abschnitt \ldots''. Dort kann dann auch, vlt auch erst wenn CC das will,
gesagt werden zB sowas wie prolongiert wird}

Dass die zahlreichen Parameter, die während des Pro\-gramm\-ab\-laufs
über- oder ausgegeben werden müssen, als Felder von
Struc\-ture Ar\-rays gespeichert werden, dient der Modifizierbarkeit des
Programms. 
So haben Korrekturen und Ergänzungen am Programm häufig nur zur Folge, dass
einige \texttt{structs} um Felder ergänzt werden müssen während die 
Methodenköpfe unverändert bleiben können.

Das Eingangssignal $f$ und eventuell weitere Funktionen, wie etwa die exakte
Lösung $u$ von \Cref{prob:continuousProblem} und ihre schwache Ableitung
$\nabla u$, welche in einer Benchmark-Datei angegeben werden müssen um sie dem
Programm zu übergeben, sind zu finden in 
\begin{center}
  \texttt{./utils/functions/}.
\end{center}

Ist eine Lösung $u$ von $\Cref{prob:continuousProblem}$ bekannt, so kann die
exakte Energie $E(u)$ approximiert werden mit der Methode
\begin{center}
  \texttt{./nonconforming/computeExactEnergyBV.m}.
\end{center}
Die so berechneten Energien werden gespeichert in 
\begin{center}
  \texttt{./nonconforming/knownExactEnergies/}
\end{center}
und können anschließend manuell in ein Benchmark aufgenommen werden.
\todo{$f=\alpha g$ und weiter Details zur Umwandlung von Bildern? Hier oder
bei Details zu ausgewählten Funktionen?}

Soll als Eingangssignal kein \texttt{function\_handle} sondern ein
Graufarbenbild gegeben werden, so muss es gespeichert sein in 
\begin{center}
  \texttt{./utils/functions/images/}.
\end{center}
Um Dirichlet-Nullranddaten des Bildes zu garantierten, was einem schwarzen Rand
entspricht, kann die Methode 
\begin{center}
  \texttt{./utils/functions/images/addBoundary2image.m}
\end{center}
genutzt werden. Diese fügt einen graduellen Übergang zu schwarzen Rand auf den 
äußeren 25 Pixeln des Bildes hinzu.

Um additives weißes gaußsches Rauschen zu einen Bild hinzuzufügen, kann die
Methode
\begin{center}
  \texttt{./utils/functions/images/addNoise2image.m}
\end{center}
genutzt werden. 

%\bigskip
%%%%%%%%%%%%%%%%%%%%%%%%%%
%cd Realisierungschapter PB
%
%Tabelle 4.2 (PB) artige Übersicht über die benutzten AFEM 
%Datenstrukturen, die im Programm nicht extra dokumentiert wurden. Für alle
%anderen In- und Outputs verweise auf Dokumentation in den Docstrings
%
%
%
%Die Ausführbare Funktionen werden im nächsten Kapitel beschrieben.
%
%%\subsection{Erstellen eines lauffähigen Benchmarks (Minimalbeispeil)}
%Beschreibung der wichtigsten Parameter
%und Idee hinter structs
%
%Ordner, in denen die Funktionen für rechte Seite, Gradient, exakte
%Lösung etc liegen müsssen
%
%Wahrscheinlich flag für flag durchgehen, erklären, welche automatisch gesetzt 
%werden u.U., und wann immer nötig sagen, was man vorher machen muss, wo man
%Funktionen erstellen muss etc.
%
%fur exakte Lösungs Beispiel usw.
%Berechnung der exakten Energie, also alles was nur mehr Möglichkeiten bietet,
%Verweis auf die nächste Section (in der dann sagen, welche Flags gesetzt werden 
%können)
%
%
%\subsubsection{\texttt{startAlgorithmCR.m}}
%\texttt{test}
%\subsubsection{\texttt{computeExactEnergyBV.m}}
%\subsubsection{\texttt{addNoise2image.m}}
%\subsubsection{\texttt{addBoundary2image.m}}


\section{Erstellen einer Benchmark-Datei}
Vielleicht (das heißt wahrscheinlich) in der 'Hinweise' Section in einer
Tabelle ala PB Realisierungstabelle. Tabellen vlt trennen nach Art der paramter
(AFEM Params, Exp Params, Funktionen, misc usw.)
Beispielhaft für eine der Paramter Arten machen um Tien fragen zu können


\section{Mathematische Grundlagen ausgewählter Funktionen}
\label{sec:mathematicalBasicsForMethods}
\todo{Funktionsname in den Sections als Überschrift oder Nutzen der 
entsprechenden Methode in die Überschrift?}


\subsection{\texttt{computeNodeValuesCR4e}}

Sei $\ucr\in\CR^1(\Tcal)$.

Für die in dieser Arbeit implementierte Prolongation von $\ucr$ auf eine
Verfeinerung der Triangulierung $\Tcal$ oder für die Berechnung der $L^1$-Norm
der Kantensprünge von $\ucr$ benötigen wir für jedes Dreieck $T\in\Tcal$ die
Werte von $\ucr$ in den Knoten von $T$. 
Diese berechnen wir mit der Methode
\begin{center}
  \texttt{./nonconforming/common/computeNodeValuesCR4e.m}
\end{center} 

Dazu sei $T = \conv\{P_1, P_2, P_3\}$ mit den Kanten
$E_1 = \conv\{P_1,P_2\}$, $E_2 = \conv\{P_2,P_3\}$ und $E_3 =
\conv\{P_3,P_1\}$. 
Die Funktion $u\coloneqq\ucr|_T$ habe in den Mittelpunkten der Kanten die Werte
$u_j\coloneqq u\left(\Mid(E_j)\right)$ für alle $j\in\{1,2,3\}$. 
Gesucht sind damit die Werte $u(P_1)$, $u(P_2)$ und $u(P_3)$.

Da $\ucr\in\CR^1(\Tcal)$, ist $u$ affin-linear. Damit gilt für eine Kante
$E=\conv\{P,Q\}\in\{E_1,E_2,E_3\}$, dass $u(\Mid(E))$ gegeben ist durch den
Mittelwert von $u(P)$ und $u(Q)$.

Somit erhalten wir die drei Gleichungen
\begin{align*}
  u_1 &= \frac{u(P_1)+u(P_2)}{2},  
  &u_2 &= \frac{u(P_2)+u(P_3)}{2},  
  &u_3 &= \frac{u(P_3)+u(P_1)}{2}.
\end{align*}
Sind $u_1$, $u_2$ und $u_3$ bekannt, können wir dieses Gleichungssystem nach 
$u(P_1)$, $u(P_2)$ und $u(P_3)$ lösen und erhalten die gesuchten Werte
\begin{align*}
 u(P_1)&=u_1+u_3-u_2, &u(P_2)&= u_1+u_2-u_3,&u(P_3)&=u_2+u_3-u_1.
\end{align*}


\subsection{Berechnung der L1 Norm der Sprünge}
\todo{hier}
Für die Berechnung des Verfeinerungsindikators [verweis auf entsprechende
section] und zur Auswertung der kontinuierlichen Energie $E(\vcr)$ einer 
Crouzeix-Raviart Funktion $\vcr$, deren diskrete Energie $\Enc(\vcr)$
bereits bekannt ist, werden die $L^1$ Normen der Kantensprünge 
$[\vcr]_F$ für alle Kanten $F\in\Fcal$ der Triangulierung benötigt,
wobei für eine Innenkante $F\in\Fcal(\Omega)$, die gemeinsame Kante der
Dreiecke $T_+$ und $T_-$ ist, gilt
$[\vcr]_F\coloneqq (\vcr|_{T_+})|_F-(\vcr|_{T_-})|_F$
und $[\vcr]_F \coloneqq \vcr|_F$ für eine Randkante
$F\in\Fcal(\partial\Omega)$. Die Konvention der Wahl von
$T_+$ und $T_-$ ist hier irrelevant, da wir
zur Berechung von $\Vert [\vcr]_F\Vert_{L^1(\Omega)}$
ausschließlich den Betrag $|[\vcr]_F|$ benötigen.

Da $\vcr\in\CR^1(\Tcal)$,
ist $[\vcr]_F$ affin linear und es gilt $[\vcr]_F(\Mid(F))=0$ für 
alle Innenkanten $F\in\Fcal(\Omega)$ und, falls 
$\vcr\in\CR^1_0(\Tcal)$, auch für alle Randkanten $F\in\Fcal(\partial\Omega)$.

Die folgenden Aussagen gelten also für Innenkanten beliebiger
Crouzeix-Raviart Funktionen, wir beschränken uns aber von nun
an auf Funktionen $\vcr\in\CR^1_0(\Tcal)$.

Betrachten wir also eine beliebige Kante $F\in\Fcal$
mit $F=\conv\{P_1,P_2\}$. 
Wir definieren eine Parametrisierung $\gamma:[0,2]\to\Rbb^2$ von $F$ durch
$\gamma(t)\coloneqq \frac{t}{2}(P_2-P_1)+P_1$. 
Es gilt $|\gamma'|\equiv \frac{1}{2}|P_2-P_1|=\frac{1}{2}|F|$.

Sei außerdem
$p(t)\coloneqq [\vcr]_F(\gamma(t))$. Dann gilt nach
 [cite Wegintegrale] 
\begin{align*}
  \Vert [\vcr]_F\Vert_{L^1(F)} 
  &=
  \int_F |[\vcr]_F|\ds 
  = \int_0^2 |p(t)|\,|\gamma'(t)|\dt
  = \frac{|F|}{2}\int_0^2 |p(t)|\dt\\
  &= \frac{|F|}{2}\left(\int_0^1 |p(t)|\dt + \int_1^2 |p(t)|\dt\right).
\end{align*}

Da $\vcr\in\CR^1_0(\Tcal)$, ist $|p|$ auf $[0,1]$ und $[1,2]$ jeweils
ein Polynom
vom Grad $1$ mit $p(1)=[\vcr]_F(\Mid(F))=0$, womit sich $|p|$ jeweils
explizit ausdrücken lässt durch
$|p|(t)=(1-t)|p|(0)$ für alle $t\in[0,1]$ und 
$|p|(t)=(t-1)|p|(2)$ für alle $t\in[1,2]$.
Die Mittelpunktsregel $\int_a^b f(x)\dx\approx (b-a)f( (a+b)/2)$ [cite] ist
exakt für Polynome vom Grad $1$ und somit gilt
\begin{align*}
  \int_0^1 |p(t)|\dt 
  &= 
  (1-0)|p|\left( \frac{1}{2} \right)
  =
  \frac{|p|(0)}{2}\quad\text{und }\\
  \int_1^2 |p(t)|\dt 
  &= 
  (2-1)|p|\left( \frac{3}{2} \right)
  =
  \frac{|p|(2)}{2}.
\end{align*}

Somit erhalten wir insgesamt 
\begin{align*}
  \Vert [\vcr]_F\Vert_{L^1(F)} 
  &=
  \frac{|F|}{2}\left(\frac{|p|(0)}{2} + \frac{|p|(2)}{2}\right)
  =
  \frac{|F|}{4}(|p|(0)+|p|(2))\\
  &= 
  \frac{|F|}{4}\big(|[\vcr]_F|(P_1)+|[\vcr]_F|(P_2)\big),
\end{align*}
beziehungsweise 
  $\Vert [\vcr]_F\Vert_{L^1(F)} =
  \frac{|F|}{4}\big(|\vcr|(P_1)+|\vcr|(P_2)\big)$ für eine Randkanten
  $F\in\Fcal(\partial\Omega)$.

Diese Berechnung ist realisiert durch die 
Funktionen \texttt{computeBlaJumps}, die die absoluten Sprünge
in den Endpunkte einer Kante berchnet, \texttt{computeAbsJumps}, die \ldots,
und \texttt{computeL1NormOfJumps}, die schließlich die $L^1$ Norm aller
Kantensprünge berechnet\ldots.



\subsection{Implementation der GLEB}

\subsection{Implementation des Refinement Indicators}

\subsection{Implementaition der exakten Energie Berechnung}




\chapter{Numerische Beispiele}
\label{chap:experiments}
\section{Konstruktion eines Experiments mit exakter Lösung}
Um eine rechte Seite zu finden, zu der die exakte Lösung bekannt
ist, wähle eine Funktion des Radius $u\in H^1_0([0,1])$ mit Träger im 
zweidimensionalen Einheitskreis. Insbesondere muss damit gelten $u(1)=0$ und
$u$ stetig.
Die rechte Seite als Funktion des Radius $f\in L^2([0,1])$ ist dann gegeben
durch 
\begin{align*}
  f \coloneqq 
  \alpha u - \partial_r(\sign(\partial_r u)) - \frac{\sign(\partial_r u)}{r},
\end{align*}
wobei für $F\in\Rbb^2\setminus\{0\}$ gilt 
$\sign(F)\coloneqq \left\{\frac{F}{|F|}\right\}$ 
und $\sign(0)\in B_1(0)$.
Damit außerdem gilt $f\in H^1_0([0,1])$, was z.B.\ für GLEB relevant ist, 
muss also noch Stetigkeit von $\sign(\partial_r u)$ und 
$\partial_r(\sign(\partial_r u))$ verlangt werden und 
$\partial_r(\sign(\partial_r u(1))=\sign(\partial_r u(1))=0$.

Auf diese Weise erhält man die Experiment TODO

Dann berechnet man Gradienten von u und f indem TODO
Und dann berechnet man exakte Energie mit Funktion TODO indem TODO

Dann kann man das Programm nutzen.


\chapter{Auswertung}
\label{chap:review}
In diesen Abschnitt werten wir die in \Cref{chap:experiments} erhaltenen
Ergebnisse aus.


%% --- APPENDIX -------------------------------------------------------------
%
%\appendix
%\chapter{Appendix}
%
\begin{frame}{Input Signal $f$}
  \begin{figure}[!ht]
    \centering
    \includegraphics[width=0.8\linewidth]
      {pictures/experiments/appendix/timeCompPrealloc.pdf}
  \end{figure}
\end{frame}

\begin{frame}{Input Signal $f$, $\theta = 1$}
  \begin{figure}[!ht]
    \centering
    \includegraphics[width=0.9\linewidth]
      {pictures/experiments/appendix/f01UniformLvl8.png}
  \end{figure}
\end{frame}

\begin{frame}
  Let $u_P:[0,\infty)\to\Rbb$ with $u_P(r)=0$ for $r\geq 1$,
  and, for all $x\in\Omega$, $u(x)= u_P\big(|x|\big)$. 
  %\pause
  Furthermore, assume the existence  of $\partial_r u_P$ a.e.\ in $[0,\infty)$,
  the existence of the derivative of
  \begin{align*}
    \operatorname{sgn}\big(\partial_r u_P(r)\big)
    \coloneqq
    \begin{cases}
      -1 &\text{für }\partial_r u_P(r)<0,\\
      x\in[0,1] &\text{für }\partial_r u_P(r)=0,\\ 
      1 &\text{für }\partial_r u_P(r)>0.
    \end{cases}
  \end{align*}
  a.e.\ in $[0,\infty)$, and
  that $\operatorname{sgn}\big(\partial_r u_P(r)\big)/r\to 0$ as $r\to 0$.
  %\pause
  For all $r\in[0,\infty)$, define 
  \begin{align*}
    f_P(r)
    \coloneqq 
    \alpha u_P(r) - \partial_r\left(\operatorname{sgn}\big(\partial_r u_P(r)\big)\right) 
    - \frac{\operatorname{sgn}\big(\partial_r u_P(r)\big)}{r}
  \end{align*}
  %\pause
  Then $u$ solves the continuous problem
  on $\Omega\supseteq \left\{w\in\Rbb^2\,\middle|\, |w|\leq 1\right\}$ if
  the input signal is $f(x)\coloneqq f_P\big(|x|\big)$.
\end{frame}


% --- BIBLIOGRAPHY -------------------------------------------------------------

\printbibliography
\addcontentsline{toc}{chapter}{Literatur}

\chapter*{Selbständigkeitserklärung}

Ich erkläre, dass ich die vorliegende Arbeit selbständig verfasst und noch nicht 
für andere Prüfungen eingereicht habe. Sämtliche Quellen, einschließlich
Internetquellen, die unverändert oder abgewandelt wiedergegeben werden,
insbesondere Quellen für Texte, Grafiken, Tabellen und Bilder, sind als solche
kenntlich gemacht. Mir ist bekannt, dass bei Verstößen gegen diese Grundsätze ein
Verfahren wegen Täuschungsversuchs bzw.\ Täuschung eingeleitet wird. 
\bigbreak
\noindent Berlin, den \today, 
\end{document}

