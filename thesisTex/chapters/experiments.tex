allgemeine Sachen (degree4Integrate, parTheta) hier aufführen (mit ,wenn nicht
anders angegeben)

überlegen, welche Parmeter für jedes Experiment nochmal gesagt werden sollen
definitiv: 

Triangulierung: hier am Anfang einmal sagen, dass es für Funktionen
als Input immer BigSquare war und für echte Bilder als Input Square

f01 und cameraman schon hier vorstellen als Beispiel mit exakter Lösung,
die anderen Funktionen aber erst dann, wenn es soweit ist

Nach \Cref{eq:constructionInputSignal} \ldots
Für einen Parameter $\beta\geq 1/2$, wir wählen stets $\beta =1$, erhalten wir
für die Funktion
\begin{align*}
  u_1(r)&\coloneqq
  \begin{cases}
    1, 
    & \text{falls } r\in \left[0,\frac{1}{6}\right],\\
    1+(6r-1)^\beta, 
    & \text{falls } r\in \left(\frac{1}{6}, \frac{1}{3}\right],\\
    2, 
    & \text{falls } r\in \left(\frac{1}{3}, \frac{1}{2}\right],\\
    2\left(\frac{5}{2}-3r\right)^\beta, 
    & \text{falls } r\in \left(\frac{1}{2}, \frac{5}{6}\right],\\
    0, 
    & \text{falls } r\in \left(\frac{5}{6}, \infty\right),\\
  \end{cases}
  \quad\text{mit der Wahl}\\
  \sgn\big(\partial_r u_1(r)\big) 
  &\coloneqq
  \begin{cases}
    12r-36r^2, 
    & \text{falls } r\in \left[0,\frac{1}{6}\right],\\
    1, 
    & \text{falls } r\in \left(\frac{1}{6}, \frac{1}{3}\right],\\
    \cos(\pi(6r-2)), 
    & \text{falls } r\in \left(\frac{1}{3}, \frac{1}{2}\right],\\
    -1, 
    & \text{falls } r\in \left(\frac{1}{2}, \frac{5}{6}\right],\\
    -\frac{1+\cos(\pi(6r-5))}{2}, 
    & \text{falls } r\in \left(\frac{5}{6}, \infty\right),\\
  \end{cases}
  \quad\text{das Eingangssignal}\\
  f_1(r)
  &=
  \begin{cases}
    \alpha-12(2-9r), 
    & \text{falls } r\in \left[0,\frac{1}{6}\right],\\
    \alpha\left(1+(6r-1)^\beta\right)-\frac{1}{r}, 
    & \text{falls } r\in \left(\frac{1}{6}, \frac{1}{3}\right],\\
    2\alpha+6\pi\sin(\pi(6r-2))-\frac{1}{r}\cos(\pi(6r-2)), 
    & \text{falls } r\in \left(\frac{1}{3}, \frac{1}{2}\right],\\
    2\alpha\left(\frac{5}{2}-3r\right)^\beta+\frac{1}{r},
    & \text{falls } r\in \left(\frac{1}{2}, \frac{5}{6}\right],\\
    -3\pi\sin(\pi(6r-5))+\frac{1+\cos(\pi(6r-5))}{2r}, 
    & \text{falls } r\in \left(\frac{5}{6}, \infty\right).
  \end{cases}
\end{align*}
Die entsprechenden Ableitungen lauten
\begin{align*}
  \partial_r f_1(r)
  &=
  \begin{cases}
    108,
    & \text{falls } r\in\left[0,\frac{1}{6}\right],\\
    6\alpha\beta(6r-1)^{\beta-1} +\frac{1}{r^2}, 
    & \text{falls } r\in\left(\frac{1}{6},\frac{1}{3}\right],\\
    \left(36\pi^2+\frac{1}{r^2}\right)\cos(\pi(6r-2))
    + \frac{6\pi}{r}\sin(\pi(6r-2)), 
    & \text{falls } r\in\left(\frac{1}{3},\frac{1}{2}\right],\\
    -\left(6\alpha\beta\left( \frac{5}{2}-3r \right)^{\beta-1}+
    \frac{1}{r^2}\right),
    & \text{falls } r\in\left(\frac{1}{2},\frac{5}{6}\right],\\
    -\left( \left( 18\pi^2+\frac{1}{2r^2} \right)\cos(\pi(6r-5))
    +\frac{1}{2r^2} + \frac{3\pi}{r}\sin(\pi(6r-5))\right), 
    &\text{falls } r\in\left(\frac{5}{6},\infty\right),
  \end{cases}\\
  \partial_r u_1(r) 
  &= 
  \begin{cases}
    0,
    & \text{falls } r\in\left[0,\frac{1}{6}\right],\\
    6\beta(6r-1)^{\beta-1}, 
    & \text{falls } r\in\left(\frac{1}{6},\frac{1}{3}\right],\\
    0, 
    & \text{falls } r\in\left(\frac{1}{3},\frac{1}{2}\right],\\
    -6\beta\left( \frac{5}{2}-3r \right)^{\beta-1},
    & \text{falls } r\in\left(\frac{1}{2},\frac{5}{6}\right],\\
    0,
    &\text{falls } r\in\left(\frac{5}{6},\infty\right).
  \end{cases}
\end{align*}

dann die Bilder von u und inSi als Plots und entlang der Achsen.

Ableitung von u wird zu Berechung von E(u) gebraucht nochmal sagen, sonst 
irrelevant für Experimt

vielleicht ,der Gradient von u ist\ldots, damit approximieren wir die
exakte Energie \ldots)

\section{Wahl der Einstellungen für die primale-duale Iteration}
alles zu tau mit hypothesse

beispiel ohne konvergenz aufzeigen und dafür auch die Triangulierung zeigen

zu abbruchkriterium


\section{Experimente mit bekannter exakter Lösung}

f01 vor allem

dann abschweif zu höhere Regularität

hier eventuell auch experimte für AFEM Params?

dann insgesamt raten abarbeiten und interpretieren

\section{Graufarbenbilder als Eingangssignale}
cameraman

abschweif zu denoise example aus intro hier erklären

zum abschluss zu approximation von white circle mit stetigeer funktion (plots
dementsprechend



%%%%% hiervor finalized
%%%%% ab hier Skizze

Die im vorherigen Kapitel beschriebenen Methoden werden wir an einigen 
Benchmark-Problemen untersuchen (umschreiben, von PB übernommen).
Dazu gehört ein Beispiel mit bekannter Lösung, das wir nach cite(section)
konstruieren, ein komplexes Beispiel ohne Lösung mit einer unstetigen Funktion,
hier ein Bild als Eingangssignal und eine einfache, diskontinuierliche Funktion
als Eingangssignal und eine, ebenfalls nach section konstruierte stetige 
Approximation dieser.




\todo[inline]{Neue Gliederung: vgl Storn oder noch besser Sophie, statt
Auswertung ,Zusammenfassung und Ausblick'. Oder einfach wie Franz, ohne
Zsmfssg, ,Numerische Experimente und SChlussfolgerung' oder so}


Alle Allgemeinen Infos unter die Hauptüberschrift, keine eigene Section,
da auch integrate gleich 10 erwähnen



Mache alle mögliche Sachen, prüfe Dinge die gelten sollen. Für ein
Beispiel (z.B. CCs) fange mit basics an, also mal an einem Iterationsplot
aufzeigen, dass die Energie tatsächlich von oben gegen was konvergiert und 
gehe dann weiter zu anderen Themen wie den Raten, GLEB etc.

\todo[inline]{Probiere auch thm:convexity irgendwie informell zu erwähnen ,,wir
sehen, dass sogar ohne die Sprünge das stimmt\ldots`` oder so}

\todo[inline]{in chap 6 cf.s auf integrate etc nicht ständig sonder einmal 
gesammelt am Anfang oder drauf achten, dass das nur beim ersten Erwähnen zitiert 
wird}

\todo[inline]{Programm gibt Sprungtermsumme aus, z.B. stagnierend bei 11 auf
allen Leveln, darauf vielleicht noch kurz in der Auswertung eingehen ,man
sieht, dass die Sprünge tatsächlch im nonkonformen Problem nicht minimiert
werden oder so'}

\bigskip
Als erste Experimente für CCs Benchmark alle möglichen Parameter untersuchen 
(z.B. die Parameter im refinement indicator, GLEB, Abbruchkriterien etc)
und danach begründen, warum für alle Experimente diese gewählt werden mit
den Beobachtungen

\todo[inline]{ascent and descent flow, gradient flow heißt doch, dass
Gradientenverfahren angewendet wurde, richtig?
Stopping criteria sind in Punkt 6.2 von Bar12}
\bigskip

\todo[inline]{Immer nur Beobachtungen aufschreiben, die Raten sagen, 
Sachverhalte einbringen. Auswertungen dann eben, wie schon der Name sagt,
in Auswertungen}
\todo[inline]{Ab hier Content}

Zuerst allgemeines, alle Infromationen, die ggf bei allen Experimten in den 
folgenden Abschnitten immer gleich sind und möglicherweise kurze Kommentare,
warum diese entspreched gewählt sind.

\bigskip

Alle Experimentparameter die stets gleich gesetzt sind Eingangs einmal erwähnen,
insbesondere degree4Integrate, was immer als 10 gesetzt wird, mit ähnlichem 
Kommentar wie PB auf S 30

\section{Betrachte Probleme}
Einzelne Experimente in eigenen Subsections
\bigskip

Hier auch alle möglichen rechten Seiten beschreiben.
Alle möglichen rechten Seiten hier auflisten, insbesondre die rechten 
Seite mit exakten Lösungen mit verweise auf Kapitel 3.
Die Funktionen entsprechend bezeichnen

\bigskip

Vielleicht auch hier dann

\section{Betrachtungen der Iteration}

Hier alles was man so über die Iteration sagen kann. Alles am Beispiel, dass

\section{Bilder als Input und Rauschverminderung}

Bei Beschreibung wie die Bilder der Einleitung zur Rauschunterdrückung
entstanden sind auch erwähnen, dass $f=\alpha g$ eine Rolle spiele usw.

\section{Alternative Kapitelstruktur: Anwendung am Beispiel der Benchmarks}
\subsection{Experimente mit exakter Lösung: Benchmarks 'f01Bench' und 'f02'
Bench}

\subsection{Bilder als Input und Rauschverminderung: 'cameramanBench' und
\ldots}

%%%%%%%%%%%%%%%%%%%%%%%%%%%%%%%%%%%%%%%%%%%%%%%%%%%%%%
\todo[inline]{Beschreibe hier nur das allgemeine Vorgehen und die 
mathematischen Hintergründe und schreibe die genutzten Beispiele an
entsprechender Stelle in Numerische Beispiele oder so. Die ungenutzten Beispiele 
die trotzdem im Ordner liegen können in einem extra Abschnitt aufgezählt werden}
Schema ist dann: Nach section Konstruktion einer exakten Lösung erhalten wir
für die Wahl u mit sgn bla die rechte Seite f.
Die Details bei bestimmten Konstruktionen, etwa ,,um H2 zu erhalten fordern wir
noch`` ergänzen beim Vorstellen des entsprechenden Experimentes, hier nur das 
allgemeinste, das, was CC wirklich geschrieben hat.
%%%%%%%%%%%%%%%%%%%%%%%%%%%%%%%%%%%%%%%%%%%%%%%%%%%%%%
Wir wollen damit Experimente konstruieren um Aussagen in \ldots oder \ldots
zu prüfen, entsprechend müssen wir noch fordern
Da in den Kapiteln GLEB und bla u in H10 relevant wird, fordern wir
weiterhin \ldots
%%%%%%%%%%%%%%%%%%%%%%%%%%%%%%%%%%%%%%%%%%%%%%%%%%%%%%


Für die Funktion
\begin{align*}
  u_2(r)\coloneqq 
  \begin{cases}
    1, & \text{wenn } 0\leq r\leq\frac{1-\beta}{2},\\
    -\frac{1}{\beta}r + \frac{1+\beta}{2\beta}, & 
    \text{wenn } \frac{1-\beta}{2}\leq r\leq \frac{1+\beta}{2},\\
    0, & \text{wenn } \frac{1+\beta}{2}\leq r,
  \end{cases}
\end{align*}
erhält man mit der Wahl
\begin{align*}
  \sgn&(\partial_r u_2(r)) \\
  &\coloneqq 
  \begin{cases}
    \frac{4}{1-\beta}r\left(\frac{1}{1-\beta}r -1\right), &
    \text{wenn } 0\leq r\leq\frac{1-\beta}{2},\\
    -1, & \text{wenn } \frac{1-\beta}{2}\leq r\leq \frac{1+\beta}{2},\\
    \frac{4}{(\beta-1)^3}
    \left( 4r^3-3(\beta+3)r^2 +6(\beta+1)r-3\beta-1\right), & 
    \text{wenn } \frac{1+\beta}{2}\leq r\leq 1,
  \end{cases}
\end{align*}
die rechte Seite
\begin{align*}
  f_2(r)\coloneqq 
  \begin{cases}
    \alpha - \frac{4}{1-\beta}\left(\frac{3}{1-\beta}r - 2\right), &
    \text{wenn } 0\leq r\leq\frac{1-\beta}{2},\\
    -\frac{\alpha}{\beta}\left( r-\frac{1+\beta}{2} \right) +\frac{1}{r}, & 
    \text{wenn } \frac{1-\beta}{2}\leq r\leq \frac{1+\beta}{2},\\
    \frac{-4}{(\beta-1)^3}
    \left( 16r^2 -9(\beta+3)r + 12(\beta+1) - \frac{3\beta+1}{r}\right), & 
    \text{wenn } \frac{1+\beta}{2}\leq r\leq 1.
  \end{cases}
\end{align*}

Es folgen zwei Beispiele mit exakter Lösung $u_3=u_4 \in H^2_0((0,1)^2)$, 
gegeben durch 
\begin{align*}
  u_3(r)=u_4(r)\coloneqq 
  \begin{cases}
    1, & \text{wenn } 0\leq r\leq\frac{1}{3},\\
    54r^3 - 81r^2 + 36r - 4, & 
    \text{wenn } \frac{1}{3}\leq r\leq \frac{2}{3},\\
    0, & \text{wenn } \frac{2}{3}\leq r.
  \end{cases}
\end{align*}
Mit der Wahl
\begin{align*}
  \sgn&(\partial_r u_3(r)) \\
  &\coloneqq 
  \begin{cases}
    54r^3-27r^2, & \text{wenn } 0\leq r\leq\frac{1}{3},\\
    -1, & \text{wenn } \frac{1}{3}\leq r\leq \frac{2}{3},\\
    -54r^3 + 135r^2 - 108r + 27, & \text{wenn } \frac{2}{3}\leq r\leq 1,
  \end{cases}
\end{align*}
erhalten wir die rechte Seite
\begin{align*}
  f_3(r)\coloneqq 
  \begin{cases}
    \alpha - 216r^2 + 81r, &
    \text{wenn } 0\leq r\leq\frac{1}{3},\\
    \alpha\left(54r^3 - 81r^2 + 36r - 4\right)) + \frac{1}{r}, & 
    \text{wenn } \frac{1}{3}\leq r\leq \frac{2}{3},\\
    216r^2 - 405r + 216 - \frac{27}{r}, & 
    \text{wenn } \frac{2}{3}\leq r\leq 1,
  \end{cases}
\end{align*}
für die gilt $f_3\in H^1_0$
und mit der Wahl
\begin{align*}
  \sgn&(\partial_r u_4(r)) \\
  &\coloneqq 
  \begin{cases}
    -1458r^5 + 1215r^4 - 270r^3, & \text{wenn } 0\leq r\leq\frac{1}{3},\\
    -1, & \text{wenn } \frac{1}{3}\leq r\leq \frac{2}{3},\\
    -243r^4 + 756r^3 - 864r^2 + 432r - 81, 
    & \text{wenn } \frac{2}{3}\leq r\leq 1,
  \end{cases}
\end{align*}
erhalten wir die rechte Seite
\begin{align*}
  f_4(r)\coloneqq 
  \begin{cases}
    \alpha + 8748r^4 - 6075r^3 + 1080r^2, &
    \text{wenn } 0\leq r\leq\frac{1}{3},\\
    \alpha\left(54r^3 - 81r^2 + 36r - 4\right) + \frac{1}{r}, & 
    \text{wenn } \frac{1}{3}\leq r\leq \frac{2}{3},\\
    1215r^3 - 3024r^2 + 2592r - 864 + \frac{81}{r}, & 
    \text{wenn } \frac{2}{3}\leq r\leq 1,
  \end{cases}
\end{align*}
für die gilt $f_4\in H^2_0$.


\begin{align*}
  \partial_r f_2(r) &= 
  \begin{cases}
    -\frac{12}{(1-\beta)^2},&\text{wenn }0\leq r\leq\frac{1-\beta}{2},\\
    -\frac{\alpha}{\beta}-\frac{1}{r^2},&
    \text{wenn } \frac{1-\beta}{2}\leq r\leq \frac{1+\beta}{2},\\
    -\frac{4}{(1-\beta)^3}\left( 32r-9(\beta+3)+\frac{3\beta+1}{r^2} \right),&
    \text{wenn } \frac{1+\beta}{2}\leq r\leq 1,\\
  \end{cases}\\
  \partial_r u_2(r) &= 
  \begin{cases}
    0,&\text{wenn }0\leq r\leq\frac{1-\beta}{2},\\
    -\frac{1}{\beta},&
    \text{wenn } \frac{1-\beta}{2}\leq r\leq \frac{1+\beta}{2},\\
    0,&\text{wenn } \frac{1+\beta}{2}\leq r,
  \end{cases}\\
  \partial_r f_3(r) &=
  \begin{cases}
    - 432r + 81, & \text{wenn } 0\leq r\leq\frac{1}{3},\\
    \alpha\left(162r^2 - 162r + 36\right) - \frac{1}{r^2}, & 
    \text{wenn } \frac{1}{3}\leq r\leq \frac{2}{3},\\
    432r - 405 + \frac{27}{r^2}, & 
    \text{wenn } \frac{2}{3}\leq r\leq 1,
  \end{cases}\\
  \partial_r f_4(r) &=
  \begin{cases}
    34992r^3 - 18225r^2 + 2160r, & \text{wenn } 0\leq r\leq\frac{1}{3},\\
    \alpha\left(162r^2 - 162r + 36\right) - \frac{1}{r^2}, & 
    \text{wenn } \frac{1}{3}\leq r\leq \frac{2}{3},\\
    3645r^2 - 6048r + 2592 - 864 - \frac{81}{r^2}, & 
    \text{wenn } \frac{2}{3}\leq r\leq 1,
  \end{cases}\\
  \partial_r u_{3,4}(r) &=
  \begin{cases}
    0, & \text{wenn } 0\leq r\leq\frac{1}{3},\\
    162r^2 - 162r + 36, & 
    \text{wenn } \frac{1}{3}\leq r\leq \frac{2}{3},\\
    0, & \text{wenn } \frac{2}{3}\leq r\leq 1,
  \end{cases}\\
\end{align*}
