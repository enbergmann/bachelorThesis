Mache alle mögliche Sachen, prüfe Dinge die gelten sollen. Für ein
Beispiel (z.B. CCs) fange mit basics an, also mal an einem Iterationsplot
aufzeigen, dass die Energie tatsächlich von oben gegen was konvergiert und 
gehe dann weiter zu anderen Themen wie den Raten, GLEB etc.

\todo[inline]{Programm gibt Sprungtermsumme aus, z.B. stagnierend bei 11 auf
allen Leveln, darauf vielleicht noch kurz in der Auswertung eingehen ,man
sieht, dass die Sprünge tatsächlch im nonkonformen Problem nicht minimiert
werden oder so'}
\section{Bilder als Input und Rauschverminderung}
\section{Alternative Kapitelstruktur: Anwendung am Beispiel der Benchmarks}
\subsection{Experimente mit exakter Lösung: Benchmarks 'f01Bench' und 'f02'
Bench}

\subsection{Bilder als Input und Rauschverminderung: 'cameramanBench' und
\ldots}
