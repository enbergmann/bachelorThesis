\section{Konstruktion eines Experiments mit exakter Lösung}
Um eine rechte Seite zu finden, zu der die exakte Lösung bekannt
ist, wähle eine Funktion des Radius $u\in H^1_0([0,1])$ mit Träger im 
zweidimensionalen Einheitskreis. Insbesondere muss damit gelten $u(1)=0$ und
$u$ stetig.
Die rechte Seite als Funktion des Radius $f\in L^2([0,1])$ ist dann gegeben
durch 
\begin{align*}
  f \coloneqq 
  \alpha u - \partial_r(\sign(\partial_r u)) - \frac{\sign(\partial_r u)}{r},
\end{align*}
wobei für $F\in\Rbb^2\setminus\{0\}$ gilt 
$\sign(F)\coloneqq \left\{\frac{F}{|F|}\right\}$ 
und $\sign(0)\in B_1(0)$.
Damit außerdem gilt $f\in H^1_0([0,1])$, was z.B.\ für GLEB relevant ist, 
muss also noch Stetigkeit von $\sign(\partial_r u)$ und 
$\partial_r(\sign(\partial_r u))$ verlangt werden und 
$\partial_r(\sign(\partial_r u(1))=\sign(\partial_r u(1))=0$.

Auf diese Weise erhält man die rechten Seiten
\begin{align*}
  f_1(r)\coloneqq 
  \begin{cases}
    \alpha-12(2-9r), & \text{wenn } 0\leq r\leq\frac{1}{6},\\
    \alpha(1+(6r-1)^\beta)-\frac{1}{r}, & \text{wenn } \frac{1}{6}\leq r\leq
    \frac{1}{3},\\
    2\alpha+6\pi\sin(\pi(6r-2))-\frac{1}{r}\cos(\pi(6r-2)), &
    \text{wenn } \frac{1}{3}\leq r\leq\frac{1}{2},\\
    2\alpha(\frac{5}{2}-3r)^\beta+\frac{1}{r},&
    \text{wenn } \frac{1}{2}\leq r\leq\frac{5}{6},\\
    -3\pi\sin(\pi(6r-5))+\frac{1+\cos(\pi(6r-5))}{2r}, &
    \text{wenn } \frac{5}{6}\leq r\leq 1,
  \end{cases}
\end{align*}
mit exakter Lösung
\begin{align*}
  u_1(r)\coloneqq
  \begin{cases}
    1, & \text{wenn } 0\leq r\leq\frac{1}{6},\\
    1+(6r-1)^\beta, & \text{wenn } \frac{1}{6}\leq r\leq
    \frac{1}{3},\\
    2, &
    \text{wenn } \frac{1}{3}\leq r\leq\frac{1}{2},\\
    2(\frac{5}{2}-3r)^\beta, &
    \text{wenn } \frac{1}{2}\leq r\leq\frac{5}{6},\\
    0, &
    \text{wenn } \frac{5}{6}\leq r\leq 1
  \end{cases}
\end{align*}
und
\begin{align*}
  f_2(r)\coloneqq 
  \begin{cases}
    \alpha(-2r+1), & \text{wenn } 0\leq r\leq\frac{1}{2},\\
    64r^2-108r+48-\frac{4}{r}, & \text{wenn } \frac{1}{2}\leq r\leq 1,
  \end{cases}
\end{align*}
mit exakter Lösung
\begin{align*}
  u_2(r)\coloneqq 
  \begin{cases}
    -2r+1, & \text{wenn } 0\leq r\leq\frac{1}{2},\\
    0, & \text{wenn } \frac{1}{2}\leq r\leq 1.
  \end{cases}
\end{align*}

Damit können Experimente durchgeführt werden bei denen 
\texttt{exactSolutionKnown = true} gesetzt werden kann und entsprechend auch 
der $L^2$-Fehler berechnet wird.

Soll nun auch die Differenz der exakten Energie mit der garantierten unteren 
Energie Schranke (GLEB) berechnet werden, dann werden die stückweisen
Gradienten der exakten Lösung und der rechten Seite benötigt.

Dabei gelten folgende Ableitungsregeln für die Ableitungen einer Funktion 
$g$, wenn man ihr Argument $x=(x_1,x_2)\in\Rbb^2$ in Polarkoordinaten mit Länge
$r=\sqrt{x_1^2+x_2^2}$ und Winkel
$\varphi = \atan(x_2,x_1)$, wobei 
\begin{align*}
  \atan(x_2,x_1)\coloneqq
  \begin{cases}
    \arctan\left( \frac{x_2}{x_1} \right),& \text{wenn }x_1>0,\\
    \arctan\left( \frac{x_2}{x_1} \right) +\pi,& \text{wenn }x_1<0,x_2\geq 0,\\
    \arctan\left( \frac{x_2}{x_1} \right) -\pi,& \text{wenn }x_1<0,x_2<0,\\
    \frac{\pi}{2},& \text{wenn }x_1=0,x_2>0,\\
    -\frac{\pi}{2},& \text{wenn }x_1=0,x_2<0,\\
    \text{undefiniert},& \text{wenn }x_1=x_2=0,\\
  \end{cases}
\end{align*}
auffasst,
\begin{align*}
  \partial_{x_1} &= 
  \cos(\varphi)\partial_r - \frac{1}{r}\sin(\varphi)\partial_\varphi,\\
  \partial_{x_2} &= 
  \sin(\varphi)\partial_r - \frac{1}{r}\cos(\varphi)\partial_\varphi.
\end{align*}
Ist $g$ vom Winkel $\varphi$ unabhängig, so ergibt sich
\begin{align*}
  \nabla_{(x_1,x_2)}g = (\cos(\varphi),\sin(\varphi))\partial_r g.
\end{align*}
Unter Beachtung der trigonometrischen Zusammenhänge
\begin{align*}
  \sin(\arctan(y)) = \frac{y}{\sqrt{1+y^2}},\\
  \cos(\arctan(y)) = \frac{1}{\sqrt{1+y^2}}
\end{align*}
ergibt sich 
\begin{align*}
  (\cos(\varphi),\sin(\varphi)) = (x_1,x_2)\frac{1}{r}
\end{align*}
und damit 
\begin{align*}
  \nabla_{(x_1,x_2)}g = (x_1,x_2)\frac{\partial_r g}{r},
\end{align*} 
es muss also nur $\partial_r g$ bestimmt werden.

Die entsprechenden Ableitung lauten
\begin{align*}
  \partial_r f_1(r) &= 
  \begin{cases}
    108,&\text{wenn }0\leq r\leq\frac{1}{6},\\
    6\alpha\beta(6r-1)^{\beta-1} +\frac{1}{r^2}, &
    \text{wenn } \frac{1}{6}\leq r\leq\frac{1}{3},\\
    (36\pi^2+\frac{1}{r^2})\cos(\pi(6r-2))+
    \frac{6\pi}{r}\sin(\pi(6r-2)), &
    \text{wenn } \frac{1}{3}\leq r\leq\frac{1}{2},\\
    -\left(6\alpha\beta\left( \frac{5}{2}-3r \right)^{\beta-1}+
    \frac{1}{r^2}\right),&
    \text{wenn } \frac{1}{2}\leq r\leq\frac{5}{6},\\
    -\left( \left( 18\pi^2+\frac{1}{2r^2} \right)\cos(\pi(6r-5))+ 
    \frac{1}{2r^2}+\frac{3\pi}{r}\sin(\pi(6r-5))\right),&
    \text{wenn } \frac{5}{6}\leq r\leq 1,
  \end{cases}\\
  \partial_r u_1(r) &= 
  \begin{cases}
    0,&\text{wenn }0\leq r\leq\frac{1}{6},\\
    6\beta(6r-1)^{\beta-1}, &\text{wenn } \frac{1}{6}\leq r\leq\frac{1}{3},\\
    0, &\text{wenn } \frac{1}{3}\leq r\leq\frac{1}{2},\\
    -6\beta\left( \frac{5}{2}-3r \right)^{\beta-1},&
    \text{wenn } \frac{1}{2}\leq r\leq\frac{5}{6},\\
    0,&\text{wenn } \frac{5}{6}\leq r\leq 1,
  \end{cases}\\
  \partial_r f_2(r) &= 
  \begin{cases}
    -\left( 2\alpha +\frac{1}{r^2} \right),&\text{wenn }0\leq r\leq\frac{1}{6},\\
    2\left( 64r-54+\frac{2}{r^2} \right),&
    \text{wenn } \frac{5}{6}\leq r\leq 1,
  \end{cases}\\
  \partial_r u_2(r) &= 
  \begin{cases}
    -2,&\text{wenn }0\leq r\leq\frac{1}{2},\\
    0,&\text{wenn } \frac{1}{2}\leq r\leq 1.
  \end{cases}
\end{align*}

Mit diesen Informationen kann mit \texttt{computeExactEnergyBV.m} die exakte 
Energie berechnet werden und somit durch eintragen der exakten Energie
in die Variable \texttt{exactEnergy} im Benchmark und setzen der Flag
\texttt{useExactEnergy=true} das Experiment durch anschließendes Ausführen
von \texttt{startAlgorithmCR.m} gestartet werden.
