\section{Konstruktion eines Experiments mit exakter Lösung}
Um eine rechte Seite zu finden, zu der die exakte Lösung bekannt
ist, wähle eine Funktion des Radius $u\in H^1_0([0,1])$ mit Träger im 
zweidimensionalen Einheitskreis. Insbesondere muss damit gelten $u(1)=0$ und
$u$ stetig.
Die rechte Seite als Funktion des Radius $f\in L^2([0,1])$ ist dann gegeben
durch 
\begin{align*}
  f \coloneqq 
  \alpha u - \partial_r(\sign(\partial_r u)) - \frac{\sign(\partial_r u)}{r},
\end{align*}
wobei für $F\in\Rbb^2\setminus\{0\}$ gilt 
$\sign(F)\coloneqq \left\{\frac{F}{|F|}\right\}$ 
und $\sign(0)\in B_1(0)$.
Damit außerdem gilt $f\in H^1_0([0,1])$, was z.B.\ für GLEB relevant ist, 
muss also noch Stetigkeit von $\sign(\partial_r u)$ und 
$\partial_r(\sign(\partial_r u))$ verlangt werden und 
$\partial_r(\sign(\partial_r u(1))=\sign(\partial_r u(1))=0$.

Auf diese Weise erhält man die Experiment TODO

Dann berechnet man Gradienten von u und f indem TODO
Und dann berechnet man exakte Energie mit Funktion TODO indem TODO

Dann kann man das Programm nutzen.
