%Dies(easymotion-prefix)er Abschnitt folgt dem einführenden Kapitel von [braidesApprox], 
%mit der Einschränkung $\Omega \subset \Rbb^n$ offen und beschränkt, und für
%genauere Informationen sei an dieser Stelle darauf verwiesen.
%
%Die Familie $\Bcal(\Omega)$ aller Borelmengen und die Familie
%$\Bcal_C(\Omega)$ aller Borelmengen mit kompaktem Abschluss in
%$\Omega$ stimmen in diesem Fall überein. 

% TODO Maßtheorie für Vektormaße ist absolut nicht notwenig, da meine Anwendung
% ausschließlich von R^2 nach R geht
% Möglicherweise kann ich also doch eine durchgehende Geschichte erzählen und
% insbesondere alles verstehen

Dieser Abschnitt folgt Kapitel 10 von [bartels].
Dabei sei $\Omega \subset \Rbb^n$ ein offenes, polygonal berandetes
Lipschitz-Gebiet.

\section{Vorbereitung}

\begin{definition}[braides]\label{def:masz}
  Eine Funktion $\mu:\Bcal(\Omega)\to\Rbb^N$ heißt (Vektor-) Maß auf $\Omega$,
  wenn sie abzählbar additiv ist, d.h.\ für alle
  $(B_j)_{j\in\Nbb}\subseteq\Bcal(\Omega)$ mit $B_j \cap B_k = \emptyset$ für
  $j\neq k$ gilt
  \begin{align*}
    \mu\left( \dot{\bigcup_{j\in\Nbb}} B_j  \right) = \sum_{j\in\Nbb} \mu(B_j).
  \end{align*}
  Die Menge aller dieser Maße sei $\Mcal(\Omega;\Rbb^N)$.

  Im Fall $N=1$ heißt $\mu$ skalares Maß. Falls $\mu$ zusätzlich nur Werte in 
  $[0,\infty)$
  annimmt, heißt es positives Maß. Die Menge aller skalaren Maße sei 
  $\Mcal(\Omega)$ und die Menge aller positiven Maße sei 
  $\Mcal_+(\Omega)$.
\end{definition}

\begin{remark}
  Die übliche Anforderung $\mu(\emptyset) = 0$  an ein Maß ist äquivalent zur 
  Bedingung, dass $A\in \Bcal(\Omega)$ existiert, sodass $\mu(A)$ in jeder 
  Komponente endliche Werte annimmt. 

  Da in \Cref{def:masz} als Wertebereich $\Rbb^N$ gefordert wird, ist
  diese äquivalente Bedingung erfüllt.
\end{remark}

\begin{definition}\label{def:radonmasz}
  Eine Funktion $\mu:\Bcal(\Omega)\to\Rbb^N$ heißt Radonmaß auf 
  $\Omega$, wenn $\mu|_{\Bcal(\Omega')}$ ein Maß auf jeder Menge 
  $\Omega'\ssubset \Omega$ (d.h.\,$\closure({\Omega'})\subset \interior({\Omega})$) 
  ist.
\end{definition}

\section{Direkte Methode der Variationsrechnung}
\section{Subdifferential}
\section{Karush-Kuhn-Tucker Bedingungen}


\section{Funktionen Beschränkter Variation}

\begin{definition}[Funktionen beschränkter Variation]
  Eine Funktion $u\in L^1(\Omega)$ ist von beschränkter Variation, wenn ihre
  distributionelle Ableitung ein Radonmaß definiert, d.h.\ eine Konstante
  $c\geq 0$ existiert, sodass 
  \begin{align}
    \label{eq:boundedVariation}
    \langle Du,\Phi\rangle = - \int_\Omega u\Div (\phi) \dx 
    \leq c\Vert\phi\Vert_{L^\infty(\Omega)}
  \end{align}
  für alle $\phi\in C^1_C(\Omega;\Rbb^n)$.

  Die minimale Konstante $c\geq 0$, die \eqref{eq:boundedVariation} erfüllt,
  heißt totale Variation von $Du$ und besitzt die Darstellung
  \begin{align*}
    |u|_{\BV(\Omega)} = \sup_{\substack{\phi\in C^1_C(\Omega;\Rbb^n)\\
    \Vert\phi\Vert_{L^\infty(\Omega)}\leq 1}}-\int_\Omega u\Div (\phi)\dx.
  \end{align*}

  Der Raum aller Funktionen beschränkter Variation $\BV(\Omega)$
  ist ausgestattet mit der Norm 
  \begin{align*}
    \Vert u \Vert_{\BV(\Omega)} \coloneqq \Vert u\Vert_{L^1(\Omega)} +
    |u|_{\BV(\Omega)}.
  \end{align*}
\end{definition}

\begin{definition}
  Sei $(u_n)_{n\in\Nbb}\subset \BV(\Omega)$ und sei $u\in \BV(\Omega)$ mit
  $u_n\rightarrow u$ in $L^1(\Omega)$ für $n\rightarrow\infty$.
  \begin{itemize}
    \item[(i)]
      Die Folge $(u_n)_{n\in\Nbb}$ konvergiert strikt gegen $u$,
      wenn $|u_n|_{\BV(\Omega)}\rightarrow |u|_{\BV(\Omega)}$ für $n\rightarrow\infty$.
    \item[(ii)] Die Folge $(u_n)_{n\in\Nbb}$ konvergiert
      schwach gegen $u$, wenn
      $Du_n\rightharpoonup^\ast Du$ in 
      $\Mcal(\Omega;\Rbb^n)$ für $n\rightarrow\infty$, d.h.\ für alle
      $\phi\in C_0(\Omega;\Rbb^n)$ gilt 
      $\langle Du_n,\phi\rangle\rightarrow \langle Du,\phi\rangle$ für 
      $n\rightarrow\infty$.
  \end{itemize}
\end{definition}

\begin{theorem}[Schwache Unterhalbstetigkeit]
  \label{thm:wlsc}
  Seien $(u_n)_{n\in\Nbb}\subset\BV(\Omega)$ und $u\in L^1(\Omega)$ mit
  $|u_n|_{\BV(\Omega)}\leq c$ für ein $c>0$ und alle $n\in\Nbb$ und
  $u_n\rightarrow u$ in $L^1(\Omega)$ für $n\rightarrow\infty$.

  Dann gilt $u\in\BV(\Omega)$ und $|u|_{\BV(\Omega)}\leq
  \liminf_{n\rightarrow\infty}|u_n|_{\BV(\Omega)}.$
  Außerdem gilt $u_n\rightharpoonup u$ in $\BV(\Omega)$ für $n\rightarrow
  \infty$.
\end{theorem}

\begin{theorem}[Appoximation mit glatten Funktionen]
  \label{thm:approximationBySmoothFunctions}
  Die Räume $C^\infty(\overline\Omega)$ und $C^\infty(\Omega)\cap\BV(\Omega)$
  liegen dicht in $\BV(\Omega)$ bezüglich strikter Konvergenz.
\end{theorem}

\begin{theorem}
  \label{thm:compactness}
  Sei $(u_n)_{n\in\Nbb}\subset \BV(\Omega)$ eine beschränkte Folge. Dann 
  existiert eine Teilfolge $(u_{n_k})_{k\in\Nbb}$ und ein $u\in\BV(\Omega)$,
  sodass $u_{n_k}\rightharpoonup u$ in $\BV(\Omega)$ für $k\rightarrow\infty$.
\end{theorem}
