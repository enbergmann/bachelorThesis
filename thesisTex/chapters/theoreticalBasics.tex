%Dies(easymotion-prefix)er Abschnitt folgt dem einführenden Kapitel von [braidesApprox], 
%mit der Einschränkung $\Omega \subset \Rbb^n$ offen und beschränkt, und für
%genauere Informationen sei an dieser Stelle darauf verwiesen.
%
%Die Familie $\Bcal(\Omega)$ aller Borelmengen und die Familie
%$\Bcal_C(\Omega)$ aller Borelmengen mit kompaktem Abschluss in
%$\Omega$ stimmen in diesem Fall überein. 

\todo[inline]{TODO quote \cite[chapter 10, maybe also 4]{ABM14} for BV things,
e.g. BV is Banach space is proven there and also some wlsc statements, so 
quote it for many things instead of Bar15}
\section{Notation}
In dieser Arbeit sei $\Omega\subset\mathbb{R}^2$ stets ein polygonal-berandetes
Lipschitz-Gebiet.

\todo[inline]{TODO usw alle Notationen einführen, die in der ganzen Arbeit
gelten}
\todo[inline]{FRAGE Dann zB CR Notation erst im entsprechenden Kapitel 
einführen oder auch das schon hier? Oder hier nur ABSOLUTE Grundlagen, extrem 
basic, und alles andere dann on the fly?}

\todo[inline]{TODO
 nur die Sachen rausschreiben/zusammentragen/zitieren (um später Theoreme und
 Gleichungen zitieren zu können statt Bücher) die auch wirklich gebraucht 
 werden später in Beweisen. Insbesondere am Ende nochmal durchgucken, was 
 wirklich gebraucht wurde und ungebrauchtes und/oder uninteressanter und/oder
 unwichtiges rauswerfen}

\section{Maßtheoretische Grundlagen}
 
\todo[inline]{TODO doch noch wenigstens eine einfach Def für Radon Maße (vor
allem mit Zitat zu einer Quelle}

\todo[inline]{TODO Maßtheorie für Vektormaße ist absolut nicht notwenig, da
meine Anwendung ausschließlich von R2 nach R geht Möglicherweise kann ich also
doch eine durchgehende Geschichte erzählen und insbesondere alles verstehen,
im Quelltext sind noch auskommentierte Theorem zur Maßtheorie}

% \begin{definition}[braides]\label{def:masz}
%   Eine Funktion $\mu:\Bcal(\Omega)\to\Rbb^N$ heißt (Vektor-) Maß auf $\Omega$,
%   wenn sie abzählbar additiv ist, d.h.\ für alle
%   $(B_j)_{j\in\Nbb}\subseteq\Bcal(\Omega)$ mit $B_j \cap B_k = \emptyset$ für
%   $j\neq k$ gilt
%   \begin{align*}
%     \mu\left( \dot{\bigcup_{j\in\Nbb}} B_j  \right) = \sum_{j\in\Nbb} \mu(B_j).
%   \end{align*}
%   Die Menge aller dieser Maße sei $\Mcal(\Omega;\Rbb^N)$.
% 
%   Im Fall $N=1$ heißt $\mu$ skalares Maß. Falls $\mu$ zusätzlich nur Werte in 
%   $[0,\infty)$
%   annimmt, heißt es positives Maß. Die Menge aller skalaren Maße sei 
%   $\Mcal(\Omega)$ und die Menge aller positiven Maße sei 
%   $\Mcal_+(\Omega)$.
% \end{definition}
% 
% \begin{remark}
%   Die übliche Anforderung $\mu(\emptyset) = 0$  an ein Maß ist äquivalent zur 
%   Bedingung, dass $A\in \Bcal(\Omega)$ existiert, sodass $\mu(A)$ in jeder 
%   Komponente endliche Werte annimmt. 
% 
%   Da in \Cref{def:masz} als Wertebereich $\Rbb^N$ gefordert wird, ist
%   diese äquivalente Bedingung erfüllt.
% \end{remark}
% 
% \begin{definition}\label{def:radonmasz}
%   Eine Funktion $\mu:\Bcal(\Omega)\to\Rbb^N$ heißt Radonmaß auf 
%   $\Omega$, wenn $\mu|_{\Bcal(\Omega')}$ ein Maß auf jeder Menge 
%   $\Omega'\ssubset \Omega$ (d.h.\,$\closure({\Omega'})\subset \interior({\Omega})$) 
%   ist.
% \end{definition}

\section{Direkte Methode der Variationsrechnung}
\todo[inline]{braucht es eigentlich nicht, im Beweis selbst werden ja alle 
Argumente gebracht, also diese section ist überflüssig}

\section{Subdifferential}
In diesem Abschnitt trage ich die in dieser Arbeit benötigten Eigenschaften 
des Subdifferentials eines Funktionals $F:X\to [-\infty,\infty]$ 
auf einem Banachraum 
$(X,\Vert\bullet\Vert_X)$ und die dafür benötigten Begriffe zusammen.

Zunächst eine grundlegende Definition.

\begin{definition}[\protect{\cite[S. 245, Definition 42.1]{Zei85}}]
  Sei $X$ ein Vektorraum, $M\subseteq X$ und $F:M\to\Rbb$. 
  
  Dann heißt die Menge $M$ konvex, wenn für alle $u,v\in M$ und alle $t\in
  [0,1]$ gilt $(1-t)u+tv\in M$.

  Ist $M$ konvex, so heißt $F$ konvex, falls für alle $u,v\in M$ und alle
  $t\in[0,1]$ gilt $F\big( (1-t)u+tv\big)\leq (1-t)F(u)+t F(v).$
\end{definition}

In \cite{Zei85} werden einige der folgenden Aussagen auf reellen lokal konvexen
Räumen $X$ formuliert.
Da nach \cite[S. 781, (43)]{Zei86} alle Banachräume 
(in \cite{Zei86} und \cite{Zei85} genannt \glqq B-spaces\grqq, \cite[S.
786]{Zei86}) lokal
konvex sind und in dieser Arbeit die Aussagen nur auf Banachräumen benötigt
werden, beschränke ich
die folgenden Aussagen, falls nicht anders spezifiziert, wie folgt.
Sei $(X,\Vert\bullet\Vert_X)$ ein reeller Banachraum und
$F:X\to [-\infty,\infty]$ ein Funktional auf $X$.

\begin{definition}[Subdifferential, \protect{\cite[S. 385,
  Definition 47.8]{Zei85}}]
  Für $u\in X$ mit $F(u)\neq\pm\infty$ heißt
  \begin{equation}
    \label{equ:subdifferential}
    \partial F(u)\coloneq 
    \{u^\ast\in X^\ast\ |\ 
    \forall v\in X\quad F(v)\geq F(u)+\langle u^\ast,v-u\rangle\}  
  \end{equation}
  Subdifferential von $F$ an der Stelle $u$. Für $F(u)=\pm\infty$ definiere
  $\partial F(u)\coloneq\emptyset$.

  Ein Element $u^\ast\in\partial F(u)$ heißt Subgradient von $F$ an der Stelle
  $u$.
\end{definition}

\begin{theorem}[\protect{\cite[S. 387, Proposition 47.12]{Zei85}}]
  \label{thm:extremalprinzip}
  Falls $F: X\to (-\infty,\infty]$ mit $F\nequiv\infty$, gilt
  $F(u)=\inf_{v\in X}F(v)$ genau dann, wenn $0\in\partial F(u)$.
\end{theorem}

\begin{theorem}[\protect{\cite[S. 387, Proposition 47.13]{Zei85}}]
  \label{thm:subdiffGateaux}
  Falls $F$ konvex ist und G\^{a}teaux-differenzierbar
  (in \cite{Zei86} und \cite{Zei85} genannt \glqq G-differentiable\grqq, \cite[S.
  135f.]{Zei86})
  an der Stelle $u\in X$ mit G\^{a}teaux-Differential $F'(u)$,
  gilt $\partial F(u)=\{F'(u)\}$.
  \todo[inline]{TODO checke Notation und Definition mit der
  Gateaux/Frechet-differenzierbarkeit, für die ich mich entschieden habe (d.h.
  meint Zeidler das gleiche  
  
  Bemerke, die Prop liefert noch wann das umgekehrte gilt, aber nur
  aufschreiben, wenn das mal benötigt wird in dieser Arbeit

  nutze vielleicht Zeidler I als Quelle für die Differentiale und vielleicht
  auch Notation?}
\end{theorem}

Das folgende Theorem folgt aus \cite[S. 389, Theorem 47.B]{Zei85} unter 
Beachtung der Tatsache, dass die Addition von Funktionalen 
$F_1,F_2,\ldots,F_n:X\to (-\infty,\infty]$ und die Addition von
Menge in $X^\ast$ kommutieren.

\begin{theorem}
  \label{thm:subdifferentialSumRule}
  Seien für $n\geq 2$ die Funktionale $F_1,F_2,\ldots,F_n: X\to
  (-\infty,\infty]$ konvex und es existiere
  ein $u_0\in X$ und ein $j\in\{1,2,\ldots,n\}$ mit $F_k(u_0)<\infty$
  für alle $k\in\{1,2\ldots,n\}$, 
  sodass für alle $k\in\{1,2,\ldots,n\}\setminus\{j\}$ das Funktional
  $F_k$ stetig an der Stelle $u_0$ ist.

  Dann gilt 
  \begin{align*}
    \partial (F_1+F_2+\ldots+ F_n)(u) 
    = \partial F_1(u)+\partial F_2(u)+ \ldots + \partial F_n(u) \quad\text{für
    alle } u\in X.
  \end{align*}
\end{theorem}

\begin{theorem}[\protect{\cite[S. 396f., Definition 47.15, Theorem
  47.F]{Zei85}}]
  \label{thm:subdifferentialMonotonicity}
  Sei $F:X\to (-\infty,\infty]$ konvex und unterhalbstetig mit $F\nequiv\infty$.

  Dann ist $\partial F(\bullet)$ monoton, das heißt 
  \begin{align*}
    \langle u^\ast-v^\ast,u-v\rangle\geq 0\quad \text{für alle } u,v\in X, 
    u^\ast \in \partial F(u), v^\ast \in \partial F(v).
  \end{align*}
\end{theorem}

\section{Karush-Kuhn-Tucker Bedingungen}

\section{Funktionen Beschränkter Variation}

Dieser Abschnitt folgt Kapitel 10 von \cite{Bar15}.
Dabei sei $\Omega \subset \Rbb^n$ ein offenes, polygonal berandetes
Lipschitz-Gebiet.
\todo[inline]{Direkt von R2 ausgehen, weil mehr im Programm nicht geht?}
\todo[inline]{TODO jede übernommene Definition/Theorem/etc. zitieren trotz
Disclaimer oben?  jede Notation erklären bzw. definieren? Falls ja; am Anfang
oder Ende der Arbeit?}

\begin{definition}
  \todo{TODO vielleicht zu Grundlagen über Radonmaße verschieben}
  Die Vervollständigung des Raums $C^\infty_C(\Omega;\Rbb^m)$ bezüglich der 
  Norm
  $\Vert\bullet\Vert_{L^\infty(\Omega)}$ ist ein separabler Banachraum und wird
  bezeichnet mit 
  $C_0(\Omega; \Rbb^m)$.
  Der Dualraum $\Mcal(\Omega;\Rbb^m)$ von $C_0(\Omega; \Rbb^m)$ wird
  durch den Riesz'schen Darstellungssatz \todo{TODO zitiere?} identifiziert mit
  dem Raum aller (vektoriellen) Radonmaße. Dabei wird die Anwendung
  von $\mu\in \Mcal(\Omega;\Rbb^m)$
  auf $\phi\in C_0(\Omega;\Rbb^m)$ identifiziert mit
  \begin{align*}
    \langle \mu, \phi\rangle \coloneqq \int_\Omega \phi \dmu =
    \int_\Omega \phi(x) \dmu(x).
  \end{align*}
\end{definition}

\begin{definition}[Funktionen beschränkter Variation]
  Eine Funktion $u\in L^1(\Omega)$ ist von beschränkter Variation, wenn ihre
  distributionelle Ableitung ein Radonmaß definiert, d.h.\ eine Konstante
  $c\geq 0$ existiert, sodass 
  \begin{align}
    \label{eq:boundedVariation}
    \langle Du,\phi\rangle \coloneqq - \int_\Omega u\Div (\phi) \dx 
    \leq c\Vert\phi\Vert_{L^\infty(\Omega)}
  \end{align}
  für alle $\phi\in C^1_C(\Omega;\Rbb^n)$.

  Die minimale Konstante $c\geq 0$, die \eqref{eq:boundedVariation} erfüllt,
  heißt totale Variation von $Du$ und besitzt die Darstellung
  \begin{align*}
    |u|_{\BV(\Omega)} = \sup_{\substack{\phi\in C^1_C(\Omega;\Rbb^n)\\
    \Vert\phi\Vert_{L^\infty(\Omega)}\leq 1}}-\int_\Omega u\Div (\phi)\dx.
  \end{align*}

  Durch $|\bullet|_{\BV(\Omega)}$ ist eine Seminorm auf $\BV(\Omega)$
  gegeben.

  Der Raum aller Funktionen beschränkter Variation $\BV(\Omega)$
  ist ausgestattet mit der Norm 
  \begin{align*}
    \Vert u \Vert_{\BV(\Omega)} \coloneqq \Vert u\Vert_{L^1(\Omega)} +
    |u|_{\BV(\Omega)}
  \end{align*}
  für $u\in\BV(\Omega)$.
\end{definition}

\begin{remark}
  Es gilt $W^{1,1}(\Omega)\subset\BV(\Omega)$ und 
  $\Vert u \Vert_{\BV(\Omega)}=\Vert u \Vert_{W^{1,1}(\Omega)}$ für alle
  $u\in W^{1,1}(\Omega)$.
  {\color{red}es gilt für diese u tatsächlich (nach BV lecture04) ca.
  $|u|_{\BV(\Omega)=\Vert \nabla u \Vert_{L^1(\Omega)}}$}
\end{remark}

\begin{definition}
  Sei $(u_n)_{n\in\Nbb}\subset \BV(\Omega)$ und sei $u\in \BV(\Omega)$ mit
  $u_n\rightarrow u$ in $L^1(\Omega)$ für $n\rightarrow\infty$.
  \begin{itemize}
    \item[(i)]
      Die Folge $(u_n)_{n\in\Nbb}$ konvergiert strikt gegen $u$,
      wenn $|u_n|_{\BV(\Omega)}\rightarrow |u|_{\BV(\Omega)}$ für $n\rightarrow\infty$.
      {\color{red} strikte Konvergenz gdw. ($\Vert u-u_n\Vert_{L^1(\Omega)}
      \to 0$ und $|u_n|_{\BV(\Omega)}\rightarrow |u|_{\BV(\Omega)}$)
      was impliziert $\Vert u_n\Vert_\BV\to \Vert u\Vert_{\BV}$ aber nicht
      unbedingt $\Vert u_n - u \Vert_\BV\to 0$, da nicht folgt, dass 
      $|u_n - u|_\BV\to 0$
      
      aus BV Konvergenz, also $\Vert u_n - u \Vert_\BV\to 0$, folgt hingegen 
      aber ($\Vert u-u_n\Vert_{L^1(\Omega)}
      \to 0$ und $|u_n - u|_{\BV(\Omega)}\rightarrow 0$), also insbesondere
      ($\Vert u-u_n\Vert_{L^1(\Omega)}
      \to 0$ und $|u_n|_{\BV(\Omega)}\rightarrow |u|_{\BV(\Omega)}$), d.h.
      strikte Konvergenz
      
      jede BV konvergente Folge ist also strikt konvergent aber nicht umgekehrt,
      es gibt also mehr strikt konvergente Folgen, deshalb klingt es sinnvoll,
      dass wir BV Funktionen durch $C^\infty$ Funktionen (usw.) approximieren
      können bzgl strikter Konvergenz aber nicht bzgl BV Konvergenz (strong 
      topology, vgl.  Ende von BV lecture04}
    \item[(ii)] Die Folge $(u_n)_{n\in\Nbb}$ konvergiert
      schwach gegen $u$, wenn
      $Du_n\rightharpoonup^\ast Du$ in 
      $\Mcal(\Omega;\Rbb^n)$ für $n\rightarrow\infty$, d.h.\ für alle
      $\phi\in C_0(\Omega;\Rbb^n)$ gilt 
      $\langle Du_n,\phi\rangle\rightarrow \langle Du,\phi\rangle$ für 
      $n\rightarrow\infty$.
  \end{itemize}
\end{definition}

\begin{theorem}[Schwache Unterhalbstetigkeit]
  \label{thm:wlsc}
  Seien $(u_n)_{n\in\Nbb}\subset\BV(\Omega)$ und $u\in L^1(\Omega)$ mit
  $|u_n|_{\BV(\Omega)}\leq c$ für ein $c>0$ und alle $n\in\Nbb$ und
  $u_n\rightarrow u$ in $L^1(\Omega)$ für $n\rightarrow\infty$.

  Dann gilt $u\in\BV(\Omega)$ und $|u|_{\BV(\Omega)}\leq
  \liminf_{n\rightarrow\infty}|u_n|_{\BV(\Omega)}.$
  Außerdem gilt $u_n\rightharpoonup u$ in $\BV(\Omega)$ für $n\rightarrow
  \infty$.
\end{theorem}

\begin{theorem}[Appoximation mit glatten Funktionen]
  \label{thm:approximationBySmoothFunctions}
  Die Räume $C^\infty(\overline\Omega)$ und $C^\infty(\Omega)\cap\BV(\Omega)$
  liegen dicht in $\BV(\Omega)$ bezüglich strikter Konvergenz.

  {\color{red}BV lecture05 Thm 2.4 liefert sogar (Folge in
  $C^\infty\cap W^{1,1}$) sowohl strikte als auch schwache Konvergenz gegen
  gegebenes $u\in\BV$, also wir haben nach diesem Thm die Dichte von
  $C^\infty\cap W^{1,1}$ bzgl. strikter und schwacher BV Konvergenz
  
  da für Folgen in $W^{1,1}$ BV und $W^{1,1}$ Norm übereinstimmen und da
  $W^{1,1}$ der Abschluss von $C^\infty$ bzgl. der $W^{1,1}$ Norm ist (
  $C^\infty$ dicht in $W^{ 1,1 }$ bzgl. W11 Norm), ist
  für $W^{1,1}$ Funktionen W11 auch Abschluss von Cinfty bzgl
  der BV Norm (Cinfty dicht in W11 bzhl BV Norm)
  
  JETZT DER KNACKPUNKT und wie aus Thmeorem 2.6 gefolgert werden kann was in BV lecture
  steht: da BV Konvergenz strikte Konvergenz impliziert, ist 
  also Cinfty auch dicht in W11 bzgl strikter Konvergenz und W11 ist Teilmenge
  von BV. Da A dicht in B und B Teilmenge C impliziert das B dicht in C (wiki,
  natürlich beides bzgl gleicher Metrik), folgt insgesamt W11 dicht in BV bzgl
  strikter Konvergenz
  
  MORGEN CONTINUE IN WITH THIS IN EXISTENCE PROOF: NOW I might know WHY
  infimizing sequence in BV can simply be choosen in W11 or something
  (Think about it and what bartels did)}
\end{theorem}

\begin{theorem}
  \label{thm:compactness}
  Sei $(u_n)_{n\in\Nbb}\subset \BV(\Omega)$ eine beschränkte Folge. Dann 
  existiert eine Teilfolge $(u_{n_k})_{k\in\Nbb}$ und ein $u\in\BV(\Omega)$,
  sodass $u_{n_k}\rightharpoonup u$ in $\BV(\Omega)$ für $k\rightarrow\infty$.
  {\color{red}augenscheinlich nicht in BV lecture}
\end{theorem}

\begin{theorem}
  \label{thm:embeddingBVintoLp}
  Die Einbettung $\BV(\Omega)\to L^p(\Omega)$ ist stetig für 
  $1\leq p\leq n/(n-1)$ und kompakt für $1\leq p< n/(n-1)$
\end{theorem}\todo[inline]{TODO vielleicht wichtig, Quelle braucht es noch}

\begin{theorem}[Spuroperator]
  \label{thm:traceOperator}
  Es existiert ein linearer Operator $T:\BV(\Omega)\to L^1(\partial\Omega)$
  mit $T(u) = u|_{\partial\Omega}$ für alle $u\in\BV(\Omega)\cap
  C(\overline\Omega)$.

  Der Operator $T$ ist stetig bezüglich strikter Konvergenz in $\BV(\Omega)$,
  aber nicht stetig bezüglich schwacher Konvergenz in $\BV(\Omega)$. 
\end{theorem}\todo[inline]{TODO finde Quelle, nur ein Remark in Bartels (wird 
aber für CCs Funktional offensichtlich gebraucht, es gibt noch weitere Aussagen
in Bartels (zB integration by parts aber erstmal nur Existens und Stetigkeit 
hier (wie gesagt, nur was gebraucht wird zitieren, oder?)}
