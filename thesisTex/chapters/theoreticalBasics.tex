\section{Notation}
\todo[inline]{TODO usw alle Notationen einführen, die in der ganzen Arbeit
gelten, vergleiche Bringmann 2.1 Einleitung. Da wir aber erst sehr allgemein
sind bzw hier für Subdiffs und Variationsrechnung etc zwischen Räumen 
umhersprigen, dieses Kapitel doch lieber nicht machen sondern diese Themen (eben
theoretische Grundlagen) abarbeiten und dann immer weiter einschränken Kapitel
weise (continuous Problem schränkt Omega ein, dann discrete schränkt zu 2D ein
und führt CR ein, usw.) Notation hier also mglw einstampfen und on the fly 
machen}
\todo[inline]{FRAGE Dann zB CR Notation erst im entsprechenden Kapitel 
einführen oder auch das schon hier? Oder hier nur ABSOLUTE Grundlagen, extrem 
basic, und alles andere dann on the fly?}

\todo[inline]{TODO
 nur die Sachen rausschreiben/zusammentragen/zitieren (um später Theoreme und
 Gleichungen zitieren zu können statt Bücher) die auch wirklich gebraucht 
 werden später in Beweisen. Insbesondere am Ende nochmal durchgucken, was 
 wirklich gebraucht wurde und ungebrauchtes und/oder uninteressanter und/oder
 unwichtiges rauswerfen}


\section{Benötigte Begriffe der Variationsrechnung in Banachräumen}
\todo[inline]{im Zeidler lokal konvex Raum. Das irgendwie noch ausdrücken und
die Aussage, dass B-spaces lokal konvex sind, einmal rechtzeitig erwähnen.}

Die folgenden Aussagen basieren auf \cite[S. 189-192]{Zei85}.
Wir betrachten einen Banachraum $X$, eine Teilmenge $V\subseteq X$ und ein
Funktional $F:V\to\Rbb$. Sei $u$ ein innerer Punkt von $V$. Außerdem
definieren wir für $h\in X$ die Funktion $\varphi_h:\Rbb\to\Rbb$ durch 
$\varphi_h(t)\coloneqq F(u+th)$ für alle $t\in\Rbb$.

\begin{definition}[$n$-te Variation]
  Die $n$-te Variation von $F$ an der Stelle $u$ in Richtung $h\in X$ ist 
  \begin{align*}
    \delta^n F(u;h)\coloneqq \varphi_h^{(n)}(0)=
    \left. \frac{d^n F(u+th)}{dt^n}\right|_{t=0},
  \end{align*}
  falls $\varphi_h^{(n)}(0)$ existiert. Wir schreiben $\delta$ für $\delta^1$.
\end{definition}

\begin{definition}[G\^ateaux- und Fr\'echet-Differential]
  $F$ heißt G\^ateaux-differenzierbar an der Stelle $u$, falls ein 
  Funktional $F'(u)\in X^\ast$ existiert mit 
  \begin{align*}
    \lim_{t\to 0}\frac{F(u+th)-F(u)}{t} = \langle F'(u), h\rangle\quad
    \text{für alle } h\in X.
  \end{align*}
  $F'(u)$ heißt dann G\^ateaux-Differential von $F$ an der Stelle $u$.

  $F$ heißt Fr\'echet-differenzierbar an der Stelle $u$, falls ein Funktional
  $F'(u)\in X^\ast$ existiert, sodass
  \begin{align*}
    \lim_{\Vert h\Vert_X\to 0}\frac{|F(u+th)-F(u)-
    \langle F'(u),h\rangle|}{\Vert h\Vert_X} =0.
  \end{align*}
  $F'(u)$ heißt dann Fr\'echet-Differential von $F$ an der Stelle $u$.
  Das Fr\'echet-Differential von $F$ an der Stelle $u$ in Richtung $h\in X$
  ist definiert durch $dF(u;h)\coloneq \langle F'(u),h\rangle.$
\end{definition}

Die wichtigsten Eigenschaften sind noch einmal in folgender Bemerkung 
zusammengefasst.

\begin{remark}
  \begin{itemize}
    \item Existiert das Fr\'echet-Differential $F'(u)$ von $F$ an der Stelle
      $u$, so ist $F'(u)$ auch das G\^ateaux-Differential von $F$ an der Stelle
      $u$ und es gilt 
      \begin{align*}
        \delta F(u;h)=dF(u;h)=\langle F'(u),h\rangle\quad\text{für alle } h\in
        X.
      \end{align*}
%      Außerdem ist $F$ dann stetig an der Stelle $u$.
  \end{itemize}
\end{remark}

Damit können wir eine wichtige Aussage der Variationsrechnung formulieren,
basierend auf \cite[S. 193ff., Theorem 40.A, Theorem 40.B]{Zei85}.

\begin{theorem}[Notwendige Optimalitätsbedingung erster Ordnung]
  \label{thm:necessaryConditionFreeLocalExtrema}
  Sei $u\in \interior(V)$ lokaler Minimierer von $F$, das heißt
  es existiere eine Umgebung 
  \todo[inline]{Definiere Umgebung so wie in Zeilder 'neigborhood', 
  \cite[S. 751, (5)]{Zei86} (also es gibt eine offene Menge in der Umgebung,
  die den Punkt enthällt} 
  $U$ von $u$, so dass $F(v)\geq F(u)$ für alle $v\in U$. Dann gilt für alle
  $h\in X$, dass $\delta F(u;h) = 0$, falls diese Variation für alle $h\in X$
  existiert, beziehungsweise $F'(u) = 0$, falls $F'(u)$ als 
  G\^ateaux- oder Fr\'echet-Differential existiert.
\end{theorem}

\section{Subdifferential}
In diesem Abschnitt trage ich die in dieser Arbeit benötigten Eigenschaften 
des Subdifferentials eines Funktionals $F:X\to [-\infty,\infty]$ 
auf einem Banachraum 
$(X,\Vert\bullet\Vert_X)$ und die dafür benötigten Begriffe zusammen.

Zunächst eine grundlegende Definition.

\begin{definition}[\protect{\cite[S. 245, Definition 42.1]{Zei85}}]
  Sei $X$ ein Vektorraum, $M\subseteq X$ und $F:M\to\Rbb$. 
  
  Dann heißt die Menge $M$ konvex, wenn für alle $u,v\in M$ und alle $t\in
  [0,1]$ gilt $(1-t)u+tv\in M$.

  Ist $M$ konvex, so heißt $F$ konvex, falls für alle $u,v\in M$ und alle
  $t\in[0,1]$ gilt $F\big( (1-t)u+tv\big)\leq (1-t)F(u)+t F(v).$
\end{definition}

In \cite{Zei85} werden einige der folgenden Aussagen auf reellen lokal konvexen
Räumen $X$ formuliert.
Da nach \cite[S. 781, (43)]{Zei86} alle Banachräume 
(in \cite{Zei86} und \cite{Zei85} genannt \glqq B-spaces\grqq, \cite[S.
786]{Zei86}) lokal
konvex sind und in dieser Arbeit die Aussagen nur auf Banachräumen benötigt
werden, beschränke ich
die folgenden Aussagen, falls nicht anders spezifiziert, wie folgt.
Sei $(X,\Vert\bullet\Vert_X)$ ein reeller Banachraum und
$F:X\to [-\infty,\infty]$ ein Funktional auf $X$.

\begin{definition}[Subdifferential, \protect{\cite[S. 385,
  Definition 47.8]{Zei85}}]
  \label{def:subdifferential}
  Für $u\in X$ mit $F(u)\neq\pm\infty$ heißt
  \begin{equation}
    \label{eq:subdifferential}
    \partial F(u)\coloneq 
    \{u^\ast\in X^\ast\ |\ 
    \forall v\in X\quad F(v)\geq F(u)+\langle u^\ast,v-u\rangle\}  
  \end{equation}
  Subdifferential von $F$ an der Stelle $u$. Für $F(u)=\pm\infty$ definiere
  $\partial F(u)\coloneq\emptyset$.

  Ein Element $u^\ast\in\partial F(u)$ heißt Subgradient von $F$ an der Stelle
  $u$.
\end{definition}

\begin{theorem}[\protect{\cite[S. 387, Proposition 47.12]{Zei85}}]
  \label{thm:extremalprinciple}
  Falls $F: X\to (-\infty,\infty]$ mit $F\nequiv\infty$, gilt
  $F(u)=\inf_{v\in X}F(v)$ genau dann, wenn $0\in\partial F(u)$.
\end{theorem}

\begin{theorem}[\protect{\cite[S. 387, Proposition 47.13]{Zei85}}]
  \label{thm:subdiffGateaux}
  Falls $F$ konvex ist und G\^{a}teaux-differenzierbar
  (in \cite{Zei86} und \cite{Zei85} genannt \glqq G-differentiable\grqq, \cite[S.
  135f.]{Zei86})
  an der Stelle $u\in X$ mit G\^{a}teaux-Differential $F'(u)$,
  gilt $\partial F(u)=\{F'(u)\}$.
  \todo[inline]{TODO checke Notation und Definition mit der
  Gateaux/Frechet-differenzierbarkeit, für die ich mich entschieden habe (d.h.
  meint Zeidler das gleiche  
  
  Bemerke, die Prop liefert noch wann das umgekehrte gilt, aber nur
  aufschreiben, wenn das mal benötigt wird in dieser Arbeit

  nutze vielleicht Zeidler I als Quelle für die Differentiale und vielleicht
  auch Notation?}
\end{theorem}

Das folgende Theorem folgt aus \cite[S. 389, Theorem 47.B]{Zei85} unter 
Beachtung der Tatsache, dass die Addition von Funktionalen 
$F_1,F_2,\ldots,F_n:X\to (-\infty,\infty]$ und die Addition von
Menge in $X^\ast$ kommutieren.

\begin{theorem}
  \label{thm:subdifferentialSumRule}
  Seien für $n\geq 2$ die Funktionale $F_1,F_2,\ldots,F_n: X\to
  (-\infty,\infty]$ konvex und es existiere
  ein $u_0\in X$ und ein $j\in\{1,2,\ldots,n\}$ mit $F_k(u_0)<\infty$
  für alle $k\in\{1,2\ldots,n\}$, 
  sodass für alle $k\in\{1,2,\ldots,n\}\setminus\{j\}$ das Funktional
  $F_k$ stetig an der Stelle $u_0$ ist.

  Dann gilt 
  \begin{align*}
    \partial (F_1+F_2+\ldots+ F_n)(u) 
    = \partial F_1(u)+\partial F_2(u)+ \ldots + \partial F_n(u) \quad\text{für
    alle } u\in X.
  \end{align*}
\end{theorem}

\begin{theorem}[\protect{\cite[S. 396f., Definition 47.15, Theorem
  47.F]{Zei85}}]
  \label{thm:subdifferentialMonotonicity}
  Sei $F:X\to (-\infty,\infty]$ konvex und unterhalbstetig mit $F\nequiv\infty$.

  Dann ist $\partial F(\bullet)$ monoton, das heißt 
  \begin{align*}
    \langle u^\ast-v^\ast,u-v\rangle\geq 0\quad \text{für alle } u,v\in X, 
    u^\ast \in \partial F(u), v^\ast \in \partial F(v).
  \end{align*}
\end{theorem}


\section{Funktionen Beschränkter Variation}

Dieser Abschnitt präsentiert die für diese Arbeite benötigten Aussagen 
über Funktionen beschrankter Variation und basiert auf
Kapitel 10
in \cite{ABM14}.
\todo[inline]{Schreibe vlt noch sowas wie ,Für weitere Aussagen und Details
zu BV Funktionen und den maßtheoretischen Grundlagen dafür siehe [EG92,
Braides, ABM14]'}
Dabei sei $\Omega$ eine offene Teilmenge des $\Rbb^n$.
Der Raum aller $\Rbb^n$-wertigen Borelmaße wird bezeichnet mit
$M(\Omega;\Rbb^n)$ und ist nach Riesz identifizierbar mit dem Dualraum von
$C_0(\Omega;\Rbb^n)$ ausgestattet mit der Norm
$\Vert\phi\Vert_\infty\coloneqq
(\sum_{j=1}^n\sup_{x\in\Omega}|\phi_j(x)|^2)^{1/2}$ für $\phi\in
C_0(\Omega;\Rbb^n)$.

Die folgende Definition basiert auf \cite[S. 393 f.]{ABM14}.
\todo[inline]{Wenn einmal gesagt wurde worauf diese Section basiert, sind
dann noch Zitate wie dieses notwendig? Ich würde einfach nur noch zitieren,
wenn eine Aussage mal aus einer anderen Quelle kommt.}

\begin{definition}[Funktionen beschränkter Variation]
  Eine Funktion $u\in L^1(\Omega)$ ist von beschränkter Variation, wenn ihre
  distributionelle Ableitung $Du$ ein Element in $M(\Omega;\Rbb^n)$ definiert.
  Das ist äquivalent zu der Bedingung
  \begin{align}
    \label{eq:boundedVariation}
    |u|_{\BV(\Omega)}
    \coloneqq
    \sup_{\substack{\phi\in C^1_C(\Omega;\Rbb^n)\\
    \Vert\phi\Vert_{L^\infty(\Omega)}\leq 1}}\int_\Omega u\Div (\phi)\dx
    <
    \infty.
  \end{align}

  Durch $|\bullet|_{\BV(\Omega)}$ ist eine Seminorm auf $\BV(\Omega)$
  gegeben.

  Der Raum aller Funktionen beschränkter Variation $\BV(\Omega)$
  ist ausgestattet mit der Norm 
  \begin{align*}
    \Vert u \Vert_{\BV(\Omega)} \coloneqq \Vert u\Vert_{L^1(\Omega)} +
    |u|_{\BV(\Omega)}
  \end{align*}
  für $u\in\BV(\Omega)$.

  Nach \cite[S. 395, Theorem 10.1.1.]{ABM14} ist $(\BV(\Omega),
  \Vert\bullet\Vert_{\BV(\Omega)})$ ein Banachraum.
\end{definition}

\begin{remark}
  Es gilt $W^{1,1}(\Omega)\subset\BV(\Omega)$ und 
  $\Vert u \Vert_{\BV(\Omega)}=\Vert u \Vert_{W^{1,1}(\Omega)}$ für alle
  $u\in W^{1,1}(\Omega)$. (\cite[S. 394]{ABM14})
\end{remark}

\begin{definition}
  Sei $(u_n)_{n\in\Nbb}\subset \BV(\Omega)$ und sei $u\in \BV(\Omega)$ mit
  $u_n\rightarrow u$ in $L^1(\Omega)$ für $n\rightarrow\infty$.
  \begin{itemize}
    \item[(i)]
      Die Folge $(u_n)_{n\in\Nbb}$ konvergiert strikt gegen $u$,
      wenn $|u_n|_{\BV(\Omega)}\rightarrow |u|_{\BV(\Omega)}$ für
      $n\rightarrow\infty$
      (unter Beachtung von \cite[Remark 10.1.1]{ABM14}).
    \item[(ii)] Die Folge $(u_n)_{n\in\Nbb}$ konvergiert
      schwach gegen $u$, wenn
      $Du_n$ in 
      $M(\Omega;\Rbb^n)$ schwach gegen $Du$ konvergiert.
  \end{itemize}
\end{definition}

\begin{theorem}[Schwache Unterhalbstetigkeit, Prop. 10.1.1]
  \label{thm:wlsc}
  Sei $(u_n)_{n\in\Nbb}$ eine Folge in $\BV(\Omega)$ mit
  $\sup_{n\in\Nbb}|u_n|_{\BV(\Omega)}< \infty$ und $u\in L^1(\Omega)$ 
  mit $u_n\rightarrow u$ in $L^1(\Omega)$ für $n\rightarrow\infty$.

  Dann gilt $u\in\BV(\Omega)$ und $|u|_{\BV(\Omega)}\leq
  \liminf_{n\rightarrow\infty}|u_n|_{\BV(\Omega)}.$
  Außerdem konvergiert $u_n$ schwach gegen $u$ in $\BV(\Omega)$.
\end{theorem}

Mit Blick auf die folgenden Kapitel, sei $\Omega\subset\Rbb^n$ nun ein
polygonal berandetes Lipschitz-Gebiet.
\todo[inline]{Nochmal nachfragen: das heißt tatsächlich offen, beschränkt mit
Lipschitz Rand, korrekt?}
\todo[inline]{Nochmal nachfragen: $\sup u_k<\infty\Leftrightarrow u_k$ bounded,
korrekt? Ich übersehe da nichts, oder? Falls doch, alle Theoreme nochmal 
nachschlagen und sichergehen, dass sie richtig zitiert sind.}
\todo[inline]{Die BV Aussagen aus den nächsten Kapitel wieder hierher holen, soweit 
sinnvoll. Vielleicht auch nur die Spuroperator Aussage. Kann kopiert werden
aus delTexts.tex.}

\begin{theorem}[\protect{\cite[S. 176, Theorem 4]{EG92}}]
  \label{thm:l1ConvergentSubsequence}
  Sei $(u_n)_{n\in\Nbb}\subset \BV(\Omega)$ eine beschränkte Folge. Dann 
  existiert eine Teilfolge $(u_{n_k})_{k\in\Nbb}$ von
  $(u_n)_{n\in\Nbb}$ und ein $u\in\BV(\Omega)$, sodass
  $u_{n_k}\to u$ in $L^1(\Omega)$ für $k\to \infty$.
\end{theorem}
\todo[inline]{Das kann vielleciht auch noch im kontinuierlichen Existenzbeweis 
eingebracht werden und den etwas vereinfachen/verkürzen. Prüf das Zukunfts-Ich}

\begin{theorem}
  \label{thm:compactness}
  Sei $(u_n)_{n\in\Nbb}\subset \BV(\Omega)$ eine beschränkte Folge. Dann 
  existiert eine Teilfolge $(u_{n_k})_{k\in\Nbb}$ und ein $u\in\BV(\Omega)$,
  sodass $u_{n_k}\rightharpoonup u$ in $\BV(\Omega)$ für $k\rightarrow\infty$.
\end{theorem}

\begin{proof}
  Nach \Cref{thm:l1ConvergentSubsequence} besitzt $(u_n)_{n\in\Nbb}$ eine
  Teilfolge $(u_{n_k})_{k\in\Nbb}$, die in $L^1(\Omega)$ gegen ein
  $u\in\BV(\Omega)$ konvergiert.
  Diese Folge ist nach Voraussetzung ebenfalls beschränkt in 
  $\BV(\Omega)$, woraus nach Definition der Norm auf $\BV(\Omega)$ insbesondere
  folgt, dass
  $\sup_{k\in\Nbb}|u_{n_k}|_{\BV(\Omega)}< \infty$. 
  
  Insgesamt liefert \Cref{thm:wlsc} dann die schwache Konvergenz von
  $(u_{n_k})_{k\in\Nbb}$ in $\BV(\Omega)$ gegen $u\in\BV(\Omega)$.
\end{proof}
