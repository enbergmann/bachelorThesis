Dieser Abschnitt folgt dem einführenden Kapitel von [braidesApprox] und für
genauere Informationen sei an dieser Stelle darauf verwiesen.

\section{Vorbereitung}

\begin{definition}
  Sei $\Bcal(\Omega)$ die Familie aller Borelmengen in $\Omega$ und
  $\Bcal_C(\Omega)$ die Menge aller Borelmengen mit kompaktem Abschluss in
  $\Omega$. \\
  Eine Funktion $\mu:\Bcal(\Omega)\to\Rbb^N$ heißt (Vektor-) Maß auf $\Omega$,
  wenn sie abzählbar additiv ist, d.h. für $B\in \Bcal(\Omega)$ und
  $(B_j)_{j\in\Nbb}\subseteq\Bcal(\Omega)$ mit $B_j \cap B_k = \emptyset$ für
  $j\neq k$ gilt
  \begin{align*}
    B=\dot{\bigcup_{j\in\Nbb}} B_j \Rightarrow \mu(B) = \sum_{j\in\Nbb} \mu(B_j).
  \end{align*}
  Die Menge aller dieser Maße sei $\Mcal(\Omega;\Rbb^N)$.\\

  Im Fall $N=1$ heißt $\mu$ skalares Maß. Falls $\mu$ zusätzlich nur Werte in 
  $[0,\infty)$
  annimmt, heißt es positives Maß. Die Menge aller skalaren Maße sei 
  $\Mcal(\Omega)$ und die Menge aller positiven Maße sei 
  $\Mcal_+(\Omega)$.\\
  Eine Funktion $\mu:\Bcal_C(\Omega)\to\Rbb^N$ heißt Radonmaß auf 
  $\Omega$, wenn $\mu|_{\Bcal(\Omega')}$ ein Maß auf jeder Menge 
  $\Omega'\ssubset \Omega$ (d.h.\,$\closure({\Omega'})\subset \interior({\Omega})$) 
  ist. \\


    
\end{definition}

\section{Funktionen beschränkter Variation}
test
