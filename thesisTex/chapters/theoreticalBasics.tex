\section{Vorausgesetztes Wissen}
\label{sec:basicKnowledge}

In dieser Arbeit werden Grundbegriffe in topologischen Räumen und Kentnisse zu
Banach- und Hilberträumen sowie zu Lebesgue- und Sobolev-Räumen vorausgesetzt.
Dazu gehören insbesondere wichtige Ungleichungen (z.B.\ Cauchy-Schwarz, Hölder,
Young), grundlegende Einbettungssätze, Dualraumtheorie, Aussagen zur schwachen
Konvergenz sowie der Rieszsche Darstellungssatz und seine Implikationen.
Außerdem sollte ein grundlegendes Verständnis von Konzepten in der Optimierung
vorhanden sein.

Benötigte topologische Begriffe und grundlegende Aussagen zu Banach- und
Hilberträumen können beispielsweise in \cite{Zei86} nachgeschlagen werden.
Grundlagen der Optimierung sind in in \cite{Zei85} nachlesbar und alles Weitere
eben genannte in \cite{Zei90a} und \cite{Zei90b}. 

Dabei eignen sich die Register von \cite{Zei90b} und \cite{Zei85} hervorragend
zum schnellen Auffinden von Begriffen in allen eben aufgeführten
Referenzen.


\section{Notation}

Wir wählen für die natürlichen Zahlen die Konvention
$\Nbb=\{1,2,3,\ldots\}$ und $\Nbb_0=\Nbb\cup\{0\}$. 
Die Menge der positiven reellen Zahlen notieren wir mit $\Rbb_+$.

Um auszudrücken, dass eine Menge $A$ Teilmenge einer Menge $B$ ist, schreiben
wir $A\subseteq B$. Falls wir hervorheben wollen, dass $A$ sogar eine echte
Teilmenge von $B$ ist, so schreiben wir $A\subset B$.

Ist $A$ Teilmenge eines topologischen Raumes, so notieren wir den Rand von $A$
mit $\partial A$, das Innere von $A$ mit $\interior(A)$ und den Abschluss
von $A$ mit $\overline A$.
Wir nennen weiterhin eine Teilmenge $B\subseteq A$ Umgebung eines Punktes $x\in
A$, wenn es eine offene Teilmenge von $B$ gibt, die $x$ enthält.

Für den Rest dieses Abschnitts seien $d,m\in\Nbb$, $k\in\Nbb_0$,
$p\in[1,\infty]$ und $U$ eine nichtleere, offene Teilmenge von $\Rbb^d$.

Für das euklidische Skalarprodukt zweier Vektoren $v,w\in \Rbb^m$ schreiben wir
$v\cdot w$.

Betrachten wir einen Funktionenraum mit Werten in $\Rbb$, so verzichten wir auf
$\Rbb$ beim Notieren des Funktionenraums. Zum Beispiel schreiben wir 
$C(U)\coloneqq C(U;\Rbb)$ für den Raum der stetigen Funktionen von $U$ nach
$\Rbb$.


Mit der Menge $\Omega\subset\Rbb^d$ bezeichnen wir stets ein polygonal
berandetes Lipschitz-Gebiet. 
Falls $\Omega\subset\Rbb^2$, nutzen wir die Notation $\Tcal$ für eine reguläre
Triangulierung von $\Omega$ im Sinne von Philippe Ciarlet in eine Menge von
abgeschlossenen Dreiecken (cf.\ \cite[Kapitel 1.3.1]{Car09b}).
Die Menge der stückweisen konstanten Funktionen auf der Triangulierung $\Tcal$
mit Werten in $\Rbb^m$ notieren wir dann mit $\Pbb_0(\Tcal,\Rbb^m)$ und die
Menge der stückweisen affinen Funktionen auf $\Tcal$ mit Werten in $\Rbb^m$ mit
$\Pbb_1(\Tcal,\Rbb^m)$.
Für ein Dreieck $T\in\Tcal$ sei die längste Seitenlänge $h_T$.
Damit können wir die stückweise konstante Funktion $h_\Tcal\in\Pbb_0(\Tcal)$
für alle $T\in\Tcal$ durch $h_\Tcal|_T\coloneqq h_T$ definieren.
Die Menge der Knoten der Triangulierung sei $\Ncal$, wobei die Menge der
inneren Knoten mit $\Ncal(\Omega)$, die Menge der Randknoten mit
$\Ncal(\partial\Omega)$ und die Menge der Knoten eines Dreiecks $T\in\Tcal$ mit
$\Ncal(T)$ bezeichnet werde. 
Für die Kanten der Triangulierung seien die Mengen $\Ecal$, $\Ecal(\Omega)$,
$\Ecal(\partial\Omega)$ und $\Ecal(T)$ analog definiert. 
Den Mittelpunkt einer Kante $E\in\Fcal$ bezeichnen wir mit $\Mid(E)$.
Der Normaleneinheitsvektor auf dem Rand eines Dreiecks $T\in\Tcal$ sei
$\nu_T$ und der Normaleneinheitsvektor auf einer Kante $E\in\Ecal$ sei
$\nu_E$. 
Für eine Innenkante $E\in\Ecal(\Omega)$ bezeichnen wir dann die beiden
Dreiecke, die $E$ als gemeinsame Kanten haben, so mit $T_+$ und $T_-$, dass
$\nu_{T_+}$ und $\nu_E$ gleichorientiert sind, also $\nu_{T_+}\cdot\nu_E=1$.
Damit können wir den Sprung entlang einer Innenkante $E\in\Ecal(\Omega)$
definieren als $[\bullet]_E\coloneqq \bullet|_{T_+} -\bullet|_{T_-}$.
Für eine Randkante $E\in\Ecal(\partial\Omega)$ definieren wir
$[\bullet]_E\coloneqq \bullet|_E$.

Mit $|\bullet|$ bezeichnen wir, je nach Argument, die euklidischen Norm eines
Vektors $v\in\Rbb^m$, den Inhalt eines Dreiecks $T\in\Tcal$, die Länge
einer Kante $E\in\Ecal$ oder $|\Omega|=\int_\Omega 1\dx$.

Ist $V$ ein normierter Vektorraum, so bezeichnen wir die entsprechende Norm auf
$V$ mit $\Vert\bullet\Vert_V$. 
Falls $V$ sogar ein Prähilbertraum ist, bezeichnen wir das Skalarprodukt auf
$V$, welches $\Vert\bullet\Vert_V$ induziert, mit $(\bullet,\bullet)_V$.

Betrachten wir eine Eigenschaft, für die gegeben sein muss, bezüglich welcher
Norm Folgen auf $V$ konvergieren, so ist stets die Konvergenz in der Norm
$\Vert\bullet\Vert_V$ gewählt, sofern nicht anders angegeben.
Beispielsweise meinen wir mit der Folgenstetigkeit eines Funktionals
$F:V\to\Rbb$ im Detail die Folgenstetigkeit mit der Normkonvergenz
bezüglich $\Vert\bullet\Vert_V$ in $V$ und der Konvergenz bezüglich $|\bullet|$
in $\Rbb$.

Für den Dualraum eines Banachraums $X$ über $\Rbb$ 
schreiben wir $X^\ast$. Die Auswertung eines Funktionals $F\in X^\ast$ an
der Stelle $u\in X$ notieren wir, vor eventueller 
Anwendung des Rieszschen Darstellungssatzes, mit $\langle F,u\rangle$.
Identifizieren wir einen Raum $Y$ mit dem Dualraum $X^\ast$, so schreiben
wir $Y\cong X^\ast$.

Weiterhin benutzen wir die übliche Notation für Lebesgue-Räume
$L^p(U)$ und die So\-bo\-lev-Räume $W^{k,p}(U)$ sowie
$H^k(U)\coloneqq W^{k,2}(U)$ und $H^k_0(U)$. Die Normen auf diesen Räumen
definieren wir ebenfalls wie üblich.
Falls $p=2$, schreiben wir auch $\Vert\bullet\Vert \coloneqq
\Vert\bullet\Vert_{L^2(\Omega)}$
und $(\bullet,\bullet)\coloneqq(\bullet,\bullet)_{L^2(\Omega)}$.


\section{Variationsrechnung auf Banachräumen}
\label{sec:variationalCalculus}

Wie in \Cref{sec:basicKnowledge} beschrieben, setzen wir in dieser Arbeit
Kentnisse über die Variationsrechung voraus. 
Da wir aber einige Aussagen auch für Funktionale benötigen, die auf
unedlichdimensionalen reellen Banachräumen definiert sind, führen
wir in diesen Abschnitt die grundlegenden Notationen dafür ein und formulieren
schließlich die notwendige Optimalitätsbedingung erster Ordnung.

Dabei folgen wir \cite[S. 189-194]{Zei85}. \todo{noch sagen, dass Beweise da zu
finden sind oder ist das mit dem Satz schon klar?}
Dort werden die Aussagen auf einen reellen lokal konvexen
Raum formuliert. Da nach \cite[S. 781, (43)]{Zei86} alle Banachräume 
lokal konvex sind und wir die Aussagen in dieser Arbeit
nur auf Banachräumen benötigen, formulieren wir sie hier auf einen reellen 
Banachraum $X$. 
Außerdem betrachten wir eine Teilmenge $V\subseteq X$, einen
inneren Punkt $u$ von $V$ und ein Funktional $F:V\to\Rbb$. 
Schließlich definieren wir noch für alle $h\in X$ eine Funktion
$\varphi_h:\Rbb\to\Rbb$, die für alle $t\in\Rbb$ gegeben ist durch
$\varphi_h(t)\coloneqq F(u+th)$.

Damit können wir die $n$-te Variation, das G\^ateaux- und das 
Fr\'echet-Differential von $F$ definieren.

\begin{definition}[$n$-te Variation]
  Die $n$-te Variation von $F$ an der Stelle $u$ in Richtung $h\in X$ ist,
  falls die $n$-te Ableitung $\varphi_h^{(n)}(0)$ von $\varphi_h$ in $0$
  existiert, definiert durch 
  \begin{align*}
    \delta^n F(u;h)\coloneqq \varphi_h^{(n)}(0)=
    \left. \frac{d^n F(u+th)}{dt^n}\right|_{t=0}.
  \end{align*}

  Wir schreiben $\delta$ für $\delta^1$.
\end{definition}

\begin{definition}[G\^ateaux- und Fr\'echet-Differential]
  $F$ heißt G\^ateaux-differenzierbar an der Stelle $u$, falls ein 
  Funktional $F'(u)\in X^\ast$ existiert, sodass
  \begin{align*}
    \lim_{t\to 0}\frac{F(u+th)-F(u)}{t} = \langle F'(u), h\rangle\quad
    \text{für alle } h\in X.
  \end{align*}
  $F'(u)$ heißt dann G\^ateaux-Differential von $F$ an der Stelle $u$.

  $F$ heißt Fr\'echet-differenzierbar an der Stelle $u$, falls ein Funktional
  $F'(u)\in X^\ast$ existiert, sodass
  \begin{align*}
    \lim_{\Vert h\Vert_X\to 0}\frac{|F(u+th)-F(u)-
    \langle F'(u),h\rangle|}{\Vert h\Vert_X} =0.
  \end{align*}
  $F'(u)$ heißt dann Fr\'echet-Differential von $F$ an der Stelle $u$.
  Das Fr\'echet-Differential von $F$ an der Stelle $u$ in Richtung $h\in X$
  ist definiert durch $dF(u;h)\coloneqq \langle F'(u),h\rangle.$
\end{definition}

\begin{remark}
  Existiert das Fr\'echet-Differential $F'(u)$ von $F$ an der Stelle
    $u$, so ist $F'(u)$ auch das G\^ateaux-Differential von $F$ an der Stelle
    $u$ und es gilt 
    \begin{align*}
      \delta F(u;h)=dF(u;h)=\langle F'(u),h\rangle\quad\text{für alle } h\in X.
    \end{align*}
\end{remark}

Nachdem die relevante Notation eingeführt ist, können wir zum Abschluss
die notwendige Bedingung erster Ordnung für einen lokalen Minimierer von $F$
formulieren.

\begin{theorem}[Notwendige Optimalitätsbedingung erster Ordnung]
  \label{thm:necessaryConditionFreeLocalExtrema}
  Sei $u\in \interior(V)$ lokaler Minimierer von $F$, das heißt
  es existiere eine Umgebung 
  $U$ von $u$, sodass $F(v)\geq F(u)$ für alle $v\in U$. Dann gilt für alle
  $h\in X$, dass $\delta F(u;h) = 0$, falls diese Variation für alle $h\in X$
  existiert, beziehungsweise $F'(u) = 0$, falls $F'(u)$ als 
  G\^ateaux- oder Fr\'echet-Differential existiert.
\end{theorem}

\section{Subdifferentiale}

Für diesen Abschnitt betrachten wir stets einen reellen Banachraum $X$ und,
falls nicht anders spezifiziert, ein Funktional $F:X\to [-\infty,\infty]$.

Wir wollen die in dieser Arbeit benötigten Notationen und
Eigenschaften des Subdifferentials von $F$ zusammentragen.

Zuvor starten wir mit einer grundlegenden Definition.

\begin{definition}[\protect{\cite[S. 245, Definition 42.1]{Zei85}}]
  Sei $V$ ein Vektorraum, $M\subseteq V$ und $F:M\to\Rbb$. 
  
  Dann heißt die Menge $M$ konvex, wenn für alle $u,v\in M$ und alle $t\in
  [0,1]$ gilt $$(1-t)u+tv\in M.$$

  Ist $M$ konvex, so heißt $F$ konvex, falls für alle $u,v\in M$ und alle
  $t\in[0,1]$ gilt 
  \begin{equation}
    \label{eq:convexity}
    F\big( (1-t)u+tv\big)\leq (1-t)F(u)+t F(v).
  \end{equation}
  Gilt Ungleichung \eqref{eq:convexity} mit \glqq$<$\grqq, so heißt
  $F$ strikt konvex. Falls $-F$ konvex ist, so heißt $F$ konkav.
\end{definition}

Für den Rest dieses Abschnitts folgen wir \cite[S. 385-397]{Zei85}. \todo{noch
sagen, dass Beweise da zu finden sind oder ist das mit dem Satz schon klar? 
Besser sowas sagen wie 'Die Aussagen stammen aus'?}
Analog zur Begründung zum Beginn von \Cref{sec:variationalCalculus}, schränken
wir auch hier die Definitionen und Aussagen, die in \cite{Zei85} auf reellen
lokal konvexen Räumen formuliert sind, auf den reellen Banachraum $X$ ein.

Zunächst definieren wir das Subdifferential von $F$ an einer Stelle $u\in X$.

\begin{definition}[Subdifferential]
  \label{def:subdifferential}
  Für $u\in X$ mit $F(u)\neq\pm\infty$ heißt
  \begin{equation}
    \label{eq:subdifferential}
    \partial F(u)\coloneqq
    \{u^\ast\in X^\ast\ |\ 
    \forall v\in X\quad F(v)\geq F(u)+\langle u^\ast,v-u\rangle\}  
  \end{equation}
  Subdifferential von $F$ an der Stelle $u$. Für $F(u)=\pm\infty$ ist
  $\partial F(u)\coloneq\emptyset$.

  Ein Element $u^\ast\in\partial F(u)$ heißt Subgradient von $F$ an der Stelle
  $u$.
\end{definition}

Es folgen für Optimierungsprobleme wichtige Aussagen über das Subdifferential 
von $F$.

\begin{theorem}
  \label{thm:extremalprinciple}
  Falls $F: X\to (-\infty,\infty]$ mit $F\nequiv\infty$, gilt
  $F(u)=\inf_{v\in X}F(v)$ genau dann, wenn $0\in\partial F(u)$.
\end{theorem}

\begin{theorem}
  \label{thm:subdiffGateaux}
  Falls $F$ konvex ist und G\^{a}teaux-differenzierbar
  an der Stelle $u\in X$ mit G\^{a}teaux-Differential $F'(u)$,
  gilt $\partial F(u)=\{F'(u)\}$.
\end{theorem}

Das folgende Theorem folgt aus \cite[S. 389, Theorem 47.B]{Zei85} unter 
Beachtung der Tatsache, dass die Addition von Funktionalen 
$F_1,F_2,\ldots,F_n:X\to (-\infty,\infty]$ und die Addition von
Menge in $X^\ast$ kommutieren.

\begin{theorem}
  \label{thm:subdifferentialSumRule}
  Seien für $n\geq 2$ die Funktionale $F_1,F_2,\ldots,F_n: X\to
  (-\infty,\infty]$ konvex und es existiere
  ein $u_0\in X$, sodass $F_k(u_0)<\infty$
  für alle $k\in\{1,2\ldots,n\}$. 
  Außerdem seien mindestens $n-1$ der $n$ Funktionale $F_1,F_2,\ldots,F_n$
  stetig an der Stelle $u_0$.

  Dann gilt 
  \begin{align*}
    \partial (F_1+F_2+\ldots+ F_n)(u) 
    = \partial F_1(u)+\partial F_2(u)+ \ldots + \partial F_n(u) \quad\text{für
    alle } u\in X.
  \end{align*}
\end{theorem}

Zum Abschluss formulieren wir noch die Monotonie des Subdifferentials.

\begin{theorem}
  \label{thm:subdifferentialMonotonicity}
  Sei $F:X\to (-\infty,\infty]$ konvex und unterhalbstetig mit $F\nequiv\infty$.

  Dann ist $\partial F(\bullet)$ monoton, das heißt 
  \begin{align*}
    \langle u^\ast-v^\ast,u-v\rangle\geq 0\quad \text{für alle } u,v\in X, 
    u^\ast \in \partial F(u), v^\ast \in \partial F(v).
  \end{align*}
\end{theorem}


\section{Funktionen beschränkter Variation}
\label{sec:bvFunctions}

In diesen Abschnitt führen wir den Raum der Funktionen beschränkter Variation
ein.
Wir vermeiden dabei, soweit möglich, für den weiteren Verlauf dieser Arbeit
nicht benötigte Notation und Theorie, indem wir die Definitionen und Aussagen 
entsprechend aus- und umformulieren.
Für weit detailliertere Ausführungen und die
maßtheoretischen Hintergründe siehe zum Beispiel \cite{ABM14, EG92, Bra98}. 

Soweit nicht anders angegeben, folgen die Definitionen und Aussagen
dieses Abschnitts aus \cite[S. 393-395]{ABM14}
\todo{noch sagen, dass Beweise da zu finden sind oder ist das mit dem Satz
schon klar? Besser sowas sagen wie 'Die Aussagen stammen aus'?}

Sei im Weiteren $U$ eine offene Teilmenge des $\Rbb^d$.

Zunächst definieren wir den Raum der Funktionen beschränkter Variation.

\begin{definition}[Funktionen beschränkter Variation]
  Eine Funktion $u\in L^1(U)$ ist von beschränkter Variation, wenn   
  \begin{align}
    \label{eq:boundedVariation}
    |u|_{\BV(U)}
    \coloneqq
    \sup_{\substack{\phi\in C^1_C(U;\Rbb^d)\\
    \Vert\phi\Vert_{L^\infty(U)}\leq 1}}\int_U u\Div (\phi)\dx
    <
    \infty.
  \end{align}
  Die Menge aller Funktionen beschränkter Variation ist $\BV(U)$.
\end{definition}

\begin{remark}
  Durch $|\bullet|_{\BV(U)}$ ist eine Seminorm auf $\BV(U)$
  gegeben.

  Ausgestattet mit der Norm
  \begin{align*}
    \Vert \bullet \Vert_{\BV(U)} \coloneqq \Vert \bullet\Vert_{L^1(U)} +
    |\bullet|_{\BV(U)}
  \end{align*}
  ist $\BV(U)$ ein Banachraum.

  Außerdem gilt $W^{1,1}(U)\subset\BV(U)$ und 
  $\Vert u \Vert_{\BV(U)}=\Vert u \Vert_{W^{1,1}(U)}$ für alle
  $u\in W^{1,1}(U)$.
\end{remark}

In der Anwendung ist Konvergenz in $\BV(U)$ bezüglich der Norm
$\Vert\bullet\Vert_{\BV(U)}$ zu restriktiv (cf.\ \cite[300]{Bar15}). 
Deshalb führen wir zwei schwächere Konvergenzbegriffe ein.

\begin{definition}
  Sei $(u_n)_{n\in\Nbb}\subset \BV(U)$ und sei $u\in \BV(U)$ mit
  $u_n\rightarrow u$ in $L^1(U)$ für $n\rightarrow\infty$.
  \begin{itemize}
    \item[(i)]
      Die Folge $(u_n)_{n\in\Nbb}$ konvergiert strikt gegen $u$,
      wenn $|u_n|_{\BV(U)}\rightarrow |u|_{\BV(U)}$ für
      $n\rightarrow\infty$. 
      \todo[inline]{Nochmal gucken, das wird glaube ich nirgends gebraucht.
      Falls das stimmt, rausnehmen. Wird für Dichte von $C^\infty$ usw in $BV$
      gebraucht bzgl dieser Konvergenz. Gucken, ob das relevant wird. Falls
      es rausgenomme wird, Text vor und nach dieser Def noch anpassen}
    \item[(ii)] Die Folge $(u_n)_{n\in\Nbb}$ konvergiert
      schwach gegen $u$, wenn für alle $\phi\in C_0(U;\Rbb^d)$ gilt, dass
      $\int_U u_n\Div(\phi)\dx\rightarrow \int_U u\Div(\phi)\dx$ für
      $n\to\infty$. Wir schreiben dann $u_n\rightharpoonup u$ für $n\to\infty$.
  \end{itemize}
\end{definition}

Damit können wir das folgende Theorem formulieren, welches unmittelbar die
schwache Unterhalbstetigkeit der Seminorm $|\bullet|_{\BV(U)}$ auf $\BV(U)$
impliziert.

\begin{theorem}
  \label{thm:wlsc}
  Sei $u\in L^1(U)$ und sei $(u_n)_{n\in\Nbb}\subset\BV(U)$ mit
  $\sup_{n\in\Nbb}|u_n|_{\BV(U)}< \infty$ und
  $u_n\rightarrow u$ in $L^1(U)$ für $n\rightarrow\infty$.

  Dann gilt $u\in\BV(U)$ und $|u|_{\BV(U)}\leq
  \liminf_{n\rightarrow\infty}|u_n|_{\BV(U)}.$
  Außerdem gilt dann $u_n\rightharpoonup u$ in $\BV(U)$.
\end{theorem}

Mit Blick auf die folgenden Kapitel betrachten wir nun
das polygonal berandete Lip\-schitz-Gebiet $\Omega\subset\Rbb^d$. 
Unter dieser Voraussetzung können wir zeigen, dass jede in $\BV(\Omega)$
beschränkte Folge eine in $\BV(\Omega)$ schwach konvergente Teilfolge besitzt
mit schwachen Grenzwert in $\BV(\Omega)$. Für den Beweis dieser Aussage 
benötigen wir noch das folgende Theorem aus \cite[S. 176, Theorem 4]{EG92}.

\todo[inline]{Nochmal nachfragen: das heißt tatsächlich offen, beschränkt mit
Lipschitz Rand, korrekt?}
\todo[inline]{Nochmal nachfragen: $\sup u_k<\infty\Leftrightarrow u_k$ bounded,
korrekt? Ich übersehe da nichts, oder? Falls doch, alle Theoreme nochmal 
nachschlagen und sichergehen, dass sie richtig zitiert sind.}

\begin{theorem}
  \label{thm:l1ConvergentSubsequence}
  Sei $(u_n)_{n\in\Nbb}\subset \BV(\Omega)$ eine beschränkte Folge. Dann 
  existiert eine Teilfolge $(u_{n_k})_{k\in\Nbb}$ von
  $(u_n)_{n\in\Nbb}$ und ein $u\in\BV(\Omega)$, sodass
  $u_{n_k}\to u$ in $L^1(\Omega)$ für $k\to \infty$.
\end{theorem}

Damit können wir nun das folgende Theorem beweisen.
\begin{theorem}
  \label{thm:compactness}
  Sei $(u_n)_{n\in\Nbb}\subset \BV(\Omega)$ eine beschränkte Folge. Dann 
  existiert eine Teilfolge $(u_{n_k})_{k\in\Nbb}$ und ein $u\in\BV(\Omega)$,
  sodass $u_{n_k}\rightharpoonup u$ in $\BV(\Omega)$ für $k\rightarrow\infty$.
\end{theorem}

\begin{proof}
  Nach \Cref{thm:l1ConvergentSubsequence} besitzt $(u_n)_{n\in\Nbb}$ eine
  Teilfolge $(u_{n_k})_{k\in\Nbb}$, die in $L^1(\Omega)$ gegen ein
  $u\in\BV(\Omega)$ konvergiert.
  Diese Teilfolge ist nach Voraussetzung ebenfalls beschränkt in 
  $\BV(\Omega)$, woraus nach Definition der Norm auf $\BV(\Omega)$ insbesondere
  folgt, dass
  $\sup_{k\in\Nbb}|u_{n_k}|_{\BV(\Omega)}< \infty$. 
  
  Somit ist \Cref{thm:wlsc} anwendbar und impliziert die schwache Konvergenz von
  $(u_{n_k})_{k\in\Nbb}$ in $\BV(\Omega)$ gegen $u\in\BV(\Omega)$.
\end{proof}
