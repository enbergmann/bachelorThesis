In dieser Arbeit werden Grundbegriffe in topologischen Räumen und Kenntnisse zu
Ba\-nach- und Hilberträumen sowie zu Lebesgue- und Sobolev-Räumen vorausgesetzt.
Dazu gehören insbesondere wichtige Ungleichungen (beispielsweise
Cauchy-Schwarz, Hölder, Young), grundlegende Einbettungssätze, Dualraumtheorie,
Aussagen zur schwachen Konvergenz sowie der Rieszsche Darstellungssatz und
seine Implikationen.
Benötigte topologische Begriffe und grundlegende Aussagen zu Banach- und
Hilberträumen können beispielsweise in \cite{Zei86} nachgeschlagen werden.
Grundlagen der Optimierung sind in \cite{Zei85} nachlesbar und alles Weitere
eben genannte in \cite{Zei90a} und \cite{Zei90b}. 
Dabei eignen sich die Register von \cite{Zei90b} und \cite{Zei85} hervorragend
zum schnellen Auffinden von Begriffen in allen eben aufgeführten
Referenzen.

Schließlich sollte ein Verständnis des AFEM-Softwarepakets \cite{Car09}, von
den darin verwendeten Datenstrukturen und von den mathematischen Konzepten
hinter dessen Funktionen vorhanden sein. 
Dazu sei an dieser Stelle auf die Dokumentation \cite{CGKNRR10} dieses
Softwarepakets verwiesen.
Die in dieser Arbeit benötigten Begriffe und Notationen führen wir im nächsten
Abschnitt nochmals ein, wobei wir den Definitionen und Aussagen zum
Crouzeix-Raviart-Finite-Elemente-Raum einen separaten
\Cref{sec:crouzeixRaviartFunctions} widmen.


\section{Notation}
\label{sec:notation}

Wir wählen für die natürlichen Zahlen die Konvention
$\Nbb=\{1,2,3,\ldots\}$ und $\Nbb_0\coloneqq\Nbb\cup\{0\}$. 
Die Menge der positiven reellen Zahlen notieren wir mit $\Rbb_+$.
Um auszudrücken, dass eine Menge $A$ Teilmenge einer Menge $B$ ist, schreiben
wir $A\subseteq B$. Falls wir hervorheben wollen, dass $A$ sogar eine echte
Teilmenge von $B$ ist, so schreiben wir $A\subset B$.
Ist $A$ Teilmenge eines topologischen Raumes, so notieren wir den Rand von $A$
mit $\partial A$, das Innere von $A$ mit $\interior(A)$ und den Abschluss
von $A$ mit $\overline A$.
Wir nennen weiterhin eine Teilmenge $B\subseteq A$ Umgebung eines Punktes $x\in
A$, wenn es eine offene Teilmenge von $B$ gibt, die $x$ enthält.
Ist eine Funktion $F:X\to Y$ zwischen nichtleeren Mengen $X$ und $Y$ konstant,
das heißt nimmt sie nur genau einen Wert $y\in Y$ an, schreiben wir $F\equiv
y$.

Für den Rest dieses Abschnitts seien $d,m\in\Nbb$, $k\in\Nbb_0$,
$p\in[1,\infty]$ und $U$ eine nichtleere, offene Teilmenge von $\Rbb^d$.
Die Einheitsmatrix in $\Rbb^{m\times m}$ bezeichnen wir mit $I_m$.
Für das euklidische Skalarprodukt zweier Vektoren $v,w\in \Rbb^m$ schreiben wir
$v\cdot w$.
Betrachten wir einen Funktionenraum mit Werten in $\Rbb$, so verzichten wir auf
$\Rbb$ beim Notieren des Funktionenraums. Zum Beispiel schreiben wir 
$C(U)\coloneqq C(U;\Rbb)$ für den Raum der stetigen Funktionen von $U$ nach
$\Rbb$.

Mit $\Omega\subset\Rbb^2$ bezeichnen wir stets ein polygonal berandetes
Lipschitz-Gebiet. 
Eine reguläre Triangulierung $\Tcal$ von $\Omega$ im Sinne von Ciarlet (cf.\
\cites[34]{Car09b}[345]{CGR12}[S. 8 f.]{CGKNRR10}) ist eine endliche Menge von
abgeschlossenen Dreiecken $T$ mit positiven Flächeninhalt $|T|$, sodass
\begin{align*}
  \bigcup_{T\in\Tcal} T=\overline{\Omega}
\end{align*}
und zwei Dreiecke $T_1,T_2\in\Tcal$ mit $T_1\neq T_2$ und $T_1\cap
T_2\neq\emptyset$ genau eine gemeinsame Ecke oder genau eine gemeinsame Kante
haben.
Die Menge der Knoten der Triangulierung sei $\Ncal$, wobei die Menge der
inneren Knoten mit $\Ncal(\Omega)$, die Menge der Randknoten mit
$\Ncal(\partial\Omega)$ und die Menge der Knoten eines Dreiecks $T\in\Tcal$ mit
$\Ncal(T)$ bezeichnet werde. 
Für die Kanten der Triangulierung seien die Mengen $\Ecal$, $\Ecal(\Omega)$,
$\Ecal(\partial\Omega)$ und $\Ecal(T)$ analog definiert.  
Außerdem definieren wir für $z\in\Ncal$ die Menge $\Tcal(z)\coloneqq
\{T\in\Tcal\,|\,z\in\Ncal(T)\}$ und für $E\in\Ecal$ die Menge
$\Tcal(E)\coloneqq \{T\in\Tcal\,|\,E\in\Ecal(T)\}$. 
Den Mittelpunkt einer Kante $E\in\Ecal$ bezeichnen wir mit $\Mid(E)$.
Der Normaleneinheitsvektor auf dem Rand eines Dreiecks $T\in\Tcal$ sei
$\nu_T$ und der Normaleneinheitsvektor auf einer Kante $E\in\Ecal$ sei
$\nu_E$. 
Für eine Innenkante $E\in\Ecal(\Omega)$ bezeichnen wir dann die beiden
Dreiecke in $\Tcal(E)$  so mit $T_+$ und $T_-$, dass
$\nu_{T_+}$ und $\nu_E$ gleich orientiert sind, also $\nu_{T_+}\cdot\nu_E=1$.
Damit können wir den Sprung entlang einer Innenkante $E\in\Ecal(\Omega)$
definieren als $[\bullet]_E\coloneqq \bullet|_{T_+} -\bullet|_{T_-}$.
Für eine Randkante $E\in\Ecal(\partial\Omega)$ definieren wir
$[\bullet]_E\coloneqq \bullet|_E$.
Die Menge der stückweise konstanten Funktionen auf der Triangulierung $\Tcal$
mit Werten in $\Rbb^m$ notieren wir dann mit $ P_0\!\left(\Tcal,\Rbb^m\right)$
und die Menge der stückweise affinen Funktionen auf $\Tcal$ mit Werten in
$\Rbb^m$ mit $ P_1\!\left(\Tcal,\Rbb^m\right)$. 
Weiterhin ist der Courant-Finite-Elemente-Raum (cf. \cite[12]{CGKNRR10})
definiert als
\begin{align*}
  S^1(\Tcal)\coloneqq P_1(\Tcal)\cap C\!\left(\overline{\Omega}\right).
\end{align*}
Für ein Dreieck $T\in\Tcal$ sei die Länge der längsten Seite $h_T$. 
Damit können wir die stückweise konstante Funktion $h_\Tcal\in P_0(\Tcal)$ für
alle $T\in\Tcal$ durch $h_\Tcal|_T\coloneqq h_T$ und die Länge der längsten
Seite der Triangulierung durch $h\coloneqq\max_{T\in\Tcal}h_T$ definieren.

Mit $|\bullet|$ bezeichnen wir, je nach Argument, die euklidischen Norm eines
Vektors $v\in\Rbb^m$, das Lebesgue-Maß einer Menge $M\subset\Rbb^2$, die Länge
einer Kante $E\in\Ecal$ oder die Kardinalität einer endlichen Menge $A$.

Ist $V$ ein Vektorraum, so notieren wir die konvexe Hülle einer Teilmenge
$X\subseteq V$ mit $\conv\,X$.
Falls $V$ ein normierter Vektorraum ist, dann bezeichnen wir die entsprechende
Norm auf $V$ mit $\Vert\bullet\Vert_V$. 
Die Einheitskugel auf $V$ ist damit $B_V\coloneqq\{v\in V\,|\, \Vert v\Vert_V<
1\}$ und die abgeschlossene Einheitskugel auf $V$ ist
$\overline{B_V}=\{v\in V\,|\, \Vert v\Vert_V\leq 1\}$.
Ist $V$ sogar ein Prähilbertraum, so bezeichnen wir das Skalarprodukt auf
$V$, welches $\Vert\bullet\Vert_V$ induziert, mit $(\bullet,\bullet)_V$.
Betrachten wir eine Eigenschaft, für die gegeben sein muss, bezüglich welcher
Norm Folgen auf $V$ konvergieren, so ist stets die Konvergenz in der Norm
$\Vert\bullet\Vert_V$ gewählt, sofern nicht anders angegeben.
Beispielsweise meinen wir mit der Folgenstetigkeit eines Funktionals
$F:V\to\Rbb$ im Detail die Folgenstetigkeit mit der Normkonvergenz
bezüglich $\Vert\bullet\Vert_V$ in $V$ und der Konvergenz bezüglich $|\bullet|$
in $\Rbb$.

Die Signumfunktion auf dem $\Rbb^m$ definieren wir für einen Vektor
$v\in\Rbb^m$ durch
\begin{align}
  \label{eq:signumFunction}
  \sign(v)\coloneqq
  \begin{cases}
    \left\{\frac{v}{|v|}\right\},&\text{ falls }v\in\Rbb^m\setminus\{0\},\\
    \overline{B_{\Rbb^m}},&\text{ sonst.}
  \end{cases}
\end{align}

Für den Dualraum eines Banachraums $X$ über $\Rbb$ 
schreiben wir $X^\ast$. Die Auswertung eines Funktionals $F\in X^\ast$ an
der Stelle $u\in X$ notieren wir, vor eventueller 
Anwendung des Rieszschen Darstellungssatzes, mit $\langle F,u\rangle$.
Identifizieren wir einen Raum $Y$ mit dem Dualraum $X^\ast$, so schreiben
wir $Y\cong X^\ast$.

Weiterhin benutzen wir die übliche Notation für Lebesgue-Räume
$L^p\!\left(U;\Rbb^m\right)$ und die So\-bo\-lev-Räume $W^{k,p}(U)$ sowie
$H^k(U)\coloneqq W^{k,2}(U)$ und $H^k_0(U)$. Die Normen auf diesen Räumen
definieren wir ebenfalls wie üblich.
Außerdem schreiben wir kurz $\Vert\bullet\Vert \coloneqq
\Vert\bullet\Vert_{L^2(\Omega)}$
und $(\bullet,\bullet)\coloneqq(\bullet,\bullet)_{L^2(\Omega)}$.
Mit $H^1(\Tcal)$ bezeichnen wir den Raum der stückweisen $H^1$-Funktionen auf
$\Tcal$.


\section{Crouzeix-Raviart-Finite-Elemente-Funktionen}
\label{sec:crouzeixRaviartFunctions}

In \Cref{sec:discreteProblemFormulation} wollen wir für eine nichtkonforme
Diskretisierung des Minimierungsproblems \ref{prob:continuousProblem} den
Crouzeix-Raviart-Finite-Elemente-Raum nutzen. 
Die dafür benötigten Definitionen und Aussagen tragen wir in diesem Abschnitt
aus \cites{Car09b}{CGR12}{CGKNRR10} zusammen. 

Die Räume der Crouzeix-Raviart-Finite-Elemente-Funktionen sind definiert durch
\begin{align*}
  \CR^1(\Tcal)
  &\coloneqq
  \{\vcr\in P_1(\Tcal)\,\mid\, \vcr \text{ ist stetig in }\Mid(E)\text{ für
  alle } E\in\Ecal\} \quad\text{und}\\
  \CR^1_0(\Tcal)
  &\coloneqq
  \left\{\vcr\in\CR^1(\Tcal)\,\middle|\, \forall E\in\Ecal(\partial\Omega)\quad
  \vcr(\Mid(E))=0\right\}.
\end{align*}
Wir wollen $\CR^1_0(\Tcal)$ mit einem diskreten Skalarprodukt ausstatten. 
Dafür führen wir den stückweisen Gradienten $\gradnc:H^1(\Tcal)\to
L^2\!\left(\Omega;\Rbb^2\right)$ ein. 
Dieser ist für alle $v\in H^1(\Tcal)$ und alle $T\in\Tcal$ definiert durch
$\left(\gradnc v\right)\!|_T\coloneqq \nabla v|_T$.
Das ermöglicht uns die Definition der diskreten Bilinearform
$\anc(\bullet,\bullet):\CR^1_0(\Tcal)\times\CR^1_0(\Tcal)\to\Rbb$ 
durch
\begin{align*}
  \anc(\ucr,\vcr)\coloneqq\int_\Omega\gradnc\ucr\cdot\gradnc\vcr\dx
  \quad\text{für alle } \ucr,\vcr\in\CR^1_0(\Tcal).
\end{align*}
Diese ist ein Skalarprodukt auf $\CR^1_0(\Tcal)$.
Die von $\anc$ induzierte Norm bezeichnen wir mit $\vvvert\bullet\vvvert_\NC$.

Nun definieren wir für jede Kante $E\in\Ecal$ eine Funktion
$\psi_E\in\CR^1(\Tcal)$ durch ihre Werte
\begin{align*}
  \psi_E(\Mid(E))&\coloneqq 1 
  &&\text{sowie} 
  &\psi_E(\Mid(F))&\coloneqq 0\quad\text{für alle } F\in\Ecal.
\end{align*}
Die Menge $\{\psi_E\,\mid\,E\in\Ecal\}$ bildet dann eine Basis von
$\CR^1(\Tcal)$ und die Menge
$\{\psi_E\,\mid\,E\in\Ecal(\Omega)\}$ bildet eine Basis von
$\CR^1_0(\Tcal)$. 
Für \Cref{chap:implementation} werden wir außerdem die folgenden Aussagen
benötigen.
Sei $T\in\Tcal$ mit $T=\conv\{P_1,P_2,P_3\}$. 
Dann sind die baryzentrischen Koordinaten $\lambda_1,\lambda_2,\lambda_3\in
P_1(T)$ für $j,k\in\{1,2,3\}$ charakterisiert durch
\begin{align*}
  \lambda_j(P_k)
  =
  \begin{cases}
    1,&\text{falls }  j=k,\\
    0,&\text{sonst.}
  \end{cases}
\end{align*}
Seien die Kanten von $T$ nun so mit $E_1,E_2$ und $E_3$ bezeichnet, dass für
$j\in\{1,2,3\}$ die Kante $E_j$ gegenüber von $P_j$ liegt. 
Dann können wir die lokalen Crouzeix-Raviart-Basisfunktionen 
$\psi_{E_j}\!|_T$, $j\in\{1,2,3\}$, für alle $x\in T$ darstellen durch
\begin{equation}
  \label{eq:connectionCrBarycentric}
  \psi_{E_j}\!|_T(x)=1-2\lambda_j(x).
\end{equation}
Dadurch ist für alle $k_1,k_2,k_3\in\Nbb$ die Integrationsformel
\begin{equation}
  \label{eq:formulaIntegrationBarycentricCoordinates}
  \int_T \lambda_1^{k_1}\lambda_2^{k_2}\lambda_3^{k_3}\dx
  =
  2|T|\frac{k_1!k_2!k_3!}{(2+k_1+k_2+k_3)!}
\end{equation}
auch für die Berechnung von Integralen über Produkte der lokalen
Crouzeix-Raviart-Basisfunktionen auf $T$ nutzbar. Ein Beweis der Formel
\eqref{eq:formulaIntegrationBarycentricCoordinates} ist in 
\cite{EM73} zu finden.


\section{Variationsrechnung auf Banachräumen}
\label{sec:variationalCalculus}

In dieser Arbeit setzen wir Kenntnisse über die Variationsrechnung voraus. 
Da wir aber auch Funktionale betrachten, die auf
unendlichdimensionalen, reellen Banachräumen definiert sind, führen
wir in diesem Abschnitt die grundlegenden Notationen dafür ein und formulieren
schließlich die notwendige Optimalitätsbedingung erster Ordnung.

Dabei folgen wir \cite[S. 189-194]{Zei85}. 
Dort werden einige der Aussagen auf einem reellen, lokal konvexen
Raum formuliert. Da nach \cite[S. 781, (43)]{Zei86} alle Banachräume 
lokal konvex sind und wir die Aussagen in dieser Arbeit
nur auf Banachräumen benötigen, formulieren wir sie hier auf einem reellen 
Banachraum $X$. 
Außerdem betrachten wir eine Teilmenge $V\subseteq X$, einen
inneren Punkt $u$ von $V$ und ein Funktional $F:V\to\Rbb$. 
Schließlich definieren wir noch für alle $h\in X$ eine Funktion
$\varphi_h:\Rbb\to\Rbb$, die für alle $t\in\Rbb$ gegeben ist durch
$\varphi_h(t)\coloneqq F(u+th)$.
Damit können wir die $n$-te Variation, das G\^ateaux- und das 
Fr\'echet-Differential von $F$ definieren.

\begin{definition}[$n$-te Variation]
  Die $n$-te Variation von $F$ an der Stelle $u$ in Richtung $h\in X$ ist,
  falls die $n$-te Ableitung $\varphi_h^{(n)}(0)$ von $\varphi_h$ in $0$
  existiert, definiert durch 
  \begin{align*}
    \delta^n F(u;h)\coloneqq \varphi_h^{(n)}(0)=
    \left. \frac{d^n F(u+th)}{dt^n}\right|_{t=0}.
  \end{align*}
  Wir schreiben $\delta$ für $\delta^1$.
\end{definition}

\begin{definition}[G\^ateaux- und Fr\'echet-Differential]
  $F$ heißt G\^ateaux-differenzierbar an der Stelle $u$, falls ein 
  Funktional $F'(u)\in X^\ast$ existiert, sodass
  \begin{align*}
    \lim_{t\to 0}\frac{F(u+th)-F(u)}{t} = \langle F'(u), h\rangle\quad
    \text{für alle } h\in X.
  \end{align*}
  $F'(u)$ heißt dann G\^ateaux-Differential von $F$ an der Stelle $u$.
  $F$ heißt Fr\'echet-dif\-fe\-ren\-zier\-bar an der Stelle $u$, falls ein
  Funktional $F'(u)\in X^\ast$ existiert, sodass
  \begin{align*}
    \lim_{\Vert h\Vert_X\to 0}\frac{|F(u+th)-F(u)-
    \langle F'(u),h\rangle|}{\Vert h\Vert_X} =0.
  \end{align*}
  $F'(u)$ heißt dann Fr\'echet-Differential von $F$ an der Stelle $u$.
  Das Fr\'echet-Differential von $F$ an der Stelle $u$ in Richtung $h\in X$
  ist definiert durch $dF(u;h)\coloneqq \langle F'(u),h\rangle.$
\end{definition}

\begin{remark}
  Existiert das Fr\'echet-Differential $F'(u)$ von $F$ an der Stelle
    $u$, so ist $F'(u)$ auch das G\^ateaux-Differential von $F$ an der Stelle
    $u$ und es gilt 
    \begin{align*}
      \delta F(u;h)=dF(u;h)=\langle F'(u),h\rangle\quad\text{für alle } h\in X.
    \end{align*}
\end{remark}

Nachdem nun die relevante Notation eingeführt ist, können wir zum Abschluss
die notwendige Bedingung erster Ordnung für einen lokalen Minimierer von $F$
formulieren.

\begin{theorem}[Notwendige Optimalitätsbedingung erster Ordnung]
  \label{thm:necessaryConditionFreeLocalExtrema}
  Sei $u\in \interior(V)$ lokaler Minimierer von $F$, das heißt es existiere
  eine Umgebung $U\subseteq V$ von $u$, sodass $F(v)\geq F(u)$ für alle $v\in
  U$. 
  Dann gilt für alle $h\in X$, dass $\delta F(u;h) = 0$, falls diese Variation
  für alle $h\in X$ existiert, beziehungsweise $F'(u) = 0$, falls $F'(u)$ als
  G\^ateaux- oder Fr\'echet-Differential existiert.
\end{theorem}

\begin{proof}
  Die Aussage folgt direkt aus \cite[S. 193 f., Theorem 40.A, Theorem
  40.B]{Zei85}.
\end{proof}


\section{Subdifferentiale}

Für diesen Abschnitt betrachten wir stets einen reellen Banachraum $X$ und,
falls nicht anders spezifiziert, ein Funktional $F:X\to [-\infty,\infty]$.
Wir wollen die in dieser Arbeit benötigten Notationen und
Eigenschaften des Subdifferentials von $F$ zusammentragen.
Zuvor starten wir mit einer grundlegenden Definition.

\begin{definition}[\protect{\cite[S. 245, Definition 42.1]{Zei85}}]
  Sei $V$ ein Vektorraum, $M\subseteq V$ und $F:M\to\Rbb$. 
  Dann heißt die Menge $M$ konvex, wenn für alle $u,v\in M$ und alle $t\in
  [0,1]$ gilt $$(1-t)u+tv\in M.$$
  Ist $M$ konvex, so heißt $F$ konvex, falls für alle $u,v\in M$ und alle
  $t\in[0,1]$ gilt 
  \begin{equation}
    \label{eq:convexity}
    F\big( (1-t)u+tv\big)\leq (1-t)F(u)+t F(v).
  \end{equation}
  Gilt Ungleichung \eqref{eq:convexity} für alle $t\in
  (0,1)$mit \glqq$<$\grqq, so heißt $F$ strikt konvex. 
  Falls $-F$ konvex ist, so heißt $F$ konkav.
\end{definition}

Für den Rest dieses Abschnitts folgen wir \cite[S. 385-397]{Zei85}. 
Analog zur Begründung zum Beginn von \Cref{sec:variationalCalculus}, schränken
wir auch hier die Definitionen und Aussagen, die in \cite{Zei85} auf reellen,
lokal konvexen Räumen formuliert sind, auf den reellen Banachraum $X$ ein.
Zunächst definieren wir das Subdifferential von $F$ an einer Stelle $u\in X$.

\begin{definition}[Subdifferential]
  \label{def:subdifferential}
  Für $u\in X$ mit $F(u)\neq\pm\infty$ heißt
  \begin{equation}
    \label{eq:subdifferential}
    \partial F(u)\coloneqq
    \{u^\ast\in X^\ast\ |\ 
    \forall v\in X\quad F(v)\geq F(u)+\langle u^\ast,v-u\rangle\}  
  \end{equation}
  Subdifferential von $F$ an der Stelle $u$. Für $F(u)=\pm\infty$ ist
  $\partial F(u)\coloneq\emptyset$.
  Ein Element $u^\ast\in\partial F(u)$ heißt Subgradient von $F$ an der Stelle
  $u$.
\end{definition}

Es folgen für Optimierungsprobleme wichtige Aussagen über das Subdifferential 
von $F$.

\begin{theorem}[\protect{\cite[S. 387, Proposition 47.12]{Zei85}}]
  \label{thm:extremalprinciple}
  Falls $F: X\to (-\infty,\infty]$ mit $F\nequiv\infty$, gilt
  $F(u)=\inf_{v\in X}F(v)$ genau dann, wenn $0\in\partial F(u)$.
\end{theorem}

\begin{theorem}[\protect{\cite[S. 387, Proposition 47.13 (i)]{Zei85}}]
  \label{thm:subdiffGateaux}
  Falls $F$ konvex und G\^{a}teaux-differenzierbar an der Stelle $u\in X$ mit
  G\^{a}teaux-Differential $F'(u)$ ist, gilt $\partial F(u)=\{F'(u)\}$.
\end{theorem}

Das folgende Theorem folgt aus \cite[S. 389, Theorem 47.B]{Zei85} unter 
Beachtung der Tatsache, dass die Addition von Funktionalen 
$F_1,F_2,\ldots,F_n:X\to (-\infty,\infty]$ und die Addition von
Mengen in $X^\ast$ kommutieren.

\begin{theorem}
  \label{thm:subdifferentialSumRule}
  Seien für $n\geq 2$ die Funktionale $F_1,F_2,\ldots,F_n: X\to
  (-\infty,\infty]$ konvex und es existiere ein $u_0\in X$, sodass
  $F_k(u_0)<\infty$ für alle $k\in\{1,2\ldots,n\}$. 
  Außerdem seien mindestens $n-1$ der $n$ Funktionale $F_1,F_2,\ldots,F_n$
  stetig an der Stelle $u_0$.
  Dann gilt 
  \begin{align*}
    \partial (F_1+F_2+\ldots+ F_n)(u) 
    = \partial F_1(u)+\partial F_2(u)+ \ldots + \partial F_n(u) \quad\text{für
    alle } u\in X.
  \end{align*}
\end{theorem}

Zum Abschluss formulieren wir noch die Monotonie des Subdifferentials.
Diese folgt aus \cite[S. 396 f., Definition 47.15, Theorem 47.F (1)]{Zei85}.

\begin{theorem}
  \label{thm:subdifferentialMonotonicity}
  Sei $F:X\to (-\infty,\infty]$ konvex und unterhalbstetig mit $F\nequiv\infty$.
  Dann ist $\partial F(\bullet)$ monoton, das heißt 
  \begin{align*}
    \langle u^\ast-v^\ast,u-v\rangle\geq 0\quad \text{für alle } u,v\in X, 
    u^\ast \in \partial F(u), v^\ast \in \partial F(v).
  \end{align*}
\end{theorem}


\section{Funktionen beschränkter Variation}
\label{sec:bvFunctions}

In diesem Abschnitt führen wir den Raum der Funktionen beschränkter Variation
ein.
Wir vermeiden dabei für den weiteren Verlauf dieser Arbeit
nicht benötigte Notation und Theorie, indem wir die Definitionen und Aussagen 
entsprechend aus- und umformulieren.
Für detaillierte Ausführungen und die maßtheoretischen Hintergründe siehe zum
Beispiel \cite{ABM14, EG92, Bra98}. 
Soweit nicht anders angegeben, folgen die Definitionen und Aussagen
dieses Abschnitts aus \cite[S. 393-395]{ABM14}.
Sei im Weiteren $U$ eine offene Teilmenge des $\Rbb^d$.
Zunächst definieren wir den Raum der Funktionen beschränkter Variation.

\begin{definition}[Funktionen beschränkter Variation]
  Eine Funktion $u\in L^1(U)$ ist von beschränkter Variation, wenn   
  \begin{align}
    \label{eq:boundedVariation}
    |u|_{\BV(U)}
    \coloneqq
    \sup_{\substack{\phi\in C^1_C(U;\Rbb^d)\\
    \Vert\phi\Vert_{L^\infty(U)}\leq 1}}\int_U u\Div (\phi)\dx
    <
    \infty.
  \end{align}
  Die Menge aller Funktionen beschränkter Variation ist $\BV(U)$.
\end{definition}

\begin{remark}
  \label{rem:bvSeminorm}
  Durch $|\bullet|_{\BV(U)}$ ist eine Seminorm auf $\BV(U)$
  gegeben.
  Ausgestattet mit der Norm
  \begin{align*}
    \Vert \bullet \Vert_{\BV(U)} \coloneqq \Vert \bullet\Vert_{L^1(U)} +
    |\bullet|_{\BV(U)}
  \end{align*}
  ist $\BV(U)$ ein Banachraum.
  Außerdem gilt $W^{1,1}(U)\subset\BV(U)$ und 
  $\Vert u \Vert_{\BV(U)}=\Vert u \Vert_{W^{1,1}(U)}$ für alle
  $u\in W^{1,1}(U)$.
\end{remark}

In der Anwendung ist Konvergenz in $\BV(U)$ bezüglich der Norm
$\Vert\bullet\Vert_{\BV(U)}$ zu restriktiv (cf.\ \cite[300]{Bar15}). 
Deshalb führen wir einen schwächeren Konvergenzbegriff ein.

\begin{definition}
  Sei $(u_n)_{n\in\Nbb}\subset \BV(U)$ und sei $u\in \BV(U)$ mit
  $u_n\rightarrow u$ in $L^1(U)$ für $n\rightarrow\infty$.
  Dann konvergiert die Folge $(u_n)_{n\in\Nbb}$ schwach gegen $u$ in $\BV(U)$,
  wenn für alle $\phi\in C_0(U;\Rbb^d)$ gilt, dass 
  \begin{align*}
    \int_U u_n\Div(\phi)\dx\rightarrow \int_U u\Div(\phi)\dx 
    \quad\text{für } n\to\infty. 
  \end{align*}
  Wir schreiben dann $u_n\rightharpoonup u$ für $n\to\infty$.
\end{definition}

Damit können wir das folgende Theorem formulieren, welches unmittelbar die
schwache Unterhalbstetigkeit der Seminorm $|\bullet|_{\BV(U)}$ auf $\BV(U)$
impliziert.

\begin{theorem}[\protect{\cite[S. 394, Proposition 10.1.1]{ABM14}}]
  \label{thm:wlsc}
  Sei $u\in L^1(U)$ und sei $(u_n)_{n\in\Nbb}\subset\BV(U)$ mit
  $\sup_{n\in\Nbb}|u_n|_{\BV(U)}< \infty$ und
  $u_n\rightarrow u$ in $L^1(U)$ für $n\rightarrow\infty$.
  Dann gilt $u\in\BV(U)$ und $|u|_{\BV(U)}\leq
  \liminf_{n\rightarrow\infty}|u_n|_{\BV(U)}.$
  Außerdem gilt dann $u_n\rightharpoonup u$ in $\BV(U)$.
\end{theorem}

Für den Rest dieses Abschnitts sei $U$ ein beschränktes Lip\-schitz-Gebiet.
Unter dieser Voraussetzung können wir zeigen, dass jede in $\BV(U)$
beschränkte Folge eine in $\BV(U)$ schwach konvergente Teilfolge besitzt
mit schwachem Grenzwert in $\BV(U)$. Für den Beweis dieser Aussage 
benötigen wir zunächst das folgende Theorem. 

\begin{theorem}[\protect{\cite[S. 176, Theorem 4]{EG92}}]
  \label{thm:l1ConvergentSubsequence}
  Sei $(u_n)_{n\in\Nbb}\subset \BV(U)$ eine beschränkte Folge. Dann 
  existiert eine Teilfolge $(u_{n_k})_{k\in\Nbb}$ von
  $(u_n)_{n\in\Nbb}$ und ein $u\in\BV(U)$, sodass
  $u_{n_k}\to u$ in $L^1(U)$ für $k\to \infty$.
\end{theorem}

Damit können wir nun das folgende Theorem beweisen.
\begin{theorem}
  \label{thm:compactness}
  Sei $(u_n)_{n\in\Nbb}\subset \BV(U)$ eine beschränkte Folge. Dann 
  existiert eine Teilfolge $(u_{n_k})_{k\in\Nbb}$ und ein $u\in\BV(U)$,
  sodass $u_{n_k}\rightharpoonup u$ in $\BV(U)$ für $k\rightarrow\infty$.
\end{theorem}

\begin{proof}
  Nach \Cref{thm:l1ConvergentSubsequence} besitzt $(u_n)_{n\in\Nbb}$ eine
  Teilfolge $(u_{n_k})_{k\in\Nbb}$, die in $L^1(U)$ gegen ein
  $u\in\BV(U)$ konvergiert.
  Diese Teilfolge ist nach Voraussetzung beschränkt in $\BV(U)$,
  woraus nach Definition der Norm $\Vert\bullet\Vert_{\BV(U)}$ insbesondere
  folgt, dass $\sup_{k\in\Nbb}|u_{n_k}|_{\BV(U)}< \infty$. 
  Somit ist \Cref{thm:wlsc} anwendbar und impliziert die schwache Konvergenz von
  $(u_{n_k})_{k\in\Nbb}$ in $\BV(U)$ gegen $u\in\BV(U)$.
\end{proof}
