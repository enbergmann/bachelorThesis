\section{Formulierung}
\label{sec:continuousProblemFormulation}
Für einen Parameter $\alpha\in\Rbb_+$ und eine Funktion
$f\in L^2(\Omega)$ betrachten wir das folgende Minimierungsproblem. 

\begin{problem}\label{prob:continuousProblem}
  Finde $u\in \BV(\Omega)\cap L^2(\Omega)$, sodass
  $u$ das Funktional
  \begin{align}\label{eq:continuousProblem}
    E(v)\coloneqq \frac{\alpha}{2}\Vert v\Vert^2 + |v|_{\BV(\Omega)}
    +\Vert v\Vert_{L^1(\partial\Omega)}-\int_\Omega fv\dx
  \end{align}
  unter allen $v\in\BV(\Omega)\cap L^2(\Omega)$ minimiert.

  Dabei ist der Term $\Vert v\Vert_{L^1(\partial\Omega)}$ wohldefiniert, da
  nach \cite[S. 400, Theorem 10.2.1]{ABM14} eine lineare, stetige Abbildung
  $T:\BV(\Omega)\to L^1(\partial\Omega)$ existiert mit $T(u) =
  u|_{\partial\Omega}$ für alle $u\in\BV(\Omega)\cap C(\overline\Omega)$.
\end{problem}

\begin{remark}
  Nach \cite[S. 399, Theorem 10.1.3]{ABM14} ist 
  \todo[inline]{noch fragen, was 1-regular nochmal heißt und ob das hier 
  glatt geht (tut es sehr wahrscheinlich)}
  die Einbettung $\BV(\Omega)\hookrightarrow L^p(\Omega)$ stetig für 
  $1\leq p\leq d/(d-1)$. 
  Damit ist $\BV(\Omega)$ für $d=2$ Teilmenge von $L^2(\Omega)$ und die
  Lösung von \Cref{prob:continuousProblem} kann in
  $\BV(\Omega)$ gesucht werden. Für beliebige $d\in\Nbb$, die wir in diesen
  Abschnitt betrachten, gilt dies im Allgemeinen nicht.
\end{remark}


\section{Existenz eines eindeutigen Minimierers}
Zunächst zeigen wir, dass \Cref{prob:continuousProblem} eine Lösung besitzt.
Dafür benötigen wir die folgende Formulierung der Youngschen Ungleichung.

\begin{lemma}[Youngsche Ungleichung]
  \label{lem:young}
  Seien $a,b\in\Rbb$ und $\varepsilon\in\Rbb_+$ beliebig. Dann gilt
  \begin{align*}
    ab\leq\frac{1}{\varepsilon}a^2+\frac{\varepsilon}{4}b^2. 
  \end{align*}
\end{lemma}

Außerdem wird im Beweis folgende Aussage benötigt, die direkt aus \cite[S. 183,
Theorem 1]{EG92} folgt, da
$0\in\BV\!\left(\Rbb^d\setminus\overline\Omega\right)$,
$|0|_{\BV\!\left(\Rbb^d\setminus\overline\Omega\right)}=0$ und
$0|_{\partial\Omega}=0$.

\begin{lemma}
  \label{lem:bvExtension}
  Sei $v\in\BV(\Omega)$.
  Definiere, für alle $x\in\Rbb^d$,
  \begin{align*}
    \tilde{v}(x)\coloneqq
    \begin{cases}
      v(x),  &\text{ falls } x\in\Omega,\\
      0,     &\text{ falls } x\in\Rbb^d\setminus\overline\Omega.
    \end{cases} 
  \end{align*}
  Dann gilt $\tilde{v}\in\BV\!\left(\Rbb^d\right)$ und
  $\left|\tilde{v}\right|_{\BV\!\left(\Rbb^d\right)}
  = |v|_{\BV(\Omega)}+\Vert v\Vert_{L^1(\partial\Omega)}$.
\end{lemma}

\begin{theorem}[Existenz einer Lösung]
  \label{thm:contProblemExistence}
  \Cref{prob:continuousProblem} besitzt eine Lösung \\$u\in\BV(\Omega)\cap
  L^2(\Omega)$.
\end{theorem}

\begin{proof}
  Die Beweisidee ist die Anwendung der direkten Methode der Variationsrechnung
  (cf.\ z.B.\ \cite{Dac89}) unter Nutzung der in \Cref{sec:bvFunctions}
  aufgeführten Eigenschaften der schwachen Konvergenz in $\BV(\Omega)$.

  Für alle $v\in L^2(\Omega)\subseteq L^1(\Omega)$ gilt mit der Hölderschen
  Ungleichung für $p=q=2$, dass
  \begin{equation}\label{eq:hoelderL2BiggerL1}
    \Vert v\Vert_{L^1(\Omega)} 
    = \Vert 1\cdot v\Vert_{L^1(\Omega)}
    \leq \Vert 1\Vert\Vert v\Vert
    =\sqrt{|\Omega|} \Vert v\Vert.
  \end{equation}
  Dann folgt für das Funktional $E$ in \eqref{eq:continuousProblem} für alle
  $v\in \BV(\Omega)\cap L^2(\Omega)$ durch die Cauchy-Schwarzsche Ungleichung,
  die Youngsche Ungleichung aus \cref{lem:young} und Ungleichung
  \eqref{eq:hoelderL2BiggerL1}, dass

  \begin{equation}
    \label{eq:contProbBddFromBelow}
    \begin{aligned}
      E(v)&=\frac{\alpha}{2}\Vert v\Vert^2 + |v|_{\BV(\Omega)}
      +\Vert v\Vert_{L^1(\partial\Omega)}-\int_\Omega fv\dx\\
      &\geq 
      \frac{\alpha}{2}\Vert v\Vert^2 + |v|_{\BV(\Omega)}
      +\Vert v\Vert_{L^1(\partial\Omega)}
      -\Vert f\Vert\Vert v\Vert\\
      &\geq 
      \frac{\alpha}{2}\Vert v\Vert^2 + |v|_{\BV(\Omega)}
      +\Vert v\Vert_{L^1(\partial\Omega)}
      -\frac{1}{\alpha}\Vert f\Vert^2
      -\frac{\alpha}{4}\Vert v\Vert^2\\
      &\geq 
      \frac{\alpha}{4}\Vert v\Vert^2 + |v|_{\BV(\Omega)}
      +\Vert v\Vert_{L^1(\partial\Omega)}-\frac{1}{\alpha}\Vert
      f\Vert^2\\
      &\geq 
      \frac{\alpha}{4|\Omega|}\Vert v\Vert_{L^1(\Omega)}^2 + |v|_{\BV(\Omega)}
      +\Vert v\Vert_{L^1(\partial\Omega)}-\frac{1}{\alpha}\Vert
      f\Vert^2\\
      &\geq -\frac{1}{\alpha}\Vert f\Vert^2.
    \end{aligned}
  \end{equation}
  Somit ist $E$ nach unten beschränkt, was die Existenz einer infimierenden
  Folge $(u_n)_{n\in\Nbb}\subset\BV(\Omega)\cap L^2(\Omega)$ von $E$ 
  impliziert, das heißt
  $(u_n)_{n\in\Nbb}$ erfüllt $$\lim_{n\rightarrow\infty}E(u_n) =
  \inf_{v\in\BV(\Omega)\cap L^2(\Omega)}E(v).$$ 

  Ungleichung \eqref{eq:contProbBddFromBelow} impliziert außerdem, dass
  $E(u_n)\to\infty$ für $n\to\infty$, falls $|u_n|_{\BV(\Omega)}\to\infty$ oder
  $\Vert u_n\Vert_{L^1(\Omega)}\to\infty$ für $n\to\infty$. 
  Daraus folgt insbesondere, dass $E(u_n)\to\infty$ für $n\to\infty$, falls
  $\Vert u_n\Vert_{\BV(\Omega)}\to\infty$ für $n\to\infty$ .
  Deshalb muss die Folge $(u_n)_{n\in\Nbb}$ beschränkt in $\BV(\Omega)$ sein.

  Nun garantiert \cref{thm:compactness} die Existenz einer in $\BV(\Omega)$
  schwach konvergenten Teilfolge $(u_{n_k})_{k\in\Nbb}$ von $(u_n)_{n\in\Nbb}$
  mit schwachen Grenzwert $u\in\BV(\Omega)$. 
  Ohne Beschränkung der Allgemeinheit ist
  $(u_{n_k})_{k\in\Nbb}=(u_n)_{n\in\Nbb}$.
  Aus der schwachen Konvergenz von $(u_n)_{n\in\Nbb}$ in $\BV(\Omega)$ gegen
  $u$ folgt nach Definition, dass $(u_n)_{n\in\Nbb}$ stark, und damit
  insbesondere auch schwach, in $L^1(\Omega)$ gegen $u$ konvergiert.

  Weiterhin folgt aus (\ref{eq:contProbBddFromBelow}), dass
  $E(v)\rightarrow\infty$ für $\Vert v\Vert\rightarrow\infty$. 
  Somit muss $(u_n)_{n\in\Nbb}$ auch beschränkt sein bezüglich der Norm
  $\Vert\bullet\Vert$ und besitzt deshalb, wegen der Reflexivität von
  $L^2(\Omega)$, eine Teilfolge (ohne Beschränkung der Allgemeinheit weiterhin
  bezeichnet mit $(u_n)_{n\in\Nbb}$), die in $L^2(\Omega)$ schwach gegen einen
  Grenzwert $\overline{u}\in L^2(\Omega)$ konvergiert. 
  Damit gilt für alle $w\in L^2(\Omega)\cong L^2(\Omega)^\ast$ und, da
  $L^\infty(\Omega)\subseteq L^2(\Omega)$, insbesondere auch für alle $w\in
  L^\infty(\Omega)\cong L^1(\Omega)^\ast$, dass 
  \begin{align*}
    \lim_{n\to\infty}\int_\Omega u_n w\dx =\int_\Omega \overline{u} w\dx.
  \end{align*}
  Das bedeutet, dass $(u_n)_{n\in\Nbb}$ auch schwach in $L^1(\Omega)$ gegen
  $\overline{u}\in L^2(\Omega)\subseteq L^1(\Omega)$ konvergiert. 

  Da schwache Grenzwerte eindeutig bestimmt sind, gilt insgesamt $u=\overline u
  \in L^2(\Omega)$, das heißt $u\in\BV(\Omega)\cap
  L^2(\Omega)$.

  Nun definieren wir für alle
  $n\in\Nbb$ und für alle 
  $x\in\Rbb^d$
  \begin{align*}
    \tilde{u}_n(x)
    &\coloneqq
    \begin{cases}
      u_n(x),  &\text{ falls } x\in\Omega,\\
      0,     &\text{ falls } x\in\Rbb^d\setminus\overline\Omega
    \end{cases} 
    &&\text{und }
    &\tilde{u}(x)
    &\coloneqq
    \begin{cases}
      u(x),  &\text{ falls } x\in\Omega,\\
      0,     &\text{ falls } x\in\Rbb^d\setminus\overline\Omega.
    \end{cases} 
  \end{align*}

  Dann gilt nach \cref{lem:bvExtension} sowohl
  \begin{align*}
    \tilde{u}_n
    &\in
    \BV\!\left(\Rbb^d\right)
    &&\text{und}
    &\left|\tilde{u}_n\right|_{\BV\!\left(\Rbb^d\right)} 
    &= 
    |u_n|_{\BV(\Omega)}+\Vert u_n\Vert_{L^1(\partial\Omega)}
    \quad\text{für alle }n\in\Nbb \text{ als auch}\\
    \tilde{u}
    &\in
    \BV\!\left(\Rbb^d\right)
    &&\text{und}
    &\left|\tilde{u}\right|_{\BV\!\left(\Rbb^d\right)} 
    &=
    |u|_{\BV(\Omega)}+\Vert u\Vert_{L^1(\partial\Omega)}.
  \end{align*}
  Da $(u_n)_{n\in \Nbb}$ infimierende Folge von $E$ ist, muss die Folge
  \begin{align*}
    \left(\left|\tilde{u}_n\right|_{\BV\!\left(\Rbb^d\right)}\right)_{n\in\Nbb} 
    = \left(|u_n|_{\BV(\Omega)}+
    \Vert u_n\Vert_{L^1(\partial\Omega)}\right)_{n\in\Nbb}
  \end{align*}
  beschränkt sein.
  Außerdem folgt aus den Definitionen von $\tilde{u}$ und 
  $\tilde{u}_n$ für alle $n\in\Nbb$ und der bereits bekannten Eigenschaft 
  $u_n\to u$ in $L^1(\Omega)$ für $n\to\infty$, dass
  \begin{align*}
    \left\Vert \tilde{u}_n - \tilde{u}\right\Vert_{L^1\!\left(\Rbb^d\right)} 
    &= \int_{\Rbb^d} \left|\tilde{u}_n - \tilde{u}\right|\dx
    = \int_\Omega |u_n - u|\dx
    = \Vert u_n - u\Vert_{L^1(\Omega)}\to 0\quad\text{für }n\to\infty,
  \end{align*}
  das heißt $\tilde{u}_n \to \tilde{u}$ in $L^1\left(\Rbb^d\right)$ für
  $n\to\infty$.

  Insgesamt ist also $\left(\tilde{u}_n\right)_{n\in\Nbb}$ eine Folge in
  $\BV\!\left(\Rbb^d\right)$, die in $L^1\!\left(\Rbb^d\right)$ gegen
  $\tilde{u}\in\BV\!\left(\Rbb^d\right)\subseteq L^1\!\left(\Rbb^d\right)$
  konvergiert und 
  $\sup_{n\in\Nbb} \left|\tilde{u}_n\right|_{\BV\!\left(\Rbb^d\right)}<\infty$
  erfüllt.
  Somit folgt mit
  \cref{thm:wlsc}  
  \begin{equation}
    \label{eq:wlscOfExtension}
    \begin{aligned}
      |u|_{\BV(\Omega)} +\Vert u\Vert_{L^1(\partial\Omega)}
      = \left|\tilde{u}\right|_{\BV\left(\Rbb^d\right)}
      &\leq\liminf_{n\to\infty}
      \left|\tilde{u}_n\right|_{\BV\left(\Rbb^d\right)}\\
      &= \liminf_{n\to\infty} \left(|u_n|_{\BV(\Omega)} +
      \Vert u_n\Vert_{L^1(\partial\Omega)}\right).
    \end{aligned}
  \end{equation}

  Die Funktionen $\Vert\bullet\Vert^2$ und $-\int_\Omega
  f\bullet\dx$ sind auf $L^2(\Omega)$ stetig und konvex, was impliziert,
  dass sie schwach unterhalbstetig auf $L^2(\Omega)$ sind. Da wir bereits
  wissen, dass $u_n\rightharpoonup u$ in $L^2(\Omega)$ für $n\to\infty$, 
  folgt
  \begin{align*}
    \frac{\alpha}{2}\Vert u\Vert-\int_\Omega fu\dx
    \leq \liminf_{n\to\infty}
    \left(\frac{\alpha}{2}\Vert u_n\Vert
    -\int_\Omega fu_n\dx\right).
  \end{align*}
  
  Damit und mit Ungleichung \eqref{eq:wlscOfExtension} gilt insgesamt
  \begin{align*}
    \inf_{v\in\BV(\Omega)\cap L^2(\Omega)}E(v)\leq
    E(u)\leq\liminf_{n\rightarrow\infty} E\left(u_n\right) =
    \lim_{n\rightarrow\infty}E\left(u_n\right) = \inf_{v\in\BV(\Omega)\cap
    L^2(\Omega)}E(v),
  \end{align*}
  das heißt $\min_{v\in\BV(\Omega)\cap L^2(\Omega)} E(v) = E(u)$.
\end{proof}

Nachdem wir gezeigt haben, dass für \Cref{prob:continuousProblem} eine
Lösung existiert, beweisen wir als nächstes ein Theorem, das direkt impliziert,
dass diese Lösung eindeutig ist. 
\begin{theorem}[Stabilität und Eindeutigkeit]
  \label{thm:contProbStabAndUniqu}
  Seien $u_1,u_2\in \BV(\Omega)\cap L^2(\Omega)$ die Minimierer des Problems
  \ref{prob:continuousProblem} mit $f_1,f_2\in L^2(\Omega)$ anstelle von $f$,
  das heißt für $\ell\in\{1,2\}$ minimiere $u_\ell$ das Funktional
  \begin{align*}
    E_\ell
    \coloneqq 
    \frac{\alpha}{2}\Vert v\Vert^2 + |v|_{\BV(\Omega)} 
    + \Vert v\Vert_{L^1(\partial\Omega)} - \int_\Omega f_\ell v\dx
  \end{align*}
  unter allen $v\in\BV(\Omega)\cap L^2(\Omega)$.

  Dann gilt 
  \begin{align*}
    \Vert u_1 - u_2\Vert 
    \leq\frac{1}{\alpha}\Vert f_1-f_2\Vert.
  \end{align*}
\end{theorem}

\begin{proof}
  Wir folgen der Argumentation im Beweis von \cite[S. 304, Theorem 10.6]{Bar15}.

  Zunächst definieren wir die Funktionale $F: L^2(\Omega)\to
  \Rbb\cup\{\infty\}$ und $G_\ell:L^2(\Omega)\to \Rbb$, $\ell\in\{1,2\}$, für
  alle
  $u\in L^2(\Omega)$ durch
  \begin{align*}
    F(u) 
    &\coloneqq 
    \begin{cases}
      |u|_{\BV(\Omega)} + \Vert u \Vert_{L^1(\partial\Omega)}, 
      &\text{ falls } u\in\BV(\Omega)\cap L^2(\Omega),\\
      \infty,&\text{ falls } u\in L^2(\Omega)\setminus\BV(\Omega)
    \end{cases}
    \quad\text{und }\\
    G_\ell(u)
    &\coloneqq 
    \frac{\alpha}{2}\Vert u\Vert^2 - \int_\Omega f_\ell u\dx.
  \end{align*}
  Damit gilt für $\ell\in\{1,2\}$ und alle
  $u\in\BV(\Omega)\cap L^2(\Omega)$, dass $E_\ell(u) =  F(u)+G_\ell(u)$.

  Für $\ell\in\{1,2\}$ ist $G_\ell$ Fr\'echet-differenzierbar und die
  Fr\'echet-Ableitung $G_\ell'(u): L^2(\Omega)\to\Rbb$ von $G_\ell$ an der
  Stelle $u\in L^2(\Omega)$ ist für alle $v\in L^2(\Omega)$
  gegeben durch
  \begin{align*}
    dG_\ell(u;v) = \alpha (u,v) - \int_\Omega f_\ell v\dx 
    = (\alpha u-f_\ell ,v).
  \end{align*}

  Das Funktional $F$ ist konvex, unterhalbstetig und es gilt $F\nequiv\infty$.
  Deshalb ist nach \cref{thm:subdifferentialMonotonicity} das Subdifferential
  $\partial F$ von $F$ monoton, das heißt für alle $\mu_\ell\in \partial
  F(u_\ell)$, $\ell\in\{1,2\}$, gilt
  \begin{align}\label{eq:stabilityAndUniqueness:monotonicityOfSubdifferential}
    (\mu_1-\mu_2,u_1-u_2)\geq 0.
  \end{align}

  Für $\ell\in\{1,2\}$ gilt, dass $E_\ell$ konvex ist und von $u_\ell$ in
  $\BV(\Omega)\cap L^2(\Omega)$ minimiert wird. 
  Außerdem gilt $E_\ell\nequiv\infty$ und $G_\ell$ ist stetig.
  Somit gilt nach \cref{thm:extremalprinciple}, 
  \cref{thm:subdifferentialSumRule} und \cref{thm:subdiffGateaux}, dass
  $0\in\partial E_\ell(u_\ell) = \partial F(u_\ell)+\partial
  G_\ell(u_\ell)=\partial F(u_\ell)+ \{G_\ell'(u_\ell)\}.$ 
  Daraus folgt
  $-G_\ell'(u_\ell)\in\partial F(u_\ell)$.
  Zusammen mit Ungleichung
  \eqref{eq:stabilityAndUniqueness:monotonicityOfSubdifferential}
  impliziert das
  \begin{align*}
    \big( -(\alpha u_1 - f_1) -(- (\alpha u_2 - f_2)), u_1 - u_2\big)
    \geq 0.
  \end{align*}
  Durch Umformen und Anwenden der Cauchy-Schwarzschen Ungleichung erhalten wir
  \begin{align*}
    \alpha \Vert u_1 - u_2 \Vert^2
    &\leq
    \big(f_1 -f_2, u_1-u_2 \big)\\
    &\leq
    \Vert f_1-f_2\Vert\Vert u_1 - u_2\Vert.
  \end{align*}

  Falls $\Vert u_1 - u_2 \Vert = 0$, gilt die zu zeigende Aussage.
  Ansonsten führt Division durch $\alpha\Vert u_1 - u_2 \Vert\neq 0$ den
  Beweis zum Abschluss.
\end{proof}

\section{Konstruktion eines Eingangssignals für eine gegebene Lösung}
Für die numerische Untersuchung der primalen-dualen Iteration aus
\Cref{chap:algorithm} ist es sinnvoll
Eingangssignale $f$ für \Cref{prob:continuousProblem} gegeben zu haben, für die
der entsprechende gesuchte Minimierer bekannt ist. 
Als Grundlage für die Konstruktion solcher Signale nutzen wir die folgende 
Aussage von Professor Carstensen.

Sei $u$ gegeben als Funktion in Polarkoordinaten. 
Dabei beschränken wir uns auf vom Polarwinkel unabhängige Funktionen
$u:[0,\infty)\to\Rbb$ des Radius $r$ mit $u(r)=0$ falls $r\geq 1$, deren
partielle Ableitungen $\partial_r u$ fast überall in $[0,\infty)$ existiere.
Außerdem existiere auch fast überall in $[0,\infty)$ die partielle Ableitung
des für $r\in[0,\infty)$ definierten Ausdrucks
\begin{align*}
  \sgn\big(\partial_r u(r)\big)
  \coloneqq
  \begin{cases}
    -1 &\text{für }\partial_r u(r)<0,\\
    x\in[0,1] &\text{für }\partial_r u(r)=0,\\ 
    1 &\text{für }\partial_r u(r)>0.
  \end{cases}
\end{align*}

Dann ist $u$ Lösung von \Cref{prob:continuousProblem}, wenn das Eingangssignal
auf $\Omega\supseteq \left\{w\in\Rbb^2\,\middle|\, |w|\leq
1\right\}$ für fast alle $x\in\Omega$ gegeben ist durch $f(x)=g(|x|)$ mit
\begin{align*}
  g(r)
  \coloneqq 
  \alpha u(r) - \partial_r\left(\sgn\big(\partial_r u(r)\big)\right) 
  - \frac{\sgn\big(\partial_r u(r)\big)}{r}
  \quad\text{für alle }r\in[0,\infty).
\end{align*}

Damit können wir nun weitere Einschränkungen für $u$ formulieren,
um Experimente zu konstruieren, mit denen wir für uns interessante
Eigenschaften von \Cref{alg:primalDualIteration} untersuchen können.
\todo{hier}

%%%%%%%%%%%%%%%%%%%%%%%%%%%%%%%%%%%%%%%%%%%%%%%%%%%%%%
\todo[inline]{Beschreibe hier nur das allgemeine Vorgehen und die 
mathematischen Hintergründe und schreibe die genutzten Beispiele an
entsprechender Stelle in Numerische Beispiele oder so. Die ungenutzten Beispiele 
die trotzdem im Ordner liegen können in einem extra Abschnitt aufgezählt werden}
Schema ist dann: Nach section Konstruktion einer exakten Lösung erhalten wir
für die Wahl u mit sgn bla die rechte Seite f.
Die Details bei bestimmten Konstruktionen, etwa ,,um H2 zu erhalten fordern wir
noch`` ergänzen beim Vorstellen des entsprechenden Experimentes, hier nur das 
allgemeinste, das, was CC wirklich geschrieben hat.
%%%%%%%%%%%%%%%%%%%%%%%%%%%%%%%%%%%%%%%%%%%%%%%%%%%%%%
Wir wollen damit Experimente konstruieren um Aussagen in \ldots oder \ldots
zu prüfen, entsprechend müssen wir noch fordern
Da in den Kapiteln GLEB und bla u in H10 relevant wird, fordern wir
weiterhin \ldots
%%%%%%%%%%%%%%%%%%%%%%%%%%%%%%%%%%%%%%%%%%%%%%%%%%%%%%
Professor Carstensen stellte folgende KOnsturkution \ldots
Um eine rechte Seite zu finden, zu der die exakte Lösung bekannt
ist, wähle eine Funktion des Radius $u\in H^1_0([0,1])$ mit Träger im 
zweidimensionalen Einheitskreis. Insbesondere muss damit gelten $u(1)=0$ und
$u$ stetig.
Die rechte Seite als Funktion des Radius $f\in L^2([0,1])$ ist dann gegeben
durch 
\begin{align*}
  f \coloneqq 
  \alpha u - \partial_r(\sgn(\partial_r u)) - \frac{\sgn(\partial_r u)}{r},
\end{align*}
wobei für $F\in\Rbb^2\setminus\{0\}$ gilt 
$\sgn(F)\coloneqq \left\{\frac{F}{|F|}\right\}$ 
und $\sgn(0)\in B_1(0)$.
Damit außerdem gilt $f\in H^1_0([0,1])$, was z.B.\ für GLEB relevant ist, 
muss also noch Stetigkeit von $\sgn(\partial_r u)$ und 
$\partial_r(\sgn(\partial_r u))$ verlangt werden und 
$\partial_r(\sgn(\partial_r u(1))=\sgn(\partial_r u(1))=0$.
Damit $f$ in $0$ definierbar ist, muss auch gelten 
$\sgn(\partial_r u) \in o(r)$ für $r\to 0$.

Damit erhält man für die Funktion
\begin{align*}
  u_1(r)\coloneqq
  \begin{cases}
    1, & \text{wenn } 0\leq r\leq\frac{1}{6},\\
    1+(6r-1)^\beta, & \text{wenn } \frac{1}{6}\leq r\leq\frac{1}{3},\\
    2, &\text{wenn } \frac{1}{3}\leq r\leq\frac{1}{2},\\
    2(\frac{5}{2}-3r)^\beta, &\text{wenn } \frac{1}{2}\leq r\leq\frac{5}{6},\\
    0, &\text{wenn } \frac{5}{6}\leq r,
  \end{cases}
\end{align*}
wobei $\beta\geq 1/2$, mit der Wahl
\begin{align*}
  \sgn(\partial_r u_1(r)) =
  \begin{cases}
    12r-36r^2, & \text{wenn } 0\leq r\leq\frac{1}{6},\\
    1, & \text{wenn } \frac{1}{6}\leq r\leq\frac{1}{3},\\
    \cos(\pi(6r-2)), &\text{wenn } \frac{1}{3}\leq r\leq\frac{1}{2},\\
    -1, &\text{wenn } \frac{1}{2}\leq r\leq\frac{5}{6},\\
    -\frac{1+\cos(\pi(6r-5))}{2}, &\text{wenn } \frac{5}{6}\leq r\leq 1,
  \end{cases}
\end{align*}
die rechte Seite
\begin{align*}
  f_1(r)\coloneqq 
  \begin{cases}
    \alpha-12(2-9r), & \text{wenn } 0\leq r\leq\frac{1}{6},\\
    \alpha(1+(6r-1)^\beta)-\frac{1}{r}, & \text{wenn } \frac{1}{6}\leq r\leq
    \frac{1}{3},\\
    2\alpha+6\pi\sin(\pi(6r-2))-\frac{1}{r}\cos(\pi(6r-2)), &
    \text{wenn } \frac{1}{3}\leq r\leq\frac{1}{2},\\
    2\alpha(\frac{5}{2}-3r)^\beta+\frac{1}{r},&
    \text{wenn } \frac{1}{2}\leq r\leq\frac{5}{6},\\
    -3\pi\sin(\pi(6r-5))+\frac{1+\cos(\pi(6r-5))}{2r}, &
    \text{wenn } \frac{5}{6}\leq r\leq 1.
  \end{cases}
\end{align*}

Für die Funktion
\begin{align*}
  u_2(r)\coloneqq 
  \begin{cases}
    1, & \text{wenn } 0\leq r\leq\frac{1-\beta}{2},\\
    -\frac{1}{\beta}r + \frac{1+\beta}{2\beta}, & 
    \text{wenn } \frac{1-\beta}{2}\leq r\leq \frac{1+\beta}{2},\\
    0, & \text{wenn } \frac{1+\beta}{2}\leq r,
  \end{cases}
\end{align*}
erhält man mit der Wahl
\begin{align*}
  \sgn&(\partial_r u_2(r)) \\
  &\coloneqq 
  \begin{cases}
    \frac{4}{1-\beta}r\left(\frac{1}{1-\beta}r -1\right), &
    \text{wenn } 0\leq r\leq\frac{1-\beta}{2},\\
    -1, & \text{wenn } \frac{1-\beta}{2}\leq r\leq \frac{1+\beta}{2},\\
    \frac{4}{(\beta-1)^3}
    \left( 4r^3-3(\beta+3)r^2 +6(\beta+1)r-3\beta-1\right), & 
    \text{wenn } \frac{1+\beta}{2}\leq r\leq 1,
  \end{cases}
\end{align*}
die rechte Seite
\begin{align*}
  f_2(r)\coloneqq 
  \begin{cases}
    \alpha - \frac{4}{1-\beta}\left(\frac{3}{1-\beta}r - 2\right), &
    \text{wenn } 0\leq r\leq\frac{1-\beta}{2},\\
    -\frac{\alpha}{\beta}\left( r-\frac{1+\beta}{2} \right) +\frac{1}{r}, & 
    \text{wenn } \frac{1-\beta}{2}\leq r\leq \frac{1+\beta}{2},\\
    \frac{-4}{(\beta-1)^3}
    \left( 16r^2 -9(\beta+3)r + 12(\beta+1) - \frac{3\beta+1}{r}\right), & 
    \text{wenn } \frac{1+\beta}{2}\leq r\leq 1.
  \end{cases}
\end{align*}

Es folgen zwei Beispiele mit exakter Lösung $u_3=u_4 \in H^2_0((0,1)^2)$, 
gegeben durch 
\begin{align*}
  u_3(r)=u_4(r)\coloneqq 
  \begin{cases}
    1, & \text{wenn } 0\leq r\leq\frac{1}{3},\\
    54r^3 - 81r^2 + 36r - 4, & 
    \text{wenn } \frac{1}{3}\leq r\leq \frac{2}{3},\\
    0, & \text{wenn } \frac{2}{3}\leq r.
  \end{cases}
\end{align*}
Mit der Wahl
\begin{align*}
  \sgn&(\partial_r u_3(r)) \\
  &\coloneqq 
  \begin{cases}
    54r^3-27r^2, & \text{wenn } 0\leq r\leq\frac{1}{3},\\
    -1, & \text{wenn } \frac{1}{3}\leq r\leq \frac{2}{3},\\
    -54r^3 + 135r^2 - 108r + 27, & \text{wenn } \frac{2}{3}\leq r\leq 1,
  \end{cases}
\end{align*}
erhalten wir die rechte Seite
\begin{align*}
  f_3(r)\coloneqq 
  \begin{cases}
    \alpha - 216r^2 + 81r, &
    \text{wenn } 0\leq r\leq\frac{1}{3},\\
    \alpha\left(54r^3 - 81r^2 + 36r - 4\right)) + \frac{1}{r}, & 
    \text{wenn } \frac{1}{3}\leq r\leq \frac{2}{3},\\
    216r^2 - 405r + 216 - \frac{27}{r}, & 
    \text{wenn } \frac{2}{3}\leq r\leq 1,
  \end{cases}
\end{align*}
für die gilt $f_3\in H^1_0$
und mit der Wahl
\begin{align*}
  \sgn&(\partial_r u_4(r)) \\
  &\coloneqq 
  \begin{cases}
    -1458r^5 + 1215r^4 - 270r^3, & \text{wenn } 0\leq r\leq\frac{1}{3},\\
    -1, & \text{wenn } \frac{1}{3}\leq r\leq \frac{2}{3},\\
    -243r^4 + 756r^3 - 864r^2 + 432r - 81, 
    & \text{wenn } \frac{2}{3}\leq r\leq 1,
  \end{cases}
\end{align*}
erhalten wir die rechte Seite
\begin{align*}
  f_4(r)\coloneqq 
  \begin{cases}
    \alpha + 8748r^4 - 6075r^3 + 1080r^2, &
    \text{wenn } 0\leq r\leq\frac{1}{3},\\
    \alpha\left(54r^3 - 81r^2 + 36r - 4\right) + \frac{1}{r}, & 
    \text{wenn } \frac{1}{3}\leq r\leq \frac{2}{3},\\
    1215r^3 - 3024r^2 + 2592r - 864 + \frac{81}{r}, & 
    \text{wenn } \frac{2}{3}\leq r\leq 1,
  \end{cases}
\end{align*}
für die gilt $f_4\in H^2_0$.

Damit können Experimente durchgeführt werden bei denen 
\texttt{exactSolutionKnown = true} gesetzt werden kann und entsprechend auch 
der $L^2$-Fehler berechnet wird.

Soll nun auch die Differenz der exakten Energie mit der garantierten unteren 
Energie Schranke (GLEB) berechnet werden, dann werden die stückweisen
Gradienten der exakten Lösung und der rechten Seite benötigt.

Dabei gelten folgende Ableitungsregeln für die Ableitungen einer Funktion 
$g$, wenn man ihr Argument $x=(x_1,x_2)\in\Rbb^2$ in Polarkoordinaten mit Länge
$r=\sqrt{x_1^2+x_2^2}$ und Winkel
$\varphi = \atan(x_2,x_1)$, wobei 
\begin{align*}
  \atan(x_2,x_1)\coloneqq
  \begin{cases}
    \arctan\left( \frac{x_2}{x_1} \right),& \text{wenn }x_1>0,\\
    \arctan\left( \frac{x_2}{x_1} \right) +\pi,& \text{wenn }x_1<0,x_2\geq 0,\\
    \arctan\left( \frac{x_2}{x_1} \right) -\pi,& \text{wenn }x_1<0,x_2<0,\\
    \frac{\pi}{2},& \text{wenn }x_1=0,x_2>0,\\
    -\frac{\pi}{2},& \text{wenn }x_1=0,x_2<0,\\
    \text{undefiniert},& \text{wenn }x_1=x_2=0,\\
  \end{cases}
\end{align*}
auffasst,
\begin{align*}
  \partial_{x_1} &= 
  \cos(\varphi)\partial_r - \frac{1}{r}\sin(\varphi)\partial_\varphi,\\
  \partial_{x_2} &= 
  \sin(\varphi)\partial_r - \frac{1}{r}\cos(\varphi)\partial_\varphi.
\end{align*}
Ist $g$ vom Winkel $\varphi$ unabhängig, so ergibt sich
\begin{align*}
  \nabla_{(x_1,x_2)}g = (\cos(\varphi),\sin(\varphi))\partial_r g.
\end{align*}
Unter Beachtung der trigonometrischen Zusammenhänge
\begin{align*}
  \sin(\arctan(y)) = \frac{y}{\sqrt{1+y^2}},\\
  \cos(\arctan(y)) = \frac{1}{\sqrt{1+y^2}}
\end{align*}
ergibt sich 
\begin{align*}
  (\cos(\varphi),\sin(\varphi)) = (x_1,x_2)\frac{1}{r}
\end{align*}
und damit 
\begin{align*}
  \nabla_{(x_1,x_2)}g = (x_1,x_2)\frac{\partial_r g}{r},
\end{align*} 
es muss also nur $\partial_r g$ bestimmt werden.

Die entsprechenden Ableitungen lauten
\begin{align*}
  \partial_r f_1(r)&=
  \begin{cases}
    108,&
    \text{für } r\in\left[0,\frac{1}{6}\right],\\
    6\alpha\beta(6r-1)^{\beta-1} +\frac{1}{r^2}, &
    \text{für } r\in\left[\frac{1}{6},\frac{1}{3}\right],\\
    (36\pi^2+\frac{1}{r^2})\cos(\pi(6r-2))+
    \frac{6\pi}{r}\sin(\pi(6r-2)), &
    \text{für } r\in\left[\frac{1}{3},\frac{1}{2}\right],\\
    -\left(6\alpha\beta\left( \frac{5}{2}-3r \right)^{\beta-1}+
    \frac{1}{r^2}\right),&
    \text{für } r\in\left[\frac{1}{2},\frac{5}{6}\right],\\
    -\left( \left( 18\pi^2+\frac{1}{2r^2} \right)\cos(\pi(6r-5))
    +\frac{1}{2r^2} + \frac{3\pi}{r}\sin(\pi(6r-5))\right), 
    &\text{für } r\in\left[\frac{5}{6},1\right],
  \end{cases}\\
  \partial_r u_1(r) &= 
  \begin{cases}
    0,&\text{wenn }0\leq r\leq\frac{1}{6},\\
    6\beta(6r-1)^{\beta-1}, &\text{wenn } \frac{1}{6}\leq r\leq\frac{1}{3},\\
    0, &\text{wenn } \frac{1}{3}\leq r\leq\frac{1}{2},\\
    -6\beta\left( \frac{5}{2}-3r \right)^{\beta-1},&
    \text{wenn } \frac{1}{2}\leq r\leq\frac{5}{6},\\
    0,&\text{wenn } \frac{5}{6}\leq r,
  \end{cases}\\
  \partial_r f_2(r) &= 
  \begin{cases}
    -\frac{12}{(1-\beta)^2},&\text{wenn }0\leq r\leq\frac{1-\beta}{2},\\
    -\frac{\alpha}{\beta}-\frac{1}{r^2},&
    \text{wenn } \frac{1-\beta}{2}\leq r\leq \frac{1+\beta}{2},\\
    -\frac{4}{(1-\beta)^3}\left( 32r-9(\beta+3)+\frac{3\beta+1}{r^2} \right),&
    \text{wenn } \frac{1+\beta}{2}\leq r\leq 1,\\
  \end{cases}\\
  \partial_r u_2(r) &= 
  \begin{cases}
    0,&\text{wenn }0\leq r\leq\frac{1-\beta}{2},\\
    -\frac{1}{\beta},&
    \text{wenn } \frac{1-\beta}{2}\leq r\leq \frac{1+\beta}{2},\\
    0,&\text{wenn } \frac{1+\beta}{2}\leq r,
  \end{cases}\\
  \partial_r f_3(r) &=
  \begin{cases}
    - 432r + 81, & \text{wenn } 0\leq r\leq\frac{1}{3},\\
    \alpha\left(162r^2 - 162r + 36\right) - \frac{1}{r^2}, & 
    \text{wenn } \frac{1}{3}\leq r\leq \frac{2}{3},\\
    432r - 405 + \frac{27}{r^2}, & 
    \text{wenn } \frac{2}{3}\leq r\leq 1,
  \end{cases}\\
  \partial_r f_4(r) &=
  \begin{cases}
    34992r^3 - 18225r^2 + 2160r, & \text{wenn } 0\leq r\leq\frac{1}{3},\\
    \alpha\left(162r^2 - 162r + 36\right) - \frac{1}{r^2}, & 
    \text{wenn } \frac{1}{3}\leq r\leq \frac{2}{3},\\
    3645r^2 - 6048r + 2592 - 864 - \frac{81}{r^2}, & 
    \text{wenn } \frac{2}{3}\leq r\leq 1,
  \end{cases}\\
  \partial_r u_{3,4}(r) &=
  \begin{cases}
    0, & \text{wenn } 0\leq r\leq\frac{1}{3},\\
    162r^2 - 162r + 36, & 
    \text{wenn } \frac{1}{3}\leq r\leq \frac{2}{3},\\
    0, & \text{wenn } \frac{2}{3}\leq r\leq 1,
  \end{cases}\\
\end{align*}

Mit diesen Informationen kann mit \texttt{computeExactEnergyBV.m} die exakte 
Energie berechnet werden und somit durch eintragen der exakten Energie
in die Variable \texttt{exact\-Energy} im Benchmark und setzen der Flag
\texttt{useExactEnergy=true} das Experiment durch anschließendes Ausführen
von \texttt{startAlgorithmCR.m} gestartet werden.
