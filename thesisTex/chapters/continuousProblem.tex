\section{Formulierung}
Für einen Parameter $\alpha\in\Rbb_+$ und eine Funktion
$f\in L^2(\Omega)$ betrachten wir das folgende Minimierungsproblem. 

\begin{problem}\label{prob:continuousProblem}
  Finde $u\in \BV(\Omega)\cap L^2(\Omega)$, sodass
  $u$ das Funktional
  \begin{align}\label{eq:continuousProblem}
    E(v)\coloneqq \frac{\alpha}{2}\Vert v\Vert^2 + |v|_{\BV(\Omega)}
    +\Vert v\Vert_{L^1(\partial\Omega)}-\int_\Omega fv\dx
  \end{align}
  unter allen $v\in\BV(\Omega)\cap L^2(\Omega)$ minimiert.

  Dabei ist der Term $\Vert v\Vert_{L^1(\partial\Omega)}$ wohldefiniert, da
  nach \cite[S. 400, Theorem 10.2.1]{ABM14} eine lineare, stetige Abbildung
  $T:\BV(\Omega)\to L^1(\partial\Omega)$ existiert mit $T(u) =
  u|_{\partial\Omega}$ für alle $u\in\BV(\Omega)\cap C(\overline\Omega)$.
\end{problem}

\begin{remark}
  Nach \cite[S. 399, Theorem 10.1.3]{ABM14} ist 
  \todo[inline]{noch fragen, was 1-regular nochmal heißt und ob das hier 
  glatt geht (tut es sehr wahrscheinlich)}
  die Einbettung $\BV(\Omega)\hookrightarrow L^p(\Omega)$ stetig für 
  $1\leq p\leq d/(d-1)$. 
  Damit ist $\BV(\Omega)$ für $d=2$ Teilmenge von $L^2(\Omega)$ und die
  Lösung von \Cref{prob:continuousProblem} kann in
  $\BV(\Omega)$ gesucht werden. Für beliebige $d\in\Nbb$, die wir in diesen
  Abschnitt betrachten, gilt dies im Allgemeinen nicht.
\end{remark}


\section{Existenz eines eindeutigen Minimierers}
Zunächst zeigen wir, dass \Cref{prob:continuousProblem} eine Lösung besitzt.
Dafür benötigen wir die folgende Formulierung der Youngschen Ungleichung.

\begin{lemma}[Youngsche Ungleichung]
  \label{lem:young}
  Seien $a,b\in\Rbb$ und $\varepsilon\in\Rbb_+$ beliebig. Dann gilt
  \begin{align*}
    ab\leq\frac{1}{\varepsilon}a^2+\frac{\varepsilon}{4}b^2. 
  \end{align*}
\end{lemma}

Außerdem wird im Beweis folgende Aussage benötigt, die direkt aus \cite[S. 183,
Theorem 1]{EG92} folgt, da
$0\in\BV\left(\Rbb^d\setminus\overline\Omega\right)$,
$|0|_{\BV\left(\Rbb^d\setminus\overline\Omega\right)}=0$ und
$0|_{\partial\Omega}=0$.

\begin{lemma}
  \label{lem:bvExtension}
  Sei $v\in\BV(\Omega)$.
  Definiere, für alle $x\in\Rbb^d$,
  \begin{align*}
    \tilde{v}(x)\coloneqq
    \begin{cases}
      v(x),  &\text{ falls } x\in\Omega,\\
      0,     &\text{ falls } x\in\Rbb^d\setminus\overline\Omega.
    \end{cases} 
  \end{align*}
  Dann gilt $\tilde{v}\in\BV\left(\Rbb^d\right)$ und
  $\left|\tilde{v}\right|_{\BV\left(\Rbb^d\right)}
  = |v|_{\BV(\Omega)}+\Vert v\Vert_{L^1(\partial\Omega)}$.
\end{lemma}

\begin{theorem}[Existenz einer Lösung]
  \label{thm:contProblemExistence}
  \Cref{prob:continuousProblem} besitzt eine Lösung \\$u\in\BV(\Omega)\cap
  L^2(\Omega)$.
\end{theorem}

\begin{proof}
  Die Beweisidee ist die Anwendung der direkten Methode der Variationsrechnung
  (cf.\ z.B.\ \cite{Dac89}) unter Nutzung der in \Cref{sec:bvFunctions}
  aufgeführten Eigenschaften der schwachen Konvergenz in $\BV(\Omega)$.

  Für alle $v\in L^2(\Omega)\subseteq L^1(\Omega)$ gilt mit der Hölderschen
  Ungleichung für $p=q=2$, dass
  \begin{equation}\label{eq:hoelderL2BiggerL1}
    \Vert v\Vert_{L^1(\Omega)} 
    = \Vert 1\cdot v\Vert_{L^1(\Omega)}
    \leq \Vert 1\Vert\Vert v\Vert
    =\sqrt{|\Omega|} \Vert v\Vert.
  \end{equation}
  Dann folgt für das Funktional $E$ in \eqref{eq:continuousProblem} für alle
  $v\in \BV(\Omega)\cap L^2(\Omega)$ durch die Cauchy-Schwarzsche Ungleichung,
  die Youngsche Ungleichung aus \cref{lem:young} und Ungleichung
  \eqref{eq:hoelderL2BiggerL1}, dass

  \begin{equation}
    \label{eq:contProbBddFromBelow}
    \begin{aligned}
      E(v)&=\frac{\alpha}{2}\Vert v\Vert^2 + |v|_{\BV(\Omega)}
      +\Vert v\Vert_{L^1(\partial\Omega)}-\int_\Omega fv\dx\\
      &\geq 
      \frac{\alpha}{2}\Vert v\Vert^2 + |v|_{\BV(\Omega)}
      +\Vert v\Vert_{L^1(\partial\Omega)}
      -\Vert f\Vert\Vert v\Vert\\
      &\geq 
      \frac{\alpha}{2}\Vert v\Vert^2 + |v|_{\BV(\Omega)}
      +\Vert v\Vert_{L^1(\partial\Omega)}
      -\frac{1}{\alpha}\Vert f\Vert^2
      -\frac{\alpha}{4}\Vert v\Vert^2\\
      &\geq 
      \frac{\alpha}{4}\Vert v\Vert^2 + |v|_{\BV(\Omega)}
      +\Vert v\Vert_{L^1(\partial\Omega)}-\frac{1}{\alpha}\Vert
      f\Vert^2\\
      &\geq 
      \frac{\alpha}{4|\Omega|}\Vert v\Vert_{L^1(\Omega)}^2 + |v|_{\BV(\Omega)}
      +\Vert v\Vert_{L^1(\partial\Omega)}-\frac{1}{\alpha}\Vert
      f\Vert^2\\
      &\geq -\frac{1}{\alpha}\Vert f\Vert^2.
    \end{aligned}
  \end{equation}
  Somit ist $E$ nach unten beschränkt, was die Existenz einer infimierenden
  Folge $(u_n)_{n\in\Nbb}\subset\BV(\Omega)\cap L^2(\Omega)$ von $E$ 
  impliziert, das heißt
  $(u_n)_{n\in\Nbb}$ erfüllt $$\lim_{n\rightarrow\infty}E(u_n) =
  \inf_{v\in\BV(\Omega)\cap L^2(\Omega)}E(v).$$ 

  Ungleichung \eqref{eq:contProbBddFromBelow} impliziert außerdem, dass
  $E(u_n)\to\infty$ für $n\to\infty$, falls $|u_n|_{\BV(\Omega)}\to\infty$ oder
  $\Vert u_n\Vert_{L^1(\Omega)}\to\infty$ für $n\to\infty$. 
  Daraus folgt insbesondere, dass $E(u_n)\to\infty$ für $n\to\infty$, falls
  $\Vert u_n\Vert_{\BV(\Omega)}\to\infty$ für $n\to\infty$ .
  Deshalb muss die Folge $(u_n)_{n\in\Nbb}$ beschränkt in $\BV(\Omega)$ sein.

  Nun garantiert \cref{thm:compactness} die Existenz einer in $\BV(\Omega)$
  schwach konvergenten Teilfolge $(u_{n_k})_{k\in\Nbb}$ von $(u_n)_{n\in\Nbb}$
  mit schwachen Grenzwert $u\in\BV(\Omega)$. 
  Ohne Beschränkung der Allgemeinheit ist
  $(u_{n_k})_{k\in\Nbb}=(u_n)_{n\in\Nbb}$.
  Aus der schwachen Konvergenz von $(u_n)_{n\in\Nbb}$ in $\BV(\Omega)$ gegen
  $u$ folgt nach Definition, dass $(u_n)_{n\in\Nbb}$ stark, und damit
  insbesondere auch schwach, in $L^1(\Omega)$ gegen $u$ konvergiert.

  Weiterhin folgt aus (\ref{eq:contProbBddFromBelow}), dass
  $E(v)\rightarrow\infty$ für $\Vert v\Vert\rightarrow\infty$. 
  Somit muss $(u_n)_{n\in\Nbb}$ auch beschränkt sein bezüglich der Norm
  $\Vert\bullet\Vert$ und besitzt deshalb, wegen der Reflexivität von
  $L^2(\Omega)$, eine Teilfolge (ohne Beschränkung der Allgemeinheit weiterhin
  bezeichnet mit $(u_n)_{n\in\Nbb}$), die in $L^2(\Omega)$ schwach gegen einen
  Grenzwert $\overline{u}\in L^2(\Omega)$ konvergiert. 
  Damit gilt für alle $w\in L^2(\Omega)\cong L^2(\Omega)^\ast$ und, da
  $L^\infty(\Omega)\subseteq L^2(\Omega)$, insbesondere auch für alle $w\in
  L^\infty(\Omega)\cong L^1(\Omega)^\ast$, dass 
  \begin{align*}
    \lim_{n\to\infty}\int_\Omega u_n w\dx =\int_\Omega \overline{u} w\dx.
  \end{align*}
  Das bedeutet, dass $(u_n)_{n\in\Nbb}$ auch schwach in $L^1(\Omega)$ gegen
  $\overline{u}\in L^2(\Omega)\subseteq L^1(\Omega)$ konvergiert. 

  Da schwache Grenzwerte eindeutig bestimmt sind, gilt insgesamt $u=\overline u
  \in L^2(\Omega)$, das heißt $u\in\BV(\Omega)\cap
  L^2(\Omega)$.

  Nun definieren wir für alle
  $n\in\Nbb$ und für alle 
  $x\in\Rbb^d$
  \begin{align*}
    \tilde{u}_n(x)
    &\coloneqq
    \begin{cases}
      u_n(x),  &\text{ falls } x\in\Omega,\\
      0,     &\text{ falls } x\in\Rbb^d\setminus\overline\Omega
    \end{cases} 
    &&\text{und }
    &\tilde{u}(x)
    &\coloneqq
    \begin{cases}
      u(x),  &\text{ falls } x\in\Omega,\\
      0,     &\text{ falls } x\in\Rbb^d\setminus\overline\Omega.
    \end{cases} 
  \end{align*}

  Dann gilt nach \cref{lem:bvExtension} sowohl
  \begin{align*}
    \tilde{u}_n
    &\in
    \BV\left(\Rbb^d\right)
    &&\text{und}
    &\left|\tilde{u}_n\right|_{\BV\left(\Rbb^d\right)} 
    &= 
    |u_n|_{\BV(\Omega)}+\Vert u_n\Vert_{L^1(\partial\Omega)}
    \quad\text{für alle }n\in\Nbb \text{ als auch}\\
    \tilde{u}
    &\in
    \BV\left(\Rbb^d\right)
    &&\text{und}
    &\left|\tilde{u}\right|_{\BV\left(\Rbb^d\right)} 
    &=
    |u|_{\BV(\Omega)}+\Vert u\Vert_{L^1(\partial\Omega)}.
  \end{align*}
  Da $(u_n)_{n\in \Nbb}$ infimierende Folge von $E$ ist, muss die Folge
  \begin{align*}
    \left(\left|\tilde{u}_n\right|_{\BV\left(\Rbb^d\right)}\right)_{n\in\Nbb} 
    = \left(|u_n|_{\BV(\Omega)}+
    \Vert u_n\Vert_{L^1(\partial\Omega)}\right)_{n\in\Nbb}
  \end{align*}
  beschränkt sein.
  Außerdem folgt aus den Definitionen von $\tilde{u}$ und 
  $\tilde{u}_n$ für alle $n\in\Nbb$ und der bereits bekannten Eigenschaft 
  $u_n\to u$ in $L^1(\Omega)$ für $n\to\infty$, dass
  \begin{align*}
    \left\Vert \tilde{u}_n - \tilde{u}\right\Vert_{L^1\left(\Rbb^d\right)} 
    &= \int_{\Rbb^d} \left|\tilde{u}_n - \tilde{u}\right|\dx
    = \int_\Omega |u_n - u|\dx
    = \Vert u_n - u\Vert_{L^1(\Omega)}\to 0\quad\text{für }n\to\infty,
  \end{align*}
  das heißt $\tilde{u}_n \to \tilde{u}$ in $L^1\left(\Rbb^d\right)$ für
  $n\to\infty$.

  Insgesamt ist also $\left(\tilde{u}_n\right)_{n\in\Nbb}$ eine Folge in
  $\BV\left(\Rbb^d\right)$, die in $L^1\left(\Rbb^d\right)$ gegen
  $\tilde{u}\in\BV\left(\Rbb^d\right)\subseteq L^1\left(\Rbb^d\right)$
  konvergiert und 
  $\sup_{n\in\Nbb} \left|\tilde{u}_n\right|_{\BV\left(\Rbb^d\right)}<\infty$
  erfüllt.
  Somit folgt mit
  \cref{thm:wlsc}  
  \begin{equation}
    \label{eq:wlscOfExtension}
    \begin{aligned}
      |u|_{\BV(\Omega)} +\Vert u\Vert_{L^1(\partial\Omega)}
      = \left|\tilde{u}\right|_{\BV\left(\Rbb^d\right)}
      &\leq\liminf_{n\to\infty}
      \left|\tilde{u}_n\right|_{\BV\left(\Rbb^d\right)}\\
      &= \liminf_{n\to\infty} \left(|u_n|_{\BV(\Omega)} +
      \Vert u_n\Vert_{L^1(\partial\Omega)}\right).
    \end{aligned}
  \end{equation}

  Die Funktionen $\Vert\bullet\Vert^2$ und $-\int_\Omega
  f\bullet\dx$ sind auf $L^2(\Omega)$ stetig und konvex, was impliziert,
  dass sie schwach unterhalbstetig auf $L^2(\Omega)$ sind. Da wir bereits
  wissen, dass $u_n\rightharpoonup u$ in $L^2(\Omega)$ für $n\to\infty$, 
  folgt
  \begin{align*}
    \frac{\alpha}{2}\Vert u\Vert-\int_\Omega fu\dx
    \leq \liminf_{n\to\infty}
    \left(\frac{\alpha}{2}\Vert u_n\Vert
    -\int_\Omega fu_n\dx\right).
  \end{align*}
  
  Damit und mit Ungleichung \eqref{eq:wlscOfExtension} gilt insgesamt
  \begin{align*}
    \inf_{v\in\BV(\Omega)\cap L^2(\Omega)}E(v)\leq
    E(u)\leq\liminf_{n\rightarrow\infty} E\left(u_n\right) =
    \lim_{n\rightarrow\infty}E\left(u_n\right) = \inf_{v\in\BV(\Omega)\cap
    L^2(\Omega)}E(v),
  \end{align*}
  das heißt $\min_{v\in\BV(\Omega)\cap L^2(\Omega)} E(v) = E(u)$.
\end{proof}

Nachdem wir gezeigt haben, dass für \Cref{prob:continuousProblem} eine
Lösung existiert, beweisen wir als nächstes ein Theorem, das direkt impliziert,
dass diese Lösung eindeutig ist. 
\begin{theorem}[Stabilität und Eindeutigkeit]
  \label{thm:contProbStabAndUniqu}
  Seien $u_1,u_2\in \BV(\Omega)\cap L^2(\Omega)$ die Minimierer des Problems
  \ref{prob:continuousProblem} mit $f_1,f_2\in L^2(\Omega)$ anstelle von $f$,
  das heißt für $\ell\in\{1,2\}$ minimiere $u_\ell$ das Funktional
  \begin{align*}
    E_\ell
    \coloneqq 
    \frac{\alpha}{2}\Vert v\Vert^2 + |v|_{\BV(\Omega)} 
    + \Vert v\Vert_{L^1(\partial\Omega)} - \int_\Omega f_\ell v\dx
  \end{align*}
  unter allen $v\in\BV(\Omega)\cap L^2(\Omega)$.

  Dann gilt 
  \begin{align*}
    \Vert u_1 - u_2\Vert 
    \leq\frac{1}{\alpha}\Vert f_1-f_2\Vert.
  \end{align*}
\end{theorem}

\begin{proof}
  Wir folgen der Argumentation im Beweis von \cite[S. 304, Theorem 10.6]{Bar15}.

  Zunächst definieren wir die Funktionale $F: L^2(\Omega)\to
  \Rbb\cup\{\infty\}$ und $G_\ell:L^2(\Omega)\to \Rbb$, $\ell\in\{1,2\}$, für
  alle
  $u\in L^2(\Omega)$ durch
  \begin{align*}
    F(u) 
    &\coloneqq 
    \begin{cases}
      |u|_{\BV(\Omega)} + \Vert u \Vert_{L^1(\partial\Omega)}, 
      &\text{ falls } u\in\BV(\Omega)\cap L^2(\Omega),\\
      \infty,&\text{ falls } u\in L^2(\Omega)\setminus\BV(\Omega)
    \end{cases}
    \quad\text{und }\\
    G_\ell(u)
    &\coloneqq 
    \frac{\alpha}{2}\Vert u\Vert^2 - \int_\Omega f_\ell u\dx.
  \end{align*}
  Damit gilt für $\ell\in\{1,2\}$ und alle
  $u\in\BV(\Omega)\cap L^2(\Omega)$, dass $E_\ell(u) =  F(u)+G_\ell(u)$.

  Für $\ell\in\{1,2\}$ ist $G_\ell$ Fr\'echet-differenzierbar und die
  Fr\'echet-Ableitung $G_\ell'(u): L^2(\Omega)\to\Rbb$ von $G_\ell$ an der
  Stelle $u\in L^2(\Omega)$ ist für alle $v\in L^2(\Omega)$
  gegeben durch
  \begin{align*}
    dG_\ell(u;v) = \alpha (u,v) - \int_\Omega f_\ell v\dx 
    = (\alpha u-f_\ell ,v).
  \end{align*}

  Das Funktional $F$ ist konvex, unterhalbstetig und es gilt $F\nequiv\infty$.
  Deshalb ist nach \cref{thm:subdifferentialMonotonicity} das Subdifferential
  $\partial F$ von $F$ monoton, das heißt für alle $\mu_\ell\in \partial
  F(u_\ell)$, $\ell\in\{1,2\}$, gilt
  \begin{align}\label{eq:stabilityAndUniqueness:monotonicityOfSubdifferential}
    (\mu_1-\mu_2,u_1-u_2)\geq 0.
  \end{align}

  Für $\ell\in\{1,2\}$ gilt, dass $E_\ell$ konvex ist und von $u_\ell$ in
  $\BV(\Omega)\cap L^2(\Omega)$ minimiert wird. 
  Außerdem gilt $E_\ell\nequiv\infty$ und $G_\ell$ ist stetig.
  Somit gilt nach \cref{thm:extremalprinciple}, 
  \cref{thm:subdifferentialSumRule} und \cref{thm:subdiffGateaux}, dass
  $0\in\partial E_\ell(u_\ell) = \partial F(u_\ell)+\partial
  G_\ell(u_\ell)=\partial F(u_\ell)+ \{G_\ell'(u_\ell)\}.$ 
  Daraus folgt
  $-G_\ell'(u_\ell)\in\partial F(u_\ell)$.
  Zusammen mit Ungleichung
  \eqref{eq:stabilityAndUniqueness:monotonicityOfSubdifferential}
  impliziert das
  \begin{align*}
    \big( -(\alpha u_1 - f_1) -(- (\alpha u_2 - f_2)), u_1 - u_2\big)
    \geq 0.
  \end{align*}
  Durch Umformen und Anwenden der Cauchy-Schwarzschen Ungleichung erhalten wir
  \begin{align*}
    \alpha \Vert u_1 - u_2 \Vert^2
    &\leq
    \big(f_1 -f_2, u_1-u_2 \big)\\
    &\leq
    \Vert f_1-f_2\Vert\Vert u_1 - u_2\Vert.
  \end{align*}

  Falls $\Vert u_1 - u_2 \Vert = 0$, gilt die zu zeigende Aussage.
  Ansonsten führt Division durch $\alpha\Vert u_1 - u_2 \Vert\neq 0$ den
  Beweis zum Abschluss.
\end{proof}
