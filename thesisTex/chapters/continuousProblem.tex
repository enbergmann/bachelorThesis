Betrachte für gegebenes $\alpha>0$ und rechte Seite $f\in L^2(\Omega)$ das 
folgende Minimierungsproblem. 

\begin{problem}\label{prob:continuousProblem}
  Finde $u\in \BV(\Omega)\cap L^2(\Omega)$, sodass
  $u$ das Funktional
  \begin{align}\label{eq:continuousProblem}
    E(v)\coloneqq \frac{\alpha}{2}\Vert v\Vert_{L^2(\Omega)} + |v|_{\BV(\Omega)}
    +\Vert v\Vert_{L^1(\partial\Omega)}-\int_\Omega fv\dx
  \end{align}
  unter allen $v\in V\coloneqq\BV(\Omega)\cap L^2(\Omega)$ minimiert.
\end{problem}

\begin{theorem}[Existenz einer Lösung]
  \label{thm:contProblemExistence}
  \Cref{prob:continuousProblem} besitzt eine Lösung \\$u\in\BV(\Omega)\cap
  L^2(\Omega)$.
\end{theorem}

\begin{proof}
  Das Funktional $E$ in \eqref{eq:continuousProblem} ist nach unten beschränkt,
  denn für alle $v\in \BV(\Omega)\cap L^2(\Omega)$ gilt mit der Cauchy-Schwarzschen
  Ungleichung und der Youngschen Ungleichung
  \begin{equation}
    \label{eq:contProbBddFromBelow}
    \begin{aligned}
      E(v)&=\frac{\alpha}{2}\Vert v\Vert_{L^2(\Omega)} + |v|_{\BV(\Omega)}
      +\Vert v\Vert_{L^1(\partial\Omega)}-\int_\Omega fv\dx\\
      &\geq 
      \frac{\alpha}{4}\Vert v\Vert_{L^2(\Omega)} + |v|_{\BV(\Omega)}
      +\Vert v\Vert_{L^1(\partial\Omega)}-\frac{1}{\alpha}\Vert
      f\Vert_{L^2(\Omega)}^2\\
      &\geq -\Vert f\Vert_{L^2(\Omega)}^2.
    \end{aligned}
  \end{equation}
  Somit existiert eine infimierende Folge
  $(u_n)_{n\in\Nbb}\subset\BV(\Omega)\cap
  L^2(\Omega)$ von $E$, d.h.\ $(u_n)_{n\in\Nbb}$ erfüllt
  $\lim_{n\rightarrow\infty}E(u_n) =
  \inf_{v\in\BV(\Omega)\cap
    L^2(\Omega)}E(v)$. 
    % bezüglich der Normen $\Vert\bullet\Vert_{BV(\Omega)}$ 
    % und $\Vert\bullet\Vert_{L^2(\Omega)}$.

  \Cref{eq:contProbBddFromBelow} impliziert außerdem, dass
  $E(v)\rightarrow\infty$, falls
  $\Vert v\Vert_{\BV(\Omega)}\rightarrow\infty$.
  Die Folge $(u_n)_{n\in\Nbb}$ muss also insbesondere beschränkt sein.

  \medbreak
  Nun garantiert \cref{thm:compactness} die Existenz einer schwach konvergenten
  Teilfolge $(u_{n_k})_{k\in\Nbb}$ von $(u_n)_{n\in\Nbb}$ mit schwachem Grenzwert
  $u\in\BV(\Omega)$. O.B.d.A.\ ist $(u_{n_k})_{k\in\Nbb}=(u_n)_{n\in\Nbb}$.

  Analog ist $(u_n)_{n\in\Nbb}$ nach \cref{eq:contProbBddFromBelow} ebenfalls
  beschränkt bezüglich der Norm $\Vert\bullet\Vert_{L^2(\Omega)}$, besitzt also
  eine Teilfolge (O.B.d.A.\ weiterhin bezeichnet mit $(u_n)_{n\in\Nbb}$), die
  in $L^2(\Omega)$ schwach gegen ein $\tilde{u}\in L^2(\Omega)$ konvergiert.
  Da die Einbettung $L^2(\Omega)\hookrightarrow L^1(\Omega)$ kompakt ist, 
  %TODO besser zitieren, Rellich Kondrachov erwähnen oder sogar nachrechnen?
  existiert eine Teilfolge (O.B.d.A\ $(u_n)_{n\in\Nbb}$), die stark gegen
  $\tilde u$ konvergiert bezüglich der Norm $\Vert\bullet\Vert_{L^1(\Omega)}$.

  Allerdings bedeutet die schwache Konvergenz $u_n\rightharpoonup u$ in
  $\BV(\Omega)$  
  insbesondere, dass $u_n\rightarrow u$ in $L^1(\Omega)$.
  Es gilt also $u=\tilde u \in L^2(\Omega)$, d.h.\ $u\in\BV(\Omega)\cap
  L^2(\Omega)$.

  \medbreak
  \cref{thm:wlsc} liefert die schwache Unterhalbstetigkeit der Seminorm
  $|\bullet|_{\BV(\Omega)}$ \\
  bezüglich schwacher Konvergenz in $\BV(\Omega)$.
  %und $\Vert\bullet\Vert_{L^2(\Omega)}$ und $-\int_\Omega f\bullet\dx$ sind 
  %stetig (d.h.\ insbesondere 
  \bigbreak
  todo: Randterm sufs? Die beiden verbleibenden Terme sind ufs, aber auch sufs?
  \bigbreak
  Damit gilt insgesamt
  \begin{align*}
    \inf_{v\in\BV(\Omega)\cap L^2(\Omega)}E(v)\leq
    E(u)\leq\liminf_{n\rightarrow\infty} E(u_n) =
    \lim_{n\rightarrow\infty}E(u_n) = \inf_{v\in\BV(\Omega)\cap
    L^2(\Omega)}E(v),
  \end{align*}
  d.h.\ $\min_{v\in\BV(\Omega)\cap L^2(\Omega)} E(v) = E(u)$.
\end{proof}

\begin{theorem}[Stabilität und Eindeutigkeit]
  \label{thm:contProbStabAndUniqu}
  Seien $u_1,u_2\in \BV(\Omega)\cap L^2(\Omega)$ die Minimierer des Problems
  \ref{prob:continuousProblem} mit $f_1,f_2\in L^2(\Omega)$ anstelle von $f$.

  Dann gilt 
  \begin{align*}
    \Vert u_1 - u_2\Vert_{L^2(\Omega)} \leq \Vert f_1-f_2\Vert_{L^2(\Omega)}.
  \end{align*}
\end{theorem}

\begin{proof}
  Definiere die konvexen Funktionale $F:\BV(\Omega)\to \Rbb$ und 
  $G_\ell:L^2(\Omega)\to \Rbb$, $\ell=1,2$, durch
  \begin{align*}
    F(u) &\coloneqq |u|_{\BV(\Omega)} + \Vert u \Vert_{L^1(\partial\Omega)},&
    G_\ell(u)\coloneqq \frac{\alpha}{2}\Vert u\Vert_{L^2(\Omega)}^2 -
    \int_\Omega fu\dx.
  \end{align*}
  Bezeichne $E_\ell\coloneqq F+E_\ell$ und setze $F$ auf $L^2(\Omega)$
  durch $\infty$ fort.

  $G_\ell$ ist Fr\'echet-differenzierbar mit Fr\'echet-Ableitung
  \begin{align*}
    \delta G_\ell(u)[v] = \alpha (u,v)_{L^2(\Omega)} - \int_\Omega fv\dx 
    = (\alpha u-f,v)_{L^2(\Omega)} \quad \text{ für alle } v\in L^2(\Omega).
  \end{align*}

  Das Funktional $F$ ist konvex, deshalb (TODO quote Rf) ist das Subdifferential
  %TODO quote
  $\partial F$ von $F$ monoton, d.h.\ für alle $\mu_\ell\in \partial F(u_\ell)$,
  $\ell=1,2$, gilt
  \begin{align*}
    (\mu_1-\mu_2,u_1-u_2)_{L^2(\Omega)}\geq 0.
  \end{align*}

  Für $\ell=1,2$ wird $E_\ell$ von $u_\ell$ minimiert, 
  deshalb gilt $0\in\partial E_\ell(u_\ell)
  = \partial F(u_\ell)+\partial G_\ell(u_\ell)=\partial F(u_\ell)+
  \{\delta G_\ell(u_\ell)\}$ (TODO quote) und somit folgt
  $-\delta G_\ell(u_\ell)\in\partial F(u_\ell)$.
  Daraus folgt
\end{proof}
