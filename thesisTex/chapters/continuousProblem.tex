Betrachte für gegebene $\alpha>0$ und $f\in L^2(\Omega)$ das 
folgende Minimierungsproblem. 

\begin{problem}\label{prob:continuousProblem}
  Finde $u\in \BV(\Omega)\cap L^2(\Omega)$, sodass
  $u$ das Funktional
  \begin{align}\label{eq:continuousProblem}
    E(v)\coloneqq \frac{\alpha}{2}\Vert v\Vert_{L^2(\Omega)}^2 + |v|_{\BV(\Omega)}
    +\Vert v\Vert_{L^1(\partial\Omega)}-\int_\Omega fv\dx
  \end{align}
  unter allen $v\in\BV(\Omega)\cap L^2(\Omega)$ minimiert.
\end{problem}

\begin{remark}
  \todo[inline]{TODO Vielleicht auch erst beim Diskreten Problem,
  da es dort Nullranddaten gibt? Beachte insbesondere, dass $f=\alpha g$ }
  In \cite[Kapitel~10.1.3]{Bar15} wird \Cref{prob:continuousProblem} für ein
  gegebenes $g\in L^2(\Omega)$ formuliert
  mit dem Funktional 
  \begin{align*}
    I(v)\coloneqq |v|_{\BV(\Omega)} + \frac{\alpha}{2}\int_\Omega (v-g)^2\dx
  \end{align*}
  für $v\in \BV(\Omega)\cap L^2(\Omega)$.
  Für $f = \alpha g$ gilt
  $I(v) = E(v) - \Vert v\Vert_{L^1(\partial \Omega)}+ 
  \frac{\alpha}{2}\Vert g\Vert_{L^2(\Omega)}^2$ für alle 
  $v\in \BV(\Omega)\cap L^2(\Omega)$. Da der Term $\frac{\alpha}{2}\Vert
  g\Vert_{L^2(\Omega)}^2$ konstant ist, haben beide Funktionale die
  gleichen Minimierer in $\left\{v\in\BV(\Omega)\cap L^2(\Omega)\mid 
  \Vert v\Vert_{L^1(\partial\Omega)}=0\right\}$.
\end{remark}

\begin{theorem}[Existenz einer Lösung]
  \label{thm:contProblemExistence}
  \Cref{prob:continuousProblem} besitzt eine Lösung \\$u\in\BV(\Omega)\cap
  L^2(\Omega)$.
\end{theorem}

\begin{proof}
  Das Funktional $E$ in \eqref{eq:continuousProblem} ist nach unten beschränkt,
  denn für alle $v\in \BV(\Omega)\cap L^2(\Omega)$ gilt mit der
  Cauchy-Schwarzschen Ungleichung und der Youngschen Ungleichung
  \todo[inline]{TODO Ungleichungen erwähnen/zitieren in Kapitel 1}
  \begin{equation}
    \label{eq:contProbBddFromBelow}
    \begin{aligned}
      E(v)&=\frac{\alpha}{2}\Vert v\Vert_{L^2(\Omega)}^2 + |v|_{\BV(\Omega)}
      +\Vert v\Vert_{L^1(\partial\Omega)}-\int_\Omega fv\dx\\
      &\geq 
      \frac{\alpha}{2}\Vert v\Vert_{L^2(\Omega)}^2 + |v|_{\BV(\Omega)}
      +\Vert v\Vert_{L^1(\partial\Omega)}
      -\Vert f\Vert_{L^2(\Omega)}\Vert v\Vert_{L^2(\Omega)}\\
      &\geq 
      \frac{\alpha}{2}\Vert v\Vert_{L^2(\Omega)}^2 + |v|_{\BV(\Omega)}
      +\Vert v\Vert_{L^1(\partial\Omega)}
      -\frac{1}{\alpha}\Vert f\Vert_{L^2(\Omega)}^2
      -\frac{\alpha}{4}\Vert v\Vert_{L^2(\Omega)}^2\\
      &\geq 
      \frac{\alpha}{4}\Vert v\Vert_{L^2(\Omega)}^2 + |v|_{\BV(\Omega)}
      +\Vert v\Vert_{L^1(\partial\Omega)}-\frac{1}{\alpha}\Vert
      f\Vert_{L^2(\Omega)}^2\\
      &\geq -\frac{1}{\alpha}\Vert f\Vert_{L^2(\Omega)}^2.
    \end{aligned}
  \end{equation}
  Somit existiert eine infimierende Folge
  $(u_n)_{n\in\Nbb}\subset\BV(\Omega)\cap
  L^2(\Omega)$ von $E$, d.h.\ $(u_n)_{n\in\Nbb}$ erfüllt
  $\lim_{n\rightarrow\infty}E(u_n) =
  \inf_{v\in\BV(\Omega)\cap
    L^2(\Omega)}E(v)$. 

  \Cref{eq:contProbBddFromBelow} impliziert außerdem, dass
  $E(u_n)\rightarrow\infty$ falls
  $|u_n|_{\BV(\Omega)}\rightarrow\infty$ für $n\to\infty$.
  Falls andererseits $|u_n|_{\BV(\Omega)}\leq c$ für alle $n\in\Nbb$ und ein
  $c\in\Rbb_{+}$ aber $\Vert u_n\Vert_{\BV(\Omega)}\to\infty$ für $n\to\infty$,
  folgt nach Definition von $\Vert\bullet\Vert_{\BV(\Omega)}$, dass
  $\int_\Omega|u_n|\dx = \Vert u_n\Vert_{L^1(\Omega)}\to \infty$ für
  $n\to\infty$. Dafür ist für fast alle $x\in\Rbb$ notwendig, dass
  $|u_n(x)|\to\infty$ für $n\to\infty$, 
  also insbesondere $|u_n(x)|\geq 1$ für hinreichend große $n\in\Nbb$. Daraus 
  folgt für solche hinreichend große $n\in\Nbb$, dass $\Vert
  u_n\Vert_{L^1(\Omega)} = \int_\Omega|u_n|\dx \leq
  \int_\Omega |u_n|^2\dx=\Vert u_n\Vert^2_{L^2(\Omega)}$ und somit ebenfalls
  durch \Cref{eq:contProbBddFromBelow} $E(u_n)\to\infty$ für
  $n\to\infty$.

  Die Folge $(u_n)_{n\in\Nbb}$ muss also beschränkt in $\BV(\Omega)$ sein, da 
  sonst $(E(u_n))_{n\in\Nbb}$ bestimmt divergiert, im Widerspruch dazu, dass
  $(u_n)_{n\in\Nbb}$ infimierende Folge ist.

  \medbreak
  Nun garantiert \cref{thm:compactness} die Existenz einer schwach konvergenten
  Teilfolge $(u_{n_k})_{k\in\Nbb}$ von $(u_n)_{n\in\Nbb}$ mit schwachem Grenzwert
  $u\in\BV(\Omega)$. Ohne Beschränkung der Allgemeinheit
  ist $(u_{n_k})_{k\in\Nbb}=(u_n)_{n\in\Nbb}$.

  Aus \cref{eq:contProbBddFromBelow} folgt 
  $E(v)\rightarrow\infty$ für
  $\Vert v\Vert_{L^2(\Omega)}\rightarrow\infty$. Somit muss $(u_n)_{n\in\Nbb}$ 
  auch beschränkt sein bezüglich der Norm $\Vert\bullet\Vert_{L^2(\Omega)}$ und
  besitzt deshalb wegen der Reflexivität von $L^2(\Omega)$
  eine Teilfolge (ohne Beschränkung der Allgemeinheit weiterhin bezeichnet mit
  $(u_n)_{n\in\Nbb}$), die
  in $L^2(\Omega)$ schwach gegen $\tilde{u}\in L^2(\Omega)$ konvergiert.

  Allerdings bedeutet die schwache Konvergenz $u_n\rightharpoonup u$ in
  $\BV(\Omega)$  
  insbesondere, dass $u_n\rightarrow u$ in $L^1(\Omega)$.
  Es gilt 
  \begin{align*}
    u_n \to u \text{ in } L^1(\Omega)&\Rightarrow
    \Vert u_n-u\Vert_{L^1(\Omega)}=\int_\Omega |u_n - u|\dx \to 0\\
    &\Rightarrow |u_n(x) - u(x)| \to 0 \text{ a.e. in } \Omega\\
    &\Rightarrow |u_n(x) - u(x)|^2 \to 0 \text{ a.e. in } \Omega\\
    &\Rightarrow \Vert u_n-u\Vert^2_{L^2(\Omega)}=
    \int_\Omega |u_n - u|^2\dx \to 0\\
    &\Rightarrow u_n \to u \text{ in } L^2(\Omega)\\
    &\Rightarrow u_n \rightharpoonup u \text{ in } L^2(\Omega).
  \end{align*}
  \todo[inline]{TODO wahrscheinlich falsch, Konvergenz in Norm impliziert
  i.A. nicht Konvergenz f.ü. in Omega oder umgekehrt.
  Das andere Argument war, dass L2 nach L1 kompakt sei, also die schwache
  Konvergenz in L2 wäre stark in L1, aber das liefert zumindest Sobolev nicht.
  Stimmt das jemals, also hier?}
  Da schwache Grenzwerte eindeutig bestimmt sind, gilt also $u=\tilde u
  \in L^2(\Omega)$, d.h.\ $u\in\BV(\Omega)\cap
  L^2(\Omega)$.

  \medbreak
  \cref{thm:wlsc} liefert die schwache Unterhalbstetigkeit der Seminorm
  $|\bullet|_{\BV(\Omega)}$ \\
  bezüglich schwacher Konvergenz in $\BV(\Omega)$.
  %und $\Vert\bullet\Vert_{L^2(\Omega)}$ und $-\int_\Omega f\bullet\dx$ sind 
  %stetig (d.h.\ insbesondere 
  \bigbreak
  %TODO
  \todo[inline]{Randterm sufs? Die beiden verbleibenden Terme sind ufs, aber
  auch sufs? Die Norm ist schwach unterhalbstetig. Es muss aber vorher 
  geklärt sein, welche Konvergenzen jetzt zutreffen um das untersuchen zu 
  können, also es muss geklärt werden, wie richtig bewiesen wird, dass der 
  Grenzwert auch in L2 liegt eben}
  \bigbreak
  Damit gilt insgesamt
  \begin{align*}
    \inf_{v\in\BV(\Omega)\cap L^2(\Omega)}E(v)\leq
    E(u)\leq\liminf_{n\rightarrow\infty} E(u_n) =
    \lim_{n\rightarrow\infty}E(u_n) = \inf_{v\in\BV(\Omega)\cap
    L^2(\Omega)}E(v),
  \end{align*}
  d.h.\ $\min_{v\in\BV(\Omega)\cap L^2(\Omega)} E(v) = E(u)$.
\end{proof}

\begin{theorem}[Stabilität und Eindeutigkeit]
  \label{thm:contProbStabAndUniqu}
  Seien $u_1,u_2\in \BV(\Omega)\cap L^2(\Omega)$ die Minimierer des Problems
  \ref{prob:continuousProblem} mit $f_1,f_2\in L^2(\Omega)$ anstelle von $f$.

  Dann gilt 
  \begin{align*}
    \Vert u_1 - u_2\Vert_{L^2(\Omega)} \leq \Vert f_1-f_2\Vert_{L^2(\Omega)}.
  \end{align*}
\end{theorem}

\begin{proof}
  Definiere die konvexen Funktionale $F:\BV(\Omega)\to \Rbb$ und 
  $G_\ell:L^2(\Omega)\to \Rbb$, $\ell=1,2$, durch
  \begin{align*}
    F(u) &\coloneqq |u|_{\BV(\Omega)} + \Vert u \Vert_{L^1(\partial\Omega)},&
    G_\ell(u)\coloneqq \frac{\alpha}{2}\Vert u\Vert_{L^2(\Omega)}^2 -
    \int_\Omega f_\ell u\dx.
  \end{align*}
  Bezeichne $E_\ell\coloneqq F+G_\ell$ und setze $F$ auf $L^2(\Omega)$
  durch $\infty$ fort.

  $G_\ell$ ist Fr\'echet-differenzierbar \todo[inline]{TODO nachrechnen,
  Gateaux ist klar, aber auch Frechet?} mit
  Fr\'echet-Ableitung
  \begin{align*}
    \delta G_\ell(u)[v] = \alpha (u,v)_{L^2(\Omega)} - \int_\Omega f_\ell v\dx 
    = (\alpha u-f_\ell ,v)_{L^2(\Omega)} \quad 
    \text{ für alle } v\in L^2(\Omega).
  \end{align*}

  Das Funktional $F$ ist konvex {\color{red} wegen Dreiecksungleichung und 
  homogenität ($\lambda$ und $1-\lambda$ sind größer gleich 0)}, deshalb 
  \todo[inline]{TODO quote Rockafella} ist das Subdifferential
  $\partial F$ von $F$ monoton, d.h.\ für alle $\mu_\ell\in \partial F(u_\ell)$,
  $\ell=1,2$, gilt
  \begin{align}\label{eq:stabilityAndUniqueness:monotonicityOfSubdifferential}
    (\mu_1-\mu_2,u_1-u_2)_{L^2(\Omega)}\geq 0.
  \end{align}

  Für $\ell=1,2$ wird $E_\ell$ von $u_\ell$ minimiert, 
  deshalb gilt $0\in\partial E_\ell(u_\ell)
  = \partial F(u_\ell)+\partial G_\ell(u_\ell)=\partial F(u_\ell)+
  \{\delta G_\ell(u_\ell)\}$ \todo[inline]{TODO quote as well} und es folgt
  $-\delta G_\ell(u_\ell)\in\partial F(u_\ell)$.
  Daraus folgt zusammen mit
  \eqref{eq:stabilityAndUniqueness:monotonicityOfSubdifferential}
  \begin{align*}
    \big( -(\alpha u_1 - f_1) + (\alpha u_2 - f_2), u_1 - u_2\big)_{L^2(\Omega)}
    \geq 0.
  \end{align*}
  Umformen und Anwenden der Cauchy-Schwarzschen Ungleichung impliziert
  \begin{align*}
    \alpha \Vert u_1 - u_2 \Vert_{L^2(\Omega)}^2
    &\leq
    \big(f_1 -f_2, u_1-u_2 \big)_{L^2(\Omega)}\\
    &\leq
    \Vert f_1-f_2\Vert_{L^2(\Omega)}\Vert u_1 - u_2\Vert_{L^2(\Omega)}.
  \end{align*}

  \todo[inline]{TODO wie verschwindet das $\alpha$ hier um dann die
  Abschätzung aus dem Satz zu bekommen (für $alpha\geq 1$ haben wir sie ja
  auch, aber $\alpha>0$ ist nur gefordert? Oder verpasst man der
  Abschätzung im Satz noch eine Konstante $1/\alpha$ vor der oberen Schranke?

  Zumindest Eindeutigkeit bekommt man auch hier immernoch (man
  setzt f1 und f2 im Satz gleich f), was gut ist.
  Reicht das also sogar?}

  Falls $\Vert u_1 - u_2 \Vert_{L^2(\Omega)} = 0$, gilt der Satz.
  Ansonsten führt Division durch \\
  $\Vert u_1 - u_2 \Vert_{L^2(\Omega)}\neq 0$ den 
  Beweis zum Abschluss.
\end{proof}
