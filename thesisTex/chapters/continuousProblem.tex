\section{Formulierung}
Für einen Parameter $\alpha\in\Rbb_+$ und eine Funktion
$f\in L^2(\Omega)$ betrachten wir das folgende Minimierungsproblem. 

\begin{problem}\label{prob:continuousProblem}
  Finde $u\in \BV(\Omega)\cap L^2(\Omega)$, sodass
  $u$ das Funktional
  \begin{align}\label{eq:continuousProblem}
    E(v)\coloneqq \frac{\alpha}{2}\Vert v\Vert_{L^2(\Omega)}^2 + |v|_{\BV(\Omega)}
    +\Vert v\Vert_{L^1(\partial\Omega)}-\int_\Omega fv\dx
  \end{align}
  unter allen $v\in\BV(\Omega)\cap L^2(\Omega)$ minimiert.

  Dabei ist der Term $\Vert v\Vert_{L^1(\partial\Omega)}$ wohldefiniert, da
  nach \cite[S. 400, Theorem 10.2.1]{ABM14} ein linearer Operator
  $T:\BV(\Omega)\to L^1(\partial\Omega)$ existiert mit $T(u) =
  u|_{\partial\Omega}$ für alle $u\in\BV(\Omega)\cap C(\overline\Omega)$.
\end{problem}

\begin{remark}
  Nach \cite[S. 399, Theorem 10.1.3]{ABM14} ist 
  \todo[inline]{noch fragen, was 1-regular nochmal heißt und ob das hier 
  glatt geht (tut es sehr wahrscheinlich)}
  die Einbettung $\BV(\Omega)\hookrightarrow L^p(\Omega)$ stetig für 
  $1\leq p\leq d/(d-1)$. 
  Damit ist $\BV(\Omega)$ für $d=2$ Teilmenge von $L^2(\Omega)$ und die
  Lösung von \Cref{prob:continuousProblem} kann in
  $\BV(\Omega)$ gesucht werden. Für beliebige $d\in\Nbb$, die wir in diesen
  Abschnitt betrachten, gilt dies im Allgemeinen nicht.
\end{remark}

\section{Existenz und Eindeutigkeit von Minimierern}
Zunächst zeigen wir, dass \Cref{prob:continuousProblem} eine Lösung besitzt.
Dafür benötigen wir die folgende Formulierung der Youngschen Ungleichung.

\begin{lemma}[Youngsche Ungleichung]
  \label{lem:young}
  Seien $a,b\in\Rbb$ und $\varepsilon\in\Rbb_+$ beliebig. Dann gilt
  \begin{align*}
    ab\leq\frac{1}{\varepsilon}a^2+\frac{\varepsilon}{4}b^2. 
  \end{align*}
\end{lemma}

Außerdem wird im Beweis folgende Aussage benötigt, die direkt aus \cite[S. 183,
Theorem 1]{EG92} folgt, da $0\in\BV(\Rbb^d\setminus\overline\Omega)$,
$|0|_{\BV(\Rbb^d\setminus\overline\Omega)}=0$ und $0|_{\partial\Omega}=0$.

\begin{lemma}
  \label{lem:bvExtension}
  Sei $v\in\BV(\Omega)$.
  Definiere, für alle $x\in\Rbb^d$,
  \begin{align*}
    \tilde{v}(x)\coloneqq
    \begin{cases}
      v(x),  &\text{ falls } x\in\Omega,\\
      0,     &\text{ falls } x\in\Rbb^d\setminus\overline\Omega.
    \end{cases} 
  \end{align*}
  Dann gilt $\tilde{v}\in\BV\left(\Rbb^d\right)$ und
  $|\tilde{v}|_{\BV\left(\Rbb^d\right)}
  = |v|_{\BV(\Omega)}+\Vert v\Vert_{L^1(\partial\Omega)}$.
\end{lemma}

\begin{theorem}[Existenz einer Lösung]
  \label{thm:contProblemExistence}
  \Cref{prob:continuousProblem} besitzt eine Lösung \\$u\in\BV(\Omega)\cap
  L^2(\Omega)$.
\end{theorem}

\begin{proof}
  Die Beweisidee ist die Anwendung der direkten Methode der Variationsrechnung
  \todo{(cf. cite Dracoragna) bzw., falls das da nicht so klar ist, bestimmt
  im Zeidler zu finden} 
  unter Nutzung der in \Cref{sec:bvFunctions} aufgeführten Eigenschaften
  der schwachen Konvergenz in $\BV(\Omega)$.

  Für alle $v\in L^2(\Omega)\subset L^1(\Omega)$ gilt mit der Hölderschen
  Ungleichung für $p=q=2$, dass
  \begin{equation}\label{eq:hoelderL2BiggerL1}
    \Vert v\Vert_{L^1} 
    = \Vert 1\cdot v\Vert_{L^1(\Omega)}
    \leq \Vert 1\Vert_{L^2(\Omega)}\Vert v\Vert_{L^2(\Omega)}
    =\sqrt{|\Omega|} \Vert v\Vert_{L^2(\Omega)}.
  \end{equation}

  Dann folgt für das Funktional $E$ in \eqref{eq:continuousProblem} für alle
  $v\in \BV(\Omega)\cap L^2(\Omega)$ durch die Cauchy-Schwarzsche Ungleichung,
  die Youngsche Ungleichung aus \cref{lem:young} und Ungleichung
  \eqref{eq:hoelderL2BiggerL1}, dass
  \begin{equation}
    \label{eq:contProbBddFromBelow}
    \begin{aligned}
      E(v)&=\frac{\alpha}{2}\Vert v\Vert_{L^2(\Omega)}^2 + |v|_{\BV(\Omega)}
      +\Vert v\Vert_{L^1(\partial\Omega)}-\int_\Omega fv\dx\\
      &\geq 
      \frac{\alpha}{2}\Vert v\Vert_{L^2(\Omega)}^2 + |v|_{\BV(\Omega)}
      +\Vert v\Vert_{L^1(\partial\Omega)}
      -\Vert f\Vert_{L^2(\Omega)}\Vert v\Vert_{L^2(\Omega)}\\
      &\geq 
      \frac{\alpha}{2}\Vert v\Vert_{L^2(\Omega)}^2 + |v|_{\BV(\Omega)}
      +\Vert v\Vert_{L^1(\partial\Omega)}
      -\frac{1}{\alpha}\Vert f\Vert_{L^2(\Omega)}^2
      -\frac{\alpha}{4}\Vert v\Vert_{L^2(\Omega)}^2\\
      &\geq 
      \frac{\alpha}{4}\Vert v\Vert_{L^2(\Omega)}^2 + |v|_{\BV(\Omega)}
      +\Vert v\Vert_{L^1(\partial\Omega)}-\frac{1}{\alpha}\Vert
      f\Vert_{L^2(\Omega)}^2\\
      &\geq 
      \frac{\alpha}{4|\Omega|}\Vert v\Vert_{L^1(\Omega)}^2 + |v|_{\BV(\Omega)}
      +\Vert v\Vert_{L^1(\partial\Omega)}-\frac{1}{\alpha}\Vert
      f\Vert_{L^2(\Omega)}^2\\
      &\geq -\frac{1}{\alpha}\Vert f\Vert_{L^2(\Omega)}^2.
    \end{aligned}
  \end{equation}

  Somit ist $E$ nach unten beschränkt, was die Existenz einer infimierenden
  Folge $(u_n)_{n\in\Nbb}\subset\BV(\Omega)\cap L^2(\Omega)$ von $E$ 
  impliziert, das heißt
  $(u_n)_{n\in\Nbb}$ erfüllt $$\lim_{n\rightarrow\infty}E(u_n) =
  \inf_{v\in\BV(\Omega)\cap L^2(\Omega)}E(v).$$ 

  Ungleichung \eqref{eq:contProbBddFromBelow} impliziert außerdem, dass
  $E(u_n)\to\infty$ für $n\to\infty$, falls $|u_n|_{\BV(\Omega)}\to\infty$ oder
  $\Vert u_n\Vert_{L^1(\Omega)}\to\infty$ für $n\to\infty$. 
  Daraus folgt insbesondere, dass $E(u_n)\to\infty$ für $n\to\infty$, falls
  $\Vert u_n\Vert_{\BV(\Omega)}\to\infty$ für $n\to\infty$ .

  Deshalb muss die Folge $(u_n)_{n\in\Nbb}$ beschränkt in $\BV(\Omega)$ sein.

  \medbreak
  Nun garantiert \cref{thm:compactness} die Existenz einer schwach konvergenten
  Teilfolge $(u_{n_k})_{k\in\Nbb}$ von $(u_n)_{n\in\Nbb}$ mit schwachem Grenzwert
  $u\in\BV(\Omega)$. Ohne Beschränkung der Allgemeinheit
  ist $(u_{n_k})_{k\in\Nbb}=(u_n)_{n\in\Nbb}$.
  \todo{hier vielleicht THm 2.15 anwendbar zum Abkürzen}
  Schwache Konvergenz von $(u_n)_{n\in\Nbb}$ in $\BV(\Omega)$ gegen
  $u$ bedeutet nach Definition insbesondere, dass $(u_n)_{n\in\Nbb}$ stark und
  damit auch schwach in $L^1(\Omega)$ gegen $u$ konvergiert.

  \medbreak
  Weiterhin folgt aus (\ref{eq:contProbBddFromBelow}), dass
  $E(v)\rightarrow\infty$ für
  $\Vert v\Vert_{L^2(\Omega)}\rightarrow\infty$. Somit muss $(u_n)_{n\in\Nbb}$ 
  auch beschränkt sein bezüglich der Norm $\Vert\bullet\Vert_{L^2(\Omega)}$ und
  besitzt deshalb, wegen der Reflexivität von $L^2(\Omega)$,
  eine Teilfolge (ohne Beschränkung der Allgemeinheit weiterhin bezeichnet mit
  $(u_n)_{n\in\Nbb}$), die
  in $L^2(\Omega)$ schwach gegen einen Grenzwert 
  $\tilde{u}\in L^2(\Omega)$ konvergiert. Somit gilt für alle $w\in
  L^2(\Omega)\cong L^2(\Omega)^\ast$ und, da $L^\infty(\Omega)\subset
  L^2(\Omega)$, insbesondere 
  auch für alle $w\in
  L^\infty(\Omega)\cong L^1(\Omega)^\ast$, dass 
  $\int_\Omega u_n w\dx \overset{n\to\infty}{\to}\int_\Omega \tilde{u} w\dx$. 
  Damit konvergiert $(u_n)_{n\in\Nbb}$ also auch schwach in $L^1(\Omega)$ gegen
  $\tilde{u}\in L^2(\Omega)\subset L^1(\Omega)$. 

  \medbreak
  Da schwache Grenzwerte eindeutig bestimmt sind, gilt insgesamt $u=\tilde u
  \in L^2(\Omega)$, das heißt $u\in\BV(\Omega)\cap
  L^2(\Omega)$.

  %\medbreak
  %Wir wissen, dass $(u_n)_{n\in\Nbb}$ beschränkt in $\BV(\Omega)$
  %ist und in $L^1(\Omega)$ gegen $u\in \BV(\Omega)\cap L^2(\Omega)\subset
  %L^1(\Omega)$ konvergiert. \cref{thm:wlsc} impliziert unter diesen 
  %Voraussetzungen, dass $|u|_{\BV(\Omega)}\leq \liminf_{n\to\infty} 
  %|u_n|_{\BV(\Omega)}$.

  \medbreak
  Nun definieren wir, für alle
  $n\in\Nbb$ und für alle 
  $x\in\Rbb^d$,
  \begin{align*}
    \tilde{u}_n(x)\coloneqq
    \begin{cases}
      u_n(x),  &\text{ falls } x\in\Omega,\\
      0,     &\text{ falls } x\in\Rbb^d\setminus\Omega
    \end{cases} 
  \end{align*}
  und
  \begin{align*}
    \tilde{u}(x)\coloneqq
    \begin{cases}
      u(x),  &\text{ falls } x\in\Omega,\\
      0,     &\text{ falls } x\in\Rbb^d\setminus\Omega.
    \end{cases} 
  \end{align*}

  Dann gilt nach \cref{lem:bvExtension} sowohl $\tilde{u}\in\BV(\Rbb^d)$ und
  $|\tilde{u}|_{\BV\left(\Rbb^d\right)} = |u|_{\BV(\Omega)}+\Vert
  u\Vert_{L^1(\partial\Omega)}$ als auch $\tilde{u}_n\in\BV(\Rbb^d)$
   und
   $|\tilde{u}_n|_{\BV\left(\Rbb^d\right)}
  = |u_n|_{\BV(\Omega)}+\Vert u_n\Vert_{L^1(\partial\Omega)}$ für alle
  $n\in\Nbb$.
  Da $(u_n)_{n\in \Nbb}$ infimierende Folge von $E$ ist, muss aufgrund der
  Form von $E$ die Folge
  $(|\tilde{u}_n|_{\BV(\Rbb^d)})_{n\in\Nbb} = (|u_n|_{\BV(\Omega)}+\Vert
  u_n\Vert_{L^1(\partial\Omega)})_{n\in\Nbb}$
  beschränkt sein.
  Außerdem gilt $\tilde{u}_n \overset{n\to\infty}{\to} \tilde{u}$ in
  $L^1(\Rbb^d)$, da aus der Definition von $\tilde{u}$ und 
  $\tilde{u}_n$ für alle $n\in\Nbb$ und der bereits bekannten Eigenschaft 
  $u_n\overset{n\to\infty}{\to} u$ folgt
  \begin{align*}
    \Vert \tilde{u}_n - \tilde{u}\Vert_{L^1(\Rbb^d)} 
    &= \int_{\Rbb^d} |\tilde{u}_n - \tilde{u}|\dx\\
    &= \int_\Omega |u_n - u|\dx\\
    &= \Vert u_n - u\Vert_{L^1(\Omega)} \overset{n\to\infty}{\to} 0.
  \end{align*}

  Insgesamt ist also $(\tilde{u}_n)_{n\in\Nbb}$ eine Folge in $\BV(\Rbb^d)$,
  die in $L^1(\Rbb^d)$ gegen $\tilde{u}\in\BV(\Rbb^d)\subset
  L^1(\Rbb^d)$ konvergiert und 
  erfüllt, dass $(|\tilde{u}_n|_{\BV(\Rbb^d)})_{n\in\Nbb}$ beschränkt
  ist. Somit folgt mit
  \cref{thm:wlsc}  
  \begin{equation}
    \label{eq:wlscOfExtension}
    \begin{aligned}
      |u|_{\BV(\Omega)} +\Vert u\Vert_{L^1(\partial\Omega)}
      = |\tilde{u}|_{\BV(\Rbb^d)}
      &\leq\liminf_{n\to\infty} |\tilde{u}_n|_{\BV(\Rbb^d)}\\
      &= \liminf_{n\to\infty} (|u_n|_{\BV(\Omega)} +
      \Vert u_n\Vert_{L^1(\partial\Omega)}).
    \end{aligned}
  \end{equation}

  \medbreak
  Die Funktionen $\Vert\bullet\Vert_{L^2(\Omega)}^2$ und $-\int_\Omega
  f\bullet\dx$ sind auf $L^2(\Omega)$ stetig und konvex, was impliziert,
  dass sie schwach unterhalbstetig auf $L^2(\Omega)$ sind. Da wir bereits
  wissen, dass $u_n \overset{n\to\infty}{\rightharpoonup} u$ in $L^2(\Omega)$, 
  folgt daraus 
  \begin{align*}
    \frac{\alpha}{2}\Vert u\Vert_{L^2(\Omega)}-\int_\Omega fu\dx
    \leq \liminf_{n\to\infty}
    \left(\frac{\alpha}{2}\Vert u_n\Vert_{L^2(\Omega)}
    -\int_\Omega fu_n\dx\right).
  \end{align*}
  
  \medbreak
  Damit und mit \cref{eq:wlscOfExtension} gilt insgesamt
  \begin{align*}
    \inf_{v\in\BV(\Omega)\cap L^2(\Omega)}E(v)\leq
    E(u)\leq\liminf_{n\rightarrow\infty} E(u_n) =
    \lim_{n\rightarrow\infty}E(u_n) = \inf_{v\in\BV(\Omega)\cap
    L^2(\Omega)}E(v),
  \end{align*}
  d.h.\ $\min_{v\in\BV(\Omega)\cap L^2(\Omega)} E(v) = E(u)$.
\end{proof}

\begin{theorem}[Stabilität und Eindeutigkeit]
  \label{thm:contProbStabAndUniqu}
  Seien $u_1,u_2\in \BV(\Omega)\cap L^2(\Omega)$ die Minimierer des Problems
  \ref{prob:continuousProblem} mit $f_1,f_2\in L^2(\Omega)$ anstelle von $f$.

  Dann gilt 
  \begin{align*}
    \Vert u_1 - u_2\Vert_{L^2(\Omega)} 
    \leq\frac{1}{\alpha}\Vert f_1-f_2\Vert_{L^2(\Omega)}.
  \end{align*}
\end{theorem}

\begin{proof}
  Definiere die konvexen Funktionale $F:\BV(\Omega)\cap L^2(\Omega)\to \Rbb$
  und 
  $G_\ell:\BV(\Omega)\cap L^2(\Omega)\to \Rbb$, $\ell=1,2$, durch
  \begin{align*}
    F(u) &\coloneqq |u|_{\BV(\Omega)} + \Vert u \Vert_{L^1(\partial\Omega)},&
    G_\ell(u)\coloneqq \frac{\alpha}{2}\Vert u\Vert_{L^2(\Omega)}^2 -
    \int_\Omega f_\ell u\dx.
  \end{align*}
  Bezeichne $E_\ell\coloneqq F+G_\ell$.

  $G_\ell$ ist Fr\'echet-differenzierbar 

  \todo[inline]{TODO nachrechnen,
  Gateaux ist klar, aber auch Frechet? EDIT: nachgerechnet, es funktioniert
  nach WIKI Def.
  
  Außerdem: Im Grundlagen Kapitel noch einführen, was hier in dieser Arbeit
  mit Gateaux, Frechet etc gemeint ist? (ist ja von Autor zu Autor anders (cf
  Wiki)) und insbesondere irgendwo einmal alle Notationen einführen, was ist 
  welche Ableitung} 

  und die Fr\'echet-Ableitung $G_\ell'(u):
  L^2(\Omega)\to\Rbb$ von $G_\ell$ an der
  Stelle
  $u\in \BV(\Omega)\cap L^2(\Omega)$ ist 
  für alle $v\in L^2(\Omega)$ gegeben durch
  \begin{align*}
    dG_\ell(u;v) = \alpha (u,v)_{L^2(\Omega)} - \int_\Omega f_\ell v\dx 
    = (\alpha u-f_\ell ,v)_{L^2(\Omega)}.
  \end{align*}

  Das Funktional $F$ ist konvex 
  und stetig, also insbesondere unterhalbstetig, deshalb 
  ist nach \cref{thm:subdifferentialMonotonicity} das
  Subdifferential
  $\partial F$ von $F$ monoton, das heißt für alle $\mu_\ell\in \partial
  F(u_\ell)$, $\ell=1,2$, gilt
  \begin{align}\label{eq:stabilityAndUniqueness:monotonicityOfSubdifferential}
    (\mu_1-\mu_2,u_1-u_2)_{L^2(\Omega)}\geq 0.
  \end{align}
  \todo[inline]{TODO eigentlich auch mal über Dualraumtheorie reden,
  insbesondere für Lp Räume und wie die Sachen identifiziert werden können 
  nach Riesz?} 

  Für $\ell=1,2$ wird $E_\ell$ von $u_\ell$ minimiert und $G_\ell$ ist stetig.
  Nach \cref{thm:extremalprinciple} und
  \cref{thm:subdifferentialSumRule} gilt deshalb $0\in\partial E_\ell(u_\ell)
  = \partial F(u_\ell)+\partial G_\ell(u_\ell)=\partial F(u_\ell)+
  \{G_\ell'(u_\ell)\}$ 
  und es folgt
  $-G_\ell'(u_\ell)\in\partial F(u_\ell)$.
  Daraus folgt zusammen mit
  \eqref{eq:stabilityAndUniqueness:monotonicityOfSubdifferential}
  \begin{align*}
    \big( -(\alpha u_1 - f_1) + (\alpha u_2 - f_2), u_1 - u_2\big)_{L^2(\Omega)}
    \geq 0.
  \end{align*}
  Umformen und Anwenden der Cauchy-Schwarzschen Ungleichung impliziert
  \begin{align*}
    \alpha \Vert u_1 - u_2 \Vert_{L^2(\Omega)}^2
    &\leq
    \big(f_1 -f_2, u_1-u_2 \big)_{L^2(\Omega)}\\
    &\leq
    \Vert f_1-f_2\Vert_{L^2(\Omega)}\Vert u_1 - u_2\Vert_{L^2(\Omega)}.
  \end{align*}

  Falls $\Vert u_1 - u_2 \Vert_{L^2(\Omega)} = 0$, gilt der Satz.
  Ansonsten führt Division durch\\
  $\alpha\Vert u_1 - u_2 \Vert_{L^2(\Omega)}\neq 0$ den 
  Beweis zum Abschluss.
\end{proof}

\begin{theorem}
  \label{thm:convexity}
  Sei $u\in\BV(\Omega)\cap L^2(\Omega)$ Lösung von 
  \Cref{prob:continuousProblem}.

  Dann gilt 
  \begin{align*}
    \frac{\alpha}{2}\Vert u-v\Vert_{L^2(\Omega)}^2 \leq E(v)-E(u)\quad
    \text{für alle } v\in\BV(\Omega)\cap L^2(\Omega).
  \end{align*}
\end{theorem}

\begin{proof}
  Definiere die konvexen Funktionale $F:\BV(\Omega)\cap L^2(\Omega)\to \Rbb$
  und 
  $G:\BV(\Omega)\cap L^2(\Omega)\to \Rbb$ durch
  \begin{align*}
    F(u) &\coloneqq |u|_{\BV(\Omega)} + \Vert u \Vert_{L^1(\partial\Omega)},&
    G(u)\coloneqq \frac{\alpha}{2}\Vert u\Vert_{L^2(\Omega)}^2 -
    \int_\Omega f u\dx.
  \end{align*}
  Es gilt $E= F+G$.

  $G$ ist Fr\'echet-differenzierbar und die Fr\'echet-Ableitung $G'(u):
  L^2(\Omega)\to\Rbb$ von $G$ an der
  Stelle
  $u\in \BV(\Omega)\cap L^2(\Omega)$ ist 
  für alle $v\in L^2(\Omega)$ gegeben durch
  \begin{align*}
    dG(u;v) = \alpha (u,v)_{L^2(\Omega)} - \int_\Omega f v\dx 
    = (\alpha u-f ,v)_{L^2(\Omega)}.
  \end{align*}

  Das impliziert mit wenigen Rechenschritten
  \begin{align}\label{eq:strongConvexityG}
    dG(u;v-u) +\frac{\alpha}{2}\Vert u-v\Vert^2_{L^2(\Omega)}+G(u) 
    =
    G(v)
  \end{align}
  für alle $u,v\in \BV(\Omega)\cap L^2(\Omega)$.

  Da $u$ Minimierer von $E$ ist, gilt mit \Cref{thm:extremalprinciple},
  \Cref{thm:subdifferentialSumRule} und \Cref{thm:subdiffGateaux}, dass
  \begin{align*}
    0\in\partial E(u) = \partial F(u)+\{G'(u)\},
  \end{align*}
  woraus folgt 
  \begin{align*}
    -G'(u)\in\partial F(u),
  \end{align*}
  was nach \Cref{def:subdifferential} äquivalent ist zu
  \begin{align*}
    -dG(u;v-u)\leq F(v)-F(u)\quad\text{für alle }v\in\BV(\Omega)\cap
    L^2(\Omega).
  \end{align*}

  Daraus folgt zusammen mit \Cref{eq:strongConvexityG}, dass
  \begin{align*}
    \frac{\alpha}{2}\Vert u-v\Vert_{L^2(\Omega)}^2+G(u)-G(v)+F(u)
    = -dG(u;v-u)+F(u)\leq F(v)
  \end{align*}
  für alle $v\in \BV(\Omega)\cap L^2(\Omega)$.

  Da $E=F+G$, folgt daraus die Aussage.
\end{proof}
