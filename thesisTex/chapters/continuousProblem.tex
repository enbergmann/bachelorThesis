Wir betrachten für ein gegebenes $\alpha\in\Rbb^+$ und eine Funktion
$f\in L^2(\Omega)$ das 
folgende Minimierungsproblem. 

\todo[inline]{TODO/Q nach \protect{\cite[Thm. 10.1.4 (first sentence in
proof)]{ABM14}}
ist in 2D BV enthalten in $L^2$, d.h. $u\in \BV(\Omega)\cap L^2(\Omega)$ muss
nur als $u\in \BV(\Omega)$ formuliert werden}

\todo[inline]{der eine Beweis der für 2D leichter wäre damit steht aber schon
und alles 
andere sollte auch in nD gelten, also vielleciht für dieses Kapitel in nD 
bleiben (kommt auch drauf an, wie das vorherige Kapitel aussieht, ob es
BV in nD formuliert oder ob ich zur Vereinfachung direkt 2D mache)
Momentan ist das hier gemischt zwischen 2 und n: Überarbeiten!!}


\begin{problem}\label{prob:continuousProblem}
  Finde $u\in \BV(\Omega)\cap L^2(\Omega)$, sodass
  $u$ das Funktional
  \begin{align}\label{eq:continuousProblem}
    E(v)\coloneqq \frac{\alpha}{2}\Vert v\Vert_{L^2(\Omega)}^2 + |v|_{\BV(\Omega)}
    +\Vert v\Vert_{L^1(\partial\Omega)}-\int_\Omega fv\dx
  \end{align}
  unter allen $v\in\BV(\Omega)\cap L^2(\Omega)$ minimiert.

  Nach \cref{thm:traceOperator} ist der Term $\Vert
  v\Vert_{L^1(\partial\Omega)}$ wohldefiniert.
\end{problem}

\begin{remark}
  \todo[inline]{TODO Vielleicht auch erst beim Diskreten Problem,
  da es dort Nullranddaten gibt? Beachte insbesondere, dass $f=\alpha g$ 

  falls tatsächlich das Problem nur in BV formuliert werden muss, da BV in 2D
  in L2 enthalten ist, dann sollte das hier auch erwähnt werden: 'Bartels
  formuliert das so und so in BV cap L2, aber da wir uns auf 2D beschränken,
  gilt \ldots'}
  In \cite[Kapitel~10.1.3]{Bar15} wird \Cref{prob:continuousProblem} für ein
  gegebenes $g\in L^2(\Omega)$ formuliert
  mit dem Funktional 
  \begin{align*}
    I(v)\coloneqq |v|_{\BV(\Omega)} + \frac{\alpha}{2}\int_\Omega (v-g)^2\dx
  \end{align*}
  für $v\in \BV(\Omega)\cap L^2(\Omega)$.

  Nun wählen wir $f = \alpha g$. Dann gilt
  $I(v) = E(v) - \Vert v\Vert_{L^1(\partial \Omega)}+ 
  \frac{\alpha}{2}\Vert g\Vert_{L^2(\Omega)}^2$ für alle 
  $v\in \BV(\Omega)\cap L^2(\Omega)$. Da der Term $\frac{\alpha}{2}\Vert
  g\Vert_{L^2(\Omega)}^2$ konstant ist, haben die Funktionale $E$ und $I$ somit
  die gleichen Minimierer in $\left\{v\in\BV(\Omega)\cap L^2(\Omega)\mid 
  \Vert v\Vert_{L^1(\partial\Omega)}=0\right\}$.
\end{remark}

Zunächst zeigen wir, dass \Cref{prob:continuousProblem} eine Lösung besitzt.
Dafür benötigen wir die folgenden Ungleichungen.

\begin{lemma}[Cauchy-Schwarzsche Ungleichung]
  \label{lem:csu}
  Sei $V$ ein reeller oder komplexer Vektorraum mit Skalarprodukt
  $(\cdot,\cdot)_V$. Dann gilt für alle $x,y\in V$
  \begin{align*}
    |(x,y)_V|^2\leq (x,x)_V (y,y)_V.
  \end{align*}
  Gleichheit gilt genau dann, wenn $x$ und $y$ linear unabhängig sind.
\end{lemma}

\begin{lemma}[Youngsche Ungleichung]
  \label{lem:young}
  Seien $a,b\in\Rbb$ und $\varepsilon>0$ beliebig. Dann gilt
  \begin{align*}
    ab\leq\frac{1}{\varepsilon}a^2+\frac{\varepsilon}{4}b^2. 
  \end{align*}
\end{lemma}

\begin{lemma}[Höldersche Ungleichung]
  \label{lem:hoelder}
  Seien $p,q\in [1,\infty]$ mit $1/p+1/q=1$, $f\in L^p(\Omega)$ und 
  $g\in L^q(\Omega)$. Dann gilt $fg\in L^1(\Omega)$ mit 
  \begin{align*}
    \Vert fg\Vert_{L^1(\Omega)}\leq 
    \Vert f\Vert_{L^p(\Omega)}\Vert g\Vert_{L^q(\Omega)}.
  \end{align*}
\end{lemma}

\todo[inline]{Q alle drei zitieren mit irgendeiner Quelle mit der passenden
Formulierung oder ist das zu basic}

\todo[inline]{In Grundlagen einmal darüber reden, wie für fast alle hier zu 
verstehen ist? Es gibt verschiedene Konventionen und hier ist natürlich 
gemeint für alle x bis auf die aus Nullmengen}

Außerdem wird im Beweis folgende Aussage benötigt, die direkt aus \cite[S. 183,
Theorem 1]{EG92} folgt, da $0\in\BV(\Rbb^n\setminus\Omega)$,
$|0|_{\BV(\Rbb^n\setminus\Omega)}=0$ und $0|_{\partial\Omega}=0$.
\todo[inline]{Q vielleicht den Trace Operator T immer mitnehmen?}

\begin{lemma}
  \label{lem:bvExtension}
  Sei $v\in\BV(\Omega)$.
  Definiere, für alle $x\in\Rbb^n$,
  \begin{align*}
    \tilde{v}(x)\coloneqq
    \begin{cases}
      v(x),  &\text{ falls } x\in\Omega,\\
      0,     &\text{ falls } x\in\Rbb^n\setminus\Omega.
    \end{cases} 
  \end{align*}
  Dann gilt $\tilde{v}\in\BV\left(\Rbb^n\right)$ und
  $|\tilde{v}|_{\BV\left(\Rbb^n\right)}
  = |v|_{\BV(\Omega)}+\Vert v\Vert_{L^1(\partial\Omega)}$.
\end{lemma}

\begin{theorem}[Existenz einer Lösung]
  \label{thm:contProblemExistence}
  \Cref{prob:continuousProblem} besitzt eine Lösung \\$u\in\BV(\Omega)\cap
  L^2(\Omega)$.
\end{theorem}

\begin{proof}
  Für alle $v\in L^2(\Omega)\subset L^1(\Omega)$ gilt mit der Hölderschen
  Ungleichung
  \todo[inline]{Q wo und wie einmal erwähnen, dass die Lp Räume geschachtelt 
  sind da Omega bdd ist? In Grundlagenkapitel 'Da bdd gilt hier immer die
  Inklusion ..' vielleicht einmal allgemein, dann noch mit Zitierung.}
  (\cref{lem:hoelder}) für $p=q=2$, dass
  \begin{equation}\label{equ:hoelderL2BiggerL1}
    \Vert v\Vert_{L^1} 
    = \Vert 1\cdot v\Vert_{L^1(\Omega)}
    \leq \Vert 1\Vert_{L^2(\Omega)}\Vert v\Vert_{L^2(\Omega)}
    =\sqrt{|\Omega|} \Vert v\Vert_{L^2(\Omega)}.
  \end{equation}

  Dann folgt für das Funktional $E$ in \eqref{eq:continuousProblem}
  für alle $v\in \BV(\Omega)\cap L^2(\Omega)$ durch die
  Cauchy-Schwarzsche Ungleichung (\cref{lem:csu}), die Youngsche
  Ungleichung \eqref{lem:young} und \cref{equ:hoelderL2BiggerL1}, dass
  \begin{equation}
    \label{eq:contProbBddFromBelow}
    \begin{aligned}
      E(v)&=\frac{\alpha}{2}\Vert v\Vert_{L^2(\Omega)}^2 + |v|_{\BV(\Omega)}
      +\Vert v\Vert_{L^1(\partial\Omega)}-\int_\Omega fv\dx\\
      &\geq 
      \frac{\alpha}{2}\Vert v\Vert_{L^2(\Omega)}^2 + |v|_{\BV(\Omega)}
      +\Vert v\Vert_{L^1(\partial\Omega)}
      -\Vert f\Vert_{L^2(\Omega)}\Vert v\Vert_{L^2(\Omega)}\\
      &\geq 
      \frac{\alpha}{2}\Vert v\Vert_{L^2(\Omega)}^2 + |v|_{\BV(\Omega)}
      +\Vert v\Vert_{L^1(\partial\Omega)}
      -\frac{1}{\alpha}\Vert f\Vert_{L^2(\Omega)}^2
      -\frac{\alpha}{4}\Vert v\Vert_{L^2(\Omega)}^2\\
      &\geq 
      \frac{\alpha}{4}\Vert v\Vert_{L^2(\Omega)}^2 + |v|_{\BV(\Omega)}
      +\Vert v\Vert_{L^1(\partial\Omega)}-\frac{1}{\alpha}\Vert
      f\Vert_{L^2(\Omega)}^2\\
      &\geq 
      \frac{\alpha}{4|\Omega|}\Vert v\Vert_{L^1(\Omega)}^2 + |v|_{\BV(\Omega)}
      +\Vert v\Vert_{L^1(\partial\Omega)}-\frac{1}{\alpha}\Vert
      f\Vert_{L^2(\Omega)}^2\\
      &\geq -\frac{1}{\alpha}\Vert f\Vert_{L^2(\Omega)}^2.
    \end{aligned}
  \end{equation}

  Somit ist $E$ nach unten beschränkt, was die Existenz einer infimierenden
  Folge $(u_n)_{n\in\Nbb}\subset\BV(\Omega)\cap L^2(\Omega)$ von $E$ 
  impliziert, d.h.\
  $(u_n)_{n\in\Nbb}$ erfüllt $\lim_{n\rightarrow\infty}E(u_n) =
  \inf_{v\in\BV(\Omega)\cap L^2(\Omega)}E(v)$. 

  Ungleichung \eqref{eq:contProbBddFromBelow} impliziert außerdem, dass
  $E(u_n)\overset{n\to\infty}{\to}\infty$, falls
  $|u_n|_{\BV(\Omega)}\overset{n\to\infty}{\to}\infty$ oder 
  $\Vert u_n\Vert_{L^1(\Omega)}\overset{n\to\infty}{\to}\infty$, also 
  insgesamt, falls
  $\Vert u_n\Vert_{\BV(\Omega)}\overset{n\to\infty}{\to}\infty$.

  Deshalb muss die Folge $(u_n)_{n\in\Nbb}$ beschränkt in $\BV(\Omega)$ sein.

  \medbreak
  Nun garantiert \cref{thm:compactness} die Existenz einer schwach konvergenten
  Teilfolge $(u_{n_k})_{k\in\Nbb}$ von $(u_n)_{n\in\Nbb}$ mit schwachem Grenzwert
  $u\in\BV(\Omega)$. Ohne Beschränkung der Allgemeinheit
  ist $(u_{n_k})_{k\in\Nbb}=(u_n)_{n\in\Nbb}$.
  Schwache Konvergenz von $(u_n)_{n\in\Nbb}$ in $\BV(\Omega)$ gegen
  $u$ bedeutet nach Definition insbesondere, dass $(u_n)_{n\in\Nbb}$ stark und
  damit auch schwach in $L^1(\Omega)$ gegen $u$ konvergiert.

  \medbreak
  Weiterhin folgt aus (\ref{eq:contProbBddFromBelow}), dass
  $E(v)\rightarrow\infty$ für
  $\Vert v\Vert_{L^2(\Omega)}\rightarrow\infty$. Somit muss $(u_n)_{n\in\Nbb}$ 
  auch beschränkt sein bezüglich der Norm $\Vert\bullet\Vert_{L^2(\Omega)}$ und
  besitzt deshalb, wegen der Reflexivität von $L^2(\Omega)$,
  eine Teilfolge (ohne Beschränkung der Allgemeinheit weiterhin bezeichnet mit
  $(u_n)_{n\in\Nbb}$), die
  in $L^2(\Omega)$ schwach gegen einen Grenzwert 
  $\tilde{u}\in L^2(\Omega)$ konvergiert. Somit gilt für alle $w\in
  L^2(\Omega)\cong L^2(\Omega)^\ast$ und, da $L^\infty(\Omega)\subset
  L^2(\Omega)$, insbesondere 
  auch für alle $w\in
  L^\infty(\Omega)\cong L^1(\Omega)^\ast$, dass 
  $\int_\Omega u_n w\dx \overset{n\to\infty}{\to}\int_\Omega \tilde{u} w\dx$. 
  Damit konvergiert $(u_n)_{n\in\Nbb}$ also auch schwach in $L^1(\Omega)$ gegen
  $\tilde{u}\in L^2(\Omega)\subset L^1(\Omega)$. 

  \medbreak
  Da schwache Grenzwerte eindeutig bestimmt sind, gilt insgesamt $u=\tilde u
  \in L^2(\Omega)$, das heißt $u\in\BV(\Omega)\cap
  L^2(\Omega)$.

  %\medbreak
  %Wir wissen, dass $(u_n)_{n\in\Nbb}$ beschränkt in $\BV(\Omega)$
  %ist und in $L^1(\Omega)$ gegen $u\in \BV(\Omega)\cap L^2(\Omega)\subset
  %L^1(\Omega)$ konvergiert. \cref{thm:wlsc} impliziert unter diesen 
  %Voraussetzungen, dass $|u|_{\BV(\Omega)}\leq \liminf_{n\to\infty} 
  %|u_n|_{\BV(\Omega)}$.

  \medbreak
  Nun definieren wir, für alle
  $n\in\Nbb$ und für alle 
  $x\in\Rbb^n$,
  \begin{align*}
    \tilde{u}_n(x)\coloneqq
    \begin{cases}
      u_n(x),  &\text{ falls } x\in\Omega,\\
      0,     &\text{ falls } x\in\Rbb^n\setminus\Omega
    \end{cases} 
  \end{align*}
  und
  \begin{align*}
    \tilde{u}(x)\coloneqq
    \begin{cases}
      u(x),  &\text{ falls } x\in\Omega,\\
      0,     &\text{ falls } x\in\Rbb^n\setminus\Omega.
    \end{cases} 
  \end{align*}

  Dann gilt nach \cref{lem:bvExtension} sowohl $\tilde{u}\in\BV(\Rbb^n)$ und
  $|\tilde{u}|_{\BV\left(\Rbb^n\right)} = |u|_{\BV(\Omega)}+\Vert
  u\Vert_{L^1(\partial\Omega)}$ als auch $\tilde{u}_n\in\BV(\Rbb^n)$
   und
   $|\tilde{u}_n|_{\BV\left(\Rbb^n\right)}
  = |u_n|_{\BV(\Omega)}+\Vert u_n\Vert_{L^1(\partial\Omega)}$ für alle
  $n\in\Nbb$.
  Da $(u_n)_{n\in \Nbb}$ infimierende Folge von $E$ ist, muss aufgrund der
  Form von $E$ die Folge
  $(|\tilde{u}_n|_{\BV(\Rbb^n)})_{n\in\Nbb} = (|u_n|_{\BV(\Omega)}+\Vert
  u_n\Vert_{L^1(\partial\Omega)})_{n\in\Nbb}$
  beschränkt sein.
  Außerdem gilt $\tilde{u}_n \overset{n\to\infty}{\to} \tilde{u}$ in
  $L^1(\Rbb^n)$, da aus der Definition von $\tilde{u}$ und 
  $\tilde{u}_n$ für alle $n\in\Nbb$ und der bereits bekannten Eigenschaft 
  $u_n\overset{n\to\infty}{\to} u$ folgt
  \begin{align*}
    \Vert \tilde{u}_n - \tilde{u}\Vert_{L^1(\Rbb^n)} 
    &= \int_{\Rbb^n} |\tilde{u}_n - \tilde{u}|\dx\\
    &= \int_\Omega |u_n - u|\dx\\
    &= \Vert u_n - u\Vert_{L^1(\Omega)} \overset{n\to\infty}{\to} 0.
  \end{align*}

  Insgesamt ist also $(\tilde{u}_n)_{n\in\Nbb}$ eine Folge in $\BV(\Rbb^n)$,
  die in $L^1(\Rbb^n)$ gegen $\tilde{u}\in\BV(\Rbb^n)\subset
  L^1(\Rbb^n)$ konvergiert und 
  erfüllt, dass $(|\tilde{u}_n|_{\BV(\Rbb^n)})_{n\in\Nbb}$ beschränkt
  ist. Somit folgt mit
  \cref{thm:wlsc}  
  \todo[inline]{Need to research this theorem, does it only hold on special
  $\Omega$ or can it be applied to function in $\BV(\Rbb^n)$?}
  \begin{equation}
    \label{eq:wlscOfExtension}
    \begin{aligned}
      |u|_{\BV(\Omega)} +\Vert u\Vert_{L^1(\partial\Omega)}
      = |\tilde{u}|_{\BV(\Rbb^n)}
      &\leq\liminf_{n\to\infty} |\tilde{u}_n|_{\BV(\Rbb^n)}\\
      &= \liminf_{n\to\infty} (|u_n|_{\BV(\Omega)} +
      \Vert u_n\Vert_{L^1(\partial\Omega)}).
    \end{aligned}
  \end{equation}

  \medbreak
  Die Funktionen $\Vert\bullet\Vert_{L^2(\Omega)}^2$ und $-\int_\Omega
  f\bullet\dx$ sind auf $L^2(\Omega)$ stetig und konvex, was impliziert,
  dass sie schwach unterhalbstetig auf $L^2(\Omega)$ sind. Da wir bereits
  wissen, dass $u_n \overset{n\to\infty}{\rightharpoonup} u$ in $L^2(\Omega)$, 
  folgt daraus 
  \begin{align*}
    \frac{\alpha}{2}\Vert u\Vert_{L^2(\Omega)}-\int_\Omega fu\dx
    \leq \liminf_{n\to\infty}
    \left(\frac{\alpha}{2}\Vert u_n\Vert_{L^2(\Omega)}
    -\int_\Omega fu_n\dx\right).
  \end{align*}
  
  \medbreak
  Damit und mit \cref{eq:wlscOfExtension} gilt insgesamt
  \begin{align*}
    \inf_{v\in\BV(\Omega)\cap L^2(\Omega)}E(v)\leq
    E(u)\leq\liminf_{n\rightarrow\infty} E(u_n) =
    \lim_{n\rightarrow\infty}E(u_n) = \inf_{v\in\BV(\Omega)\cap
    L^2(\Omega)}E(v),
  \end{align*}
  d.h.\ $\min_{v\in\BV(\Omega)\cap L^2(\Omega)} E(v) = E(u)$.
\end{proof}

\begin{theorem}[Stabilität und Eindeutigkeit]
  \label{thm:contProbStabAndUniqu}
  Seien $u_1,u_2\in \BV(\Omega)\cap L^2(\Omega)$ die Minimierer des Problems
  \ref{prob:continuousProblem} mit $f_1,f_2\in L^2(\Omega)$ anstelle von $f$.

  Dann gilt 
  \begin{align*}
    \Vert u_1 - u_2\Vert_{L^2(\Omega)} 
    \leq\frac{1}{\alpha}\Vert f_1-f_2\Vert_{L^2(\Omega)}.
  \end{align*}
\end{theorem}

\begin{proof}
  Definiere die konvexen Funktionale $F:\BV(\Omega)\cap L^2(\Omega)\to \Rbb$
  und 
  $G_\ell:\BV(\Omega)\cap L^2(\Omega)\to \Rbb$, $\ell=1,2$, durch
  \begin{align*}
    F(u) &\coloneqq |u|_{\BV(\Omega)} + \Vert u \Vert_{L^1(\partial\Omega)},&
    G_\ell(u)\coloneqq \frac{\alpha}{2}\Vert u\Vert_{L^2(\Omega)}^2 -
    \int_\Omega f_\ell u\dx.
  \end{align*}
  Bezeichne $E_\ell\coloneqq F+G_\ell$.

  $G_\ell$ ist Fr\'echet-differenzierbar 

  \todo[inline]{TODO nachrechnen,
  Gateaux ist klar, aber auch Frechet? EDIT: nachgerechnet, es funktioniert
  nach WIKI Def.
  
  Außerdem: Im Grundlagen Kapitel noch einführen, was hier in dieser Arbeit
  mit Gateaux, Frechet etc gemeint ist? (ist ja von Autor zu Autor anders (cf
  Wiki)) und insbesondere irgendwo einmal alle Notationen einführen, was ist 
  welche Ableitung} 

  und die Fr\'echet-Ableitung $G_\ell'(u):
  L^2(\Omega)\to\Rbb$ an der
  Stelle
  $u\in \BV(\Omega)\cap L^2(\Omega)$ ist 
  für alle $v\in L^2(\Omega)$ gegeben durch
  \begin{align*}
    dG_\ell(u;v) = \alpha (u,v)_{L^2(\Omega)} - \int_\Omega f_\ell v\dx 
    = (\alpha u-f_\ell ,v)_{L^2(\Omega)}.
  \end{align*}

  Das Funktional $F$ ist konvex 
  und stetig, also insbesondere unterhalbstetig, deshalb 
  ist nach \cref{thm:subdifferentialMonotonicity} das
  Subdifferential
  $\partial F$ von $F$ monoton, das heißt für alle $\mu_\ell\in \partial
  F(u_\ell)$, $\ell=1,2$, gilt
  \begin{align}\label{eq:stabilityAndUniqueness:monotonicityOfSubdifferential}
    (\mu_1-\mu_2,u_1-u_2)_{L^2(\Omega)}\geq 0.
  \end{align}
  \todo[inline]{TODO eigentlich auch mal über Dualraumtheorie reden,
  insbesondere für Lp Räume und wie die Sachen identifiziert werden können 
  nach Riesz?} 

  Für $\ell=1,2$ wird $E_\ell$ von $u_\ell$ minimiert und $G_\ell$ ist stetig.
  Nach \cref{thm:extremalprinciple} und
  \cref{thm:subdifferentialSumRule} gilt deshalb $0\in\partial E_\ell(u_\ell)
  = \partial F(u_\ell)+\partial G_\ell(u_\ell)=\partial F(u_\ell)+
  \{G_\ell'(u_\ell)\}$ 
  und es folgt
  $-G_\ell'(u_\ell)\in\partial F(u_\ell)$.
  Daraus folgt zusammen mit
  \eqref{eq:stabilityAndUniqueness:monotonicityOfSubdifferential}
  \begin{align*}
    \big( -(\alpha u_1 - f_1) + (\alpha u_2 - f_2), u_1 - u_2\big)_{L^2(\Omega)}
    \geq 0.
  \end{align*}
  Umformen und Anwenden der Cauchy-Schwarzschen Ungleichung impliziert
  \begin{align*}
    \alpha \Vert u_1 - u_2 \Vert_{L^2(\Omega)}^2
    &\leq
    \big(f_1 -f_2, u_1-u_2 \big)_{L^2(\Omega)}\\
    &\leq
    \Vert f_1-f_2\Vert_{L^2(\Omega)}\Vert u_1 - u_2\Vert_{L^2(\Omega)}.
  \end{align*}

  Falls $\Vert u_1 - u_2 \Vert_{L^2(\Omega)} = 0$, gilt der Satz.
  Ansonsten führt Division durch\\
  $\alpha\Vert u_1 - u_2 \Vert_{L^2(\Omega)}\neq 0$ den 
  Beweis zum Abschluss.
\end{proof}
