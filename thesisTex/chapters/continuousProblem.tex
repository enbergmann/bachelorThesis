Betrachte für gegebene $\alpha>0$ und $f\in L^2(\Omega)$ das 
folgende Minimierungsproblem. 

\begin{problem}\label{prob:continuousProblem}
  Finde $u\in \BV(\Omega)\cap L^2(\Omega)$, sodass
  $u$ das Funktional
  \begin{align}\label{eq:continuousProblem}
    E(v)\coloneqq \frac{\alpha}{2}\Vert v\Vert_{L^2(\Omega)}^2 + |v|_{\BV(\Omega)}
    +\Vert v\Vert_{L^1(\partial\Omega)}-\int_\Omega fv\dx
  \end{align}
  unter allen $v\in\BV(\Omega)\cap L^2(\Omega)$ minimiert.

  Nach \cref{thm:traceOperator} ist der Term $\Vert
  v\Vert_{L^1(\partial\Omega)}$ wohldefiniert.
\end{problem}

\begin{remark}
  \todo[inline]{TODO Vielleicht auch erst beim Diskreten Problem,
  da es dort Nullranddaten gibt? Beachte insbesondere, dass $f=\alpha g$ }
  In \cite[Kapitel~10.1.3]{Bar15} wird \Cref{prob:continuousProblem} für ein
  gegebenes $g\in L^2(\Omega)$ formuliert
  mit dem Funktional 
  \begin{align*}
    I(v)\coloneqq |v|_{\BV(\Omega)} + \frac{\alpha}{2}\int_\Omega (v-g)^2\dx
  \end{align*}
  für $v\in \BV(\Omega)\cap L^2(\Omega)$.

  Nun wählen wir $f = \alpha g$. Dann gilt
  $I(v) = E(v) - \Vert v\Vert_{L^1(\partial \Omega)}+ 
  \frac{\alpha}{2}\Vert g\Vert_{L^2(\Omega)}^2$ für alle 
  $v\in \BV(\Omega)\cap L^2(\Omega)$. Da der Term $\frac{\alpha}{2}\Vert
  g\Vert_{L^2(\Omega)}^2$ konstant ist, haben die Funktionale $E$ und $I$ somit
  die gleichen Minimierer in $\left\{v\in\BV(\Omega)\cap L^2(\Omega)\mid 
  \Vert v\Vert_{L^1(\partial\Omega)}=0\right\}$.
\end{remark}

Zunächst zeigen wir, dass \Cref{prob:continuousProblem} eine Lösung besitzt.
Dafür benötigen wir die folgenden Ungleichungen.

\begin{lemma}[Cauchy-Schwarzsche Ungleichung]
  \label{lem:csu}
  Sei $V$ ein reeller oder komplexer Vektorraum mit Skalarprodukt
  $(\cdot,\cdot)_V$. Dann gilt für alle $x,y\in V$
  \begin{align*}
    |(x,y)_V|^2\leq (x,x)_V (y,y)_V.
  \end{align*}
  Gleichheit gilt genau dann, wenn $x$ und $y$ linear unabhängig sind.
\end{lemma}

\begin{lemma}[Youngsche Ungleichung]
  \label{lem:young}
  Seien $a,b\in\Rbb$ und $\varepsilon>0$ beliebig. Dann gilt
  \begin{align*}
    ab\leq\frac{1}{\varepsilon}a^2+\frac{\varepsilon}{4}b^2. 
  \end{align*}
\end{lemma}

\begin{lemma}[Höldersche Ungleichung]
  \label{lem:hoelder}
  Seien $p,q\in [1,\infty]$ mit $1/p+1/q=1$, $f\in L^p(\Omega)$ und 
  $g\in L^q(\Omega)$. Dann gilt $fg\in L^1(\Omega)$ mit 
  \begin{align*}
    \Vert fg\Vert_{L^1(\Omega)}\leq 
    \Vert f\Vert_{L^p(\Omega)}\Vert g\Vert_{L^q(\Omega)}.
  \end{align*}
\end{lemma}

\todo[inline]{Q alle drei zitieren mit irgendeiner Quelle mit der passenden
Formulierung oder ist das zu basic}

\begin{theorem}[Existenz einer Lösung]
  \label{thm:contProblemExistence}
  \Cref{prob:continuousProblem} besitzt eine Lösung \\$u\in\BV(\Omega)\cap
  L^2(\Omega)$.
\end{theorem}

\begin{proof}
  Für alle $v\in L^2(\Omega)\subset L^1(\Omega)$ gilt mit der Hölderschen
  Ungleichung
  \todo[inline]{Q wo und wie einmal erwähnen, dass die Lp Räume geschachtelt 
  sind da Omega bdd ist?}
  (\cref{lem:hoelder}) für $p=q=2$, dass
  \begin{equation}\label{equ:hoelderL2BiggerL1}
    \Vert v\Vert_{L^1} 
    = \Vert 1\cdot v\Vert_{L^1(\Omega)}
    \leq \Vert 1\Vert_{L^2(\Omega)}\Vert v\Vert_{L^2(\Omega)}
    =|\Omega|^2 \Vert v\Vert_{L^1(\Omega)}.
  \end{equation}

  Dann folgt für das Funktional $E$ in \eqref{eq:continuousProblem}
  für alle $v\in \BV(\Omega)\cap L^2(\Omega)$ durch die
  Cauchy-Schwarzschen Ungleichung (\cref{lem:csu}), die Youngschen
  Ungleichung (\cref{lem:young}) und \cref{equ:hoelderL2BiggerL1}, dass
  \begin{equation}
    \label{eq:contProbBddFromBelow}
    \begin{aligned}
      E(v)&=\frac{\alpha}{2}\Vert v\Vert_{L^2(\Omega)}^2 + |v|_{\BV(\Omega)}
      +\Vert v\Vert_{L^1(\partial\Omega)}-\int_\Omega fv\dx\\
      &\geq 
      \frac{\alpha}{2}\Vert v\Vert_{L^2(\Omega)}^2 + |v|_{\BV(\Omega)}
      +\Vert v\Vert_{L^1(\partial\Omega)}
      -\Vert f\Vert_{L^2(\Omega)}\Vert v\Vert_{L^2(\Omega)}\\
      &\geq 
      \frac{\alpha}{2}\Vert v\Vert_{L^2(\Omega)}^2 + |v|_{\BV(\Omega)}
      +\Vert v\Vert_{L^1(\partial\Omega)}
      -\frac{1}{\alpha}\Vert f\Vert_{L^2(\Omega)}^2
      -\frac{\alpha}{4}\Vert v\Vert_{L^2(\Omega)}^2\\
      &\geq 
      \frac{\alpha}{4}\Vert v\Vert_{L^2(\Omega)}^2 + |v|_{\BV(\Omega)}
      +\Vert v\Vert_{L^1(\partial\Omega)}-\frac{1}{\alpha}\Vert
      f\Vert_{L^2(\Omega)}^2\\
      &\geq 
      \frac{\alpha}{4|\Omega|}\Vert v\Vert_{L^1(\Omega)}^2 + |v|_{\BV(\Omega)}
      +\Vert v\Vert_{L^1(\partial\Omega)}-\frac{1}{\alpha}\Vert
      f\Vert_{L^2(\Omega)}^2\\
      &\geq -\frac{1}{\alpha}\Vert f\Vert_{L^2(\Omega)}^2.
    \end{aligned}
  \end{equation}

  Somit ist $E$ nach unten beschränkt, was die Existenz einer infimierenden
  Folge $(u_n)_{n\in\Nbb}\subset\BV(\Omega)\cap L^2(\Omega)$ von $E$ 
  impliziert, d.h.\
  $(u_n)_{n\in\Nbb}$ erfüllt $\lim_{n\rightarrow\infty}E(u_n) =
  \inf_{v\in\BV(\Omega)\cap L^2(\Omega)}E(v)$. 

  \Cref{eq:contProbBddFromBelow} impliziert außerdem, dass
  $E(u_n)\overset{n\to\infty}{\to}\infty$, falls
  $|u_n|_{\BV(\Omega)}\overset{n\to\infty}{\to}\infty$ oder 
  $\Vert u_n\Vert_{L^1(\Omega)}\overset{n\to\infty}{\to}\infty$, also 
  insgesamt, falls
  $\Vert u_n\Vert_{\BV(\Omega)}\overset{n\to\infty}{\to}\infty$.

  Deshalb muss die Folge $(u_n)_{n\in\Nbb}$ beschränkt in $\BV(\Omega)$ sein.

  \medbreak
  Nun garantiert \cref{thm:compactness} die Existenz einer schwach konvergenten
  Teilfolge $(u_{n_k})_{k\in\Nbb}$ von $(u_n)_{n\in\Nbb}$ mit schwachem Grenzwert
  $u\in\BV(\Omega)$. Ohne Beschränkung der Allgemeinheit
  ist $(u_{n_k})_{k\in\Nbb}=(u_n)_{n\in\Nbb}$.
  Schwache Konvergenz von $(u_n)_{n\in\Nbb}$ in $\BV(\Omega)$ gegen
  $u$ bedeutet insbesondere, dass $(u_n)_{n\in\Nbb}$ auch stark, und damit
  auch schwach, in $L^1(\Omega)$ gegen $u$ konvergiert.

  \medbreak
  Weiterhin folgt aus \cref{eq:contProbBddFromBelow}, dass
  $E(v)\rightarrow\infty$ für
  $\Vert v\Vert_{L^2(\Omega)}\rightarrow\infty$. Somit muss $(u_n)_{n\in\Nbb}$ 
  auch beschränkt sein bezüglich der Norm $\Vert\bullet\Vert_{L^2(\Omega)}$ und
  besitzt deshalb, wegen der Reflexivität von $L^2(\Omega)$,
  eine Teilfolge (ohne Beschränkung der Allgemeinheit weiterhin bezeichnet mit
  $(u_n)_{n\in\Nbb}$), die
  in $L^2(\Omega)$ schwach gegen einen Grenzwert 
  $\tilde{u}\in L^2(\Omega)$ konvergiert. Somit gilt für alle $w\in
  L^2(\Omega)$ und, da $L^\infty(\Omega)\subset L^2(\Omega)$, insbesondere 
  auch für alle $w\in
  L^\infty(\Omega)$, dass 
  $\int_\Omega u_n w\dx \overset{n\to\infty}{\to}\int_\Omega \tilde{u} w\dx$. 
  Damit konvergiert $(u_n)_{n\in\Nbb}$ also auch schwach in $L^1(\Omega)$ gegen
  $\tilde{u}\in L^2(\Omega)\subset L^1(\Omega)$. 

  \medbreak
  Da schwache Grenzwerte eindeutig bestimmt sind, gilt insgesamt $u=\tilde u
  \in L^2(\Omega)$, das heißt $u\in\BV(\Omega)\cap
  L^2(\Omega)$.

  \medbreak
  Wir wissen, dass $(u_n)_{n\in\Nbb}$ beschränkt in $\BV(\Omega)$
  ist und in $L^1(\Omega)$ gegen $u\in \BV(\Omega)\cap L^2(\Omega)\subset
  L^1(\Omega)$ konvergiert. \cref{thm:wlsc} impliziert unter diesen 
  Voraussetzungen, dass $|u|_{\BV(\Omega)}\leq \liminf_{n\to\infty} 
  |u_n|_{\BV(\Omega)}$.
  %und $\Vert\bullet\Vert_{L^2(\Omega)}$ und $-\int_\Omega f\bullet\dx$ sind 
  %stetig (d.h.\ insbesondere 
  \bigbreak
  %TODO

  Damit gilt insgesamt
  \begin{align*}
    \inf_{v\in\BV(\Omega)\cap L^2(\Omega)}E(v)\leq
    E(u)\leq\liminf_{n\rightarrow\infty} E(u_n) =
    \lim_{n\rightarrow\infty}E(u_n) = \inf_{v\in\BV(\Omega)\cap
    L^2(\Omega)}E(v),
  \end{align*}
  d.h.\ $\min_{v\in\BV(\Omega)\cap L^2(\Omega)} E(v) = E(u)$.
\end{proof}

\begin{theorem}[Stabilität und Eindeutigkeit]
  \label{thm:contProbStabAndUniqu}
  Seien $u_1,u_2\in \BV(\Omega)\cap L^2(\Omega)$ die Minimierer des Problems
  \ref{prob:continuousProblem} mit $f_1,f_2\in L^2(\Omega)$ anstelle von $f$.

  Dann gilt 
  \begin{align*}
    \Vert u_1 - u_2\Vert_{L^2(\Omega)} \leq \Vert f_1-f_2\Vert_{L^2(\Omega)}.
  \end{align*}
\end{theorem}
\todo{hier noch alpha entsprechend ergänzen?}

\begin{proof}
  Definiere die konvexen Funktionale $F:\BV(\Omega)\to \Rbb$ und 
  $G_\ell:L^2(\Omega)\to \Rbb$, $\ell=1,2$, durch
  \begin{align*}
    F(u) &\coloneqq |u|_{\BV(\Omega)} + \Vert u \Vert_{L^1(\partial\Omega)},&
    G_\ell(u)\coloneqq \frac{\alpha}{2}\Vert u\Vert_{L^2(\Omega)}^2 -
    \int_\Omega f_\ell u\dx.
  \end{align*}
  Bezeichne $E_\ell\coloneqq F+G_\ell$ und setze $F$ auf $L^2(\Omega)$
  durch $\infty$ fort.

  $G_\ell$ ist Fr\'echet-differenzierbar 

  \todo[inline]{TODO nachrechnen,
  Gateaux ist klar, aber auch Frechet? EDIT: nachgerechnet, es funktioniert
  nach WIKI Def.
  
  Außerdem: Im Grundlagen Kapitel noch einführen, was hier in dieser Arbeit
  mit Gateaux, Frechet etc gemeint ist? (ist ja von Autor zu Autor anders (cf
  Wiki))} 

  und die Fr\'echet-Ableitung $\delta G_\ell(u):L^2(\Omega)\to\Rbb^2 $ an der
  Stelle
  $u\in L^2(\Omega)$ ist gegeben durch
  \begin{align*}
    \delta G_\ell(u)[v] = \alpha (u,v)_{L^2(\Omega)} - \int_\Omega f_\ell v\dx 
    = (\alpha u-f_\ell ,v)_{L^2(\Omega)} \quad 
    \text{ für alle } v\in L^2(\Omega).
  \end{align*}

  Das Funktional $F$ ist konvex {\color{red} wegen Dreiecksungleichung und 
  homogenität ($\lambda$ und $1-\lambda$ sind größer gleich 0)}, deshalb 
  \todo[inline]{TODO quote Rockafella (pp. 213/Section 23)} ist das
  Subdifferential
  $\partial F$ von $F$ monoton, d.h.\ für alle $\mu_\ell\in \partial F(u_\ell)$,
  $\ell=1,2$, gilt
  \begin{align}\label{eq:stabilityAndUniqueness:monotonicityOfSubdifferential}
    (\mu_1-\mu_2,u_1-u_2)_{L^2(\Omega)}\geq 0.
  \end{align}

  Für $\ell=1,2$ wird $E_\ell$ von $u_\ell$ minimiert, 
  deshalb gilt $0\in\partial E_\ell(u_\ell)
  = \partial F(u_\ell)+\partial G_\ell(u_\ell)=\partial F(u_\ell)+
  \{\delta G_\ell(u_\ell)\}$ \todo[inline]{TODO quote as well} und es folgt
  $-\delta G_\ell(u_\ell)\in\partial F(u_\ell)$.
  Daraus folgt zusammen mit
  \eqref{eq:stabilityAndUniqueness:monotonicityOfSubdifferential}
  \begin{align*}
    \big( -(\alpha u_1 - f_1) + (\alpha u_2 - f_2), u_1 - u_2\big)_{L^2(\Omega)}
    \geq 0.
  \end{align*}
  Umformen und Anwenden der Cauchy-Schwarzschen Ungleichung impliziert
  \begin{align*}
    \alpha \Vert u_1 - u_2 \Vert_{L^2(\Omega)}^2
    &\leq
    \big(f_1 -f_2, u_1-u_2 \big)_{L^2(\Omega)}\\
    &\leq
    \Vert f_1-f_2\Vert_{L^2(\Omega)}\Vert u_1 - u_2\Vert_{L^2(\Omega)}.
  \end{align*}

  \todo[inline]{TODO wie verschwindet das $\alpha$ hier um dann die
  Abschätzung aus dem Satz zu bekommen (für $alpha\geq 1$ haben wir sie ja
  auch, aber $\alpha>0$ ist nur gefordert? Oder verpasst man der
  Abschätzung im Satz noch eine Konstante $1/\alpha$ vor der oberen Schranke?

  Zumindest Eindeutigkeit bekommt man auch hier immernoch (man
  setzt f1 und f2 im Satz gleich f), was gut ist.
  Reicht das also sogar?
  
  Das entspricht auch der Abschätzung aus Bartels, da f=alpha g. Wenn man das
  ganze mit g schreibt, steht exakt das aus Bartels da}

  Falls $\Vert u_1 - u_2 \Vert_{L^2(\Omega)} = 0$, gilt der Satz.
  Ansonsten führt Division durch \\
  $\Vert u_1 - u_2 \Vert_{L^2(\Omega)}\neq 0$ den 
  Beweis zum Abschluss.
\end{proof}
