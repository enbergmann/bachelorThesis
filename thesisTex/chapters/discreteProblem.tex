\section{Formulierung}
\label{sec:discreteProblemFormulation}
Bevor wir \Cref{prob:continuousProblem} diskretisieren, merken wir an,
dass $\CR^1(\Tcal)\subset\BV(\Omega)$, da
\begin{align*}
  |\vcr|_{\BV(\Omega)} 
  = 
  \Vert \gradnc \vcr\Vert_{L^1(\Omega)} 
  + \sum_{F\in\Ecal(\Omega)}\Vert[\vcr]_F\Vert_{L^1(F)}
  \quad\text{für alle }\vcr\in\CR^1(\Tcal).
\end{align*} 
Dies wird für $|\Tcal|=2$ zum Beispiel von \cites[S. 404, Example
10.2.1]{ABM14}[S. 301, Proposition 10.1]{Bar15} impliziert und kann
analog für beliebige reguläre Triangulierungen von $\Omega$ bewiesen
werden.
Damit gilt für alle $\vcr\in\CR^1(\Tcal)$ insbesondere
\begin{align*}
  |\vcr|_{\BV(\Omega)} +\Vert\vcr\Vert_{L^1(\partial\Omega)} 
  = \Vert \gradnc \vcr\Vert_{L^1(\Omega)} +
  \sum_{F\in\Ecal}\Vert[\vcr]_F\Vert_{L^1(F)}.
\end{align*}
Um eine nichtkonforme Formulierung von \Cref{prob:continuousProblem} zu 
erhalten, ersetzen wir die Terme 
$|\bullet|_{\BV(\Omega)} +\Vert\bullet\Vert_{L^1(\partial\Omega)}$ des
Funktionals $E$ durch 
$\Vert \gradnc \bullet\Vert_{L^1(\Omega)}$, das heißt wir vernachlässigen
bei der nichtkonformen Formulierung die Terme
$\sum_{F\in\Ecal}\Vert[\bullet]_F\Vert_{L^1(F)}$.
Somit erhalten wir das folgende Minimierungsproblem für den Parameter
$\alpha\in\Rbb_+$ und das Eingangssignal $f\in L^2(\Omega)$.

\begin{problem}\label{prob:discreteProblem}
  Finde $\ucr\in \CR^1_0(\Tcal)$,
  sodass $\ucr$ das Funktional
  \begin{align}\label{eq:discreteProblem}
    \Enc(\vcr)\coloneqq \frac{\alpha}{2}\Vert \vcr\Vert^2
    +\Vert \gradnc\vcr\Vert_{L^1(\Omega)}-\int_\Omega f\vcr\dx
  \end{align}
  unter allen $\vcr\in \CR^1_0(\Tcal)$ minimiert.
\end{problem}

\section{Charakterisierung und Existenz eines eindeutigen Minimierers}

In diesem Abschnitt führen wir die Argumente in \cite[S. 313]{Bar15}, angepasst
für unsere Formulierung in \Cref{prob:discreteProblem}, detailliert aus. 
Zunächst zeigen wir, dass \Cref{prob:discreteProblem} eine eindeutige Lösung
besitzt. Dafür benötigen wir folgendes Lemma.
\begin{lemma}
  \label{lem:normOfGradNcContiuous}
  Das Funktional $\Enc$ aus \Cref{eq:discreteProblem} ist stetig bezüglich der
  Konvergenz in $L^2(\Omega)$.
\end{lemma}

\begin{proof}
  Die Folge $(v_n)_{n\in\Nbb}\subset\CR^1_0(\Tcal)$ konvergiere
  gegen $\vcr\in\CR^1_0(\Tcal)$ bezüglich der Norm $\Vert\bullet\Vert$.
  Damit ist $(v_n)_{n\in\Nbb}$ insbesondere beschränkt in $L^2(\Omega)$ und es
  gilt mit einer binomischen Formel und der umgekehrten Dreiecksungleichung,
  dass
  \begin{align*}
    \left|\Vert\vcr\Vert^2-\Vert v_n\Vert^2\right|
    &=
    \big|\Vert\vcr\Vert-\Vert v_k\Vert\big|\, 
    \big|\Vert \vcr\Vert+\Vert v_k\Vert\big|\\
    &\leq
    \Vert\vcr- v_k\Vert\, \big|\Vert \vcr\Vert+\Vert v_k\Vert\big|
    \to 0\quad\text{für }n\to\infty.
  \end{align*}
  Außerdem gilt mit der Hölderschen Ungleichung
  \begin{align*}
    \left|\int_\Omega f(\vcr-v_k)\dx\right|
    \leq \Vert f\Vert \Vert\vcr-v_k\Vert\to 0\quad\text{für }n\to\infty.
  \end{align*}
  Schließlich gilt für alle $n\in\Nbb$ und alle $T\in\Tcal$ mit der 
  inversen Ungleichung (cf. \cite[S. 53, Lemma 3.5]{Bar15})
  mit Konstante $c_T\in\Rbb_+$ und der Hölderschen Ungleichung, dass
  \begin{equation*}
    \label{eq:continuityProofTriangleWiseEstimate}
    \Vert\gradnc(\vcr- v_n)\Vert_{L^1(T)}
    \leq
    c_T h_T^{-1}\Vert\vcr- v_n\Vert_{L^1(T)}
    \leq
    c_T h_T^{-1}\sqrt{|T|}\Vert\vcr- v_n\Vert_{L^2(T)}.
  \end{equation*}
  Damit folgt zusammen mit der umgekehrten Dreiecksungleichung
  \begin{align*}
    \left|\Vert\gradnc\vcr\Vert_{L^1(\Omega)}-
    \Vert \gradnc v_n\Vert_{L^1(\Omega)}\right|
    &\leq 
    \Vert\gradnc(\vcr- v_n)\Vert_{L^1(\Omega)}\\
    &=
    \sum_{T\in\Tcal}\Vert\gradnc(\vcr- v_n)\Vert_{L^1(T)}\\
    &\leq
    \max_{T\in\Tcal}\left(c_T h_T^{-1}\sqrt{|T|}\right)
    \sum_{T\in\Tcal}\Vert\vcr- v_n\Vert_{L^2(T)}\\
    &=
    \max_{T\in\Tcal}\left(c_T h_T^{-1}\sqrt{|T|}\right) \Vert\vcr- v_n\Vert
    \to 0\quad\text{für }n\to\infty.
  \end{align*}
  Somit ist $\Enc$ Summe von drei Termen, die bezüglich der Norm
  $\Vert\bullet\Vert$ folgenstetig sind, und deshalb stetig bezüglich der
  Konvergenz in $L^2(\Omega)$.
\end{proof}

\begin{theorem}
  \label{thm:discreteProblemExistenceUniqueness}
  Es existiert eine eindeutige Lösung $\ucr\in\CR^1_0(\Tcal)$ von
  \Cref{prob:discreteProblem}.
\end{theorem}

\begin{proof}
  Mit analogen Abschätzungen wie in \eqref{eq:contProbBddFromBelow}
  erhalten wir für das Funktional $\Enc$ aus \Cref{prob:discreteProblem} 
  für alle $\vcr\in\CR^1_0(\Tcal)\subset L^2(\Omega)$ die Ungleichung 
  \begin{equation}
    \label{eq:discreteEnergyCoercivity}
    \Enc(\vcr) 
    \geq 
    \frac{\alpha}{4}\Vert \vcr\Vert^2
    +\Vert \gradnc\vcr\Vert_{L^1(\Omega)}
    -\frac{1}{\alpha}\Vert f\Vert^2
    \geq 
    -\frac{1}{\alpha}\Vert f\Vert^2.
  \end{equation}
  Somit ist $\Enc$ nach unten beschränkt und es existiert eine infimierende
  Folge $(v_n)_{n\in\Nbb} \subset \CR^1_0(\Tcal)$ von $\Enc$. 
  Ungleichung \eqref{eq:discreteEnergyCoercivity} impliziert weiterhin, dass
  diese Folge beschränkt bezüglich der Norm $\Vert\bullet\Vert$ sein muss.
  Der endlichdimensionale Raum $\CR^1_0(\Tcal)$ ist, ausgestattet mit der Norm
  $\Vert\bullet\Vert$, ein Banachraum und damit reflexiv. 
  Demnach existiert eine in $\CR^1_0(\Tcal)$ schwach konvergente Teilfolge von
  $(v_n)_{n\in\Nbb}$.
  Da $\CR^1_0(\Tcal)$ endlichdimensional ist, konvergiert diese sogar stark
  in $L^2(\Omega)$. 
  Weil $\CR^1_0(\Tcal)$ ein Banachraum und damit abgeschlossen bezüglich der
  Konvergenz in $\Vert\bullet\Vert$ ist, gilt für den Grenzwert $\ucr$ dieser
  Teilfolge, dass $\ucr\in\CR^1_0(\Tcal)$.
  Nach \Cref{lem:normOfGradNcContiuous} ist $\Enc$ stetig bezüglich der
  Konvergenz in $L^2(\Omega)$, was impliziert, dass $\ucr$ Minimierer von
  $\Enc$ in $\CR^1_0(\Tcal)$ sein muss.   
  Dieser Minimierer $\ucr$ ist eindeutig, da $\Enc$ strikt konvex ist.
\end{proof}

Als Nächstes wollen wir äquivalente Charakterisierungen der eindeutigen Lösung
von \Cref{prob:discreteProblem} beweisen, die von Professor Carstensen
formuliert wurden.
Dazu leiten wir zunächst ein zu \Cref{prob:discreteProblem} äquivalentes
Minimaxproblem nach \cite[Section 36]{Roc70} her.
Wir betrachten die konvexe Menge 
\begin{align*}
  K
  \coloneqq 
  \left\{\Lambda\in L^\infty\!\left(\Omega;\Rbb^2\right)
  \,\middle|\,|\Lambda(\bullet)| \leq 1 \text{ fast überall in }\Omega\right\}
\end{align*}
und das dazugehörige Indikatorfunktional
$I_K:L^\infty\!\left(\Omega;\Rbb^2\right)\to\Rbb\cup\{\infty\}$, das für
$\Lambda\in L^\infty\!\left(\Omega;\Rbb^2\right)$ gegeben ist durch
\begin{align*}
  I_K(\Lambda)
  &\coloneqq
  \begin{cases}
    \infty, & \text{falls } \Lambda\notin K,\\
    0,       & \text{falls } \Lambda\in K.
  \end{cases}
\end{align*} 
Aufgrund der Konvexität von $K$ ist $I_K$ konvex.
Für $\vcr\in\CR^1_0(\Tcal)$ und $\Lambda_0\in
P_0\!\left(\Tcal;\Rbb^2\right)\subset L^\infty\!\left(\Omega;\Rbb^2\right)$
können wir damit die Sattelfunktion $L:\CR^1_0(\Tcal)\times
P_0\!\left(\Tcal;\Rbb^2\right)\to [-\infty,\infty)$ nach \cite[Section
33]{Roc70} definieren durch
\begin{align}\label{eq:discreteProblemLagrangeFunctional}
  L(\vcr,\Lambda_0) \coloneqq \int_\Omega\Lambda_0\cdot\gradnc\vcr\dx +
  \frac{\alpha}{2}\Vert \vcr\Vert^2 -\int_\Omega f\vcr\dx
  - I_K(\Lambda_0).
\end{align}
Nun wählen wir $\vcr\in\CR^1_0(\Tcal)$ beliebig. 
Mit der Cauchy-Schwarzschen Ungleichung gilt für alle
$\Lambda_0\in P_0\!\left(\Tcal;\Rbb^2\right)\cap K$, dass
\begin{align*}
  \int_\Omega \Lambda_0\cdot\gradnc\vcr\dx
  \leq 
  \int_\Omega |\Lambda_0||\gradnc\vcr|\dx
  \leq 
  \Vert\gradnc\vcr\Vert_{L^1(\Omega)}.
\end{align*}
Daraus folgt
\begin{align}
  \label{eq:saddlepointLeqEnergy}
  \sup_{\Lambda_0\in P_0\left(\Tcal;\Rbb^2\right)\cap K}L(\vcr,\Lambda_0)
  \leq \Enc(\vcr).
\end{align}
Weiterhin gilt, wenn wir $\Lambda_0\in P_0\!\left(\Tcal;\Rbb^2\right)\cap K$
mit der Signumfunktion aus \Cref{eq:signumFunction} elementweise auf allen
$T\in\Tcal$ definieren durch $\Lambda_0(x)\in\sign\left(\gradnc\vcr(x)\right)$
für alle $x\in \interior(T)$, dass $L(\vcr,\Lambda_0)=\Enc(\vcr)$ und deshalb
auch
\begin{align}
  \label{eq:saddlepointGeqEnergy}
  \Enc(\vcr)
  \leq
  \sup_{\Lambda_0\in P_0\left(\Tcal;\Rbb^2\right)\cap K}L(\vcr,\Lambda_0).
\end{align}
Außerdem ist $L(\vcr,\Lambda_0)>-\infty$ genau dann, wenn $\Lambda_0\in K$.
Damit folgt aus den Ungleichungen \eqref{eq:saddlepointLeqEnergy} und
\eqref{eq:saddlepointGeqEnergy} insgesamt
\begin{equation*}
  \label{eq:discreteEnergySaddlefunctionalEquality}
  \Enc(\vcr)
  =\sup_{\Lambda_0\in P_0\left(\Tcal;\Rbb^2\right)\cap K}L(\vcr,\Lambda_0)
  =\sup_{\Lambda_0\in P_0\left(\Tcal;\Rbb^2\right)}L(\vcr,\Lambda_0).
\end{equation*}
Wenn also das folgende Minimaxproblem
\ref{prob:discreteSaddlepointProblem} eine Lösung $\left(
\tilde{u}_\CR,\bar\Lambda_0 \right)\in\CR^1_0(\Tcal)\times
P_0\!\left(\Tcal;\Rbb^2\right)$ hat, dann löst die Funktion $\tilde{u}_\CR$
\Cref{prob:discreteProblem}.

\begin{problem}\label{prob:discreteSaddlepointProblem}
  Finde $\left( \tilde{u}_\CR,\bar\Lambda_0 \right)\in\CR^1_0(\Tcal)\times
  P_0\!\left(\Tcal;\Rbb^2\right)$,
  sodass
  \begin{align*}
    L(\tilde{u}_\CR,\bar\Lambda_0) 
    = 
    \inf_{\vcr\in\CR^1_0(\Tcal)}\sup_{\Lambda_0\in P_0\left(\Tcal;\Rbb^2\right)}
    L(\vcr,\Lambda_0).
  \end{align*}
\end{problem}

\begin{lemma}
  \label{lem:existenceSaddlepoint}
  Es existiert eine Lösung $\left( \tilde{u}_\CR,\bar\Lambda_0
  \right)\in\CR^1_0(\Tcal)\times \left(P_0\!\left(\Tcal;\Rbb^2\right)\cap
  K\right)$ von \Cref{prob:discreteSaddlepointProblem}.
\end{lemma}

\begin{proof}
  Die Sattelfunktion $L$ aus \Cref{eq:discreteProblemLagrangeFunctional} ist,
  wenn ihre zweite Komponente in $P_0\!\left(\Tcal;\Rbb^2\right)\cap K$ fixiert
  ist, in ihrer ersten Komponente eine konvexe, unterhalbstetige, auf
  $\CR^1_0(\Tcal)$ reellwertige Funktion und in ihrer zweiten Komponente eine
  konkave, oberhalbstetige, auf $P_0\!\left(\Tcal;\Rbb^2\right)\cap K$
  reellwertige Funktion.
  Somit ist $L$ in beiden Komponenten abgeschlossen nach \cite[S. 52,
  308]{Roc70}.  
  Insgesamt ist $L$ damit eine konvex-konkave, propere und abgeschlossene
  Funktion nach \cite[S. 349, 362 f.]{Roc70}, deren effektiver
  Definitionsbereich nach \cite[362]{Roc70} die Menge $\CR^1_0(\Tcal)\times
  \left( P_0\!\left(\Tcal;\Rbb^2\right)\cap K\right)$ ist.
  Unter Beachtung der Isomorphie von $\CR^1_0(\Tcal)$ zu
  $\Rbb^{|\Ecal(\Omega)|}$ und der Isomorphie von
  $P_0\!\left(\Tcal;\Rbb^2\right)$ zu $\Rbb^{2|\Tcal|}$, folgt aus 
  \cite[S. 397, Theorem 37.6]{Roc70} die Existenz eines Sattelpunkts
  $\left(\tilde{u}_\CR,\bar\Lambda_0\right)\in \CR^1_0(\Tcal)\times \left(
  P_0\!\left(\Tcal;\Rbb^2\right)\cap K\right)$ von $L$ nach \cite[380]{Roc70}.
  Für diesen impliziert \cite[S. 380, Lemma 36.2]{Roc70}, dass 
  \begin{align*}
    \sup_{\Lambda_0\in P_0\left(\Tcal;\Rbb^2\right)}\inf_{\vcr\in\CR^1_0(\Tcal)}
    L(\vcr,\Lambda_0)
    =
    L(\tilde{u}_\CR,\bar\Lambda_0) 
    = 
    \inf_{\vcr\in\CR^1_0(\Tcal)}\sup_{\Lambda_0\in P_0\left(\Tcal;\Rbb^2\right)}
    L(\vcr,\Lambda_0).
  \end{align*}
  Somit löst $\left(\tilde{u}_\CR,\bar\Lambda_0\right)\in \CR^1_0(\Tcal)\times
  \left( P_0\!\left(\Tcal;\Rbb^2\right)\cap K\right)$
  \Cref{prob:discreteSaddlepointProblem}.
\end{proof}

Nachdem diese Vorbereitungen abgeschlossen sind, können wir nun folgendes
Theorem beweisen.

\begin{theorem}
  \label{thm:discProbCharacterizationOfDiscreteSolutions}
  Für eine Funktion $\tilde{u}_\CR\in\CR^1_0(\Tcal)$ sind die folgenden drei
  Aussagen äquivalent.
  \begin{itemize}
    \item[(i)] \Cref{prob:discreteProblem} wird von $\tilde{u}_\CR$ gelöst.
    \item[(ii)] Es existiert ein
      $\bar\Lambda_0\in P_0\!\left(\Tcal;\Rbb^2\right)$ mit
      $\left|\bar\Lambda_0(\bullet)\right|\leq 1$
      fast überall in $\Omega$, sodass
      \begin{equation}
        \label{eq:discreteMultiplierScalerProductEquality}
        \bar\Lambda_0(\bullet)\cdot\gradnc\tilde{u}_\CR(\bullet)
        =
        \left|\gradnc\tilde{u}_\CR(\bullet)\right| 
        \quad\text{fast überall in } \Omega 
      \end{equation}
      und
      \begin{equation}
        \label{eq:discreteMultiplierL2Equality}
        \left(\bar\Lambda_0,\gradnc\vcr\right)
        = \left(f-\alpha\tilde{u}_\CR,
        \vcr\right)
        \quad\text{für alle } \vcr\in\CR^1_0(\Tcal).
      \end{equation}
    \item[(iii)] Für alle $\vcr\in\CR^1_0(\Tcal)$ gilt
      \begin{equation}
        \label{eq:discreteVariationalInequality}
        \left(f-\alpha\tilde{u}_\CR,\vcr-\tilde{u}_\CR\right)\leq
        \Vert\gradnc\vcr\Vert_{L^1(\Omega)} -
        \left\Vert\gradnc\tilde{u}_\CR\right\Vert_{L^1(\Omega)}.
      \end{equation}
  \end{itemize}
\end{theorem}

\begin{proof} 
  Sei $\tilde{u}_\CR\in\CR^1_0(\Tcal)$.

  \textit{(i) $\Rightarrow$ (ii).}
  Sei $\tilde{u}_\CR$ Lösung von \Cref{prob:discreteProblem}.
  Nach \Cref{lem:existenceSaddlepoint} existiert eine Lösung
  $\left(\hat{u}_\CR,\bar\Lambda_0\right)\in \CR^1_0(\Tcal)\times \left(
  P_0\!\left(\Tcal;\Rbb^2\right)\cap K\right)$ 
  von \Cref{prob:discreteSaddlepointProblem}. 
  Außerdem wissen wir, dass damit $\hat{u}_\CR$ Lösung von
  \Cref{prob:discreteProblem} ist.
  Daraus folgt, da nach \Cref{thm:discreteProblemExistenceUniqueness} die
  Lösung von \Cref{prob:discreteProblem} eindeutig ist, dass
  $\hat{u}_\CR=\tilde{u}_\CR$ in $\CR^1_0(\Tcal)$.
  Weiterhin wissen wir aus dem Beweis von \Cref{lem:existenceSaddlepoint}, dass
  $\left(\tilde{u}_\CR,\bar\Lambda_0\right)$ Sattelpunkt der Funktion $L$ aus
  \Cref{eq:discreteProblemLagrangeFunctional} ist.
  Das bedeutet nach \cite[380]{Roc70} insbesondere, dass $\tilde{u}_\CR$
  Minimierer von $L(\bullet, \bar\Lambda_0)$ in $\CR^1_0(\Tcal)$ und
  $\bar\Lambda_0$ Maximierer von $L\!\left(\tilde{u}_\CR,\bullet\right)$ in 
  $P_0\!\left(\Tcal;\Rbb^2\right)$ ist.  
  Mit dieser Erkenntnis können wir nun die entsprechenden
  Optimalitätsbedingungen diskutieren.
  Zunächst gilt, da
  $L\!\left(\tilde{u}_\CR,\bullet\right):P_0\!\left(\Tcal;\Rbb^2\right)\to
  [-\infty,\infty)$ konkav und
  $\bar\Lambda_0$ Maximierer von $L\!\left(\tilde{u}_\CR,\bullet\right)$ in 
  $ P_0\!\left(\Tcal;\Rbb^2\right)$ ist, dass das konvexe Funktional
  $-L(\tilde{u}_\CR,\bullet):P_0\!\left(\Tcal;\Rbb^2\right)\to
  (-\infty,\infty]$ von $\bar\Lambda_0$ in $ P_0\!\left(\Tcal;\Rbb^2\right)$
  minimiert wird.
  %Nach den Theoremen \ref{thm:extremalprinciple},
  %\ref{thm:subdifferentialSumRule} und \ref{thm:subdiffGateaux} gilt somit
  Nach den Theoremen \ref{thm:extremalprinciple} --
  \ref{thm:subdifferentialSumRule} gilt somit
  \begin{align*}
    0
    \in 
    \partial \left(-L\!\left(\tilde{u}_\CR,\bullet\right)\right)
    \left(\bar\Lambda_0\right) 
    =
    \left\{-\!\left(\gradnc\tilde{u}_\CR,\bullet\right)\right\}+\partial I_K
    \left(\bar\Lambda_0\right).
  \end{align*}
  Äquivalent zu dieser Aussage ist, dass
  $\left(\gradnc\tilde{u}_\CR,\bullet\right)\in \partial
  I_K \left(\bar\Lambda_0\right)$. 
  Da $\bar\Lambda_0\in K$, folgt mit \Cref{def:subdifferential},
  dass für alle $\Lambda_0\in  P_0\!\left(\Tcal;\Rbb^2\right)$ gilt
  \begin{align*}
    \left(\gradnc\tilde{u}_\CR,\Lambda_0-\bar\Lambda_0\right) 
    \leq 
    I_K (\Lambda_0) - I_K\!\left(\bar\Lambda_0\right)
    =
    I_K (\Lambda_0).
  \end{align*}
  Falls $\Lambda_0\in  P_0\!\left(\Tcal;\Rbb^2\right)\cap K$, folgt insbesondere
  \begin{align}
    \label{eq:scalarProductInequDiscreteProof}
    \left(\gradnc\tilde{u}_\CR,\Lambda_0-\bar\Lambda_0\right) 
    &\leq 
    0,\quad\text{also }\notag\\ 
    \left(\gradnc\tilde{u}_\CR,\Lambda_0\right)
    &\leq
    \left(\gradnc\tilde{u}_\CR,\bar\Lambda_0\right).
  \end{align}
  Sei nun $\Lambda_0\in P_0\!\left(\Tcal;\Rbb^2\right)\cap K$ elementweise auf
  allen $T\in\Tcal$ durch $\Lambda_0(x)\in\sign\left(\gradnc\tilde{u}_\CR(x)\right)$
  definiert für alle $x\in\interior(T)$.
  Mit dieser Wahl von $\Lambda_0$, Ungleichung
  \eqref{eq:scalarProductInequDiscreteProof}, der Cauchy\--Schwarz\-schen
  Ungleichung und $\bar\Lambda_0\in K$ erhalten wir die Abschätzung
  \begin{align}
    \label{eq:sumOverAllTrianglesDualVariable}
    \int_\Omega\left|\gradnc\tilde{u}_\CR\right|\dx
    &=
    \int_\Omega\gradnc\tilde{u}_\CR\cdot\Lambda_0\dx
    \leq 
    \int_\Omega\gradnc\tilde{u}_\CR\cdot\bar\Lambda_0\dx \notag\\
    &\leq 
    \int_\Omega\left|\gradnc\tilde{u}_\CR\right|\left|\bar\Lambda_0\right|\dx
    \leq
    \int_\Omega\left|\gradnc\tilde{u}_\CR\right|\dx,
    \quad\text{das heißt }\notag\\
    \int_\Omega\left|\gradnc\tilde{u}_\CR\right|\dx 
    &= 
    \int_\Omega\gradnc\tilde{u}_\CR\cdot\bar\Lambda_0\dx
    \quad\text{beziehungsweise }\notag\\
    \sum_{T\in\Tcal}|T|\,\big|(\gradnc\tilde{u}_\CR)\!|_T\big|
    &=
    \sum_{T\in\Tcal}|T|\left(\gradnc\tilde{u}_\CR\cdot \bar\Lambda_0\right)\!\!|_T.
  \end{align}
  Außerdem gilt für alle $T\in\Tcal$ mit der Cauchy-Schwarzschen Ungleichung
  und $\bar\Lambda_0\in K$, dass 
  \begin{align*}
    \left(\gradnc\tilde{u}_\CR\cdot \bar\Lambda_0\right)\!\!|_T
  \leq
  \big|(\gradnc\tilde{u}_\CR)\!|_{T}\big|\,\left|\bar\Lambda_0|_T\right|
  \leq
  \big|(\gradnc\tilde{u}_\CR)\!|_{T}\big|.
  \end{align*}
  Mit \Cref{eq:sumOverAllTrianglesDualVariable} folgt daraus für alle
  $T\in\Tcal$, dass $\left(\gradnc\tilde{u}_\CR\cdot
  \bar\Lambda_0\right)\!\!|_T=\big|(\gradnc\tilde{u}_\CR)\!|_T\big|$, das heißt fast
  überall in $\Omega$ gilt $\bar\Lambda_0(\bullet)\cdot\gradnc\tilde{u}_\CR(\bullet)
  =|\gradnc\tilde{u}_\CR(\bullet)|$. 
  Damit ist \Cref{eq:discreteMultiplierScalerProductEquality} gezeigt.
  Als Nächstes betrachten wir das reellwertige Funktional
  $L\left(\bullet,\bar\Lambda_0\right):\CR^1_0(\Tcal)\to\Rbb$.
  Es ist Fr\'echet-differenzierbar mit
  \begin{align*}
    dL\!\left(\bullet,\bar\Lambda_0\right)\!\left(\tilde{u}_\CR;\vcr\right)
    =
    \int_\Omega\bar\Lambda_0\cdot \gradnc\vcr\dx
    +\alpha\! \left(\tilde{u}_\CR,\vcr\right) - \int_\Omega f\vcr\dx
  \end{align*}
  für alle $\vcr\in\CR^1_0(\Tcal)$.
  Da $\tilde{u}_\CR$ Minimierer von  $L\!\left(\bullet, \bar\Lambda_0\right)$
  in $\CR^1_0(\Tcal)$ ist, gilt nach
  \Cref{thm:necessaryConditionFreeLocalExtrema}, dass $0 =
  dL\!\left(\bullet,\bar\Lambda_0\right)\!\left(\tilde{u}_\CR;\vcr\right)$ für
  alle $\vcr\in\CR^1_0(\Tcal)$.
  Diese Bedingung ist für alle $\vcr\in\CR^1_0(\Tcal)$ äquivalent zu
  $\left(\bar\Lambda_0,\gradnc\vcr\right) = (f-\alpha \tilde{u}_\CR,\vcr)$.
  Somit ist auch \Cref{eq:discreteMultiplierL2Equality} gezeigt.

  \textit{(ii) $\Rightarrow$ (iii).}
  Die Funktion $\bar\Lambda_0\in P_0\!\left(\Tcal;\Rbb^2\right)$ erfülle
  $\left|\bar\Lambda_0(\bullet)\right|\leq 1$ fast überall in $\Omega$ sowie
  die Gleichungen \eqref{eq:discreteMultiplierScalerProductEquality} und 
  \eqref{eq:discreteMultiplierL2Equality}. 
  Sei $\vcr\in\CR^1_0(\Tcal)$.
  Mit den Gleichungen 
  \eqref{eq:discreteMultiplierL2Equality} und 
  \eqref{eq:discreteMultiplierScalerProductEquality} gilt
  \begin{equation}
    \label{eq:equivalentCharacterizationApplicationTwoEquations}
    \begin{aligned}
      \left(f-\alpha\tilde{u}_\CR,\vcr-\tilde{u}_\CR\right) 
      &=
      \left(\bar\Lambda_0,\gradnc\vcr\right)
      - \left(\bar\Lambda_0,\gradnc\tilde{u}_\CR\right)\\
      &=
      \int_\Omega\bar\Lambda_0\cdot\gradnc\vcr\dx
      - \int_\Omega\left|\gradnc\tilde{u}_\CR\right|\dx.
    \end{aligned}
  \end{equation}
  Weiterhin gilt mit der Cauchy-Schwarzschen
  Ungleichung und $\left|\bar\Lambda_0(\bullet)\right|\leq 1$ fast überall in
  $\Omega$, dass
  \begin{align*}
    \int_\Omega\bar\Lambda_0\cdot\gradnc\vcr\dx
    &\leq 
    \int_\Omega\left|\bar\Lambda_0\right|\,|\gradnc\vcr|\dx
    \leq 
    \int_\Omega|\gradnc\vcr|\dx.
  \end{align*}
  Zusammen mit \Cref{eq:equivalentCharacterizationApplicationTwoEquations}
  folgt daraus Ungleichung \eqref{eq:discreteVariationalInequality}.

  \textit{(iii) $\Rightarrow$ (i)}.
  Es gelte Ungleichung \eqref{eq:discreteVariationalInequality} für alle
  $\vcr\in\CR^1_0(\Tcal)$, also
  \begin{align*}
    \left(f-\alpha\tilde{u}_\CR,\vcr-\tilde{u}_\CR\right) 
    \leq
    \left\Vert\gradnc\vcr\right\Vert_{L^1(\Omega)}
    -\left\Vert\gradnc\tilde{u}_\CR\right\Vert_{L^1(\Omega)}.
  \end{align*}
  Nach \Cref{thm:discreteProblemExistenceUniqueness} existiert eine eindeutige
  Lösung $\ucr\in\CR^1_0(\Tcal)$ von \Cref{prob:discreteProblem}.
  Wir haben bereits gezeigt, dass somit für alle
  $\vcr\in\CR^1_0(\Tcal)$ gilt
  \begin{align*}
    \left(f-\alpha\ucr,\vcr-\ucr\right) 
    \leq
    \left\Vert\gradnc\vcr\right\Vert_{L^1(\Omega)}
    -\Vert\gradnc\ucr\Vert_{L^1(\Omega)}.
  \end{align*}
  Um nun zu beweisen, dass $\tilde{u}_\CR$ \Cref{prob:discreteProblem} löst, genügt
  es $\tilde{u}_\CR=\ucr$ in $\CR^1_0(\Tcal)$ zu zeigen.
  Es gilt
  \begin{align*}
    \left(f-\alpha\ucr,\tilde{u}_\CR-\ucr\right) 
    &\leq
    \left\Vert\gradnc\tilde{u}_\CR\right\Vert_{L^1(\Omega)}
    -\Vert\gradnc\ucr\Vert_{L^1(\Omega)}\quad\text{und }\\
    \left(f-\alpha\tilde{u}_\CR,\ucr-\tilde{u}_\CR\right) 
    &\leq
    \left\Vert\gradnc\ucr\right\Vert_{L^1(\Omega)}
    -\Vert\gradnc\tilde{u}_\CR\Vert_{L^1(\Omega)}. 
  \end{align*}
  Die Addition dieser Ungleichungen
  impliziert
  \begin{align*}
    \alpha\left\Vert\tilde{u}_\CR-\ucr\right\Vert^2=
    \left(-\alpha\ucr,\tilde{u}_\CR-\ucr\right) 
    + \left(-\alpha\tilde{u}_\CR,\ucr-\tilde{u}_\CR\right) 
    \leq
    0.
  \end{align*}
  Da $\alpha>0$, folgt daraus $\left\Vert\tilde{u}_\CR-\ucr\right\Vert^2=0$,
  also $\tilde{u}_\CR=\ucr$ in $\CR^1_0(\Tcal)$.
\end{proof}

Zum Schluss dieses Abschnitts wollen wir noch zwei Bemerkungen von Professor
Carstensen erwähnen und kurz deren Gültigkeit begründen.
Die erste Bemerkung ist eine äquivalente Charakterisierung der dualen Variable
$\bar\Lambda_0\in P_0\!\left(\Tcal;\Rbb^2\right)$ aus
\Cref{thm:discProbCharacterizationOfDiscreteSolutions} zur diskreten Lösung
$\ucr\in\CR^1_0(\Tcal)$ von \Cref{prob:discreteProblem}.

\begin{remark}
  Dass $\bar\Lambda_0\in P_0\!\left(\Tcal;\Rbb^2\right)$ fast überall in $\Omega$
  \Cref{eq:discreteMultiplierScalerProductEquality} und
  $|\bar\Lambda_0(\bullet)|\leq 1$ erfüllt, ist äquivalent zu der Bedingung
  $\bar\Lambda_0(x)\in\sign(\gradnc \ucr(x))$ für alle $x\in\interior(T)$ für
  alle $T\in\Tcal$.   
\end{remark}

\begin{proof}
  Dass die genannte Bedingung hinreichend ist, folgt direkt aus der Definition
  der Signumfunktion.
  Ihre Notwendigkeit folgt aus der folgenden Beobachtung.
  Da $\left|\bar\Lambda_0(\bullet)\right|\leq 1$ fast überall in $\Omega$, ist
  \Cref{eq:discreteMultiplierScalerProductEquality} eine Cauchy-Schwarzsche
  Ungleichung, bei der sogar Gleichheit gilt. 
  Dies ist genau dann der Fall, wenn $\bar\Lambda_0(\bullet)$ und
  $\gradnc\ucr(\bullet)$ fast überall in $\Omega$ linear abhängig sind.
\end{proof}
 
Daraus können wir folgern, unter welchen Umständen die duale Variable
$\bar\Lambda_0$ auf einem Dreieck $T\in\Tcal$ eindeutig bestimmt ist.

\begin{remark}
  Falls $\gradnc\ucr\neq 0$ auf $T\in\Tcal$, gilt nach Definition der
  Signumfunktion, dass $\bar\Lambda_0=\gradnc\ucr/|\gradnc\ucr|$ eindeutig
  bestimmt ist auf $T$.
  Im Allgemeinen ist $\bar\Lambda_0$ allerdings nicht eindeutig bestimmbar. 
  Betrachten wir zum Beispiel $f\equiv 0$ in \Cref{prob:discreteProblem} mit
  eindeutiger Lösung $\ucr\equiv 0$ fast überall in $\Omega$. 
  Dann erfüllt nach der diskreten Helmholtz Zerlegung \cite[S. 193, Theorem
  3.32]{Car09b} die Wahl $\bar\Lambda_0\coloneqq \Curl v_\C$ für ein beliebiges
  $v_\C\in S^1(\Tcal)$ mit $|\Curl v_\C|\leq 1$ die Charakterisierung
  \textit{(ii)} aus \Cref{thm:discProbCharacterizationOfDiscreteSolutions}.
\end{remark}


\section{Verfeinerungsindikator und garantierte Energieschranken}

Professor Carstensen stellte für die numerischen Untersuchungen eine Aussage
über eine garantierte untere Energieschranke und einen Verfeinerungsindikator
zur adaptiven Netzverfeinerung zur Verfügung, die wir in diesem Abschnitt
aufführen wollen.

\begin{theorem}[Garantierte untere Energieschranke]
  \label{thm:gleb}
  Sei $\Omega$ konvex, $f\in H^1_0(\Omega)$ das Eingangssignal für
  \Cref{prob:continuousProblem} mit Lösung $u\in H^1_0(\Omega)$ sowie für
  \Cref{prob:discreteProblem} mit Lösung $\ucr\in \CR^1_0(\Omega)$.
  Dann gilt
  \begin{align*}
    \Enc(\ucr)+\frac{\alpha}{2}\Vert u-\ucr\Vert^2
    -\frac{\kappa_\CR}{\alpha}\Vert
    h_\Tcal(f-\alpha\ucr)\Vert \Vert\nabla f\Vert\leq E(u).
  \end{align*}
  Dabei ist die Konstante $\kappa_\CR\coloneqq\sqrt{1/48+1/j_{1,1}^2}$ mit der
  kleinsten positiven Nullstelle $j_{1,1}$ der Bessel-Funktion erster Art.
  Insbesondere gilt dann für 
  \begin{align}
    \label{eq:gleb}
    \Egleb 
    \coloneqq 
    \Enc(\ucr) - \frac{\kappa_\CR}{\alpha}\Vert h_\Tcal(f-\alpha\ucr)\Vert
    \Vert \nabla f\Vert,
  \end{align}
    dass $\Enc(\ucr)\geq \Egleb$ und $E(u)\geq \Egleb$.
\end{theorem}

\begin{definition}[Verfeinerungsindikator]
  \label{def:refinementIndicator}
  Für $d\in\mathbb{N}$ (in dieser Arbeit stets $d=2$) und $0<\gamma\leq 1$
  definieren wir für alle $T\in\Tcal$ und $\ucr\in\CR^1_0(\Tcal)$ die
  Funktionen
  \begin{align*}
    \eta_\text{V}(T)
    &\coloneqq
    |T|^{2/d}\Vert f-\alpha \ucr\Vert^2_{L^2(T)}\quad\text{und }\\
    \eta_\text{J}(T)
    &\coloneqq
    |T|^{\gamma/d}\sum_{F\in\Ecal(T)}\left\Vert [\ucr]_F\right\Vert_{L^1(F)}.
  \end{align*} 
  Damit definieren wir den Verfeinerungsindikator
  $\eta\coloneqq\sum_{T\in\Tcal}\eta(T)$, wobei
  \begin{align} \label{eq:refinementIndicator} 
    \eta (T)
    \coloneqq
    \eta_\text{V}(T) + \eta_\text{J}(T)\quad\text{für alle } T\in\Tcal.
  \end{align} 
\end{definition}

Darüber hinaus können wir eine garantierte obere Energieschranke 
formulieren.
Dabei nutzen wir den Operator $J_1:\CR^1_0(\Tcal)\to P_1(\Tcal)\cap
C_0(\Omega)$ aus \cite[Section 4]{CH18}, wobei $J_1\vcr$ für eine Funktion
$\vcr\in\CR^1_0(\Tcal)$ in allen Innenknoten $z\in\Ncal(\Omega)$ definiert ist
durch
\begin{align}
  \label{eq:enrichmentOperator}
  J_1\vcr(z)\coloneqq |\Tcal(z)|^{-1}\sum_{T\in\Tcal(z)}\vcr|_T(z).
\end{align}
%Dieser erfüllt nach \cite[Section 4]{CH18} für eine vom Innenwinkel (?)
%abhängige Konstante $c>0$, dass
%\begin{align*}
%  \Vert h_\Tcal^{-1}(1-J_1)\vcr\Vert\leq c\Vert\gradnc \vcr\Vert.
%\end{align*}
Da für die Lösung $u$ von \Cref{prob:continuousProblem} und die Lösung
$\ucr$ von \Cref{prob:discreteProblem} gilt 
\begin{align}
  \label{eq:gueb}
  E(u)\leq E(J_1\ucr)=\Enc(J_1\ucr),
\end{align}
wählen wir $\Enc(J_1\ucr)$ als garantierte obere Energieschranke.
