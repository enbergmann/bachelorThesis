\section{Formulierung}
\label{sec:discreteProblemFormulation}
Für den Rest dieser Arbeit sei $\Omega\subset\Rbb^2$.

Bevor wir \Cref{prob:continuousProblem} diskretisieren, merken wir an,
dass $\CR^1(\Tcal)\subset\BV(\Omega)$, da
\begin{align*}
  |\vcr|_{\BV(\Omega)} 
  = 
  \Vert \gradnc \vcr\Vert_{L^1(\Omega)} 
  + \sum_{F\in\Ecal(\Omega)}\Vert[\vcr]_F\Vert_{L^1(F)}
  \quad\text{für alle }\vcr\in\CR^1(\Tcal).
\end{align*} 
Dies wird für $|\Tcal|=2$ zum Beispiel von \cites[S. 404, Example
10.2.1]{ABM14}[S. 301, Proposition 10.1]{Bar15} impliziert und kann
analog für beliebige reguläre Triangulierungen von $\Omega$ bewiesen
werden.
Damit gilt dann insbesondere
\begin{align*}
  |\vcr|_{\BV(\Omega)} +\Vert\vcr\Vert_{L^1(\partial\Omega)} 
  = \Vert \gradnc \vcr\Vert_{L^1(\Omega)} +
  \sum_{F\in\Ecal}\Vert[\vcr]_F\Vert_{L^1(F)}.
\end{align*}

Um eine nichtkonforme Formulierung von \Cref{prob:continuousProblem} zu 
erhalten, ersetzen wir die Terme 
$|\bullet|_{\BV(\Omega)} +\Vert\bullet\Vert_{L^1(\partial\Omega)}$ des
Funktionals $E$ durch 
$\Vert \gradnc \bullet\Vert_{L^1(\Omega)}$, das heißt wir vernachlässigen
bei der nichtkonformen Formulierung die Terme
$\sum_{F\in\Ecal}\Vert[\bullet]_F\Vert_{L^1(F)}$.

Somit erhalten wir das folgende Minimierungsproblem für den Parameter
$\alpha\in\Rbb_+$ und die rechte Seite $f\in
L^2(\Omega)$.

\begin{problem}\label{prob:discreteProblem}
  Finde $\ucr\in \CR^1_0(\Tcal)$,
  sodass $\ucr$ das Funktional
  \begin{align}\label{eq:discreteProblem}
    \Enc(\vcr)\coloneqq \frac{\alpha}{2}\Vert \vcr\Vert^2
    +\Vert \gradnc\vcr\Vert_{L^1(\Omega)}-\int_\Omega f\vcr\dx
  \end{align}
  unter allen $\vcr\in \CR^1_0(\Tcal)$ minimiert.
\end{problem}

\section{Charakterisierung und Existenz eines eindeutigen Minimierers}

In diesen Abschnitt führen wir die Argumente in \cite[S. 313]{Bar15}, angepasst
für unsere Formulierung in \Cref{prob:discreteProblem}, detailiert aus. 

Bevor wir die Existenz und Eindeutigkeit einer Lösung von
\Cref{prob:discreteProblem} sowie äqui\-va\-lente Charakterisierungen für diese 
beweisen können, leiten wir ein zu \Cref{prob:discreteProblem} äquivalentes
Sattelpunktsproblem her.

Dafür definieren wir zunächst die konvexe Menge 
\begin{align*}
  K
  \coloneqq 
  \left\{\Lambda\in L^\infty\left(\Omega;\Rbb^2\right)
  \,\middle|\,|\Lambda(\bullet)| \leq 1 \text{ fast überall in }\Omega\right\}
\end{align*}
und das dazugehörige Indikatorfunktional
$I_K:L^\infty(\Omega;\Rbb^2)\to\Rbb\cup\{\infty\}$, das für $\Lambda\in
L^\infty(\Omega;\Rbb^2)$ gegeben ist durch
\begin{align*}
  I_K(\Lambda)
  &\coloneqq
  \begin{cases}
    \infty, & \text{falls } \Lambda\notin K,\\
    0,       & \text{falls } \Lambda\in K
  \end{cases}
\end{align*} 
und aufgrund der Konvexität von $K$ konvex ist.
Für $\vcr\in\CR^1_0(\Tcal)$ 
und $\Lambda\in P_0\left(\Tcal;\Rbb^2\right)\subset L^\infty(\Omega;\Rbb^2)$ können
wir damit das Funktional
$L:\CR^1_0(\Tcal)\times  P_0\left(\Tcal;\Rbb^2\right)\to [-\infty,\infty)$ 
definieren durch
\begin{align}\label{eq:discreteProblemLagrangeFunctional}
  L(\vcr,\Lambda) \coloneqq \int_\Omega\Lambda\cdot\gradnc\vcr\dx +
  \frac{\alpha}{2}\Vert \vcr\Vert^2 -\int_\Omega f\vcr\dx
  - I_K(\Lambda).
\end{align}

Dabei gilt $L(\vcr,\Lambda)>-\infty$ genau dann, wenn $\Lambda\in K$. 
Außerdem gilt mit der Cauchy-Schwarzschen Ungleichung für beliebige
$\Lambda\in P_0\left(\Tcal;\Rbb^2\right)\cap K$, dass
\begin{align*}
  \int_\Omega \Lambda\cdot\gradnc\vcr\dx
  \leq \int_\Omega |\Lambda\cdot\gradnc\vcr|\dx
  &\leq \int_\Omega |\Lambda||\gradnc\vcr|\dx\\
  &\leq \int_\Omega 1|\gradnc\vcr|\dx\
  = \Vert\gradnc\vcr\Vert_{L^1(\Omega)}.
\end{align*}
Insgesamt folgt
\begin{align*}
  \sup_{\Lambda\in P_0\left(\Tcal;\Rbb^2\right)}L(\vcr,\Lambda)=
  \sup_{\Lambda\in P_0\left(\Tcal;\Rbb^2\right)\cap K}L(\vcr,\Lambda)
  \leq \Enc(\vcr).
\end{align*}

Weiterhin gilt für alle $\vcr\in\CR^1_0(\Tcal)$, wenn wir
$\Lambda\in P_0\left(\Tcal;\Rbb^2\right)\cap K$ elementweise auf allen $T\in\Tcal$
definieren durch $\Lambda(x)\in\sign\left(\gradnc\vcr(x)\right)$ für alle
$x\in \interior(T)$, dass $L(\vcr,\Lambda)=\Enc(\vcr)$ und deshalb auch
$\Enc(\vcr)\leq\sup_{\Lambda\in P_0\left(\Tcal;\Rbb^2\right)}L(\vcr,\Lambda)$.

Insgesamt können wir festhalten, dass für alle $\vcr\in\CR^1_0(\Tcal)$ gilt
\begin{equation}
  \label{eq:discreteEnergySaddlefunctionalEquality}
  \Enc(\vcr)
  =\sup_{\Lambda\in P_0\left(\Tcal;\Rbb^2\right)\cap K}L(\vcr,\Lambda)
  =\sup_{\Lambda\in P_0\left(\Tcal;\Rbb^2\right)}L(\vcr,\Lambda).
\end{equation}

Insbesondere ist damit die Funktion $\ucr\in\CR^1_0(\Tcal)$ 
aus der Lösung $\left( \ucr,\Lambda \right)\in\CR^1_0(\Tcal)\times
 P_0\left(\Tcal;\Rbb^2\right)$ des folgenden Sattelpunktsproblems Lösung von
\Cref{prob:discreteProblem}.
\begin{problem}\label{prob:discreteSaddlepointProblem}
  Finde $\left( \ucr,\Lambda \right)\in\CR^1_0(\Tcal)\times
  P_0\left(\Tcal;\Rbb^2\right)$,
  sodass
  \begin{align*}
    L(\ucr,\Lambda) 
    = 
    \inf_{\vcr\in\CR^1_0(\Tcal)}\sup_{\Lambda\in P_0\left(\Tcal;\Rbb^2\right)}
    L(\vcr,\Lambda).
  \end{align*}
\end{problem}

Nachdem diese Vorbereitungen abgeschlossen sind, können wir nun beweisen,
dass \Cref{prob:discreteProblem} eine eindeutige Lösung besitzt und
äquivalente Charakterisierungen für diese zeigen. Der Inhalt des folgenden
Theorems wurde von Professor Carstensen formuliert.
\begin{theorem}
  \label{thm:discProbCharacterizationOfDiscreteSolutions}
  Es existiert eine eindeutige Lösung $\ucr\in\CR^1_0(\Tcal)$ von
  \Cref{prob:discreteProblem}.

  Außerdem sind die folgenden drei Aussagen
  für eine Funktion $\ucr\in\CR^1_0(\Tcal)$ äquivalent.
  \begin{itemize}
    \item[(i)] \Cref{prob:discreteProblem} wird von $\ucr$ gelöst.
    \item[(ii)] Es existiert ein
      $\bar\Lambda\in P_0\left(\Tcal;\Rbb^2\right)$ mit
      $\left|\bar\Lambda(\bullet)\right|\leq 1$
      fast überall in $\Omega$, sodass
      \begin{equation}
        \label{eq:discreteMultiplierScalerProductEquality}
        \bar\Lambda(\bullet)\cdot\gradnc\ucr(\bullet)
        =
        |\gradnc\ucr(\bullet)| \quad\text{fast überall in } \Omega 
      \end{equation}
      und
      \begin{equation}
        \label{eq:discreteMultiplierL2Equality}
        \left(\bar\Lambda,\gradnc\vcr\right)
        = \left(f-\alpha\ucr,
        \vcr\right)
        \quad\text{für alle } \vcr\in\CR^1_0(\Tcal).
      \end{equation}
    \item[(iii)] Für alle $\vcr\in\CR^1_0(\Tcal)$ gilt
      \begin{equation}
        \label{eq:discreteVariationalInequality}
        (f-\alpha\ucr,\vcr-\ucr)\leq
        \Vert\gradnc\vcr\Vert_{L^1(\Omega)} -
        \Vert\gradnc\ucr\Vert_{L^1(\Omega)}.
      \end{equation}
  \end{itemize}
\end{theorem}

\begin{proof} 
  Mit analogen Abschätzungen wie in Ungleichung \eqref{eq:contProbBddFromBelow}
  erhalten wir für das Funktional $\Enc$ aus \Cref{prob:discreteProblem} 
  für alle $\vcr\in\CR^1_0(\Tcal)\subset L^2(\Omega)$ die Abschätzung 
  \begin{equation*}
    \Enc(\vcr) \geq -\frac{1}{\alpha}\Vert f\Vert^2.
  \end{equation*}
  Somit ist $\Enc$ nach unten beschränkt und es existiert eine infimierende
  Folge $(v_n)_{n\in\Nbb} \subset \CR^1_0(\Tcal)$ von $\Enc$. 
  Aufgrund der Form von $\Enc$ ist diese Folge beschränkt bezüglich der Norm
  $\Vert\bullet\Vert$. 

  Der endlichdimensionale Raum $\CR^1_0(\Tcal)$ ist, ausgestattet mit der Norm
  $\Vert\bullet\Vert$, ein Banachraum und damit reflexiv. 
  Demnach existiert eine in $\CR^1_0(\Tcal)$ schwach konvergente Teilfolge von
  $(v_n)_{n\in\Nbb}$.
  Da $\CR^1_0(\Tcal)$ endlichdimensional ist, konvergiert diese sogar stark
  in $L^2(\Omega)$. 
  Weil $\CR^1_0(\Tcal)$ ein Banachraum und damit abgeschlossen bezüglich der
  Konvergenz in $\Vert\bullet\Vert$ ist, gilt für den Grenzwert $\ucr$ dieser
  Teilfolge, dass $\ucr\in\CR^1_0(\Tcal)$.

  Außerdem ist $\Enc$ stetig bezüglich der Konvergenz in $L^2(\Omega)$,
  was impliziert, dass $\ucr$ Minimierer von $\Enc$ in $\CR^1_0(\Tcal)$ sein
  muss. \todo{noch erklären, warum stetig oder reicht das so? S. 17 im Ausdruck}

  Die Lösung $\ucr\in \CR^1_0(\Tcal)$ ist eindeutig, da $\Enc$ strikt konvex
  ist.

  Nachdem wir die Existenz eines eindeutigen Minimierers von
  \Cref{prob:discreteProblem} bewiesen haben, zeigen wir nun die äquivalenten
  Charakterisierung für diesen.

  Für den Rest dieses Beweises sei $\ucr\in\CR^1_0(\Tcal)$, falls
  nicht anderes angegeben, eine beliebige Funktion in $\CR^1_0(\Tcal)$, das
  heißt insbesondere nicht unbedingt Lösung von \Cref{prob:discreteProblem}.

  \textit{(i) $\Rightarrow$ (ii).}
  Sei $\ucr$ Lösung von \Cref{prob:discreteProblem}.

  Durch \Cref{eq:discreteEnergySaddlefunctionalEquality} wissen wir bereits,
  dass $\Enc(\vcr) = \sup_{\Lambda\in
  P_0\left(\Tcal;\Rbb^2\right)}L(\vcr,\Lambda)$ für alle
  $\vcr\in\CR^1_0(\Tcal)$.
  Somit existiert für $\ucr$ ein $\bar\Lambda\in
  P_0\left(\Tcal;\Rbb^2\right)\cap K$ mit $\Enc(\ucr) =
  L\left(\ucr,\bar\Lambda\right)$. 
  Da außerdem $\ucr$ \Cref{prob:discreteProblem} löst, folgt insgesamt
  \begin{align}
    \label{eq:lagrangianEqualsInfSup}
    L\left(\ucr,\bar\Lambda\right)
    &=
    \Enc(\ucr)
    =
    \inf_{\vcr\in\CR^1_0(\Tcal)}\Enc(\vcr)
    =
    \inf_{\vcr\in\CR^1_0(\Tcal)}\sup_{\Lambda\in P_0\left(\Tcal;\Rbb^2\right)} 
    L(\vcr,\Lambda).
  \end{align}

  Weiterhin gilt nach \cite[S. 379, Lemma 36.1]{Roc70}, dass
  \begin{align*}
    \inf_{\vcr\in\CR^1_0(\Tcal)}\sup_{\Lambda\in P_0\left(\Tcal;\Rbb^2\right)} 
    L(\vcr,\Lambda)
    \geq 
    \sup_{\Lambda\in P_0\left(\Tcal;\Rbb^2\right)} \inf_{\vcr\in\CR^1_0(\Tcal)} 
    L(\vcr,\Lambda).
  \end{align*}
  Zusammen mit \Cref{eq:lagrangianEqualsInfSup} impliziert das
  \begin{align*}
    \inf_{\vcr\in\CR^1_0(\Tcal)}\sup_{\Lambda\in P_0\left(\Tcal;\Rbb^2\right)} 
    L(\vcr,\Lambda)
    =
    L\left(\ucr,\bar\Lambda\right)
    =
    \sup_{\Lambda\in P_0\left(\Tcal;\Rbb^2\right)} \inf_{\vcr\in\CR^1_0(\Tcal)} 
    L(\vcr,\Lambda).
  \end{align*}
  Somit ist $\left(\ucr,\bar\Lambda\right)\in 
  \CR^1_0(\Tcal)\times ( P_0\left(\Tcal;\Rbb^2\right)\cap K)$ nach \cite[S. 380,
  Lemma 36.2]{Roc70} Sattelpunkt 
  von $L$ bezüglich der Maximierung über $ P_0\left(\Tcal;\Rbb^2\right)$ und
  der Minimierung über $\CR^1_0(\Tcal)$. 
  Das bedeutet nach \cite[380]{Roc70} insbesondere, dass $\ucr$ Minimierer von 
  $L(\bullet, \bar\Lambda)$ in $\CR^1_0(\Tcal)$ ist und $\bar\Lambda$
  Maximierer von $L(\ucr,\bullet)$ über $ P_0\left(\Tcal;\Rbb^2\right)$.
  Mit dieser Erkenntnis können wir nun die entsprechenden
  Optimalitätsbedingungen diskutieren.

  Zunächst bemerken wir, dass $L(\ucr,\bullet):P_0\left(\Tcal;\Rbb^2\right)\to
  [-\infty,\infty)$ konkav ist.
  Da $\bar\Lambda$ Maximierer von $L(\ucr,\bullet)$ in 
  $ P_0\left(\Tcal;\Rbb^2\right)$ ist, wird das konvexe Funktional
  $-L(\ucr,\bullet):P_0\left(\Tcal;\Rbb^2\right)\to
  (-\infty,\infty]$ von $\bar\Lambda$ in $ P_0\left(\Tcal;\Rbb^2\right)$
  minimiert.
  Nach den Theoremen \ref{thm:extremalprinciple},
  \ref{thm:subdifferentialSumRule} und \ref{thm:subdiffGateaux} gilt somit
  \begin{align*}
    0\in \partial \left(-L(\ucr,\bullet)\right)\left(\bar\Lambda\right) 
    =
    \{-(\gradnc\ucr,\bullet)\}+\partial I_K
    \left(\bar\Lambda\right).
  \end{align*}
  Äquivalent zu dieser Aussage ist, dass
  $(\gradnc\ucr,\bullet)\in \partial
  I_K \left(\bar\Lambda\right)$. 
  Da $\bar\Lambda\in K$, folgt mit \Cref{def:subdifferential},
  dass für alle $\Lambda\in  P_0\left(\Tcal;\Rbb^2\right)$ gilt
  \begin{align*}
    (\gradnc\ucr,\Lambda-\bar\Lambda) 
    \leq 
    I_K (\Lambda) - I_K (\bar\Lambda)
    =
    I_K (\Lambda).
  \end{align*}
  Falls $\Lambda\in  P_0\left(\Tcal;\Rbb^2\right)\cap K$, folgt insbesondere
  \begin{align}
    \label{eq:scalarProductInequDiscreteProof}
    \left(\gradnc\ucr,\Lambda-\bar\Lambda\right) 
    &\leq 
    0,\quad\text{also }\notag\\ 
    \left(\gradnc\ucr,\Lambda\right)
    &\leq
    \left(\gradnc\ucr,\bar\Lambda\right).
  \end{align}
  Sei nun $\Lambda\in P_0\left(\Tcal;\Rbb^2\right)\cap K$ elementweise auf
  allen $T\in\Tcal$ durch $\Lambda(x)\in\sign\left(\gradnc\ucr(x)\right)$
  definiert für alle $x\in\interior(T)$.
  Mit dieser Wahl von $\Lambda$, Ungleichung
  \eqref{eq:scalarProductInequDiscreteProof}, der Cauchy\--Schwarz\-schen
  Ungleichung und $\bar\Lambda\in K$ erhalten wir die Abschätzung
  \begin{align}
    \label{eq:sumOverAllTrianglesDualVariable}
    \int_\Omega|\gradnc\ucr|\dx
    &=
    \int_\Omega\gradnc\ucr\cdot\Lambda\dx
    =
    (\gradnc\ucr,\Lambda)\notag\\
    &\leq 
    \left(\gradnc\ucr,\bar\Lambda\right)
    =
    \int_\Omega\gradnc\ucr\cdot\bar\Lambda\dx \notag\\
    &\leq 
    \int_\Omega|\gradnc\ucr|\,\left|\bar\Lambda\right|\dx
    \leq
    \int_\Omega|\gradnc\ucr|\dx,\quad\text{das heißt }\notag\\
    \int_\Omega|\gradnc\ucr|\dx 
    &= 
    \int_\Omega\gradnc\ucr\cdot\bar\Lambda\dx
    \quad\text{beziehungsweise }\notag\\
    \sum_{T\in\Tcal}|T|\,\big|(\gradnc\ucr)\!|_T\big|
    &=
    \sum_{T\in\Tcal}|T|\,\left(\gradnc\ucr\cdot \bar\Lambda\right)\!\!|_T.
  \end{align}
  Außerdem gilt für alle $T\in\Tcal$ mit der Cauchy-Schwarzschen Ungleichung
  und $\bar\Lambda\in K$, dass 
  \begin{align*}
  \left(\gradnc\ucr\cdot \bar\Lambda\right)\!\!|_T
  \leq
  \big|(\gradnc\ucr)\!|_{T}\big|\,\left|\bar\Lambda\!|_T\right|
  \leq
  \big|(\gradnc\ucr)\!|_{T}\big|.
  \end{align*}
  Mit \Cref{eq:sumOverAllTrianglesDualVariable} folgt daraus für alle
  $T\in\Tcal$, dass $\left(\gradnc\ucr\cdot
  \bar\Lambda\right)\!\!|_T=\big|(\gradnc\ucr)\!|_T\big|$, das heißt fast
  überall in $\Omega$ gilt $\bar\Lambda(\bullet)\cdot\gradnc\ucr(\bullet)
  =|\gradnc\ucr(\bullet)|$. 
  Damit ist \Cref{eq:discreteMultiplierScalerProductEquality} gezeigt.

  Als Nächstes betrachten wir das reellwertige Funktional
  $L\left(\bullet,\bar\Lambda\right):\CR^1_0(\Tcal)\to\Rbb$.
  Es ist Fr\'echet-differenzierbar mit
  \begin{align*}
    dL(\bullet,\bar\Lambda)(\ucr;\vcr)=
    \int_\Omega\bar\Lambda\cdot \gradnc\vcr\dx
    +\alpha (\ucr,\vcr) - \int_\Omega f\vcr\dx
  \end{align*}
  für alle $\vcr\in\CR^1_0(\Tcal)$.
  Da $\ucr$ Minimierer von  $L\left(\bullet, \bar\Lambda\right)$ in
  $\CR^1_0(\Tcal)$ ist, gilt nach
  \Cref{thm:necessaryConditionFreeLocalExtrema}, dass $0 =
  dL(\bullet,\bar\Lambda)(\ucr;\vcr)$ für alle $\vcr\in\CR^1_0(\Tcal)$.
  Diese Bedingung ist für alle $\vcr\in\CR^1_0(\Tcal)$ äquivalent zu
    $\left(\bar\Lambda,\gradnc\vcr\right) = (f-\alpha \ucr,\vcr)$.
  Somit ist \Cref{eq:discreteMultiplierL2Equality} gezeigt.

  \textit{(ii) $\Rightarrow$ (iii).}
  Die Funktion $\bar\Lambda\in P_0(\Tcal;\Rbb^2)$ erfülle
  $\left|\bar\Lambda(\bullet)\right|\leq 1$ fast überall in $\Omega$ sowie
  die Gleichungen \eqref{eq:discreteMultiplierScalerProductEquality} und 
  \eqref{eq:discreteMultiplierL2Equality}.

  Dann gilt für alle $\vcr\in\CR^1_0(\Tcal)$ mit
  \Cref{eq:discreteMultiplierL2Equality},
  \Cref{eq:discreteMultiplierScalerProductEquality}, der Cauchy-Schwarzschen
  Ungleichung und $\left|\bar\Lambda(\bullet)\right|\leq 1$ fast überall in
  $\Omega$ , dass
  \begin{align*}
    (f-\alpha\ucr,\vcr-\ucr) 
    &= 
    (f-\alpha\ucr,\vcr) 
    - (f-\alpha\ucr,\ucr) \\
    &=
    \left(\bar\Lambda,\gradnc\vcr\right)
    - \left(\bar\Lambda,\gradnc\ucr\right)\\
    &=
    \int_\Omega\bar\Lambda\cdot\gradnc\vcr\dx
    - \int_\Omega\bar\Lambda\cdot\gradnc\ucr\dx\\
    &=
    \int_\Omega\bar\Lambda\cdot\gradnc\vcr\dx
    - \int_\Omega|\gradnc\ucr|\dx\\
    &\leq 
    \int_\Omega\left|\bar\Lambda\right|\,|\gradnc\vcr|\dx
    - \int_\Omega|\gradnc\ucr|\dx\\
    &\leq 
    \int_\Omega|\gradnc\vcr|\dx
    - \int_\Omega|\gradnc\ucr|\dx\\
    &=
    \Vert\gradnc\vcr\Vert_{L^1(\Omega)}
    -\Vert\gradnc\ucr\Vert_{L^1(\Omega)}.
  \end{align*}
  Somit ist Ungleichung \eqref{eq:discreteVariationalInequality} gezeigt.

  \textit{(iii) $\Rightarrow$ (i)}.
  Es gelte Ungleichung \eqref{eq:discreteVariationalInequality} für alle
  $\vcr\in\CR^1_0(\Tcal)$, also
  \begin{align*}
    \left(f-\alpha\ucr,\vcr-\ucr\right) 
    \leq
    \left\Vert\gradnc\vcr\right\Vert_{L^1(\Omega)}
    -\Vert\gradnc\ucr\Vert_{L^1(\Omega)}.
  \end{align*}


  Wir haben in diesen Beweis bereits gezeigt, dass stets eine eindeutige Lösung
  $\tilde{u}_\CR\in\CR^1_0(\Tcal)$ von \Cref{prob:discreteProblem} existiert
  und außerdem, dass für diese
  für alle $\vcr\in\CR^1_0(\Tcal)$ gilt
  \begin{align*}
    \left(f-\alpha\tilde{u}_\CR,\vcr-\tilde{u}_\CR\right) 
    \leq
    \left\Vert\gradnc\vcr\right\Vert_{L^1(\Omega)}
    -\Vert\gradnc\tilde{u}_\CR\Vert_{L^1(\Omega)}.
  \end{align*}

  Um zu beweisen, dass $\ucr$ \Cref{prob:discreteProblem} löst, genügt es
  aufgrund der Eindeutigkeit der Lösung $\tilde{u}_\CR$ von
  \Cref{prob:discreteProblem} zu zeigen, dass $\ucr=\tilde{u}_\CR$ in
  $\CR^1_0(\Tcal)$.

  Es gilt
  \begin{align*}
    \left(f-\alpha\ucr,\tilde{u}_\CR-\ucr\right) 
    &\leq
    \left\Vert\gradnc\tilde{u}_\CR\right\Vert_{L^1(\Omega)}
    -\Vert\gradnc\ucr\Vert_{L^1(\Omega)}\quad\text{und }\\
    \left(f-\alpha\tilde{u}_\CR,\ucr-\tilde{u}_\CR\right) 
    &\leq
    \left\Vert\gradnc\ucr\right\Vert_{L^1(\Omega)}
    -\Vert\gradnc\tilde{u}_\CR\Vert_{L^1(\Omega)}. 
  \end{align*}
  Die Addition dieser Ungleichungen
  liefert die Ungleichung
  \begin{align*}
    \left(-\alpha\ucr,\tilde{u}_\CR-\ucr\right) 
    + \left(-\alpha\tilde{u}_\CR,\ucr-\tilde{u}_\CR\right) 
    \leq
    0,
  \end{align*}
  welche äquivalent ist zu
  \begin{align*}
    \alpha\left\Vert\tilde{u}_\CR-\ucr\right\Vert^2
    \leq
    0.
  \end{align*}
  Da $\alpha>0$, impliziert das
  $\left\Vert\tilde{u}_\CR-\ucr\right\Vert^2=0$, also
  $\tilde{u}_\CR=\ucr$ in $\CR^1_0(\Tcal)$.
\end{proof}

Zum Schluss dieses Abschnitts wollen wir noch zwei Bermerkungen von Professor
Carstensen erwähnen und kurz deren Gültigkeit begründen.

Die erste ist in eine äquivalente Charakterisierung der dualen Variable
$\bar\Lambda\in P_0\left(\Tcal;\Rbb^2\right)$ aus
\Cref{thm:discProbCharacterizationOfDiscreteSolutions} zur diskreten Lösung
$\ucr\in\CR^1_0(\Tcal)$ von \Cref{prob:discreteProblem}.

\begin{remark}
  Das $\bar\Lambda\in P_0\left(\Tcal;\Rbb^2\right)$ fast überall in $\Omega$
  \Cref{eq:discreteMultiplierScalerProductEquality} und
  $|\bar\Lambda(\bullet)|\leq 1$ erfüllt, ist äquivalent zu der Bedingung
  $\bar\Lambda(x)\in\sign(\gradnc \ucr(x))$ für alle $x\in\interior(T)$ für
  alle $T\in\Tcal$.   
\end{remark}

\begin{proof}
  Dass die genannte Bedingung hinreichend ist, folgt direkt aus der Definition
  der Signumfunktion.

  Ihre Notwendigkeit folgt aus der folgenden Beobachtung.
  Da $\left|\bar\Lambda(\bullet)\right|\leq 1$ fast überall in $\Omega$, ist
  \Cref{eq:discreteMultiplierScalerProductEquality} eine Cauchy-Schwarzsche
  Ungleichung, bei der sogar Gleichheit gilt. 
  Dies ist genau dann der Fall, wenn $\bar\Lambda(\bullet)$ und
  $\gradnc\ucr(\bullet)$ fast überall in $\Omega$ linear abhängig sind.
\end{proof}
 
Daraus können wir folgern, unter welchen Umständen die duale Variable
$\bar\Lambda$ auf einem Dreieck $T\in\Tcal$ eindeutig bestimmt ist.

\begin{remark}
  Falls $\gradnc\ucr\neq 0$ auf $T\in\Tcal$, gilt nach Definition der
  Signumfunktion, dass $\bar\Lambda=\gradnc\ucr/|\gradnc\ucr|$ eindeutig
  bestimmt ist auf $T$.

  Im Allgemeinen ist $\bar\Lambda$ nicht eindeutig bestimmbar. 
  Betrachten wir zum Beispiel $f\equiv 0$ in \Cref{prob:discreteProblem} mit
  eindeutiger Lösung $\ucr\equiv 0$ fast überall in $\Omega$. 
  Dann erfüllt nach der diskreten Helmholtz Zerlegung \cite[S. 193, Theorem
  3.32]{Car09b} die Wahl $\bar\Lambda\coloneqq \Curl(v_\C)$ für ein beliebiges
  $v_\C\in S^1(\Tcal)$ mit $|\Curl(v_\C)|\leq 1$ die Charakterisierung
  \textit{(ii)} aus \Cref{thm:discProbCharacterizationOfDiscreteSolutions}.
\end{remark}


\section{Kontrolle des Abstandes zwischen diskreter und kontinuierlicher
Lösung}
Wir möchten in diesen Abschnitt zwei Abschätzungen aufführen, mit denen
möglicherweise der $L^2$-Abstand zwischen der Lösung des diskreten Problems
\ref{prob:discreteProblem} und der Lösung des kontinuierlichen Problems
\ref{prob:continuousProblem} kontrolliert werden kann.

Die Gültigkeit dieser Abschätzungen werden wir in \Cref{chap:experiments}
untersuchen.

Zunächst betrachten wir \cite[S. 309, Theorem 10.7]{Bar15}. Diese
Abschätzung kontrolliert den
$L^2$-Fehler zwischen den Minimierern $u_\C\in S^1(\Tcal)$ und
$u\in\BV(\Omega)\cap L^2(\Omega)$ des Funktionals $I$ aus \Cref{eq:rofModel} in
den entsprechenden Räumen. Obwohl wir eine andere Formulierung des ROF-Modells 
betrachten und sich insbesondere das Funktional $\Enc$ aus unserer diskreten,
nichtkonformen Formulierung von $I$ unterscheidet, ähneln sich die 
Probleme möglicherweise genug, um die folgende Rate für unsere Formulierungen
experimentell feststellen zu können.

\begin{theorem}
  \label{thm:errorEstimateCourant}
  Sei $\Omega\subset\Rbb^2$ sternförmig und $g\in L^\infty(\Omega)$.  Seien
  weiterhin $u\in\BV(\Omega)\cap L^2(\Omega)$ und $u_\C\in S^1(\Tcal)$ die
  Minimierer des Funktionals $I$ aus \Cref{eq:rofModel} in den entsprechenden
  Räumen.

  Dann existiert eine Konstante $c\in\Rbb_+$, sodass
  \begin{align*}
    \frac{\alpha}{2}\Vert u-u_\C\Vert^2\leq
    ch^{1/2}.
  \end{align*}
\end{theorem}

Als Nächstes möchten wir beweisen, dass der $L^2$-Abstand zwischen der Lösung
$u$ von \Cref{prob:continuousProblem} und einer beliebigen Funktion
$v\in\BV(\Omega)\cap L^2(\Omega)$ beschränkt werden kann durch die Differenz
der Werte des Funktionals $E$ an den Stellen $u$ und $v$.
Daraus wollen wir anschließend die zweite Möglichkeit folgern,
wie der Abstand in $L^2(\Omega)$ zwischen den Lösungen des kontinuierlichen 
und des diskreten Problems kontrolliert werden kann.

\begin{theorem}
  \label{thm:convexity}
  Sei $u\in\BV(\Omega)\cap L^2(\Omega)$ Lösung von 
  \Cref{prob:continuousProblem}.

  Dann gilt 
  \begin{align*}
    \frac{\alpha}{2}\Vert u-v\Vert^2 \leq E(v)-E(u)\quad
    \text{für alle } v\in\BV(\Omega)\cap L^2(\Omega).
  \end{align*}
\end{theorem}

\begin{proof}
  Wir folgen der Argumentation im Beweis von \cite[S. 309, Lemma 10.2]{Bar15}.
  Da viele der Schritte ähnlich zum Beweis von \Cref{thm:contProbStabAndUniqu}
  sind, präsentieren wir die entsprechenden Argumente verkürzt.

  Wir definieren die konvexen Funktionale
  $F:L^2(\Omega)\to \Rbb\cup\{\infty\}$ und $G:L^2(\Omega)\to \Rbb$, wobei 
  $F$ wie im Beweis von \Cref{thm:contProbStabAndUniqu} definiert ist und $G$
  für alle $v\in L^2(\Omega)$ gegeben ist durch 
  \begin{align*}
    G(v)\coloneqq \frac{\alpha}{2}\Vert v\Vert^2 - \int_\Omega f v\dx.
  \end{align*}
  Es gilt $E = F+G$.

  Die Fr\'echet-Ableitung $G'(u): L^2(\Omega)\to\Rbb$ von $G$ an der Stelle
  $u\in \BV(\Omega)\cap L^2(\Omega)$ ist für alle $v\in L^2(\Omega)$ gegeben
  durch
  \begin{align*}
    dG(u;v) = \alpha (u,v) - \int_\Omega f v\dx 
    = (\alpha u-f ,v).
  \end{align*}

  Das impliziert mit wenigen Rechenschritten
  \begin{align}\label{eq:strongConvexityG}
    dG(u;v-u) +\frac{\alpha}{2}\Vert u-v\Vert^2+G(u) 
    =
    G(v)
    \quad\text{für alle } v\in L^2(\Omega).
  \end{align}

  Da $u$ Minimierer von $E$ ist, erhalten wir mit \Cref{thm:extremalprinciple},
  \Cref{thm:subdifferentialSumRule} und \Cref{thm:subdiffGateaux} die Aussage
  \begin{align*}
    0\in\partial E(u) = \partial F(u)+\{G'(u)\},
  \end{align*}
  woraus folgt 
  \begin{align*}
    -G'(u)\in\partial F(u).
  \end{align*}
  Das ist nach \Cref{def:subdifferential} äquivalent zu
  \begin{align*}
    -dG(u;v-u)\leq F(v)-F(u)\quad\text{für alle }v\in\BV(\Omega)\cap
    L^2(\Omega).
  \end{align*}

  Daraus folgt zusammen mit \Cref{eq:strongConvexityG} für alle $v\in
  \BV(\Omega)\cap L^2(\Omega)$, dass
  \begin{align*}
    \frac{\alpha}{2}\Vert u-v\Vert^2+G(u)-G(v)+F(u)
    = -dG(u;v-u)+F(u)\leq F(v).
  \end{align*}
  

  Da $E=F+G$, folgt daraus die zu zeigende Aussage.
\end{proof}

Zusammen mit den Betrachtungen in \Cref{sec:discreteProblemFormulation} 
erhalten wir damit folgendes Korollar.

\begin{corollary}
  Sei $u\in\BV(\Omega)\cap L^2(\Omega)$ Lösung von
  \Cref{prob:continuousProblem} und $\ucr\in\CR^1_0(\Tcal)$ Lösung von
  \Cref{prob:discreteProblem}.

  Dann gilt
  \begin{align*}
    \frac{\alpha}{2}\Vert u-\ucr\Vert^2\leq
    E(\ucr)-E(u)=\Enc(\ucr)+\sum_{F\in\Ecal}\Vert[\ucr]_F\Vert_{L^1(F)}-E(u).
  \end{align*}

  Insbesondere gilt auch 
  \begin{align*}
    \frac{\alpha}{2}\Vert u-\ucr\Vert^2
    \leq
    \left|\Enc(\ucr)-E(u)\right|+
    \left|\sum_{F\in\Ecal}\Vert[\ucr]_F\Vert_{L^1(F)}\right|.
  \end{align*}
\end{corollary}

\section{Verfeinerungsindikator und garantierte untere Energieschranke}

Professor Carstensen stellte für die numerischen Untersuchungen 
den folgenden Verfeinerungsindikator zur adaptiven Netzverfeinerung und eine 
garantierte untere Energieschranke zur Verfügung.

\begin{definition}[Verfeinerungsindikator]
  \label{def:refinementIndicator}
  Für $d\in\mathbb{N}$ (in dieser Arbeit stets $d=2$) und $0<\gamma\leq 1$
  definieren wir für alle $T\in\Tcal$ und $\ucr\in\CR^1_0(\Tcal)$ die
  Funktionen
  \begin{align*}
    \eta_\text{V}(T)
    &\coloneqq
    |T|^{2/n}\Vert f-\alpha \ucr\Vert^2_{L^2(T)}\quad\text{und }\\
    \eta_\text{J}(T)
    &\coloneqq
    |T|^{\gamma/n}\sum_{F\in\Ecal(T)}\left\Vert [\ucr]_F\right\Vert_{L^1(F)}.
  \end{align*} 
  Damit definieren wir den Verfeinerungsindikator
  $\eta\coloneqq\sum_{T\in\Tcal}\eta(T)$, wobei
  \begin{align} \label{eq:refinementIndicator} 
    \eta (T)
    \coloneqq
    \eta_\text{V}(T) + \eta_\text{J}(T)\quad\text{für alle } T\in\Tcal.
  \end{align} 
\end{definition}

\begin{theorem}
  \label{thm:gleb}
  Sei $\Omega$ konvex, $f\in H^1_0(\Omega)$ das Eingangssignal für
  \Cref{prob:continuousProblem} mit Lösung $u\in H^1_0(\Omega)$ und minimaler
  Energie $E(u)$ sowie für \Cref{prob:discreteProblem} mit Lösung $\ucr\in
  \CR^1_0(\Omega)$ und minimaler Energie $\Enc(\ucr)$.

  Dann gilt
  \begin{align*}
    \Enc(\ucr)+\frac{\alpha}{2}\Vert u-\ucr\Vert^2
    -\frac{\kappa_\CR}{\alpha}\Vert
    h_\Tcal(f-\alpha\ucr)\Vert \Vert\nabla f\Vert\leq E(u).
  \end{align*}
  Dabei ist mit der kleinsten positiven Nullstelle $j_{1,1}$ der
  Bessel-Funktion erster Art die Konstante
  $\kappa_\CR\coloneqq\sqrt{1/48+1/j_{1,1}^2}$.

  Insbesondere gilt dann für 
  \begin{align}
    \label{eq:gleb}
    \Egleb 
    \coloneqq 
    \Enc(\ucr) - \frac{\kappa_\CR}{\alpha}\Vert h_\Tcal(f-\alpha\ucr)\Vert
    \Vert \nabla f\Vert,
  \end{align}
    dass $\Enc(\ucr)\geq \Egleb$ und $E(u)\geq \Egleb$.
\end{theorem}
