\section{Formulierung}
\label{sec:discreteProblemFormulation}
Bevor wir \Cref{prob:continuousProblem} diskretisieren, merken wir an,
dass $\CR^1(\Tcal)\subset\BV(\Omega)$, da
\begin{align*}
  |\vcr|_{\BV(\Omega)} 
  = 
  \Vert \gradnc \vcr\Vert_{L^1(\Omega)} 
  + \sum_{F\in\Ecal(\Omega)}\Vert[\vcr]_F\Vert_{L^1(F)}
  \quad\text{für alle }\vcr\in\CR^1(\Tcal).
\end{align*} 
Dies wird für $|\Tcal|=2$ zum Beispiel von \cites[S. 404, Example
10.2.1]{ABM14}[S. 301, Proposition 10.1]{Bar15} impliziert und kann
analog für beliebige reguläre Triangulierungen von $\Omega$ bewiesen
werden.
Damit gilt dann für alle $\vcr\in\CR^1(\Tcal)$ insbesondere
\begin{align*}
  |\vcr|_{\BV(\Omega)} +\Vert\vcr\Vert_{L^1(\partial\Omega)} 
  = \Vert \gradnc \vcr\Vert_{L^1(\Omega)} +
  \sum_{F\in\Ecal}\Vert[\vcr]_F\Vert_{L^1(F)}.
\end{align*}
Um eine nichtkonforme Formulierung von \Cref{prob:continuousProblem} zu 
erhalten, ersetzen wir die Terme 
$|\bullet|_{\BV(\Omega)} +\Vert\bullet\Vert_{L^1(\partial\Omega)}$ des
Funktionals $E$ durch 
$\Vert \gradnc \bullet\Vert_{L^1(\Omega)}$, das heißt wir vernachlässigen
bei der nichtkonformen Formulierung die Terme
$\sum_{F\in\Ecal}\Vert[\bullet]_F\Vert_{L^1(F)}$.
Somit erhalten wir das folgende Minimierungsproblem für den Parameter
$\alpha\in\Rbb_+$ und die rechte Seite $f\in
L^2(\Omega)$.

\begin{problem}\label{prob:discreteProblem}
  Finde $\ucr\in \CR^1_0(\Tcal)$,
  sodass $\ucr$ das Funktional
  \begin{align}\label{eq:discreteProblem}
    \Enc(\vcr)\coloneqq \frac{\alpha}{2}\Vert \vcr\Vert^2
    +\Vert \gradnc\vcr\Vert_{L^1(\Omega)}-\int_\Omega f\vcr\dx
  \end{align}
  unter allen $\vcr\in \CR^1_0(\Tcal)$ minimiert.
\end{problem}

\section{Charakterisierung und Existenz eines eindeutigen Minimierers}

In diesen Abschnitt führen wir die Argumente in \cite[S. 313]{Bar15}, angepasst
für unsere Formulierung in \Cref{prob:discreteProblem}, detailiert aus. 
Zunächst zeigen wir, dass \Cref{prob:discreteProblem} eine eindeutige Lösung
besitzt. Dafür benötigen wir folgendes Lemma.
\begin{lemma}
  \label{lem:normOfGradNcContiuous}
  Das Funktional $\Enc$ aus \Cref{eq:discreteProblem} ist stetig bezüglich der
  Konvergenz in $L^2(\Omega)$.
\end{lemma}

\begin{proof}
  Die Folge $(v_n)_{n\in\Nbb}\subset\CR^1_0(\Tcal)$ konvergiere
  gegen $\vcr\in\CR^1_0(\Tcal)$ bezüglich der Norm $\Vert\bullet\Vert$.
  Damit ist $(v_n)_{n\in\Nbb}$ insbesondere beschränkt in $L^2(\Omega)$ und es
  gilt mit einer binomischen Formel und der umgekehrten Dreiecksungleichung,
  dass
  \begin{align*}
    \left|\Vert\vcr\Vert^2-\Vert v_n\Vert^2\right|
    &=
    \big|\Vert\vcr\Vert-\Vert v_k\Vert\big|\, 
    \big|\Vert \vcr\Vert+\Vert v_k\Vert\big|\\
    &\leq
    \Vert\vcr- v_k\Vert\, \big|\Vert \vcr\Vert+\Vert v_k\Vert\big|
    \to 0\quad\text{für }n\to\infty.
  \end{align*}
  Außerdem gilt mit der Hölderschen Ungleichung
  \begin{align*}
    \left|\int_\Omega f(\vcr-v_k)\dx\right|
    \leq \Vert f\Vert \Vert\vcr-v_k\Vert\to 0\quad\text{für }n\to\infty.
  \end{align*}
  Schließlich gilt für alle $n\in\Nbb$ und alle $T\in\Tcal$ mit der 
  inversen Ungleichung (cf. \cite[S. 53, Lemma 3.5]{Bar15})
  mit Konstante $c_T\in\Rbb_+$ und der Hölderschen Ungleichung, dass
  \begin{equation*}
    \label{eq:continuityProofTriangleWiseEstimate}
    \Vert\gradnc(\vcr- v_n)\Vert_{L^1(T)}
    \leq
    c_T h_T^{-1}\Vert\vcr- v_n\Vert_{L^1(T)}
    \leq
    c_T h_T^{-1}\sqrt{|T|}\Vert\vcr- v_n\Vert_{L^2(T)}.
  \end{equation*}
  Damit folgt zusammen mit der umgekehrten Dreiecksungleichung
  \begin{align*}
    \left|\Vert\gradnc\vcr\Vert-\Vert \gradnc v_n\Vert\right|
    &\leq 
    \Vert\gradnc(\vcr- v_n)\Vert_{L^1(\Omega)}\\
    &=
    \sum_{T\in\Tcal}\Vert\gradnc(\vcr- v_n)\Vert_{L^1(T)}\\
    &\leq
    \max_{T\in\Tcal}\left(c_T h_T^{-1}\sqrt{|T|}\right)
    \sum_{T\in\Tcal}\Vert\vcr- v_n\Vert_{L^2(T)}\\
    &=
    \max_{T\in\Tcal}\left(c_T h_T^{-1}\sqrt{|T|}\right) \Vert\vcr- v_n\Vert
    \to 0\quad\text{für }n\to\infty.
  \end{align*}
  Somit ist $\Enc$ Summe von drei Termen, die bezüglich der Norm
  $\Vert\bullet\Vert$ folgenstetig sind, und deshalb stetig bezüglich der
  Konvergenz in $L^2(\Omega)$.
\end{proof}

\begin{theorem}
  \label{thm:discreteProblemExistenceUniqueness}
  Es existiert eine eindeutige Lösung $\ucr\in\CR^1_0(\Tcal)$ von
  \Cref{prob:discreteProblem}.
\end{theorem}

\begin{proof}
  Mit analogen Abschätzungen wie in \eqref{eq:contProbBddFromBelow}
  erhalten wir für das Funktional $\Enc$ aus \Cref{prob:discreteProblem} 
  für alle $\vcr\in\CR^1_0(\Tcal)\subset L^2(\Omega)$ die Ungleichung 
  \begin{equation}
    \label{eq:discreteEnergyCoercivity}
    \Enc(\vcr) 
    \geq 
    \frac{\alpha}{4}\Vert \vcr\Vert^2
    +\Vert \gradnc\vcr\Vert_{L^1(\Omega)}
    -\frac{1}{\alpha}\Vert f\Vert^2
    \geq 
    -\frac{1}{\alpha}\Vert f\Vert^2.
  \end{equation}
  Somit ist $\Enc$ nach unten beschränkt und es existiert eine infimierende
  Folge $(v_n)_{n\in\Nbb} \subset \CR^1_0(\Tcal)$ von $\Enc$. 
  Ungleichung \eqref{eq:discreteEnergyCoercivity} impliziert weiterhin, dass
  diese Folge beschränkt bezüglich der Norm $\Vert\bullet\Vert$ sein muss.
  Der endlichdimensionale Raum $\CR^1_0(\Tcal)$ ist, ausgestattet mit der Norm
  $\Vert\bullet\Vert$, ein Banachraum und damit reflexiv. 
  Demnach existiert eine in $\CR^1_0(\Tcal)$ schwach konvergente Teilfolge von
  $(v_n)_{n\in\Nbb}$.
  Da $\CR^1_0(\Tcal)$ endlichdimensional ist, konvergiert diese sogar stark
  in $L^2(\Omega)$. 
  Weil $\CR^1_0(\Tcal)$ ein Banachraum und damit abgeschlossen bezüglich der
  Konvergenz in $\Vert\bullet\Vert$ ist, gilt für den Grenzwert $\ucr$ dieser
  Teilfolge, dass $\ucr\in\CR^1_0(\Tcal)$.
  Nach \Cref{lem:normOfGradNcContiuous} ist $\Enc$ stetig bezüglich der
  Konvergenz in $L^2(\Omega)$, was impliziert, dass $\ucr$ Minimierer von
  $\Enc$ in $\CR^1_0(\Tcal)$ sein muss.   
  Dieser Minimierer $\ucr$ ist eindeutig, da $\Enc$ strikt konvex ist.
\end{proof}

Als nächstes wollen wir äquivalente Charakterisierungen der eindeutigen Lösung
von \Cref{prob:discreteProblem} beweisen, die von Professor Carstensen
formuliert wurden.
Dazu leiten wir zunächst ein zu \Cref{prob:discreteProblem} äquivalentes
Minimaxproblem nach \cite[Section 36]{Roc70} her.
Wir betrachten die konvexe Menge 
\begin{align*}
  K
  \coloneqq 
  \left\{\Lambda\in L^\infty\!\left(\Omega;\Rbb^2\right)
  \,\middle|\,|\Lambda(\bullet)| \leq 1 \text{ fast überall in }\Omega\right\}
\end{align*}
und das dazugehörige Indikatorfunktional
$I_K:L^\infty\!\left(\Omega;\Rbb^2\right)\to\Rbb\cup\{\infty\}$, das für
$\Lambda\in L^\infty\!\left(\Omega;\Rbb^2\right)$ gegeben ist durch
\begin{align*}
  I_K(\Lambda)
  &\coloneqq
  \begin{cases}
    \infty, & \text{falls } \Lambda\notin K,\\
    0,       & \text{falls } \Lambda\in K.
  \end{cases}
\end{align*} 
Aufgrund der Konvexität von $K$ ist $I_K$ konvex.
Für $\vcr\in\CR^1_0(\Tcal)$ und $\Lambda_0\in
P_0\!\left(\Tcal;\Rbb^2\right)\subset L^\infty\!\left(\Omega;\Rbb^2\right)$
können wir damit die Sattelfunktion $L:\CR^1_0(\Tcal)\times
P_0\!\left(\Tcal;\Rbb^2\right)\to [-\infty,\infty)$ nach \cite[Section
33]{Roc70} definieren durch
\begin{align}\label{eq:discreteProblemLagrangeFunctional}
  L(\vcr,\Lambda_0) \coloneqq \int_\Omega\Lambda_0\cdot\gradnc\vcr\dx +
  \frac{\alpha}{2}\Vert \vcr\Vert^2 -\int_\Omega f\vcr\dx
  - I_K(\Lambda_0).
\end{align}
Nun wählen wir $\vcr\in\CR^1_0(\Tcal)$ beliebig. 
Mit der Cauchy-Schwarzschen Ungleichung gilt für alle
$\Lambda_0\in P_0\!\left(\Tcal;\Rbb^2\right)\cap K$, dass
\begin{align*}
  \int_\Omega \Lambda_0\cdot\gradnc\vcr\dx
  \leq 
  \int_\Omega |\Lambda_0||\gradnc\vcr|\dx
  \leq 
  \Vert\gradnc\vcr\Vert_{L^1(\Omega)}.
\end{align*}
Daraus folgt
\begin{align}
  \label{eq:saddlepointLeqEnergy}
  \sup_{\Lambda_0\in P_0\left(\Tcal;\Rbb^2\right)\cap K}L(\vcr,\Lambda_0)
  \leq \Enc(\vcr).
\end{align}
Weiterhin gilt, wenn wir $\Lambda_0\in P_0\!\left(\Tcal;\Rbb^2\right)\cap K$
mit der Signumfunktion aus \Cref{eq:signumFunction} elementweise auf allen
$T\in\Tcal$ definieren durch $\Lambda_0(x)\in\sign\left(\gradnc\vcr(x)\right)$
für alle $x\in \interior(T)$, dass $L(\vcr,\Lambda_0)=\Enc(\vcr)$ und deshalb
auch
\begin{align}
  \label{eq:saddlepointGeqEnergy}
  \Enc(\vcr)
  \leq
  \sup_{\Lambda_0\in P_0\left(\Tcal;\Rbb^2\right)\cap K}L(\vcr,\Lambda_0).
\end{align}
Außerdem ist $L(\vcr,\Lambda_0)>-\infty$ genau dann, wenn $\Lambda_0\in K$.
Damit folgt aus den Ungleichungen \eqref{eq:saddlepointLeqEnergy} und
\eqref{eq:saddlepointGeqEnergy} insgesamt
\begin{equation*}
  \label{eq:discreteEnergySaddlefunctionalEquality}
  \Enc(\vcr)
  =\sup_{\Lambda_0\in P_0\left(\Tcal;\Rbb^2\right)\cap K}L(\vcr,\Lambda_0)
  =\sup_{\Lambda_0\in P_0\left(\Tcal;\Rbb^2\right)}L(\vcr,\Lambda_0).
\end{equation*}
Wenn also das folgende Minimaxproblem
\ref{prob:discreteSaddlepointProblem} eine Lösung $\left(
\tilde{u}_\CR,\bar\Lambda_0 \right)\in\CR^1_0(\Tcal)\times
P_0\!\left(\Tcal;\Rbb^2\right)$ hat, dann löst die Funktion $\tilde{u}_\CR$
\Cref{prob:discreteProblem}.

\begin{problem}\label{prob:discreteSaddlepointProblem}
  Finde $\left( \tilde{u}_\CR,\bar\Lambda_0 \right)\in\CR^1_0(\Tcal)\times
  P_0\!\left(\Tcal;\Rbb^2\right)$,
  sodass
  \begin{align*}
    L(\tilde{u}_\CR,\bar\Lambda_0) 
    = 
    \inf_{\vcr\in\CR^1_0(\Tcal)}\sup_{\Lambda_0\in P_0\left(\Tcal;\Rbb^2\right)}
    L(\vcr,\Lambda_0).
  \end{align*}
\end{problem}

\begin{lemma}
  \label{lem:existenceSaddlepoint}
  Es existiert eine Lösung $\left( \tilde{u}_\CR,\bar\Lambda_0
  \right)\in\CR^1_0(\Tcal)\times \left(P_0\!\left(\Tcal;\Rbb^2\right)\cap
  K\right)$ von \Cref{prob:discreteSaddlepointProblem}.
\end{lemma}

\begin{proof}
  Die Sattelfunktion $L$ aus \Cref{eq:discreteProblemLagrangeFunctional} ist in
  ihrer ersten Komponente eine konvexe, unterhalbstetige, auf $\CR^1_0(\Tcal)$
  reellwertige Funktion und
  in ihrer zweiten Komponente eine konkave, oberhalbstetige, auf
  $P_0\!\left(\Tcal;\Rbb^2\right)\cap K$ reellwertige Funktion.
  Somit ist $L$ in beiden Komponenten abgeschlossen 
  nach \cite[S. 52, 308]{Roc70}.  
  Insgesamt ist $L$ damit eine konvex-konkave, propere und abgeschlossene
  Funktion nach \cite[S. 349, 362 f.]{Roc70}, deren effektiver
  Definitionsbereich nach \cite[362]{Roc70} die Menge $\CR^1_0(\Tcal)\times
  \left( P_0\!\left(\Tcal;\Rbb^2\right)\cap K\right)$ ist.
  Unter Beachtung der Isomorphie von $\CR^1_0(\Tcal)$ zu
  $\Rbb^{|\Ecal(\Omega)|}$ und der Isomorphie von
  $P_0\!\left(\Tcal;\Rbb^2\right)$ zu $\Rbb^{2|\Tcal|}$, folgt aus 
  \cite[S. 397, Theorem 37.6]{Roc70} die Existenz eines Sattelpunkts
  $\left(\tilde{u}_\CR,\bar\Lambda_0\right)\in \CR^1_0(\Tcal)\times \left(
  P_0\!\left(\Tcal;\Rbb^2\right)\cap K\right)$ von $L$ nach \cite[380]{Roc70}.
  Für diesen impliziert \cite[S. 380, Lemma 36.2]{Roc70}, dass 
  \begin{align*}
    \sup_{\Lambda_0\in P_0\left(\Tcal;\Rbb^2\right)}\inf_{\vcr\in\CR^1_0(\Tcal)}
    L(\vcr,\Lambda_0)
    =
    L(\tilde{u}_\CR,\bar\Lambda_0) 
    = 
    \inf_{\vcr\in\CR^1_0(\Tcal)}\sup_{\Lambda_0\in P_0\left(\Tcal;\Rbb^2\right)}
    L(\vcr,\Lambda_0).
  \end{align*}
  Somit löst $\left(\tilde{u}_\CR,\bar\Lambda_0\right)\in \CR^1_0(\Tcal)\times
  \left( P_0\!\left(\Tcal;\Rbb^2\right)\cap K\right)$
  \Cref{prob:discreteSaddlepointProblem}.
\end{proof}

Nachdem diese Vorbereitungen abgeschlossen sind, können wir nun folgendes
Theorem beweisen.

\begin{theorem}
  \label{thm:discProbCharacterizationOfDiscreteSolutions}
  Für eine Funktion $\tilde{u}_\CR\in\CR^1_0(\Tcal)$ sind die folgenden drei
  Aussagen äquivalent.
  \begin{itemize}
    \item[(i)] \Cref{prob:discreteProblem} wird von $\tilde{u}_\CR$ gelöst.
    \item[(ii)] Es existiert ein
      $\bar\Lambda_0\in P_0\!\left(\Tcal;\Rbb^2\right)$ mit
      $\left|\bar\Lambda_0(\bullet)\right|\leq 1$
      fast überall in $\Omega$, sodass
      \begin{equation}
        \label{eq:discreteMultiplierScalerProductEquality}
        \bar\Lambda_0(\bullet)\cdot\gradnc\tilde{u}_\CR(\bullet)
        =
        \left|\gradnc\tilde{u}_\CR(\bullet)\right| 
        \quad\text{fast überall in } \Omega 
      \end{equation}
      und
      \begin{equation}
        \label{eq:discreteMultiplierL2Equality}
        \left(\bar\Lambda_0,\gradnc\vcr\right)
        = \left(f-\alpha\tilde{u}_\CR,
        \vcr\right)
        \quad\text{für alle } \vcr\in\CR^1_0(\Tcal).
      \end{equation}
    \item[(iii)] Für alle $\vcr\in\CR^1_0(\Tcal)$ gilt
      \begin{equation}
        \label{eq:discreteVariationalInequality}
        \left(f-\alpha\tilde{u}_\CR,\vcr-\tilde{u}_\CR\right)\leq
        \Vert\gradnc\vcr\Vert_{L^1(\Omega)} -
        \left\Vert\gradnc\tilde{u}_\CR\right\Vert_{L^1(\Omega)}.
      \end{equation}
  \end{itemize}
\end{theorem}

\begin{proof} 
  Sei $\tilde{u}_\CR\in\CR^1_0(\Tcal)$.

  \textit{(i) $\Rightarrow$ (ii).}
  Sei $\tilde{u}_\CR$ Lösung von \Cref{prob:discreteProblem}.
  Nach \Cref{lem:existenceSaddlepoint} existiert eine Lösung
  $\left(\hat{u}_\CR,\bar\Lambda_0\right)\in \CR^1_0(\Tcal)\times \left(
  P_0\!\left(\Tcal;\Rbb^2\right)\cap K\right)$ 
  von \Cref{prob:discreteSaddlepointProblem}. 
  Außerdem wissen wir, dass damit $\hat{u}_\CR$ Lösung von
  \Cref{prob:discreteProblem} ist.
  Daraus folgt, da nach \Cref{thm:discreteProblemExistenceUniqueness} die
  Lösung von \Cref{prob:discreteProblem} eindeutig ist, dass
  $\hat{u}_\CR=\tilde{u}_\CR$ in $\CR^1_0(\Tcal)$.
  Weiterhin wissen wir aus dem Beweis von \Cref{lem:existenceSaddlepoint}, dass
  $\left(\tilde{u}_\CR,\bar\Lambda_0\right)$ Sattelpunkt der Funktion $L$ aus
  \Cref{eq:discreteProblemLagrangeFunctional} ist.
  Das bedeutet nach \cite[380]{Roc70} insbesondere, dass $\tilde{u}_\CR$
  Minimierer von $L(\bullet, \bar\Lambda_0)$ in $\CR^1_0(\Tcal)$ und
  $\bar\Lambda_0$ Maximierer von $L\!\left(\tilde{u}_\CR,\bullet\right)$ über $
  P_0\!\left(\Tcal;\Rbb^2\right)$ ist.  
  Mit dieser Erkenntnis können wir nun die entsprechenden
  Optimalitätsbedingungen diskutieren.
  Zunächst gilt, da
  $L\!\left(\tilde{u}_\CR,\bullet\right):P_0\!\left(\Tcal;\Rbb^2\right)\to
  [-\infty,\infty)$ konkav und
  $\bar\Lambda_0$ Maximierer von $L\!\left(\tilde{u}_\CR,\bullet\right)$ in 
  $ P_0\!\left(\Tcal;\Rbb^2\right)$ ist, dass das konvexe Funktional
  $-L(\tilde{u}_\CR,\bullet):P_0\!\left(\Tcal;\Rbb^2\right)\to
  (-\infty,\infty]$ von $\bar\Lambda_0$ in $ P_0\!\left(\Tcal;\Rbb^2\right)$
  minimiert wird.
  %Nach den Theoremen \ref{thm:extremalprinciple},
  %\ref{thm:subdifferentialSumRule} und \ref{thm:subdiffGateaux} gilt somit
  Nach den Theoremen \ref{thm:extremalprinciple} --
  \ref{thm:subdifferentialSumRule} gilt somit
  \begin{align*}
    0
    \in 
    \partial \left(-L\!\left(\tilde{u}_\CR,\bullet\right)\right)
    \left(\bar\Lambda_0\right) 
    =
    \left\{-\!\left(\gradnc\tilde{u}_\CR,\bullet\right)\right\}+\partial I_K
    \left(\bar\Lambda_0\right).
  \end{align*}
  Äquivalent zu dieser Aussage ist, dass
  $\left(\gradnc\tilde{u}_\CR,\bullet\right)\in \partial
  I_K \left(\bar\Lambda_0\right)$. 
  Da $\bar\Lambda_0\in K$, folgt mit \Cref{def:subdifferential},
  dass für alle $\Lambda_0\in  P_0\!\left(\Tcal;\Rbb^2\right)$ gilt
  \begin{align*}
    \left(\gradnc\tilde{u}_\CR,\Lambda_0-\bar\Lambda_0\right) 
    \leq 
    I_K (\Lambda_0) - I_K\!\left(\bar\Lambda_0\right)
    =
    I_K (\Lambda_0).
  \end{align*}
  Falls $\Lambda_0\in  P_0\!\left(\Tcal;\Rbb^2\right)\cap K$, folgt insbesondere
  \begin{align}
    \label{eq:scalarProductInequDiscreteProof}
    \left(\gradnc\tilde{u}_\CR,\Lambda_0-\bar\Lambda_0\right) 
    &\leq 
    0,\quad\text{also }\notag\\ 
    \left(\gradnc\tilde{u}_\CR,\Lambda_0\right)
    &\leq
    \left(\gradnc\tilde{u}_\CR,\bar\Lambda_0\right).
  \end{align}
  Sei nun $\Lambda_0\in P_0\!\left(\Tcal;\Rbb^2\right)\cap K$ elementweise auf
  allen $T\in\Tcal$ durch $\Lambda_0(x)\in\sign\left(\gradnc\tilde{u}_\CR(x)\right)$
  definiert für alle $x\in\interior(T)$.
  Mit dieser Wahl von $\Lambda_0$, Ungleichung
  \eqref{eq:scalarProductInequDiscreteProof}, der Cauchy\--Schwarz\-schen
  Ungleichung und $\bar\Lambda_0\in K$ erhalten wir die Abschätzung
  \begin{align}
    \label{eq:sumOverAllTrianglesDualVariable}
    \int_\Omega\left|\gradnc\tilde{u}_\CR\right|\dx
    &=
    \int_\Omega\gradnc\tilde{u}_\CR\cdot\Lambda_0\dx
    \leq 
    \int_\Omega\gradnc\tilde{u}_\CR\cdot\bar\Lambda_0\dx \notag\\
    &\leq 
    \int_\Omega\left|\gradnc\tilde{u}_\CR\right|\left|\bar\Lambda_0\right|\dx
    \leq
    \int_\Omega\left|\gradnc\tilde{u}_\CR\right|\dx,
    \quad\text{das heißt }\notag\\
    \int_\Omega\left|\gradnc\tilde{u}_\CR\right|\dx 
    &= 
    \int_\Omega\gradnc\tilde{u}_\CR\cdot\bar\Lambda_0\dx
    \quad\text{beziehungsweise }\notag\\
    \sum_{T\in\Tcal}|T|\,\big|(\gradnc\tilde{u}_\CR)\!|_T\big|
    &=
    \sum_{T\in\Tcal}|T|\left(\gradnc\tilde{u}_\CR\cdot \bar\Lambda_0\right)\!\!|_T.
  \end{align}
  Außerdem gilt für alle $T\in\Tcal$ mit der Cauchy-Schwarzschen Ungleichung
  und $\bar\Lambda_0\in K$, dass 
  \begin{align*}
    \left(\gradnc\tilde{u}_\CR\cdot \bar\Lambda_0\right)\!\!|_T
  \leq
  \big|(\gradnc\tilde{u}_\CR)\!|_{T}\big|\,\left|\bar\Lambda_0|_T\right|
  \leq
  \big|(\gradnc\tilde{u}_\CR)\!|_{T}\big|.
  \end{align*}
  Mit \Cref{eq:sumOverAllTrianglesDualVariable} folgt daraus für alle
  $T\in\Tcal$, dass $\left(\gradnc\tilde{u}_\CR\cdot
  \bar\Lambda_0\right)\!\!|_T=\big|(\gradnc\tilde{u}_\CR)\!|_T\big|$, das heißt fast
  überall in $\Omega$ gilt $\bar\Lambda_0(\bullet)\cdot\gradnc\tilde{u}_\CR(\bullet)
  =|\gradnc\tilde{u}_\CR(\bullet)|$. 
  Damit ist \Cref{eq:discreteMultiplierScalerProductEquality} gezeigt.
  Als Nächstes betrachten wir das reellwertige Funktional
  $L\left(\bullet,\bar\Lambda_0\right):\CR^1_0(\Tcal)\to\Rbb$.
  Es ist Fr\'echet-differenzierbar mit
  \begin{align*}
    dL\!\left(\bullet,\bar\Lambda_0\right)\!\left(\tilde{u}_\CR;\vcr\right)
    =
    \int_\Omega\bar\Lambda_0\cdot \gradnc\vcr\dx
    +\alpha\! \left(\tilde{u}_\CR,\vcr\right) - \int_\Omega f\vcr\dx
  \end{align*}
  für alle $\vcr\in\CR^1_0(\Tcal)$.
  Da $\tilde{u}_\CR$ Minimierer von  $L\!\left(\bullet, \bar\Lambda_0\right)$
  in $\CR^1_0(\Tcal)$ ist, gilt nach
  \Cref{thm:necessaryConditionFreeLocalExtrema}, dass $0 =
  dL\!\left(\bullet,\bar\Lambda_0\right)\!\left(\tilde{u}_\CR;\vcr\right)$ für
  alle $\vcr\in\CR^1_0(\Tcal)$.
  Diese Bedingung ist für alle $\vcr\in\CR^1_0(\Tcal)$ äquivalent zu
  $\left(\bar\Lambda_0,\gradnc\vcr\right) = (f-\alpha \tilde{u}_\CR,\vcr)$.
  Somit ist auch \Cref{eq:discreteMultiplierL2Equality} gezeigt.

  \textit{(ii) $\Rightarrow$ (iii).}
  Die Funktion $\bar\Lambda_0\in P_0\!\left(\Tcal;\Rbb^2\right)$ erfülle
  $\left|\bar\Lambda_0(\bullet)\right|\leq 1$ fast überall in $\Omega$ sowie
  die Gleichungen \eqref{eq:discreteMultiplierScalerProductEquality} und 
  \eqref{eq:discreteMultiplierL2Equality}. 
  Sei $\vcr\in\CR^1_0(\Tcal)$.
  Mit den Gleichungen 
  \eqref{eq:discreteMultiplierL2Equality} und 
  \eqref{eq:discreteMultiplierScalerProductEquality} gilt
  \begin{equation}
    \label{eq:equivalentCharacterizationApplicationTwoEquations}
    \begin{aligned}
      \left(f-\alpha\tilde{u}_\CR,\vcr-\tilde{u}_\CR\right) 
      &=
      \left(\bar\Lambda_0,\gradnc\vcr\right)
      - \left(\bar\Lambda_0,\gradnc\tilde{u}_\CR\right)\\
      &=
      \int_\Omega\bar\Lambda_0\cdot\gradnc\vcr\dx
      - \int_\Omega\left|\gradnc\tilde{u}_\CR\right|\dx.
    \end{aligned}
  \end{equation}
  Weiterhin gilt mit der Cauchy-Schwarzschen
  Ungleichung und $\left|\bar\Lambda_0(\bullet)\right|\leq 1$ fast überall in
  $\Omega$, dass
  \begin{align*}
    \int_\Omega\bar\Lambda_0\cdot\gradnc\vcr\dx
    &\leq 
    \int_\Omega\left|\bar\Lambda_0\right|\,|\gradnc\vcr|\dx
    \leq 
    \int_\Omega|\gradnc\vcr|\dx.
  \end{align*}
  Zusammen mit \Cref{eq:equivalentCharacterizationApplicationTwoEquations}
  folgt daraus Ungleichung \eqref{eq:discreteVariationalInequality}.

  \textit{(iii) $\Rightarrow$ (i)}.
  Es gelte Ungleichung \eqref{eq:discreteVariationalInequality} für alle
  $\vcr\in\CR^1_0(\Tcal)$, also
  \begin{align*}
    \left(f-\alpha\tilde{u}_\CR,\vcr-\tilde{u}_\CR\right) 
    \leq
    \left\Vert\gradnc\vcr\right\Vert_{L^1(\Omega)}
    -\left\Vert\gradnc\tilde{u}_\CR\right\Vert_{L^1(\Omega)}.
  \end{align*}
  Nach \Cref{thm:discreteProblemExistenceUniqueness} exisitert eine eindeutige
  Lösung $\ucr\in\CR^1_0(\Tcal)$ von \Cref{prob:discreteProblem}.
  Wir haben bereits gezeigt, dass somit für alle
  $\vcr\in\CR^1_0(\Tcal)$ gilt
  \begin{align*}
    \left(f-\alpha\ucr,\vcr-\ucr\right) 
    \leq
    \left\Vert\gradnc\vcr\right\Vert_{L^1(\Omega)}
    -\Vert\gradnc\ucr\Vert_{L^1(\Omega)}.
  \end{align*}
  Um nun zu beweisen, dass $\tilde{u}_\CR$ \Cref{prob:discreteProblem} löst, genügt
  es $\tilde{u}_\CR=\ucr$ in $\CR^1_0(\Tcal)$ zu zeigen.
  Es gilt
  \begin{align*}
    \left(f-\alpha\ucr,\tilde{u}_\CR-\ucr\right) 
    &\leq
    \left\Vert\gradnc\tilde{u}_\CR\right\Vert_{L^1(\Omega)}
    -\Vert\gradnc\ucr\Vert_{L^1(\Omega)}\quad\text{und }\\
    \left(f-\alpha\tilde{u}_\CR,\ucr-\tilde{u}_\CR\right) 
    &\leq
    \left\Vert\gradnc\ucr\right\Vert_{L^1(\Omega)}
    -\Vert\gradnc\tilde{u}_\CR\Vert_{L^1(\Omega)}. 
  \end{align*}
  Die Addition dieser Ungleichungen
  impliziert
  \begin{align*}
    \alpha\left\Vert\tilde{u}_\CR-\ucr\right\Vert^2=
    \left(-\alpha\ucr,\tilde{u}_\CR-\ucr\right) 
    + \left(-\alpha\tilde{u}_\CR,\ucr-\tilde{u}_\CR\right) 
    \leq
    0.
  \end{align*}
  Da $\alpha>0$, folgt daraus $\left\Vert\tilde{u}_\CR-\ucr\right\Vert^2=0$,
  also $\tilde{u}_\CR=\ucr$ in $\CR^1_0(\Tcal)$.
\end{proof}

Zum Schluss dieses Abschnitts wollen wir noch zwei Bemerkungen von Professor
Carstensen erwähnen und kurz deren Gültigkeit begründen.
Die erste ist in eine äquivalente Charakterisierung der dualen Variable
$\bar\Lambda_0\in P_0\!\left(\Tcal;\Rbb^2\right)$ aus
\Cref{thm:discProbCharacterizationOfDiscreteSolutions} zur diskreten Lösung
$\ucr\in\CR^1_0(\Tcal)$ von \Cref{prob:discreteProblem}.

\begin{remark}
  Dass $\bar\Lambda_0\in P_0\!\left(\Tcal;\Rbb^2\right)$ fast überall in $\Omega$
  \Cref{eq:discreteMultiplierScalerProductEquality} und
  $|\bar\Lambda_0(\bullet)|\leq 1$ erfüllt, ist äquivalent zu der Bedingung
  $\bar\Lambda_0(x)\in\sign(\gradnc \ucr(x))$ für alle $x\in\interior(T)$ für
  alle $T\in\Tcal$.   
\end{remark}

\begin{proof}
  Dass die genannte Bedingung hinreichend ist, folgt direkt aus der Definition
  der Signumfunktion.
  Ihre Notwendigkeit folgt aus der folgenden Beobachtung.
  Da $\left|\bar\Lambda_0(\bullet)\right|\leq 1$ fast überall in $\Omega$, ist
  \Cref{eq:discreteMultiplierScalerProductEquality} eine Cauchy-Schwarzsche
  Ungleichung, bei der sogar Gleichheit gilt. 
  Dies ist genau dann der Fall, wenn $\bar\Lambda_0(\bullet)$ und
  $\gradnc\ucr(\bullet)$ fast überall in $\Omega$ linear abhängig sind.
\end{proof}
 
Daraus können wir folgern, unter welchen Umständen die duale Variable
$\bar\Lambda_0$ auf einem Dreieck $T\in\Tcal$ eindeutig bestimmt ist.

\begin{remark}
  Falls $\gradnc\ucr\neq 0$ auf $T\in\Tcal$, gilt nach Definition der
  Signumfunktion, dass $\bar\Lambda_0=\gradnc\ucr/|\gradnc\ucr|$ eindeutig
  bestimmt ist auf $T$.
  Im Allgemeinen ist $\bar\Lambda_0$ allerdings nicht eindeutig bestimmbar. 
  Betrachten wir zum Beispiel $f\equiv 0$ in \Cref{prob:discreteProblem} mit
  eindeutiger Lösung $\ucr\equiv 0$ fast überall in $\Omega$. 
  Dann erfüllt nach der diskreten Helmholtz Zerlegung \cite[S. 193, Theorem
  3.32]{Car09b} die Wahl $\bar\Lambda_0\coloneqq \Curl v_\C$ für ein beliebiges
  $v_\C\in S^1(\Tcal)$ mit $|\Curl v_\C|\leq 1$ die Charakterisierung
  \textit{(ii)} aus \Cref{thm:discProbCharacterizationOfDiscreteSolutions}.
\end{remark}

\section{Verfeinerungsindikator und garantierte untere Energieschranke}
\todo[inline]{vielleicht: Verfeinerungsindikator und garantierte 
Energieschranken/ garantierte unter und obere Energieschranke}

Professor Carstensen stellte für die numerischen Untersuchungen 
einen Verfeinerungsindikator zur adaptiven Netzverfeinerung und eine 
Aussage über eine garantierte untere Energieschranke zur Verfügung.

\begin{theorem}
  \label{thm:gleb}
  \todo[inline]{,,Mit minimaler Energie \ldots`` rausnehmen? Das ist halt
  wirklich überflüssig als Info}
  Sei $\Omega$ konvex, $f\in H^1_0(\Omega)$ das Eingangssignal für
  \Cref{prob:continuousProblem} mit Lösung $u\in H^1_0(\Omega)$ und minimaler
  Energie $E(u)$ sowie für \Cref{prob:discreteProblem} mit Lösung $\ucr\in
  \CR^1_0(\Omega)$ und minimaler Energie $\Enc(\ucr)$.
  Dann gilt
  \begin{align*}
    \Enc(\ucr)+\frac{\alpha}{2}\Vert u-\ucr\Vert^2
    -\frac{\kappa_\CR}{\alpha}\Vert
    h_\Tcal(f-\alpha\ucr)\Vert \Vert\nabla f\Vert\leq E(u).
  \end{align*}
  Dabei ist die Konstante $\kappa_\CR\coloneqq\sqrt{1/48+1/j_{1,1}^2}$ mit der
  kleinsten positiven Nullstelle $j_{1,1}$ der Bessel-Funktion erster Art.
  Insbesondere gilt dann für 
  \begin{align}
    \label{eq:gleb}
    \Egleb 
    \coloneqq 
    \Enc(\ucr) - \frac{\kappa_\CR}{\alpha}\Vert h_\Tcal(f-\alpha\ucr)\Vert
    \Vert \nabla f\Vert,
  \end{align}
    dass $\Enc(\ucr)\geq \Egleb$ und $E(u)\geq \Egleb$.
\end{theorem}

\begin{definition}[Verfeinerungsindikator]
  \label{def:refinementIndicator}
  Für $d\in\mathbb{N}$ (in dieser Arbeit stets $d=2$) und $0<\gamma\leq 1$
  definieren wir für alle $T\in\Tcal$ und $\ucr\in\CR^1_0(\Tcal)$ die
  Funktionen
  \begin{align*}
    \eta_\text{V}(T)
    &\coloneqq
    |T|^{2/d}\Vert f-\alpha \ucr\Vert^2_{L^2(T)}\quad\text{und }\\
    \eta_\text{J}(T)
    &\coloneqq
    |T|^{\gamma/d}\sum_{F\in\Ecal(T)}\left\Vert [\ucr]_F\right\Vert_{L^1(F)}.
  \end{align*} 
  Damit definieren wir den Verfeinerungsindikator
  $\eta\coloneqq\sum_{T\in\Tcal}\eta(T)$, wobei
  \begin{align} \label{eq:refinementIndicator} 
    \eta (T)
    \coloneqq
    \eta_\text{V}(T) + \eta_\text{J}(T)\quad\text{für alle } T\in\Tcal.
  \end{align} 
\end{definition}

\section{Kontrolle des Abstandes zwischen diskreter und kontinuierlicher
Lösung}
\todo[inline]{section womöglich rausnehmen und das alles zwischen EGLEB Thm
und Verfeinerungsind Def schieben.}
Wir möchten in diesen Abschnitt eine Abschätzungen herleiten, mit welcher
der $L^2$-Abstand zwischen der Lösung des diskreten Problems
\ref{prob:discreteProblem} und der Lösung des kontinuierlichen Problems
\ref{prob:continuousProblem} kontrolliert werden kann.
Dafür benötigen wir zunächst folgendes Theorem.
\todo[inline]{Dann wohl auch den Text hier ändern und sagen, dass es als
Vorbereitung für die GUEB gebraucht wird}

\todo[inline]{definitiv reinnehmen, aber vielleicht nur als Bemerkung und 
für Beweis auf Bartels verweisen}
\begin{theorem}[Eher ein Lemma draus machen?]
  \label{thm:convexity}
  Sei $u\in\BV(\Omega)\cap L^2(\Omega)$ Lösung von 
  \Cref{prob:continuousProblem}.
  Dann gilt 
  \begin{align*}
    \frac{\alpha}{2}\Vert u-v\Vert^2 \leq E(v)-E(u)\quad
    \text{für alle } v\in\BV(\Omega)\cap L^2(\Omega).
  \end{align*}
\end{theorem}

\begin{proof}
  Wir folgen der Argumentation im Beweis von \cite[S. 309, Lemma 10.2]{Bar15}.
  Da viele der Schritte ähnlich zum Beweis von \Cref{thm:contProbStabAndUniqu}
  sind, präsentieren wir die entsprechenden Argumente verkürzt.

  Wir definieren die konvexen Funktionale
  $F:L^2(\Omega)\to \Rbb\cup\{\infty\}$ und $G:L^2(\Omega)\to \Rbb$, wobei 
  $F$ wie im Beweis von \Cref{thm:contProbStabAndUniqu} definiert ist und $G$
  für alle $v\in L^2(\Omega)$ gegeben ist durch 
  \begin{align*}
    G(v)\coloneqq \frac{\alpha}{2}\Vert v\Vert^2 - \int_\Omega f v\dx.
  \end{align*}
  Es gilt $E = F+G$.
  Die Fr\'echet-Ableitung $G'(u): L^2(\Omega)\to\Rbb$ von $G$ an der Stelle
  $u\in \BV(\Omega)\cap L^2(\Omega)$ ist für alle $v\in L^2(\Omega)$ gegeben
  durch
  \begin{align*}
    dG(u;v) = \alpha (u,v) - \int_\Omega f v\dx 
    = (\alpha u-f ,v).
  \end{align*}
  Das impliziert mit wenigen Rechenschritten
  \begin{align}\label{eq:strongConvexityG}
    dG(u;v-u) +\frac{\alpha}{2}\Vert u-v\Vert^2+G(u) 
    =
    G(v)
    \quad\text{für alle } v\in L^2(\Omega).
  \end{align}
  Da $u$ der Minimierer von $E$ ist, erhalten wir mit den Theoremen
  %\ref{thm:extremalprinciple}, \ref{thm:subdifferentialSumRule} und
  %\ref{thm:subdiffGateaux} die Aussage
  \ref{thm:extremalprinciple} -- \ref{thm:subdifferentialSumRule} die Aussage
  \begin{align*}
    0\in\partial E(u) = \partial F(u)+\{G'(u)\},
  \end{align*}
  woraus folgt 
  $ -G'(u)\in\partial F(u).$
  Das ist nach \Cref{def:subdifferential} äquivalent zu
  \begin{align*}
    -dG(u;v-u)\leq F(v)-F(u)\quad\text{für alle }v\in\BV(\Omega)\cap
    L^2(\Omega).
  \end{align*}
  Daraus folgt zusammen mit \Cref{eq:strongConvexityG} für alle $v\in
  \BV(\Omega)\cap L^2(\Omega)$, dass
  \begin{align*}
    \frac{\alpha}{2}\Vert u-v\Vert^2+G(u)-G(v)+F(u)
    = -dG(u;v-u)+F(u)\leq F(v).
  \end{align*}
  Da $E=F+G$, folgt daraus die zu zeigende Aussage.
\end{proof}

\begin{theorem}[Garantierte obere Energieschranke]
  \label{thm:gueb}
  \todo[inline]{Write it, aber Tien noch fragen stellen}
  Sei $\Omega$ konvex, $f\in H^1_0(\Omega)$ das Eingangssignal für
  \Cref{prob:continuousProblem} mit Lösung $u\in H^1_0(\Omega)$ und minimaler
  Energie $E(u)$ sowie für \Cref{prob:discreteProblem} mit Lösung $\ucr\in
  \CR^1_0(\Omega)$ und minimaler Energie $\Enc(\ucr)$.
  Dann gilt
  \begin{align*}
    \Enc(\ucr)+\frac{\alpha}{2}\Vert u-\ucr\Vert^2
    -\frac{\kappa_\CR}{\alpha}\Vert
    h_\Tcal(f-\alpha\ucr)\Vert \Vert\nabla f\Vert\leq E(u).
  \end{align*}
  Dabei ist die Konstante $\kappa_\CR\coloneqq\sqrt{1/48+1/j_{1,1}^2}$ mit der
  kleinsten positiven Nullstelle $j_{1,1}$ der Bessel-Funktion erster Art.
  Insbesondere gilt dann für 
  \begin{align}
    \label{eq:gleb}
    \Egleb 
    \coloneqq 
    \Enc(\ucr) - \frac{\kappa_\CR}{\alpha}\Vert h_\Tcal(f-\alpha\ucr)\Vert
    \Vert \nabla f\Vert,
  \end{align}
    dass $\Enc(\ucr)\geq \Egleb$ und $E(u)\geq \Egleb$.

\end{theorem}

Zusammen mit den Betrachtungen in \Cref{sec:discreteProblemFormulation} 
erhalten wir damit die folgende Abschätzung.

\begin{corollary}[TODO löschen]
  Sei $u\in\BV(\Omega)\cap L^2(\Omega)$ Lösung von
  \Cref{prob:continuousProblem} und $\ucr\in\CR^1_0(\Tcal)$ Lösung von
  \Cref{prob:discreteProblem}.
  Dann gilt
  \begin{align*}
    \frac{\alpha}{2}\Vert u-\ucr\Vert^2\leq
    E(\ucr)-E(u)=\Enc(\ucr)+\sum_{F\in\Ecal}\Vert[\ucr]_F\Vert_{L^1(F)}-E(u).
  \end{align*}
  Insbesondere gilt auch 
  \begin{align*}
    \frac{\alpha}{2}\Vert u-\ucr\Vert^2
    \leq
    \left|\Enc(\ucr)-E(u)\right|+
    \left|\sum_{F\in\Ecal}\Vert[\ucr]_F\Vert_{L^1(F)}\right|.
  \end{align*}
\end{corollary}

\todo[inline]{Hier Fragen und Formeln}
Von Theorem 4.9:
  \begin{align*}
    \Enc(\ucr)+\frac{\alpha}{2}\Vert u-\ucr\Vert^2
    -\frac{\kappa_\CR}{\alpha}\Vert
    h_\Tcal(f-\alpha\ucr)\Vert \Vert\nabla f\Vert\leq E(u).
  \end{align*}
  impliziert für
  \begin{align*}
    \Egleb 
    \coloneqq 
    \Enc(\ucr) - \frac{\kappa_\CR}{\alpha}\Vert h_\Tcal(f-\alpha\ucr)\Vert
    \Vert \nabla f\Vert,
  \end{align*}
    dass $\Enc(\ucr)\geq \Egleb$ und $E(u)\geq \Egleb$. 

      deshalb garantierte untere Energieschranke?

      Falls ja, ist dann noch eine Abschätzung $E(u)\leq E_\textup{gueb}$ gewollt?
      Diese bekommen wir nicht so wirklich aus Thm. 4.11
      Es gilt natürlich z.B.
      $E(u)\leq E(J_1\ucr)=\Enc(J_1\ucr)$. Bringt das irgendwas? Hat nur
      leider keine h Potenz dabei.

      Weitere von Thm 4.9 implizierte Aussagen, die nützlich aussehen:

      \begin{align*}
        \Vert u-\ucr\Vert\leq 
        \sqrt{\frac{\alpha}{2}}\sqrt{E(u)-\Egleb}
        \lesssim
        \sqrt{E(u)-\Egleb}
      \end{align*}

      diese rechte Seite, ohne die Konstante, wurde geplottet (quadriert)
      und war dann parallel. Also gilt in den Experimenten sogar mehr. 
      Parallelität gilt aber damit nicht. (Bild zeigen, auch das angepasste
      mit sqrt)

      Diese Aussage ist keine A-priori Abschätzung, weil die rechte
      Seite immer noch von $\ucr$ abhängt, oder?
      Bringt der plot dieser RHS dann noch irgendwas
      

      Theorem 4.11 liefert
  \begin{align*}
    \frac{\alpha}{2}\Vert u-v\Vert^2 \leq E(v)-E(u)\quad
    \text{für alle } v\in\BV(\Omega)\cap L^2(\Omega).
  \end{align*}

  Daraus folgt
  \begin{align*}
    \Vert u-\ucr\Vert &\leq
    \sqrt{2/\alpha}\sqrt{\Enc(J_1{\ucr})-\Egleb} + \Vert J_1\ucr-\ucr\Vert\\
    &\lesssim
    \sqrt{\Enc(J_1{\ucr})-\Egleb} + \Vert J_1\ucr-\ucr\Vert
  \end{align*}

  Die unterscheidet sich dann aber nicht mehr von der anderen Aussage, die auch 
  eine obere Schranke für $\Vert u-\ucr\Vert$ war.
  Ich hatte mit einer unteren Schranke gerechnet, weil CC was von ,,zwischen
  zwei Graphen`` liegen meinte.
  ACHSO, die hier ist natürlich besser, weil die auch ohne Kentniss der exakten
  Lösung berechnet werden kann (A-posteriori, korrekt?). Den Gradienten von $f$
  braucht man allerdings noch, also kann man das für Cameraman immer noch nicht
  nutzen.

  Außerdem impliziert Theorem 4.11
  \begin{align*}
    E(u)\leq E(J_1 \ucr)-\frac{\alpha}{2}\Vert u - J_1 \ucr\Vert^2 
    = \Enc(J_1 \ucr)-\frac{\alpha}{2}\Vert u - J_1 \ucr\Vert^2 
  \end{align*}
  Bringt das was als GUEB oder zählt das nicht, da die rechte Seite immernoch von
  $u$ abhängt.

