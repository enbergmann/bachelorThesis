Betrachte für gegebenes $\alpha>0$ und rechte Seite $f\in L^2(\Omega)$ 
folgende Diskretisierung von \Cref{prob:continuousProblem}. 

\todo[inline]{Quote all CR and discretisation stuff right here somewhere (mglw
in einer subsection)}
\todo[inline]{Sowas wie $\Vnc$ nutzen als (geringfügige) Abkürzung oder einfach
immer die Räume ausschreiben?}

\begin{problem}\label{prob:discreteProblem}
  \todo[inline]{zitier BV stuff der relevant ist hierfür, überlege, wieso die
  Diskretisierung genau so aussieht (Normen fallen zusammen, gewisse Terme
  werden weggelassen, etc., siehe auch Draft on BV project für einige wichtige,
  noch zu beweisende, Statements)}
  Finde $\ucr\in \Vnc(\Tcal) \coloneqq \CR^1_0(\Tcal)$,
  sodass $\ucr$ das Funktional
  \begin{align}\label{eq:discreteProblem}
    \Enc(\vcr)\coloneqq \frac{\alpha}{2}\Vert \vcr\Vert_{L^2(\Omega)}^2
    +\Vert \gradnc\vcr\Vert_{L^1(\Omega)}-\int_\Omega f\vcr\dx
  \end{align}
  unter allen $\vcr\in \Vnc(\Tcal)$ minimiert.
\end{problem}

Definiere für $\vcr\in\Vnc(\Tcal)$, $\Lambda\in\Pbb_0(\Tcal;\Rbb^n)\subset
L^\infty(\Omega;\Rbb^n)$ \todo{stimmt das?},
\begin{align*}
  K_1(0)
  &\coloneqq 
  \{\Lambda\in L^\infty(\Omega;\Rbb^n)\ | \ |\Lambda(\bullet)|
  \leq 1 \text{ fast überall in }\Omega\},\\
  I_{K_1(0)}(\Lambda)
  &\coloneqq
  \begin{cases}
    \infty, & \text{falls } \Lambda\notin K_1(0),\\
    0,       & \text{falls } \Lambda\in K_1(0)
  \end{cases}
\end{align*}
und das Funktional $\Lcal_h:\CR^1_0(\Tcal)\times \Pbb_0(\Tcal;\Rbb^n)\to
[-\infty,\infty)$ durch
\begin{align}\label{eq:discreteProblemLagrangeFunctional}
  \Lcal_h(\vcr,\Lambda) \coloneqq \int_\Omega\Lambda\cdot\gradnc\vcr\dx +
  \frac{\alpha}{2}\Vert \vcr\Vert^2_{L^2(\Omega)} -\int_\Omega f\vcr\dx
  - I_{K_1(0)}(\Lambda).
\end{align}

Falls $\Lambda\notin K_1(0)$, gilt $\Lcal(\vcr,\Lambda)=-\infty$. Da
außerdem für beliebige $\Lambda\in\Pbb_0(\Tcal;\Rbb^n)\cap K_1(0)$ (d.h.\
$|\Lambda|\leq 1$ fast überall in $\Omega$ und
außerdem $I_{K_1(0)}(\Lambda)=0$) mit
der CSU \todo{stimmt das so} gilt, dass 
\begin{align*}
  \int_\Omega \Lambda\cdot\gradnc\vcr\dx
  \leq \int_\Omega |\Lambda\cdot\gradnc\vcr|\dx
  &\leq \int_\Omega |\Lambda||\gradnc\vcr|\dx\\
  &\leq \int_\Omega 1|\gradnc\vcr|\dx\
  = \Vert\gradnc\vcr\Vert_{L^1(\Omega)},
\end{align*}
folgt zunächst 
\begin{align*}
  \sup_{\Lambda\in\Pbb_0(\Tcal;\Rbb^n)}\Lcal(\vcr,\Lambda)=
  \sup_{\Lambda\in\Pbb_0(\Tcal;\Rbb^n)\cap K_1(0)}\Lcal(\vcr,\Lambda)
  \leq \Enc(\vcr).
\end{align*}

Weiterhin gilt für $\Lambda\in\sign(\gradnc\vcr)\subset \Pbb_0(\Tcal;\Rbb^n)\cap
K_1(0)$, dass
$\Enc(\vcr)=\Lcal(\vcr,\Lambda)$ und deshalb
$\Enc(\vcr)\leq\sup_{\Lambda\in\Pbb_0(\Tcal;\Rbb^n)}\Lcal(\vcr,\Lambda)$

Somit ist das folgende Sattelpunktsproblem äquivalent zu
\Cref{prob:discreteProblem}.
\begin{problem}\label{prob:discreteSaddlepointProblem}
  Löse
  \begin{align*}
    \inf_{\vcr\in\Vnc(\Tcal)}\sup_{\Lambda\in\Pbb_0(\Tcal;\Rbb^n)} 
    \Lcal_h(\vcr,\Lambda).
  \end{align*}
\end{problem}

\begin{theorem}[Charakterisierung diskreter Lösungen]
  \label{thm:discProbCharacterizationOfDiscreteSolutions}
  Es existiert eine eindeutiges Lösung $\ucr\in\Vnc(\Tcal)$ von
  \Cref{prob:discreteProblem}. Außerdem gelten folgende äquivalente 
  Charakterisierungen von $\ucr$.
  \begin{itemize}
    \item[(i)] Es existiert ein $\Lambda\in\Pbb_0(\Tcal;\Rbb^n)$ mit
      $|\Lambda(\bullet)|\leq 1$ fast überall in $\Omega$, sodass
      \begin{align*}
        \Lambda(\bullet)\cdot\gradnc\ucr(\bullet)
        &=
        |\gradnc\ucr(\bullet)| \quad\text{ fast überall in } \Omega \text{ und}\\
        \left(\Lambda,\gradnc\vcr\right)_{L^2(\Omega)}
        &= \left(f-\alpha\ucr,
        \vcr\right)_{L^2(\Omega)}
        \quad\text{ für alle } \vcr\in\Vnc(\Tcal).
      \end{align*}
    \item[(ii)] Für alle $\vcr\in\Vnc(\Tcal)$ gilt die Variationsungleichung
      \begin{align*}
        (f-\alpha\ucr,\vcr-\ucr)_{L^2(\Omega)}\leq
        \Vert\gradnc\vcr\Vert_{L^1(\Omega)} -
        \Vert\gradnc\ucr\Vert_{L^1(\Omega)}.
      \end{align*}
  \end{itemize}
\end{theorem}

\begin{proof}
  Mit analogen Abschätzungen wie in Ungleichung \eqref{eq:contProbBddFromBelow}
  erhalten wir für das Funktional $\Enc$ aus \Cref{prob:discreteProblem} 
  für alle $\vcr\in\CR^1_0(\Tcal)\subset L^2(\Omega)$ die Abschätzung 
  \begin{equation*}
    \Enc(\vcr) \geq -\frac{1}{\alpha}\Vert f\Vert_{L^2(\Omega)}^2.
  \end{equation*}
  Somit ist $\Enc$ nach unten beschränkt und es existiert eine infimierende
  Folge $(v_k)_{k\in\Nbb} \subset \CR^1_0(\Tcal)$ von $\Enc$. Aufgrund der
  Form von $\Enc$ ist diese Folge beschränkt bezüglich der Norm
  $\Vert\bullet\Vert_{L^2(\Omega)}$ und wegen der  Reflexivität des
  abgeschlossenen Unterraums $\CR^1_0(\Tcal)$ des reflexiven
  Raums $L^2(\Omega)$ besitzt diese Folge eine schwach konvergente 
  Teilfolge in $\CR^1_0(\Tcal)$
  bezüglich der Norm $L^2(\Omega)$, welche auch stark konvergent ist, da 
  $\CR^1_0(\Tcal)$ endlichdimensional ist. Der Grenzwert dieser Folge
  liegt aufgrund der Abgeschlossenheit von $\CR^1_0(\Tcal)$ in
  $\CR^1_0(\Tcal)$ und minimiert $\Enc$, da 
  $\Enc$ stetig ist bezüglich der Konvergenz in $L^2(\Omega)$.

  \todo[inline]{Absatz above: Sachen noch näher begründen? All die benutzten 
  grundlegenden Aussagen noch zusammen suchen und zitieren irgendwo?}

  Die Lösung $\ucr\in \CR^1_0(\Tcal)$ ist eindeutig, da das Funktional $\Enc$
  aus \Cref{prob:discreteProblem} strikt konvex ist {\color{red} der erste
  Term ist quadratisch, also strikt konvex, der zweite ist konvex und der 
  dritte linear, also ist deren Summe strikt konvex}.
  \todo[inline]{grundlegende Aussagen der Optimierung wie diese noch zitieren?
  Beweis ist einfach bei dieser, schneller Widerspruchsbeweis}

  Nachdem wir die Existenz eines eindeutigen Minimierers
  $\ucr\in\CR^1_0(\Tcal)$ von \Cref{prob:discreteProblem} bewiesen haben,
  zeigen wir nun die äquivalenten Charakterisierung von $\ucr$.
  Zunächst sei erwähnt, dass aus der Existenz des Minimierers von 
  \Cref{prob:discreteProblem} und der Äquivalenz von
  \Cref{prob:discreteProblem} und \Cref{prob:discreteSaddlepointProblem},
  wobei wir insbesondere beretis gezeigt haben, dass 
  $\Enc(\vcr)=\sup_{\Lambda\in\Pbb_0(\Tcal;\Rbb^n)}\Lcal_h(\vcr,\Lambda)$
  für alle $\vcr\in\CR^1_0(\Tcal)$, folgt,
  dass $\bar\Lambda\in\Pbb_0(\Tcal;\Rbb^n)\cap K_1(0)$ {\color{red} (denn
  sonst ist das innere $\sup$ nicht erfüllt, da sonst
  $-I_{K_1(0)}(\bar\Lambda)=-\infty$})
  existiert mit 
  \begin{align*}
    \Lcal_h\left(\ucr,\bar\Lambda\right)=
    \inf_{\vcr\in\CR^1_0(\Tcal)}\sup_{\Lambda\in\Pbb_0(\Tcal;\Rbb^n)} 
    \Lcal_h(\vcr,\Lambda).
  \end{align*}
  Da nach \cite[S. 379, Lemma 36.1]{Roc70} gilt, dass
  \begin{align*}
    \inf_{\vcr\in\CR^1_0(\Tcal)}\sup_{\Lambda\in\Pbb_0(\Tcal;\Rbb^n)} 
    \Lcal_h(\vcr,\Lambda)
    \geq 
    \sup_{\Lambda\in\Pbb_0(\Tcal;\Rbb^n)} \inf_{\vcr\in\CR^1_0(\Tcal)} 
    \Lcal_h(\vcr,\Lambda),
  \end{align*}
  folgt insgesamt
  \begin{align*}
    \inf_{\vcr\in\CR^1_0(\Tcal)}\sup_{\Lambda\in\Pbb_0(\Tcal;\Rbb^n)} 
    \Lcal_h(\vcr,\Lambda)
    =
    \Lcal_h\left(\ucr,\bar\Lambda\right)
    =
    \sup_{\Lambda\in\Pbb_0(\Tcal;\Rbb^n)} \inf_{\vcr\in\CR^1_0(\Tcal)} 
    \Lcal_h(\vcr,\Lambda).
  \end{align*}
  Somit ist $\left(\ucr,\bar\Lambda\right)\in 
  \CR^1_0(\Tcal)\times (\Pbb_0(\Tcal;\Rbb^n)\cap K_1(0))$ nach \cite[S. 380,
  Lemma
  36.2]{Roc70} Sattelpunkt 
  von $\Lcal_h$ bezüglich der Maximierung über $\Pbb_0(\Tcal;\Rbb^n)$ und
  der Minimierung über $\CR^1_0(\Tcal)$. 
  Das bedeutet insbesondere, dass $\ucr$ Minimierer von 
  $\Lcal_h(\bullet, \bar\Lambda)$ in $\CR^1_0(\Tcal)$ ist und $\bar\Lambda$
  Maximierer von $\Lcal_h(\ucr,\bullet)$ über $\Pbb_0(\Tcal;\Rbb^n)$.

  \textit{(i).} 
  In der ersten Komponente ist das Lagrange-Funktional \todo{'Lagrange' 
  weglassen, falls nicht doch noch benötigt? Keine Quelle nannte das bisher so}
  Fr\'echet-\\
  differenzierbar {\color{red} (und für $\bar\Lambda$ ist das Funktional
  reellwertig und nimmt nicht $-\infty$ an, also ist Zeidler anwendbar)} mit 
  \begin{align*}
    \delta_{\ucr}\Lcal_h(\ucr,\bar\Lambda)[\vcr]=
    \int_\Omega\bar\Lambda\cdot \gradnc\vcr\dx
    +\alpha (\ucr,\vcr)_{L^2(\Omega)} - \int_\Omega f\vcr\dx
  \end{align*}
  für alle $\vcr\in\CR^1_0(\Tcal)$.
  Da $\ucr$ Minimierer von  $\Lcal_h(\bullet, \bar\Lambda)$ {\color{red}
  (reellwertig!!!)} in $\CR^1_0(\Tcal)$
  ist, gilt $0 = \delta_{\ucr}\Lcal_h(\ucr,\bar\Lambda)[\vcr]$ 
  (\cite[S. 193, Theorem 40.A (oder vlt sogar 40.B nutzen)]{Zei85}).

  
  Diese Bedingung ist für alle $\vcr\in\Vnc(\Tcal)$ äquivalent zu
  \begin{align*}
    (\bar\Lambda,\gradnc\vcr)_{L^2(\Omega)}
    =
    (f-\alpha \ucr,\vcr)_{L^2(\Omega)}.
  \end{align*}
  Somit ist die zweite Aussage der (i) gezeigt.

  Die Karush-Kuhn-Tucker-Bedingungen für eine Lösung \\
  $(\ucr,\Lambda)\in\Vnc\times(\Pbb_0(\Tcal;\Rbb^n)\cap K_1(0))$
  des Sattelpunktsproblems \ref{prob:discreteSaddlepointProblem} lauten damit
  \begin{align*}
    0 
    &= 
    \delta_{\ucr}\Lcal_h(\ucr,\Lambda)[\vcr]\\
    &=
    \int_\Omega\Lambda\cdot \gradnc\vcr\dx
    +\alpha (\ucr,\vcr)_{L^2(\Omega)} - \int_\Omega f\vcr\dx \quad\text{ für 
    alle } \vcr\in\Vnc\quad\text{ und}\\
    0&\in \partial_\Lambda \Lcal_h(\ucr,\Lambda) 
    =
    \{(\gradnc\ucr,\bullet)_{L^2(\Omega)}\}-\partial I_{K_1(0)}(\Lambda).
  \end{align*}

  Die zweite Bedingung bedeutet, dass $(\gradnc\ucr,\bullet)_{L^2(\Omega)}\in -\partial
  I_{K_1(0)}(\Lambda)$, d.h.\ für
  alle $q_0\in \Pbb_0(\Tcal;\Rbb^n)$ gilt
  \begin{align*}
    (\gradnc\ucr,q_0-\Lambda)_{L^2(\Omega)} 
    \leq 
    I_{K_1(0)}(q_0) - I_{K_1(0)}(\Lambda)
    =
    I_{K_1(0)}(q_0). 
  \end{align*}
  Für $q_0\in \Pbb_0(\Tcal;\Rbb^n)\cap K_1(0)$ folgt insbesondere
  \begin{align*}
    (\gradnc\ucr,q_0-\Lambda)_{L^2(\Omega)}&\leq 0, \quad\text{ also}\\
    (\gradnc\ucr,q_0)_{L^2(\Omega)}&\leq(\gradnc\ucr,\Lambda)_{L^2(\Omega)}.
  \end{align*}
  Mit der Wahl $q_0\coloneqq \sign\gradnc\ucr$, der Cauchy-Schwarzschen
  Ungleichung und $\Lambda\in K_1(0)$ impliziert das
  \begin{align*}
    \int_\Omega|\gradnc\ucr|\dx
    &\leq
    (\gradnc\ucr,\Lambda)_{L^2(\Omega)}\\
    &\leq 
    \int_\Omega|\gradnc\ucr|\,|\Lambda|\dx
    \leq
    \int_\Omega|\gradnc\ucr|\dx\quad\text{ bzw. }\\
    \sum_{T\in\Tcal}|T|\,\left|(\gradnc\ucr)_{|_{T}}\right|
    &=
    \sum_{T\in\Tcal}|T|\,(\gradnc\ucr\cdot \Lambda)_{|_T}
  \end{align*}
  Außerdem gilt mit der Cauchy-Schwarzschen Ungleichung auf allen $T\in\Tcal$,
  dass $(\gradnc\ucr\cdot \Lambda)_{|_T}\leq(\gradnc\ucr)_{|_{T}}$.
  Dementsprechend muss sogar für alle $T\in\Tcal$ gelten, dass
  $(\gradnc\ucr\cdot \Lambda)_{|_T}=(\gradnc\ucr)_{|_{T}}$, d.h.\ fast überall
  in $\Omega$ gilt $\Lambda(\bullet)\cdot\gradnc\ucr(\bullet)
  =|\gradnc\ucr(\bullet)|$.
\end{proof}

