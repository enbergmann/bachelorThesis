Betrachte für gegebenes $\alpha>0$ und rechte Seite $f\in L^2(\Omega)$ 
folgende Diskretisierung von \Cref{prob:continuousProblem}. 

\todo[inline]{Quote all CR and discretisation stuff right here somewhere (mglw
in einer subsection)}

\begin{problem}\label{prob:discreteProblem}
  \todo[inline]{zitier BV stuff der relevant ist hierfür, überlege, wieso die
  Diskretisierung genau so aussieht (Normen fallen zusammen, gewisse Terme
  werden weggelassen, etc., siehe auch Draft on BV project für einige wichtige,
  noch zu beweisende, Statements)}
  Finde $\ucr\in \Vnc(\Tcal) \coloneqq \CR^1_0(\Tcal)$,
  sodass $\ucr$ das Funktional
  \begin{align}\label{eq:discreteProblem}
    \Enc(\vcr)\coloneqq \frac{\alpha}{2}\Vert \vcr\Vert_{L^2(\Omega)} 
    +\Vert \gradnc\vcr\Vert_{L^1(\Omega)}-\int_\Omega f\vcr\dx
  \end{align}
  unter allen $\vcr\in \Vnc(\Tcal)$ minimiert.
\end{problem}

Definiere für $\vcr\in\Vnc(\Tcal)$, $\Lambda\in\Pbb_0(\Tcal;\Rbb^n)\subset
L^\infty(\Omega;\Rbb^n)$ \todo{stimmt das?},
\begin{align*}
  K_1(0)
  &\coloneqq 
  \{\Lambda\in L^\infty(\Omega;\Rbb^n)\ | \ |\Lambda(\bullet)|
  \leq 1 \text{ fast überall in }\Omega\},\\
  I_{K_1(0)}(\Lambda)
  &\coloneqq
  \begin{cases}
    \infty, & \text{falls } \Lambda\notin K_1(0),\\
    0,       & \text{falls } \Lambda\in K_1(0)
  \end{cases}
\end{align*}
und das Funktional $\Lcal_h:\CR^1_0(\Tcal)\times \Pbb_0(\Tcal;\Rbb^n)\to
[-\infty,\infty)$ durch
\begin{align}\label{eq:discreteProblemLagrangeFunctional}
  \Lcal_h(\vcr,\Lambda) \coloneqq \int_\Omega\Lambda\cdot\gradnc\vcr\dx +
  \frac{\alpha}{2}\Vert \vcr\Vert^2_{L^2(\Omega)} -\int_\Omega f\vcr\dx
  - I_{K_1(0)}(\Lambda).
\end{align}

Falls $\Lambda\notin K_1(0)$, gilt $\Lcal(\vcr,\Lambda)=-\infty$. Da
außerdem für beliebige $\Lambda\in\Pbb_0(\Tcal;\Rbb^n)\cap K_1(0)$ (d.h.\
$|\Lambda|\leq 1$ fast überall in $\Omega$ und
außerdem $I_{K_1(0)}(\Lambda)=0$) mit
der CSU \todo{stimmt das so} gilt, dass 
\begin{align*}
  \int_\Omega \Lambda\cdot\gradnc\vcr\dx
  \leq \int_\Omega |\Lambda\cdot\gradnc\vcr|\dx
  &\leq \int_\Omega |\Lambda||\gradnc\vcr|\dx\\
  &\leq \int_\Omega 1|\gradnc\vcr|\dx\
  = \Vert\gradnc\vcr\Vert_{L^1(\Omega)},
\end{align*}
folgt zunächst 
\begin{align*}
  \sup_{\Lambda\in\Pbb_0(\Tcal;\Rbb^n)}\Lcal(\vcr,\Lambda)=
  \sup_{\Lambda\in\Pbb_0(\Tcal;\Rbb^n)\cap K_1(0)}\Lcal(\vcr,\Lambda)
  \leq \Enc(\vcr).
\end{align*}

Weiterhin gilt für $\Lambda\in\sign(\gradnc\vcr)\subset \Pbb_0(\Tcal;\Rbb^n)\cap
K_1(0)$, dass
$\Enc(\vcr)=\Lcal(\vcr,\Lambda)$ und deshalb
$\Enc(\vcr)\leq\sup_{\Lambda\in\Pbb_0(\Tcal;\Rbb^n)}\Lcal(\vcr,\Lambda)$

Somit ist das folgende Sattelpunktsproblem äquivalent zu
\Cref{prob:discreteProblem}.
\begin{problem}\label{prob:discreteSaddlepointProblem}
  Löse
  \begin{align*}
    \inf_{\vcr\in\Vnc(\Tcal)}\sup_{\Lambda\in\Pbb_0(\Tcal;\Rbb^n)} 
    \Lcal_h(\vcr,\Lambda).
  \end{align*}
\end{problem}

\begin{theorem}[Charakterisierung diskreter Lösungen]
  \label{thm:discProbCharacterizationOfDiscreteSolutions}
  Es existiert eine eindeutiges Lösung $\ucr\in\Vnc(\Tcal)$ von
  \Cref{prob:discreteProblem}. Außerdem gelten folgende äquivalente 
  Charakterisierungen von $\ucr$.
  \begin{itemize}
    \item[(i)] Es existiert ein $\Lambda\in\Pbb_0(\Tcal;\Rbb^n)$ mit
      $|\Lambda(\bullet)|\leq 1$ fast überall in $\Omega$, sodass
      \begin{align*}
        \Lambda(\bullet)\cdot\gradnc\ucr(\bullet)
        &=
        |\gradnc\ucr(\bullet)| \quad\text{ fast überall in } \Omega \text{ und}\\
        \left(\Lambda,\gradnc\vcr\right)_{L^2(\Omega)}
        &= \left(f-\alpha\ucr,
        \vcr\right)_{L^2(\Omega)}
        \quad\text{ für alle } \vcr\in\Vnc(\Tcal).
      \end{align*}
    \item[(ii)] Für alle $\vcr\in\Vnc(\Tcal)$ gilt die Variationsungleichung
      \begin{align*}
        (f-\alpha\ucr,\vcr-\ucr)_{L^2(\Omega)}\leq
        \Vert\gradnc\vcr\Vert_{L^1(\Omega)} -
        \Vert\gradnc\ucr\Vert_{L^1(\Omega)}.
      \end{align*}
  \end{itemize}
\end{theorem}

\begin{proof}
  %TODO Existenzbeweis für eindeutige Lösbarkeit des diskreten Problems.
  \textit{(i).} Es existiert eine eindeutige Lösung
  $\ucr\in\Vnc(\Tcal)$ von \Cref{prob:discreteProblem}, deshalb besitzt
  das Sattelpunktsproblem \ref{prob:discreteSaddlepointProblem} eine 
  Lösung $(\ucr,\Lambda)\in \Vnc(\Tcal)\times(\Pbb_0(\Tcal;\Rbb^n)\cap
  K_1(0)$. 
  In der ersten Komponente ist das Lagrange-Funktional
  Fr\'echet-\\
  differenzierbar mit 
  \begin{align*}
    \delta_{\ucr}\Lcal_h(\ucr,\Lambda)[\vcr]=
    \int_\Omega\Lambda\cdot \gradnc\vcr\dx
    +\alpha (\ucr,\vcr)_{L^2(\Omega)} - \int_\Omega f\vcr\dx.
  \end{align*}
  Die Karush-Kuhn-Tucker-Bedingungen für eine Lösung \\
  $(\ucr,\Lambda)\in\Vnc\times(\Pbb_0(\Tcal;\Rbb^n)\cap K_1(0))$
  des Sattelpunktsproblems \ref{prob:discreteSaddlepointProblem} lauten damit
  \begin{align*}
    0 
    &= 
    \delta_{\ucr}\Lcal_h(\ucr,\Lambda)[\vcr]\\
    &=
    \int_\Omega\Lambda\cdot \gradnc\vcr\dx
    +\alpha (\ucr,\vcr)_{L^2(\Omega)} - \int_\Omega f\vcr\dx \quad\text{ für 
    alle } \vcr\in\Vnc\quad\text{ und}\\
    0&\in \partial_\Lambda \Lcal_h(\ucr,\Lambda) 
    =
    \{(\gradnc\ucr,\bullet)_{L^2(\Omega)}\}-\partial I_{K_1(0)}(\Lambda).
  \end{align*}

  Die erste Bedingung ist für alle $\vcr\in\Vnc(\Tcal)$ äquivalent zu
  \begin{align*}
    (\Lambda,\gradnc\vcr)_{L^2(\Omega)}
    =
    (f-\alpha \ucr,\vcr)_{L^2(\Omega)}.
  \end{align*}

  Die zweite Bedingung bedeutet, dass $(\gradnc\ucr,\bullet)_{L^2(\Omega)}\in -\partial
  I_{K_1(0)}(\Lambda)$, d.h.\ für
  alle $q_0\in \Pbb_0(\Tcal;\Rbb^n)$ gilt
  \begin{align*}
    (\gradnc\ucr,q_0-\Lambda)_{L^2(\Omega)} 
    \leq 
    I_{K_1(0)}(q_0) - I_{K_1(0)}(\Lambda)
    =
    I_{K_1(0)}(q_0). 
  \end{align*}
  Für $q_0\in \Pbb_0(\Tcal;\Rbb^n)\cap K_1(0)$ folgt insbesondere
  \begin{align*}
    (\gradnc\ucr,q_0-\Lambda)_{L^2(\Omega)}&\leq 0, \quad\text{ also}\\
    (\gradnc\ucr,q_0)_{L^2(\Omega)}&\leq(\gradnc\ucr,\Lambda)_{L^2(\Omega)}.
  \end{align*}
  Mit der Wahl $q_0\coloneqq \sign\gradnc\ucr$, der Cauchy-Schwarzschen
  Ungleichung und $\Lambda\in K_1(0)$ impliziert das
  \begin{align*}
    \int_\Omega|\gradnc\ucr|\dx
    &\leq
    (\gradnc\ucr,\Lambda)_{L^2(\Omega)}\\
    &\leq 
    \int_\Omega|\gradnc\ucr|\,|\Lambda|\dx
    \leq
    \int_\Omega|\gradnc\ucr|\dx\quad\text{ bzw. }\\
    \sum_{T\in\Tcal}|T|\,\left|(\gradnc\ucr)_{|_{T}}\right|
    &=
    \sum_{T\in\Tcal}|T|\,(\gradnc\ucr\cdot \Lambda)_{|_T}
  \end{align*}
  Außerdem gilt mit der Cauchy-Schwarzschen Ungleichung auf allen $T\in\Tcal$,
  dass $(\gradnc\ucr\cdot \Lambda)_{|_T}\leq(\gradnc\ucr)_{|_{T}}$.
  Dementsprechend muss sogar für alle $T\in\Tcal$ gelten, dass
  $(\gradnc\ucr\cdot \Lambda)_{|_T}=(\gradnc\ucr)_{|_{T}}$, d.h.\ fast überall
  in $\Omega$ gilt $\Lambda(\bullet)\cdot\gradnc\ucr(\bullet)
  =|\gradnc\ucr(\bullet)|$.
\end{proof}

