
\section{Formulierung}
\todo[inline]{Quote all CR and discretisation stuff right here somewhere (mglw
in einer subsection)}

bevor wir das diskrete problem von (cref probcont) formulieren, bemerken wir,
dass jede CR Funtkion in BV ist, was induktiv aus
\cite[S. 404, Example 10.2.1]{ABM14} folgt unter Nutzung (noch irgendwas
Dichte Argument mäßiges für die totale Variation)
\begin{align*}
  |\vcr|_{\BV(\Omega)} = \Vert \gradnc \vcr\Vert_{L^1(\Omega)} +
  \sum_{F\in\Fcal(\Omega)}\int_F |[\vcr]_F|\ds,
\end{align*}
woraus folgt
\begin{align*}
  |\vcr|_{\BV(\Omega)} +\Vert\vcr\Vert_{L^1(\partial\Omega)} 
  = \Vert \gradnc \vcr\Vert_{L^1(\Omega)} +
  \sum_{F\in\Fcal}\int_F |[\vcr]_F|\ds,
\end{align*}
Wir betrachten eine nichtkonforme Diskretisierung, da wir die Sprungterme
weglassen, wenn wir 
  $|\vcr|_{\BV(\Omega)} +\Vert\vcr\Vert_{L^1(\partial\Omega)}$
durch 
  $\Vert \gradnc \vcr\Vert_{L^1(\Omega)}$ ersetzen.
  Somit erhalten wir:
  \todo[inline]{Above was just WIP, write that properly and cite stuff}

Betrachte für gegebenes $\alpha>0$ und rechte Seite $f\in L^2(\Omega)$ 
folgende Diskretisierung von \Cref{prob:continuousProblem}. 

\begin{problem}\label{prob:discreteProblem}
  Finde $\ucr\in \CR^1_0(\Tcal)$,
  sodass $\ucr$ das Funktional
  \begin{align}\label{eq:discreteProblem}
    \Enc(\vcr)\coloneqq \frac{\alpha}{2}\Vert \vcr\Vert_{L^2(\Omega)}^2
    +\Vert \gradnc\vcr\Vert_{L^1(\Omega)}-\int_\Omega f\vcr\dx
  \end{align}
  unter allen $\vcr\in \CR^1_0(\Tcal)$ minimiert.
\end{problem}

\todo[inline]{Auf die Schranke die durch die starke Konvexität von E
existiert verweisen und als Lemma hier formulieren und das 'Courant
Lösungen konvergieren für Netzweite gegen 0 gegen die kontinuieliche 
Lösung' von Bartels zitieren und sagen, dass wir so eine Aussage hier
schuldig bleiben, aber mit dieser aussage und der entsprechenden Rate
in den Experimenten vergleichen können. Auch nochmal angucken, was Bartels
noch zu der Rate sagt (er sagt in einer Bemerkung irgendwas von 
suboptimal oder so, nachgucken)}



\section{Existenz und eindeutige Lösbarkeit}

Definiere für $\vcr\in\CR^1_0(\Tcal)$, $\Lambda\in\Pbb_0(\Tcal;\Rbb^n)\subset
L^\infty(\Omega;\Rbb^n)$ \todo{stimmt das?},
\begin{align*}
  K_1(0)
  &\coloneqq 
  \{\Lambda\in L^\infty(\Omega;\Rbb^n)\ | \ |\Lambda(\bullet)|
  \leq 1 \text{ fast überall in }\Omega\},\\
  I_{K_1(0)}(\Lambda)
  &\coloneqq
  \begin{cases}
    \infty, & \text{falls } \Lambda\notin K_1(0),\\
    0,       & \text{falls } \Lambda\in K_1(0)
  \end{cases}
\end{align*}
und das Funktional $\Lcal_h:\CR^1_0(\Tcal)\times \Pbb_0(\Tcal;\Rbb^n)\to
[-\infty,\infty)$ durch
\begin{align}\label{eq:discreteProblemLagrangeFunctional}
  \Lcal_h(\vcr,\Lambda) \coloneqq \int_\Omega\Lambda\cdot\gradnc\vcr\dx +
  \frac{\alpha}{2}\Vert \vcr\Vert^2_{L^2(\Omega)} -\int_\Omega f\vcr\dx
  - I_{K_1(0)}(\Lambda).
\end{align}

Falls $\Lambda\notin K_1(0)$, gilt $\Lcal(\vcr,\Lambda)=-\infty$. Da
außerdem für beliebige $\Lambda\in\Pbb_0(\Tcal;\Rbb^n)\cap K_1(0)$ (d.h.\
$|\Lambda|\leq 1$ fast überall in $\Omega$ und
außerdem $I_{K_1(0)}(\Lambda)=0$) mit
der CSU \todo{stimmt das so} gilt, dass 
\begin{align*}
  \int_\Omega \Lambda\cdot\gradnc\vcr\dx
  \leq \int_\Omega |\Lambda\cdot\gradnc\vcr|\dx
  &\leq \int_\Omega |\Lambda||\gradnc\vcr|\dx\\
  &\leq \int_\Omega 1|\gradnc\vcr|\dx\
  = \Vert\gradnc\vcr\Vert_{L^1(\Omega)},
\end{align*}
folgt zunächst 
\begin{align*}
  \sup_{\Lambda\in\Pbb_0(\Tcal;\Rbb^n)}\Lcal(\vcr,\Lambda)=
  \sup_{\Lambda\in\Pbb_0(\Tcal;\Rbb^n)\cap K_1(0)}\Lcal(\vcr,\Lambda)
  \leq \Enc(\vcr).
\end{align*}

Weiterhin gilt für $\Lambda\in\sign(\gradnc\vcr)\subset \Pbb_0(\Tcal;\Rbb^n)\cap
K_1(0)$, dass
$\Enc(\vcr)=\Lcal(\vcr,\Lambda)$ und deshalb
$\Enc(\vcr)\leq\sup_{\Lambda\in\Pbb_0(\Tcal;\Rbb^n)}\Lcal(\vcr,\Lambda)$

Somit ist das folgende Sattelpunktsproblem äquivalent zu
\Cref{prob:discreteProblem}.
\begin{problem}\label{prob:discreteSaddlepointProblem}
  Löse
  \begin{align*}
    \inf_{\vcr\in\CR^1_0(\Tcal)}\sup_{\Lambda\in\Pbb_0(\Tcal;\Rbb^n)} 
    \Lcal_h(\vcr,\Lambda).
  \end{align*}
\end{problem}

\begin{theorem}[Charakterisierung diskreter Lösungen]
  \label{thm:discProbCharacterizationOfDiscreteSolutions}
  Es existiert eine eindeutige Lösung $\ucr\in\CR^1_0(\Tcal)$ von
  \Cref{prob:discreteProblem}.

  Außerdem sind die folgenden drei Aussagen
  für eine Funktion $\ucr\in\CR^1_0(\Tcal)$ äquivalent.
  \begin{itemize}
    \item[(i)] \Cref{prob:discreteProblem} wird von $\ucr$ gelöst.
    \item[(ii)] Es existiert ein
      $\bar\Lambda\in\Pbb_0(\Tcal;\Rbb^n)$ mit $|\bar\Lambda(\bullet)|\leq 1$
      fast überall in $\Omega$, sodass
      \begin{equation}
        \label{eq:discreteMultiplierScalerProductEquality}
        \bar\Lambda(\bullet)\cdot\gradnc\ucr(\bullet)
        =
        |\gradnc\ucr(\bullet)| \quad\text{fast überall in } \Omega 
      \end{equation}
      und
      \begin{equation}
        \label{eq:discreteMultiplierL2Equality}
        \left(\bar\Lambda,\gradnc\vcr\right)_{L^2(\Omega)}
        = \left(f-\alpha\ucr,
        \vcr\right)_{L^2(\Omega)}
        \quad\text{für alle } \vcr\in\CR^1_0(\Tcal).
      \end{equation}
    \item[(iii)] Für alle $\vcr\in\CR^1_0(\Tcal)$ gilt
      \begin{equation}
        \label{eq:discreteVariationalInequality}
        (f-\alpha\ucr,\vcr-\ucr)_{L^2(\Omega)}\leq
        \Vert\gradnc\vcr\Vert_{L^1(\Omega)} -
        \Vert\gradnc\ucr\Vert_{L^1(\Omega)}.
      \end{equation}
  \end{itemize}
\end{theorem}

\begin{proof}
  Mit analogen Abschätzungen wie in Ungleichung \eqref{eq:contProbBddFromBelow}
  erhalten wir für das Funktional $\Enc$ aus \Cref{prob:discreteProblem} 
  für alle $\vcr\in\CR^1_0(\Tcal)\subset L^2(\Omega)$ die Abschätzung 
  \begin{equation*}
    \Enc(\vcr) \geq -\frac{1}{\alpha}\Vert f\Vert_{L^2(\Omega)}^2.
  \end{equation*}
  Somit ist $\Enc$ nach unten beschränkt und es existiert eine infimierende
  Folge $(v_k)_{k\in\Nbb} \subset \CR^1_0(\Tcal)$ von $\Enc$. Aufgrund der
  Form von $\Enc$ ist diese Folge beschränkt bezüglich der Norm
  $\Vert\bullet\Vert_{L^2(\Omega)}$ und wegen der  Reflexivität des
  abgeschlossenen Unterraums $\CR^1_0(\Tcal)$ des reflexiven
  Raums $L^2(\Omega)$ besitzt diese Folge eine schwach konvergente 
  Teilfolge in $\CR^1_0(\Tcal)$
  bezüglich der Norm $L^2(\Omega)$, welche auch stark konvergent ist, da 
  $\CR^1_0(\Tcal)$ endlichdimensional ist. Der Grenzwert dieser Folge
  liegt aufgrund der Abgeschlossenheit von $\CR^1_0(\Tcal)$ in
  $\CR^1_0(\Tcal)$ und minimiert $\Enc$, da 
  $\Enc$ stetig ist bezüglich der Konvergenz in $L^2(\Omega)$.

  \todo[inline]{Absatz above: Sachen noch näher begründen? All die benutzten 
  grundlegenden Aussagen noch zusammen suchen und zitieren irgendwo?}

  Die Lösung $\ucr\in \CR^1_0(\Tcal)$ ist eindeutig, da das Funktional $\Enc$
  aus \Cref{prob:discreteProblem} strikt konvex ist {\color{red} der erste
  Term ist quadratisch, also strikt konvex, der zweite ist konvex und der 
  dritte linear, also ist deren Summe strikt konvex}.
  \todo[inline]{grundlegende Aussagen der Optimierung wie diese noch zitieren?
  Beweis ist einfach bei dieser, schneller Widerspruchsbeweis}

  Nachdem wir die Existenz eines eindeutigen Minimierers
  $\ucr\in\CR^1_0(\Tcal)$ von \Cref{prob:discreteProblem} bewiesen haben,
  zeigen wir nun die äquivalenten Charakterisierung von $\ucr$.

  \bigskip
  \textit{(i) $\Rightarrow$ (ii).}
  Zunächst sei erwähnt, dass aus der Existenz des Minimierers $\ucr$ von 
  \Cref{prob:discreteProblem} und der Äquivalenz des Minimierungsproblems
  \ref{prob:discreteProblem} und des
  Sattelpunktsproblems \ref{prob:discreteSaddlepointProblem},
  wobei wir insbesondere bereits gezeigt haben, dass 
  $\Enc(\vcr) = \sup_{\Lambda\in\Pbb_0(\Tcal;\Rbb^n)}\Lcal_h(\vcr,\Lambda)$
  für alle $\vcr\in\CR^1_0(\Tcal)$, folgt,
  dass $\bar\Lambda\in\Pbb_0(\Tcal;\Rbb^n)\cap K_1(0)$ {\color{red} (denn
  sonst ist das innere $\sup$ nicht erfüllt, da sonst
  $-I_{K_1(0)}(\bar\Lambda)=-\infty$})
  existiert mit 
  \begin{align*}
    \Lcal_h\left(\ucr,\bar\Lambda\right)=
    \inf_{\vcr\in\CR^1_0(\Tcal)}\sup_{\Lambda\in\Pbb_0(\Tcal;\Rbb^n)} 
    \Lcal_h(\vcr,\Lambda).
  \end{align*}
  Da nach \cite[S. 379, Lemma 36.1]{Roc70} gilt, dass
  \begin{align*}
    \inf_{\vcr\in\CR^1_0(\Tcal)}\sup_{\Lambda\in\Pbb_0(\Tcal;\Rbb^n)} 
    \Lcal_h(\vcr,\Lambda)
    \geq 
    \sup_{\Lambda\in\Pbb_0(\Tcal;\Rbb^n)} \inf_{\vcr\in\CR^1_0(\Tcal)} 
    \Lcal_h(\vcr,\Lambda),
  \end{align*}
  folgt insgesamt
  \begin{align*}
    \inf_{\vcr\in\CR^1_0(\Tcal)}\sup_{\Lambda\in\Pbb_0(\Tcal;\Rbb^n)} 
    \Lcal_h(\vcr,\Lambda)
    =
    \Lcal_h\left(\ucr,\bar\Lambda\right)
    =
    \sup_{\Lambda\in\Pbb_0(\Tcal;\Rbb^n)} \inf_{\vcr\in\CR^1_0(\Tcal)} 
    \Lcal_h(\vcr,\Lambda).
  \end{align*}
  Somit ist $\left(\ucr,\bar\Lambda\right)\in 
  \CR^1_0(\Tcal)\times (\Pbb_0(\Tcal;\Rbb^n)\cap K_1(0))$ nach \cite[S. 380,
  Lemma
  36.2]{Roc70} Sattelpunkt 
  von $\Lcal_h$ bezüglich der Maximierung über $\Pbb_0(\Tcal;\Rbb^n)$ und
  der Minimierung über $\CR^1_0(\Tcal)$. 
  Das bedeutet nach \cite[380]{Roc70} insbesondere, dass $\ucr$ Minimierer von 
  $\Lcal_h(\bullet, \bar\Lambda)$ in $\CR^1_0(\Tcal)$ ist und $\bar\Lambda$
  Maximierer von $\Lcal_h(\ucr,\bullet)$ über $\Pbb_0(\Tcal;\Rbb^n)$.

  \medskip
  In der zweiten Komponente ist $\Lcal_h$ konkav {(\color{red} $K_1(0)$ ist 
  konvex, somit ist $I_{K_1(0)}$ konvex, also $-I_{K_1(0)}$ konkav. Die 
  restlichen Terme sind konstant oder linear in $\Lambda$)}
  \todo[inline]{diese grundlegende Aussage über Indikatorfunktionen irgendwo
  (vielleicht sogar in Grundlagen) einmal zitieren}.
  Da wir bereits wissen, dass $\Lcal_h(\ucr,\bullet)$ von $\bar\Lambda$ in
  $\Pbb_0(\Tcal;\Rbb^n)$ maximiert wird, wird das konvexe Funktional 
  $-\Lcal_h(\ucr,\bullet)$ 
  \todo[inline]{zitieren, was konkav ist und das -konvex=konkav}
  von $\bar\Lambda$ in $\Pbb_0(\Tcal;\Rbb^n)$ minimiert 
  \todo[inline]{auch diese basic optimierungsaussage noch zitieren}.
  Nach den Theoremen \ref{thm:extremalprinciple},
  \ref{thm:subdifferentialSumRule} und \ref{thm:subdiffGateaux} gilt somit
  \begin{align*}
    0\in \partial \left(-\Lcal_h(\ucr,\bullet)\right)(\bar\Lambda) 
    =
    \{-(\gradnc\ucr,\bullet)_{L^2(\Omega)}\}+\partial I_{K_1(0)}(\bar\Lambda).
  \end{align*}
  Äquivalent zu dieser Aussage ist, dass
  $(\gradnc\ucr,\bullet)_{L^2(\Omega)}\in \partial
  I_{K_1(0)}(\bar\Lambda)$, das heißt nach \Cref{def:subdifferential} gilt für
  alle $\Lambda\in \Pbb_0(\Tcal;\Rbb^n)$ 
  \begin{align*}
    (\gradnc\ucr,\Lambda-\bar\Lambda)_{L^2(\Omega)} 
    \leq 
    I_{K_1(0)}(\Lambda) - I_{K_1(0)}(\bar\Lambda)
    =
    I_{K_1(0)}(\Lambda),
  \end{align*}
  da $\bar\Lambda\in K_1(0)$.
  Für $\Lambda\in \Pbb_0(\Tcal;\Rbb^n)\cap K_1(0)$ folgt insbesondere
  \begin{align*}
    (\gradnc\ucr,\Lambda-\bar\Lambda)_{L^2(\Omega)} 
    &\leq 0,\quad\text{also }\\ 
    (\gradnc\ucr,\Lambda)_{L^2(\Omega)}
    &\leq(\gradnc\ucr,\bar\Lambda)_{L^2(\Omega)}.
  \end{align*}
  Damit und der Wahl $\Lambda\in \sign\left(\gradnc\ucr\right)\subset 
  \Pbb_0(\Tcal;\Rbb^n)\cap K_1(0)$
  impliziert die Cauchy-Schwarzsche Ungleichung, dass
  \begin{align*}
    \int_\Omega|\gradnc\ucr|\dx
    &=
    (\gradnc\ucr,\Lambda)_{L^2(\Omega)}
    \leq 
    (\gradnc\ucr,\bar\Lambda)_{L^2(\Omega)}\\
    &\leq 
    \int_\Omega|\gradnc\ucr|\,|\bar\Lambda|\dx
    \leq
    \int_\Omega|\gradnc\ucr|\dx,\quad\text{also }\\
    \int_\Omega|\gradnc\ucr|\dx 
    &= 
    (\gradnc\ucr,\bar\Lambda)_{L^2(\Omega)}
    \quad\text{beziehungsweise }\\
    \sum_{T\in\Tcal}|T|\,\big|(\gradnc\ucr)|_T\big|
    &=
    \sum_{T\in\Tcal}|T|\,(\gradnc\ucr\cdot \bar\Lambda)|_T.
  \end{align*}
  Außerdem gilt mit der Cauchy-Schwarzschen Ungleichung auf allen $T\in\Tcal$,
  dass $(\gradnc\ucr\cdot
  \bar\Lambda)|_T\leq\big|(\gradnc\ucr)|_{T}\big|$, da
  $\bar\Lambda\in K_1(0)$.
  Dementsprechend muss {\color{red} (da somit alle Summanden der rechten Summe
  kleiner-gleich den entsprechenden Summanden (d.h. zum gleichen $T$) der
  linken Summe sind und Gleichheit der Summen somit nur noch möglich ist, wenn
  die entsprechenden Summanden tatsächlich gleich sind)} für alle
  $T\in\Tcal$ gelten, dass
  $(\gradnc\ucr\cdot \bar\Lambda)|_T=\big|(\gradnc\ucr)|_T\big|$,
  das heißt fast überall
  in $\Omega$ gilt $\bar\Lambda(\bullet)\cdot\gradnc\ucr(\bullet)
  =|\gradnc\ucr(\bullet)|$. 

  Damit ist \Cref{eq:discreteMultiplierScalerProductEquality} gezeigt.

  \medskip
  In der ersten Komponente ist das Lagrange-Funktional \todo{'Lagrange' 
  weglassen, falls nicht doch noch benötigt? Keine Quelle nannte das bisher so}
  Fr\'echet-\\
  differenzierbar {\color{red} (und für $\bar\Lambda$ ist das Funktional
  reellwertig und nimmt nicht $-\infty$ an, also ist Zeidler anwendbar)} mit 
  \begin{align*}
    d\Lcal_h(\bullet,\bar\Lambda)(\ucr;\vcr)=
    \int_\Omega\bar\Lambda\cdot \gradnc\vcr\dx
    +\alpha (\ucr,\vcr)_{L^2(\Omega)} - \int_\Omega f\vcr\dx
  \end{align*}
  für alle $\vcr\in\CR^1_0(\Tcal)$.
  Da $\ucr$ Minimierer von  $\Lcal_h(\bullet, \bar\Lambda)$ {\color{red}
  (reellwertig!!!)} in $\CR^1_0(\Tcal)$
  ist, gilt nach \Cref{thm:necessaryConditionFreeLocalExtrema}, dass $0 =
  d\Lcal_h(\bullet,\bar\Lambda)(\ucr;\vcr)$ .

  
  Diese Bedingung ist für alle $\vcr\in\CR^1_0(\Tcal)$ äquivalent zu
  \begin{align*}
    (\bar\Lambda,\gradnc\vcr)_{L^2(\Omega)}
    =
    (f-\alpha \ucr,\vcr)_{L^2(\Omega)}.
  \end{align*}

  Somit ist \Cref{eq:discreteMultiplierL2Equality} gezeigt.

  \bigskip
  \textit{(ii) $\Rightarrow$ (iii).}
  Für alle $\vcr\in\CR^1_0(\Tcal)$ gilt mit
  \Cref{eq:discreteMultiplierL2Equality}, der CSU,
  \Cref{eq:discreteMultiplierScalerProductEquality} und
  $\left|\bar\Lambda(\bullet)\right|\leq 1$ fast überall in $\Omega$ , dass
  \begin{align*}
    (f-\alpha\ucr,\vcr-\ucr)_{L^2(\Omega)} 
    &= 
    (f-\alpha\ucr,\vcr)_{L^2(\Omega)} 
    - (f-\alpha\ucr,\ucr)_{L^2(\Omega)} \\
    &=
    \left(\bar\Lambda,\gradnc\vcr\right)_{L^2(\Omega)}
    - \left(\bar\Lambda,\gradnc\ucr\right)_{L^2(\Omega)}\\
    &=
    \int_\Omega\bar\Lambda\cdot\gradnc\vcr\dx
    - \int_\Omega\bar\Lambda\cdot\gradnc\ucr\dx\\
    &\leq 
    \int_\Omega\left|\bar\Lambda\right|\,|\gradnc\vcr|\dx
    - \int_\Omega|\gradnc\ucr|\dx\\
    &\leq 
    \int_\Omega|\gradnc\vcr|\dx
    - \int_\Omega|\gradnc\ucr|\dx\\
    &=
    \Vert\gradnc\vcr\Vert_{L^1(\Omega)}
    -\Vert\gradnc\ucr\Vert_{L^1(\Omega)}.
  \end{align*}

  Damit löst $\ucr$ Ungleichung \eqref{eq:discreteVariationalInequality}
  für alle $\vcr\in\CR^1_0(\Tcal)$ in $\CR^1_0(\Tcal)$.

  \bigskip
  \textit{(iii) $\Rightarrow$ (i)}.
  Sei $\ucr\in\CR^1_0(\Tcal)$ Lösung von
  \Cref{prob:discreteProblem}.
  Wir haben bereits gezeigt, dass $\ucr$ stets existiert und außerdem, dass
  $\ucr$ insbesondere für alle $\vcr\in\CR^1_0(\Tcal)$ eine Lösung von
  Ungleichung \eqref{eq:discreteVariationalInequality} in $\CR^1_0(\Tcal)$ ist.

  Zu zeigen ist somit nur noch, dass eine beliebige Funktion
  $\tilde{u}_\CR\in\CR^1_0(\Tcal)$, die
  Ungleichung \eqref{eq:discreteVariationalInequality} in $\CR^1_0(\Tcal)$ für
  alle $\vcr\in\CR^1_0(\Tcal)$ löst, auch eine Lösung von
  \Cref{prob:discreteProblem} ist, das heißt zu zeigen ist
  $\tilde{u}_\CR=\ucr$.

  \medskip
  Für ein solches $\tilde{u}_\CR$ gilt
  \begin{align*}
    \left(f-\alpha\ucr,\tilde{u}_\CR-\ucr\right)_{L^2(\Omega)} 
    &\leq
    \left\Vert\gradnc\tilde{u}_\CR\right\Vert_{L^1(\Omega)}
    -\Vert\gradnc\ucr\Vert_{L^1(\Omega)}\quad\text{und }\\
    \left(f-\alpha\tilde{u}_\CR,\ucr-\tilde{u}_\CR\right)_{L^2(\Omega)} 
    &\leq
    \left\Vert\gradnc\ucr\right\Vert_{L^1(\Omega)}
    -\Vert\gradnc\tilde{u}_\CR\Vert_{L^1(\Omega)}. 
  \end{align*}
  Addition dieser Ungleichungen
  liefert die Ungleichung
  \begin{align*}
    \left(-\alpha\ucr,\tilde{u}_\CR-\ucr\right)_{L^2(\Omega)} 
    + \left(-\alpha\tilde{u}_\CR,\ucr-\tilde{u}_\CR\right)_{L^2(\Omega)} 
    \leq
    0,
  \end{align*}
  welche äquivalent ist zu
  \begin{align*}
    \alpha\left\Vert\tilde{u}_\CR-\ucr\right\Vert_{L^2(\Omega)}^2
    \leq
    0.
  \end{align*}
  Da $\alpha>0$, impliziert das
  $\left\Vert\tilde{u}_\CR-\ucr\right\Vert_{L^2(\Omega)}^2=0$, also
  $\tilde{u}_\CR=\ucr$ in $\CR^1_0(\Tcal)$.
\end{proof}

Als letztes noch eine von Prof. Carstensen angemerkte äquivalente 
Charakterisierung der dualen Variable $\bar\Lambda$ aus
  \Cref{thm:discProbCharacterizationOfDiscreteSolutions} zur diskreten Lösung
  $\ucr$.
\begin{remark}
  Das $\bar\Lambda$ fast überall in $\Omega$
  \Cref{eq:discreteMultiplierScalerProductEquality} und
  $|\bar\Lambda(\bullet)|\leq 1$ erfüllt, ist äquivalent zu 
  $\bar\Lambda\in\sign(\gradnc \ucr)$.   

  Die Notwendigkeit folgt hierbei unter anderem daraus, dass Gleichheit in der
  CSU gilt genau dann, wenn die Vektoren linear abhänigig sind, da
  $|\bar\Lambda|\leq 1$ gilt.

  Falls $\gradnc\ucr\neq 0$ auf $T\in\Tcal$, gilt somit, nach Definition von
  $\sign$, dass $\bar\Lambda=\gradnc\ucr/|\gradnc\ucr|$ eindeutig auf
  $T$.

  Im Allgemeinen ist $\bar\Lambda$ nicht eindeutig. So erfüllt zum Beispiel
  für $f\equiv 0$ in \Cref{prob:discreteProblem} mit eindeutiger Lösung
  $\ucr\equiv 0$ die Wahl $\bar\Lambda\coloneqq \Curl(v_\C)$ für ein beliebiges
  $v_\C\in S^1(\Tcal)$ mit $|\Curl(v_\C)|\leq 1$ die Eigenschaft \textit{(ii)}
  aus \Cref{thm:discProbCharacterizationOfDiscreteSolutions}.
\end{remark}

\section{Verfeinerungsindikator und garantierte untere Energieschranke}
Prof. Carstensen stellte zur numerischen Untersuchung den folgenden
Verfeinerungsindikator zur adaptiven Netzverfeinerung  und eine 
garantierte untere Energieschranke zur Verfügung.

\begin{definition}[Verfeinerungsindikator]\label{def:refinementIndicator}
  Für $n\in\mathbb{N}$ (hier $n=2$) und $0<\gamma\leq 1$ definieren wir den
  Verfeinerungsindikator $\eta\coloneqq\sum_{T\in\Tcal}\eta(T)$, wobei
  \begin{align} \label{eq:refinementIndicator} 
    \eta (T)\coloneqq
    \underbrace{|T|^{2/n}\Vert f-\alpha \ucr\Vert^2_{L^2(T)}}_{\eqqcolon
    \eta_\text{V}(T)} +\underbrace{|T|^{\gamma/n}\sum_{F\in\Fcal(T)}\Vert
    [\ucr]_F\Vert_{L^1(F)}}_{ \eqqcolon \eta_\text{J}(T)} 
  \end{align} 
  für alle $T\in\Tcal$.
\end{definition}

\begin{theorem}
  \label{thm:gleb}
  Ist $f\in H^1_0(\Omega)$, $u\in H^1_0(\Omega)$ Lösung von
  \Cref{prob:continuousProblem} mit minimaler Energie $E(u)$ und $\ucr\in
  \CR^1_0(\Omega)$ Lösung von \Cref{prob:discreteProblem} mit minimaler Energie
  $\Enc(\ucr)$, dann gilt
  \begin{align}
    \label{eq:gleb}
    \Enc(\ucr)+\frac{\alpha}{2}\Vert u-\ucr\Vert_{L^2(\Omega)}^2
    -\frac{\kappa_\CR}{\alpha}\Vert
    h_\Tcal(f-\alpha\ucr)\Vert_{L^2(\Omega)} |f|_{1,2}\leq E(u),
  \end{align}
  wobei $|\bullet|_{1,2}=\Vert\nabla \bullet\Vert_{L^2(\Omega)}$.
  
  Insbesondere gilt für 
  $\Egleb \coloneqq 
    \Enc(\ucr) - \frac{\kappa_\CR}{\alpha}\Vert
    h_\Tcal(f-\alpha\ucr)\Vert_{L^2(\Omega)} |f|_{1,2}$, dass
    $\Enc(\ucr)\geq \Egleb$ und $E(u)\geq \Egleb$.
    \todo[inline]{$\kappa_\CR$ (Bessel Nullstelle) zitieren und wohl noch was
    zur Regularität der Lösung bei dieser RHS sagen, also das Lösung 
    in H10 falls RHS in H10 (s. CC unausführliche Erwähung).}
\end{theorem}
