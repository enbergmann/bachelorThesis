Betrachte für gegebenes $\alpha>0$ und rechte Seite $f\in L^2(\Omega)$ 
folgende Diskretisierung von \Cref{prob:continuousProblem}. 

\begin{problem}\label{prob:discreteProblem}
  Finde $\ucr\in \Vnc(\Tcal) \coloneqq \CR^1_0(\Tcal)$,
  sodass $\ucr$ das Funktional
  \begin{align}\label{eq:discreteProblem}
    \Enc(\vcr)\coloneqq \frac{\alpha}{2}\Vert \vcr\Vert_{L^2(\Omega)} 
    +\Vert \gradnc\vcr\Vert_{L^1(\Omega)}-\int_\Omega f\vcr\dx
  \end{align}
  unter allen $\vcr\in \Vnc(\Tcal)$ minimiert.
\end{problem}

\noindent Für $\vcr\in\Vnc(\Tcal)$, $\Lambda\in\Pbb_0(\Tcal;\Rbb^n)$,
\begin{align*}
  K_1(0)
  &\coloneqq 
  \{\Lambda\in L^\infty(\Omega;\Rbb^n)\ | \ |\Lambda(\bullet)|
  \leq 1 \text{ fast überall in }\Omega\},\\
  I_{K_1(0)}(\Lambda)
  &\coloneqq
  \begin{cases}
    -\infty, & \text{falls } \Lambda\notin K_1(0) \text{ und}\\
    0,       & \text{falls } \Lambda\in K_1(0),
  \end{cases}
\end{align*}
ist das
dazugehörige Lagrange-Funktional definert als
\begin{align}\label{eq:discreteProblemLagrangeFunctional}
  \Lcal_h(\vcr,\Lambda) \coloneqq \int_\Omega\Lambda\cdot\gradnc\vcr\dx +
  \frac{\alpha}{2}\Vert \vcr\Vert^2_{L^2(\Omega)} -\int_\Omega f\vcr\dx
  - I_{K_1(0)}(\Lambda)
\end{align}
und das Sattelpunktsproblem dem entsprechend wie folgt.
\begin{problem}\label{prob:discreteSaddlepointProblem}
  Löse
  \begin{align*}
    \inf_{\vcr\in\Vnc(\Tcal)}\sup_{\Lambda\in\Pbb_0(\Tcal;\Rbb^n)} 
    \Lcal_h(\vcr,\Lambda).
  \end{align*}
\end{problem}

\begin{theorem}[Charakterisierung diskreter Lösungen]
  \label{thm:discProbCharacterizationOfDiscreteSolutions}
  Es existiert eine eindeutiges Lösung $\ucr\in\Vnc(\Tcal)$ von
  \Cref{prob:discreteProblem}. Außerdem gelten folgende äquivalente 
  Charakterisierungen von $\ucr$.
  \begin{itemize}
    \item[(i)] Es existiert ein $\Lambda\in\Pbb_0(\Tcal;\Rbb^n)$ mit
      $|\Lambda(\bullet)|\leq 1$ fast überall in $\Omega$, sodass
      \begin{align*}
        \big(\Lambda(\bullet),\gradnc\ucr(\bullet)\big)
        &=
        |\gradnc\ucr(\bullet)| \quad\text{ fast überall in } \Omega \text{ und}\\
        \left(\Lambda,\gradnc\vcr\right)_{L^2(\Omega)}
        &= \left(f-\alpha\ucr,
        \vcr\right)_{L^2(\Omega)}
        \quad\text{ für alle } \vcr\in\Vnc(\Tcal).
      \end{align*}
    \item[(ii)] Für alle $\vcr\in\Vnc$ gilt die Variationsungleichung
      \begin{align*}
        (f-\alpha\ucr,\vcr-\ucr)_{L^2(\Omega)}\leq
        \Vert\gradnc\vcr\Vert_{L^1(\Omega)} -
        \Vert\gradnc\ucr\Vert_{L^1(\Omega)}.
      \end{align*}
  \end{itemize}
\end{theorem}

\begin{proof}
  In der ersten Komponente ist das Lagrange-Funktional
  Fr\'echet-\\
  differenzierbar mit 
  \begin{align*}
    \delta_{\ucr}\Lcal_h(\ucr,\Lambda)[\vcr]=
    \int_\Omega\Lambda\cdot \gradnc\vcr\dx
    +\alpha (\ucr,\vcr)_{L^2(\Omega)} - \int_\Omega f\vcr\dx.
  \end{align*}
  Die Karush-Kuhn-Tucker-Bedingungen lauten damit
  \begin{align*}
    0 
    &= 
    \delta_{\ucr}\Lcal_h(\ucr,\Lambda)[\vcr]\\
    &=
    \int_\Omega\Lambda\cdot \gradnc\vcr\dx
    +\alpha (\ucr,\vcr)_{L^2(\Omega)} - \int_\Omega f\vcr\dx \quad\text{ für 
    alle } \vcr\in\Vnc\quad\text{ und}\\
    0&\in \partial_\Lambda \Lcal_h(\ucr,\Lambda) 
    =
    (\gradnc\ucr,\bullet)_{L^2(\Omega)}-\partial I_{K_1(0)}(\Lambda).
  \end{align*}

  Des Weiteren ist ein $\xi_h\in \Pbb_0(\Tcal;\Rbb^n)\cap
  \partial I_{K_1(0)}(\Lambda)$, wenn für alle $q_0\in \Pbb(\Tcal;\Rbb^n)$ gilt
  \begin{align*}
    (\xi_h,q_0-\Lambda)_{L^2(\Omega)} 
    \leq 
    I_{K_1(0)}(q_0) - I_{K_1(0)(\Lambda)}
    =
    I_{K_1(0)}(q_0). 
  \end{align*}
  Für $q_0\in \Pbb_0(\Tcal;\Rbb^n)\cap K_1(0)$ folgt insbesondere
  \begin{align*}
    (\xi_h,q_0-\Lambda)_{L^2(\Tcal;\Rbb^n)}&\leq 0, \quad\text{ also}\\
    (\xi_h,q_0)_{L^2(\Tcal;\Rbb^n)}&\leq(\xi_h,\Lambda)_{L^2(\Tcal;\Rbb^n)}.
  \end{align*}
  Mit der Wahl $q_0\coloneqq \sign\gradnc\ucr$ ergibt das



\end{proof}
