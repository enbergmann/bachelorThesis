\section{Seitennummerierungskonvention lokal}
welche Funktionen haben welche, an 
welche Stellen wird also umnummeriert

\section{Aufstellung des zu lösenden LGS}
Aufstellung der Gradienten etc.
 
\section{Berechnung der Werte einer CR Funktion auf einem Element in den Knoten}
Sei $T\in\Tcal$ mit $T = \conv\{P_1, P_2, P_3\}$. Die Seiten 
von $T$ seien $F_1 = \conv\{P_1,P_2\}$, $F_2 = \conv\{P_2,P_3\}$ und $F_3 =
\conv\{P_3,P_1\}$. Die Funktion
$\ucr\in\CR^1(\Tcal)$ habe in den Mittelpunkte der Seiten die Werte $u_j =
\ucr(\Mid(F_j))$ für alle $j=1,2,3$. Gesucht sind die Werte in den Knoten
$\ucr(P_1)$, $\ucr(P_2)$ und $\ucr(P_3)$.

Da $\ucr\in\CR^1(\Tcal)$ affin-linear ist, gilt für eine Kante
$F=\conv\{P,Q\}$, dass der Wert $\ucr(\Mid(F))$ von $\ucr$ im Mittlpunkt 
der Kante gegeben ist durch den Mittelwert der Werte von $\ucr$ in $P$ und $Q$.

Somit erhalten wir die drei Gleichungen
\begin{align*}
  u_1 &= \frac{\ucr(P_1)+\ucr(P_2)}{2},  
  &u_2 &= \frac{\ucr(P_2)+\ucr(P_3)}{2},  
  &u_3 &= \frac{\ucr(P_3)+\ucr(P_1)}{2}.
\end{align*}
Sind $u_1$, $u_2$ und $u_3$ bekannt, können wir dieses Gleichungssystem nach 
$\ucr(P_1)$, $\ucr(P_2)$ und $\ucr(P_3)$ lösen und erhalten
\begin{align*}
 \ucr(P_1)&=u_1+u_3-u_2, &\ucr(P_2)&= u_1+u_2-u_3,&\ucr(P_3)&=u_2+u_3-u_1.
\end{align*}

Dies wird realisiert in der Methode \texttt{computeNodeValuesCR4e}.

\section{Berechnung der L1 Norm der Sprünge}
Für die Berechnung des Verfeinerungsindikators [verweis auf entsprechende
section] und zur Auswertung der kontinuierlichen Energie $E(\vcr)$ einer 
Crouzeix-Raviart Funktion $\vcr$, deren diskrete Energie $\Enc(\vcr)$
bereits bekannt ist, werden die $L^1$ Normen der Kantensprünge 
$[\vcr]_F$ für alle Kanten $F\in\Fcal$ der Triangulierung benötigt,
wobei für eine Innenkante $F\in\Fcal(\Omega)$, die gemeinsame Kante der
Dreiecke $T_+$ und $T_-$ ist, gilt
$[\vcr]_F\coloneqq (\vcr|_{T_+})|_F-(\vcr|_{T_-})|_F$
und $[\vcr]_F \coloneqq \vcr|_F$ für eine Randkante
$F\in\Fcal(\partial\Omega)$. Die Konvention der Wahl von
$T_+$ und $T_-$ ist hier irrelevant, da wir
zur Berechung von $\Vert [\vcr]_F\Vert_{L^1(\Omega)}$
ausschließlich den Betrag $|[\vcr]_F|$ benötigen.

Da $\vcr\in\CR^1(\Tcal)$,
ist $[\vcr]_F$ affin linear und es gilt $[\vcr]_F(\Mid(F))=0$ für 
alle Innenkanten $F\in\Fcal(\Omega)$ und, falls 
$\vcr\in\CR^1_0(\Tcal)$, auch für alle Randkanten $F\in\Fcal(\partial\Omega)$.

Die folgenden Aussagen gelten also für Innenkanten beliebiger
Crouzeix-Raviart Funktionen, wir beschränken uns aber von nun
an auf Funktionen $\vcr\in\CR^1_0(\Tcal)$.

Betrachten wir also eine beliebige Kante $F\in\Fcal$
mit $F=\conv\{P_1,P_2\}$. 
Wir definieren eine Parametrisierung $\gamma:[0,2]\to\Rbb^2$ von $F$ durch
$\gamma(t)\coloneqq \frac{t}{2}(P_2-P_1)+P_1$. 
Es gilt $|\gamma'|\equiv \frac{1}{2}|P_2-P_1|=\frac{1}{2}|F|$.

Sei außerdem
$p(t)\coloneqq [\vcr]_F(\gamma(t))$. Dann gilt nach
 [cite Wegintegrale] 
\begin{align*}
  \Vert [\vcr]_F\Vert_{L^1(F)} 
  &=
  \int_F |[\vcr]_F|\ds 
  = \int_0^2 |p(t)|\,|\gamma'(t)|\dt
  = \frac{|F|}{2}\int_0^2 |p(t)|\dt\\
  &= \frac{|F|}{2}\left(\int_0^1 |p(t)|\dt + \int_1^2 |p(t)|\dt\right).
\end{align*}

Da $\vcr\in\CR^1_0(\Tcal)$, ist $|p|$ auf $[0,1]$ und $[1,2]$ jeweils
ein Polynom
vom Grad $1$ mit $p(1)=[\vcr]_F(\Mid(F))=0$, womit sich $|p|$ jeweils
explizit ausdrücken lässt durch
$|p|(t)=(1-t)|p|(0)$ für alle $t\in[0,1]$ und 
$|p|(t)=(t-1)|p|(2)$ für alle $t\in[1,2]$.
Die Mittelpunktsregel $\int_a^b f(x)\dx\approx (b-a)f( (a+b)/2)$ [cite] ist
exakt für Polynome vom Grad $1$ und somit gilt
\begin{align*}
  \int_0^1 |p(t)|\dt 
  &= 
  (1-0)|p|\left( \frac{1}{2} \right)
  =
  \frac{|p|(0)}{2}\quad\text{und }\\
  \int_1^2 |p(t)|\dt 
  &= 
  (2-1)|p|\left( \frac{3}{2} \right)
  =
  \frac{|p|(2)}{2}.
\end{align*}

Somit erhalten wir insgesamt 
\begin{align*}
  \Vert [\vcr]_F\Vert_{L^1(F)} 
  &=
  \frac{|F|}{2}\left(\frac{|p|(0)}{2} + \frac{|p|(2)}{2}\right)
  =
  \frac{|F|}{4}(|p|(0)+|p|(2))\\
  &= 
  \frac{|F|}{4}\big(|[\vcr]_F|(P_1)+|[\vcr]_F|(P_2)\big),
\end{align*}
beziehungsweise 
  $\Vert [\vcr]_F\Vert_{L^1(F)} =
  \frac{|F|}{4}\big(|\vcr|(P_1)+|\vcr|(P_2)\big)$ für eine Randkanten
  $F\in\Fcal(\partial\Omega)$.

Diese Berechnung ist realisiert durch die 
Funktionen \texttt{computeBlaJumps}, die die absoluten Sprünge
in den Endpunkte einer Kante berchnet, \texttt{computeAbsJumps}, die \ldots,
und \texttt{computeL1NormOfJumps}, die schließlich die $L^1$ Norm aller
Kantensprünge berechnet\ldots.



\section{Implementation der GLEB}

\section{Implementation des Refinement Indicators}

\section{Implementaition der exakten Energie Berechnung}
