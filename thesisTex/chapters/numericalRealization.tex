





\section{Seitennummerierungskonvention lokal}
welche Funktionen haben welche, an 
welche Stellen wird also umnummeriert

\section{Aufstellung des zu lösenden LGS}
Aufstellung der Gradienten etc.
 
\section{Berechnung der Werte einer CR Funktion auf einem Element in den Knoten}
Sei $T\in\Tcal$ mit $T = \conv\{P_1, P_2, P_3\}$. Die Seiten 
von $T$ seien $F_1 = \conv\{P_1,P_2\}$, $F_2 = \conv\{P_2,P_3\}$ und $F_3 =
\conv\{P_3,P_1\}$. Die Funktion
$\ucr\in\CR^1(\Tcal)$ habe in den Mittelpunkte der Seiten die Werte $u_j =
\ucr(\Mid(F_j))$ für alle $j=1,2,3$. Gesucht sind die Werte in den Knoten
$\ucr(P_1)$, $\ucr(P_2)$ und $\ucr(P_3)$.

Da $\ucr\in\CR^1(\Tcal)$ affin-linear ist, gilt für eine Kante
$F=\conv\{P,Q\}$, dass der Wert $\ucr(\Mid(F))$ von $\ucr$ im Mittlpunkt 
der Kante gegeben ist durch den Mittelwert der Werte von $\ucr$ in $P$ und $Q$.

Somit erhalten wir die drei Gleichungen
\begin{align*}
  u_1 &= \frac{\ucr(P_1)+\ucr(P_2)}{2},  
  &u_2 &= \frac{\ucr(P_2)+\ucr(P_3)}{2},  
  &u_3 &= \frac{\ucr(P_3)+\ucr(P_1)}{2}.
\end{align*}
Sind $u_1$, $u_2$ und $u_3$ bekannt, können wir dieses Gleichungssystem nach 
$\ucr(P_1)$, $\ucr(P_2)$ und $\ucr(P_3)$ lösen und erhalten
\begin{align*}
 \ucr(P_1)&=u_1+u_3-u_2, &\ucr(P_2)&= u_1+u_2-u_3,&\ucr(P_3)&=u_2+u_3-u_1.
\end{align*}

Dies wird realisiert in der Methode \texttt{computeNodeValuesCR4e}.

\section{Berechnung der L1 Norm der Sprünge}
Für die Berechnung des Verfeinerungsindikators [verweis auf entsprechende
section] und zur Auswertung der kontinuierlichen Energie $E(\vcr)$ einer 
Crouzeix-Raviart Funktion $\vcr$, deren diskrete Energie $\Enc(\vcr)$
bereits bekannt ist, werden die $L^1$ Normen der Kantensprünge 
$[\vcr]_F$ für alle Kanten $F\in\Fcal$ der Triangulierung benötigt,
wobei für eine Innenkante $F\in\Fcal(\Omega)$, die gemeinsame Kante der
Dreiecke $T_+$ und $T_-$ ist, gilt
$[\vcr]_F\coloneqq (\vcr|_{T_+})|_F-(\vcr|_{T_-})|_F$
und $[\vcr]_F \coloneqq \vcr|_F$ für eine Randkante
$F\in\Fcal(\partial\Omega)$. Die Konvention der Wahl von
$T_+$ und $T_-$ ist hier irrelevant, da wir
zur Berechung von $\Vert [\vcr]_F\Vert_{L^1(\Omega)}$
ausschließlich den Betrag $|[\vcr]_F|$ benötigen.

Da $\vcr\in\CR^1(\Tcal)$,
ist $[\vcr]_F$ affin linear und es gilt $[\vcr]_F(\Mid(F))=0$ für 
alle Innenkanten $F\in\Fcal(\Omega)$ und, falls 
$\vcr\in\CR^1_0(\Tcal)$, auch für alle Randkanten $F\in\Fcal(\partial\Omega)$.

Die folgenden Aussagen gelten also für Innenkanten beliebiger
Crouzeix-Raviart Funktionen, wir beschränken uns aber von nun
an auf Funktionen $\vcr\in\CR^1_0(\Tcal)$.

Betrachten wir also eine beliebige Kante $F\in\Fcal$
mit $F=\conv\{P_1,P_2\}$. 
Wir definieren eine Parametrisierung $\gamma:[0,2]\to\Rbb^2$ von $F$ durch
$\gamma(t)\coloneqq \frac{t}{2}(P_2-P_1)+P_1$. 
Es gilt $|\gamma'|\equiv \frac{1}{2}|P_2-P_1|=\frac{1}{2}|F|$.

Sei außerdem
$p(t)\coloneqq [\vcr]_F(\gamma(t))$. Dann gilt nach
 [cite Wegintegrale] 
\begin{align*}
  \Vert [\vcr]_F\Vert_{L^1(F)} 
  &=
  \int_F |[\vcr]_F|\ds 
  = \int_0^2 |p(t)|\,|\gamma'(t)|\dt
  = \frac{|F|}{2}\int_0^2 |p(t)|\dt\\
  &= \frac{|F|}{2}\left(\int_0^1 |p(t)|\dt + \int_1^2 |p(t)|\dt\right).
\end{align*}

Da $\vcr\in\CR^1_0(\Tcal)$, ist $|p|$ auf $[0,1]$ und $[1,2]$ jeweils
ein Polynom
vom Grad $1$ mit $p(1)=[\vcr]_F(\Mid(F))=0$, womit sich $|p|$ jeweils
explizit ausdrücken lässt durch
$|p|(t)=(1-t)|p|(0)$ für alle $t\in[0,1]$ und 
$|p|(t)=(t-1)|p|(2)$ für alle $t\in[1,2]$.
Die Mittelpunktsregel $\int_a^b f(x)\dx\approx (b-a)f( (a+b)/2)$ [cite] ist
exakt für Polynome vom Grad $1$ und somit gilt
\begin{align*}
  \int_0^1 |p(t)|\dt 
  &= 
  (1-0)|p|\left( \frac{1}{2} \right)
  =
  \frac{|p|(0)}{2}\quad\text{und }\\
  \int_1^2 |p(t)|\dt 
  &= 
  (2-1)|p|\left( \frac{3}{2} \right)
  =
  \frac{|p|(2)}{2}.
\end{align*}

Somit erhalten wir insgesamt 
\begin{align*}
  \Vert [\vcr]_F\Vert_{L^1(F)} 
  &=
  \frac{|F|}{2}\left(\frac{|p|(0)}{2} + \frac{|p|(2)}{2}\right)
  =
  \frac{|F|}{4}(|p|(0)+|p|(2))\\
  &= 
  \frac{|F|}{4}\big(|[\vcr]_F|(P_1)+|[\vcr]_F|(P_2)\big),
\end{align*}
beziehungsweise 
  $\Vert [\vcr]_F\Vert_{L^1(F)} =
  \frac{|F|}{4}\big(|\vcr|(P_1)+|\vcr|(P_2)\big)$ für eine Randkanten
  $F\in\Fcal(\partial\Omega)$.

Diese Berechnung ist realisiert durch die 
Funktionen \texttt{computeBlaJumps}, die die absoluten Sprünge
in den Endpunkte einer Kante berchnet, \texttt{computeAbsJumps}, die \ldots,
und \texttt{computeL1NormOfJumps}, die schließlich die $L^1$ Norm aller
Kantensprünge berechnet\ldots.



\section{Implementation der GLEB}

\section{Implementation des Refinement Indicators}

\section{Implementaition der exakten Energie Berechnung}


\section{Erstellen eines lauffähigen Benchmarks (Minimalbeispeil)}
Beschreibung der wichtigsten Parameter
und Idee hinter structs

Ordner, in denen die Funktionen für rechte Seite, Gradient, exakte
Lösung etc liegen müsssen

Wahrscheinlich flag für flag durchgehen, erklären, welche automatisch gesetzt 
werden u.U., und wann immer nötig sagen, was man vorher machen muss, wo man
Funktionen erstellen muss etc.

fur exakte Lösungs Beispiel usw.
Berechnung der exakten Energie, also alles was nur mehr Möglichkeiten bietet,
Verweis auf die nächste Section (in der dann sagen, welche Flags gesetzt werden 
können)

\section{Konstruktion eines Experiments mit exakter Lösung}
Um eine rechte Seite zu finden, zu der die exakte Lösung bekannt
ist, wähle eine Funktion des Radius $u\in H^1_0([0,1])$ mit Träger im 
zweidimensionalen Einheitskreis. Insbesondere muss damit gelten $u(1)=0$ und
$u$ stetig.
Die rechte Seite als Funktion des Radius $f\in L^2([0,1])$ ist dann gegeben
durch 
\begin{align*}
  f \coloneqq 
  \alpha u - \partial_r(\sign(\partial_r u)) - \frac{\sign(\partial_r u)}{r},
\end{align*}
wobei für $F\in\Rbb^2\setminus\{0\}$ gilt 
$\sign(F)\coloneqq \left\{\frac{F}{|F|}\right\}$ 
und $\sign(0)\in B_1(0)$.
Damit außerdem gilt $f\in H^1_0([0,1])$, was z.B.\ für GLEB relevant ist, 
muss also noch Stetigkeit von $\sign(\partial_r u)$ und 
$\partial_r(\sign(\partial_r u))$ verlangt werden und 
$\partial_r(\sign(\partial_r u(1))=\sign(\partial_r u(1))=0$.
Damit $f$ in $0$ definierbar ist, muss auch gelten 
$\sign(\partial_r u) \in o(r)$ für $r\to 0$.

Damit erhält man für die Funktion
\begin{align*}
  u_1(r)\coloneqq
  \begin{cases}
    1, & \text{wenn } 0\leq r\leq\frac{1}{6},\\
    1+(6r-1)^\beta, & \text{wenn } \frac{1}{6}\leq r\leq\frac{1}{3},\\
    2, &\text{wenn } \frac{1}{3}\leq r\leq\frac{1}{2},\\
    2(\frac{5}{2}-3r)^\beta, &\text{wenn } \frac{1}{2}\leq r\leq\frac{5}{6},\\
    0, &\text{wenn } \frac{5}{6}\leq r,
  \end{cases}
\end{align*}
wobei $\beta\geq 1/2$, mit der Wahl
\begin{align*}
  \sign(\partial_r u_1(r)) =
  \begin{cases}
    12r-36r^2, & \text{wenn } 0\leq r\leq\frac{1}{6},\\
    1, & \text{wenn } \frac{1}{6}\leq r\leq\frac{1}{3},\\
    \cos(\pi(6r-2)), &\text{wenn } \frac{1}{3}\leq r\leq\frac{1}{2},\\
    -1, &\text{wenn } \frac{1}{2}\leq r\leq\frac{5}{6},\\
    -\frac{1+\cos(\pi(6r-5))}{2}, &\text{wenn } \frac{5}{6}\leq r\leq 1,
  \end{cases}
\end{align*}
die rechte Seite
\begin{align*}
  f_1(r)\coloneqq 
  \begin{cases}
    \alpha-12(2-9r), & \text{wenn } 0\leq r\leq\frac{1}{6},\\
    \alpha(1+(6r-1)^\beta)-\frac{1}{r}, & \text{wenn } \frac{1}{6}\leq r\leq
    \frac{1}{3},\\
    2\alpha+6\pi\sin(\pi(6r-2))-\frac{1}{r}\cos(\pi(6r-2)), &
    \text{wenn } \frac{1}{3}\leq r\leq\frac{1}{2},\\
    2\alpha(\frac{5}{2}-3r)^\beta+\frac{1}{r},&
    \text{wenn } \frac{1}{2}\leq r\leq\frac{5}{6},\\
    -3\pi\sin(\pi(6r-5))+\frac{1+\cos(\pi(6r-5))}{2r}, &
    \text{wenn } \frac{5}{6}\leq r\leq 1.
  \end{cases}
\end{align*}

Für die Funktion
\begin{align*}
  u_2(r)\coloneqq 
  \begin{cases}
    1, & \text{wenn } 0\leq r\leq\frac{1-\beta}{2},\\
    -\frac{1}{\beta}r + \frac{1+\beta}{2\beta}, & 
    \text{wenn } \frac{1-\beta}{2}\leq r\leq \frac{1+\beta}{2},\\
    0, & \text{wenn } \frac{1+\beta}{2}\leq r,
  \end{cases}
\end{align*}
erhält man mit der Wahl
\begin{align*}
  \sign&(\partial_r u_2(r)) \\
  &\coloneqq 
  \begin{cases}
    \frac{4}{1-\beta}r\left(\frac{1}{1-\beta}r -1\right), &
    \text{wenn } 0\leq r\leq\frac{1-\beta}{2},\\
    -1, & \text{wenn } \frac{1-\beta}{2}\leq r\leq \frac{1+\beta}{2},\\
    \frac{4}{(\beta-1)^3}
    \left( 4r^3-3(\beta+3)r^2 +6(\beta+1)r-3\beta-1\right), & 
    \text{wenn } \frac{1+\beta}{2}\leq r\leq 1,
  \end{cases}
\end{align*}
die rechte Seite
\begin{align*}
  f_2(r)\coloneqq 
  \begin{cases}
    \alpha - \frac{4}{1-\beta}\left(\frac{3}{1-\beta}r - 2\right), &
    \text{wenn } 0\leq r\leq\frac{1-\beta}{2},\\
    -\frac{\alpha}{\beta}\left( r-\frac{1+\beta}{2} \right) +\frac{1}{r}, & 
    \text{wenn } \frac{1-\beta}{2}\leq r\leq \frac{1+\beta}{2},\\
    \frac{-4}{(\beta-1)^3}
    \left( 16r^2 -9(\beta+3)r + 12(\beta+1) - \frac{3\beta+1}{r}\right), & 
    \text{wenn } \frac{1+\beta}{2}\leq r\leq 1.
  \end{cases}
\end{align*}

Es folgen zwei Beispiele mit exakter Lösung $u_3=u_4 \in H^2_0((0,1)^2)$, 
gegeben durch 
\begin{align*}
  u_3(r)=u_4(r)\coloneqq 
  \begin{cases}
    1, & \text{wenn } 0\leq r\leq\frac{1}{3},\\
    54r^3 - 81r^2 + 36r - 4, & 
    \text{wenn } \frac{1}{3}\leq r\leq \frac{2}{3},\\
    0, & \text{wenn } \frac{2}{3}\leq r.
  \end{cases}
\end{align*}
Mit der Wahl
\begin{align*}
  \sign&(\partial_r u_3(r)) \\
  &\coloneqq 
  \begin{cases}
    54r^3-27r^2, & \text{wenn } 0\leq r\leq\frac{1}{3},\\
    -1, & \text{wenn } \frac{1}{3}\leq r\leq \frac{2}{3},\\
    -54r^3 + 135r^2 - 108r + 27, & \text{wenn } \frac{2}{3}\leq r\leq 1,
  \end{cases}
\end{align*}
erhalten wir die rechte Seite
\begin{align*}
  f_3(r)\coloneqq 
  \begin{cases}
    \alpha - 216r^2 + 81r, &
    \text{wenn } 0\leq r\leq\frac{1}{3},\\
    \alpha\left(54r^3 - 81r^2 + 36r - 4\right)) + \frac{1}{r}, & 
    \text{wenn } \frac{1}{3}\leq r\leq \frac{2}{3},\\
    216r^2 - 405r + 216 - \frac{27}{r}, & 
    \text{wenn } \frac{2}{3}\leq r\leq 1,
  \end{cases}
\end{align*}
für die gilt $f_3\in H^1_0$
und mit der Wahl
\begin{align*}
  \sign&(\partial_r u_4(r)) \\
  &\coloneqq 
  \begin{cases}
    -1458r^5 + 1215r^4 - 270r^3, & \text{wenn } 0\leq r\leq\frac{1}{3},\\
    -1, & \text{wenn } \frac{1}{3}\leq r\leq \frac{2}{3},\\
    -243r^4 + 756r^3 - 864r^2 + 432r - 81, 
    & \text{wenn } \frac{2}{3}\leq r\leq 1,
  \end{cases}
\end{align*}
erhalten wir die rechte Seite
\begin{align*}
  f_4(r)\coloneqq 
  \begin{cases}
    \alpha + 8748r^4 - 6075r^3 + 1080r^2, &
    \text{wenn } 0\leq r\leq\frac{1}{3},\\
    \alpha\left(54r^3 - 81r^2 + 36r - 4\right) + \frac{1}{r}, & 
    \text{wenn } \frac{1}{3}\leq r\leq \frac{2}{3},\\
    1215r^3 - 3024r^2 + 2592r - 864 + \frac{81}{r}, & 
    \text{wenn } \frac{2}{3}\leq r\leq 1,
  \end{cases}
\end{align*}
für die gilt $f_4\in H^2_0$.

Damit können Experimente durchgeführt werden bei denen 
\texttt{exactSolutionKnown = true} gesetzt werden kann und entsprechend auch 
der $L^2$-Fehler berechnet wird.

Soll nun auch die Differenz der exakten Energie mit der garantierten unteren 
Energie Schranke (GLEB) berechnet werden, dann werden die stückweisen
Gradienten der exakten Lösung und der rechten Seite benötigt.

Dabei gelten folgende Ableitungsregeln für die Ableitungen einer Funktion 
$g$, wenn man ihr Argument $x=(x_1,x_2)\in\Rbb^2$ in Polarkoordinaten mit Länge
$r=\sqrt{x_1^2+x_2^2}$ und Winkel
$\varphi = \atan(x_2,x_1)$, wobei 
\begin{align*}
  \atan(x_2,x_1)\coloneqq
  \begin{cases}
    \arctan\left( \frac{x_2}{x_1} \right),& \text{wenn }x_1>0,\\
    \arctan\left( \frac{x_2}{x_1} \right) +\pi,& \text{wenn }x_1<0,x_2\geq 0,\\
    \arctan\left( \frac{x_2}{x_1} \right) -\pi,& \text{wenn }x_1<0,x_2<0,\\
    \frac{\pi}{2},& \text{wenn }x_1=0,x_2>0,\\
    -\frac{\pi}{2},& \text{wenn }x_1=0,x_2<0,\\
    \text{undefiniert},& \text{wenn }x_1=x_2=0,\\
  \end{cases}
\end{align*}
auffasst,
\begin{align*}
  \partial_{x_1} &= 
  \cos(\varphi)\partial_r - \frac{1}{r}\sin(\varphi)\partial_\varphi,\\
  \partial_{x_2} &= 
  \sin(\varphi)\partial_r - \frac{1}{r}\cos(\varphi)\partial_\varphi.
\end{align*}
Ist $g$ vom Winkel $\varphi$ unabhängig, so ergibt sich
\begin{align*}
  \nabla_{(x_1,x_2)}g = (\cos(\varphi),\sin(\varphi))\partial_r g.
\end{align*}
Unter Beachtung der trigonometrischen Zusammenhänge
\begin{align*}
  \sin(\arctan(y)) = \frac{y}{\sqrt{1+y^2}},\\
  \cos(\arctan(y)) = \frac{1}{\sqrt{1+y^2}}
\end{align*}
ergibt sich 
\begin{align*}
  (\cos(\varphi),\sin(\varphi)) = (x_1,x_2)\frac{1}{r}
\end{align*}
und damit 
\begin{align*}
  \nabla_{(x_1,x_2)}g = (x_1,x_2)\frac{\partial_r g}{r},
\end{align*} 
es muss also nur $\partial_r g$ bestimmt werden.

Die entsprechenden Ableitungen lauten
\begin{align*}
  \partial_r f_1(r) &= 
  \begin{cases}
    108,&
    \text{wenn }0\leq r\leq\frac{1}{6},\\
    6\alpha\beta(6r-1)^{\beta-1} +\frac{1}{r^2}, &
    \text{wenn } \frac{1}{6}\leq r\leq\frac{1}{3},\\
    (36\pi^2+\frac{1}{r^2})\cos(\pi(6r-2))+
    \frac{6\pi}{r}\sin(\pi(6r-2)), &
    \text{wenn } \frac{1}{3}\leq r\leq\frac{1}{2},\\
    -\left(6\alpha\beta\left( \frac{5}{2}-3r \right)^{\beta-1}+
    \frac{1}{r^2}\right),&
    \text{wenn } \frac{1}{2}\leq r\leq\frac{5}{6},\\
    -\left( \left( 18\pi^2+\frac{1}{2r^2} \right)\cos(\pi(6r-5))
    +\frac{1}{2r^2}+\frac{3\pi}{r}\sin(\pi(6r-5))\right),
    &\text{wenn } \frac{5}{6}\leq r\leq 1,
  \end{cases}\\
  \partial_r u_1(r) &= 
  \begin{cases}
    0,&\text{wenn }0\leq r\leq\frac{1}{6},\\
    6\beta(6r-1)^{\beta-1}, &\text{wenn } \frac{1}{6}\leq r\leq\frac{1}{3},\\
    0, &\text{wenn } \frac{1}{3}\leq r\leq\frac{1}{2},\\
    -6\beta\left( \frac{5}{2}-3r \right)^{\beta-1},&
    \text{wenn } \frac{1}{2}\leq r\leq\frac{5}{6},\\
    0,&\text{wenn } \frac{5}{6}\leq r,
  \end{cases}\\
  \partial_r f_2(r) &= 
  \begin{cases}
    -\frac{12}{(1-\beta)^2},&\text{wenn }0\leq r\leq\frac{1-\beta}{2},\\
    -\frac{\alpha}{\beta}-\frac{1}{r^2},&
    \text{wenn } \frac{1-\beta}{2}\leq r\leq \frac{1+\beta}{2},\\
    -\frac{4}{(1-\beta)^3}\left( 32r-9(\beta+3)+\frac{3\beta+1}{r^2} \right),&
    \text{wenn } \frac{1+\beta}{2}\leq r\leq 1,\\
  \end{cases}\\
  \partial_r u_2(r) &= 
  \begin{cases}
    0,&\text{wenn }0\leq r\leq\frac{1-\beta}{2},\\
    -\frac{1}{\beta},&
    \text{wenn } \frac{1-\beta}{2}\leq r\leq \frac{1+\beta}{2},\\
    0,&\text{wenn } \frac{1+\beta}{2}\leq r,
  \end{cases}\\
  \partial_r f_3(r) &=
  \begin{cases}
    - 432r + 81, & \text{wenn } 0\leq r\leq\frac{1}{3},\\
    \alpha\left(162r^2 - 162r + 36\right) - \frac{1}{r^2}, & 
    \text{wenn } \frac{1}{3}\leq r\leq \frac{2}{3},\\
    432r - 405 + \frac{27}{r^2}, & 
    \text{wenn } \frac{2}{3}\leq r\leq 1,
  \end{cases}\\
  \partial_r f_4(r) &=
  \begin{cases}
    34992r^3 - 18225r^2 + 2160r, & \text{wenn } 0\leq r\leq\frac{1}{3},\\
    \alpha\left(162r^2 - 162r + 36\right) - \frac{1}{r^2}, & 
    \text{wenn } \frac{1}{3}\leq r\leq \frac{2}{3},\\
    3645r^2 - 6048r + 2592 - 864 - \frac{81}{r^2}, & 
    \text{wenn } \frac{2}{3}\leq r\leq 1,
  \end{cases}\\
  \partial_r u_{3,4}(r) &=
  \begin{cases}
    0, & \text{wenn } 0\leq r\leq\frac{1}{3},\\
    162r^2 - 162r + 36, & 
    \text{wenn } \frac{1}{3}\leq r\leq \frac{2}{3},\\
    0, & \text{wenn } \frac{2}{3}\leq r\leq 1,
  \end{cases}\\
\end{align*}

Mit diesen Informationen kann mit \texttt{computeExactEnergyBV.m} die exakte 
Energie berechnet werden und somit durch eintragen der exakten Energie
in die Variable \texttt{exactEnergy} im Benchmark und setzen der Flag
\texttt{useExactEnergy=true} das Experiment durch anschließendes Ausführen
von \texttt{startAlgorithmCR.m} gestartet werden.

\section{Bilder als Input und Rauschverminderung}


\section{Alternative Kapitelstruktur: Anwendung am Beispiel der Benchmarks}
\subsection{Experimente mit exakter Lösung: Benchmarks 'f01Bench' und 'f02'
Bench}

\subsection{Bilder als Input und Rauschverminderung: 'cameramanBench' und
\ldots}
