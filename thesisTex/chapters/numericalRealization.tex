\section{Hinweise zur Benutzung des Programms}
%\subsection{Aufbau des Programms}
Ziel der für diese Arbeit implementierten Methoden ist die Realisierung von
\Cref{alg:primalDualIteration} für den Solve-Schritt des AFEM-Algorithmus aus
Abbildung \ldots. Wir gehen davon aus, dass der AFEM-Algorithmus und die im
AFEM-Softwarepaket realisierten Methoden sowie deren Datenstrukturen bekannt
sind und verweisen für weitere Details auf \cite{CGKNRR10}.

Die zur korrekten Funktionsweise dieses Programms nötigen Methoden und
Dateien des AFEM-Softwarepakets sind enthalten im Ordner
\texttt{./utils/afemPackage/} sowie \texttt{./utils/geometries/}.

Die ausführbare Methode, welche den AFEM-Algorithmus realisiert, ist
\begin{center}
  \texttt{./nonconforming/startAlgorithmCR.m}.\\ 
\end{center}
Als optionale Eingabe ist dabei ein 
String \texttt{benchmark} möglich. Wird die Methode ohne Eingabe ausgeführt, 
so wird \texttt{benchmark = 'editable'} gesetzt.
Alle nötigen Parameter und Einstellungen für das jeweilige
Experiment werden nach Ausführen von \texttt{startAlgorithmCR} aus der
entsprechenden Benchmark-Datei im Ordner 
\begin{center}
  \texttt{./nonconforming/benchmarks/}
\end{center}
geladen und als Felder des Structure Arrays \texttt{params} übergeben. Für
jedes in dieser Arbeit dokumentierte Experiment verweisen wir in
\Cref{chap:experiments} an entsprechender Stelle auf das dafür benutzte 
Benchmark, welches in \texttt{./nonconforming/benchmarks/} zu finden ist und
somit die Reproduzierbarkeit des Experiments garantiert. Die wählbaren
Paramter und Einstellungen sind in
\begin{center}
  \texttt{./nonconforming/benchmarks/editable.m}
\end{center}
dokumentiert.

Dass die zahlreichen Paramter, die während des Pro\-gramm\-ab\-laufs
über- oder ausgegeben werden müssen, als Felder von
Struc\-ture Ar\-rays gespeichert werden, dient der Modifizierbarkeit des
Programms. 
So haben Korrekturen und Ergänzungen am Programm häufig nur zur Folge, dass
einige \texttt{structs} um Felder ergänzt werden müssen während die 
Methodenköpfe unverändert bleiben können.
\todo{alle Benchmark einstellungen hier in einer Tabelle auflisten und
dokumentieren?}

Das Eingangssignal $f$ und eventuell weitere Funktionen, wie etwa die exakte
Lösung $u$ von \Cref{prob:continuousProblem} und ihre schwache Ableitung
$\nabla u$, welche in einer Benchmark-Datei angegeben werden müssen um sie dem
Programm zu übergeben, sind zu finden in 
\begin{center}
  \texttt{./utils/functions/}.
\end{center}

Ist eine Lösung $u$ von $\Cref{prob:continuousProblem}$ bekannt, so kann die
exakte Energie $E(u)$ approximiert werden mit der Methode
\begin{center}
  \texttt{./nonconforming/computeExactEnergyBV.m}.
\end{center}
Die so berechneten Energien werden gespeichert in 
\begin{center}
  \texttt{./nonconforming/knownExactEnergies/}
\end{center}
und können anschließend manuell in ein Benchmark aufgenommen werden.
\todo{$f=\alpha g$ und weiter Details zur Umwandlung von Bildern? Hier oder
bei Details zu ausgewählten Funktionen?}

Soll als Eingangssignal kein \texttt{function\_handle} sondern ein
Graufarbenbild gegeben werden, so muss es gespeichert werden in 
\begin{center}
  \texttt{./utils/functions/images/}.
\end{center}
Um Dirichlet-Nullranddaten des Bildes zu garantierten, was einem schwarzen Rand
entspricht, kann die Methode 
\begin{center}
  \texttt{./utils/functions/images/addBoundary2image.m}
\end{center}
genutzt werden. Diese fügt einen graduellen Übergang zu schwarzen Rand auf den 
äußeren 25 Pixeln des Bildes hinzu.

Um additives weißes gaußsches Rauschen zu einen Bild hinzuzufügen, kann die
Methode
\begin{center}
  \texttt{./utils/functions/images/addNoise2image.m}
\end{center}
genutzt werden. 

\bigskip
%%%%%%%%%%%%%%%%%%%%%%%%%
cd Realisierungschapter PB

Tabelle 4.2 (PB) artige Übersicht über die benutzten AFEM 
Datenstrukturen, die im Programm nicht extra dokumentiert wurden. Für alle
anderen In- und Outputs verweise auf Dokumentation in den Docstrings



Die Ausführbare Funktionen werden im nächsten Kapitel beschrieben.

%\subsection{Erstellen eines lauffähigen Benchmarks (Minimalbeispeil)}
Beschreibung der wichtigsten Parameter
und Idee hinter structs

Ordner, in denen die Funktionen für rechte Seite, Gradient, exakte
Lösung etc liegen müsssen

Wahrscheinlich flag für flag durchgehen, erklären, welche automatisch gesetzt 
werden u.U., und wann immer nötig sagen, was man vorher machen muss, wo man
Funktionen erstellen muss etc.

fur exakte Lösungs Beispiel usw.
Berechnung der exakten Energie, also alles was nur mehr Möglichkeiten bietet,
Verweis auf die nächste Section (in der dann sagen, welche Flags gesetzt werden 
können)


\subsubsection{\texttt{startAlgorithmCR.m}}
\texttt{test}
\subsubsection{\texttt{computeExactEnergyBV.m}}
\subsubsection{\texttt{addNoise2image.m}}
\subsubsection{\texttt{addBoundary2image.m}}



\section{Mathematische Grundlagen ausgewählter Funktionen}
\subsection{Berechnung der Werte einer CR Funktion auf einem Element in den Knoten}
Sei $T\in\Tcal$ mit $T = \conv\{P_1, P_2, P_3\}$. Die Seiten 
von $T$ seien $F_1 = \conv\{P_1,P_2\}$, $F_2 = \conv\{P_2,P_3\}$ und $F_3 =
\conv\{P_3,P_1\}$. Die Funktion
$\ucr\in\CR^1(\Tcal)$ habe in den Mittelpunkte der Seiten die Werte $u_j =
\ucr(\Mid(F_j))$ für alle $j=1,2,3$. Gesucht sind die Werte in den Knoten
$\ucr(P_1)$, $\ucr(P_2)$ und $\ucr(P_3)$.

Da $\ucr\in\CR^1(\Tcal)$ affin-linear ist, gilt für eine Kante
$F=\conv\{P,Q\}$, dass der Wert $\ucr(\Mid(F))$ von $\ucr$ im Mittlpunkt 
der Kante gegeben ist durch den Mittelwert der Werte von $\ucr$ in $P$ und $Q$.

Somit erhalten wir die drei Gleichungen
\begin{align*}
  u_1 &= \frac{\ucr(P_1)+\ucr(P_2)}{2},  
  &u_2 &= \frac{\ucr(P_2)+\ucr(P_3)}{2},  
  &u_3 &= \frac{\ucr(P_3)+\ucr(P_1)}{2}.
\end{align*}
Sind $u_1$, $u_2$ und $u_3$ bekannt, können wir dieses Gleichungssystem nach 
$\ucr(P_1)$, $\ucr(P_2)$ und $\ucr(P_3)$ lösen und erhalten
\begin{align*}
 \ucr(P_1)&=u_1+u_3-u_2, &\ucr(P_2)&= u_1+u_2-u_3,&\ucr(P_3)&=u_2+u_3-u_1.
\end{align*}

Dies wird realisiert in der Methode \texttt{computeNodeValuesCR4e}.

\subsection{Berechnung der L1 Norm der Sprünge}
Für die Berechnung des Verfeinerungsindikators [verweis auf entsprechende
section] und zur Auswertung der kontinuierlichen Energie $E(\vcr)$ einer 
Crouzeix-Raviart Funktion $\vcr$, deren diskrete Energie $\Enc(\vcr)$
bereits bekannt ist, werden die $L^1$ Normen der Kantensprünge 
$[\vcr]_F$ für alle Kanten $F\in\Fcal$ der Triangulierung benötigt,
wobei für eine Innenkante $F\in\Fcal(\Omega)$, die gemeinsame Kante der
Dreiecke $T_+$ und $T_-$ ist, gilt
$[\vcr]_F\coloneqq (\vcr|_{T_+})|_F-(\vcr|_{T_-})|_F$
und $[\vcr]_F \coloneqq \vcr|_F$ für eine Randkante
$F\in\Fcal(\partial\Omega)$. Die Konvention der Wahl von
$T_+$ und $T_-$ ist hier irrelevant, da wir
zur Berechung von $\Vert [\vcr]_F\Vert_{L^1(\Omega)}$
ausschließlich den Betrag $|[\vcr]_F|$ benötigen.

Da $\vcr\in\CR^1(\Tcal)$,
ist $[\vcr]_F$ affin linear und es gilt $[\vcr]_F(\Mid(F))=0$ für 
alle Innenkanten $F\in\Fcal(\Omega)$ und, falls 
$\vcr\in\CR^1_0(\Tcal)$, auch für alle Randkanten $F\in\Fcal(\partial\Omega)$.

Die folgenden Aussagen gelten also für Innenkanten beliebiger
Crouzeix-Raviart Funktionen, wir beschränken uns aber von nun
an auf Funktionen $\vcr\in\CR^1_0(\Tcal)$.

Betrachten wir also eine beliebige Kante $F\in\Fcal$
mit $F=\conv\{P_1,P_2\}$. 
Wir definieren eine Parametrisierung $\gamma:[0,2]\to\Rbb^2$ von $F$ durch
$\gamma(t)\coloneqq \frac{t}{2}(P_2-P_1)+P_1$. 
Es gilt $|\gamma'|\equiv \frac{1}{2}|P_2-P_1|=\frac{1}{2}|F|$.

Sei außerdem
$p(t)\coloneqq [\vcr]_F(\gamma(t))$. Dann gilt nach
 [cite Wegintegrale] 
\begin{align*}
  \Vert [\vcr]_F\Vert_{L^1(F)} 
  &=
  \int_F |[\vcr]_F|\ds 
  = \int_0^2 |p(t)|\,|\gamma'(t)|\dt
  = \frac{|F|}{2}\int_0^2 |p(t)|\dt\\
  &= \frac{|F|}{2}\left(\int_0^1 |p(t)|\dt + \int_1^2 |p(t)|\dt\right).
\end{align*}

Da $\vcr\in\CR^1_0(\Tcal)$, ist $|p|$ auf $[0,1]$ und $[1,2]$ jeweils
ein Polynom
vom Grad $1$ mit $p(1)=[\vcr]_F(\Mid(F))=0$, womit sich $|p|$ jeweils
explizit ausdrücken lässt durch
$|p|(t)=(1-t)|p|(0)$ für alle $t\in[0,1]$ und 
$|p|(t)=(t-1)|p|(2)$ für alle $t\in[1,2]$.
Die Mittelpunktsregel $\int_a^b f(x)\dx\approx (b-a)f( (a+b)/2)$ [cite] ist
exakt für Polynome vom Grad $1$ und somit gilt
\begin{align*}
  \int_0^1 |p(t)|\dt 
  &= 
  (1-0)|p|\left( \frac{1}{2} \right)
  =
  \frac{|p|(0)}{2}\quad\text{und }\\
  \int_1^2 |p(t)|\dt 
  &= 
  (2-1)|p|\left( \frac{3}{2} \right)
  =
  \frac{|p|(2)}{2}.
\end{align*}

Somit erhalten wir insgesamt 
\begin{align*}
  \Vert [\vcr]_F\Vert_{L^1(F)} 
  &=
  \frac{|F|}{2}\left(\frac{|p|(0)}{2} + \frac{|p|(2)}{2}\right)
  =
  \frac{|F|}{4}(|p|(0)+|p|(2))\\
  &= 
  \frac{|F|}{4}\big(|[\vcr]_F|(P_1)+|[\vcr]_F|(P_2)\big),
\end{align*}
beziehungsweise 
  $\Vert [\vcr]_F\Vert_{L^1(F)} =
  \frac{|F|}{4}\big(|\vcr|(P_1)+|\vcr|(P_2)\big)$ für eine Randkanten
  $F\in\Fcal(\partial\Omega)$.

Diese Berechnung ist realisiert durch die 
Funktionen \texttt{computeBlaJumps}, die die absoluten Sprünge
in den Endpunkte einer Kante berchnet, \texttt{computeAbsJumps}, die \ldots,
und \texttt{computeL1NormOfJumps}, die schließlich die $L^1$ Norm aller
Kantensprünge berechnet\ldots.



\subsection{Implementation der GLEB}

\subsection{Implementation des Refinement Indicators}

\subsection{Implementaition der exakten Energie Berechnung}


