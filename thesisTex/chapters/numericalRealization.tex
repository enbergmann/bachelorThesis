\todo{Algorithmus und Konvergenzbeweis als eigenes Kapitel 5, Implemtierung
dann als Kapitel 6}

Mehr zur Herleitung des Algorithmus, siehe BartelsBVPaper S. 1163, schwache
Formulierung der vorletzten Gleichung

Schreibe sowas wie ,Die GLeichungen basieren auf der primalen dualen
Formulierung des Problems, auf einem descent flow \ldots usw,
Details finden sich in barBV, wobei Gleichung (Glg im Alg) die schwache
Formulierung von (Glg in Bartels bzw der Herleitung)'. d.h. Stichworte
abklappern, damit Leute mit Ahnung wissen was los ist, aber für Details
auf andere Leute zeigen.

Weitere Details zur Herleitung auf S. 1168, da noch Details rausschreiben
aber nur grob aber dafür darauf verweisen für die Herleitung. Nur das 
nötigste hier.


Stopping criteria sind in Punkt 6.2




\section{Algorithmus}
\todo[inline]{Irgendwo, wahrscheinlich bei ,,alles zu $\CR_0$'' muss noch
$\anc(u,v)\coloneqq\int_\Omega\gradnc u\cdot\gradnc v\dx$ erwähnt werden (und
warum das ein SP ist muss angerissen werden, Stichwort Friedrichs Ungleichung)} 

Für unsere Formulierung \Cref{prob:discreteProblem} nutzen wir \cite[S. 314,
Algorithm 10.1]{Bar15} unter Beachtung von \cite[S. 314, Remark 10.11]{Bar15}
als Algorithmus als iterativen Löser und benutzen als inneres Produkt $\anc$
(definiert in Kapitel \ldots hier in dieser Arbeit).
Weitere Details dazu finden sich in \cite[S. 118-121]{Bar15}.
\todo[inline]{Beim Zitieren z.B. 'Remark' lassen, weil es in Bartels so heißt,
oder das lieber übersetzen? Außerdem natürlich, passt das so als Einleitung 
für den Alg?}
\begin{algorithm}[Primale-Duale Iteration]
  \label{alg:primalDualIteration}
\begin{algorithmic}\\
  \Require $u_0\in\textup{CR}_0^1(\mathcal{T}),$
  $\Lambda_0\in \Pbb_0(\mathcal{T};
  \overline{B(0,1)}),\tau>0$  \\
  Initialisiere $v_0\coloneqq 0$ in $\textup{CR}^1_0(\mathcal T)$.
  \For{$j = 1,2,\dots$}
  \begin{equation}
    \label{eq:primalDualAlgUj}
    \tilde{u}_j\coloneqq u_{j-1}+\tau v_{j-1},
  \end{equation}
  \begin{equation}
    \label{eq:primalDualAlgLambdaJ}
    \Lambda_j\coloneqq
    (\Lambda_{j-1}+\tau\nabla_{\textup{NC}} \tilde{u}_j)/
      (\operatorname{max}\{1,|\Lambda_{j-1}+\tau\nabla_{\textup{NC}}\tilde{u}_j|\}),
  \end{equation}
      \State
  \State bestimme $u_j\in\textup{CR}^1_0(\mathcal{T})$
  als Lösung des linearen Gleichungssystems
  \begin{align}
    \label{eq:linSysPrimalDualAlg}
    \frac{1}{\tau}&a_{\textup{NC}}(u_j,\bullet)+\alpha(u_j,\bullet)_{L^2(\Omega)}
    \notag \\
    &=
    \frac{1}{\tau}a_{\textup{NC}}(u_{j-1},\bullet) + (f,\bullet)_{L^2(\Omega)}
    - (\Lambda_j,\nabla_{\textup{NC}}\bullet)_{L^2(\Omega)} 
  \end{align}
  \State in $\CR^1_0(\mathcal{T})$, \\
  \begin{equation*}
    v_j\coloneqq(u_j-u_{j-1})/\tau.
  \end{equation*}
  \EndFor
  \Ensure Folge $(u_j,\Lambda_j)_{j\in\mathbb N}$ in
  $\CR^1_0(\mathcal{T})\times
  \Pbb_0(\mathcal{T};\overline{B(0,1)})$   
  \end{algorithmic}
\end{algorithm}

\begin{theorem}
  Sei $\ucr\in \CR^1_0(\Tcal)$ Lösung von \Cref{prob:discreteProblem}
  und $\bar\Lambda\in\operatorname{sign}(\gradnc\ucr)$ erfülle Eigenschaft 
  \textit{(ii)} in \Cref{thm:discProbCharacterizationOfDiscreteSolutions}.
  Falls $0 < \tau \leq 1$, dann konvergieren die Iterate $(u_j)_{j\in\Nbb}$ von
  \Cref{alg:primalDualIteration}
  gegen $\ucr.$
\end{theorem}

\begin{proof}
  Der Beweis folgt einer Skizze von Prof. Carstensen.
  
  Seien $\tilde{u}_j$, $v_j$ und $\Lambda_j$ definiert wie in
  \Cref{alg:primalDualIteration}.
  Definiere außerdem $e_j \coloneqq \ucr-u_j$ und $E_j\coloneqq
  \bar\Lambda-\Lambda_j$. 

  Testen wir nun \eqref{eq:linSysPrimalDualAlg} mit $e_j$, erhalten wir
  \begin{align*}
    \anc(v_j,e_j) + \alpha(u_j,e_j)_{L^2(\Omega)} 
    + (\Lambda_j,\gradnc e_j)_{L^2(\Omega)}
    = 
    (f,e_j)_{L^2(\Omega)}.
  \end{align*}
  Äquivalent dazu ist, da $\ucr$ \Cref{eq:discreteMultiplierL2Equality} löst, 
  \begin{align}
    \label{eq:convProofE}
    \anc(v_j,e_j) &= 
    \alpha(\ucr-u_j,e_j)_{L^2(\Omega)} 
    + (\bar\Lambda-\Lambda_j,\gradnc e_j)_{L^2(\Omega)} \notag\\
    &= 
    \alpha\Vert e_j\Vert_{L^2(\Omega)}^2
    + (E_j,\gradnc e_j)_{L^2(\Omega)}.
  \end{align}
  Sei $\mu_j\coloneqq \max\{1,|\Lambda_{j-1}+\tau
  \gradnc \tilde{u}_j|\}$.
  Nutzen wir \eqref{eq:primalDualAlgLambdaJ} erhalten wir damit
  \begin{align}
    \label{eq:convProofA}
    \Lambda_{j-1}-\Lambda_j+\tau \gradnc \tilde{u}_j 
    = (\mu_j-1)\Lambda_j \quad\text{fast überall in }\Omega.
  \end{align}
  Für fast alle $x\in\Omega$ liefert die CSU, da $|\bar\Lambda|\leq 1$ fast
  überall in $\Omega$,
  $\Lambda_j(x)\cdot\bar\Lambda(x)\leq|\Lambda_j(x)|$ und damit folgt 
  aus \Cref{eq:primalDualAlgLambdaJ} und einer einfachen Fallunterscheidung
  zwischen $1\geq |\Lambda_{j-1}+\tau\gradnc \tilde{u}_j|$ und
  $1< |\Lambda_{j-1}+\tau\gradnc \tilde{u}_j|$,
  dass $(1-|\Lambda_j(x)|)(\mu_j(x)-1)=0$.
  Testen wir nun \eqref{eq:convProofA} mit $E_j$, erhalten wir 
  unter Nutzung von $\mu_j\geq 1$ und der CSU damit
  \begin{align*}
    ( \Lambda_{j-1}-\Lambda_j+\tau\gradnc \tilde{u}_j,
    E_j)_{L^2(\Omega)}
    &= 
    ( (\mu_j-1)\Lambda_j,\bar\Lambda-\Lambda_j)_{L^2(\Omega)}\\
    &=
    \int_\Omega
    (\mu_j-1)(\Lambda_j\cdot\bar\Lambda-\Lambda_j\cdot\Lambda_j)\,\mathrm dx\\
    &\leq
    \int_\Omega (\mu_j-1)(|\Lambda_j|-|\Lambda_j|^2)\,\mathrm dx\\
    &=
    \int_\Omega |\Lambda_j|
    (1-|\Lambda_j|)(\mu_j-1)\,\mathrm dx \\
    &=
    \int_\Omega |\Lambda_j|\cdot
    0\,\mathrm dx =0.
  \end{align*}
  Damit und mit $\Lambda_{j-1}-\Lambda_j=E_j-E_{j-1}$, 
  $\tilde{u}_j=u_{j-1}+\tau v_{j-1}=u_{j-1}+u_{j-1}-u_{j-2}=
  u_{j-1}-(e_{j-1}-e_{j-2})$
  für $j\geq 2$ und der Konvention $e_{-1}\coloneqq e_0$ für $j=1$ erhalten wir
  insgesamt
  \begin{align}
    \label{eq:convProofB}
    \left(\frac{E_j-E_{j-1}}{\tau}+ \gradnc u_{j-1}-\gradnc
    (e_{j-1}-e_{j-2}),E_j\right)_{L^2(\Omega)}\leq 0\quad\text{für alle }
    j\in\Nbb.
  \end{align}
  Falls $|\gradnc \ucr|\neq 0$, gilt somit zusammen mit
  der CSU, $\bar\Lambda\in\sign\gradnc\ucr$ und $|\Lambda_j|\leq 1$,  dass
  \begin{align*}
    \gradnc\ucr\cdot E_j 
    &=
    \gradnc\ucr\cdot\bar\Lambda - \gradnc\ucr\cdot\Lambda_j\\
    &\geq 
    \gradnc\ucr\cdot\bar\Lambda - |\gradnc\ucr||\Lambda_j| \\
    &=
    |\gradnc\ucr|^2/|\gradnc\ucr|-|\gradnc\ucr||\Lambda_j| \\
    &= 
    |\gradnc\ucr|(1-|\Lambda_j|)\\
    &\geq
    0. 
  \end{align*}
  Falls $|\gradnc\ucr|=0$, gilt diese Ungleichung ebenfalls.
  Daraus folgt
  \begin{align}
    \label{eq:convProofC}
    (\gradnc\ucr,E_j)_{L^2(\Omega)}=\int_\Omega \gradnc\ucr\cdot E_j\dx\geq 0.
  \end{align}
  Mit \eqref{eq:convProofB} und \eqref{eq:convProofC} folgt nun
  \begin{align*}
    \left( \frac{E_j-E_{j-1}}{\tau}+ \gradnc u_{j-1}
    -\nabla_\nc(e_{j-1}-e_{j-2}),E_j\right)_{L^2(\Omega)}
    \leq
    (\gradnc\ucr,E_j)_{L^2(\Omega)}
  \end{align*}
  für alle $j\in\Nbb$, was nach Definition von $e_{j-1}$
  äquivalent ist zu
  \begin{align}
    \label{eq:convProofD}
    \left( \frac{E_j-E_{j-1}}{\tau} 
    -\gradnc(2e_{j-1}-e_{j-2}),E_j\right)_{L^2(\Omega)}\leq 0.
  \end{align}
  Unter Nutzung von $-v_j=(e_j-e_{j-1})/\tau$, \eqref{eq:convProofE}, $\tau>0$
  und \eqref{eq:convProofD} erhalten wir 
  \begin{align*}
    &\vvvert e_j \vvvert^2_\nc   -
    \vvvert e_{j-1}\vvvert_\nc^2 +
    \Vert E_j \Vert_{L^2(\Omega)}^2 - \Vert E_{j-1}\Vert_{L^2(\Omega)}^2 +
    \vvvert e_j-e_{j-1}\vvvert_\nc^2 +
    \Vert E_j - E_{j-1} \Vert_{L^2(\Omega)}^2\\
    &\quad =
    2a_\nc(e_j,e_j-e_{j-1}) + 2(E_j,E_j-E_{j-1})_{L^2(\Omega)}\\
    &\quad =
    -2\tau a_\nc(e_j,v_j) + 2(E_j,E_j-E_{j-1})_{L^2(\Omega)}\\
    &\quad =
    -2\tau\alpha\Vert e_j\Vert_{L^2(\Omega)}^2 + 2\tau\left(E_j,
    -\nabla_\nc e_j+\frac{E_j-E_{j-1}}{\tau}\right)_{L^2(\Omega)} \\
    &\quad \leq
    -2\tau\alpha\Vert e_j\Vert_{L^2(\Omega)}^2 + 2\tau\left(E_j,
    -\nabla_\nc e_j+\frac{E_j-E_{j-1}}{\tau}\right)_{L^2(\Omega)}\\ 
    &\quad\quad -2\tau\left( \frac{E_j-E_{j-1}}{\tau}
    -\gradnc(2e_{j-1}-e_{j-2}),E_j\right)_{L^2(\Omega)}\\
    &\quad =
    -2\tau\alpha\Vert e_j\Vert_{L^2(\Omega)}^2 - 
    2\tau\big(E_j,\gradnc(e_j-2e_{j-1}+e_{j-2})\big)_{L^2(\Omega)}.
  \end{align*}
  Für jedes $J\in\Nbb$ führt die Summation über $j=1,\ldots,J$ und eine
  Äquivalenz\-umfomung zu
  \begin{align}
    \label{eq:convProofF}
    &\vvvert e_J \vvvert^2_\nc +\Vert E_J \Vert_{L^2(\Omega)}^2 
    +\sum_{j=1}^J\left(\vvvert e_j-e_{j-1} \vvvert_\nc^2 + 
    \Vert E_j-E_{j-1}\Vert_{L^2(\Omega)}^2\right)\notag \\
    &\quad \leq 
    \vvvert e_0 \vvvert_\nc^2 + \Vert E_0 \Vert_{L^2(\Omega)}^2 
    -2\tau\alpha\sum_{j=1}^J \Vert e_j\Vert^2_{L^2(\Omega)} \\
    &\quad\quad
    -2\tau \sum_{j=1}^J\big(E_j,\gradnc
    (e_j-2e_{j-1}+e_{j-2})\big)_{L^2(\Omega)}.\notag
  \end{align}
  Dabei lässt sich die letzt Summe, unter Beachtung von $e_{-1}=e_0$, umformen
  zu
  \begin{align*}
    &\sum_{j=1}^J\big(E_j,\gradnc
    (e_j-2e_{j-1}+e_{j-2})\big)_{L^2(\Omega)}\\
    &\quad=\sum_{j=1}^J(E_j,\gradnc(e_j-e_{j-1}))_{L^2(\Omega)}
    -
    \sum_{j=0}^{J-1}(E_{j+1},\gradnc(e_j-e_{j-1}))_{L^2(\Omega)} \\
    &\quad = 
    \sum_{j=1}^{J-1} 
    \big(E_j-E_{j+1},\gradnc(e_j-e_{j-1})\big)_{L^2(\Omega)}
    +(E_J,\gradnc(e_J-e_{J-1}))_{L^2(\Omega)}\\
    &\quad\quad 
    - (E_1, \gradnc(e_0-e_{-1}))_{L^2(\Omega)} \\
    &\quad = 
    \sum_{j=1}^{J-1} 
    \big(E_j-E_{j+1},\gradnc(e_j-e_{j-1})\big)_{L^2(\Omega)}
    +(E_J,\gradnc(e_J-e_{J-1}))_{L^2(\Omega)}
  \end{align*}
  und da die linke Seite von
  \eqref{eq:convProofF} nicht negativ ist, gilt damit
  für jedes $0<\tau\leq 1$, dass
  \begin{align*}
    &\tau\left(\vvvert e_J \vvvert^2_\nc +\Vert E_J \Vert_{L^2(\Omega)}^2 
    +\sum_{j=1}^J\left(\vvvert e_j-e_{j-1} \vvvert_\nc^2 + 
    \Vert E_j-E_{j-1}\Vert_{L^2(\Omega)}^2\right)\right) \\
    &\quad \leq 
    \vvvert e_0 \vvvert_\nc^2 + \Vert E_0 \Vert_{L^2(\Omega)}^2 
    -2\tau\alpha\sum_{j=1}^J \Vert e_j\Vert^2_{L^2(\Omega)} \\
    &\quad\quad
    -2\tau\left( 
    \sum_{j=1}^{J-1} 
    (E_j-E_{j+1},\gradnc(e_j-e_{j-1}))_{L^2(\Omega)}
    +(E_J,\gradnc(e_J-e_{J-1}))_{L^2(\Omega)}\right).
  \end{align*}
  Division durch $\tau$ ergibt
  \begin{align}
    \label{eq:convProofG}
    &\vvvert e_J \vvvert^2_\nc +\Vert E_J \Vert_{L^2(\Omega)}^2 
    +\sum_{j=1}^J\left(\vvvert e_j-e_{j-1} \vvvert_\nc^2 + 
    \Vert E_j-E_{j-1}\Vert_{L^2(\Omega)}^2\right) \notag\\
    &\quad \leq 
    \tau^{-1}(\vvvert e_0 \vvvert_\nc^2 + \Vert E_0 \Vert_{L^2(\Omega)}^2 )
    -2\alpha\sum_{j=1}^J \Vert e_j\Vert^2_{L^2(\Omega)} \\
    &\quad\quad
    -2 \sum_{j=1}^{J-1} (E_j-E_{j+1},\gradnc(e_j-e_{j-1}))_{L^2(\Omega)}
    -2(E_J,\gradnc(e_J-e_{J-1}))_{L^2(\Omega)}.\notag
  \end{align}

%  \begin{align*}
%    &2\tau\sum_{j=1}^J(E_j,\gradnc(-e_j+e_{j-1}))_{L^2(\Omega)} +
%    2\tau\sum_{j=0}^{J-1}(E_{j+1},\gradnc(e_j-e_{j-1}))_{L^2(\Omega)}\\
%    &\quad \le
%    2\sum_{j=1}^J(E_j,\gradnc(-e_j+e_{j-1}))_{L^2(\Omega)} +
%    2\sum_{j=0}^{J-1}(E_{j+1},\gradnc(e_j-e_{j-1}))_{L^2(\Omega)}\\
%    &\quad =
%    2\sum_{j=1}^J(E_j,\gradnc(-e_j+e_{j-1}))_{L^2(\Omega)} 
%    +
%    2\sum_{j=1}^{J-1}(E_{j+1},\gradnc(e_j-e_{j-1}))_{L^2(\Omega)} 
%    \tag{since $e_{-1}\coloneqq e_0$}\\
%    &\quad =
%    2\sum_{j=1}^{J-1}(E_{j+1}-E_j,\gradnc(e_j-e_{j-1}))_{L^2(\Omega)}
%    -2(E_{J},\gradnc(e_J-e_{J-1}))_{L^2(\Omega)}
%    .
%  \end{align*}

  Schließlich ergibt eine Abschätzung unter Nutzung von \eqref{eq:convProofG}, 
  dass
  \begin{align*}
    2\alpha\sum_{j=1}^J\Vert e_j\Vert^2_{L^2(\Omega)} 
    &\leq
    2\alpha\sum_{j=1}^J\Vert e_j\Vert^2_{L^2(\Omega)}\\
    &\quad
    +\Vert E_J + \gradnc(e_J-e_{J-1}) \Vert_{L^2(\Omega)}^2 
    + \vvvert e_J \vvvert^2_\nc 
    + \Vert E_1 - E_0 \Vert^2_{L^2(\Omega)} \\
    &\quad 
    + \sum_{j=1}^{J-1}  
      \Vert \gradnc(e_j-e_{j-1}) - (E_{j+1} - E_j ) \Vert^2_{L^2(\Omega)} \\
    & = 
    2\alpha\sum_{j=1}^J\Vert e_j\Vert^2_{L^2(\Omega)}\\
    &\quad 
    +\vvvert e_J \vvvert^2_\nc + \Vert E_J \Vert_{L^2(\Omega)}^2 
    + \sum_{j=1}^J ( \vvvert e_j-e_{j-1} \vvvert^2_\nc
    + \Vert E_j - E_{j-1} \Vert^2_{L^2(\Omega)} )\\
    &\quad
    + 2\sum_{j=1}^{J-1}(E_j-E_{j+1},\gradnc(e_j-e_{j-1}))_{L^2(\Omega)}\\
    &\quad 
    + 2(E_{J},\gradnc(e_J-e_{J-1}))_{L^2(\Omega)} \\
    &\leq
    \tau^{-1}(\vvvert e_0\vvvert^2_\nc + \Vert E_0\Vert^2_{L^2(\Omega)}).
  \end{align*}

  Das zeigt, dass
  $\sum_{j=1}^\infty \Vert e_j\Vert _{L^2(\Omega)}^2$ nach oben beschränkt ist,
  was impliziert, dass $\Vert e_j\Vert_{L^2(\Omega)}\rightarrow 0$
  für $j\rightarrow \infty$.
\end{proof}


\section{Seitennummerierungskonvention lokal}
welche Funktionen haben welche, an 
welche Stellen wird also umnummeriert

\section{Aufstellung des zu lösenden LGS}
Aufstellung der Gradienten etc.
 
\section{Berechnung der Werte einer CR Funktion auf einem Element in den Knoten}
Sei $T\in\Tcal$ mit $T = \conv\{P_1, P_2, P_3\}$. Die Seiten 
von $T$ seien $F_1 = \conv\{P_1,P_2\}$, $F_2 = \conv\{P_2,P_3\}$ und $F_3 =
\conv\{P_3,P_1\}$. Die Funktion
$\ucr\in\CR^1(\Tcal)$ habe in den Mittelpunkte der Seiten die Werte $u_j =
\ucr(\Mid(F_j))$ für alle $j=1,2,3$. Gesucht sind die Werte in den Knoten
$\ucr(P_1)$, $\ucr(P_2)$ und $\ucr(P_3)$.

Da $\ucr\in\CR^1(\Tcal)$ affin-linear ist, gilt für eine Kante
$F=\conv\{P,Q\}$, dass der Wert $\ucr(\Mid(F))$ von $\ucr$ im Mittlpunkt 
der Kante gegeben ist durch den Mittelwert der Werte von $\ucr$ in $P$ und $Q$.

Somit erhalten wir die drei Gleichungen
\begin{align*}
  u_1 &= \frac{\ucr(P_1)+\ucr(P_2)}{2},  
  &u_2 &= \frac{\ucr(P_2)+\ucr(P_3)}{2},  
  &u_3 &= \frac{\ucr(P_3)+\ucr(P_1)}{2}.
\end{align*}
Sind $u_1$, $u_2$ und $u_3$ bekannt, können wir dieses Gleichungssystem nach 
$\ucr(P_1)$, $\ucr(P_2)$ und $\ucr(P_3)$ lösen und erhalten
\begin{align*}
 \ucr(P_1)&=u_1+u_3-u_2, &\ucr(P_2)&= u_1+u_2-u_3,&\ucr(P_3)&=u_2+u_3-u_1.
\end{align*}

Dies wird realisiert in der Methode \texttt{computeNodeValuesCR4e}.

\section{Berechnung der L1 Norm der Sprünge}
Für die Berechnung des Verfeinerungsindikators [verweis auf entsprechende
section] und zur Auswertung der kontinuierlichen Energie $E(\vcr)$ einer 
Crouzeix-Raviart Funktion $\vcr$, deren diskrete Energie $\Enc(\vcr)$
bereits bekannt ist, werden die $L^1$ Normen der Kantensprünge 
$[\vcr]_F$ für alle Kanten $F\in\Fcal$ der Triangulierung benötigt,
wobei für eine Innenkante $F\in\Fcal(\Omega)$, die gemeinsame Kante der
Dreiecke $T_+$ und $T_-$ ist, gilt
$[\vcr]_F\coloneqq (\vcr|_{T_+})|_F-(\vcr|_{T_-})|_F$
und $[\vcr]_F \coloneqq \vcr|_F$ für eine Randkante
$F\in\Fcal(\partial\Omega)$. Die Konvention der Wahl von
$T_+$ und $T_-$ ist hier irrelevant, da wir
zur Berechung von $\Vert [\vcr]_F\Vert_{L^1(\Omega)}$
ausschließlich den Betrag $|[\vcr]_F|$ benötigen.

Da $\vcr\in\CR^1(\Tcal)$,
ist $[\vcr]_F$ affin linear und es gilt $[\vcr]_F(\Mid(F))=0$ für 
alle Innenkanten $F\in\Fcal(\Omega)$ und, falls 
$\vcr\in\CR^1_0(\Tcal)$, auch für alle Randkanten $F\in\Fcal(\partial\Omega)$.

Die folgenden Aussagen gelten also für Innenkanten beliebiger
Crouzeix-Raviart Funktionen, wir beschränken uns aber von nun
an auf Funktionen $\vcr\in\CR^1_0(\Tcal)$.

Betrachten wir also eine beliebige Kante $F\in\Fcal$
mit $F=\conv\{P_1,P_2\}$. 
Wir definieren eine Parametrisierung $\gamma:[0,2]\to\Rbb^2$ von $F$ durch
$\gamma(t)\coloneqq \frac{t}{2}(P_2-P_1)+P_1$. 
Es gilt $|\gamma'|\equiv \frac{1}{2}|P_2-P_1|=\frac{1}{2}|F|$.

Sei außerdem
$p(t)\coloneqq [\vcr]_F(\gamma(t))$. Dann gilt nach
 [cite Wegintegrale] 
\begin{align*}
  \Vert [\vcr]_F\Vert_{L^1(F)} 
  &=
  \int_F |[\vcr]_F|\ds 
  = \int_0^2 |p(t)|\,|\gamma'(t)|\dt
  = \frac{|F|}{2}\int_0^2 |p(t)|\dt\\
  &= \frac{|F|}{2}\left(\int_0^1 |p(t)|\dt + \int_1^2 |p(t)|\dt\right).
\end{align*}

Da $\vcr\in\CR^1_0(\Tcal)$, ist $|p|$ auf $[0,1]$ und $[1,2]$ jeweils
ein Polynom
vom Grad $1$ mit $p(1)=[\vcr]_F(\Mid(F))=0$, womit sich $|p|$ jeweils
explizit ausdrücken lässt durch
$|p|(t)=(1-t)|p|(0)$ für alle $t\in[0,1]$ und 
$|p|(t)=(t-1)|p|(2)$ für alle $t\in[1,2]$.
Die Mittelpunktsregel $\int_a^b f(x)\dx\approx (b-a)f( (a+b)/2)$ [cite] ist
exakt für Polynome vom Grad $1$ und somit gilt
\begin{align*}
  \int_0^1 |p(t)|\dt 
  &= 
  (1-0)|p|\left( \frac{1}{2} \right)
  =
  \frac{|p|(0)}{2}\quad\text{und }\\
  \int_1^2 |p(t)|\dt 
  &= 
  (2-1)|p|\left( \frac{3}{2} \right)
  =
  \frac{|p|(2)}{2}.
\end{align*}

Somit erhalten wir insgesamt 
\begin{align*}
  \Vert [\vcr]_F\Vert_{L^1(F)} 
  &=
  \frac{|F|}{2}\left(\frac{|p|(0)}{2} + \frac{|p|(2)}{2}\right)
  =
  \frac{|F|}{4}(|p|(0)+|p|(2))\\
  &= 
  \frac{|F|}{4}\big(|[\vcr]_F|(P_1)+|[\vcr]_F|(P_2)\big),
\end{align*}
beziehungsweise 
  $\Vert [\vcr]_F\Vert_{L^1(F)} =
  \frac{|F|}{4}\big(|\vcr|(P_1)+|\vcr|(P_2)\big)$ für eine Randkanten
  $F\in\Fcal(\partial\Omega)$.

Diese Berechnung ist realisiert durch die 
Funktionen \texttt{computeBlaJumps}, die die absoluten Sprünge
in den Endpunkte einer Kante berchnet, \texttt{computeAbsJumps}, die \ldots,
und \texttt{computeL1NormOfJumps}, die schließlich die $L^1$ Norm aller
Kantensprünge berechnet\ldots.



\section{Implementation der GLEB}

\section{Implementation des Refinement Indicators}

\section{Implementaition der exakten Energie Berechnung}
