\todo[inline]{Wenigstens sagen worauf der basiert (kann aus Bartels übernommen
werden), vorher nochmal prüfen, ob CC und Bartels Alg wirklich gleich sind, 
ansonsten kann mglw trotzdem ,basiert' gessagt werden}
\begin{algorithm}[Primale-Duale Iteration]
  \label{alg:primalDualIteration}
\begin{algorithmic}\\
  \Require $u_0\in\textup{CR}_0^1(\mathcal{T}),$
  $\Lambda_0\in P_0(\mathcal{T};
  \overline{B(0,1)}),\tau>0$  \\
  Initialisiere $v_0\coloneqq 0$ in $\textup{CR}^1_0(\mathcal T)$.
  \For{$j = 1,2,\dots$}
  \begin{equation}
    \label{equ:primalDualAlgUj}
    \tilde{u}_j\coloneqq u_{j-1}+\tau v_{j-1},
  \end{equation}
  \begin{equation}
    \label{equ:primalDualAlgLambdaJ}
    \Lambda_j\coloneqq
    (\Lambda_{j-1}+\tau\nabla_{\textup{NC}} \tilde{u}_j)/
      (\operatorname{max}\{1,|\Lambda_{j-1}+\tau\nabla_{\textup{NC}}\tilde{u}_j|\}),
  \end{equation}
      \State
  \State bestimme $u_j\in\textup{CR}^1_0(\mathcal{T})$
  als Lösung des linearen Gleichungssystems
  \begin{align}
    \label{equ:linSysPrimalDualAlg}
    \frac{1}{\tau}&a_{\textup{NC}}(u_j,\bullet)+\alpha(u_j,\bullet)_{L^2(\Omega)}
    \notag \\
    &=
    \frac{1}{\tau}a_{\textup{NC}}(u_{j-1},\bullet) + (f,\bullet)_{L^2(\Omega)}
    - (\Lambda_j,\nabla_{\textup{NC}}\bullet)_{L^2(\Omega)} 
  \end{align}
  \State in $\textup{CR}^1_0(\mathcal{T})$, \\
  \begin{equation*}
    v_j\coloneqq(u_j-u_{j-1})/\tau.
  \end{equation*}
  \EndFor
  \Ensure Folge $(u_j,\Lambda_j)_{j\in\mathbb N}$ in
  $\textup{CR}^1_0(\mathcal{T})\times
  P_0(\mathcal{T};\overline{B(0,1)})$   
  \end{algorithmic}
\end{algorithm}

\begin{theorem}
  Sei $\ucr\in \CR^1_0(\Tcal)$ Lösung von \Cref{prob:discreteProblem}
  und $\bar\Lambda\in\operatorname{sign}(\gradnc\ucr)$.
  Falls $0 < \tau \leq 1$, dann konvergieren die Iterate des Algorithmus 
  \Cref{alg:primalDualIteration}
  gegen $\ucr.$
\end{theorem}
