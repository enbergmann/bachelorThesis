\section{Hinweise zur Benutzung des Programms}
Ziel der für diese Arbeit implementierten Methoden ist die Realisierung von
\Cref{alg:primalDualIteration} im Solve-Schritt des AFEM-Algorithmus aus
\Cref{fig:afemLoop}. 
\begin{figure}
  \centering
  \tikzstyle{block} = [rectangle, text width=5em, text centered, rounded corners,
  thick, draw = black]
\tikzstyle{line} = [draw = black, thick, -Triangle, rounded corners]

\begin{tikzpicture}[node distance = 3.5cm]
	%Placing nodes
  \node [block] (solve) {\texttt{Solve}};
  \node [block, right of=solve] (estimate) {\texttt{Estimate}};
  \node [block, right of=estimate] (mark) {\texttt{Mark}};
  \node [block, right of=mark] (refine) {\texttt{Refine}};
	
	\path [line] (solve) -- (estimate);
	\path [line] (estimate) -- (mark);
	\path [line] (mark) -- (refine);
	\path [line] (refine) |- (0,-1) -| (solve);
\end{tikzpicture}

  \caption{AFEM-Schleife}
  \label{fig:afemLoop}
\end{figure}
Wir gehen davon aus, dass dieser und die im AFEM-Softwarepaket \cite{Car09}
realisierten Methoden sowie deren Datenstrukturen bekannt sind und verweisen
für weitere Details auf \cite{CGKNRR10}. 
Im Estimate-Schritt nutzen wir anstelle eines Fehlerschätzers den
Verfeinerungsindikator aus \Cref{def:refinementIndicator}.
Alle in diesem Kapitel aufgeführten Pfadnamen sind relative Pfade in Relation
zum Ordner \texttt{code} im Anhang.
Die zur korrekten Funktionsweise dieses Programms nötigen Methoden und
Dateien des AFEM-Softwarepakets sind enthalten in den Ordnern
\begin{center}
    \texttt{./utils/afemPackage/} sowie \texttt{./utils/geometries/}.
\end{center}
Dort ist eine veränderte Version der Methode \texttt{plotTriangulation}
gespeichert, die nun den Plot einer Triangulierung als Schwarz-Weiß-Bild ohne
Titel erzeugt.
Alle Ein- und Ausgabeparameter der im Rahmen dieser Arbeit implementierten
Methoden sind in den entsprechenden Dateien im Anhang detailliert dokumentiert. 
Ausgenommen davon sind die bekannten Datenstrukturen aus dem
AFEM-Softwarepaket, welche die Triangulierung beschreiben. 
Diese werden lediglich aufgelistet.
Entwickelt und getestet wurde das Programm in Matlab R2017b.
Die mathematischen Grundlagen für die Realisierung einiger Methoden diskutieren
wir in \Cref{sec:mathematicalBasicsForMethods}.

Das Hauptprogramm, das heißt die ausführbare Methode, welche den
AFEM\--Al\-go\-rith\-mus realisiert, ist
\begin{center}
  \texttt{./nonconforming/startAlgorithmCR.m}.
\end{center}
Beim Ausführen dieser Methode sollte das aktuelle Verzeichnis 
\texttt{./nonconforming/} sein.
Als optionaler Eingabeparameter ist dabei ein nichtleeres \texttt{char array}
mit genau einer Zeile möglich, das den Namen einer Matlab-Funktion enthält. 
Diese muss die Parameter und Einstellungen für ein Experiment als Felder eines
Structure Arrays zurückgegeben und in 
\begin{center}
  \texttt{./nonconforming/benchmarks/}
\end{center}
gespeichert sein.
Als Muster für eine solche Bechmark-Datei dient
\begin{center}
  \texttt{./nonconforming/benchmarks/editable.m}.
\end{center}
Wird das Hauptprogramm ohne Übergabe eines Parameters ausgeführt, nutzt
es den Standardwert \texttt{benchmark = 'editable'}.
Für jedes in dieser Arbeit dokumentierte Experiment verweisen wir in
\Cref{chap:experiments} an entsprechender Stelle auf das dafür benutzte
Benchmark, welches in \texttt{./nonconforming/benchmarks/} hinterlegt ist und
somit die Reproduzierbarkeit des Experiments garantiert. 
Eine Übersicht über die in einer Benchmark-Datei wählbaren Parameter ist
zu finden in den Tabellen \ref{tab:paramsMisc} -- \ref{tab:paramsDoc}
sowie über die definierbaren Funktionen in \Cref{tab:paramsFunctions}.
Details zu den Datentypen sind in \texttt{editable.m} aufgeführt.
Dass die zahlreichen Parameter, die während des Pro\-gramm\-ab\-laufs
über- oder ausgegeben werden müssen, als Felder von
Struc\-ture Ar\-rays gespeichert werden, dient der Modifizierbarkeit des
Programms. 
So haben Änderungen am Programm häufig nur zur Folge, dass einige
\texttt{structs} um Felder ergänzt werden müssen, während Methodenköpfe
unverändert bleiben können.
Das Eingangssignal $f$ und eventuell weitere Funktionen, wie etwa die exakte
Lösung $u$ von \Cref{prob:continuousProblem} oder die schwache Ableitung
$\nabla f$ eines schwach differenzierbaren Eingangssignals, müssen in der
Benchmark-Datei definiert werden, um sie dem Programm zu übergeben. 
Die für die Experimente in \Cref{chap:experiments} genutzten Funktionen sind zu
finden in 
\begin{center}
  \texttt{./utils/functions/}.
\end{center}
Ist eine Lösung $u\in H^1_0(\Omega)$ von \Cref{prob:continuousProblem}
bekannt, so kann die exakte Energie $E(u)$ approximiert werden mit der Methode
\begin{center}
  \texttt{./nonconforming/computeExactEnergyBV.m}.
\end{center}
Die so berechneten Energien werden gespeichert im Ordner
\begin{center}
  \texttt{./nonconforming/knownExactEnergies/}
\end{center}
und können anschließend manuell in ein Benchmark aufgenommen werden.
Soll als Eingangssignal ein Graufarbenbild gegeben sein, so muss es in einem
mit der Matlab-Funktion \texttt{imread} kompatiblen Format gespeichert sein in 
\begin{center}
  \texttt{./utils/functions/images/}.
\end{center}
Um Dirichlet-Nullranddaten des Bildes zu garantierten, was einem schwarzen Rand
entspricht, kann die Methode 
\begin{center}
  \texttt{./utils/functions/images/addBoundary2image.m}
\end{center}
genutzt werden. Diese fügt wahlweise einen graduellen Übergang zu schwarzem
Rand auf den äußeren 25 Pixeln des Bildes hinzu oder ergänzt das Bild um einen
10 Pixel breiten schwarzen Rand.
Um additives weißes gaußsches Rauschen zu einem Bild hinzuzufügen, kann die
Methode
\begin{center}
  \texttt{./utils/functions/images/addNoise2image.m}
\end{center}
genutzt werden, welche dafür die Matlab-Funktion \texttt{awgn} verwendet. 

\begin{table}
  \centering
  \begin{tabular}{c|p{9.9cm}}
    \hline
    Parametername  & Beschreibung\\  
    \hline
    \texttt{showPlots} 
    & \texttt{true}, wenn während des Programmablaufs Plots angezeigt werden
    sollen, sonst \texttt{false}\\
    \texttt{plotModeGrayscale} 
    & \texttt{true}, wenn während des Programmablaufs Plots von Funktionen als
    Graufarbenbilder mit Blick aus positiver z-Rich\-tung auf die xy-Ebene
    angezeigt werden sollen, sonst \texttt{false} (Wahl irrelevant, wenn
    \texttt{showPlots==false})\\
    \texttt{showProgress}
    & \texttt{true}, wenn während der primalen-dualen Iteration Informationen
    über den aktuellen Iterationsschritt angezeigt werden sollen, sonst
    \texttt{false} \\
    \texttt{degree4Integrate}
    & algebraischer Exaktheitsgrad für Aufrufe der Methode \texttt{in\-tegrate} 
    \cite[Abschnitt 1.8.2]{CGKNRR10} des AFEM-Soft\-ware\-pakets\\
    \texttt{plotGivenFunctions}
    & \texttt{true}, wenn Plots des Eingangssignals $f$ und, falls angegeben,
    der Lösung $u$ von \Cref{prob:continuousProblem} erstellt und gespeichert
    werden sollen, sonst \texttt{false}\\
    \texttt{refinementLevel4Plots}
    & Anzahl der Rotverfeinerungen der geladenen Geometrie um Plots des
    Eingangssignal $f$ und, falls angegeben, der Lösung $u$ von
    \Cref{prob:continuousProblem} zu erstellen (Wahl irrelevant, falls
    \texttt{plotGivenFunctions==false})\\
    \texttt{debugIfError}
    & \texttt{true}, wenn Matlab beim Auftreten eines Fehlers den Debug-Modus
    starten soll, sonst \texttt{false}\\
    \hline
  \end{tabular}
  \caption{Diverse Parameter zur Kontrolle des Programmablaufs}
  \label{tab:paramsMisc}
\end{table} 

\begin{table}
  \centering
  \begin{tabular}{c|p{9.7cm}}
    \hline
    Parametername  & Beschreibung\\  
    \hline
    \texttt{geometry} &
    Name der Geometrie auf deren Triangulierung der AFEM-Algorithmus angewendet
    werden soll (wird \texttt{'Square'} gesetzt, falls \texttt{useImage}
    aus \Cref{tab:paramsExperiment} den Wert \texttt{true} hat)\\
    \texttt{initialRefinementLevel} &
    Anzahl der Rotverfeinerungen, die auf die Triangulierung der Geometrie
    \texttt{geometry} angewendet werden sollen vor Start des AFEM\--Algorithmus 
    (cf. \texttt{loadGeometry} in \cite[Abschnitt 1.9.1]{CGKNRR10})\\
    \texttt{parTheta}& Bulk-Parameter $\theta$ für den Mark-Schritt des
    AFEM-Al\-go\-rith\-mus ($\theta\in(0,1)$ für adaptive und $\theta=1$ für
    uniforme Netzverfeinerung)\\
    \texttt{minNrDof}& 
    Anzahl der Freiheitsgrade der Triangulierung eines Levels, die mindestens
    erreicht werden soll, bevor der AFEM-Algorithmus abbricht\\
    \texttt{useProlongation}
    & \texttt{true}, wenn eine Prolongation der Lösung der pri\-ma\-len-dualen
    Iteration als Startwert für die Iteration des nächsten Levels genutzt
    werden soll, sonst \texttt{false}\\
    \texttt{parGamma}& 
    Parameter $\gamma$ aus \Cref{def:refinementIndicator}\\
    \texttt{d}& 
    Dimension $d$ aus \Cref{def:refinementIndicator} (muss für diese 
    Implementierung stets 2 sein)\\
    \hline
  \end{tabular}
  \caption{Parameter für den AFEM-Algorithmus}
  \label{tab:paramsAFEM}
\end{table} 

\begin{table}
  \centering
  \begin{tabular}{c|p{11.5cm}}
    \hline
    Parametername  & Beschreibung\\  
    \hline
    \texttt{u0Mode} 
    & Startwert $u_0$ für \Cref{alg:primalDualIteration} auf dem ersten Level
    und, falls \texttt{useProlongation==false}, für die Iteration auf allen
    Leveln der AFEM-Routine (\texttt{'zeros'} für $u_0=0$ und
    \texttt{'interpolationInSi'}, falls für $u_0$ die
    Crouzeix-Raviart-Interpolation des Eingangssignals $f$ genutzt werden
    soll)\\
    \texttt{epsStop} & Parameter $\epsstop$ für das Abbruchkriterium
    \eqref{eq:terminationCriterion} von \Cref{alg:primalDualIteration}\\
    \texttt{parTau}& Parameter $\tau$ aus \Cref{alg:primalDualIteration}\\
    \texttt{maxIter}& 
    Anzahl der Iterationsschritte der primalen-dualen Iteration, die auf jedem
    Level maximal durchgeführt werden sollen\\
    \hline
  \end{tabular}
  \caption{Parameter für die primale-duale Iteration}
  \label{tab:paramsIteration}
\end{table} 

\begin{table}
  \centering
  \begin{tabular}{c|p{10.5cm}}
    \hline
    Parametername  & Beschreibung\\  
    \hline
    \texttt{useImage}
    & \texttt{true}, wenn für das Eingangssignal $f$ ein Graufarbenbild
    anstelle eines \texttt{function\_handle} genutzt werden soll, sonst
    \texttt{false}\\
    \texttt{imageName} 
    & Name (mit Dateiendung) des Graufarbenbildes im Ordner
    \texttt{./utils/functions/images/}, das als Eingangssignal $f$
    genutzt werden soll (Wahl irrelevant, falls
    \texttt{useImage==false})\\
    \texttt{parAlpha}
    & Parameter $\alpha$ aus \Cref{prob:continuousProblem} und
    \Cref{prob:discreteProblem}\\
    \texttt{parBeta} 
    & Parameter $\beta$ einiger Experimente aus \Cref{chap:experiments}\\
    \texttt{inSiGradientKnown}
    & \texttt{true}, wenn ein \texttt{function\_handle} des schwachen
    Gradienten $\nabla f$ eines Eingangssignals $f\in H^1_0(\Omega)$ gegeben
    ist und zur Berechnung der garantierten unteren Energieschranke aus
    \Cref{thm:gleb} genutzt werden soll, sonst \texttt{false} (wird
    \texttt{false} gesetzt, falls \texttt{useImage==true})\\
    \texttt{exactSolutionKnown}
    & \texttt{true}, wenn die Lösung $u$ von \Cref{prob:continuousProblem}
    bekannt ist und zur Berechnung von exakten Fehlern zu Iteraten von
    \Cref{alg:primalDualIteration} genutzt werden soll, sonst \texttt{false}
    (wird \texttt{false} gesetzt, falls \texttt{useImage==true})\\
    \texttt{useExactEnergy}
    & \texttt{true}, wenn $E(u)$ für die Lösung $u$ von
    \Cref{prob:continuousProblem} bekannt ist und in Auswertungen der
    Ergebnisse genutzt werden soll, sonst \texttt{false} (wird \texttt{false}
    gesetzt, falls \texttt{exactSolutionKnown==false})\\ 
    \texttt{exactEnergy} 
    & Wert $E(u)$ für die Lösung $u$ von \Cref{prob:continuousProblem} (Wahl
    irrelevant, falls \texttt{useExactEnergy==false})\\
    \hline
  \end{tabular}
  \caption{Parameter des Experiments}
  \label{tab:paramsExperiment}
\end{table} 

\begin{table}
  \centering
  \begin{tabular}{c|p{11.5cm}}
    \hline
    Parametername  & Beschreibung\\  
    \hline
    \texttt{expName} 
    & Name des Ordners, der in \texttt{./../results/nonconforming/} zum
    Speichern der Ergebnisse von Durchläufen des Programms für ein Experiment
    erstellt werden soll\\
    \texttt{dirInfoName} 
    & Name des Ordners, der in \texttt{./../results/nonconforming/expName/} zum
    Speichern der Ergebnisse eines Durchlaufs des Programms erstellt werden
    soll \\
    \hline
  \end{tabular}
  \caption{Parameter zum Speichern der Ergebnisse}
  \label{tab:paramsDoc}
\end{table} 

\begin{table}
  \centering
  \begin{tabular}{c|p{10.3cm}}
    \hline
    Parametername  & Beschreibung\\  
    \hline
    \texttt{inputSignal} 
    & \texttt{function\_handle} des Eingangssignals $f$ (Wahl irrelevant, falls
    \texttt{useImage} aus \Cref{tab:paramsExperiment} den Wert \texttt{true}
    hat)\\
    \texttt{gradientInputSignal} 
    & \texttt{function\_handle} des schwachen Gradienten $\nabla f$ des
    Eingangssignals $f$ (Wahl irrelevant, falls \texttt{inSiGradientKnown} aus
    \Cref{tab:paramsExperiment} den Wert \texttt{false} hat)\\ 
    \texttt{exactSolution}
    & \texttt{function\_handle} der Lösung $u$ von
    \Cref{prob:continuousProblem} (Wahl irrelevant, falls
    \texttt{exactSolutionKnown} aus \Cref{tab:paramsExperiment} den Wert
    \texttt{false} hat)\\
    \hline
  \end{tabular}
  \caption{Definierbare \texttt{function\_handles}}
  \label{tab:paramsFunctions}
\end{table} 


\section{Programmablauf}
\label{sec:programFlow}

In diesem Abschnitt betrachten wir eine beispielhafte Ausführung des Programms,
bei der alle relevanten Methoden genutzt werden.
In der Benchmark-Datei für den Programmaufruf müssen dafür, neben dem
Eingangssignal $f\in H^1_0(\Omega)$, die Lösung $u\in H^1_0(\Omega)$, die
exakte Energie $E(u)$ und der schwache Gradient von $f$ gegeben sein. 
Damit diese Informationen tatsächlich genutzt werden, müssen zusätzlich einige
Parameter aus \Cref{tab:paramsExperiment} passend gewählt werden. 
Diese Wahlen sind
\begin{itemize}
  \item \texttt{useImage = false},
  \item \texttt{inSiGradientKnown = true},
  \item \texttt{exactSolutionKnown = true},
  \item \texttt{useExactEnergy = true}.
\end{itemize}
Noch in der Benchmark-Datei werden mit den gewählten Einstellungen die
benötigten geometrischen Datenstrukturen für die Triangulierung $\Tcal$
initialisiert. 
Diese werden mit allen weiteren gegebenen Informationen als Felder eines
Structure Arrays gespeichert und an das Hauptprogramm übergeben. 
Dort werden zunächst Structure Arrays mit Feldern für die benötigten Daten
während der AFEM-Routine sowie deren Ergebnisse erstellt. 
Insbesondere wird der Startwert $u_0$ für die primale-duale Iteration
auf dem ersten Level des AFEM-Algorithmus initialisiert. 
Dieser hängt von der Wahl des Parameters \texttt{u0Mode} aus
\Cref{tab:paramsIteration} ab.
Falls dabei die Crouzeix-Raviart-Interpolation des Eingangssignals $f$ 
bezüglich $\Tcal$ genutzt werden soll, wird diese in
\begin{center}
  \texttt{./nonconforming/common/interpolationCR.m}
\end{center}
ermittelt.
Nach dem Berechnen einiger weiterer Informationen beginnt die AFEM-Schleife.

Zu Beginn der Schleife werden die für das aktuelle Level benötigten Daten
ermittelt, zum Beispiel die Flächeninhalte der Dreiecke und die Freiheitsgrade,
das heißt in dieser Implementierung die inneren Kanten, der aktuellen
Triangulierung $\Tcal$. 
Insbesondere werden die auf dem Level mehrmals benötigten Gradienten der
lokalen Crouzeix-Raviart-Basisfunktionen aus
\Cref{sec:crouzeixRaviartFunctions} auf allen Dreiecken berechnet (cf.
\cite[Abschnitt 1.4.2]{CGKNRR10}) in
\begin{center}
  \texttt{./nonconforming/common/computeGradsCR4e.m}.
\end{center}
Mit diesen wird der stückweise Gradient des Startwerts $u_0$ aus
\Cref{alg:primalDualIteration} berechnet in
\begin{center}
  \texttt{./nonconforming/common/gradientCR.m}.
\end{center}
Mit $\gradnc u_0$ wird anschließend $\Lambda_0$ aus
\Cref{alg:primalDualIteration} auf allen $T\in\Tcal$ initialisiert als 
\begin{align*}
  \Lambda_0|_T\coloneqq
  \begin{cases}
    \frac{\nabla u_0|_T}{|\nabla u_0|_T|},&\text{falls }\nabla u_0|_T\neq 0,\\
    0,&\text{falls }\nabla u_0|_T= 0.
  \end{cases}
\end{align*}
Außerdem werden die Steifigkeits- und Massenmatrix aus
\Cref{rem:primalDualMatrixEquations} aufgestellt in
\begin{center}
  \texttt{./nonconforming/common/computeFeMatricesCR.m}.
\end{center}
Dabei benutzten wir für jedes $T\in\Tcal$, dass die lokale Massenmatrix auf
$T$ mithilfe der Gleichungen
\eqref{eq:connectionCrBarycentric} und
\eqref{eq:formulaIntegrationBarycentricCoordinates}
berechnet werden kann durch
\begin{align}
  \label{eq:localMassmatrix}
  M_T
  =
  \frac{|T|}{3}I_3.
\end{align}
Das Aufstellen der lokalen Steifigkeitsmatrizen sowie die Berechnung der
globalen Stei\-fig\-keits- und Massenmatrix aus ihren lokalen Versionen geschieht
analog zu \cite[Abschnitt 1.4.2]{CGKNRR10}).
Danach werden in
\begin{center}
  \texttt{./nonconforming/common/integralsWithInSi.m}
\end{center}
mit den Crouzeix-Raviart-Basisfunktionen
$\psi_F$, $F\in\Ecal$, für jedes Dreieck $T\in\Tcal$ mit Kanten
$\Ecal(T)=\{E_1,E_2,E_3\}$ die Integrale $\int_T f\psi_{E_k}\!|_T\dx$,
$k\in\{1,2,3\}$, berechnet. 
Dies geschieht mithilfe der \texttt{integrate} Methode \cite[Abschnitt
1.8.2]{CGKNRR10} des AFEM-Softwarepakets.
Die Berechnung der dafür benötigten lokalen Crouzeix-Raviart-Basisfunktionen
$\psi_{E_k}\!|_T$, $k\in\{1,2,3\}$, diskutieren wir in
\Cref{sec:localCRBasisRealization}.
Anschließend kann für alle $F\in\Ecal$ auch das Integral 
\begin{align*}
  \int_\Omega f\psi_F\dx
  &=
  \sum_{T\in\Tcal}\int_T f\psi_F\!|_T\dx
  =
  \sum_{T\in\Tcal(F)}\int_T f\psi_F\!|_T\dx
\end{align*}
bestimmt werden.
Damit sind alle notwendigen Daten ermittelt, um die primale-duale Iteration für
den Solve-Schritt des aktuellen Levels zu starten. 
Genutzt wird dafür die Methode
\begin{center}
  \texttt{./nonconforming/main/solvePrimalDualFormulation.m}.
\end{center}
Details dazu diskutieren wir in \Cref{sec:implementationPrimalDualIteration}.
Mit dem letzten Iterat $u_J\in\CR^1_0(\Tcal)$ als Approximation der Lösung von
\Cref{prob:discreteProblem} auf dem Level berechnen wir unter anderem
$|E(u)-\Enc(u_J)|$ und, mithilfe der Methode \texttt{error4eCRL2}
\cite[Abschnitt 1.8.3]{CGKNRR10} des AFEM-Soft\-ware\-pakets, den Fehler $\Vert
u - u_J\Vert$.
Außerdem werden Informationen über die Iteration und ihre Ergebnisse
gespeichert und gegebenenfalls ausgegeben. 
%%%%TODO formulieren. Außerdem: muss noch eine dieser Funktionen detaillierter 
  % beschrieben werden?
Zur Berechnung der garantierten oberen Energieschranke $\Enc(J_1 u_J)$ nach
\eqref{eq:gueb} nutzen wir anschließend die Methoden
\begin{center}
    \texttt{./nonconforming/common/computeJ1.m},
\end{center}
die $J_1 u_J\in P_1(\Tcal)\cap C_0(\Omega)\subset \CR^1_0(\Tcal)$ mit dem
Operator $J_1$ aus \eqref{eq:enrichmentOperator} bestimmt,
\begin{center}
    \texttt{./nonconforming/common/courant2CR.m},
\end{center}
welche die Koeffizienten von $J_1 u_J$ bezüglich der
Crouzeix-Raviart-Basisfunktionen aus \Cref{sec:crouzeixRaviartFunctions} 
ermittelt, und schließlich
\begin{center}
    \texttt{./nonconforming/common/computeDiscreteEnergyCR.m},
\end{center}
welche $\Enc(J_1 u_J)$, wie zum Abschluss des nächsten Abschnitts
\ref{sec:implementationPrimalDualIteration} beschrieben, berechnet.
Zum Bestimmen der garantierten unteren Energieschranke aus \Cref{thm:gleb} und
des Verfeinerungsindikators aus \Cref{def:refinementIndicator} benötigen wir
dann für alle $T\in\Tcal$ den Term $\Vert f-\alpha u_J\Vert_{L^2(T)}^2$.
Zur Berechnung dieser Terme nutzen wir die Methode
\begin{center}
  \texttt{./nonconforming/common/computeNormDiffInSiSolCrSquared4e.m}
\end{center}
wie folgt. 
Sei $T\in\Tcal$ mit $\Ecal(T)=\{E_1,E_2,E_3\}$ und $u_J|_T$ habe mit
$v_1,v_2,v_3\in\Rbb$ die Darstellung $u_J|_T=\sum_{k=1}^3 v_k\psi_{E_k}\!|_T$.
Dann gilt
\begin{align*}
  \Vert f-\alpha u_J\Vert_{L^2(T)}^2 
  &=
  \Vert f\Vert_{L^2(T)}^2 - 2\alpha\left(f,u_J\right)_{L^2(T)} + \alpha^2 \Vert
  u_J\Vert_{L^2(T)}^2\\
  &=
  \Vert f\Vert_{L^2(T)}^2 
  - 2\alpha\sum_{k=1}^3 v_k\int_T f\psi_{E_k}\!|_T\dx
  + \alpha^2 \sum_{k=1}^3 \sum_{\ell=1}^3 
  v_k \left(\psi_{E_k}\!|_T,\psi_{E_\ell}|_T\right)_{L^2(T)}v_\ell.
\end{align*} 
Um den ersten Term zu berechnen, nutzen wir die \texttt{integrate} Methode des
AFEM-Soft\-ware\-pakets und für die Berechnung des zweiten Terms merken wir an,
dass die Integrale $\int_T f\psi_{E_k}\!|_T\dx$, $k\in\{1,2,3\}$, bereits
ermittelt wurden.
Der letzte Term vereinfacht sich durch Nutzung der expliziten Darstellung
\eqref{eq:localMassmatrix} der lokalen Massenmatrix $M_T$ zu
\begin{align*}
  \alpha^2 \sum_{k=1}^3 \sum_{\ell=1}^3 
    v_k \left(\psi_{E_k}\!|_T,\psi_{E_\ell}|_T\right)_{L^2(T)}v_\ell
  =
  \alpha^2 \sum_{k=1}^3 \sum_{\ell=1}^3 
  v_k M_{T,k\ell} v_\ell
  =
  \alpha^2\frac{|T|}{3}\sum_{k=1}^3 v_k^2.
\end{align*}
Unter erneuter Zuhilfenahme der \texttt{integrate} Methode wird anschließend
in
\begin{center}
  \texttt{./nonconforming/common/computeGleb.m}
\end{center}
die garantierte untere Energieschranke berechnet. 
Zur Berechnung des Verfeinerungsindikators benötigen wir noch
für alle $F\in\Ecal$ den Term $\Vert[u_J]_F\Vert_{L^1(F)}$. Diese ermitteln wir
in der Methode 
\begin{center}
  \texttt{./nonconforming/common/computeL1NormOfJump4s.m}.
\end{center} 
Details dazu diskutieren wir in \Cref{sec:jumptTermsImplementation}.
Danach werden der Verfeinerungsindikator
und seine Anteile $\eta_\textup{V}$ und $\eta_J$ berechnet in der Methode
\begin{center}
  \texttt{./nonconforming/estimate/estimateErrorCR4e.m}.
\end{center}
Es folgt die Ausgabe einiger relevanter Informationen über das Level und das
Abspeichern dieser und weiterer Ergebnisse in den durch die Paramter in
\Cref{tab:paramsDoc} gewählten Ordner mittels der Methode
\begin{center}
  \texttt{./nonconforming/misc/saveResultsCR.m}.
\end{center}
Endet nun der AFEM-Algorithmus nach Überprüfung des Abbruchkriteriums nicht,
das heißt ist die Anzahl der Freiheitsgrade der Triangulierung des aktuellen
Levels geringer als die Zahl \texttt{minNrDof} aus \Cref{tab:paramsAFEM}, so
werden mithilfe des Verfeinerungsindikators $\eta$ die Dreiecke markiert, die
verfeinert werden sollen. 
Dafür nutzen wir die folgenden Methoden des AFEM-Softwarepakets, die in
\cite[Abschnitt 1.6]{CGKNRR10} beschriebenen sind.
Gilt für den Bulk-Parameter $\theta$, gewählt durch \texttt{parTheta} aus
\Cref{tab:paramsAFEM}, dass $\theta=1$, so nutzen
wir die Methode \texttt{markUniform}. Gilt $\theta\in(0,1)$, so nutzen wir
die Methode \texttt{markBulk}.
Zum Verfeinern der Triangulierung wird anschließend die Methode
\texttt{refineRGB} \cite[1.7.2]{CGKNRR10} des AFEM-Softwarepakets verwendet.

Zum Abschluss werden bereits Informationen für das nächste Level vorbereitet. 
Dazu gehören die Datenstrukturen, welche die verfeinerte Triangulierung
$\hat\Tcal$ beschreiben, sowie einige weitere Eigenschaften von
$\hat\Tcal$, zum Beispiel die Anzahl und Längen der Seiten der Triangulierung.
Soll als Startwert für das nächste Level eine Prolongation $\hat
u_J\in\CR^1_0\big(\hat\Tcal\big)$ von $u_J$ auf $\hat\Tcal$ genutzt werden,
festgelegt durch den Parameter \texttt{useProlongation} aus
\Cref{tab:paramsAFEM}, wird dessen Berechnung in der Methode
\begin{center}
  \texttt{./nonconforming/common/prolongationJ1.m}
\end{center}
realisiert. 
Als Prolongation von $u_J$ wird ebenda mit dem Operator $J_1$ aus
\eqref{eq:enrichmentOperator} die Funktion $\hat u_J\coloneqq J_1 u_J\in
P_1(\Tcal)\cap C_0(\Omega)\subseteq P_1\big(\hat\Tcal\big)\cap C_0(\Omega)
\subseteq \CR^1_0\big(\hat\Tcal\big)$ gewählt.
Die Berechnung der dabei benötigten lokalen Knotenwerte von $u_J$ geschieht
wie in \Cref{sec:localNodalValuesCR} beschrieben.

\section{Realisierung der primalen-dualen Iteration}
\label{sec:implementationPrimalDualIteration}
In diesem Abschnitt beschreiben wir die Methode
\begin{center}
  \texttt{./nonconforming/main/solvePrimalDualFormulation.m}.
\end{center}
Dabei benutzen wir die Bezeichnungen aus \Cref{alg:primalDualIteration}, der
in dieser Methode realisiert wird.
Wie in \Cref{sec:programFlow} beschrieben, sind vor Aufruf der Methode bereits
alle benötigten Informationen über die aktuelle Triangulierung $\Tcal$ sowie
die Steifig\-keits- und Massenmatrix aus \Cref{rem:primalDualMatrixEquations}
und die Integrale $\int_\Omega f\psi_F\dx$ für alle
Crouzeix-Raviart-Basisfunktionen $\psi_F$, $F\in\Ecal$, bekannt.
Mit der Steifigkeits- und Massenmatrix kann die Koeffizientenmatrix des
Gleichungssystems \eqref{eq:linSysPrimalDualAlgMatrixEq} ohne Weiteres
berechnet werden.
Für die Implementierung der rechten Seite $\overline b$ dieses Systems nutzten
wir die folgende Beobachtung.
Für alle $k\in\{1,2,\cdots,|\Ecal|\}$ und alle $j\in\Nbb$ gilt, da $\gradnc
u_{j-1}$, $\Lambda_j$ und $\gradnc \psi_{E_k}$ stückweise konstant sind, dass
\begin{align*}
  \left(\frac{1}{\tau}\gradnc u_{j-1}-\Lambda_j,\gradnc\psi_{E_k}\right)
  &=
  \sum_{T\in\Tcal}
  \left(\frac{1}{\tau}\gradnc u_{j-1}-\Lambda_j,
  \gradnc\psi_{E_k}\right)_{L^2(T)}\\
  &=
  \sum_{T\in\Tcal(E_k)}
  \frac{|T|}{\tau}\left(\gradnc u_{j-1}-\Lambda_j\right)\!|_T\cdot
  \nabla\psi_{E_k}|_T.
\end{align*}
Da die Terme $\left(f,\psi_{E_k}\right)=\int_\Omega f\psi_{E_k}\dx$ bereits
berechnet wurden, können wir somit die rechte Seite $\overline b$
implementieren.
Es ist direkt ersichtlich wie der verbleibende Teil von
\Cref{alg:primalDualIteration} mit Abbruchkriterium
\eqref{eq:terminationCriterion} realisiert werden kann.

Es bleibt anzumerken, dass in \texttt{solvePrimalDualFormulation}, neben 
\Cref{alg:primalDualIteration}, der Ausgabe des Fortschritts sowie
der Übergabe von Ergebnissen der Iteration, auch die
nichtkonformen Energien $\Enc(u_j)$ der Iterate berechnet werden. 
Diese Berechnung ist realisiert in
\begin{center}
  \texttt{./nonconforming/common/computeDiscreteEnergyCR.m}.
\end{center}
Dabei werden die Massenmatrix $M$ und die Integrale 
$\int_\Omega f\psi_{F}\dx$, $F\in\Ecal$, wie folgt genutzt.
Sei $\vcr\in\CR^1_0(\Tcal)$ und $v\in\Rbb^{|\Ecal|}$
enthalte die Koordinaten von $\vcr$ bezüglich der Basis
$\left\{\psi_{E_1},\psi_{E_2},\ldots,\psi_{E_{|\Ecal|}}\right\}$
von $\CR^1(\Tcal)$, das heißt
$\vcr=\sum_{k=1}^{|\Ecal|}v_k \psi_{E_k}$.
Dann gilt
\begin{align*}
  \Enc(\vcr)
  &=
  \frac{\alpha}{2}\Vert \vcr\Vert^2
  + \Vert\gradnc \vcr\Vert_{L^1(\Omega)}
  -\int_\Omega f\vcr\dx\\
  &=
  \frac{\alpha}{2}
  \sum_{k=1}^{|\Ecal|} 
  \sum_{\ell=1}^{|\Ecal|} 
  v_k\left( \psi_{E_k}, \psi_{E_\ell} \right)v_\ell
  + \sum_{T\in\Tcal}
   \int_T\left|\gradnc \vcr\right|\dx
  -\sum_{k=1}^{|\Ecal|}v_k\int_\Omega f\psi_{E_k}\dx\\
  &=
  \frac{\alpha}{2} v\cdot Mv
  + \sum_{T\in\Tcal}|T|\, \big|\!\left(\gradnc \vcr\right)\!\!|_T\big|
  -\sum_{k=1}^{|\Ecal|}v_k\int_\Omega f\psi_{E_k}\dx.
\end{align*}


\section{Mathematische Grundlagen ausgewählter Methoden}
\label{sec:mathematicalBasicsForMethods}

\subsection{Berechnung lokaler Crouzeix-Raviart-Basisfunktionen}
\label{sec:localCRBasisRealization}

In der Methode
\begin{center}
  \texttt{./nonconforming/common/integralsWithInSi.m}
\end{center}
werden auf allen Dreiecken einer Triangulierung $\Tcal$ die lokalen
Crouzeix-Raviart-Basis\-funk\-tio\-nen aus \Cref{sec:crouzeixRaviartFunctions}
benötigt. 
Sei $T\in\Tcal$ mit $T=\conv\left\{P_1,P_2,P_3\right\}$.
Da nach \Cref{eq:connectionCrBarycentric} die lokalen
Crouzeix-Raviart-Basisfunktionen $\psi_{F}|_T$, $F\in\Ecal(T)$, darstellbar
sind durch die baryzentrischen Koordinaten $\lambda_1,\lambda_2,\lambda_3\in
P_1(T)$ aus \Cref{sec:crouzeixRaviartFunctions}, genügt es hier deren
Berechnung zu diskutieren.
Dafür betrachten wir das Referenzdreieck, das definiert ist durch
\begin{align*}
  \Tref \coloneqq
  \conv\left\{
  \begin{pmatrix}
   1\\0 
  \end{pmatrix},
  \begin{pmatrix}
   0\\1 
  \end{pmatrix},
  \begin{pmatrix}
   0\\0 
  \end{pmatrix}
  \right\},
\end{align*}
und mit der Matrix $B\coloneqq (P_1-P_3,P_2-P_3)$ die affine Transformation
$p:\Tref \to T$, $x \mapsto Bx+P_3$. 
Mithilfe der Umkehrabbildung von $p$, die gegeben ist durch $p^{-1}:T\to\Tref$,
$x\mapsto B^{-1}(x-P_3)$, können die baryzentrischen Koordinaten für alle $x\in
T$ berechnet werden durch 
\begin{align*}
  \begin{pmatrix}
    \lambda_1(x)\\
    \lambda_2(x)
  \end{pmatrix}
  &=
  p^{-1}(x) &&\text{und}
  &\lambda_3(x)
  &=1-\lambda_1(x)-\lambda_2(x).
\end{align*}


\subsection{Berechnung lokaler Knotenwerte einer Crouzeix-Raviart-Funktion}
\label{sec:localNodalValuesCR}

Sei $\vcr\in\CR^1(\Tcal)$.
Um $J_1\vcr$ für den Operator $J_1$ aus \Cref{eq:enrichmentOperator} zu
bestimmen und für die Berechnung der $L^1$-Norm der Kantensprünge von $\vcr$
benötigen wir für jedes Dreieck $T\in\Tcal$ die Werte von $\vcr$ in den Knoten
von $T$. 
Dazu sei $T = \conv\{P_1, P_2, P_3\}$ mit den Kanten
$E_1 = \conv\{P_1,P_2\}$, $E_2 = \conv\{P_2,P_3\}$ und $E_3 =
\conv\{P_3,P_1\}$. 
Die Funktion $v\coloneqq\vcr|_T$ habe in den Mittelpunkten der Kanten die Werte
$v_j\coloneqq v\left(\Mid(E_j)\right)$ für alle $j\in\{1,2,3\}$. 
Gesucht sind die Werte $v(P_1)$, $v(P_2)$ und $v(P_3)$.

Da $\vcr\in\CR^1(\Tcal)$, ist $v$ affin. Damit gilt für eine Kante
$E=\conv\{P,Q\}\in\{E_1,E_2,E_3\}$, dass $v(\Mid(E))$ gegeben ist durch das
arithmetische Mittel von $v(P)$ und $v(Q)$.
Somit erhalten wir die drei Gleichungen
\begin{align*}
  v_1 &= \frac{v(P_1)+v(P_2)}{2},  
  &v_2 &= \frac{v(P_2)+v(P_3)}{2},  
  &v_3 &= \frac{v(P_3)+v(P_1)}{2}.
\end{align*}
Sind $v_1$, $v_2$ und $v_3$ bekannt, können wir dieses Gleichungssystem nach 
$v(P_1)$, $v(P_2)$ und $v(P_3)$ lösen und erhalten die gesuchten Werte
\begin{align*}
 v(P_1)&=v_1+v_3-v_2, &v(P_2)&= v_1+v_2-v_3,&v(P_3)&=v_2+v_3-v_1.
\end{align*}
Wir realisieren diese Berechnung in der Methode
\begin{center}
  \texttt{./nonconforming/common/computeNodeValuesCR4e.m}.
\end{center} 


\subsection{Berechnung von Sprungtermen}
\label{sec:jumptTermsImplementation}

Insbesondere für die Berechnung des Verfeinerungsindikators aus
\Cref{def:refinementIndicator} benötigen wir eine Methode, die
für eine Crouzeix-Raviart-Funktion
$\vcr\in\CR^1_0(\Tcal)$
%und zur Auswertung der kontinuierlichen Energie $E(\vcr)$ einer 
%Crouzeix-Raviart Funktion $\vcr$, deren diskrete Energie $\Enc(\vcr)$
%bereits bekannt ist, 
für jede Kante $F\in\Ecal$ den Term $\Vert [\vcr]_F\Vert_{L^1(F)}$ 
bestimmt.

Da $\vcr\in\CR^1_0(\Tcal)$, ist $[\vcr]_F$ affin und es gilt
$[\vcr]_F\big(\Mid(F)\big)=0$ für alle Kanten $F\in\Ecal$.
Betrachten wir nun eine beliebige Kante $F\in\Ecal$ mit $F=\conv\{P_1,P_2\}$. 
Wir definieren eine Parametrisierung $\xi:[0,2]\to\Rbb^2$ von $F$ durch
$\xi(t)\coloneqq \frac{t}{2}(P_2-P_1)+P_1$. 
Es gilt $\left|\xi'\right|\equiv \frac{1}{2}|P_2-P_1|=\frac{1}{2}|F|$.
Sei außerdem
$p(t)\coloneqq [\vcr]_F\big(\xi(t)\big)$. Dann gilt
\begin{align*}
  \Vert [\vcr]_F\Vert_{L^1(F)} 
  &=
  \int_F |[\vcr]_F|\ds 
  = \int_0^2 |p(t)|\left|\xi'(t)\right|\dt
  = \frac{|F|}{2}\int_0^2 |p(t)|\dt\\
  &= \frac{|F|}{2}\left(\int_0^1 |p(t)|\dt + \int_1^2 |p(t)|\dt\right).
\end{align*}
Da $\vcr\in\CR^1_0(\Tcal)$, ist $|p(\bullet)|$ auf $[0,1]$ und $[1,2]$ jeweils
ein Polynom vom Grad $1$ mit $|p(1)|=\left|[\vcr]_F\big(\Mid(F)\big)\right|=0$. 
Somit können wir $|p|$ explizit ausdrücken durch
\begin{align*}
  |p(t)|&=(1-t)|p(0)| \quad\text{für alle }t\in[0,1] \quad\text{und }\\
  |p(t)|&=(t-1)|p(2)| \quad\text{für alle }t\in[1,2].
\end{align*}
Damit erhalten wir, aufgrund der Exaktheit der Mittelpunktsregel für Polynome
vom Grad 1, dass
\begin{align*}
  \int_0^1 |p(t)|\dt 
  &= 
  (1-0)\left|p\left( \frac{1}{2} \right)\right|
  =
  \frac{|p(0)|}{2}\quad\text{und }\\
  \int_1^2 |p(t)|\dt 
  &= 
  (2-1)\left|p\left( \frac{3}{2} \right)\right|
  =
  \frac{|p(2)|}{2}.
\end{align*}
Insgesamt gilt also
\begin{align*}
  \left\Vert [\vcr]_F\right\Vert_{L^1(F)} 
  &=
  \frac{|F|}{2}\left(\frac{|p(0)|}{2} + \frac{|p(2)|}{2}\right)
  =
  \frac{|F|}{4}\big(|p(0)|+|p(2)|\big)\\
  &= 
  \frac{|F|}{4}\big(|[\vcr]_F(P_1)|+|[\vcr]_F(P_2)|\big).
\end{align*}
Realisiert wird dies in der Methode
\begin{center}
  \texttt{./nonconforming/common/computeL1NormOfJump4s.m}.
\end{center}
Dabei werden für $\vcr\in\CR^1_0(\Tcal)$ für jede Kante $F\in\Ecal$ mit
$F=\conv\{P_1,P_2\}$ die Terme $|[\vcr]_F(P_1)|$ und $|[\vcr]_F(P_2)|$
berechnet in
\begin{center}
  \texttt{./nonconforming/common/computeAbsNodeJumps4s.m}.
\end{center}
