\section{Primale-duale Iteration}

In diesem Abschnitt präsentieren wir ein iteratives Verfahren mit dem wir
\Cref{prob:discreteProblem} numerisch lösen möchten. 
Dieses basiert auf der primalen-dualen Iteration \cite[S. 314, Algorithm
10.1]{Bar15} unter Beachtung von \cite[S. 314, Remark 10.11]{Bar15}. 
%Diese realisiert Gradientenverfahren zum Finden eines Sattelpunkts, dessen 
%Komponenten die primale und die duale Formulierung des Minimierungsproblems
%lösen. 
Details dazu und weitere Referenzen finden sich in \cites{Bar12}[S.
118-121]{Bar15}.
Angepasst an unser Problem und die Notation dieser Arbeit lautet der
Algorithmus wie folgt.

\begin{algorithm}[Primale-duale Iteration]
  \label{alg:primalDualIteration}
\begin{algorithmic}\\
  \Require $\left(u_0,\Lambda_0\right)
  \in\textup{CR}_0^1(\mathcal{T})\times P_0\!\left(\mathcal{T}; 
  %\left\{w\in\Rbb^2\,\middle|\,|w|\leq 1\right\}\right),
  \overline{B_{\Rbb^2}}\right),
  %\overline{B_1(0)}\right),
  \tau>0$  \\
  Initialisiere $v_0\coloneqq 0$ in $\textup{CR}^1_0(\mathcal T)$.
  \For{$j = 1,2,\dots$}
  \begin{align}
    %\label{eq:primalDualAlgUj}
    \tilde{u}_j&\coloneqq u_{j-1}+\tau v_{j-1},\nonumber\\
    \label{eq:primalDualAlgLambdaJ}
    \Lambda_j
    &\coloneqq
    \frac{\Lambda_{j-1}+\tau\nabla_{\textup{NC}} \tilde{u}_j}
    {\max\left\{1,
    \left|\Lambda_{j-1}+\tau\nabla_{\textup{NC}}\tilde{u}_j\right|\right\}},\\
    \text{bestimme }u_j\in\textup{CR}^1_0(\mathcal{T})
    \text{ a}&\text{ls Lösung des linearen Gleichungssystems }\nonumber\\
    \label{eq:linSysPrimalDualAlg}
    \frac{1}{\tau}a_{\textup{NC}}(u_j,\bullet)+\alpha(u_j,\bullet)
    &=
    \frac{1}{\tau}a_{\textup{NC}}(u_{j-1},\bullet) + (f,\bullet)
    - \left(\Lambda_j,\nabla_{\textup{NC}}\bullet\right) 
    \text{ in }\CR^1_0(\Tcal),\\
    v_j &\coloneqq \frac{u_j-u_{j-1}}{\tau}.\nonumber
  \end{align}
  %\State bestimme $u_j\in\textup{CR}^1_0(\mathcal{T})$
  %als Lösung des linearen Gleichungssystems
  %\begin{align}
  %  \label{eq:linSysPrimalDualAlg}
  %  \frac{1}{\tau}a_{\textup{NC}}(u_j,\bullet)+\alpha(u_j,\bullet)
  %  &=
  %  \frac{1}{\tau}a_{\textup{NC}}(u_{j-1},\bullet) + (f,\bullet)
  %  - \left(\Lambda_j,\nabla_{\textup{NC}}\bullet\right) 
  %  \quad\text{in }\CR^1_0(\Tcal),
  %\end{align}
  %\begin{equation*}
  %  v_j\coloneqq\frac{u_j-u_{j-1}}{\tau}.
  %\end{equation*}
  \EndFor
  \Ensure Folge $(u_j,\Lambda_j)_{j\in\mathbb N}$ in
  $\CR^1_0(\mathcal{T})\times
   P_0\!\left(\mathcal{T};\overline{B_{\Rbb^2}}\right)$   
  \end{algorithmic}
\end{algorithm}

In \cite{Bar15} wird in \Cref{eq:linSysPrimalDualAlg} anstelle des diskreten
Skalarprodukts $\anc(\bullet,\bullet)$ ein diskretes Skalarprodukt
$(\bullet,\bullet)_{h,s}$, welches ungleich dem $L^2$-Skalarprodukt sein
kann, genutzt. Die Iteration wird ebenda abgebrochen, wenn die von
$(\bullet,\bullet)_{h,s}$ induzierte Norm des Terms $(u_j-u_{j-1})/\tau$
kleiner einer gegebenen Toleranz ist.
Dementsprechend nutzen wir mit $\epsstop > 0$ für
\Cref{alg:primalDualIteration} das Abbruchkriterium 
\begin{align}
  \label{eq:terminationCriterion}
  \left\vvvert \frac{u_j-u_{j-1}}{\tau}\right\vvvert_\NC<\epsstop.
\end{align}

\begin{remark} 
  \label{rem:primalDualMatrixEquations}
  Mit den kantenorientierten Crouzeix-Raviart-Basisfunktionen
  $\{\psi_E\,\mid\,E\in\Ecal\}$ aus \Cref{sec:crouzeixRaviartFunctions} können
  wir die Steifigkeitsmatrix $A\in\Rbb^{|\Ecal|\times|\Ecal|}$ und die
  Massenmatrix $M\in\Rbb^{|\Ecal|\times|\Ecal|}$ für alle
  $k,\ell\in\{1,2,\ldots,|\Ecal|\}$ definieren durch
  \begin{align*}
    A_{k\ell}\coloneqq \anc\!\left(\psi_{E_k},\psi_{E_\ell}\right)
    \quad\text{und}\quad
    M_{k\ell}\coloneqq \left(\psi_{E_k},\psi_{E_\ell}\right).
  \end{align*}
  Außerdem definieren wir mit $u_{j-1}$ und $\Lambda_j$ aus
  \Cref{alg:primalDualIteration} den Vektor $b\in\Rbb^{|\Ecal|}$ für alle
  $k\in\Nbb$ durch
  \begin{align*}
    b_k\coloneqq 
    \left(\frac{1}{\tau}\gradnc u_{j-1}-\Lambda_j,\gradnc\psi_{E_k}\right)
    + \left( f,\psi_{E_k} \right).
  \end{align*}
  Sei nun $\Ecal=\left\{E_1,E_2,\ldots,E_{|\Ecal|}\right\}$ und sei ohne
  Beschränkung der Allgemeinheit 
  $\Ecal(\Omega)\coloneqq\left\{E_1,E_2,\ldots,E_{|\Ecal(\Omega)|}\right\}$.  
  Dann ist $J\coloneqq\{|\Ecal(\Omega)|+1,|\Ecal(\Omega)|+2,\ldots,|\Ecal|\}$
  die Menge der Indizes der Randkanten in $\Ecal$.
  Damit definieren wir die Matrix $\bar
  A\in\Rbb^{|\Ecal(\Omega)|\times|\Ecal(\Omega)|}$, die durch Streichen der
  Zeilen und Spalten von $A$ mit den Indizes aus $J$ entsteht, die Matrix $\bar
  M\in\Rbb^{|\Ecal(\Omega)|\times|\Ecal(\Omega)|}$, die ebenso aus $M$
  hervorgeht und den Vektor $\bar b\in\Rbb^{|\Ecal(\Omega)|}$, der durch
  Streichen der Komponenten von $b$ mit Indizes in $J$ entsteht.
  Weiterhin sei $x\in\Rbb^{|\Ecal(\Omega)|}$, wobei $x_k$ für alle
  $k\in\{1,2\ldots,|\Ecal(\Omega)|\}$ der Koeffizient der Lösung 
  $u_j\in\CR^1_0(\Tcal)$ des
  Gleichungssystems \eqref{eq:linSysPrimalDualAlg} zur $k$-ten Basisfunktion
  von $\CR^1_0(\Tcal)$ sei, das heißt es gelte
  \begin{align*}
    u_j=\sum_{k=1}^{|\Ecal(\Omega)|} x_k\psi_{E_k}.
  \end{align*}
  Da wir das Gleichungssystem \eqref{eq:linSysPrimalDualAlg} in
  $\CR^1_0(\Tcal)$ lösen, lässt sich somit $u_j$ durch Lösen einer
  Matrixgleichung nach $x$ bestimmen. 
  Diese lautet
  \begin{align}
    \label{eq:linSysPrimalDualAlgMatrixEq}
    \left(\frac{1}{\tau}\bar A+\alpha \bar M\right)x=\bar b.
  \end{align}
\end{remark}

\section{Konvergenz der Iteration}
In diesem Abschnitt beweisen wir die Konvergenz der Iterate von
\Cref{alg:primalDualIteration} gegen die Lösung von
\Cref{prob:discreteProblem}. 
Dabei bedienen wir uns unter anderem der äquivalenten Charakterisierungen aus
\Cref{thm:discProbCharacterizationOfDiscreteSolutions}.

\begin{theorem}
  Sei $\ucr\in \CR^1_0(\Tcal)$ Lösung von \Cref{prob:discreteProblem} und
  $\bar\Lambda_0\in P_0\!\left(\Tcal;\Rbb^2\right)$ erfülle
  $\left|\bar\Lambda_0(\bullet)\right|\leq 1$ fast überall in $\Omega$ sowie
  \Cref{eq:discreteMultiplierScalerProductEquality} und
  \Cref{eq:discreteMultiplierL2Equality} aus
  \Cref{thm:discProbCharacterizationOfDiscreteSolutions} mit
  $\tilde{u}_\CR=\ucr$.
  Falls $\tau \in (0, 1]$, dann konvergieren die Iterate $(u_j)_{j\in\Nbb}$ von
  \Cref{alg:primalDualIteration} in $L^2(\Omega)$ gegen $\ucr.$
\end{theorem}

\begin{proof}
  Der Beweis folgt einer Skizze von Professor Carstensen.
  
  Sei $j\in\Nbb$. 
  Seien weiterhin $u_0$, $\Lambda_0$ und $v_0$ sowie $\tilde{u}_j$,
  $\Lambda_j$, $u_j$ und $v_j$ definiert wie in \Cref{alg:primalDualIteration}.
  Außerdem definieren wir $\mu_j\coloneqq \max\{1,|\Lambda_{j-1}+\tau \gradnc
  \tilde{u}_j|\}$ und für alle $k\in\Nbb_0$ die Abkürzungen $e_k \coloneqq
  \ucr-u_k$, $E_k\coloneqq \bar\Lambda_0-\Lambda_k$.
  Dabei nutzen wir die Konvention $e_{-1}\coloneqq e_0$.
  Wir testen zunächst \eqref{eq:linSysPrimalDualAlg} mit $e_j$ und formen das
  Resultat um. 
  Damit erhalten wir
  \begin{align*}
    \anc(v_j,e_j) + \alpha(u_j,e_j) 
    + (\Lambda_j,\gradnc e_j)
    = 
    (f,e_j).
  \end{align*}
  Zusammen mit \Cref{eq:discreteMultiplierL2Equality} folgt daraus
  \begin{equation}
    \label{eq:convProofE}
    \begin{aligned}
      \anc(v_j,e_j) &= 
      \alpha(\ucr-u_j,e_j) 
      + \left(\bar\Lambda_0-\Lambda_j,\gradnc e_j\right) 
      = 
      \alpha\Vert e_j\Vert^2 + \left(E_j,\gradnc e_j\right).
    \end{aligned}
  \end{equation}
  Als Nächstes betrachten wir \Cref{eq:primalDualAlgLambdaJ}. Es gilt
  \begin{align}
    \label{eq:convProofA}
    \Lambda_{j-1}-\Lambda_j+\tau \gradnc \tilde{u}_j 
    = (\mu_j-1)\Lambda_j \quad\text{fast überall in }\Omega.
  \end{align}
  Außerdem folgt aus \Cref{eq:primalDualAlgLambdaJ} und einer
  Fallunterscheidung zwischen $1\geq |\Lambda_{j-1}+\tau\gradnc \tilde{u}_j|$
  und $1< |\Lambda_{j-1}+\tau\gradnc \tilde{u}_j|$, dass
  \begin{align}
    \label{eq:convergenceIterationMuProductZero}
    \left(1-|\Lambda_j|\right)(\mu_j-1)=0
    \quad\text{fast überall in } \Omega.
  \end{align}
  Testen wir nun \Cref{eq:convProofA} in $L^2(\Omega)$ mit $E_j$, erhalten wir 
  unter Nutzung von $\mu_j\geq 1$, der Cauchy-Schwarzschen Ungleichung,
  $\left|\bar\Lambda_0(\bullet)\right|\leq 1$ fast überall in $\Omega$ und
  \Cref{eq:convergenceIterationMuProductZero}, dass
  \begin{align*}
    \left( \Lambda_{j-1}-\Lambda_j+\tau\gradnc \tilde{u}_j, E_j\right)
    &= 
    \left( (\mu_j-1)\Lambda_j,\bar\Lambda_0-\Lambda_j\right)\\
    &\leq
    \int_\Omega (\mu_j-1)\left(|\Lambda_j|-|\Lambda_j|^2\right)\dx\\
    &=
    \int_\Omega |\Lambda_j| (1-|\Lambda_j|)(\mu_j-1)\dx 
    =
    0.
  \end{align*}
  Daraus folgt mit $\Lambda_{j-1}-\Lambda_j = E_j-E_{j-1}$ und $\tilde{u}_j =
  u_{j-1}-(e_{j-1}-e_{j-2})$, dass nach Division durch $\tau$ gilt
  \begin{align}
    \label{eq:convProofB}
    \left(\frac{E_j-E_{j-1}}{\tau}+ \gradnc u_{j-1}-\gradnc
    (e_{j-1}-e_{j-2}),E_j\right)\leq 0.
  \end{align}
  Aus der Cauchy-Schwarzschen Ungleichung,
  \Cref{eq:discreteMultiplierScalerProductEquality} und
  $|\Lambda_j(\bullet)|\leq 1$ fast überall in $\Omega$ folgt, dass
  \begin{align*}
    \gradnc\ucr\cdot E_j 
    &=
    \gradnc\ucr\cdot\bar\Lambda_0 - \gradnc\ucr\cdot\Lambda_j\\
    &\geq 
    \gradnc\ucr\cdot\bar\Lambda_0 - |\gradnc\ucr||\Lambda_j| \\
    &= 
    |\gradnc\ucr|(1-|\Lambda_j|)
    \geq
    0\quad\text{fast überall in }\Omega.
  \end{align*}
  Daraus folgt
  \begin{align}
    \label{eq:convProofC}
    (\gradnc\ucr,E_j)=\int_\Omega \gradnc\ucr\cdot E_j\dx\geq 0.
  \end{align}
  Aus den Ungleichungen \eqref{eq:convProofB} und \eqref{eq:convProofC} folgt
  insgesamt
  \begin{align*}
    \left( \frac{E_j-E_{j-1}}{\tau}+ \gradnc u_{j-1}
    -\nabla_\nc(e_{j-1}-e_{j-2}),E_j\right)
    \leq
    (\gradnc\ucr,E_j).
  \end{align*}
  Das ist äquivalent zu
  \begin{align}
    \label{eq:convProofD}
    \left( \frac{E_j-E_{j-1}}{\tau} 
    -\gradnc(2e_{j-1}-e_{j-2}),E_j\right)\leq 0.
  \end{align}
  Weiterhin gilt
  \begin{equation}
    \begin{aligned}
      \label{eq:longVvvertFormula}
      &\vvvert e_j \vvvert^2_\nc   -
      \vvvert e_{j-1}\vvvert_\nc^2 +
      \Vert E_j \Vert^2 - \Vert E_{j-1}\Vert^2 +
      \vvvert e_j-e_{j-1}\vvvert_\nc^2 +
      \Vert E_j - E_{j-1} \Vert^2\\
      &=
      2a_\nc(e_j,e_j-e_{j-1}) + 2(E_j,E_j-E_{j-1}).
    \end{aligned}
  \end{equation}
  Unter Nutzung von $e_j-e_{j-1}=-\tau v_j$ und \Cref{eq:convProofE} 
  gilt außerdem
  \begin{align*}
    2a_\nc(e_j,e_j-e_{j-1}) + 2(E_j,E_j-E_{j-1})
    &=
    -2\tau a_\nc(e_j,v_j) + 2(E_j,E_j-E_{j-1})\\
    &=
    -2\tau\alpha\Vert e_j\Vert^2 + 2\tau\left(E_j,
    -\nabla_\nc e_j+\frac{E_j-E_{j-1}}{\tau}\right) 
  \end{align*}
  Daraus folgt durch Ungleichung \eqref{eq:convProofD} zusammen mit $\tau>0$,
  dass
  \begin{align*}
    &2a_\nc(e_j,e_j-e_{j-1}) + 2(E_j,E_j-E_{j-1})\\
    &\leq
    -2\tau\alpha\Vert e_j\Vert^2 + 2\tau\left(E_j,
    -\nabla_\nc e_j+\frac{E_j-E_{j-1}}{\tau}\right)\\
    &\quad\quad-2\tau\left( \frac{E_j-E_{j-1}}{\tau}
    -\gradnc(2e_{j-1}-e_{j-2}),E_j\right)\\
    &=
    -2\tau\alpha\Vert e_j\Vert^2 - 
    2\tau\big(E_j,\gradnc(e_j-2e_{j-1}+e_{j-2})\big).
  \end{align*}
  Damit und mit \Cref{eq:longVvvertFormula} erhalten wir insgesamt
  \begin{align*}
    &\vvvert e_j \vvvert^2_\nc   -
    \vvvert e_{j-1}\vvvert_\nc^2 +
    \Vert E_j \Vert^2 - \Vert E_{j-1}\Vert^2 +
    \vvvert e_j-e_{j-1}\vvvert_\nc^2 +
    \Vert E_j - E_{j-1} \Vert^2\\
    &\leq
    -2\tau\alpha\Vert e_j\Vert^2 - 
    2\tau\big(E_j,\gradnc(e_j-2e_{j-1}+e_{j-2})\big).
  \end{align*}
  Für jedes $J\in\Nbb$ führt die Summation dieser Ungleichung über
  $j=1,\ldots,J$ und eine Äquivalenzumfomung zu
  \begin{equation}
    \label{eq:convProofF}
    \begin{aligned}
      &\vvvert e_J \vvvert^2_\nc +\Vert E_J \Vert^2 
      +\sum_{j=1}^J\left(\vvvert e_j-e_{j-1} \vvvert_\nc^2 + 
      \Vert E_j-E_{j-1}\Vert^2\right)\\
      &\leq 
      \vvvert e_0 \vvvert_\nc^2 + \Vert E_0 \Vert^2 
      -2\tau\alpha\sum_{j=1}^J \Vert e_j\Vert^2 
      -2\tau \sum_{j=1}^J\big(E_j,\gradnc
      (e_j-2e_{j-1}+e_{j-2})\big).
    \end{aligned}
  \end{equation}
  Für die letzte Summe auf der rechten Seite dieser Ungleichung gilt, unter
  Beachtung von $e_{-1}=e_0$, dass
  \begin{align*}
    &\sum_{j=1}^J\big(E_j,\gradnc
    (e_j-2e_{j-1}+e_{j-2})\big)\\
    &=\sum_{j=1}^J\big(E_j,\gradnc(e_j-e_{j-1})\big)
    -
    \sum_{j=0}^{J-1}\big(E_{j+1},\gradnc(e_j-e_{j-1})\big) \\
    &= 
    \sum_{j=1}^{J-1} 
    \big(E_j-E_{j+1},\gradnc(e_j-e_{j-1})\big)
    +\big(E_J,\gradnc(e_J-e_{J-1})\big)
    - \big(E_1, \gradnc(e_0-e_{-1})\big) \\
    &= 
    \sum_{j=1}^{J-1} 
    \big(E_j-E_{j+1},\gradnc(e_j-e_{j-1})\big)
    +\big(E_J,\gradnc(e_J-e_{J-1})\big).
  \end{align*}
  Mit dieser Umformung erhalten wir aus Ungleichung \eqref{eq:convProofF} für
  jedes $\tau\in(0,1]$, das heißt $\tau^{-1}\geq 1$, dass
  \begin{equation}
    \label{eq:convProofG}
    \begin{aligned}
      &\vvvert e_J \vvvert^2_\nc +\Vert E_J \Vert^2 
      +\sum_{j=1}^J\left(\vvvert e_j-e_{j-1} \vvvert_\nc^2 + 
      \Vert E_j-E_{j-1}\Vert^2\right) \\
      &\leq 
      \tau^{-1}\left(\vvvert e_0 \vvvert_\nc^2 + \Vert E_0 \Vert^2 \right)
      -2\alpha\sum_{j=1}^J \Vert e_j\Vert^2 \\
      &\quad\quad
      -2 \sum_{j=1}^{J-1} \big(E_j-E_{j+1},\gradnc(e_j-e_{j-1})\big)
      -2\big(E_J,\gradnc(e_J-e_{J-1})\big).
    \end{aligned}
  \end{equation}
  Außerdem gilt
  \begin{align*}
    2\alpha\sum_{j=1}^J\Vert e_j\Vert^2 
    &\leq
    2\alpha\sum_{j=1}^J\Vert e_j\Vert^2
    +\Vert E_J + \gradnc(e_J-e_{J-1}) \Vert^2 
    + \vvvert e_J \vvvert^2_\nc 
    + \Vert E_1 - E_0 \Vert^2 \\
    &\quad\quad
    + \sum_{j=1}^{J-1}  
      \Vert \gradnc(e_j-e_{j-1}) - (E_{j+1} - E_j ) \Vert^2 \\
    &= 
    2\alpha\sum_{j=1}^J\Vert e_j\Vert^2
    +\vvvert e_J \vvvert^2_\nc + \Vert E_J \Vert^2 
    + \sum_{j=1}^J \left( \vvvert e_j-e_{j-1} \vvvert^2_\nc
    + \Vert E_j - E_{j-1} \Vert^2 \right)\\
    &\quad\quad
    + 2\sum_{j=1}^{J-1}\big(E_j-E_{j+1},\gradnc(e_j-e_{j-1})\big)
    + 2\big(E_{J},\gradnc(e_J-e_{J-1})\big).
  \end{align*}
  Zusammen mit Ungleichung \eqref{eq:convProofG} folgt daraus
  \begin{align}
    \label{eq:updateBound}
    2\alpha\sum_{j=1}^J\Vert e_j\Vert^2 
    \leq
    \tau^{-1}\left(\vvvert e_0\vvvert^2_\nc + \Vert E_0\Vert^2\right).
  \end{align}
  Somit gilt, dass $\sum_{j=1}^\infty \Vert e_j\Vert^2$ beschränkt
  ist, was impliziert $\Vert\ucr-u_j\Vert=\Vert e_j\Vert\rightarrow 0$ für
  $j\rightarrow \infty$.
\end{proof}
