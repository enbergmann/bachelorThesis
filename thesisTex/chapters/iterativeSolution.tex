\section{Primale-duale Iteration}

In diesen Abschnitt präsentieren wir ein iteratives Verfahren mit den wir
\Cref{prob:discreteProblem} numerisch lösen möchten. 
Wir nutzen  die primale-duale Iteration \cite[S. 314, Algorithm
10.1]{Bar15} unter Beachtung von \cite[S. 314, Remark 10.11]{Bar15}. 
Diese realisiert Gradientenverfahren zum Finden eines Sattelpunkts, dessen 
Komponenten die primale und die duale Formulierung des Minimierungsproblems
lösen. Details dazu und weitere Referenzen finden sich in
\cites{Bar12}[S. 118-121]{Bar15}.
\todo{ascent and descent flow, gradient flow heißt doch, dass
Gradientenverfahren angewendet wurde, richtig?
Stopping criteria sind in Punkt 6.2}

Angepasst an unser Problem und die Notation dieser Arbeit lautet der
Algorithmus wie folgt.

\begin{algorithm}[Primale-duale Iteration]
  \label{alg:primalDualIteration}
\begin{algorithmic}\\
  \Require $\left(u_0,\Lambda_0\right)
  \in\textup{CR}_0^1(\mathcal{T})\times P_0\!\left(\mathcal{T}; 
  \left\{w\in\Rbb^2\,\middle|\,|w|\leq 1\right\}\right),
  \tau>0$  \\
  Initialisiere $v_0\coloneqq 0$ in $\textup{CR}^1_0(\mathcal T)$.
  \For{$j = 1,2,\dots$}
  \begin{equation}
    \label{eq:primalDualAlgUj}
    \tilde{u}_j\coloneqq u_{j-1}+\tau v_{j-1},
  \end{equation}
  \begin{equation}
    \label{eq:primalDualAlgLambdaJ}
    \Lambda_j
    \coloneqq
    \frac{\Lambda_{j-1}+\tau\nabla_{\textup{NC}} \tilde{u}_j}
    {\max\left\{1,
    \left|\Lambda_{j-1}+\tau\nabla_{\textup{NC}}\tilde{u}_j\right|\right\}},
  \end{equation}
      \State
  \State bestimme $u_j\in\textup{CR}^1_0(\mathcal{T})$
  als Lösung des linearen Gleichungssystems
  \begin{align}
    \label{eq:linSysPrimalDualAlg}
    \frac{1}{\tau}a_{\textup{NC}}(u_j,\bullet)+\alpha(u_j,\bullet)
    &=
    \frac{1}{\tau}a_{\textup{NC}}(u_{j-1},\bullet) + (f,\bullet)
    - \left(\Lambda_j,\nabla_{\textup{NC}}\bullet\right) 
    \quad\text{in }\CR^1_0(\Tcal),
  \end{align}
  \begin{equation*}
    v_j\coloneqq\frac{u_j-u_{j-1}}{\tau}.
  \end{equation*}
  \EndFor
  \Ensure Folge $(u_j,\Lambda_j)_{j\in\mathbb N}$ in
  $\CR^1_0(\mathcal{T})\times
   P_0\!\left(\mathcal{T};\left\{w\in\Rbb^2\,\middle|\,|w|\leq
   1\right\}\right)$   
  \end{algorithmic}
\end{algorithm}

\section{Konvergenz der Iteration}
In diesen Abschnitt beweisen wir die Konvergenz der Iterate von
\Cref{alg:primalDualIteration} gegen die Lösung von
\Cref{prob:discreteProblem}. 
Dabei bedienen wir uns unter anderem der äquivalenten Charakterisierungen aus
\Cref{thm:discProbCharacterizationOfDiscreteSolutions}.

\begin{theorem}
  Sei $\ucr\in \CR^1_0(\Tcal)$ Lösung von \Cref{prob:discreteProblem} und
  $\bar\Lambda\in P_0\!\left(\Tcal;\Rbb^2\right)$ erfülle
  $\left|\bar\Lambda(\bullet)\right|\leq 1$ fast überall in $\Omega$,
  \Cref{eq:discreteMultiplierScalerProductEquality} und
  \Cref{eq:discreteMultiplierL2Equality}.

  Falls $0 < \tau \leq 1$, dann konvergieren die Iterate $(u_j)_{j\in\Nbb}$ von
  \Cref{alg:primalDualIteration} in $L^2(\Omega)$ gegen $\ucr.$
\end{theorem}

\begin{proof}
  Der Beweis folgt einer Skizze von Professor Carstensen.
  
  Sei $j\in\Nbb$. 
  Seien weiterhin $\tilde{u}_j$, $v_j$ und $\Lambda_j$ definiert wie in
  \Cref{alg:primalDualIteration}.
  Außerdem definieren wir die Abkürzungen $e_j \coloneqq \ucr-u_j$,
  $E_j\coloneqq \bar\Lambda-\Lambda_j$ und $\mu_j\coloneqq
  \max\{1,|\Lambda_{j-1}+\tau \gradnc \tilde{u}_j|\}$.
  Dabei nutzen wir die Konvention $e_{-1}\coloneqq e_0$.

  Wir testen zunächst \eqref{eq:linSysPrimalDualAlg} mit $e_j$ und formen das
  Resultat um. 
  Damit erhalten wir
  \begin{align*}
    \anc(v_j,e_j) + \alpha(u_j,e_j) 
    + (\Lambda_j,\gradnc e_j)
    = 
    (f,e_j).
  \end{align*}
  Zusammen mit \Cref{eq:discreteMultiplierL2Equality} folgt daraus
  \begin{equation}
    \label{eq:convProofE}
    \begin{aligned}
      \anc(v_j,e_j) &= 
      \alpha(\ucr-u_j,e_j) 
      + \left(\bar\Lambda-\Lambda_j,\gradnc e_j\right) \\
      &= 
      \alpha\Vert e_j\Vert^2 + \left(E_j,\gradnc e_j\right).
    \end{aligned}
  \end{equation}

  Als Nächstes betrachten wir \Cref{eq:primalDualAlgLambdaJ}. Es gilt
  \begin{align}
    \label{eq:convProofA}
    \Lambda_{j-1}-\Lambda_j+\tau \gradnc \tilde{u}_j 
    = (\mu_j-1)\Lambda_j \quad\text{fast überall in }\Omega.
  \end{align}
  Außerdem folgt aus \Cref{eq:primalDualAlgLambdaJ} und einer einfachen
  Fallunterscheidung zwischen $1\geq |\Lambda_{j-1}+\tau\gradnc \tilde{u}_j|$
  und $1< |\Lambda_{j-1}+\tau\gradnc \tilde{u}_j|$, dass
  \begin{align}
    \label{eq:convergenceIterationMuProductZero}
    \left(1-|\Lambda_j|\right)(\mu_j-1)=0
    \quad\text{fast überall in } \Omega.
  \end{align}

  Testen wir nun \Cref{eq:convProofA} in $L^2(\Omega)$ mit $E_j$, erhalten wir 
  unter Nutzung von $\mu_j\geq 1$, der Cauchy-Schwarzschen Ungleichung,
  $\left|\bar\Lambda(\bullet)\right|\leq 1$ fast überall in $\Omega$ und
  \Cref{eq:convergenceIterationMuProductZero}, dass
  \begin{align*}
    \left( \Lambda_{j-1}-\Lambda_j+\tau\gradnc \tilde{u}_j, E_j\right)
    &= 
    \left( (\mu_j-1)\Lambda_j,\bar\Lambda-\Lambda_j\right)\\
    &=
    \int_\Omega
    (\mu_j-1)(\Lambda_j\cdot\bar\Lambda-\Lambda_j\cdot\Lambda_j)\,\mathrm dx\\
    &\leq
    \int_\Omega (\mu_j-1)(|\Lambda_j|-|\Lambda_j|^2)\,\mathrm dx\\
    &=
    \int_\Omega |\Lambda_j| (1-|\Lambda_j|)(\mu_j-1)\,\mathrm dx \\
    &=
    0.
  \end{align*}
  Daraus folgt mit $\Lambda_{j-1}-\Lambda_j = E_j-E_{j-1}$ und $\tilde{u}_j =
  u_{j-1}-(e_{j-1}-e_{j-2})$, dass nach Division durch $\tau$ gilt
  \begin{align}
    \label{eq:convProofB}
    \left(\frac{E_j-E_{j-1}}{\tau}+ \gradnc u_{j-1}-\gradnc
    (e_{j-1}-e_{j-2}),E_j\right)\leq 0.
  \end{align}

  Aus der Cauchy-Schwarzschen Ungleichung,
  $\bar\Lambda(x)\in\sign\left(\gradnc\ucr(x)\right)$ für alle
  $x\in\interior(T)$ für alle $T\in\Tcal$ und $|\Lambda_j(\bullet)|\leq
  1$ fast überall in $\Omega$ folgt, dass
  \begin{align*}
    \gradnc\ucr\cdot E_j 
    &=
    \gradnc\ucr\cdot\bar\Lambda - \gradnc\ucr\cdot\Lambda_j\\
    &\geq 
    \gradnc\ucr\cdot\bar\Lambda - |\gradnc\ucr||\Lambda_j| \\
    &= 
    |\gradnc\ucr|(1-|\Lambda_j|)\\
    &\geq
    0\quad\text{fast überall in }\Omega.
  \end{align*}
  Daraus folgt
  \begin{align}
    \label{eq:convProofC}
    (\gradnc\ucr,E_j)=\int_\Omega \gradnc\ucr\cdot E_j\dx\geq 0.
  \end{align}

  Aus den Ungleichungen \eqref{eq:convProofB} und \eqref{eq:convProofC} folgt
  insgesamt
  \begin{align*}
    \left( \frac{E_j-E_{j-1}}{\tau}+ \gradnc u_{j-1}
    -\nabla_\nc(e_{j-1}-e_{j-2}),E_j\right)
    \leq
    (\gradnc\ucr,E_j).
  \end{align*}
  Das ist äquivalent zu
  \begin{align}
    \label{eq:convProofD}
    \left( \frac{E_j-E_{j-1}}{\tau} 
    -\gradnc(2e_{j-1}-e_{j-2}),E_j\right)\leq 0.
  \end{align}

  Unter Nutzung von $-\tau v_j=e_j-e_{j-1}$, \Cref{eq:convProofE} und
  Ungleichung \eqref{eq:convProofD} zusammen mit $\tau>0$ erhalten wir 
  \begin{align*}
    &\vvvert e_j \vvvert^2_\nc   -
    \vvvert e_{j-1}\vvvert_\nc^2 +
    \Vert E_j \Vert^2 - \Vert E_{j-1}\Vert^2 +
    \vvvert e_j-e_{j-1}\vvvert_\nc^2 +
    \Vert E_j - E_{j-1} \Vert^2\\
    &\quad =
    2a_\nc(e_j,e_j-e_{j-1}) + 2(E_j,E_j-E_{j-1})\\
    &\quad =
    -2\tau a_\nc(e_j,v_j) + 2(E_j,E_j-E_{j-1})\\
    &\quad =
    -2\tau\alpha\Vert e_j\Vert^2 + 2\tau\left(E_j,
    -\nabla_\nc e_j+\frac{E_j-E_{j-1}}{\tau}\right) \\
    &\quad \leq
    -2\tau\alpha\Vert e_j\Vert^2 + 2\tau\left(E_j,
    -\nabla_\nc e_j+\frac{E_j-E_{j-1}}{\tau}\right)\\ 
    &\quad\quad -2\tau\left( \frac{E_j-E_{j-1}}{\tau}
    -\gradnc(2e_{j-1}-e_{j-2}),E_j\right)\\
    &\quad =
    -2\tau\alpha\Vert e_j\Vert^2 - 
    2\tau\big(E_j,\gradnc(e_j-2e_{j-1}+e_{j-2})\big).
  \end{align*}
  Für jedes $J\in\Nbb$ führt die Summation dieser Ungleichung über
  $j=1,\ldots,J$ und eine Äquivalenzumfomung zu
  \begin{equation}
    \label{eq:convProofF}
    \begin{aligned}
      &\vvvert e_J \vvvert^2_\nc +\Vert E_J \Vert^2 
      +\sum_{j=1}^J\left(\vvvert e_j-e_{j-1} \vvvert_\nc^2 + 
      \Vert E_j-E_{j-1}\Vert^2\right)\\
      &\quad \leq 
      \vvvert e_0 \vvvert_\nc^2 + \Vert E_0 \Vert^2 
      -2\tau\alpha\sum_{j=1}^J \Vert e_j\Vert^2 
      -2\tau \sum_{j=1}^J\big(E_j,\gradnc
      (e_j-2e_{j-1}+e_{j-2})\big).
    \end{aligned}
  \end{equation}
  Die letzte Summe auf der rechten Seite dieser Ungleichung können wir, unter
  Beachtung von $e_{-1}=e_0$, umformen durch
  \begin{align*}
    &\sum_{j=1}^J\big(E_j,\gradnc
    (e_j-2e_{j-1}+e_{j-2})\big)\\
    &\quad=\sum_{j=1}^J\big(E_j,\gradnc(e_j-e_{j-1})\big)
    -
    \sum_{j=0}^{J-1}\big(E_{j+1},\gradnc(e_j-e_{j-1})\big) \\
    &\quad = 
    \sum_{j=1}^{J-1} 
    \big(E_j-E_{j+1},\gradnc(e_j-e_{j-1})\big)
    +\big(E_J,\gradnc(e_J-e_{J-1})\big)
    - \big(E_1, \gradnc(e_0-e_{-1})\big) \\
    &\quad = 
    \sum_{j=1}^{J-1} 
    \big(E_j-E_{j+1},\gradnc(e_j-e_{j-1})\big)
    +\big(E_J,\gradnc(e_J-e_{J-1})\big).
  \end{align*}
  Mit dieser Umformung folgt für jedes $0<\tau\leq 1$ aus Ungleichung
  \eqref{eq:convProofF}, dass
  \begin{align*}
    &\tau\left(\vvvert e_J \vvvert^2_\nc +\Vert E_J \Vert^2 
    +\sum_{j=1}^J\left(\vvvert e_j-e_{j-1} \vvvert_\nc^2 + 
    \Vert E_j-E_{j-1}\Vert^2\right)\right) \\
    &\quad \leq 
    \vvvert e_0 \vvvert_\nc^2 + \Vert E_0 \Vert^2 
    -2\tau\alpha\sum_{j=1}^J \Vert e_j\Vert^2 \\
    &\quad\quad
    -2\tau\left( 
    \sum_{j=1}^{J-1} 
    \big(E_j-E_{j+1},\gradnc(e_j-e_{j-1})\big)
    +\big(E_J,\gradnc(e_J-e_{J-1})\big)\right).
  \end{align*}
  Division durch $\tau$ ergibt
  \begin{equation}
    \label{eq:convProofG}
    \begin{aligned}
      &\vvvert e_J \vvvert^2_\nc +\Vert E_J \Vert^2 
      +\sum_{j=1}^J\left(\vvvert e_j-e_{j-1} \vvvert_\nc^2 + 
      \Vert E_j-E_{j-1}\Vert^2\right) \\
      &\quad \leq 
      \tau^{-1}\left(\vvvert e_0 \vvvert_\nc^2 + \Vert E_0 \Vert^2 \right)
      -2\alpha\sum_{j=1}^J \Vert e_j\Vert^2 \\
      &\quad\quad
      -2 \sum_{j=1}^{J-1} \big(E_j-E_{j+1},\gradnc(e_j-e_{j-1})\big)
      -2\big(E_J,\gradnc(e_J-e_{J-1})\big).
    \end{aligned}
  \end{equation}

  Schließlich ergibt eine Abschätzung unter Nutzung von Ungleichung
  \eqref{eq:convProofG}, dass
  \begin{align*}
    &2\alpha\sum_{j=1}^J\Vert e_j\Vert^2 \\
    &\quad\leq
    2\alpha\sum_{j=1}^J\Vert e_j\Vert^2
    +\Vert E_J + \gradnc(e_J-e_{J-1}) \Vert^2 
    + \vvvert e_J \vvvert^2_\nc 
    + \Vert E_1 - E_0 \Vert^2 \\
    &\quad\quad
    + \sum_{j=1}^{J-1}  
      \Vert \gradnc(e_j-e_{j-1}) - (E_{j+1} - E_j ) \Vert^2 \\
    & \quad= 
    2\alpha\sum_{j=1}^J\Vert e_j\Vert^2
    +\vvvert e_J \vvvert^2_\nc + \Vert E_J \Vert^2 
    + \sum_{j=1}^J \left( \vvvert e_j-e_{j-1} \vvvert^2_\nc
    + \Vert E_j - E_{j-1} \Vert^2 \right)\\
    &\quad\quad
    + 2\sum_{j=1}^{J-1}\big(E_j-E_{j+1},\gradnc(e_j-e_{j-1})\big)
    + 2\big(E_{J},\gradnc(e_J-e_{J-1})\big) \\
    &\quad\leq
    \tau^{-1}\left(\vvvert e_0\vvvert^2_\nc + \Vert E_0\Vert^2\right).
  \end{align*}
  Daraus folgt, dass $\sum_{j=1}^\infty \Vert e_j\Vert^2$ nach oben beschränkt
  ist, was impliziert $\Vert\ucr-u_j\Vert=\Vert e_j\Vert\rightarrow 0$ für
  $j\rightarrow \infty$.
\end{proof}


