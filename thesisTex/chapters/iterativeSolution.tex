\section{Primale-duale Iteration}
\todo{hier start}
Mehr zur Herleitung des Algorithmus, siehe BartelsBVPaper S. 1163, schwache
Formulierung der vorletzten Gleichung

Schreibe sowas wie ,Die GLeichungen basieren auf der primalen dualen
Formulierung des Problems, auf einem descent flow \ldots usw,
Details finden sich in barBV, wobei Gleichung (Glg im Alg) die schwache
Formulierung von (Glg in Bartels bzw der Herleitung)'. d.h. Stichworte
abklappern, damit Leute mit Ahnung wissen was los ist, aber für Details
auf andere Leute zeigen.

Weitere Details zur Herleitung auf S. 1168, da noch Details rausschreiben
aber nur grob aber dafür darauf verweisen für die Herleitung. Nur das 
nötigste hier.


Stopping criteria sind in Punkt 6.2


\todo[inline]{Irgendwo, wahrscheinlich bei ,,alles zu $\CR_0$'' muss noch
$\anc(u,v)\coloneqq\int_\Omega\gradnc u\cdot\gradnc v\dx$ erwähnt werden (und
warum das ein SP ist muss angerissen werden, Stichwort Friedrichs Ungleichung)} 

Für unsere Formulierung \Cref{prob:discreteProblem} nutzen wir \cite[S. 314,
Algorithm 10.1]{Bar15} unter Beachtung von \cite[S. 314, Remark 10.11]{Bar15}
als Algorithmus als iterativen Löser und benutzen als inneres Produkt $\anc$
(definiert in Kapitel \ldots hier in dieser Arbeit).
Weitere Details dazu finden sich in \cite[S. 118-121]{Bar15}.
\todo[inline]{Beim Zitieren z.B. 'Remark' lassen, weil es in Bartels so heißt,
oder das lieber übersetzen? Außerdem natürlich, passt das so als Einleitung 
für den Alg?}
\todo{hier ende}

\begin{algorithm}[Primale-duale Iteration]
  \label{alg:primalDualIteration}
\begin{algorithmic}\\
  \Require $u_0\in\textup{CR}_0^1(\mathcal{T}),$
  $\Lambda_0
  \in  P_0\!\left(\mathcal{T}; 
  \left\{w\in\Rbb^2\,\middle|\,|w|\leq 1\right\}\right),
  \tau>0$  \\
  Initialisiere $v_0\coloneqq 0$ in $\textup{CR}^1_0(\mathcal T)$.
  \For{$j = 1,2,\dots$}
  \begin{equation}
    \label{eq:primalDualAlgUj}
    \tilde{u}_j\coloneqq u_{j-1}+\tau v_{j-1},
  \end{equation}
  \begin{equation}
    \label{eq:primalDualAlgLambdaJ}
    \Lambda_j
    \coloneqq
    \left(\Lambda_{j-1}+\tau\nabla_{\textup{NC}} \tilde{u}_j\right)/
      \max\left\{1,
      \left|\Lambda_{j-1}+\tau\nabla_{\textup{NC}}\tilde{u}_j\right|\right\},
  \end{equation}
      \State
  \State bestimme $u_j\in\textup{CR}^1_0(\mathcal{T})$
  als Lösung des linearen Gleichungssystems
  \begin{align}
    \label{eq:linSysPrimalDualAlg}
    \frac{1}{\tau}&a_{\textup{NC}}(u_j,\bullet)+\alpha(u_j,\bullet)
    \notag \\
    &=
    \frac{1}{\tau}a_{\textup{NC}}(u_{j-1},\bullet) + (f,\bullet)
    - (\Lambda_j,\nabla_{\textup{NC}}\bullet) 
  \end{align}
  \State in $\CR^1_0(\mathcal{T})$, \\
  \begin{equation*}
    v_j\coloneqq(u_j-u_{j-1})/\tau.
  \end{equation*}
  \EndFor
  \Ensure Folge $(u_j,\Lambda_j)_{j\in\mathbb N}$ in
  $\CR^1_0(\mathcal{T})\times
   P_0\!\left(\mathcal{T};\left\{w\in\Rbb^2\,\middle|\,|w|\leq
   1\right\}\right)$   
  \end{algorithmic}
\end{algorithm}

\section{Konvergenz der Iteration}
\begin{theorem}
  Sei $\ucr\in \CR^1_0(\Tcal)$ Lösung von \Cref{prob:discreteProblem} und
  $\bar\Lambda\in P_0\!\left(\Tcal;\Rbb^2\right)$ erfülle
  $\left|\bar\Lambda(\bullet)\right|\leq 1$ fast überall in $\Omega$,
  \Cref{eq:discreteMultiplierScalerProductEquality} und
  \Cref{eq:discreteMultiplierL2Equality}.

  Falls $0 < \tau \leq 1$, dann konvergieren die Iterate $(u_j)_{j\in\Nbb}$ von
  \Cref{alg:primalDualIteration}
  gegen $\ucr.$
\end{theorem}

\begin{proof}
  \todo{hier start again}
  Der Beweis folgt einer Skizze von Professor Carstensen.
  
  Seien $\tilde{u}_j$, $v_j$ und $\Lambda_j$ definiert wie in
  \Cref{alg:primalDualIteration}.
  Definiere außerdem $e_j \coloneqq \ucr-u_j$ und $E_j\coloneqq
  \bar\Lambda-\Lambda_j$. 

  Testen wir nun \eqref{eq:linSysPrimalDualAlg} mit $e_j$, erhalten wir
  \begin{align*}
    \anc(v_j,e_j) + \alpha(u_j,e_j) 
    + (\Lambda_j,\gradnc e_j)
    = 
    (f,e_j).
  \end{align*}
  Äquivalent dazu ist, da $\ucr$ \Cref{eq:discreteMultiplierL2Equality} löst, 
  \begin{align}
    \label{eq:convProofE}
    \anc(v_j,e_j) &= 
    \alpha(\ucr-u_j,e_j) 
    + (\bar\Lambda-\Lambda_j,\gradnc e_j) \notag\\
    &= 
    \alpha\Vert e_j\Vert^2
    + (E_j,\gradnc e_j).
  \end{align}
  Sei $\mu_j\coloneqq \max\{1,|\Lambda_{j-1}+\tau
  \gradnc \tilde{u}_j|\}$.
  Nutzen wir \eqref{eq:primalDualAlgLambdaJ} erhalten wir damit
  \begin{align}
    \label{eq:convProofA}
    \Lambda_{j-1}-\Lambda_j+\tau \gradnc \tilde{u}_j 
    = (\mu_j-1)\Lambda_j \quad\text{fast überall in }\Omega.
  \end{align}
  Für fast alle $x\in\Omega$ liefert die CSU, da $|\bar\Lambda|\leq 1$ fast
  überall in $\Omega$,
  $\Lambda_j(x)\cdot\bar\Lambda(x)\leq|\Lambda_j(x)|$ und damit folgt 
  aus \Cref{eq:primalDualAlgLambdaJ} und einer einfachen Fallunterscheidung
  zwischen $1\geq |\Lambda_{j-1}+\tau\gradnc \tilde{u}_j|$ und
  $1< |\Lambda_{j-1}+\tau\gradnc \tilde{u}_j|$,
  dass $(1-|\Lambda_j(x)|)(\mu_j(x)-1)=0$.
  Testen wir nun \eqref{eq:convProofA} mit $E_j$, erhalten wir 
  unter Nutzung von $\mu_j\geq 1$ und der CSU damit
  \begin{align*}
    ( \Lambda_{j-1}-\Lambda_j+\tau\gradnc \tilde{u}_j,
    E_j)
    &= 
    ( (\mu_j-1)\Lambda_j,\bar\Lambda-\Lambda_j)\\
    &=
    \int_\Omega
    (\mu_j-1)(\Lambda_j\cdot\bar\Lambda-\Lambda_j\cdot\Lambda_j)\,\mathrm dx\\
    &\leq
    \int_\Omega (\mu_j-1)(|\Lambda_j|-|\Lambda_j|^2)\,\mathrm dx\\
    &=
    \int_\Omega |\Lambda_j|
    (1-|\Lambda_j|)(\mu_j-1)\,\mathrm dx \\
    &=
    \int_\Omega |\Lambda_j|\cdot
    0\,\mathrm dx =0.
  \end{align*}
  Damit und mit $\Lambda_{j-1}-\Lambda_j=E_j-E_{j-1}$, 
  $\tilde{u}_j=u_{j-1}+\tau v_{j-1}=u_{j-1}+u_{j-1}-u_{j-2}=
  u_{j-1}-(e_{j-1}-e_{j-2})$
  für $j\geq 2$ und der Konvention $e_{-1}\coloneqq e_0$ für $j=1$ erhalten wir
  insgesamt
  \begin{align}
    \label{eq:convProofB}
    \left(\frac{E_j-E_{j-1}}{\tau}+ \gradnc u_{j-1}-\gradnc
    (e_{j-1}-e_{j-2}),E_j\right)\leq 0\quad\text{für alle }
    j\in\Nbb.
  \end{align}
  Falls $|\gradnc \ucr|\neq 0$, gilt somit zusammen mit
  der CSU, $\bar\Lambda\in\sign\gradnc\ucr$ und $|\Lambda_j|\leq 1$,  dass
  \begin{align*}
    \gradnc\ucr\cdot E_j 
    &=
    \gradnc\ucr\cdot\bar\Lambda - \gradnc\ucr\cdot\Lambda_j\\
    &\geq 
    \gradnc\ucr\cdot\bar\Lambda - |\gradnc\ucr||\Lambda_j| \\
    &=
    |\gradnc\ucr|^2/|\gradnc\ucr|-|\gradnc\ucr||\Lambda_j| \\
    &= 
    |\gradnc\ucr|(1-|\Lambda_j|)\\
    &\geq
    0. 
  \end{align*}
  Falls $|\gradnc\ucr|=0$, gilt diese Ungleichung ebenfalls.
  Daraus folgt
  \begin{align}
    \label{eq:convProofC}
    (\gradnc\ucr,E_j)=\int_\Omega \gradnc\ucr\cdot E_j\dx\geq 0.
  \end{align}
  Mit \eqref{eq:convProofB} und \eqref{eq:convProofC} folgt nun
  \begin{align*}
    \left( \frac{E_j-E_{j-1}}{\tau}+ \gradnc u_{j-1}
    -\nabla_\nc(e_{j-1}-e_{j-2}),E_j\right)
    \leq
    (\gradnc\ucr,E_j)
  \end{align*}
  für alle $j\in\Nbb$, was nach Definition von $e_{j-1}$
  äquivalent ist zu
  \begin{align}
    \label{eq:convProofD}
    \left( \frac{E_j-E_{j-1}}{\tau} 
    -\gradnc(2e_{j-1}-e_{j-2}),E_j\right)\leq 0.
  \end{align}
  Unter Nutzung von $-v_j=(e_j-e_{j-1})/\tau$, \eqref{eq:convProofE}, $\tau>0$
  und \eqref{eq:convProofD} erhalten wir 
  \begin{align*}
    &\vvvert e_j \vvvert^2_\nc   -
    \vvvert e_{j-1}\vvvert_\nc^2 +
    \Vert E_j \Vert^2 - \Vert E_{j-1}\Vert^2 +
    \vvvert e_j-e_{j-1}\vvvert_\nc^2 +
    \Vert E_j - E_{j-1} \Vert^2\\
    &\quad =
    2a_\nc(e_j,e_j-e_{j-1}) + 2(E_j,E_j-E_{j-1})\\
    &\quad =
    -2\tau a_\nc(e_j,v_j) + 2(E_j,E_j-E_{j-1})\\
    &\quad =
    -2\tau\alpha\Vert e_j\Vert^2 + 2\tau\left(E_j,
    -\nabla_\nc e_j+\frac{E_j-E_{j-1}}{\tau}\right) \\
    &\quad \leq
    -2\tau\alpha\Vert e_j\Vert^2 + 2\tau\left(E_j,
    -\nabla_\nc e_j+\frac{E_j-E_{j-1}}{\tau}\right)\\ 
    &\quad\quad -2\tau\left( \frac{E_j-E_{j-1}}{\tau}
    -\gradnc(2e_{j-1}-e_{j-2}),E_j\right)\\
    &\quad =
    -2\tau\alpha\Vert e_j\Vert^2 - 
    2\tau\big(E_j,\gradnc(e_j-2e_{j-1}+e_{j-2})\big).
  \end{align*}
  Für jedes $J\in\Nbb$ führt die Summation über $j=1,\ldots,J$ und eine
  Äquivalenz\-umfomung zu
  \begin{align}
    \label{eq:convProofF}
    &\vvvert e_J \vvvert^2_\nc +\Vert E_J \Vert^2 
    +\sum_{j=1}^J\left(\vvvert e_j-e_{j-1} \vvvert_\nc^2 + 
    \Vert E_j-E_{j-1}\Vert^2\right)\notag \\
    &\quad \leq 
    \vvvert e_0 \vvvert_\nc^2 + \Vert E_0 \Vert^2 
    -2\tau\alpha\sum_{j=1}^J \Vert e_j\Vert^2 \\
    &\quad\quad
    -2\tau \sum_{j=1}^J\big(E_j,\gradnc
    (e_j-2e_{j-1}+e_{j-2})\big).\notag
  \end{align}
  Dabei lässt sich die letzt Summe, unter Beachtung von $e_{-1}=e_0$, umformen
  zu
  \begin{align*}
    &\sum_{j=1}^J\big(E_j,\gradnc
    (e_j-2e_{j-1}+e_{j-2})\big)\\
    &\quad=\sum_{j=1}^J(E_j,\gradnc(e_j-e_{j-1}))
    -
    \sum_{j=0}^{J-1}(E_{j+1},\gradnc(e_j-e_{j-1})) \\
    &\quad = 
    \sum_{j=1}^{J-1} 
    \big(E_j-E_{j+1},\gradnc(e_j-e_{j-1})\big)
    +(E_J,\gradnc(e_J-e_{J-1}))\\
    &\quad\quad 
    - (E_1, \gradnc(e_0-e_{-1})) \\
    &\quad = 
    \sum_{j=1}^{J-1} 
    \big(E_j-E_{j+1},\gradnc(e_j-e_{j-1})\big)
    +(E_J,\gradnc(e_J-e_{J-1}))
  \end{align*}
  und da die linke Seite von
  \eqref{eq:convProofF} nicht negativ ist, gilt damit
  für jedes $0<\tau\leq 1$, dass
  \begin{align*}
    &\tau\left(\vvvert e_J \vvvert^2_\nc +\Vert E_J \Vert^2 
    +\sum_{j=1}^J\left(\vvvert e_j-e_{j-1} \vvvert_\nc^2 + 
    \Vert E_j-E_{j-1}\Vert^2\right)\right) \\
    &\quad \leq 
    \vvvert e_0 \vvvert_\nc^2 + \Vert E_0 \Vert^2 
    -2\tau\alpha\sum_{j=1}^J \Vert e_j\Vert^2 \\
    &\quad\quad
    -2\tau\left( 
    \sum_{j=1}^{J-1} 
    (E_j-E_{j+1},\gradnc(e_j-e_{j-1}))
    +(E_J,\gradnc(e_J-e_{J-1}))\right).
  \end{align*}
  Division durch $\tau$ ergibt
  \begin{align}
    \label{eq:convProofG}
    &\vvvert e_J \vvvert^2_\nc +\Vert E_J \Vert^2 
    +\sum_{j=1}^J\left(\vvvert e_j-e_{j-1} \vvvert_\nc^2 + 
    \Vert E_j-E_{j-1}\Vert^2\right) \notag\\
    &\quad \leq 
    \tau^{-1}(\vvvert e_0 \vvvert_\nc^2 + \Vert E_0 \Vert^2 )
    -2\alpha\sum_{j=1}^J \Vert e_j\Vert^2 \\
    &\quad\quad
    -2 \sum_{j=1}^{J-1} (E_j-E_{j+1},\gradnc(e_j-e_{j-1}))
    -2(E_J,\gradnc(e_J-e_{J-1})).\notag
  \end{align}

%  \begin{align*}
%    &2\tau\sum_{j=1}^J(E_j,\gradnc(-e_j+e_{j-1})) +
%    2\tau\sum_{j=0}^{J-1}(E_{j+1},\gradnc(e_j-e_{j-1}))\\
%    &\quad \le
%    2\sum_{j=1}^J(E_j,\gradnc(-e_j+e_{j-1})) +
%    2\sum_{j=0}^{J-1}(E_{j+1},\gradnc(e_j-e_{j-1}))\\
%    &\quad =
%    2\sum_{j=1}^J(E_j,\gradnc(-e_j+e_{j-1})) 
%    +
%    2\sum_{j=1}^{J-1}(E_{j+1},\gradnc(e_j-e_{j-1})) 
%    \tag{since $e_{-1}\coloneqq e_0$}\\
%    &\quad =
%    2\sum_{j=1}^{J-1}(E_{j+1}-E_j,\gradnc(e_j-e_{j-1}))
%    -2(E_{J},\gradnc(e_J-e_{J-1}))
%    .
%  \end{align*}

  Schließlich ergibt eine Abschätzung unter Nutzung von \eqref{eq:convProofG}, 
  dass
  \begin{align*}
    2\alpha\sum_{j=1}^J\Vert e_j\Vert^2 
    &\leq
    2\alpha\sum_{j=1}^J\Vert e_j\Vert^2\\
    &\quad
    +\Vert E_J + \gradnc(e_J-e_{J-1}) \Vert^2 
    + \vvvert e_J \vvvert^2_\nc 
    + \Vert E_1 - E_0 \Vert^2 \\
    &\quad 
    + \sum_{j=1}^{J-1}  
      \Vert \gradnc(e_j-e_{j-1}) - (E_{j+1} - E_j ) \Vert^2 \\
    & = 
    2\alpha\sum_{j=1}^J\Vert e_j\Vert^2\\
    &\quad 
    +\vvvert e_J \vvvert^2_\nc + \Vert E_J \Vert^2 
    + \sum_{j=1}^J ( \vvvert e_j-e_{j-1} \vvvert^2_\nc
    + \Vert E_j - E_{j-1} \Vert^2 )\\
    &\quad
    + 2\sum_{j=1}^{J-1}(E_j-E_{j+1},\gradnc(e_j-e_{j-1}))\\
    &\quad 
    + 2(E_{J},\gradnc(e_J-e_{J-1})) \\
    &\leq
    \tau^{-1}(\vvvert e_0\vvvert^2_\nc + \Vert E_0\Vert^2).
  \end{align*}

  Das zeigt, dass
  $\sum_{j=1}^\infty \Vert e_j\Vert^2$ nach oben beschränkt ist,
  was impliziert, dass $\Vert e_j\Vert\rightarrow 0$
  für $j\rightarrow \infty$.
\end{proof}


