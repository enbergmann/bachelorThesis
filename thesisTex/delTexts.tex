%%%%%%%%%%%%%%%%%%%%%%%%%%%%%%%%%%%%%%%%%%%%%%%%%%%%%%%%%%%%%%%%%%%%%%%%%%%%%
From introduction.tex

write the introduction similar to
[bartelsErrorControlAndAdaptivityForBV], just more details. See below.

In particular, drop infos needed about BV (wir werden sehen, dass BV Funktionen
dies und das erfüllen und deshalb diese und jene Terme definiert sind)
or (wir werden zuerst grundlagen einführen aus der Optimierung und über BV 
Funktionen [dann verweis auf das entsprechende Kapitel], anschließend
das konitnuierliche Problem betrachten [verweis auf kapitel] \ldots)
\bigskip

1. BV Funtionen allgemein: lassen unstetigkeiten zu, mehr als Sobolev 
Funktionen,
[hier nochmal überlegen, wie ich dieses über 'manifold' (one codimensional
(ABM)) springen schreibe, ohne ins Detail gehen zu müssen]
\medskip

deshalb Anwendung zum Beispiel in
\medskip

[\cite{ABM14} modelization of a large number of problems in physics, mechanics,
or image processing requires the introducion of new functionals spaces
permitting discontinuities of the solution. In phase transitions, image
segmentation, plasticity theory, the study of crasks and fissures, the study of
the wake in fluid dynamics , and so forth, the solution of the problem presents
discontinuities along one-odimensionalmanifolds.
-solution of these problems cannot be found in classical Sobolev spaces
\medskip

Viele physikalischen Anwendungen können mit kontinuierlichen Funktionen nicht
beschrieben werden [Bartels, Error Control and Adaptivity for a variational
model problem defined on functions of bounded variation, und darin zitierte
Quellen].
\medskip

introduction aus [Braides] zitieren einfach mit Bemerkung, dass man dort
Beispiele finden kann (vielleicht auch nicht, da sie direkt auf SBV
eingeht und nicht auf BV)
\medskip

insbesondere in der 'Bildbearbeitung' (finde
noch das richtige Wort, Computer Vision, computergestützte Analyse von Bildern
oder so änhlich) wird das genutzt
(Quellen: \cite{AK06}, \ldots)

\cite{Get12} ist mglw ein guter Verweis für weitere Infos zum Denoisung und
wahrscheinlich in Kapitel 6 auch zur Wahl von alpha, sagt
aber auch viel über die Interpretation der Parameter
Wahrscheinlich auch sowas schreiben wie ,Die Anwendung ist kein zentraler
Punkt in dieser Arbeit, sonder \ldots, für weitere Details zum Denoising siehe
\ldots.' Dann im Experimente Kapitel eine kleine Rand-Section, in der geguckt
wird, wie verschiedene Alpha sich auf Denoising auswirken, rein optisch 'nur
ein kleiner experementeller Exkurs zur Wahr des Tuning Parameters alpha'

Vielleicht auch den Press Release zum Schwarzen Loch Bild erwähnen oder ist
das zu random?


\medskip

2. als Modell Problem dient das ROF-Modell (\cite[sec 6.2.1]{CP10}, beschreibt
auch,
was die Parameter und Terme sollen, aber vielleicht auch noch andere Quellen 
dazu (''siehe auch JZ13'')

\medskip
natürlich an dieser Stelle das Funktional einmal wie in Bartels hinschreiben

an der stelle vielleicht auch kurz schreiben, was Bartels tut und dann was wir
hier tun im Unterschied dazu

\medskip

benannt nach Rudin, Osher und Fatemi, die das Modell vorschlugen in
\cite{ROF92} (das ist auch aus CP10, also mit in das Zitat von CP10 rein)

\medskip
u ist die exakte Lösung, g das gesehene verrauschte Bild und alpha Gewichtet 
das Verhältniss zwischen Regularisierung und data fitting (tradeoff
between regularization and data fitting, CP10)
(zitiere eines der Paper/Bücher, CP10 above klärt das ganz gut)
große alpha wenig Entrauschen, kleine alpha blurry, oversmoothed u
\cite{JZ13}

\medskip
TV Term verringert Oszillationen, lässt aber Unstetigkeiten der Lösung zu.
Regularisierungs Term, der Rauschen und kleine Details entfernt.

\medskip
der zweite Term versucht die Lösung nahe an g zu halten (Treue, fidelity) (cite
getreuer und JZ13, hat aber keinen mathscinet entry), vielleicht auch nur
für mich

\medskip
INSBESONDERE erst ROF beschreiben und Parameter erklären und dann CCs 
Variante hinchreiben

\medskip
3. wir betrachten hier eine leicht abgewandlete Variante, die wie folgt
zusammenhängt (dann dieses Remark (aber wohl nicht mehr als Remark)

\begin{remark}
  In \cite[Kapitel~10.1.3]{Bar15} wird \Cref{prob:continuousProblem} für ein
  gegebenes $g\in L^2(\Omega)$ formuliert
  mit dem Funktional 
  \begin{align*}
    I(v)\coloneqq |v|_{\BV(\Omega)} + \frac{\alpha}{2}\int_\Omega (v-g)^2\dx
  \end{align*}
  für $v\in \BV(\Omega)\cap L^2(\Omega)$.

  Nun wählen wir $f = \alpha g$. Dann gilt
  $I(v) = E(v) - \Vert v\Vert_{L^1(\partial \Omega)}+ 
  \frac{\alpha}{2}\Vert g\Vert_{L^2(\Omega)}^2$ für alle 
  $v\in \BV(\Omega)\cap L^2(\Omega)$. Da der Term $\frac{\alpha}{2}\Vert
  g\Vert_{L^2(\Omega)}^2$ konstant ist, haben die Funktionale $E$ und $I$ somit
  die gleichen Minimierer in $\left\{v\in\BV(\Omega)\cap L^2(\Omega)\mid 
  \Vert v\Vert_{L^1(\partial\Omega)}=0\right\}$.
\end{remark}

4. Kurzen Überblick über Aufbau der Arbeit (,,In Kap 1 machen wir das, dann das
in Kapitel 2, dann das in Kapitel 3, \ldots``)

5. auch ein hübsches Bild wie bei Philipp (siehe Paper, sowas wie: Originalbild 
links, verrauschtes Bild (Art des Rauschens nennen) mitte, entrauschtes Bild
rechts). Falls ich die Experimente so eingestellt bekomme, dass sie entrauschen
 (todo)


%%%%%%%%%%%%%%%%%%%%%%%%%%%%%%%%%%%%%%%%%%%%%%%%%%%%%%%%%%%%%%%%%%%%%%%%%%%%%
From theoreticalBasics.tex

%Dies(easymotion-prefix)er Abschnitt folgt dem einführenden Kapitel von [braidesApprox], 
%mit der Einschränkung $\Omega \subset \Rbb^n$ offen und beschränkt, und für
%genauere Informationen sei an dieser Stelle darauf verwiesen.
%
%Die Familie $\Bcal(\Omega)$ aller Borelmengen und die Familie
%$\Bcal_C(\Omega)$ aller Borelmengen mit kompaktem Abschluss in
%$\Omega$ stimmen in diesem Fall überein. 


\section{Maßtheoretische Grundlagen}
 
\todo[inline]{TODO doch noch wenigstens eine einfach Def für Radon Maße (vor
allem mit Zitat zu einer Quelle.
NACHTRAG: auf keinen Fall, BV einfach aus ABM, dieses Kapitel hier einstampfen}

\todo[inline]{TODO Maßtheorie für Vektormaße ist absolut nicht notwenig, da
meine Anwendung ausschließlich von R2 nach R geht Möglicherweise kann ich also
doch eine durchgehende Geschichte erzählen und insbesondere alles verstehen,
im Quelltext sind noch auskommentierte Theorem zur Maßtheorie}

% \begin{definition}[braides]\label{def:masz}
%   Eine Funktion $\mu:\Bcal(\Omega)\to\Rbb^N$ heißt (Vektor-) Maß auf $\Omega$,
%   wenn sie abzählbar additiv ist, d.h.\ für alle
%   $(B_j)_{j\in\Nbb}\subseteq\Bcal(\Omega)$ mit $B_j \cap B_k = \emptyset$ für
%   $j\neq k$ gilt
%   \begin{align*}
%     \mu\left( \dot{\bigcup_{j\in\Nbb}} B_j  \right) = \sum_{j\in\Nbb} \mu(B_j).
%   \end{align*}
%   Die Menge aller dieser Maße sei $\Mcal(\Omega;\Rbb^N)$.
% 
%   Im Fall $N=1$ heißt $\mu$ skalares Maß. Falls $\mu$ zusätzlich nur Werte in 
%   $[0,\infty)$
%   annimmt, heißt es positives Maß. Die Menge aller skalaren Maße sei 
%   $\Mcal(\Omega)$ und die Menge aller positiven Maße sei 
%   $\Mcal_+(\Omega)$.
% \end{definition}
% 
% \begin{remark}
%   Die übliche Anforderung $\mu(\emptyset) = 0$  an ein Maß ist äquivalent zur 
%   Bedingung, dass $A\in \Bcal(\Omega)$ existiert, sodass $\mu(A)$ in jeder 
%   Komponente endliche Werte annimmt. 
% 
%   Da in \Cref{def:masz} als Wertebereich $\Rbb^N$ gefordert wird, ist
%   diese äquivalente Bedingung erfüllt.
% \end{remark}
% 
% \begin{definition}\label{def:radonmasz}
%   Eine Funktion $\mu:\Bcal(\Omega)\to\Rbb^N$ heißt Radonmaß auf 
%   $\Omega$, wenn $\mu|_{\Bcal(\Omega')}$ ein Maß auf jeder Menge 
%   $\Omega'\ssubset \Omega$ (d.h.\,$\closure({\Omega'})\subset \interior({\Omega})$) 
%   ist.
% \end{definition}

\section{Karush-Kuhn-Tucker Bedingungen}
\todo[inline]{Wahrscheinlich einstampfen}

\section{Sattelpunktsprobleme}
In diesem Abschnitt können wir angelehnt an \cite[S. 4ff., S. 165ff.]{Aub79}
eine propere Funktion $f:U\to (-\infty,\infty]$ betrachten, wobei $U$ ein
Vektorraum ist.
\todo[inline]{Wahrscheinlich einstampfen, das wurde an Ort und Stelle erledigt}

\section{Funktionen Beschränkter Variation}

Dieser Abschnitt folgt Kapitel 10 von \cite{ABM14}. Für Details siehe
\cite{EG92}, \todo[inline]{Vielleicht noch weitere Quellen auflisten. ,,Für 
Details siehe\ldots``}
Dabei sei $\Omega$ eine offene Teilmenge des $\Rbb^n$.

%\begin{definition}
%  \todo{TODO vielleicht zu Grundlagen über Radonmaße verschieben, vielleicht
%  auch einfach einstampfen. Fürs erste stehen gelassen. Vielleicht auch alles
%  zu Maßen weglassen, weil es hier in der Arbeit nicht wirklich gebraucht wird
%  und nur die notwendigen Dinge zitieren}
%  Die Vervollständigung des Raums $C^\infty_C(\Omega;\Rbb^m)$ bezüglich der 
%  Norm
%  $\Vert\bullet\Vert_{L^\infty(\Omega)}$ ist ein separabler Banachraum und wird
%  bezeichnet mit 
%  $C_0(\Omega; \Rbb^m)$.
%  Der Dualraum $\Mcal(\Omega;\Rbb^m)$ von $C_0(\Omega; \Rbb^m)$ wird
%  durch den Riesz'schen Darstellungssatz \todo{TODO zitiere?} identifiziert mit
%  dem Raum aller (vektoriellen) Radonmaße. Dabei wird die Anwendung
%  von $\mu\in \Mcal(\Omega;\Rbb^m)$
%  auf $\phi\in C_0(\Omega;\Rbb^m)$ identifiziert mit
%  \begin{align*}
%    \langle \mu, \phi\rangle \coloneqq \int_\Omega \phi \dmu =
%    \int_\Omega \phi(x) \dmu(x).
%  \end{align*}
%\end{definition}

Die folgende Definition basiert auf \cite[S. 393 f.]{ABM14}.

\begin{definition}[Funktionen beschränkter Variation]
  Eine Funktion $u\in L^1(\Omega)$ ist von beschränkter Variation, wenn   
  \begin{align}
    \label{eq:boundedVariation}
    |u|_{\BV(\Omega)}
    \coloneqq
    \sup_{\substack{\phi\in C^1_C(\Omega;\Rbb^n)\\
    \Vert\phi\Vert_{L^\infty(\Omega)}\leq 1}}\int_\Omega u\Div (\phi)\dx
    <
    \infty.
  \end{align}

  Durch $|\bullet|_{\BV(\Omega)}$ ist eine Seminorm auf $\BV(\Omega)$
  gegeben.

  Der Raum aller Funktionen beschränkter Variation $\BV(\Omega)$
  ist ausgestattet mit der Norm 
  \begin{align*}
    \Vert u \Vert_{\BV(\Omega)} \coloneqq \Vert u\Vert_{L^1(\Omega)} +
    |u|_{\BV(\Omega)}
  \end{align*}
  für $u\in\BV(\Omega)$.

  Nach \cite[S. 395, Theorem 10.1.1.]{ABM14} ist $(\BV(\Omega),
  \Vert\bullet\Vert_{\BV(\Omega)})$ ein Banachraum.
\end{definition}

%\begin{definition}[Funktionen beschränkter Variation]
%  Eine Funktion $u\in L^1(\Omega)$ ist von beschränkter Variation, wenn ihre
%  distributionelle Ableitung ein Radonmaß definiert, d.h.\ eine Konstante
%  $c\geq 0$ existiert, sodass 
%  \begin{align}
%    \label{eq:boundedVariation}
%    \langle Du,\phi\rangle \coloneqq - \int_\Omega u\Div (\phi) \dx 
%    \leq c\Vert\phi\Vert_{L^\infty(\Omega)}
%  \end{align}
%  für alle $\phi\in C^1_C(\Omega;\Rbb^n)$.
%
%  Die minimale Konstante $c\geq 0$, die \eqref{eq:boundedVariation} erfüllt,
%  heißt totale Variation von $Du$ und besitzt die Darstellung
%  \begin{align*}
%    |u|_{\BV(\Omega)} = \sup_{\substack{\phi\in C^1_C(\Omega;\Rbb^n)\\
%    \Vert\phi\Vert_{L^\infty(\Omega)}\leq 1}}-\int_\Omega u\Div (\phi)\dx.
%  \end{align*}
%
%  Durch $|\bullet|_{\BV(\Omega)}$ ist eine Seminorm auf $\BV(\Omega)$
%  gegeben.
%
%  Der Raum aller Funktionen beschränkter Variation $\BV(\Omega)$
%  ist ausgestattet mit der Norm 
%  \begin{align*}
%    \Vert u \Vert_{\BV(\Omega)} \coloneqq \Vert u\Vert_{L^1(\Omega)} +
%    |u|_{\BV(\Omega)}
%  \end{align*}
%  für $u\in\BV(\Omega)$.
%\end{definition}

\begin{remark}
  Es gilt $W^{1,1}(\Omega)\subset\BV(\Omega)$ und 
  $\Vert u \Vert_{\BV(\Omega)}=\Vert u \Vert_{W^{1,1}(\Omega)}$ für alle
  $u\in W^{1,1}(\Omega)$. (\cite[S. 394]{ABM14})
\end{remark}

\begin{definition}
  Sei $(u_n)_{n\in\Nbb}\subset \BV(\Omega)$ und sei $u\in \BV(\Omega)$ mit
  $u_n\rightarrow u$ in $L^1(\Omega)$ für $n\rightarrow\infty$.
  \begin{itemize}
    \item[(i)]
      Die Folge $(u_n)_{n\in\Nbb}$ konvergiert strikt gegen $u$,
      wenn $|u_n|_{\BV(\Omega)}\rightarrow |u|_{\BV(\Omega)}$ für $n\rightarrow\infty$.
      {\color{red} strikte Konvergenz gdw. ($\Vert u-u_n\Vert_{L^1(\Omega)}
      \to 0$ und $|u_n|_{\BV(\Omega)}\rightarrow |u|_{\BV(\Omega)}$)
      was impliziert $\Vert u_n\Vert_\BV\to \Vert u\Vert_{\BV}$ aber nicht
      unbedingt $\Vert u_n - u \Vert_\BV\to 0$, da nicht folgt, dass 
      $|u_n - u|_\BV\to 0$
      
      aus BV Konvergenz, also $\Vert u_n - u \Vert_\BV\to 0$, folgt hingegen 
      aber ($\Vert u-u_n\Vert_{L^1(\Omega)}
      \to 0$ und $|u_n - u|_{\BV(\Omega)}\rightarrow 0$), also insbesondere
      ($\Vert u-u_n\Vert_{L^1(\Omega)}
      \to 0$ und $|u_n|_{\BV(\Omega)}\rightarrow |u|_{\BV(\Omega)}$), d.h.
      strikte Konvergenz
      
      jede BV konvergente Folge ist also strikt konvergent aber nicht umgekehrt,
      es gibt also mehr strikt konvergente Folgen, deshalb klingt es sinnvoll,
      dass wir BV Funktionen durch $C^\infty$ Funktionen (usw.) approximieren
      können bzgl strikter Konvergenz aber nicht bzgl BV Konvergenz (strong 
      topology, vgl.  Ende von BV lecture04}
    \item[(ii)] Die Folge $(u_n)_{n\in\Nbb}$ konvergiert
      schwach gegen $u$, wenn
      $Du_n\rightharpoonup^\ast Du$ in 
      $\Mcal(\Omega;\Rbb^n)$ für $n\rightarrow\infty$, d.h.\ für alle
      $\phi\in C_0(\Omega;\Rbb^n)$ gilt 
      $\langle Du_n,\phi\rangle\rightarrow \langle Du,\phi\rangle$ für 
      $n\rightarrow\infty$.
  \end{itemize}
\end{definition}

\begin{theorem}[Schwache Unterhalbstetigkeit]
  \label{thm:wlsc}
  Seien $(u_n)_{n\in\Nbb}\subset\BV(\Omega)$ und $u\in L^1(\Omega)$ mit
  $|u_n|_{\BV(\Omega)}\leq c$ für ein $c>0$ und alle $n\in\Nbb$ und
  $u_n\rightarrow u$ in $L^1(\Omega)$ für $n\rightarrow\infty$.

  Dann gilt $u\in\BV(\Omega)$ und $|u|_{\BV(\Omega)}\leq
  \liminf_{n\rightarrow\infty}|u_n|_{\BV(\Omega)}.$
  Außerdem gilt $u_n\rightharpoonup u$ in $\BV(\Omega)$ für $n\rightarrow
  \infty$.
\end{theorem}

\begin{theorem}[Appoximation mit glatten Funktionen]
  \label{thm:approximationBySmoothFunctions}
  Die Räume $C^\infty(\overline\Omega)$ und $C^\infty(\Omega)\cap\BV(\Omega)$
  liegen dicht in $\BV(\Omega)$ bezüglich strikter Konvergenz.

  {\color{red}BV lecture05 Thm 2.4 liefert sogar (Folge in
  $C^\infty\cap W^{1,1}$) sowohl strikte als auch schwache Konvergenz gegen
  gegebenes $u\in\BV$, also wir haben nach diesem Thm die Dichte von
  $C^\infty\cap W^{1,1}$ bzgl. strikter und schwacher BV Konvergenz
  
  da für Folgen in $W^{1,1}$ BV und $W^{1,1}$ Norm übereinstimmen und da
  $W^{1,1}$ der Abschluss von $C^\infty$ bzgl. der $W^{1,1}$ Norm ist (
  $C^\infty$ dicht in $W^{ 1,1 }$ bzgl. W11 Norm), ist
  für $W^{1,1}$ Funktionen W11 auch Abschluss von Cinfty bzgl
  der BV Norm (Cinfty dicht in W11 bzhl BV Norm)
  
  JETZT DER KNACKPUNKT und wie aus Thmeorem 2.6 gefolgert werden kann was in BV lecture
  steht: da BV Konvergenz strikte Konvergenz impliziert, ist 
  also Cinfty auch dicht in W11 bzgl strikter Konvergenz und W11 ist Teilmenge
  von BV. Da A dicht in B und B Teilmenge C impliziert das B dicht in C (wiki,
  natürlich beides bzgl gleicher Metrik), folgt insgesamt W11 dicht in BV bzgl
  strikter Konvergenz
  
  MORGEN CONTINUE IN WITH THIS IN EXISTENCE PROOF: NOW I might know WHY
  infimizing sequence in BV can simply be choosen in W11 or something
  (Think about it and what bartels did)}
\end{theorem}

\begin{theorem}
  \label{thm:compactness}
  Sei $(u_n)_{n\in\Nbb}\subset \BV(\Omega)$ eine beschränkte Folge. Dann 
  existiert eine Teilfolge $(u_{n_k})_{k\in\Nbb}$ und ein $u\in\BV(\Omega)$,
  sodass $u_{n_k}\rightharpoonup u$ in $\BV(\Omega)$ für $k\rightarrow\infty$.
  {\color{red}augenscheinlich nicht in BV lecture}
\end{theorem}

\begin{theorem}
  \label{thm:embeddingBVintoLp}
  Die Einbettung $\BV(\Omega)\to L^p(\Omega)$ ist stetig für 
  $1\leq p\leq n/(n-1)$ und kompakt für $1\leq p< n/(n-1)$
\end{theorem}\todo[inline]{TODO vielleicht wichtig, Quelle braucht es noch}

\begin{theorem}[Spuroperator]
  \label{thm:traceOperator}
  Es existiert ein linearer Operator $T:\BV(\Omega)\to L^1(\partial\Omega)$
  mit $T(u) = u|_{\partial\Omega}$ für alle $u\in\BV(\Omega)\cap
  C(\overline\Omega)$.

  Der Operator $T$ ist stetig bezüglich strikter Konvergenz in $\BV(\Omega)$,
  aber nicht stetig bezüglich schwacher Konvergenz in $\BV(\Omega)$. 
\end{theorem}\todo[inline]{TODO finde Quelle, nur ein Remark in Bartels (wird 
aber für CCs Funktional offensichtlich gebraucht, es gibt noch weitere Aussagen
in Bartels (zB integration by parts aber erstmal nur Existens und Stetigkeit 
hier (wie gesagt, nur was gebraucht wird zitieren, oder?)}



%%%%%%%%%%%%%%%%%%%%%%%%%%%%%%%%%%%%%%%%%%%%%%%%%%%%%%%%%%%%%%%%%%%%%%
%BV Chapter with measure stuff

\section{Funktionen beschränkter Variation}

In diesen Abschnitt führen wir den Raum der Funktionen beschränkter Variation
ein und formulieren die in dieser Arbeit benötigten Aussagen über diesen.
Dabei beschränken wir unsere Ausführungen 
Für weit detailliertere Ausführung und die benötigten
maßtheoretischen Grundlagen siehe zum Beispiel \cite{ABM14, EG92, Bra98}. 

Soweit nicht anders angegeben, folgen wir \cite[S. 393]{ABM14}
\todo{noch sagen, dass Beweise da zu finden sind oder ist das mit dem Satz
schon klar?  Besser sowas sagen wie 'Die Aussagen stammen aus'?}

Dazu sei in den folgenden Definitionen und Aussagen $U$ eine offene Teilmenge
des $\Rbb^d$.

Der Raum aller $\Rbb^d$-wertigen Borelmaße wird bezeichnet mit
$M(U;\Rbb^d)$.
Statten wir $C_0(U;\Rbb^d)$ mit der Norm $\Vert\phi\Vert_\infty\coloneqq
(\sum_{j=1}^n\sup_{x\in U}|\phi_j(x)|^2)^{1/2}$ für $\phi\in C_0(U;\Rbb^d)$ 
aus, so ist $M(U;\Rbb^d)$ identifizierbar mit dem Dualraum von $C_0(U;\Rbb^d)$.

Zunächst können wir damit den Raum der Funktionen beschränkter Variation
definieren.

\begin{definition}[Funktionen beschränkter Variation]
  Eine Funktion $u\in L^1(U)$ ist von beschränkter Variation, wenn ihre
  distributionelle Ableitung $Du$ ein Element in $M(U;\Rbb^d)$ definiert. Das 
  ist äquivalent zu der Bedingung
  \begin{align}
    \label{eq:boundedVariation}
    |u|_{\BV(U)}
    \coloneqq
    \sup_{\substack{\phi\in C^1_C(U;\Rbb^d)\\
    \Vert\phi\Vert_{L^\infty(U)}\leq 1}}\int_U u\Div (\phi)\dx
    <
    \infty.
  \end{align}
\end{definition}

\begin{remark}
  Durch $|\bullet|_{\BV(U)}$ ist eine Seminorm auf $\BV(U)$
  gegeben.

  Der Raum aller Funktionen beschränkter Variation $\BV(U)$
  ist ausgestattet mit der Norm 
  \begin{align*}
    \Vert u \Vert_{\BV(U)} \coloneqq \Vert u\Vert_{L^1(U)} +
    |u|_{\BV(U)}
  \end{align*}
  für $u\in\BV(U)$.

  Nach \cite[S. 395, Theorem 10.1.1.]{ABM14} ist $(\BV(U),
  \Vert\bullet\Vert_{\BV(U)})$ ein Banachraum.
  Es gilt $W^{1,1}(U)\subset\BV(U)$ und 
  $\Vert u \Vert_{\BV(U)}=\Vert u \Vert_{W^{1,1}(U)}$ für alle
  $u\in W^{1,1}(U)$. (\cite[S. 394]{ABM14})
\end{remark}

\begin{definition}
  Sei $(u_n)_{n\in\Nbb}\subset \BV(U)$ und sei $u\in \BV(U)$ mit
  $u_n\rightarrow u$ in $L^1(U)$ für $n\rightarrow\infty$.
  \begin{itemize}
    \item[(i)]
      Die Folge $(u_n)_{n\in\Nbb}$ konvergiert strikt gegen $u$,
      wenn $|u_n|_{\BV(U)}\rightarrow |u|_{\BV(U)}$ für
      $n\rightarrow\infty$
      (unter Beachtung von \cite[Remark 10.1.1]{ABM14}).
    \item[(ii)] Die Folge $(u_n)_{n\in\Nbb}$ konvergiert
      schwach gegen $u$, wenn
      $Du_n$ in 
      $M(U;\Rbb^d)$ schwach gegen $Du$ konvergiert.
  \end{itemize}
\end{definition}

\begin{theorem}[Schwache Unterhalbstetigkeit, Prop. 10.1.1]
  \label{thm:wlsc}
  Sei $(u_n)_{n\in\Nbb}$ eine Folge in $\BV(U)$ mit
  $\sup_{n\in\Nbb}|u_n|_{\BV(U)}< \infty$ und $u\in L^1(U)$ 
  mit $u_n\rightarrow u$ in $L^1(U)$ für $n\rightarrow\infty$.

  Dann gilt $u\in\BV(U)$ und $|u|_{\BV(U)}\leq
  \liminf_{n\rightarrow\infty}|u_n|_{\BV(U)}.$
  Außerdem konvergiert $u_n$ schwach gegen $u$ in $\BV(U)$.
\end{theorem}

Mit Blick auf die folgenden Kapitel, betrachten wir nun
das polygonal berandete Lipschitz-Gebiet $\Omega\subset\Rbb^d$.

\todo[inline]{Nochmal nachfragen: das heißt tatsächlich offen, beschränkt mit
Lipschitz Rand, korrekt?}
\todo[inline]{Nochmal nachfragen: $\sup u_k<\infty\Leftrightarrow u_k$ bounded,
korrekt? Ich übersehe da nichts, oder? Falls doch, alle Theoreme nochmal 
nachschlagen und sichergehen, dass sie richtig zitiert sind.}
\todo[inline]{Die BV Aussagen aus den nächsten Kapitel wieder hierher holen, soweit 
sinnvoll. Vielleicht auch nur die Spuroperator Aussage. Kann kopiert werden
aus delTexts.tex.}

\begin{theorem}[\protect{\cite[S. 176, Theorem 4]{EG92}}]
  \label{thm:l1ConvergentSubsequence}
  Sei $(u_n)_{n\in\Nbb}\subset \BV(\Omega)$ eine beschränkte Folge. Dann 
  existiert eine Teilfolge $(u_{n_k})_{k\in\Nbb}$ von
  $(u_n)_{n\in\Nbb}$ und ein $u\in\BV(\Omega)$, sodass
  $u_{n_k}\to u$ in $L^1(\Omega)$ für $k\to \infty$.
\end{theorem}
\todo[inline]{Das kann vielleciht auch noch im kontinuierlichen Existenzbeweis 
eingebracht werden und den etwas vereinfachen/verkürzen. Prüf das Zukunfts-Ich}

\begin{theorem}
  \label{thm:compactness}
  Sei $(u_n)_{n\in\Nbb}\subset \BV(\Omega)$ eine beschränkte Folge. Dann 
  existiert eine Teilfolge $(u_{n_k})_{k\in\Nbb}$ und ein $u\in\BV(\Omega)$,
  sodass $u_{n_k}\rightharpoonup u$ in $\BV(\Omega)$ für $k\rightarrow\infty$.
\end{theorem}

\begin{proof}
  Nach \Cref{thm:l1ConvergentSubsequence} besitzt $(u_n)_{n\in\Nbb}$ eine
  Teilfolge $(u_{n_k})_{k\in\Nbb}$, die in $L^1(\Omega)$ gegen ein
  $u\in\BV(\Omega)$ konvergiert.
  Diese Folge ist nach Voraussetzung ebenfalls beschränkt in 
  $\BV(\Omega)$, woraus nach Definition der Norm auf $\BV(\Omega)$ insbesondere
  folgt, dass
  $\sup_{k\in\Nbb}|u_{n_k}|_{\BV(\Omega)}< \infty$. 
  
  Insgesamt liefert \Cref{thm:wlsc} dann die schwache Konvergenz von
  $(u_{n_k})_{k\in\Nbb}$ in $\BV(\Omega)$ gegen $u\in\BV(\Omega)$.
\end{proof}

%In der Anwendung ist Konvergenz in $\BV(U)$ bezüglich der Norm
%$\Vert\bullet\Vert_{\BV(U)}$ zu restriktiv (cf.\ \cite[300]{Bar15}). 
%Deshalb führen wir zwei schwächere Konvergenzbegriffe ein.
%
%\begin{definition}
%  Sei $(u_n)_{n\in\Nbb}\subset \BV(U)$ und sei $u\in \BV(U)$ mit
%  $u_n\rightarrow u$ in $L^1(U)$ für $n\rightarrow\infty$.
%  \begin{itemize}
%    \item[(i)]
%      Die Folge $(u_n)_{n\in\Nbb}$ konvergiert strikt gegen $u$,
%      wenn $|u_n|_{\BV(U)}\rightarrow |u|_{\BV(U)}$ für
%      $n\rightarrow\infty$. 
%      \todo[inline]{Nochmal gucken, das wird glaube ich nirgends gebraucht.
%      Falls das stimmt, rausnehmen. Wird für Dichte von $C^\infty$ usw in $BV$
%      gebraucht bzgl dieser Konvergenz. Gucken, ob das relevant wird. Falls
%      es rausgenomme wird, Text vor und nach dieser Def noch anpassen}
%    \item[(ii)] Die Folge $(u_n)_{n\in\Nbb}$ konvergiert
%      schwach gegen $u$, wenn für alle $\phi\in C_0(U;\Rbb^d)$ gilt, dass
%      $\int_U u_n\Div(\phi)\dx\rightarrow \int_U u\Div(\phi)\dx$ für
%      $n\to\infty$. Wir schreiben dann $u_n\rightharpoonup u$ für $n\to\infty$.
%  \end{itemize}
%\end{definition}

%%%%%%%%%%%%%%%%%%%%%%%%%%%%%%%%%%%%%%%%%%%%%%%%%%%%%%%%%%%%%%%%%%%%%%%%%%%%%
From continuousProblem.tex

  %Wir wissen, dass $(u_n)_{n\in\Nbb}$ beschränkt in $\BV(\Omega)$
  %ist und in $L^1(\Omega)$ gegen $u\in \BV(\Omega)\cap L^2(\Omega)\subset
  %L^1(\Omega)$ konvergiert. \cref{thm:wlsc} impliziert unter diesen 
  %Voraussetzungen, dass $|u|_{\BV(\Omega)}\leq \liminf_{n\to\infty} 
  %|u_n|_{\BV(\Omega)}$.

  % Nach \cite[S. 399, Theorem 10.1.3]{ABM14} gilt diese Aussage auch für
  % $d\in\Nbb$, wenn $U\subset\Rbb^d$ offen, beschränkt und 1-regulär ist.
%%%%%%%%%%%%%%%%%%%%%%%%%%%%%%%%%%%%%%%%%%%%%%%%%%%%%%%%%%%%%%%%%%%%%%%%%%%%%
From discreteProblem.tex

  Durch \Cref{eq:discreteEnergySaddlefunctionalEquality} wissen wir bereits,
  dass 
  \begin{align*}
    \Enc(\vcr) = \sup_{\Lambda_0\in
    P_0\left(\Tcal;\Rbb^2\right)}L(\vcr,\Lambda_0)\quad\text{für alle }
    \vcr\in\CR^1_0(\Tcal).
  \end{align*}
  Somit existiert für $\tilde{u}_\CR$ ein $\bar\Lambda_0\in
  P_0\left(\Tcal;\Rbb^2\right)\cap K$ mit $\Enc\!\left(\tilde{u}_\CR\right) =
  L\!\left(\tilde{u}_\CR,\bar\Lambda_0\right)$. 
  Da außerdem $\tilde{u}_\CR$ \Cref{prob:discreteProblem} löst, folgt insgesamt
  \begin{equation}
    \label{eq:lagrangianEqualsInfSup}
    \begin{aligned}
      L\!\left(\tilde{u}_\CR,\bar\Lambda_0\right)
      =
      \Enc\!\left(\tilde{u}_\CR\right)
      &=
      \inf_{\vcr\in\CR^1_0(\Tcal)}\Enc(\vcr)\\
      &=
      \inf_{\vcr\in\CR^1_0(\Tcal)}
      \sup_{\Lambda_0\in P_0\left(\Tcal;\Rbb^2\right)} L(\vcr,\Lambda_0).
    \end{aligned}
  \end{equation}
  Weiterhin gilt nach \cite[S. 379, Lemma 36.1]{Roc70}, dass
  \begin{align*}
    \inf_{\vcr\in\CR^1_0(\Tcal)}\sup_{\Lambda_0\in P_0\left(\Tcal;\Rbb^2\right)} 
    L(\vcr,\Lambda_0)
    \geq 
    \sup_{\Lambda_0\in P_0\left(\Tcal;\Rbb^2\right)} \inf_{\vcr\in\CR^1_0(\Tcal)} 
    L(\vcr,\Lambda_0).
  \end{align*}
  Zusammen mit \Cref{eq:lagrangianEqualsInfSup} impliziert das
%%
\section{Kontrolle des Abstandes zwischen diskreter und kontinuierlicher
Lösung}
\todo[inline]{section womöglich rausnehmen und das alles zwischen EGLEB Thm
und Verfeinerungsind Def schieben.}
Wir möchten in diesen Abschnitt eine Abschätzungen herleiten, mit welcher
der $L^2$-Abstand zwischen der Lösung des diskreten Problems
\ref{prob:discreteProblem} und der Lösung des kontinuierlichen Problems
\ref{prob:continuousProblem} kontrolliert werden kann.
Dafür benötigen wir zunächst folgendes Theorem.
\todo[inline]{Dann wohl auch den Text hier ändern und sagen, dass es als
Vorbereitung für die GUEB gebraucht wird}

\todo[inline]{Enriching operaotr hier und auch aus der CC Quelle eine
Abschätzung gegen h zitieren, um zu argumentieren, dass das Ding konvergiert}
\todo[inline]{definitiv reinnehmen, aber vielleicht nur als Bemerkung und 
für Beweis auf Bartels verweisen}

\begin{theorem}[Garantierte obere Energieschranke]
  \label{thm:gueb}
  \todo[inline]{Write it, aber Tien noch fragen stellen}
  Sei $\Omega$ konvex, $f\in H^1_0(\Omega)$ das Eingangssignal für
  \Cref{prob:continuousProblem} mit Lösung $u\in H^1_0(\Omega)$ und minimaler
  Energie $E(u)$ sowie für \Cref{prob:discreteProblem} mit Lösung $\ucr\in
  \CR^1_0(\Omega)$ und minimaler Energie $\Enc(\ucr)$.
  Dann gilt
  \begin{align*}
    \Enc(\ucr)+\frac{\alpha}{2}\Vert u-\ucr\Vert^2
    -\frac{\kappa_\CR}{\alpha}\Vert
    h_\Tcal(f-\alpha\ucr)\Vert \Vert\nabla f\Vert\leq E(u).
  \end{align*}
  Dabei ist die Konstante $\kappa_\CR\coloneqq\sqrt{1/48+1/j_{1,1}^2}$ mit der
  kleinsten positiven Nullstelle $j_{1,1}$ der Bessel-Funktion erster Art.
  Insbesondere gilt dann für 
  \begin{align}
    \Egleb 
    \coloneqq 
    \Enc(\ucr) - \frac{\kappa_\CR}{\alpha}\Vert h_\Tcal(f-\alpha\ucr)\Vert
    \Vert \nabla f\Vert,
  \end{align}
    dass $\Enc(\ucr)\geq \Egleb$ und $E(u)\geq \Egleb$.

\end{theorem}

Zusammen mit den Betrachtungen in \Cref{sec:discreteProblemFormulation} 
erhalten wir damit die folgende Abschätzung.

\begin{corollary}[TODO löschen]
  Sei $u\in\BV(\Omega)\cap L^2(\Omega)$ Lösung von
  \Cref{prob:continuousProblem} und $\ucr\in\CR^1_0(\Tcal)$ Lösung von
  \Cref{prob:discreteProblem}.
  Dann gilt
  \begin{align*}
    \frac{\alpha}{2}\Vert u-\ucr\Vert^2\leq
    E(\ucr)-E(u)=\Enc(\ucr)+\sum_{F\in\Ecal}\Vert[\ucr]_F\Vert_{L^1(F)}-E(u).
  \end{align*}
  Insbesondere gilt auch 
  \begin{align*}
    \frac{\alpha}{2}\Vert u-\ucr\Vert^2
    \leq
    \left|\Enc(\ucr)-E(u)\right|+
    \left|\sum_{F\in\Ecal}\Vert[\ucr]_F\Vert_{L^1(F)}\right|.
  \end{align*}
\end{corollary}

\todo[inline]{Hier Fragen und Formeln}
Von Theorem 4.9:
  \begin{align*}
    \Enc(\ucr)+\frac{\alpha}{2}\Vert u-\ucr\Vert^2
    -\frac{\kappa_\CR}{\alpha}\Vert
    h_\Tcal(f-\alpha\ucr)\Vert \Vert\nabla f\Vert\leq E(u).
  \end{align*}
  impliziert für
  \begin{align*}
    \Egleb 
    \coloneqq 
    \Enc(\ucr) - \frac{\kappa_\CR}{\alpha}\Vert h_\Tcal(f-\alpha\ucr)\Vert
    \Vert \nabla f\Vert,
  \end{align*}
    dass $\Enc(\ucr)\geq \Egleb$ und $E(u)\geq \Egleb$. 

      deshalb garantierte untere Energieschranke?

      Falls ja, ist dann noch eine Abschätzung $E(u)\leq E_\textup{gueb}$ gewollt?
      Diese bekommen wir nicht so wirklich aus Thm. 4.11
      Es gilt natürlich z.B.
      $E(u)\leq E(J_1\ucr)=\Enc(J_1\ucr)$. Bringt das irgendwas? Hat nur
      leider keine h Potenz dabei.

      Weitere von Thm 4.9 implizierte Aussagen, die nützlich aussehen:

      \begin{align*}
        \Vert u-\ucr\Vert\leq 
        \sqrt{\frac{\alpha}{2}}\sqrt{E(u)-\Egleb}
        \lesssim
        \sqrt{E(u)-\Egleb}
      \end{align*}

      diese rechte Seite, ohne die Konstante, wurde geplottet (quadriert)
      und war dann parallel. Also gilt in den Experimenten sogar mehr. 
      Parallelität gilt aber damit nicht. (Bild zeigen, auch das angepasste
      mit sqrt)

      Diese Aussage ist keine A-priori Abschätzung, weil die rechte
      Seite immer noch von $\ucr$ abhängt, oder?
      Bringt der plot dieser RHS dann noch irgendwas
      

      Theorem 4.11 liefert
  \begin{align*}
    \frac{\alpha}{2}\Vert u-v\Vert^2 \leq E(v)-E(u)\quad
    \text{für alle } v\in\BV(\Omega)\cap L^2(\Omega).
  \end{align*}

  Daraus folgt
  \begin{align*}
    \Vert u-\ucr\Vert &\leq
    \sqrt{2/\alpha}\sqrt{\Enc(J_1{\ucr})-\Egleb} + \Vert J_1\ucr-\ucr\Vert\\
    &\lesssim
    \sqrt{\Enc(J_1{\ucr})-\Egleb} + \Vert J_1\ucr-\ucr\Vert
  \end{align*}

  Die unterscheidet sich dann aber nicht mehr von der anderen Aussage, die auch 
  eine obere Schranke für $\Vert u-\ucr\Vert$ war.
  Ich hatte mit einer unteren Schranke gerechnet, weil CC was von ,,zwischen
  zwei Graphen`` liegen meinte.
  ACHSO, die hier ist natürlich besser, weil die auch ohne Kenntniss der exakten
  Lösung berechnet werden kann (A-posteriori, korrekt?). Den Gradienten von $f$
  braucht man allerdings noch, also kann man das für Cameraman immer noch nicht
  nutzen.

  Außerdem impliziert Theorem 4.11
  \begin{align*}
    E(u)\leq E(J_1 \ucr)-\frac{\alpha}{2}\Vert u - J_1 \ucr\Vert^2 
    = \Enc(J_1 \ucr)-\frac{\alpha}{2}\Vert u - J_1 \ucr\Vert^2 
  \end{align*}
  Bringt das was als GUEB oder zählt das nicht, da die rechte Seite immernoch von
  $u$ abhängt.

%%%%%%%%%%%%%%%%%%%%%%%%%%%%%%%%%%%%%%%%%%%%%%%%%%%%%%%%%%%%%%%%%%%%%%%%%%%%%
From experiments.tex


\begin{align*}
  u_\textrm{HR}(r)\coloneqq 
  \begin{cases}
    1, & \text{wenn } 0\leq r\leq\frac{1}{3},\\
    54r^3 - 81r^2 + 36r - 4, & 
    \text{wenn } \frac{1}{3}\leq r\leq \frac{2}{3},\\
    0, & \text{wenn } \frac{2}{3}\leq r.
  \end{cases}
\end{align*}
Mit der Wahl
\begin{align*}
  \sgn&(\partial_r u_3(r)) \\
  &\coloneqq 
  \begin{cases}
    54r^3-27r^2, & \text{wenn } 0\leq r\leq\frac{1}{3},\\
    -1, & \text{wenn } \frac{1}{3}\leq r\leq \frac{2}{3},\\
    -54r^3 + 135r^2 - 108r + 27, & \text{wenn } \frac{2}{3}\leq r\leq 1,
  \end{cases}
\end{align*}
erhalten wir die rechte Seite
\begin{align*}
  f_3(r)\coloneqq 
  \begin{cases}
    \alpha - 216r^2 + 81r, &
    \text{wenn } 0\leq r\leq\frac{1}{3},\\
    \alpha\left(54r^3 - 81r^2 + 36r - 4\right)) + \frac{1}{r}, & 
    \text{wenn } \frac{1}{3}\leq r\leq \frac{2}{3},\\
    216r^2 - 405r + 216 - \frac{27}{r}, & 
    \text{wenn } \frac{2}{3}\leq r\leq 1,
  \end{cases}
\end{align*}
für die gilt $f_3\in H^1_0$

\begin{align*}
  \partial_r f_3(r) &=
  \begin{cases}
    - 432r + 81, & \text{wenn } 0\leq r\leq\frac{1}{3},\\
    \alpha\left(162r^2 - 162r + 36\right) - \frac{1}{r^2}, & 
    \text{wenn } \frac{1}{3}\leq r\leq \frac{2}{3},\\
    432r - 405 + \frac{27}{r^2}, & 
    \text{wenn } \frac{2}{3}\leq r\leq 1,
  \end{cases}
\end{align*}
