%
% USER-DEFINED ENVIRONMENTS
%
\usepackage{amsthm}
\usepackage{listings}

% Re-define font of optional title
\makeatletter
\def\th@plain{%
  \thm@notefont{}% same as heading font
  \itshape % body font
}
\def\th@definition{%
  \thm@notefont{}% same as heading font
  \normalfont % body font
}
\makeatother

%%% MATH ENVIRONMENTS %%%%%%%%%%%%%%%%%%%%%%%%%%%%%%%%%%%%%%%%%%%%%%%%%%%%%%%%%
% statements
\theoremstyle{plain}
\newtheorem{theorem}{Theorem}[chapter]
\newtheorem{proposition}[theorem]{Proposition}
\newtheorem{lemma}[theorem]{Lemma}
\newtheorem{corollary}[theorem]{Corollary}
% normal text
\theoremstyle{definition}
\newtheorem{definition}[theorem]{Definition}
\newtheorem{example}[theorem]{Example}
\newtheorem{exercise}{Exercise}[chapter]
% minor text
\theoremstyle{remark}
\newtheorem{remark}[theorem]{Remark}
\newtheorem*{notation}{Notation}
% proofs
\newtheorem{claim}{Claim}
\newtheorem{subproof}{Proof~of~the~claim}
\newtheorem*{proofb}{Proof}
% algorithm
\newtheoremstyle{algorithm-style}%
  {\topsep}   % ABOVESPACE
  {\topsep}   % BELOWSPACE
  {\normalfont}% BODYFONT
  {0pt}       % INDENT (empty value is the same as 0pt)
  {\bfseries} % HEADFONT
  {.}         % HEADPUNCT
  {5pt plus 1pt minus 1pt} % HEADSPACE
  {\thmname{#1}\thmnumber{ #2}\thmnote{ (#3)}} % CUSTOM-HEAD-SPEC
\theoremstyle{algorithm-style}
\newtheorem{algorithm}[theorem]{Algorithm}

% axiom
\newtheoremstyle{axiom-style}%
  {\topsep}   % ABOVESPACE
  {\topsep}   % BELOWSPACE
  {\itshape}% BODYFONT
  {0pt}       % INDENT (empty value is the same as 0pt)
  {\bfseries} % HEADFONT
  {.}         % HEADPUNCT
  {5pt plus 1pt minus 1pt} % HEADSPACE
  {\thmname{#1} \thmnumber{A#2}\thmnote{ (#3)}} % CUSTOM-HEAD-SPEC
\theoremstyle{axiom-style}
\newtheorem{inneraxiom}{Axiom}
\newenvironment{axiom}[1]
  {\renewcommand\theinneraxiom{#1}\inneraxiom}
  {\endinneraxiom}
%To Reference an Axiom as ``(A1)'' use \cref
\crefname{inneraxiom}{}{}
%To Reference an Axiom as ``axiom (A1)'' use \Cref
\Crefname{inneraxiom}{axiom}{axioms}
\creflabelformat{inneraxiom}{#2(A#1)#3}

%Only for A12
\newtheoremstyle{axiomspecial-style}%
  {\topsep}   % ABOVESPACE
  {\topsep}   % BELOWSPACE
  {\itshape}% BODYFONT
  {0pt}       % INDENT (empty value is the same as 0pt)
  {\bfseries} % HEADFONT
  {.}         % HEADPUNCT
  {5pt plus 1pt minus 1pt} % HEADSPACE
  {\thmnumber{A#2}\thmnote{ (#3)}} % CUSTOM-HEAD-SPEC
\theoremstyle{axiomspecial-style}
\newtheorem{inneraxiomspecial}{}
\newenvironment{axiomspecial}[1]
  {\renewcommand\theinneraxiomspecial{#1}\inneraxiomspecial}
  {\endinneraxiomspecial}
  %To Reference A12 as ``(A12)'' use \crefspecial
\crefname{inneraxiomspecial}{}{}
\creflabelformat{inneraxiomspecial}{#2(A#1)#3}

%% THEOREM ENVIRONMENTS
% theorems
% \makeatletter
% \newtheoremstyle{algostyle}
% {\item[\rlap{\vbox{\hbox{\hskip\labelsep \theorem@headerfont
%    ##1\ ##2\theorem@separator}\hbox{\strut}}}]}%
% {\item[\rlap{\vbox{\hbox{\hskip\labelsep \theorem@headerfont
% ##1\ ##2\ ##3\theorem@separator}\hbox{\strut}}}]}
% \theoremstyle{algostyle}
% \theorempreskip{3\parskip}
% \theoremheaderfont{\itshape}


% % axioms
% % \makeatletter
% \newtheoremstyle{axiomstyle}%
%   {\item[\hskip\labelsep \theorem@headerfont ##1\ (A##2)\theorem@separator]}%
%   {\item[\hskip\labelsep \theorem@headerfont ##1\ (A##2)\ ##3\theorem@separator]}
% % \makeatother
% \theoremstyle{axiomstyle}

% \theoremheaderfont{\bf}

% \theoremstyle{changebreak}
% \theoremheaderfont{\bf}
% \newcommand\theoremname{Axiom}
% \newtheorem{innercustomthm}{Axiom}
% \newenvironment{axiom}[1]
%   {\renewcommand\theinnercustomthm{#1}\innercustomthm}
%   {\endinnercustomthm}

% \theoremstyle{changebreak}
% \theoremheaderfont{\bf}
% \newcommand\theoremname{Axiom}
% \newtheorem*{thm}{\theoremname}
% \newenvironment{axiom}[1][\theoremname]
%  {\edef\theoremname{#1}%
%   \begin{thm}%
%  }
%  {\end{thm}}

% \makeatother








%%% CODE ENVIRONMENTS %%%%%%%%%%%%%%%%%%%%%%%%%%%%%%%%%%%%%%%%%%%%%%%%%%%%%%%%%
\usepackage{listings}
\lstset{ %
  language=Matlab,
  basicstyle=\scriptsize\ttfamily,
  keywordstyle=\color{blue},
  commentstyle=\color{dkgreen},
  stringstyle=\color{mauve},
  numberstyle=\color{orange},
  numbers=left,
  numberstyle=\tiny\color{gray},
  stepnumber=1,
  numbersep=10pt,
  % backgroundcolor=\color{bgcolor},
  showspaces=false,
  showstringspaces=false,
  showtabs=false,
  tabsize=2,
  captionpos=t,
  breaklines=true,
  breakatwhitespace=false,
  belowskip=0pt plus 2pt minus 2pt,
}
% different font sizes
\lstdefinestyle{fullsrcsmall}{
  basicstyle=\small\ttfamily,
  xleftmargin=1em,
}
\lstdefinestyle{fullsrcfnsize}{
  basicstyle=\footnotesize\ttfamily,
  xleftmargin=2em,
}
\lstdefinestyle{fullsrcscsize}{
  basicstyle=\scriptsize\ttfamily,
  xleftmargin=2em,
}
\lstdefinestyle{inline}{
  basicstyle=\ttfamily,
  numbers=none,
  xleftmargin=1em,
}
\lstdefinestyle{inlinesmall}{
  basicstyle=\small\ttfamily,
  numbers=none,
  xleftmargin=1em,
}
\lstdefinestyle{inlinefnsize}{
  basicstyle=\footnotesize\ttfamily,
  numbers=none,
  xleftmargin=1em,
}

%%% ALGORITHM ENVIRONMENTS %%%%%%%%%%%%%%%%%%%%%%%%%%%%%%%%%%%%%%%%%%%%%%%%%%%%
% \usepackage{algorithm}
\usepackage{algpseudocode}
% endif CC style
\algrenewcommand\algorithmicrequire{\textbf{Input:}}
\algrenewcommand\algorithmicensure{\textbf{Output:}}
\newcommand\Od{\textbf{od}}
\newcommand\Fi{\textbf{fi}}
\algnotext{EndIf}
\let\oldEndIf\EndIf
\renewcommand\EndIf{\Fi\oldEndIf}
\algnotext{EndWhile}
\let\oldEndWhile\EndWhile
\renewcommand\EndWhile{\Od\oldEndWhile}
\algnotext{EndFor}
\let\oldEndFor\EndFor
\renewcommand\EndFor{\Od\oldEndFor}
% block for an inline for loop
\algblockdefx[IfInline]{IfInline}{EndIfInline}
[2]{\textbf{if} #1 \textbf{then} #2}
[0]{\textbf{fi}}
\algcblockdefx{IfInline}{ElseInline}{EndIfInline}
[1]{\textbf{else} #1}
[0]{\textbf{fi}}
\algnotext{EndIfInline}
\let\oldEndIfInline\EndIfInline
\renewcommand\EndIfInline{\Fi\oldEndIfInline}
% single else inline
\algcloopx[SingleElseInline]{If}{SingleElseInline}
[1]{\textbf{else} #1}
