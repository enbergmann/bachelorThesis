\usepackage{ngerman}
\usepackage{amsmath}
\usepackage{amssymb,amstext}
\usepackage{ntheorem}
\usepackage{marvosym}
\usepackage{mathrsfs}
\usepackage{wasysym, ulem}
\usepackage{graphicx}
\usepackage{tikz}
\usepackage{verbatim}
\usepackage{hyperref}
\usepackage{relsize}


%Die Zahlbereiche
\newcommand{\C}{\mathbb{C}}
\newcommand{\F}{\mathbb{F}}
\newcommand{\K}{\mathbb{K}}
\newcommand{\N}{\mathbb{N}}
\newcommand{\Pol}{\mathbb{P}}
\newcommand{\Q}{\mathbb{Q}}
\newcommand{\R}{\mathbb{R}}
\newcommand{\Z}{\mathbb{Z}}
\newcommand{\Prim}{\mathrm{I \! P}}

%um den \Rightarrow zu erhalten...
\let\Rightarrow\undefined
\DeclareMathSymbol{\Rightarrow}{\mathrel}{symbols}{"29} 

%Die Pfeile
\newcommand{\Lra}{\Leftrightarrow}
\newcommand{\Ra}{\Rightarrow}

\newcommand{\im}{\mathrm{im}}
\newcommand{\tr}{\mathrm{tr}}
\newcommand{\Span}{\mathrm{span}}
\newcommand{\Kern}{\mathrm{Kern}}
%Die Basis
\newcommand{\B}{\mathcal{B}}
\newcommand{\dist}{\mathrm{dist}}
\newcommand{\id}{\mathrm{id}}
%gequert
\newcommand{\ol}{\overline}
\newcommand{\ul}{\underline}
%der Rang
\newcommand{\ra}{\mathrm{Rang}}

\newcommand{\fa}{\;\forall\;}
\newcommand{\ex}{\;\exists\;}

%Spectrum und Maximalspectrum
\newcommand{\spec}{\mathrm{Spec}}
\newcommand{\specm}{\mathrm{Specm}}
\newcommand{\ord}{\mathrm{ord}}

% Die kursiv geschriebenen theorem-Umgebungen
\theoremseparator{.}
\newtheorem{satz}{Satz}[chapter]
\newtheorem{korollar}[satz]{Korollar}
\newtheorem{lemma}[satz]{Lemma}
\newtheorem{thm}[satz]{Theorem}
\newtheorem{prop}[satz]{Proposition}

% Die aufrecht geschriebenen theorem-Umgebungen
\theoremseparator{.}
\theorembodyfont{\upshape} 
\newtheorem{bsp}[satz]{Beispiel}
\newtheorem{defi}[satz]{Definition}
\newtheorem{bem}[satz]{Bemerkung}
\newtheorem{aufg}[satz]{Aufgabe}

{\theoremstyle{break}
\newtheorem*{bsps}{Beispiele}
\newtheorem{bems}[satz]{Bemerkungen}
\newtheorem*{algo}{Algorithmus}
}


\theoremsymbol{\ensuremath{ \hfill \Box }}
\newtheorem*{bew}{Beweis}

% Absatz Spezialdesign
\parindent0mm    % Kein Einrücken
\parskip3mm      % 3mm Platz

% enumerate-Spezialdesign
\usepackage{enumitem}

\setlist{topsep=0pt, itemsep=0pt, parsep = 0pt, partopsep = 0pt, leftmargin=*, font=\bfseries}
\newlist{biblio}{itemize}{1}
\setlist[biblio]{listparindent=0pt,itemindent=0pt, label=, %\textbullet, 
	itemsep=0.7ex, leftmargin=0pt, rightmargin = 0pt}

	\newlist{romanenum}{enumerate}{1}
\setlist[romanenum]{label=(\roman*)}
\setenumerate[1]{label=(\alph*)}
\setenumerate[2]{label=(\roman*)}
\setdescription{leftmargin=0pt,  labelindent=\parindent, listparindent=\parindent, labelwidth=0.5em}

\newcommand{\bbems}{\begin{bems}\begin{enumerate}}
\newcommand{\ebems}{\end{enumerate}\end{bems}}
\newcommand{\bbsps}{\begin{bsps}\begin{enumerate}}
\newcommand{\ebsps}{\end{enumerate}\end{bsps}}


%Paragraph mit Numerierung
\setcounter{secnumdepth}{5}  % Paragraph bekommt einen counter
\setcounter{tocdepth}{3}     % wird aber im Inhaltsverzeichnis nicht aufgelistet. 
\def\theparagraph{\arabic{paragraph}}
\def\labelparagraph{\arabic{paragraph}. }
\def\thesubparagraph{\Alph{paragraph}(\roman{subparagraph})}


\newcommand{\E}{\mathcal{E}}
\newcommand{\D}{\mathcal{D}}
\newcommand{\M}{\mathcal{M}}
\newcommand{\NC}{\mathcal{N}}
\newcommand{\T}{\mathcal{T}}
\newcommand{\KC}{\mathcal{K}}
\newcommand{\PC}{\mathcal{P}}
\newcommand{\DC}{\mathcal{D}}
\newcommand{\OT}{\mathcal{O}}
\newcommand{\I}{\mathcal{I}}

\newcommand{\circled}[1]{{\fontsize{14}{14}\selectfont
\textcircled{{\small #1}}}}

\newcommand{\hintf}{\int\hspace{-2.07ex}--}
%\DeclareMathOperator{\hintt}{\int\hspace{-2.07ex}--}

\newcommand{\ve}{\varepsilon}
\newcommand{\contra}{\lightning}
%\newcommand{\Div}{{\rm{div}}}
%\newcommand{\en}{\| \vspace{-2pt} |}
\newcommand{\en}{| \hspace{-1pt} | \hspace{-1pt} |}
\newcommand{\biggg}{\vline{10pt}}


%\newcommand{\3}{\ss }

%%% PERFECT BULLET %%%
\let\oldbullet\bullet
\newlength{\raisebulletlen}
\setbox1=\hbox{$\bullet$}\setbox2=\hbox{\tiny$\bullet$}
\setlength{\raisebulletlen}{\dimexpr0.5\ht1-0.5\ht2}
\renewcommand\bullet{\raisebox{\raisebulletlen}{\,\tiny$\oldbullet$}\,}


\DeclareMathOperator*{\argmin}{argmin}
\DeclareMathOperator{\conv}{conv}
\DeclareMathOperator{\Cond}{cond}
\DeclareMathOperator{\diam}{diam}
\DeclareMathOperator{\diag}{diag}
\DeclareMathOperator{\Dim}{dim}
\DeclareMathOperator{\Div}{div}
\DeclareMathOperator{\Lip}{Lip}
\DeclareMathOperator{\Rang}{Rang}
\DeclareMathOperator{\sing}{sing}
\DeclareMathOperator{\STEMA}{STEMA}
\DeclareMathOperator{\supp}{supp}
\DeclareMathOperator{\width}{width}
%\newcommand{\pg}{\selectlanguage{polutonikogreek}}
%\newcommand{\sge}{\selectlanguage{german}}

% For \operp
% copied relevant lines from stix.sty
%\DeclareFontEncoding{LS1}{}{}
%\DeclareFontSubstitution{LS1}{stix}{m}{n}
%\DeclareSymbolFont{symbols2}{LS1}{stixfrak} {m} {n}
%\DeclareMathSymbol{\operp}{\mathbin}{symbols2}{"A8}
\newcommand{\operp}{\mathsmaller {\bigcirc\hspace{-0.3cm}\perp}\;}
