% ============================================================
% === USER-DEFINED COMMANDS ==================================
% ============================================================

\usepackage{amssymb,mathabx,textcomp,bm}

\newcommand{\mathbox}[2]{\makebox[#1]{$\displaystyle #2$}}
\newcommand{\Stackrel}[2]{\mathbox{1.25em}{\stackrel{#1}{#2}}}


% === INTEGRAL MEAN ==========================================

\def\Xint#1{\mathchoice
{\XXint\displaystyle\textstyle{#1}}%
{\XXint\textstyle\scriptstyle{#1}}%
{\XXint\scriptstyle\scriptscriptstyle{#1}}%
{\XXint\scriptscriptstyle\scriptscriptstyle{#1}}%
\!\int}
\def\XXint#1#2#3{{\setbox0=\hbox{$#1{#2#3}{\int}$}
\vcenter{\hbox{$#2#3$}}\kern-.5\wd0}}
\newcommand{\intmean}{\Xint-}

% === PERFECT BULLET =========================================

\let\oldbullet\bullet
\newlength{\raisebulletlen}
\setbox1=\hbox{$\bullet$}\setbox2=\hbox{\tiny$\bullet$}
\setlength{\raisebulletlen}{\dimexpr0.5\ht1-0.5\ht2}
\renewcommand\bullet{\raisebox{\raisebulletlen}{\,\tiny$\oldbullet$}\,}

% === ORTHOGONAL SUM =========================================

\newcommand\orthsum{
\tikz[baseline=(A.base),font=\small]
     \node[draw,ellipse,inner sep=0.15ex] (A){$\perp$};
}

% === CALIGRAPHIC LETTERS ====================================

\newcommand\Acal{\mathcal{A}}
\newcommand\Bcal{\mathcal{B}}
\newcommand\Ccal{\mathcal{C}}
\newcommand\Dcal{\mathcal{D}}
\newcommand\Ecal{\mathcal{E}}
\newcommand\Fcal{\mathcal{F}}
\newcommand\Jcal{\mathcal{J}}
\newcommand\Kcal{\mathcal{K}}
\newcommand\Lcal{\mathcal{L}}
\newcommand\Mcal{\mathcal{M}}
\newcommand\Ncal{\mathcal{N}}
\newcommand\Ocal{\mathcal{O}}
\newcommand\Pcal{\mathcal{P}}
\newcommand\Rcal{\mathcal{R}}
\newcommand\Tcal{\mathcal{T}}

% === MATHBB LETTERS =========================================

\newcommand\C{\mathbb{C}}
\newcommand\N{\mathbb{N}}
\newcommand\R{\mathbb{R}}
\newcommand\T{\mathbb{T}}
\newcommand\K{\mathbb{K}}

% === MATH OPERATORS =========================================

\DeclareMathOperator*{\argmin}{argmin} % the * adjusts MathOperator for indizes beneath
\DeclareMathOperator{\bisec}{bisec}
\DeclareMathOperator{\cond}{cond}
\DeclareMathOperator{\Conv}{conv}
\DeclareMathOperator{\Curl}{Curl}
\DeclareMathOperator{\curl}{curl}
\DeclareMathOperator{\dev}{dev}
\DeclareMathOperator{\diag}{diag}
\DeclareMathOperator{\diam}{diam}
\DeclareMathOperator{\Dim}{dim}
\DeclareMathOperator{\Div}{div}
\DeclareMathOperator{\Dist}{dist}
\DeclareMathOperator{\esssup}{ess\ supp}
\DeclareMathOperator{\grad}{\nabla}
\DeclareMathOperator{\Int}{int}
\DeclareMathOperator{\Ker}{ker}
\DeclareMathOperator{\Mid}{mid}
\DeclareMathOperator{\Osc}{osc}
\DeclareMathOperator{\Red}{red}
\DeclareMathOperator{\Ref}{Ref}
\DeclareMathOperator{\Res}{Res}
\DeclareMathOperator{\sign}{sgn}
\DeclareMathOperator{\Span}{span}
\DeclareMathOperator{\supp}{supp}
\DeclareMathOperator{\tr}{tr}
\DeclareMathOperator{\Width}{width}
\DeclareMathOperator*{\arginf}{arginf}

% === COMMANDS WITH INPUT ARGUMENTS ==========================

\newcommand\abs[1]{\lvert #1 \rvert}
\newcommand\average[1]{\langle #1 \rangle}
\newcommand\jump[1]{\lbrack #1 \rbrack}
\newcommand\NormEnergy[1]{\big\vvvert #1 \big\vvvert}
\newcommand\Norm[2]{\lVert #1 \rVert_{#2}}
\newcommand\scal[2]{\left\langle #1 , #2 \right\rangle}

% === SPACES WITH NAMES ======================================

\newcommand{\CONF}{\textup{C}}
\newcommand{\NC}{\textup{NC}}
\newcommand{\CR}{\textup{CR}}
\newcommand{\RT}{\textup{RT}}

% === GENERAL STUFF ==========================================

\newcommand{\splitter}{\,:\,} % split for definition of sets
\renewcommand{\d}{\, \textup{d}} % d for differentials, i.e., \d x


% === JUST TO COMPILE CHAPTERS 4  AND 5 ==========================================

\newcommand{\LO}[0]{L^2(\Omega)}
\newcommand{\restrict}[2]{\left. #1 \right\lvert_{#2}}
\DeclareMathOperator{\mPkt}{mid}
\newcommand{\nnorm}[2]{\left\lVert #1 \right\rVert_{#2}}
\newcommand{\volf}[2]{\nnorm{h_{#1}f}{#2}}
\newcommand{\volO}[1]{\nnorm{h_{#1}f}{\LO}}
\DeclareMathOperator\Kern{ker}
\DeclareMathOperator\setspan{span}

% === JUST TO COMPILE CHAPTER 6 ==========================================


\newcommand\NormL[3]{\Norm{#1}{L^{#2}\left( { #3 } \right)}}
 \newcommand\NormH[3]{\Norm{#1}{H^{#2}\left( { #3 } \right)}}
 \newcommand\NormHdiv[1]{\Norm{#1}{H(\div)}}
 \newcommand\NormW[3]{\Norm{#1}{W^{#2}\left( { #3 } \right)}}
\newcommand\NormLDom[2]{\NormL{#1}{#2}{\Omega}}
 \newcommand\NormHDom[2]{\NormH{#1}{#2}{\Omega}}
 \newcommand\NormHdivDom[1]{\NormHdiv{#1}}
 \newcommand\NormWDom[2]{\NormW{#1}{#2}{\Omega}}
 \newcommand\NormLz[2]{\NormL{#1}{2}{#2}}
 \newcommand\NormHz[2]{\NormH{#1}{2}{#2}}
 \newcommand\NormWkp[4]{\NormW{#1}{{#2,#3}}{#4}}
 \newcommand\NormWkpDom[3]{\NormWDom{#1}{{#2,#3}}}
 \newcommand\NormLzDom[1]{\NormLDom{#1}{2}}
 \newcommand\NormHzDom[1]{\NormHDom{#1}{2}}
\newcommand\NormMax[2]{\Norm{#1}{\C({#2})}}


% === JUST TO COMPILE CHAPTER 0 ==========================================

\newcommand{\dist}{\mathrm{dist}}
\newcommand{\fa}{\;\forall\;}
\newcommand{\Lra}{\Leftrightarrow}
\newcommand{\id}{\mathrm{id}}

% === JUST TO COMPILE EXERCISES ==========================================

\newcommand{\al}{\alpha}
\newcommand{\be}{\beta}
\newcommand{\ga}{\gamma}
\newcommand{\de}{\delta}
\newcommand{\eps}{\varepsilon}
\newcommand{\vphi}{\varphi}
\newcommand{\la}{\lambda}
\newcommand{\om}{\omega}
